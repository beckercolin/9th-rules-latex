% Base sur la VO 0.11.9
% Relecture technique: 
% Relecture syntaxique: 

\part{Journal des modifications}

\section*{VO 0.99.1, VF 1.1}
\begin{itemize}
\item Plusieurs clarifications de règles (voir les changements en vert). 
\item Test de Caractéristique pour les profils combinés.
\item Changement du pourcentage dans la version simplifiée de la détermination du vainqueur.
\item Précisions sur les Touches d'impact, les Embuscades et l'épée des Héros.
\end{itemize}


\section*{VO 0.99.0, VF 1.0}
\noindent \textbf{Changement mineurs}
\begin{itemize}
\item \emph{Tirailleurs}, devenir des personnages survivants lorsque l'unité est détruite.
\item \emph{Canalisation}, figurines en plusieurs éléments
\item \emph{Stupide}, pas de test quand l'unité fuit ou est en combat
\item Choix des zones de déploiement
\item Faire un test de Commandement: utiliser le meilleur Cd est optionnel.
\item Unité combinée détruite
\item Canons, réécriture partielle pour que les \emph{Grandes Cibles} qui sont des \emph{Plateformes de guerre} ne puissent pas éviter le +1 pour toucher
\item Canons, le jet pour toucher ne peut plus être relancé
\item Profil des Monstres Montés
\end{itemize}

\noindent \textbf{Clarifications}
\begin{itemize}
\item Machines de guerre: maximiser les figurines en charge
\item Accepter ou Refuser un défi, interaction avec \emph{Tenace}
\item Dissiper les sorts \og Reste en jeu \fg{}
\item Effets simultanés, réécriture
\item Bouclier de Chance, attaques simultanées
\item Assaut et combat contre un bâtiment
\item Endurance tombée à 0 : règles
\item Unités engagées en combat: clarification (elles ne peuvent pas se déplacer)
\item Modificateurs pour toucher au tir
\item Tests de moral, clarification
\item Terrain occultant, terrain et couverts: clarifications
\item Murs, clarification
\item Armes magiques détruites
\item Charge d'un personnage hors de son unité, clarification
\item Restrictions d'armée et catégories d'unités,  clarification
\item Tenir la position et tirer, clarification
\item Mesurer les distances, ajout de la définition de \og dans un rayon de X pas \fg{}
\item \emph{Caché}, clarification
\item \emph{Reformation}, l'unité peut changer de direction
\item \emph{Parade}
\item Profil combiné
\item Objets magiques, qui est affecté
\item Distribuer les touches, les gabarits ne peuvent pas toucher les champions
\item Interaction entre \emph{Encombrant} et \emph{Tir rapide}
\end{itemize}

\noindent \textbf{Mise en page, noms}
\begin{itemize}
\item Restructuration du livre
\item Apparition de \emph{Terrain dangereux (X)}
\item \emph{Faux} remplacé par \emph{Touches d'impact (+1)}
\item \emph{Perforant (+X)} ajouté
\item Arme de base additionnelle: remplacé par Paire d'armes
\item Objectifs secondaires: mise en place déplacée dans la mise en place de la partie. Ajout d'une courte description.
\item Cristal d'anti-magie, nouveau nom: Couronne de Moquerie
\item \emph{Difficile à manier}, la regle devient \emph{Rechargez!} (nombreuses confusions avec \emph{Encombrant})
\item \emph{Reformation gratuite}, nouveau nom: \emph{Troupes légères}, pour éviter la confusion avec la manœuvre de \emph{Reformation rapide}
\end{itemize}

