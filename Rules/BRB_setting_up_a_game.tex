% Base sur la VO 0.11.9
% Relecture technique: 
% Relecture syntaxique: 

\part{Préparer une partie}

\section{Construire votre liste d'armée}
\label{construire_liste}


Une armée se compose d'une sélection de figurines, qui représentent des unités, décrites dans les livres d'armée. \emph{Le 9\ieme Âge} propose sa propre série de livres d'armée compatibles dans lesquels les joueurs peuvent trouver les règles, options et coûts en points associés à chaque figurine. Elles sont réparties entre des \emph{Personnages}, \emph{Seigneurs} et \emph{Héros}, et des troupes, divisées en \emph{Bases}, \emph{Spéciales} et \emph{Rares}.

Les \emph{Seigneurs} sont les individus les plus puissants d'une armée. Les \emph{Héros} sont des individus aux capacités exceptionnelles. Les \emph{Unités de Base} représentent les guerriers les plus communs d'une armée. Les \emph{Unités spéciales} sont des régiments d'élite ou de vétérans. Enfin, les \emph{Unités Rares} sont des troupes ou des machines de guerre peu communes...

Les armées qui s'affrontent correspondent généralement à un même total de points, qui donne une idée de la taille de l'affrontement. La première étape de la construction d'une armée est d'écrire un choix d'unités, leurs options et leurs coûts en points sur un document qu'on appelle \textbf{liste d'armée}. La composition d'une liste est sujette à des règles et restrictions, que l'on décrit en détails dans le paragraphe suivant.

\section{Coût en points}

Dans ce jeu, nous utilisons des montants en points pour chaque unité, arme, option, objet magique, etc. Le coût d'une unité est la somme des points que coûtent les figurines et options qu'elle contient. Le total des valeurs en points de toutes les unités d'une armée donne la valeur en points de l'armée.

\textbf{Demi-point}. Le coût d'une unité doit être entier. Si une amélioration coûte 0,5 point, le coût final de l'unité est arrondi à la valeur supérieure.

\textbf{Seigneur / Héros}. Si le prix d'une option ou d'un objet magique est écrit avec un "/" (par exemple, \emph{Épée de Géant} (\unit{60}{pts} / \unit{50}{pts})), la première valeur est le coût pour un \emph{Seigneur}, et la seconde pour un \emph{Héros} ou un \emph{Champion}.

\section{Restrictions}

Dans \emph{Batailles Fantastiques : Le 9\ieme Âge}, une armée doit suivre des règles simples de composition décrites dans la table \ref{table/restrictions}.

\begin{table}[!htbp]
\centering
\begin{tabular}{rcc}
\hline
 & \textbf{Limite de points} & \textbf{Limite de duplication} 			\tabularnewline
\hline
\textbf{Base} 				& min. 25 \% 			& \nouveau{max. 4} 	\tabularnewline
\textbf{Spécial} 			& - 					& max. 3 			\tabularnewline
\textbf{Rare} 				& max. 25 \% 			& max. 2 			\tabularnewline
\textbf{Héros} 				& max. \nouveau{50 \%} 	& \nouveau{max. 3} 	\tabularnewline
\textbf{Seigneurs} 			& max. \nouveau{35 \%} 	& \nouveau{max. 3} 	\tabularnewline
\textbf{Héros + Seigneurs} & max. 50 \% 			& - 				\tabularnewline
\hline
\end{tabular}
\caption{\label{table/restrictions}Restrictions de composition d'armée.}
\end{table}

\subsection*{Notes additionnelles}

\begin{itemize}[label={-}]
\item \textbf{Valeur en points de l'armée} : Le total des valeurs en points de toutes les unités de l'armée, en comptant les options et équipements, ne doit pas dépasser la limite de points décidée pour la bataille. Il peut être en dessous de 20 points au maximum. Les limites en pourcentage des différentes catégories, comme les unités de base, sont toujours calculées par rapport à la limite en points de la bataille, même si le total de l'armée est inférieur.
\item \textbf{Catégories d'unités} : Chaque unité du jeu appartient à une des 5 catégories. La limite maximum des points qu'on peut dépenser est différent d'une catégorie à une autre. De plus, une même unité ne peut être sélectionnée qu'un nombre limité de fois qui dépend aussi de sa catégorie. Parfois, certaines unités peuvent être déplacées d'une catégorie à une autre. Par exemple, une armée de Guerriers des Dieux Sombres commandée par un Général sur Char est autorisée à prendre un Char comme choix d'unité de base au lieu de choix d'unité spéciale. Dans de tels cas, il faut respecter à la fois la limite de duplication de son ancienne catégorie et de sa nouvelle. La limite de points est prise, quant à elle, à partir de la nouvelle catégorie. Dans notre exemple précédent, vous ne pouvez pas inclure plus de 3 Chars, mais ceux-ci compteront dans les 25 \% d'unités de base.
\item \textbf{Taille d'armée minimale} : \nouveau{Chaque armée doit contenir \textbf{au moins 4 unités}, sans compter les \emph{Personnages}. On ne peut pas comptabiliser qu'une seule unité du type \emph{Machine de Guerre} pour atteindre ce minimum}.
\item \textbf{Général} : Un \emph{Personnage} de l'armée doit être nommé \emph{Général}. Il faut donc qu'il y ait au moins un \emph{Personnage} dans l'armée capable de porter ce titre. Il ne peut y avoir qu'un seul \emph{Général}.
\item \textbf{Unique} : \nouveau{Les unités marquées \emph{Unique} ne peuvent être représentées qu'une fois par armée}.
\item \textbf{Limite spéciale} : Certaines unités disposent d'une limite de duplication différente qui est alors spécifiée.
\end{itemize}

\section{Patrouilles et grandes armées}

Les règles de composition d'armée peuvent être modifiées selon la taille de la partie, en dessous de 1500 points et au-dessus de 4000 points.

Les armées de 1500 points ou moins sont appelées \emph{Patrouilles}. Dans de telles armées, les limites de duplication habituelles sont divisées par deux en arrondissant au supérieur. De plus, la taille d'armée minimale est réduite à 3 unités.

Les armées de 4000 points ou plus sont appelées \emph{Grandes Armées}. Dans de telles armées, les limites de duplication habituelles sont doublées. De plus, les unités marquées \emph{Unique} peuvent exceptionnellement être prises en deux exemplaires, sauf les \emph{Personnages} et les objets magiques.

La table \ref{table/patrouille_et_grande_armee} résume les limites de duplication dans ces cas.

\begin{table}[!htbp]
\centering
\begin{tabular}{rcc}
\hline
 							& \textbf{Patrouille} 	& \textbf{Grande armée} \tabularnewline
\hline
\textbf{Base} 				& max. 2	 			& max. 8	 			\tabularnewline
\textbf{Spécial} 			& max. 2	 			& max. 6	 			\tabularnewline
\textbf{Rare} 				& max. 1	 			& max. 4	 			\tabularnewline
\textbf{Héros et Seigneurs} & max. 2	 			& max. 6	 			\tabularnewline
\textbf{Unique}			    & max. 1	 			& max. 2	 			\tabularnewline
\textbf{Limite spéciale} 	& divisée par 2			& doublée				\tabularnewline
\hline
\end{tabular}
\caption{\label{table/patrouille_et_grande_armee}Limites extrêmes de duplication des armées.}
\end{table}

\section{Liste d'armée ouverte ?}

Les règles de ce jeu ont été équilibrées avec l'idée de listes d'armée complètement révélées. Par exemple, votre adversaire doit savoir quels objets magiques vous possédez. Nous encourageons les joueurs à partager leur liste d'armée complète avec leur adversaire au début de la partie, en détaillant unités, options, objets magiques, capacités spéciales, coût en points, etc. Les seules choses qui ne doivent pas être montrées à votre adversaire sont celles explicitement citées comme cachées ou secrètes, comme un Assassin caché. Précisons que l'Assassin et son équipement sont quand même mentionnés dans la liste.

\subsection*{Listes cachées}
\label{liste_cachee}

Rappelez-vous quand même que les règles générales ont été équilibrées pour des listes ouvertes. Il peut cependant arriver que des joueurs préfèrent jouer avec des listes \og cachées \fg . Dans ce cas, nous vous proposons de suivre les règles suivantes. La plus grande partie de votre armée doit rester connue de votre adversaire avant le début de la partie. Seuls quelques aspects sont secrets, ou \og cachés \fg : les Objets magiques, et ce qui suit les règles des Objets magiques, comme les \emph{Dons Démoniaques} et les \emph{Runes Naines}. Le reste doit être présenté dans la partie ouverte de votre liste. De plus, les Objets magiques qui ont un équivalent standard doivent être présentés sur la liste ouverte comme leur équivalent standard. Une Lance magique, par exemple, apparait comme une simple Lance. Quand vous possédez au moins deux unités qui sont identiques dans la liste ouverte, mais qui ont une différence cachée, comme par exemple une Bannière magique, vous devez avoir un moyen visible de les différencier, noté sur votre liste cachée. Par exemple, la bannière rouge est la Bannière magique, tandis que la bleue est ordinaire.

\subsection*{Révéler les objets magiques}

Un Objet magique, ou équivalent, doit être révélé à la première utilisation. Un objet est considéré comme utilisé quand il a une chance d'affecter le jeu d'une quelconque façon. Entre autres :
\begin{itemize}[label={-}]
\item si cela affecte un jet de dé, même si le résultat obtenu sur le dé ne déclenche pas d'effet ;
\item si cela altère une attaque, via une Arme magique ou un objet qui ajoute une règle spéciale à l'attaque ;
\item si cela altère un jet de sauvegarde. L'objet doit alors être révélé avant que les dés ne soient jetés. Remarquez qu'un objet qui change une sauvegarde de la même manière que son équivalent standard le ferait, comme un Bouclier magique, ne doit pas être révélé.
\end{itemize}

Un objet qui augmente le Mouvement ne compte comme étant utilisé que quand l'unité se déplace plus qu'elle ne le pourrait sans l'objet, ou quand elle charge. Déclarez alors l'objet avant de lancer le jet de distance de charge, mais après que les réactions ont été déclarées. Pour les \emph{Objets Runiques Nains}, ne révélez que la \emph{Rune} qui est utilisée, pas la combinaison complète.

