% Base sur la VO 0.11.9
% Relecture technique: 
% Relecture syntaxique: 

\part{Introduction}

\section{BF : Le 9\ieme Âge, qu'est-ce que c'est ?}

\emph{Batailles Fantastiques : Le 9\ieme Âge} est un jeu de figurines créé par une communauté, sans lien avec une quelconque entreprise. Le jeu met en scène deux armées qui s'affrontent, représentées par des figurines adéquates. Chaque armée est contrôlée par un joueur, bien qu'en certaines occasions, il puisse y avoir plus d'un joueur par camp, comme dans le cas d'une alliance. Le jeu se joue sur une planche de 4{\pied} par 6{\pied} (environ 1,20 {\meter} par 1,80 {\meter}), bien que pour des batailles plus petites ou plus grandes, la taille du plateau puisse être adaptée. Les deux adversaires déploient leurs armées et jouent tour à tour pour faire agir leurs figurines. Chaque action a une certaine chance de réussite, ce pourquoi nous utilisons des dés. Pendant chaque tour de joueur, un joueur peut faire agir son armée pendant quatre phases de jeu consécutives : Mouvement, Magie, Tir et Corps à Corps. À la fin de la partie, on détermine le vainqueur ou on déclare un match nul.

Toutes les règles du jeu, ainsi que les retours et suggestions peuvent être trouvés et donnés ici :
\url{http://www.the-ninth-age.com/}

\nouveau{Pour simplifier le passage aux nouvelles règles aux anciens joueurs, les changements les plus importants par rapport à la huitième édition sont colorés comme ce paragraphe}. 
\newrule{Les changements du livre de règles intervenant d'une version à l'autre sont colorés en vert (voir également le journal des modifications en fin d'ouvrage).} 

Copyright Creative Commons licence : \url{http://www.the-ninth-age.com/license.html}

\section{Note des traducteurs sur les distances}

L'unité de mesure dans \emph{Batailles Fantastiques : Le 9\ieme Âge} est le pouce anglais, ou \og pouce technique international \fg , et se note \pouce . Il vaut 2,54 {\centi\meter} exactement. Toutes les distances et portées sont indiquées et mesurées en pouces. Le pied, noté \pied , vaut 12 pouces. Ce système d'unités, datant du Moyen-Âge, est encore utilisé aujourd'hui dans quelques régions du monde comme les États-Unis ou le Royaume-Uni, berceau de ce jeu.


\newpage
\section{L'échelle du jeu...}

Jouer à un jeu de figurines est souvent un exercice d'abstraction, en particulier lorsqu'il s'agit d'un jeu de batailles de masse comme le Neuvième Âge. Il n’y a donc pas d’échelle recommandée lors d’une partie du Neuvième Âge : une figurine pourrait représenter un seul, une douzaine, ou même une centaine de guerriers. Même si les joueurs sont encouragés à interpréter cette échelle comme ils le souhaitent, les distances utilisées dans les règles ne semblent pas réalistes en comparaison de la taille des figurines utilisées pour le jeu. L’échelle des figurines utilisées pour le Neuvième Âge est environ de 1/72. Cela signifie qu’ 1\pouce{} dans le jeu correspondrait à peu près à 1,5 \meter{} en réalité. Une figurine typique de taille humaine a une valeur de mouvement de 4\pouce{}, ce qui signifie qu’en une phase de mouvement, elle se déplacerait de seulement 6 mètres (ou 12 mètres en cas de marche forcée). De même, une arme de tir telle qu’un arc long a une portée de 30\pouce{} dans le jeu, ce qui équivaudrait environ à 45 mètres – ce qui représente à peine 20 \% de la portée historique de ces armes (environ 250 mètres).

Par exemple, les joueurs pourraient se servir de la portée réelle d’un arc long pour déterminer approximativement la distance qu’ 1\pouce{} représente dans le jeu. Ainsi, 1\pouce{} représenterait un peu plus de 8 mètres, ce qui serait plus proche des distances prises en considération en écrivant les règles de ce jeu.

De même que nous pouvons imaginer que les combattants du jeu sont en réalité plus petits que les figurines qui les personnifient, nous pouvons imaginer qu’une figurine ne représente pas forcément un seul guerrier. Par abstraction, nous pourrions envisager qu’une unité de 10 guerriers d’élite elfes représente exactement 10 elfes, ou davantage : 20, 50, ou même 100. De façon similaire, une unité de 10 gobelins gringalets pourrait représenter juste 10 gobelins, ou plus probablement un groupe plus important tel que 100, 200 ou 500 avortons. On peut alors se poser la question des personnages et des monstres : ces figurines sont faites pour incarner des individus exceptionnels et des créatures toutes puissantes, aussi dangereux et influents que des régiments entiers. Si cela facilite les choses, on peut supposer que la figurine d’un personnage représente non seulement cet individu, mais aussi la garde du corps et le personnel qui ne manqueraient pas de l’escorter sur le champ de bataille.

De façon semblable, les éléments de décor peuvent être interprétés comme les éléments exacts qu’ils sont, mais ils pourraient aussi être les représentations visuelles de paysages plus vastes dans le jeu. Ainsi, un bosquet d’arbres pourrait évoquer une forêt entière, un ruisseau pourrait évoquer une large rivière, une maison pourrait évoquer un hameau, une tour pourrait évoquer un fort.

L’échelle de temps, quant à elle, est encore plus arbitraire que l’échelle physique du jeu. Les déplacements durant la phase de mouvement pourraient prendre plusieurs minutes de temps réel, alors que les sorts et les tirs pourraient être des évènements quasi-instantanés. Le combat féroce de deux unités en corps à corps pourrait ne durer que quelques battements de cœur, tandis qu’un défi entre deux puissants personnages pourrait être un combat prolongé, qui dure plusieurs minutes ou davantage. Ainsi, aucune mesure de temps ne peut réellement être associée à un tour ou à une phase de jeu.

Il n’est pas de notre ressort d’imposer aux joueurs une façon d’imaginer leurs affrontements, ou combien d’individus chaque figurine est censée représenter ; mais nous pensons que la conversion simple d’ 1\pouce{} représentant environ 10 \meter{} est une interprétation cohérente de la taille du jeu que nous avons créé. Une partie de taille moyenne sera jouée sur une table d’environ 72\pouce{} x 48\pouce{} (soit 1,80 \meter{} x 1,20 \meter{}), ce qui représente une zone d’environ 720 \meter{} x 480 \meter{} – l’équivalent d’environ 50 terrains de football. Au Moyen-Âge (la période historique la plus similaire à notre monde fantastique), cela correspondrait à un champ de bataille de taille moyenne, où deux armées de quelques centaines à quelques milliers de soldats auraient pu se rencontrer.
