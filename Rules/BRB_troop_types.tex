% Base sur la VO 0.11.9
% Relecture technique: 
% Relecture syntaxique: 

\part{Types de troupe}
\label{types_de_troupe}

Toutes les figurines ont un \emph{Type de Troupe}, chacun ayant un certain nombre de règles spéciales décrites ci-dessous. Certaines règles peuvent également changer en fonction du \emph{Type de Troupe} de la figurine ou de l'unité. Le \emph{Type de Troupe} d'une unité est celui de la majorité de ses figurines ordinaires et \emph{Champions}, l'adversaire tranchant en cas d'égalité. Si une unité est constituée uniquement de \emph{Personnages}, utilisez le \emph{Type de Troupe} majoritaire, l'adversaire décidant en cas d'égalité. Le \emph{Type de Troupe} de figurines individuelles n'a pas à être le même que celui de leur unité.

Les figurines d'un certain \emph{Type de Troupe} ont toujours les règles associées à leur \emph{Type de Troupe} :
\begin{itemize}[label={-}, itemsep=0.2cm]
\item \textbf{Infanterie} : pas de règles additionnelles.
\item \textbf{Infanterie Monstrueuse} : \emph{Piétinement (1)}, \emph{Rangs Monstrueux}, \emph{Soutien Monstrueux}.
\item \textbf{Cavalerie} : \emph{Profil Combiné}, \emph{Protection de Monture}, \emph{Rapide}, \emph{Soutien de Cavalerie}.
\item \textbf{Cavalerie Monstrueuse} : \emph{Piétinement (1)}, \emph{Profil Combiné}, \emph{Protection de Monture}, \emph{Rangs Monstrueux}, \emph{Rapide}, \emph{Soutien de Cavalerie}, \emph{Soutien Monstrueux}.
\item \textbf{Char} :  \emph{Pas de Marche Forcée}, \emph{Profil Combiné}, \emph{Protection de Monture}, \emph{\nouveau{Rangs  Monstrueux}}, \emph{Rapide}, \emph{Soutien de Cavalerie}, \emph{Touches d'Impact (1D6)}.
\item \textbf{Bête de Guerre} : \emph{Rapide}.
\item \textbf{Bête Monstrueuse} : \emph{Piétinement (1)}, \emph{Rangs Monstrueux}, \emph{Rapide}, \emph{Soutien Monstrueux}.
\item \textbf{Machine de Guerre} : \emph{Mouvement ou Tir}, \emph{\nouveau{Pas de Marche Forcée}}, \emph{Profil de Machine de Guerre}, \emph{Rechargez !}.
\item \textbf{Monstre} : \emph{Grande Cible}, \emph{Piétinement (1D6)}, \emph{Rangs de Monstre}, \emph{Terreur}.
\item \textbf{\nouveau{Monstre Monté}} :  \emph{Grande Cible}, \emph{Piétinement (1D6)}, \emph{Profil de Monstre Monté}, \emph{Rangs de Monstre}, \emph{Terreur}.
\item \textbf{Nuée} : \emph{Indémoralisable}, \emph{Instable} ,\emph{Tirailleurs}.
\end{itemize}

\section{Figurines à pied ou montées}

Certains sorts et règles peuvent affecter les figurines montées et les figurines à pied différemment. À tous points de vue, les \emph{Bêtes de Guerre}, les \emph{Bêtes Monstrueuses}, l'\emph{Infanterie}, l'\emph{Infanterie Monstrueuse}, les \emph{Machines de Guerre}, les \emph{Monstres} et les \emph{Nuées} sont considérés comme des figurines \textbf{à pied}, à moins d'être des montures (comme un palanquin), auquel cas elles sont considérées comme des figurines montées. La \emph{Cavalerie}, la \emph{Cavalerie Monstrueuse}, les \emph{Chars} et les \emph{Monstres Montés} sont considérés comme des figurines \textbf{montées}.

\subsection{Personnage monté}
Certains personnages ont la possibilité de choisir une monture de la section "monture" d'un Livre d'armée. Quand une figurine est montée, son \emph{Type de Troupe} change :
\begin{itemize}[label={-}]
\item Une figurine montée sur une \emph{Bête de Guerre} devient de la \emph{Cavalerie}.
\item Une figurine montée sur une \emph{Bête Monstrueuse} devient de la \emph{Cavalerie Monstrueuse}.
\item \nouveau{Une figurine montée sur un \emph{Monstre} devient un \emph{Monstre Monté}}.
\item \nouveau{Une figurine montée sur n'importe quel autre \emph{Type de Troupe} devient le \emph{Type de Troupe} de sa monture} (mais compte comme une figurine montée et non à pied). Dans ce cas, la figurine gagne les règles de \emph{Profil Combiné} et \emph{Protection de Monture}.
\end{itemize}
L'ensemble de la figurine (Personnage, monture et, éventuellement, cavaliers ou membres d'équipage supplémentaires) suit les règles des Personnages. Les cavaliers ou membres d'équipage appartenant à la monture choisie ne comptent pas comme des "montures" (ce qui signifie qu'ils peuvent utiliser leurs armes ou armures, par exemple). Ces cavaliers ou membres d'équipage ont leur propre équipement qu'ils peuvent utiliser. Cependant, la figurine combinée n'a qu'une seule sauvegarde d'armure. Il faut choisir d'utiliser l'armure des cavaliers ou membres d'équipages (en comptant la \emph{Protection de Monture} et l'éventuel Caparaçon) ou l'armure du Personnage (en comptant la \emph{Protection de Monture} et l'éventuel Caparaçon).

\subsubsection*{Protection de monture}

Les figurines ayant cette règle donnent +1 à la sauvegarde d'armure de leur cavalier, comme décrit dans le paragraphe \ref{equipement_armure} à la page \pageref{equipement_armure}.

\subsubsection*{\nouveau{Rangs de monstre}}

\nouveau{L'unité a besoin d'une seule figurine pour former un \emph{Rang Complet}}.

\subsubsection*{Rangs monstrueux}

L'unité a seulement besoin d'un minimum de trois figurines dans un rang pour qu'il soit considéré comme un \emph{Rang Complet} et d'un minimum de six figurines par rang pour être considérée comme une formation de \emph{Horde}.

\subsubsection*{Soutien de cavalerie}

Les montures et l'attelage ne peuvent pas faire d'attaque de \emph{Soutien}.

\subsubsection*{Soutien monstrueux}

Les figurines avec cette règle spéciale peuvent faire jusqu'à trois attaques de \emph{Soutien} au lieu d'une seule. Notez que les montures ne peuvent toujours pas faire d'attaques de \emph{Soutien}.


\section{Règles spéciales des types de troupe}

\subsubsection*{Profil combiné}

Les figurines ayant cette règle spéciale ont des profils différents pour chacun de leurs éléments. Les figurines de \emph{Cavalerie} ont généralement un profil pour la monture et un profil pour le cavalier. Les figurines de \emph{Char} ont généralement un profil pour le véhicule, un profil pour les membres d'équipage et un profil pour les montures. Certains chars n'ont qu'un seul profil pour le véhicule et les montures.

Quand ils attaquent, chaque élément de la figurine utilise ses propres caractéristiques. Chaque élément de la figurine peut faire une attaque de tir dans la même phase, mais ils doivent tous choisir la même cible. Un membre de l'équipage d'un \emph{Char} peut choisir de tirer avec une arme de tir du \emph{Char} à la place de sa propre arme de tir.

Dans toutes les autres situations, la figurine au profil combiné est considérée comme une seule figurine avec les caractéristiques suivantes.

\begin{itemize}[label={-}]
\item \textbf{Mouvement}. Utiliser le Mouvement de la monture ou de l'attelage.
\item \textbf{Capacité de Combat}. Utiliser la CC du ou des cavalier(s), en utilisant la CC la plus haute s'il y a plusieurs cavaliers.
\item \textbf{Endurance}. Utiliser la plus haute caractéristique \nouveau{d'Endurance} disponible parmi tous les éléments de la figurine.
\item \textbf{Points de Vie}. Utiliser la plus haute caractéristique de PV disponibles parmi tous les éléments de la figurine.
\item \textbf{Commandement}. Utiliser le Cd du ou des cavalier(s), en utilisant le Cd le plus haut s'il y a plusieurs cavaliers.
\item \textbf{Sauvegarde}. Utiliser la meilleure sauvegarde, sauvegarde invulnérable et régénération disponible parmi un des éléments de la figurine. La sauvegarde d'armure d'un cavalier peut être améliorée par la \emph{Protection de la monture} et un \emph{Caparaçon}.
\end{itemize}

\newrule{Les autres caractéristiques sont ignorées sauf pour passer un test de Caractéristique. Dans un tel cas, utiliser la plus haute caractéristique disponible parmi tous les éléments de la figurine.}

\subsubsection*{\nouveau{Profil de monstre monté}}
\label{profil_monstre_monte}

Les figurines ayant cette règle spéciale ont des profils différents : un profil pour la monture et un profil pour le (ou les) cavalier(s). Quand elles sont attaquées, les attaques sont obligatoirement résolues contre le \emph{Monstre}.

Quand ils attaquent ou tirent, chaque élément de la figurine utilise ses propres caractéristiques. Chaque élément de la figurine peut faire une attaque de tir dans la même phase, mais ils doivent tous choisir la même cible. Le cavalier d'un \emph{Monstre} peut choisir de tirer avec une arme de tir portée par le \emph{Monstre} à la place de sa propre arme de tir.

Dans toutes les autres situations, la figurine au profil combiné est considérée comme une seule figurine avec les caractéristiques suivantes. 


\begin{itemize}[label={-}]
\item \textbf{Mouvement}. Utiliser le Mouvement du \emph{Monstre}.
\item \textbf{Capacité de Combat}. Utiliser la CC du \emph{Monstre}.
\item \textbf{Endurance}. Utiliser l'Endurance du \emph{Monstre}.
\item \textbf{Points de Vie}. Utiliser les PV du \emph{Monstre}. Quand le \emph{Monstre} tombe à 0 PV, retirez tous les éléments de la figurine comme pertes.
\item \textbf{Initiative}. Utiliser l'Initiative du \emph{Monstre} pour tous les tests d'Initiative.
\item \textbf{Commandement}. Utiliser le Cd du ou des cavalier(s), en utilisant le Cd le plus haut s'il y a plusieurs cavaliers.
\item \textbf{Sauvegarde}. Utiliser les sauvegardes du \emph{Monstre}. Ainsi, toute armure portée par le cavalier, sa propre \emph{Sauvegarde Invulnérable} ou \emph{Régénération} n'ont pas d'effet, à moins que le contraire ne soit précisé. Les \emph{Monstres Montés} peuvent uniquement avoir une armure de type \emph{Protection Innée}. Toute autre sorte d'armure est ignorée.
\end{itemize}

\newrule{Les autres caractéristiques sont ignorées sauf pour passer un test de Caractéristique, hormis un test d'Initiative. Dans un tel cas, utiliser la plus haute caractéristique disponible parmi tous les éléments de la figurine.}

\subsubsection*{\nouveau{Profil de machine de guerre}}

Les \emph{Machines de Guerre} ont des profils différents pour la machine et ses servants. Utilisez les PVs de la \emph{Machine de Guerre}. Utilisez  l'Endurance de la machine contre toute attaque à distance et les caractéristiques des servants pour toutes les autres situations. Les \emph{Machines de Guerre} ratent automatiquement tout test de caractéristique sauf les tests de Commandement et ne peuvent pas faire de \emph{Marche Forcée}, déclarer des charges, poursuivre une unité en fuite ou choisir de fuir en réaction à une charge. Si une \emph{Machine de Guerre} rate un test de \emph{Panique}, elle ne fuit pas, mais ne peut pas tirer lors de sa prochaine phase de tir. Si elle rate un test de \emph{Moral}, elle est simplement détruite. Si une \emph{Machine de Guerre} tombe à 0 PV, la figurine et tous ses éléments sont retirés du champ de bataille comme perte. Les \emph{Personnages} ne peuvent jamais rejoindre des \emph{Machines de Guerre}.

Quand une unité charge une \emph{Machine de Guerre}, suivez les règles normales de charge, excepté le fait que l'unité qui charge peut être amenée en contact n'importe où, les \emph{Machines de Guerre} ayant des socles ronds et n'ayant ainsi ni front, flancs ou arrière. Ignorez l'orientation de la \emph{Machine de Guerre} ainsi que la maximisation du nombre de figurines au contact socle à socle. L'unité qui charge doit cependant avoir son front en contact avec le socle de la \emph{Machine de Guerre}. Les \emph{Machines de Guerre} et les unités engagées au corps à corps contre elles ne peuvent pas effectuer de \emph{Reformation de Combat}. Au corps à corps, n'allouez pas les attaques de façon habituelle. Le joueur attaquant la \emph{Machine de Guerre} choisit jusqu'à six figurines qui ne sont pas en contact avec d'autres ennemis. Il doit en choisir 6 ou autant que possible. Les \emph{Bêtes Monstrueuses}, la \emph{Cavalerie Monstrueuse}, les \emph{Chars} et l'\emph{Infanterie Monstrueuse} comptent pour 3 figurines chacun, et les \emph{Monstres}, montés ou non, comptent comme 6 figurines chacun. Seules ces figurines peuvent attaquer la \emph{Machine de Guerre} comme si elles étaient en contact socle à socle, et la \emph{Machine de Guerre} ne peut riposter que sur les figurines sélectionnées. Les pertes dans l'unité attaquante sont retirées normalement à l'arrière de l'unité. Si des figurines ordinaires parmi celles choisies pour l'assaut meurent avant d'avoir pu frapper, elles peuvent être remplacées. Les tests de \emph{Moral} sont faits normalement, à l'exception du fait qu'une \emph{Machine de Guerre} est détruite immédiatement si elle rate un test de \emph{Moral}.

\newpage
\section*{Résumé des différents types de troupes}

\renewcommand{\arraystretch}{1.48}
\begin{table}[!htbp]
\centering
\begin{tabular}{M{2.5cm}|M{2cm}M{2cm}M{3cm}M{2.3cm}M{2cm}}
 & \textbf{Rang complet (horde)} & \textbf{Soutien} & \textbf{Règles spéciales} & \textbf{Mouvement} & \textbf{Taille} \\
\hline
\textbf{Infanterie} & 5 (10) & 1 & - & - & Petite \\
\hline
\textbf{Bête de Guerre} & 5 (10) & 1 & - & \emph{Rapide} & Petite \\
\hline
\textbf{Cavalerie} & 5 (10) & 1 (cavalier seulement) & \emph{Profil Combiné} \newline \emph{Protection de Monture} & \emph{Rapide} & Moyenne \\
\hline
\textbf{Infanterie Monstrueuse} & 3 (6) & 3 & \emph{Piétinement (1)} & - & Moyenne \\
\hline
\textbf{Bête Monstrueuse} & 3 (6) & 3 & \emph{Piétinement (1)} & \emph{Rapide} & Moyenne \\
\hline
\textbf{Cavalerie Monstrueuse} & 3 (6) & 3 & \emph{Piétinement (1)} \newline \emph{Profil Combiné} \newline \emph{Protection de Monture}  & \emph{Rapide} & Moyenne \\
\hline
\textbf{Char} & 3 (6) & 1 \newline (1 membre d'équipage seulement) & \emph{Profil Combiné} \newline \emph{Protection de Monture} \newline \emph{Touches d'Impact (1D6)} & \emph{Pas de Marche Forcée} \newline \emph{Rapide} & Moyenne \\
\hline
\textbf{Monstre} & 1 & - & \emph{Piétinement (1D6)} \newline \emph{Terreur} & - & \emph{Grande Cible} \\
\hline
\textbf{Monstre Monté} & 1 & - & \emph{Piétinement (1D6)} \newline \emph{Profil de Monstre Monté} \newline \emph{Terreur} & - & \emph{Grande Cible} \\
\hline
\textbf{Nuée} & * (10) & 1 & \emph{Indémoralisable} \newline \emph{Instable} \newline \emph{Tirailleurs} & - & Petite \\
\hline
\textbf{Machine de Guerre} & Pas de rangs & - & \emph{Mouvement ou Tir} \newline \emph{Profil de Machine de Guerre} \newline \emph{Rechargez !} & \emph{Pas de Marche Forcée} & Petite \\
\hline
\end{tabular}
* Les Nuées ne peuvent pas avoir de rang complet vu qu'elles ont la règle \emph{Tirailleurs}
\end{table}
\renewcommand{\arraystretch}{1.5}