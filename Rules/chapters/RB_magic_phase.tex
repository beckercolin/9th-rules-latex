
\part{Phase de Magie}

Lors de la Phase de Magie, vos Sorciers peuvent lancer des sorts et votre adversaire peut tenter de les dissiper.

\section{Sorciers}

Les figurines pouvant lancer des sorts non liés à des \boundspells{} sont appelées Sorciers. Tous les Sorciers disposent de la règle \channel{}.

\paragraph{Niveau de Magie} 

Le Niveau de Magie d'un Sorcier indique le nombre de sorts qu'il connaît. Si un Sorcier perd des Niveaux de Magie, il perd un sort par niveau. Sauf mention contraire, les sorts perdus sont déterminés aléatoirement. Un Niveau de Magie ne peut pas descendre en dessous de 0, mais le Sorcier continue à perdre des sorts quand il perd des niveaux. Les Sorciers de niveau 0 sont toujours des Sorciers à tous égards.

\paragraph{Sorcier Apprenti} 

\newfromWHB{Les Sorciers de niveaux 1 et 2 sont appelés des Sorciers Apprentis. Ils ajoutent \textbf{+1} au résultat de leurs jets de lancement et de dissipation de sorts.}

\paragraph{Maître Sorcier} 

\newfromWHB{Les Sorciers de niveaux 3 et 4 sont appelés des Maîtres Sorciers. Ils ajoutent \textbf{+2} au résultat de leurs jets de lancement et de dissipation de sorts.}

Si le niveau d'un Sorcier augmente ou diminue, quelle qu'en soit la raison, son bonus de lancement et de dissipation change aussi en accord avec les paragraphes ci-dessus.

\section{Sorts}

Les sorts peuvent être lancés pendant la Phase de Magie. Les sorts connus par un Sorcier sont normalement déterminés au hasard avant le début de la partie, en utilisant les règles du paragraphe \ref{generating_spells} (page \pageref{generating_spells}). La plupart des sorts font partie d'une Voie Magique. Chacun de vos Sorciers doit choisir une Voie Magique parmi celles auxquelles il a accès et générer ses sorts depuis cette Voie. La Voie choisie doit être inscrite dans la Liste d'Armée. Tous les sorts sont définis par les cinq critères suivants.

\paragraph{Nom} 

Utilisez le nom du sort pour annoncer le sort que le Sorcier va tenter de lancer.

\paragraph{Valeur de Lancement} 

C'est la valeur minimale à obtenir pour lancer le sort avec succès. Les sorts peuvent avoir différentes Valeurs de Lancement (voir le paragraphe \ref{boosted_spells}).

\paragraph{Type}

Le Type du sort détermine quelles cibles peuvent être choisies. Un sort peut avoir plusieurs types ; leurs restrictions se cumulent. Par exemple, un sort de type \range{12}, \hex{} et \direct{} ne peut cibler que les unités ennemies à moins de \distance{12} dans l'arc frontal du lanceur. Sauf mention contraire, un sort ne peut viser qu'une cible.

\paragraph{Durée}

La Durée d'un sort indique le temps pendant lequel ses effets restent actifs.

\paragraph{Effet}

L'Effet d'un sort décrit ce qui arrive à la cible si le sort est lancé avec succès. \newfromWHB{L'Effet d'un sort n'est jamais affecté par les règles spéciales, les Objets Magiques, d'autres effets de sorts ou tout type de règles donnant des avantages au lanceur, sauf indication contraire.}

\subsection{Sorts améliorés}
\label{boosted_spells}

Certains sorts ont plus d'une Valeur de lancement. La ou les valeurs plus élevées correspondent à des versions améliorées du sort. Une version améliorée d'un sort peut en changer les effets, ou en modifier les restrictions, comme la portée ou les cibles. Avant de jeter les dés pour tenter de lancer un sort, déclarez si vous souhaitez utiliser une version améliorée du sort, et laquelle. Sans déclaration de votre part, on considère toujours que le sort n'est pas amélioré.

\subsection{Types de sorts}

Le Type du sort détermine les restrictions sur la ou les cibles du sort. À moins que le contraire ne soit précisé, la cible doit être une unique unité.

\paragraph{\augment}

\newfromWHB{Ne peut cibler que des unités alliées, ou des figurines alliées si de type \focused{}.}

\paragraph{\aura}

\newfromWHB{Ce sort a un effet de zone. Toutes les cibles réglementaires, en suivant les restrictions des autres types du sort, sont touchées. Par exemple, un sort de type \aura{}, \augment{} et \range{12} touche toutes les unités alliées à moins de \distance{12}.}

\paragraph{\damage}

\newfromWHB{La cible ne peut pas être engagée au Corps à Corps.}

\paragraph{\direct}

\newfromWHB{La cible doit se trouver dans l'arc frontal du lanceur.}

\paragraph{\focused}

\newfromWHB{Seules les figurines uniques, ce qui comprend les Personnages dans des unités, peuvent être prises pour cible. Si la cible est une figurine en plusieurs éléments, comme un Char avec deux membres d'équipage et deux montures, ou un cavalier et son cheval, seul un élément peut être pris pour cible.}

\paragraph{\linetemplate}

Tracez une ligne droite depuis le centre de l'avant du socle du lanceur, dans la direction de votre choix. \newfromWHB{Toutes les figurines touchées par cette ligne sont affectées par le sort.} La ligne compte comme un Gabarit.

\paragraph{\caster}

Ne peut cibler que la figurine qui lance le sort.

\paragraph{\hex}

\newfromWHB{Ne peut cibler que des unités ennemies, ou des figurines ennemies si de type \focused{}.}

\paragraph{\ground}

\newfromWHB{Ne cible pas des unités ou des figurines, mais un point sur le champ de bataille, choisi par le joueur lançant le sort.}

\paragraph{Personnage uniquement}

Ne peut cibler que des figurines de Personnage, monture comprise.

\paragraph{\range{X}}

Les sorts ont normalement une Portée maximale indiquée par \og \range{X} \fg{}. La cible doit être située à moins de \distance{X} du lanceur.

\paragraph{\missile}

Le lanceur doit avoir une Ligne de Vue sur la cible. Ni le lanceur ni son unité ne peuvent être engagés au Corps à Corps pour pouvoir lancer ce sort.

\paragraph{\castersunit}

Ne peut cibler que l'unité dans laquelle se trouve le lanceur.

\paragraph{\universal}

Peut cibler des unités alliées ou ennemies, et des figurines alliées ou ennemies si de type \focused{}.

\paragraph{\vortex{} (\range{X}, \template{} \distance{Y})}

Placez un Gabarit de taille \distance{Y} en contact avec le socle du lanceur, avec son centre dans l'arc frontal du lanceur, et lancez 1D6.
\begin{itemize}[label={-}]
\item \textbf{1 à 5} : \newfromWHB{Multipliez le résultat du dé par la valeur X de la Portée du \vortex{}.} Le résultat est la distance sur laquelle le Gabarit, dans la direction de la cible\newfromWHB{, qui est toujours un point sur le champ de bataille, puisque tous les sorts de \vortex{} sont aussi de type \ground{}}.
\item \textbf{6} : Centrez le gabarit sur le lanceur et déplacez-le de \distance{1D6} dans une direction aléatoire.
\end{itemize}

Toutes les figurines sur le passage du Gabarit, de sa position initiale à sa position finale, sont affectées par le sort. \newfromWHB{Une fois les effets appliqués, retirez le Gabarit du jeu et le sort prend fin.}

\subsection{Durée des sorts}

La Durée d'un sort indique pendant combien de temps ses effets persistent. Quatre Durées, décrites ci-dessous, sont possibles : \lastsoneturn{}, \instant{}, \permanent{} et \remainsinplay{}.

\paragraph{\lastsoneturn}

Les effets durent jusqu'au début de la prochaine Phase de Magie du lanceur. \newfromWHB{Si une unité affectée se divise en plusieurs unités, par exemple si un Personnage quitte l'unité, chacune des parties continue d'être affectée par le sort.} Les Personnages qui rejoignent une unité déjà affectée par le sort ne sont pas concernés.

\paragraph{\instant}

Le sort n'a pas de durée : les effets sont résolus une fois, puis le sort prend immédiatement fin.

\paragraph{\permanent}

\newfromWHB{Les effets durent jusqu'à la fin de la partie ou jusqu'à ce qu'une condition décrite par le sort soit atteinte. Ces effets ne peuvent être supprimés par aucun autre moyen que cette condition. Si une unité affectée se divise en plusieurs unités, par exemple si un Personnage quitte l'unité, chacune des parties continue d'être affectée par le sort.} Les Personnages qui rejoignent une unité déjà affectée par le sort ne sont pas concernés.

\paragraph{\remainsinplay}

Les effets durent jusqu'à ce que le sort soit dissipé ou que le lanceur soit tué. Le sort peut être dissipé lors de chaque Phase de Magie suivant le lancement (voir le paragraphe \ref{dispel_remains_in_play_spells}, page \pageref{dispel_remains_in_play_spells}). \newfromWHB{Si une unité affectée se divise en plusieurs unités, par exemple si un Personnage quitte l'unité, chacune des parties continue d'être affectée par le sort. Dans ce cas, une dissipation réussie permet de faire disparaître le sort sur toutes les unités affectées. Les Personnages qui rejoignent une unité déjà affectée par le sort ne sont pas concernés.} Tant que le sort est actif, il ne peut pas être relancé par le même lanceur. \newfromWHB{Si le lanceur est tué, le sort est automatiquement dissipé et ses effets disparaissent à la première occasion où il aurait pu être normalement dissipé (voir la partie \ref{magic_phase_sequence}).}

\newpage
\section{Séquence de la Phase de Magie}
\label{magic_phase_sequence}

La Phase de Magie est divisée en cinq étapes :

\hspace*{0.3cm}
\begin{tabular}{c|m{14cm}}
1 & Début de la phase. Jetez les dés pour les Flux de Magie et la \channel{}. \tabularnewline
2 & \newfromWHB{Les sorts de type \remainsinplay{} peuvent être dissipés (voir le paragraphe \ref{dispel_remains_in_play_spells}).} \tabularnewline
3 & Le Joueur Actif peut tenter de lancer un sort (voir la partie \ref{spell_casting_sequence} page \pageref{spell_casting_sequence}). \tabularnewline
4 & Répétez les étapes 2 et 3 jusqu'à ce qu'aucun joueur ne tente une autre action. \tabularnewline
5 & Fin de la phase. Les capacités prenant effet à la fin de la phase sont déclenchées. \tabularnewline
\end{tabular}

\subsection{Flux Magiques et Canalisation}

Pendant la Phase de Magie, les sorts sont lancés et dissipés avec des Dés de Magie, appelés différemment selon leur propriétaire. Le Joueur Actif dispose de Dés de Pouvoir tandis que le Joueur Réactif défend avec des Dés de Dissipation. Chaque joueur garde ses dés dans une réserve et peut piocher dedans pour lancer ou dissiper des sorts. Une réserve ne peut jamais contenir plus de \textbf{12} Dés de Magie, même temporairement\newfromWHB{, et aucun joueur ne peut utiliser plus de \textbf{12} Dés de Magie en une phase. De plus, ils ne peuvent ajouter qu'un maximum de \textbf{2} Dés de Magie à ceux générés par les Flux Magiques en une phase.}

Au début de la Phase de Magie, le Joueur Actif lance 2D6 pour le jet des Flux Magiques. Le nombre de Dés de Pouvoir disponibles est égal à la somme des résultats des deux dés, tandis que le nombre de Dés de Dissipation est égal au résultat le plus élevé parmi les deux dés. Les deux joueurs peuvent alors faire une unique tentative de \channel{}. \newfromWHB{Pour tenter une \channel{}, lancez 1D6. Ajoutez +1 au résultat de ce jet pour chaque figurine non en fuite possédant la règle \channel{} dans votre armée. Tous les Sorciers ont la règle \channel{}. Si le total est de \textbf{7} ou plus, vous pouvez ajouter un Dé de Magie à votre réserve.}

\subsection{Dissipation d'un sort restant en jeu}
\label{dispel_remains_in_play_spells}

\newfromWHB{Chaque joueur peut essayer de dissiper des sorts de type \remainsinplay{}, en commençant par le Joueur Réactif. Chacun peut dissiper ses propres sorts de type \remainsinplay{} automatiquement et sans utiliser de Dés de Magie, tandis que son adversaire devra faire une tentative de dissipation et ne pourra dissiper que des sorts lancés lors d'une Phase de Magie précédente. Les tentatives de dissipation sont faites avec des Dés de Magie (Pouvoir ou de Dissipation), utilisés comme des Dés de Dissipation, et suivent les règles normales de dissipation.} Cependant, pour dissiper avec succès, le résultat du jet de dissipation doit être \textbf{supérieur ou égal} à la plus petite valeur de lancement, non modifiée, de la version du sort qui est de type \remainsinplay{}. Ignorez les valeurs des versions améliorées.

\paragraph{Jeu de Devinettes}

Le Joueur Réactif ne sait pas si le Joueur Actif a l'intention de dissiper un sort de type \remainsinplay{} et doit prendre sa décision en premier. Il doit choisir s'il prend le risque de dépenser des Dés de Dissipation pour un sort que le Joueur Actif pourrait dissiper de lui-même, ou garder ses dés alors que le Joueur Actif pourrait décider de ne rien faire et mettre fin à la Phase de Magie, laissant en jeu le sort de type \remainsinplay{}.

\newpage
\section{Séquence de lancement d'un sort}
\label{spell_casting_sequence}

Chacun des Sorciers non en fuite du Joueur Actif, ou chaque figurine avec un \boundspell{}, peut tenter de jeter chacun de ses sorts une fois par Phase de Magie.

\paragraph{Tentative de Lancement}

\begin{tabular}{c|m{14cm}}
1 & Le Joueur Actif indique quel Sorcier tente de lancer quel sort puis le nombre de Dés de Pouvoir utilisés. Il doit préciser s'il opte pour une version améliorée du sort, ainsi que la cible du sort \newfromWHB{et de celle de l'Attribut} si nécessaire. \newfromWHB{Le joueur peut lancer entre 1 et \textbf{5} Dés de Pouvoir, dans la limite de sa réserve.} \tabularnewline
2 & Le Joueur Actif lance le nombre de Dés de Pouvoir annoncé, en les retirant de sa réserve. Additionnez les résultats des dés avec les modificateurs de lancer (voir le paragraphe \ref{magic_modifiers} ci-dessous) pour obtenir le total de lancement. \tabularnewline
3 & La tentative de lancement réussit si le total de lancement est \textbf{supérieur ou égal} à la valeur de lancement. Sinon, le lancement de sort échoue et le lanceur subit une \lostfocus{}. \tabularnewline
\end{tabular}

Si la tentative de lancement réussit, le sort n'est pas encore lancé car le Joueur Réactif peut tenter de le dissiper et l'empêcher d'être résolu.

\paragraph{Tentative de Dissipation}

\begin{tabular}{c|m{14cm}}
1 & Le Joueur Réactif indique, s'il le souhaite, un Sorcier n'étant pas en fuite pour tenter la dissipation, et annonce combien de Dés de Dissipation il va utiliser. Il doit utiliser au moins un dé, et jusqu'à la totalité de sa réserve. Il est possible de tenter une dissipation même sans avoir de Sorcier. \tabularnewline
2 & Le Joueur Réactif lance le nombre de Dés de Dissipation annoncé, en les retirant de sa réserve. Additionnez les résultats des dés avec les modificateurs de dissipation (voir le paragraphe \ref{magic_modifiers} ci-dessous), pour obtenir le total de dissipation. \tabularnewline
3 & La tentative de dissipation réussit si le total de dissipation est \textbf{supérieur ou égal} au total de lancement. Le sort est alors dissipé et le lancement échoue. Sinon, la tentative de dissipation échoue et le Sorcier à l'origine de cette tentative subit une \lostfocus{}. \tabularnewline
\end{tabular}

\paragraph{Résolution du Sort}

Si le sort n'a pas été dissipé, il est lancé avec succès. Appliquez les effets du sort, puis les effets de l'Attribut de la Voie. Si le sort a été lancé avec un \overwhelmingpower{}, résolvez les effets du \miscast{}.

\newpage
\subsection{Modificateurs magiques}
\label{magic_modifiers}

Lors d'une tentative de lancement ou de dissipation, ajoutez le bonus donné par le Niveau de Magie du Sorcier et tout autre modificateur, par exemple celui d'un \overwhelmingpower{}, pour obtenir le total de lancement ou de dissipation. \newfromWHB{Les modificateurs cumulés ne peuvent ni dépasser un total de +3 ni descendre en dessous de -3.} Il y a quelques exceptions à cette règle, notamment lors d'un \overwhelmingpower{}. Dans ce cas, ajoutez d'abord les modificateurs normaux jusqu'à un maximum de +3, puis ajoutez les modificateurs exceptionnels qui, eux, peuvent dépasser +3.

\newpage
\subsection{\lostfocus}

\newfromWHB{Un Sorcier qui subit une \lostfocus{} ne peut ajouter aucun bonus, ce qui inclut ceux liés au Niveau de Magie ou à un \overwhelmingpower{}, à des jets de lancement ou de dissipation pour le reste de la phase. Il subit quand même les effets négatifs d'un \overwhelmingpower{}, même s'il ne profite pas du bonus sur son jet.}

\subsection{Pas Assez de Puissance}

Quand vous tentez de lancer ou de dissiper un sort \newfromWHB{avec un seul Dé de Magie}, un résultat de \result{1} ou \result{2} sur le dé est toujours un échec, peu importe les modificateurs.


\subsection{\overwhelmingpower}

\newfromWHB{Quand vous lancez ou dissipez un sort, et qu'au moins deux dés donnent des \result{6} naturels, la tentative bénéficie d'un \overwhelmingpower{}. Le nombre de Dés de magie Utilisés pour la tentative est noté NDU. Ajoutez alors immédiatement 1D3 + NDU à votre total de lancer ou de dissipation. C'est une exception à la règle des Modificateurs Magiques, le cumul des modificateurs pouvant alors dépasser +3. Si un sort est lancé avec un \overwhelmingpower{} et qu'il n'est pas dissipé, le lanceur subit un \miscast{}.}

\newpage
\section{\miscast}

\newfromWHB{Lancez 2D6 et appliquez les effets décrits dans la table \ref{table/miscast}. Le nombre de Dés de Pouvoir utilisés pour lancer le sort est noté NDU. Les touches d'un \miscast{} ont une Force de NDU + 2 et suivent les règles \magicalattacks{} et \armourpiercing{1}. La figurine du Sorcier à l'origine du \miscast{} ne peut utiliser aucune sauvegarde contre ses effets.}

\newfromWHB{\textbf{Après ceci, retirez NDU Dés de Pouvoir de la réserve du propriétaire du lanceur.}}

\renewcommand{\arraystretch}{2}
\begin{table}[!htbp]
 \centering
\begin{tabular}{cp{12cm}}
\hline
\textbf{2 à 4} & \textbf{\breachintheveil}

\vspace*{5pt}
Centrez le Gabarit de \distance{5} sur le lanceur. Toute figurine touchée par le Gabarit subit une touche.

\vspace*{5pt}
\newfromWHB{Si \textbf{4} Dés de Pouvoir ont été utilisés, lancez un dé. Sur un résultat de 1 à 3, retirez le lanceur de la partie.}

\vspace*{5pt}
\newfromWHB{Si \textbf{5} Dés de Pouvoir ont été utilisés, retirez le lanceur de la partie.}\tabularnewline
\textbf{5 à 6} & \textbf{\catastrophicdetonation}

\vspace*{5pt}
Centrez le Gabarit de \distance{3} sur le lanceur. Toute figurine touchée par le Gabarit subit une touche. Le lanceur doit subir une touche.\tabularnewline
\textbf{7} & \textbf{\newfromWHB{\witchfire}}

\vspace*{5pt}
\newfromWHB{L'unité du lanceur subit NDU touches, distribuées comme des tirs. Le lanceur ne peut cependant subir qu'une seule touche au plus.}\tabularnewline
\textbf{8 à 9} & \textbf{\sorcerousbacklash}

\vspace*{5pt}
Le lanceur et tout Sorcier allié subissent une touche. \tabularnewline
\textbf{10 à 12} & \textbf{\newfromWHB{\amnesia}}

\vspace*{5pt}
\newfromWHB{Le Niveau de Magie du lanceur est diminué de NDU-2}. Il perd un sort pour chaque Niveau de Magie perdu, en commençant par le sort ayant causé le \miscast{} et en tirant les autres au hasard.\tabularnewline
\hline
\end{tabular}
\caption{Table des \miscasts{}.}
\label{table/miscast}
\end{table}
\renewcommand{\arraystretch}{1.5}

\section{Attributs de Voie}

Les Attributs de Voie sont des sorts particuliers qui ne peuvent pas être lancés indépendamment. Ils sont lancés automatiquement, du moment qu'il y a une cible possible, après qu'un sort de la même Voie a été lancé avec succès et que ses effets ont été résolus. Les Attributs ne peuvent pas être dissipés.

\newpage
\section{\boundspell{}}

Certains sorts sont liés à des \boundspells{} : ils sont contenus dans des Objets Magiques. Ces sorts peuvent être lancés par des figurines n'étant pas des Sorciers. Posséder un \boundspell{} ne fait pas de la figurine un Sorcier. Les sorts liés ne peuvent jamais être lancés en version améliorée.

\paragraph{Lancer un sort lié}

Lancer un sort lié à un \boundspell{} se fait de la même manière qu'un sort classique, sauf qu'il est impossible d'ajouter des modificateurs positifs, et que le lanceur ne subit jamais de \lostfocus{}. L'Attribut de la Voie est lancé normalement en cas de succès. Pour lancer un sort lié, un total de lancer \textbf{supérieur ou égal} au Niveau de Puissance de l'objet doit être obtenu.

\paragraph{Niveau de Puissance}

Le Niveau de Puissance correspond à la valeur de lancement du sort. Si le sort lié a aussi une valeur de lancement classique, le Niveau de Puissance la remplace.

\paragraph{\overwhelmingpower}

\newfromWHB{Si un \overwhelmingpower{} est obtenu en lançant un sort lié, la procédure habituelle des \miscasts{} est remplacée par les instructions suivantes :}
\begin{itemize}[label={-}]
\item \newfromWHB{Retirez NDU Dés de Pouvoir de la réserve du propriétaire du lanceur.}
\item \newfromWHB{Si le sort lié a été lancé avec \textbf{4} dés ou plus, l'\boundspell{} est perdu et le sort ne peut plus être lancé de la partie.}
\end{itemize}

\paragraph{Dissipation d'un sort lié}

\newfromWHB{Quand un joueur tente de dissiper un sort lié, il gagne un modificateur de +1 pour son jet de dissipation.} C'est une exception à la règle des Modificateurs Magiques.

\newpage
\section{Effets magiques particuliers}

\subsection{Mouvement Magique}

Tout mouvement réalisé pendant la Phase de Magie est un Mouvement Magique. Ce mouvement est effectué comme lors de l'étape des Autres Mouvements et suit les même restrictions. Les unités en fuite ou au Corps à Corps ne peuvent donc pas se déplacer. Toute action habituellement autorisée pendant cette étape peut être faite, comme une Roue, une Reformation, quitter ou rejoindre une unité, à l'exception d'une Marche Forcée. Un Mouvement Magique a toujours une distance limite précisée : par exemple, \og la cible peut faire un Mouvement Magique de \distance{12} \fg{}. Cette valeur est utilisée à la place de la valeur de Mouvement de la cible. Rappelons qu'il n'est pas possible de faire une Marche Forcée. Une unité ne peut faire qu'un seul Mouvement Magique par Phase de Magie.

\subsection{Récupérer et Ressusciter des PVs}

Certains sorts ou capacités permettent de récupérer des Points de Vie perdus pendant la bataille. La quantité de Points de Vie récupérés est notée dans la capacité  : \og Récupérer X PVs \fg{}. Si une unité contient plusieurs figurines, chaque figurine doit récupérer tous ses PVs perdus avant qu'une autre figurine ne puisse en récupérer à son tour. Les Personnages à l'intérieur des unités ne peuvent jamais récupérer de PVs grâce à une capacité qui fait récupérer des PVs à leur unité. Ils ne récupèrent des PVs que lorsqu'ils sont la seule cible du sort ou de la capacité. Récupérer des PVs ne peut jamais ramener à la vie des figurines retirées comme perte, ni permettre à une figurine de dépasser ses PVs initiaux. Tout PV récupéré en excès est perdu.

Ressusciter des PVs utilise les mêmes règles que Récupérer des PVs, à l'exception que cela peut ramener à la vie des figurines retirées comme perte. D'abord, les figurines de l'unité, sauf les Personnages, récupèrent tous leurs PVs perdus, puis les figurines sont ramenées à la vie dans l'ordre suivant : Champion, Porte-Étendard, Musicien et enfin les figurines ordinaires. \newfromWHB{Un Porte-Étendard capturé ne peut pas être ressuscité.} Un effet avec la règle Une seule Utilisation déjà utilisé, ou un équipement détruit ne peut être récupéré de cette manière. Chaque figurine relevée doit récupérer la totalité de ses PVs avant qu'une autre figurine ne soit relevée. Cela ne peut pas permettre à une unité de dépasser ses effectifs initiaux. Tout PV ressuscité en excès est perdu.

\subsection{Unités invoquées}

Les unités invoquées sont des unités créées pendant la partie. Toutes les figurines d'une unité nouvellement invoquée doivent être déployées à portée du sort ou de la capacité. Si l'unité est invoquée grâce à un sort de type \ground{}, au moins une des figurines doit être déployée sur le point ciblé et toutes les autres figurines doivent être déployées à portée du sort. De plus, les figurines invoquées doivent être placées au moins à \distance{1} des autres unités et des Terrains Infranchissables. Si l'unité ne peut pas être déployée en entier, alors aucune figurine ne doit être déployée. Une fois invoquées, ces unités comptent comme des unités ordinaires, dans le camp du lanceur du sort ou de la capacité. Elles ne donnent aucun Point de Victoire à l'adversaire quand elles sont détruites.
