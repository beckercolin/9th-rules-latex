% Base sur la VO 0.11.9
% Relecture technique: 
% Relecture syntaxique: 

\part{Conditions de victoire et points}
\label{condition_victoire}

\section{Gagner des points de victoire}

À la fin de la partie, faites le total de vos points de victoire en suivant les règles de la table \ref{table/pointsdevictoire}.

\begin{table}[!htbp]
\centering
\begin{tabular}{M{2cm}|m{11cm}}
\textbf{Morts ou Déserteurs} & Pour chaque unité ennemie qui a été détruite ou qui a fui en dehors du champ de bataille, vous gagnez autant de \textbf{points de victoire que sa valeur en points}. \tabularnewline
\nouveau{\textbf{Effrayés}} & \nouveau{Pour chaque unité ennemie en fuite sur le champ de bataille, vous gagnez autant de \textbf{points de victoire que la moitié de sa valeur en points (arrondis au supérieur)}}. \tabularnewline
\nouveau{\textbf{Décimés}} & \nouveau{Pour chaque unité réduite à 25 \% ou moins de son effectif de PVs de départ, vous gagnez autant de \textbf{points de victoire que la moitié de sa valeur en points (arrondis au supérieur)}. Les \emph{Personnages} sont comptés séparément des unités qu'ils ont rejointes}. \newrule{Notez que si une unité ennemie est à la fois Effrayée et Décimée, vous gagnez autant de points de victoire que sa valeur en points.} \tabularnewline
\textbf{Leur Roi est mort} & Si le \emph{Général} ennemi est mort ou a fui le champ de bataille, vous gagnez \textbf{100 points de victoire} additionnels. \tabularnewline
\textbf{Leur bannière est à terre} & Si le \emph{Porteur de la Grande Bannière} ennemi est mort ou a perdu un combat et a raté son test de \emph{Moral}, vous gagnez \textbf{100 points de victoire} additionnels. \tabularnewline
\textbf{Étendards capturés} & Vous gagnez \textbf{\nouveau{50} points de victoire} additionnels pour chaque \emph{Porte-Étendard} ennemi tué au corps à corps, ou qui a perdu un combat et a raté un test de \emph{Moral}. \tabularnewline
\end{tabular}
\caption{\label{table/pointsdevictoire}Comment gagner des points de victoire ?}
\end{table}

\subsection{Objectifs secondaires}

\nouveau{Chaque objectif secondaire choisi en début de partie peut faire gagner des Points de Bataille supplémentaires}. Avant de sélectionner les zones de déploiement, vous pouvez choisir, en accord avec votre adversaire, si vous voulez jouer un objectif secondaire, et lequel. Vous pouvez aussi le déterminer aléatoirement en lançant un D6 et en suivant la correspondance suivante :
\begin{itemize}
\item \emph{1 ou 2}. \textbf{Tenez la ligne}. Le joueur avec le plus d'\emph{Unités de Capture} à moins de 6{\pouce} du centre du champ de bataille à la fin de la partie gagne cet objectif secondaire.
\item \emph{3 ou 4}. \textbf{Percée}. Le joueur qui a le plus d'\emph{Unités de capture} dans la zone de déploiement de son adversaire à la fin de la partie remporte cet objectif secondaire.
\item \emph{5}. \newrule{\textbf{Capturez les étendards}. Le joueur qui possède le plus grand nombre de porte-étendards en vie, parmi ceux qui avaient été désignés au début de la partie, remporte cet objectif secondaire.}
\item \emph{6}. \textbf{Sécurisez cette zone}. À la fin de la partie, le joueur qui contrôle le plus de marqueurs remporte cet objectif secondaire. Pour contrôler un marqueur, il faut avoir plus d'\emph{Unités de Capture} que son adversaire à moins de 6{\pouce} du marqueur. \newrule{Si une même unité est à moins de 6{\pouce} des deux marqueurs, elle ne compte que dans les 6{\pouce} du marqueur le plus proche de son centre (déterminez-le aléatoirement si les deux marqueurs sont à égale distance du centre de l'unité).}
\end{itemize}

\subsection{\nouveau{Unités de capture}}

Les \emph{Unités de Capture} servent à obtenir les objectifs secondaires. Toute unité avec un \emph{Porte-Étendard}, à l'exception du \emph{Porteur de la Grande Bannière}, compte comme une \emph{Unité de Capture}, sauf si:
\begin{itemize}
\item au moins une figurine dans l'unité possède la règle spéciale \emph{Troupes Légères}.
\item l'unité est en fuite.
\item l'unité est entrée en \emph{Embuscade} au Tour 4 ou plus tard.
\item l'unité a fait une \emph{Reformation Post-Combat} pendant ce tour de joueur.
\end{itemize}

\section{Qui a gagné ?}

\nouveau{Une fois que tous les points de victoire ont été comptabilisés, un total de 20 Points de Bataille est attribué aux joueurs, en fonction de l'écart entre leurs points de victoire. Calculez cette différence et référez-vous à la table ci-dessous pour la convertir en Points de Bataille.}

\nouveau{Si un objectif secondaire a été utilisé et qu'un joueur l'a remporté, il gagne 3 Points de Bataille supplémentaires, tandis que son adversaire en perd 3.} Si plus d'un objectif secondaire a été utilisé, le joueur qui en a remporté le plus gagne ces 3 Points de Bataille, tandis que son adversaire les perd.

\newpage
\subsection{Table des Points de Bataille}

\begin{table}[!htbp]
\centering
\begin{tabular}{|M{3,8cm}|M{3cm}|M{3cm}|M{3cm}|}
\hline
\textbf{Différence de points de victoire (en pourcentage)} & \textbf{Exemple pour une partie à 2500 points} & \textbf{Points de Bataille du vainqueur} & \textbf{Points de Bataille du perdant} \\
\hline
0 - 5\% & 0 - 125 & 10 & 10 \\
\hline
>5 - 10\% & 126 - 250 & 11 & 9 \\
\hline
>10 - 20\% & 251 - 500 & 12 & 8 \\
\hline
>20 - 30\% & 501 - 750 & 13 & 7 \\
\hline
>30 - 40\% & 751 - 1000 & 14 & 6 \\
\hline
>40 - 50\% & 1001 - 1250 & 15 & 5 \\
\hline
>50 - 70\% & 1251 - 1750 & 16 & 4 \\
\hline
>70\% & >1750 & 17 & 3 \\
\hline
Remporter un Objectif Secondaire & - & +3 & -3 \\
\hline
\end{tabular}
\caption{\label{table/points_bataille}Table des Points de Bataille.}
\end{table}

\subsection{Règles optionnelles simplifiées pour déterminer le vainqueur}

Remporter un objectif secondaire apporte un nombre de points de victoire égal à 20\% de la taille de la partie. Une fois que tous les points de victoire ont été comptabilisés, comparez les résultats.
\begin{itemize}
\item Si la différence entre les totaux est de moins de 10\% de la taille de la partie, il s'agit d'une \textbf{égalité}.
\item Si la différence entre les totaux est entre 10 et 50\%, il s'agit d'une \textbf{victoire} pour le joueur qui a le plus de points de victoire.
\item Si la différence entre les totaux est de plus de 50\%, il s'agit d'un \textbf{massacre}.
\end{itemize}
