% Base sur la VO 0.99.0
% Relecture technique: 
% Relecture syntaxique: 

\part{État-majors}
\label{etat_majors}

Certaines unités proposent une option pour promouvoir des figurines d'\emph{État-Major}, qui peuvent être un \emph{Champion}, un \emph{Musicien}, ou encore un \emph{Porte-Étendard}. Chacune de ces figurines est une amélioration d'une figurine ordinaire, qui devient alors une figurine d'\emph{État-Major}. Elles ont la règle \emph{Au premier rang}, décrite au paragraphe \ref{au_premier_rang} à la page \pageref{au_premier_rang}.

\section{État-major et pertes}

Les \emph{Musiciens} et \emph{Porte-Étendards} sont considérés comme des figurines ordinaires, donc les blessures qui leur sont infligées occasionnent des pertes retirées à l'arrière de l'unité, comme pour toute figurine ordinaire. Si un \emph{Musicien} ou un \emph{Porte-Étendard} devait être retiré comme perte du fait de son positionnement dans l'unité, comme par exemple s'il est au bout d'une unité formée d'un seul rang, retirez-le, puis remettez-le à la place d'une figurine ordinaire restante. Les \emph{Champions} peuvent être ciblés de la même manière que des \emph{Personnages}. Cependant, si assez de blessures sont infligées à une unité pour tuer toutes les figurines ordinaires, les éventuelles blessures excédentaires sont reportées sur le \emph{Champion}, même s'il combat dans un \emph{Défi}.

\section{Musiciens}

\subsubsection*{Reformation Rapide}

Une unité contenant un \emph{Musicien} peut faire une \emph{Reformation Rapide} en faisant un test de Commandement. Si le test est raté, l'unité doit faire une \emph{Reformation} normale. S'il est réussi, l'unité fait une \emph{Reformation} avec les avantages suivants :
\begin{itemize}[label={-}]
\item L'unité peut tirer dans la \emph{Phase de Tir} à venir.
\item L'unité peut faire un \emph{Mouvement Simple} après la \emph{Reformation}.
\end{itemize}

\subsubsection*{Du nerf !}

Si un résultat de combat est nul, le camp qui possède un \emph{Musicien} gagne le combat. Si les deux camps possèdent un \emph{Musicien}, le combat est une égalité. Si un test de \emph{Moral} est provoqué par une victoire procurée par un \emph{Musicien}, \nouveau{aucun modificateur du Commandement pour avoir perdu le combat n'est appliqué à l'unité qui perd}, ni pour le test de \emph{Moral}, ni pour le test de \emph{Reformation de Combat}. Les unités \emph{Instables} ne sont pas affectées.

\subsubsection*{Rappel à l'ordre}

Une unité en fuite qui possède un \emph{Musicien} a un bonus de +1 à son Commandement pour faire son test de \emph{Ralliement}.

\section{Porte-étendard}

\subsubsection*{Bonus de combat}

Un camp qui possède au moins un \emph{Porte-Étendard} a un bonus de +1 à son résultat de combat.

\subsubsection*{Étendard capturé}

Quand un \emph{Porte-Étendard} est retiré comme perte alors qu'il était engagé dans un corps à corps, on considère que l'étendard est capturé par l'adversaire. Si une unité contenant un \emph{Porte-Étendard} perd un combat et rate son test de \emph{Moral}, \nouveau{remplacez sa figurine par une figurine ordinaire}, et l'étendard est capturé. Il est considéré comme perdu pour tout aspect du jeu, comme le bonus au \emph{Résultat de Combat} ou toute bannière magique.

\subsection{\nouveau{Porte-étendard Vétéran}}

\nouveau{Certaines unités ont la possibilité de sélectionner un \emph{Porte-étendard Vétéran}. Un \emph{Porte-étendard Vétéran} peut sélectionner une bannière magique coûtant 25 points maximum. Un \emph{Porte-étendard Vétéran} suit la règle \emph{Unique}. Il ne peut donc être représenté qu'une seule fois par armée (ou deux fois en cas de Grande Armée).}

\section{Champions}

\subsubsection*{Premier entre ses pairs}

Tous les champions ont un bonus de \nouveau{+1 Attaque, +1 en Capacité de Combat et +1 en Capacité de Tir}. Si la figurine est composée de plusieurs éléments, seules les caractéristiques du cavalier, ou d'un seul membre d'équipage, sont augmentées.

\subsubsection*{\nouveau{Meneur de charge}}

Une unité contenant un \emph{Champion} réussit automatiquement les charges nécessitant un résultat du jet de distance de charge de \result{4} ou moins. Sinon, elle lance les dés normalement.
