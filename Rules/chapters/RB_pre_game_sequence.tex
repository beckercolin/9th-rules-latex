
\part{Séquence Pré-Partie}

Il y a plusieurs étapes à suivre dans un certain ordre afin de mettre en place une partie de Batailles Fantastiques : Le 9\ieme{} Âge. L'ensemble de ces étapes est appelé Séquence Pré-Partie. La première étape, et la plus importante, est de trouver un adversaire motivé et de vous mettre d'accord sur la taille de la partie. Les joueurs peuvent ensuite se dévoiler leurs Listes d'Armée et commencer à installer le champ de bataille. Puis ils déterminent le type de déploiement, les Objectifs Secondaires, les zones de déploiement et génèrent les sorts des \wizards{}. La dernière étape consiste à passer à ce qu'on appelle le Déploiement.

Voilà un résumé des étapes de la séquence pré-partie :

\hspace*{0.3cm}
\begin{tabular}{c|l}
1 & Décidez de la taille de la partie. \tabularnewline
2 & Montrez-vous vos Listes d'Armée. \tabularnewline
3 & Installez le champ de bataille. \tabularnewline
4 & Tirez au sort le type de déploiement. \tabularnewline
5 & Tirez au sort les Objectifs Secondaires. \tabularnewline
6 & Déterminez les zones de déploiement. \tabularnewline
7 & Générez les sorts. \tabularnewline
8 & Phase de Déploiement. \tabularnewline
\end{tabular}

\hypertarget{gamesize}{\subsection{Taille de la partie}}

Les deux armées qui se font face doivent avoir à peu près le même coût en points, afin que l'issue de la bataille dépende des stratégies et tactiques rusées des joueurs plutôt que d'une asymétrie dans la puissance des armées. Les deux joueurs doivent donc se mettre d'accord sur le coût en points des armées que chacun commandera, déterminant ainsi la taille de la partie. Des armées de 1500 à 3000 points correspondent à des petites escarmouches. De 3000 à 6000 points, la bataille commence à être plus importante. Au delà de 6000 points, le jeu met en scène un conflit massif entre des armées épiques. Nous recommandons une taille de partie autour de 4500 points pour profiter du meilleur équilibre du jeu.

\hypertarget{sharearmylist}{\subsection{Dévoiler les listes}}

Après avoir décidé de la taille de la partie, l'étape suivante est d'échanger les Listes d'Armée ainsi que toute information pertinente à propos de la partie à venir. Les joueurs peuvent aussi choisir de garder quelques aspects de leur armée secrets et de ne les révéler qu'au fur et à mesure de la partie, comme expliqué dans le paragraphe \ref{hidden_lists}.

\newpage
\hypertarget{buildbattlefield}{\subsection{Installer le champ de bataille}}

La partie se joue habituellement sur un champ de bataille de \distance{48} par \distance{72} (environ 1,20 {\meter} par 1,80 {\meter}). Pour de plus petites batailles, au format Patrouille, nous recommandons une table de \distance{36} par \distance{48} (environ 90 {\centi\meter} par 1,20 {\meter}) et pour les plus grandes parties, au format Grande Armée, ajustez le champ de bataille à la taille des armées. Bien qu'une partie puisse être jouée sur une table vierge, on y place habituellement quelques Décors. Les joueurs peuvent tout à fait se mettre d'accord sur le nombre, le type, la taille et la position des Décors. Sinon, voilà les règles standard pour générer un champ de bataille aléatoire :

\begin{itemize}[label={\textbullet}]
\item Divisez la table de jeu en sections de \distance{24} par \distance{24}, soit environ 60 {\centi\meter} par 60 {\centi\meter}. Si la table fait \distance{36} par \distance{48}, prenez plutôt des sections de \distance{18} par \distance{24}, soit environ 45 {\centi\meter} par 60 {\centi\meter}.

\item Placez chacun des trois Décors suivants au centre de sections choisies aléatoirement, avec un Décor au plus dans chaque section : un Bâtiment ou un Terrain Infranchissable (tirez au hasard), une Colline et une Forêt. Déplacez ensuite chaque Décor de \distance{2D6} dans une direction aléatoire.

\item Ajoutez 2D3 Décors (1D3 si la table fait \distance{36} par \distance{48}), en suivant les règles ci-dessus pour leur positionnement. Lancez 1D6 pour déterminer le type de chaque Décor additionnel :
\begin{enumerate}
\item Colline
\item Forêt
\item Champ
\item Eau peu profonde
\item Mur
\item Ruines
\end{enumerate}

\item Tous les Décors doivent être placés au moins à \distance{6} les uns des autres. Bougez-les aussi peu que possible de leur position initiale pour satisfaire à cette condition. Si cela n'est pas possible, retirez le Décor qui pose problème.

\item Nous vous recommandons des tailles de Décor comprises entre \distance{6} par \distance{8} (soit 15x20 {\centi\meter}) et \distance{6} par \distance{10} (15x25 {\centi\meter}), excepté pour les Murs pour lesquels nous conseillons \distance{1} par \distance{10} (2,5x25 {\centi\meter}).
\end{itemize}


\newpage
\hypertarget{deploymenttype}{\subsection{Type de déploiement}}

\hypertarget{deploymenttypefigure}{%
Tirez au hasard le type de déploiement en lançant 1D6 :
}

\hypertarget{frontlineclash}{\paragraph{1-2 : \frontlineclash{}}}
\label{figure/deployment}
\begin{minipage}[c]{0.35\textwidth}
\def\svgwidth{\textwidth}
\input{pics/deployment_one.pdf_tex}
\end{minipage}\hfill
\begin{minipage}[c]{0.62\textwidth}
La table est divisée en deux moitiés égales par une ligne centrale, dans le sens de la longueur. Les zones de déploiement commencent à \distance{12} de part et d'autre de cette ligne et s'étendent au delà. 
\end{minipage}

\hypertarget{refusedflank}{\paragraph{3-4 : \refusedflank{}}}

\begin{minipage}[c]{0.35\textwidth}
\def\svgwidth{\textwidth}
\input{pics/deployment_two.pdf_tex}
\end{minipage}\hfill
\begin{minipage}[c]{0.62\textwidth}
La table est divisée en deux moitiés par une diagonale de la table. Le joueur choisissant sa zone de déploiement décide de la diagonale à utiliser. Les zones de déploiement commencent à \distance{9} de part et d'autre de cette ligne et s'étendent au delà.
\end{minipage}

\hypertarget{encircle}{\paragraph{5 : \encircle{}}}

\newcommand{\deploymentfigAttacker}{\flufffont{Attaquant}}
\newcommand{\deploymentfigDefender}{\flufffont{Défenseur}}
\begin{minipage}[c]{0.35\textwidth}
\def\svgwidth{\textwidth}
\input{pics/deployment_three.pdf_tex}
\end{minipage}\hfill
\begin{minipage}[c]{0.62\textwidth}
La table est divisée en deux moitiés égales par une ligne centrale, dans le sens de la longueur. Le joueur choisissant sa zone de déploiement décide s'il veut être l'attaquant ou le défenseur. L'attaquant peut se déployer à plus de \distance{9} de la ligne centrale à plus du quart des bords courts de la table (\distance{18} sur une table de \distance{72} de long) et à plus de \distance{15} de la ligne médiane aux deux autres endroits, près des bords courts. Le défenseur fait le contraire : les limites de \distance{9} et \distance{15} sont inversées.
\end{minipage}

\hypertarget{counterthrust}{\paragraph{6 : \counterthrust{}}}

\begin{minipage}[c]{0.35\textwidth}
\def\svgwidth{\textwidth}
\input{pics/deployment_four.pdf_tex}
\end{minipage}\hfill
\begin{minipage}[c]{0.62\textwidth}
La table est divisée en deux moitiés égales par une ligne centrale, dans le sens de la longueur. Les zones de déploiement commencent à \distance{6} de part et d'autre de cette ligne et s'étendent au delà. Aucune unité ne peut être déployée à moins de \distance{18} d'une unité ennemie. Cela ne concerne pas les unités au déploiement spécial comme celles avec la règle \scout{}.

Les joueurs ne peuvent déployer qu'une seule unité par tour de déploiement, et posent donc une unité tour à tour. L'ensemble des Personnages et l'ensemble des Machines de Guerre comptent chacun comme une seule unité en terme de déploiement.
\end{minipage}


\newpage
\hypertarget{secondaryobjectives}{\subsection{Objectifs Secondaires}}

Avant de choisir les zones de déploiement, tirez au hasard un Objectif Secondaire en lançant 1D6 :

\paragraph{1-2 : Tenez la Ligne}

\flufffont{Tenez et défendez le centre du champ de bataille.}\newline
Placez un marqueur sur le centre du champ de bataille au besoin.

\paragraph{3-4 : Percée}

\flufffont{Envahissez le territoire ennemi.}\newline
Marquez la position des zones de déploiement.

\paragraph{5 : Capturez les Étendards}

\flufffont{Les cibles de valeur doivent être annihilées.}\newline
Après avoir déplacé les unités avec la règle \vanguard{} et avant de déterminer qui aura le premier Tour de Joueur, les deux joueurs doivent désigner chacun leur tour et ouvertement une unité ennemie avec la règle \scoring{}, jusqu'à en avoir choisi trois ou toutes les unités éligibles s'il y en a moins de trois. Les unités qui n'ont pas encore été déployées, comme les unités avec la règle \ambush{}, peuvent être désignées. Le joueur qui a fini de se déployer en premier commence à choisir.

\paragraph{6 : Sécurisez la Cible}

\flufffont{Des ressources cruciales ne doivent pas tomber entre les mains de l'ennemi.}\newline
Après avoir choisi les zones de déploiement, chaque joueur place un marqueur sur le champ de bataille, en commençant par celui qui a choisi sa zone de déploiement. Ces marqueurs doivent être positionnés à plus de \distance{12} de la zone de déploiement du joueur qui le place, et à plus de \distance{24} l'un de l'autre.

Référez-vous au chapitre \ref{scoring_and_victory_conditions} (page \pageref{scoring_and_victory_conditions}) pour plus de détails sur la capture d'un objectif et l'impact sur les Points de Victoire.

\hypertarget{deploymentzones}{\subsection{Zones de déploiement}}

Tirez au hasard afin de déterminer qui choisit les zones de déploiement. Par exemple, un joueur peut lancer 1D6 : sur 4+, c'est lui qui choisit. En cas de \frontlineclash{} ou de \counterthrust{}, il s'approprie un des bords longs du champ de bataille Pour un \refusedflank{}, il peut prendre un des quatre coins. Enfin, pour un \encircle{}, il choisit le côté et s'il est l'attaquant ou le défenseur.

\newpage
\hypertarget{generatespells}{\subsection{Générer les sorts}}
\label{generating_spells}

Chaque joueur génère les sorts pour tous ses \wizards{}, en commençant par le joueur qui a choisi sa zone de déploiement. Pour cela, sélectionnez un \wizard{} et consultez sa Voie de Magie, qui doit être inscrite sur la Liste d'Armée. Toutes les Voies de Magie peuvent être trouvées dans le Livre de Magie.

\paragraph{Sorts numérotés de 1 à 6}

Toutes les Voies de Magie comprennent des \og \learnedspells{} \fg{}, numérotés de 1 à 6. Lancez autant de D6 que de sorts que le \wizard{} connait pour voir quels sorts il pourra utiliser lors de cette bataille. Si un \result{1} est obtenu, le \wizard{} connaît le sort numéro 1, et ainsi de suite. Si un \learnedspell{} est obtenu deux fois, parce que les dés donnent un double ou parce qu'un autre \wizard{} de la même armée a déjà choisi ce sort, le sort en double doit être remplacé par un autre sort de son choix de la même Voie de Magie qui n'a pas encore été choisi.

\paragraph{Sorts numérotés 0}

Certaines Voies de Magie comprennent un sort numéroté 0. C'est un type spécial de \learnedspell{}. Tout \wizard{} peut échanger un de ses \learnedspells{} de la même Voie contre ce sort. Notez qu'il ne peut pas être dupliqué au sein d'une armée.

Deux \wizards{} de la même armée ne peuvent pas connaître le même \learnedspell{} et aucun \wizard{} ne peut disposer d'un \learnedspell{} en plusieurs exemplaires. S'il est impossible de remplacer un \learnedspell{} en double par un autre, le sort est perdu. Les sorts qui ne sont pas générés en suivant ces règles, comme les sorts prédéterminés ou les \boundspells{}, sont ignorés pour la duplication des sorts. De tels sorts peuvent ainsi être présents plus d'une fois dans une même armée.

\paragraph{Sorts numérotés \og \attributespellnumber{} \fg{} ou \og \traitspellnumber{} \fg{}}

Certaines Voies de Magie comprennent des sorts numérotés \og \attributespellnumber{} \fg{} ou \og \traitspellnumber{} \fg{}. Ce sont des sorts spéciaux appelés respectivement \attributespells{} et \traitspells{}. Tout \wizard{} connaissant au moins un \learnedspell{} d'une Voie connait aussi l'\attributespell{} et/ou le \traitspell{} de la même Voie en plus de ses autres sorts.
