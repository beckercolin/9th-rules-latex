
\part{Séquence pré-partie}

Il y a plusieurs étapes à suivre dans un certain ordre afin de mettre en place une partie de Batailles Fantastiques : Le 9\ieme Âge. L'ensemble de ces étapes est appelé séquence pré-partie. La première étape, et la plus importante, est de trouver un adversaire motivé et de vous mettre d'accord sur la taille de la partie. Les joueurs peuvent ensuite se dévoiler leurs Listes d'Armée et commencer à installer le champ de bataille. Puis ils déterminent le type de déploiement, les Objectifs Secondaires, les zones de déploiement et génèrent les sorts des Sorciers. La dernière étape consiste à passer à ce qu'on appelle le Déploiement.

Voilà un résumé des étapes de la séquence pré-partie :

\hspace*{0.3cm}
\begin{tabular}{c|l}
1 & Décidez de la taille de la partie. \tabularnewline
2 & Montrez-vous vos Listes d'Armée. \tabularnewline
3 & Installez le champ de bataille. \tabularnewline
4 & Déterminez le type de déploiement. \tabularnewline
5 & Choisissez les Objectifs Secondaires. \tabularnewline
6 & Déterminez les zones de déploiement. \tabularnewline
7 & Générez les sorts. \tabularnewline
8 & Phase de Déploiement. \tabularnewline
\end{tabular}

\subsection{Taille de la partie}

Les deux armées qui se font face doivent avoir à peu près le même coût en points, afin que l'issue de la bataille dépende des stratégies et tactiques rusées des joueurs plutôt que d'une asymétrie dans la puissance des armées. Les deux joueurs doivent donc se mettre d'accord sur le coût en points des armées que chacun commandera, déterminant ainsi la taille de la partie. Des armées de 500 à 1000 points correspondent à des petites escarmouches. De 1000 à 2000 points, la bataille commence à être plus importante. Au delà de 2000 points, le jeu met en scène un conflit massif entre des armées épiques.

\subsection{Dévoiler les listes}

Après avoir décidé de la taille de la partie, l'étape suivante est d'échanger les Listes d'Armée ainsi que toute information pertinente à propos de la partie à venir. Les joueurs peuvent aussi choisir de garder quelques aspects de leur armée secrets, et de ne les révéler qu'au fur et à mesure de la partie, comme expliqué dans le paragraphe \ref{hidden_lists}.

\newpage
\subsection[Installer le champ de bataille]{\newfromWHB{Installer le champ de bataille}}

La partie se joue habituellement sur un champ de bataille de \distance{48} par \distance{72} (environ 1,20 {\meter} par 1,80 {\meter}). Pour de plus petites batailles, au format Patrouille, nous recommandons une table de \distance{36} par \distance{48} (environ 90 {\centi\meter} par 1,20 {\meter}), et pour les plus grandes parties, au format Grande Armée, ajustez le champ de bataille à la taille des armées. Bien qu'une partie puisse être jouée sur une table vierge, on y place habituellement quelques Décors. Les joueurs peuvent tout à fait se mettre d'accord sur le nombre, le type la taille et la position des Décors. Sinon, voilà les règles standard pour générer un champ de bataille aléatoire :

\begin{itemize}[label={\textbullet}]
\item Divisez la table de jeu en sections de \distance{24} par \distance{24}, soit environ 60 {\centi\meter} par 60 {\centi\meter}. Si la table fait \distance{36} par \distance{48}, prenez plutôt des sections de \distance{18} par \distance{24}, soit environ 45 {\centi\meter} par 60 {\centi\meter}.

\item Placez chacun des trois Décors suivants au centre de sections choisies aléatoirement, avec un Décor au plus dans chaque section : un Bâtiment ou un Terrain Infranchissable (tirez au hasard), une Colline et une Forêt. Déplacez ensuite chaque Décor de \distance{2D6} dans une direction aléatoire.

\item Ajoutez 2D3 Décors (1D3 si la table fait \distance{36} par \distance{48}), en suivant les règles ci-dessus pour leur positionnement. Lancez 1D6 pour déterminer le type de chaque Décor additionnel :
\begin{enumerate}
\item Colline
\item Forêt
\item Champ
\item Eau peu profonde
\item Mur
\item Ruines
\end{enumerate}

\item Tous les Décors doivent être placés au moins à \distance{6} les uns des autres. Bougez-les aussi peu que possible de leur position initiale pour satisfaire à cette condition. Si cela n'est pas possible, retirez le Décor qui pose problème.

\item Nous vous recommandons des tailles de Décor comprises entre \distance{6} par \distance{8} (soit 15x20 {\centi\meter}) et \distance{6} par \distance{10} (15x25 {\centi\meter}), excepté pour les Murs, pour lesquels nous conseillons \distance{1} par \distance{10} (2,5x25 {\centi\meter}).
\end{itemize}

\newpage
\subsection{Type de déploiement}

Les deux joueurs peuvent se mettre d'accord sur le type de déploiement, ou le tirer au hasard en lançant 1D6 :
\begin{itemize}[label={-}]
\item \textbf{1 à 3 : Classique}\newline
La table est divisée en deux moitiés égales par une ligne centrale, dans le sens de la longueur. Les zones de déploiement commencent à \distance{12} de part et d'autre de cette ligne et s'étendent au delà. 
\item \textbf{4 à 5 : Diagonal}\newline
La table est divisée en deux moitiés par une diagonale de la table. Le joueur choisissant sa zone de déploiement décide de la diagonale à utiliser. Les zones de déploiement commencent à \newfromWHB{\distance{9}} de part et d'autre de cette ligne et s'étendent au delà.
\item \textbf{\newfromWHB{6 : Contournement}}\newline
La table est divisée en deux moitiés égales par une ligne centrale, dans le sens de la longueur. Le joueur choisissant sa zone de déploiement décide s'il veut être l'attaquant ou le défenseur. L'attaquant peut se déployer à plus de \distance{9} de la ligne centrale à plus du quart des bords courts de la table (\distance{18} sur une table de \distance{72} de long), et à plus de \distance{15} de la ligne médiane aux deux autres endroits, près des bords courts. Le défenseur fait le contraire : les limites de \distance{9} et \distance{15} sont inversées.
\end{itemize}

La figure \ref{figure/deployment} illustre ces trois types de déploiement.

\newcommand{\deploymentfigClassic}{CLASSIQUE}
\newcommand{\deploymentfigDiagonal}{DIAGONAL}
\newcommand{\deploymentfigFlankattack}{CONTOURNEMENT}
\newcommand{\deploymentfigAttacker}{\textit{Attaquant}}
\newcommand{\deploymentfigDefender}{\textit{Défenseur}}
\begin{figure}[!htbp]
\centering
\def\svgwidth{\textwidth}
\input{pics/deployment.pdf_tex}
\caption{Illustration des trois types de déploiement.}
\label{figure/deployment}
\end{figure}

\subsection[Objectifs secondaires]{\newfromWHB{Objectifs secondaires}}

%Les deux joueurs peuvent se mettre d'accord sur l'objectif secondaire, ou vous pouvez le déterminer aléatoirement en lançant 1D6 : 
%\begin{itemize}[label={-}]
%\item \emph{\result{1} ou \result{2}}. Tenez la ligne. \emph{Tenez et défendez le centre du champ de bataille.} Placez un marqueur sur le centre du champ de bataille si nécessaire.
%\item \emph{\result{3} ou \result{4}}. Percée. \emph{Envahissez le territoire ennemi.}
%\item \emph{\result{5}}. \newrule{Capturez les étendards. \emph{Des cibles de valeur doivent être annihilées.} Après les mouvements d'avant-garde, avant de déterminer qui aura le premier tour de jeu, les deux joueurs doivent désigner ouvertement 3 porte-étendards ennemis (à l'exception du porteur de la grande bannière), en commençant par le joueur qui a terminé son déploiement en premier. Si un joueur possède moins de 3 porte-étendards dans son armée, son adversaire désignera tous ses porteurs d'étendards. Les porte-étendards des \emph{Troupes Légères} ne peuvent pas être désignés, mais ceux qui n'ont pas été encore déployés (comme pour les unités en \emph{Embuscade}) peuvent l'être.}
%\item \emph{\result{6}}. Sécurisez cette zone. \emph{Des ressources cruciales ne doivent pas tomber entre les mains de l'ennemi.} Après avoir choisi les zones de déploiement, chaque joueur place un marqueur sur le champ de bataille, en commençant par celui qui a choisi sa zone de déploiement. Ces marqueurs doivent être positionnés à plus de 12{\pouce} de la zone de déploiement du joueur qui le place, et à plus de 24{\pouce} l’un de l’autre.
%\end{itemize}
%
%Voir la section \ref{condition_victoire} (page \pageref{condition_victoire}) sur les conditions de victoire pour les règles concernant les objectifs secondaires.

\subsection{Zones de déploiement}

%Tirez au hasard afin de déterminer les zones de déploiement. Par exemple, un joueur peut lancer 1D6. Sur 4+, celui qui a lancé le dé choisit son côté de déploiement. Il choisit aussi la diagonale à utiliser pour le déploiement diagonal et qui joue le rôle de l'attaquant et du défenseur dans le cas de l'attaque de flanc.

\subsection{Générer les sorts}
\label{generating_spells}

%Chaque joueur génère les sorts pour tous ses \emph{Sorciers}, \nouveau{en commençant par le joueur qui a choisi sa zone de déploiement.} Pour cela, sélectionnez un \emph{Sorcier} et consultez la \emph{Discipline Magique} choisie (qui doit être inscrite sur la liste d'armée). Toutes les \emph{Disciplines Magiques} peuvent être trouvées dans \emph{Batailles Fantastiques : Le 9\ieme Âge, Disciplines Magiques}. Les sorts sont numérotés de 0 à 6. Lancez autant de D6 que de sorts que le \emph{Sorcier} possède (normalement en nombre égal à son niveau de magie) pour voir à quels sorts le \emph{Sorcier} a accès pour cette bataille. Si un '1' est obtenu, le \emph{Sorcier} connaît le sort numéro 1, et ainsi de suite. Si un sort est obtenu deux fois (les dés donnent un double, ou bien un autre \emph{Sorcier} dans la même armée a déjà choisi le sort), le \emph{Sorcier} doit remplacer le sort en double par un autre sort de la même \emph{Discipline Magique} de son choix qui n'a pas encore été choisi. Deux \emph{Sorciers} de la même armée ne peuvent pas choisir le même sort, et aucun \emph{Sorcier} ne peut connaître un sort plus d'une fois. S'il est impossible de remplacer un sort en double par un sort disponible, le sort est perdu. De plus, le \emph{Sorcier} peut échanger un de ses sorts pour le sort primaire de sa \emph{Discipline Magique} (numéroté 0). Le sort primaire peut être choisi même si d'autres \emph{Sorciers} l'ont déjà sélectionné.
%
%Les sorts qui ne sont pas générés en suivant ces règles sont ignorés pour la duplication des sorts, comme les \emph{Sorciers} qui ont des sorts prédéterminés ou tous les objets de sorts. De tels sorts peuvent ainsi être présents plus d'une fois dans la même armée.
