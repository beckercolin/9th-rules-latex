
\hypertarget{movementphase}{\part{Phase de Mouvement}}

Durant la Phase de Mouvement, vous avez la possibilité de déplacer vos figurines sur le champ de bataille.

\section{Séquence de la Phase de Mouvement}

La Phase de Mouvement est divisée en six étapes :

\hspace*{0.3cm}
\begin{tabular}{c|l}
1 & Début de la Phase de Mouvement et du Tour de Joueur. \tabularnewline
2 & Déclaration des Charges. \tabularnewline
3 & Déplacement des unités en Charge. \tabularnewline
4 & Mouvements Obligatoires. \tabularnewline
5 & Autres Mouvements. \tabularnewline
6 & Fin de la Phase de Mouvement. \tabularnewline
\end{tabular}


\hypertarget{declarecharges}{\section{Déclaration des charges}}

Si vous voulez que l'une de vos unités engage une unité ennemie au Corps à Corps, vous devez d'abord lui faire déclarer une charge. Déclarez lesquelles de vos unités vont charger ce tour-ci, et leur cible, une par une. À chaque fois que le Joueur Actif déclare une charge, le Joueur Réactif doit déclarer une Réaction pour l'unité chargée.

Une charge ne peut être déclarée que si :
\begin{itemize}[label={-}]
\item la cible est dans le champ de vision de l'unité qui charge,
\item la charge est effectivement possible, c'est-à-dire que la cible est à une distance inférieure à la distance maximale de charge,
\item il y a assez de place pour amener l'unité chargeant au contact de l'unité chargée.
\end{itemize}
Ne prenez pas en compte une éventuelle fuite en réaction à la charge, même si elle est obligatoire, pour vérifier si une charge est possible ou non. Prenez par contre en compte les charges déjà déclarées. Une unité ayant déclaré une charge va potentiellement bouger et ne plus gêner une autre charge.

\newpage
\hypertarget{chargereaction}{\subsection{Réactions aux charges}}
\label{chargereaction}

Quand une charge vient d'être déclarée contre une unité, cette dernière doit immédiatement déclarer une réaction à la charge parmi les trois possibles : \og Tenir la Position \fg{}, \og Tenir la Position et Tirer \fg{}, et \og Fuir \fg{}.

\paragraph{Tenir la Position}

L'unité ne fait rien. Les unités déjà engagées dans un Corps à Corps ne peuvent choisir que cette réaction.

\paragraph{Tenir la Position et Tirer}

Cette réaction n'est possible qu'à condition que l'unité chargée dispose d'Armes de Tir, que l'unité ennemie la chargeant ait au moins la moitié de son front dans l'Arc Frontal de l'unité chargée, et enfin que l'unité ennemie soit éloignée d'au moins sa valeur de Mouvement (la valeur la plus basse parmi les figurines qui chargent s'il y en a plusieurs). L'unité chargée effectue immédiatement une Attaque de Tir, avant que l'unité ennemie ne se déplace, comme elle le ferait durant la Phase de Tir, et ce même si l'ennemi est au delà de la portée maximale des armes utilisées. N'oubliez pas de prendre en compte les modificateurs pour toucher appropriés, comme Longue Portée ou Tenir la Position et Tirer. Suivez ensuite les règles de Tenir la Position. Une unité ne peut déclarer cette réaction qu'une seule fois par tour, même si elle est chargée à plusieurs reprises.

\paragraph{Fuir}

L'unité chargée fuit immédiatement dans la direction opposée de l'unité chargeant, en suivant la ligne droite passant par les centres des deux unités. Après qu'une unité a fait son mouvement de fuite, toutes les unités qui avaient déclaré une charge vers elle peuvent immédiatement tenter une Redirection de charge. Une unité chargée déjà en fuite doit toujours choisir cette réaction.

\texttt{NDT :} Pour illustrer les choix possibles de réactions, une unité peut très bien déclarer Tenir la Position en réaction à une première charge, Tenir la Position et Tirer suite à une deuxième charge, et enfin Fuir face à une troisième charge.

\paragraph{Rediriger une charge}

Quand une unité choisit de fuir en réaction à une charge, l'unité chargeant peut essayer de Rediriger sa Charge, en passant un test de Commandement. Si elle le rate, elle doit essayer de finir sa charge vers l'unité ayant fui. Si elle le réussit, elle peut immédiatement déclarer une nouvelle charge vers une autre unité, qui pourra à son tour choisir une réaction normalement. Si plusieurs unités avaient déclaré une charge vers l'unité fuyant, chaque unité peut tenter une redirection, dans l'ordre choisi par le Joueur Actif. Une unité ne peut rediriger sa charge qu'une seule fois par tour. Toutefois, si l'unité vers laquelle la charge est redirigée fuit aussi, alors la charge peut être tentée vers l'une ou l'autre des unités en fuite. Déclarez laquelle avant de jeter les dés pour la distance de charge.

\newpage
\hypertarget{movechargers}{\subsection{Déplacement des unités en charge}}
\label{move_chargers}

Une fois que toutes les charges et leurs réactions ont été déclarées, les unités en charge essayent d'atteindre leur cible. Le Joueur Actif choisit une unité qui a déclaré une charge et jette les dés pour déterminer la portée de charge de cette unité, puis la déplace. Répétez ce procédé pour chaque unité qui a déclaré une charge durant cette phase.

\paragraph{Portée de charge}

La portée de charge d'une unité est normalement sa valeur de Mouvement à laquelle on ajoute 2D6. Si le résultat est \textbf{égal ou supérieur} à la distance entre l'unité et sa cible, on dit que la portée de charge est suffisante. L'unité peut alors faire son mouvement de charge, si elle en a la place. Si la portée de charge obtenue est strictement inférieure à la distance entre les unités, ou s'il n'y a pas assez de place pour faire le mouvement de charge, la charge est ratée. L'unité en charge doit alors faire un mouvement de charge ratée (voir plus bas).

\paragraph{Mouvement de charge}

Un mouvement de charge suit les règles suivantes :
\begin{itemize}[label={-}]
\item L'unité chargeant peut avancer tout droit autant qu'elle le souhaite.
\item Une seule roue peut être effectuée pendant ce mouvement. Cette roue ne peut dépasser un angle de 90{\text{\degree}}.
\item L'ennemi doit se trouver en contact socle à socle avec l'avant de l'unité en charge, du côté de l'arc où se trouvait la majorité de l'avant de l'unité en charge quand la charge a été déclarée (voir la figure \ref{figure/chargefrontage}). Si le front de l'unité en charge est divisé en deux équitablement, tirez au hasard de quelle côté l'unité se trouve avant de déclarer une charge.
\item L'unité en charge ignore la Règle du Pouce d'Écart. Elle ne peut cependant entrer en contact socle à socle avec un ennemi que si elle lui a déclaré une charge.
\end{itemize}

\paragraph{Alignement}
\label{aligning_units}

Si l'unité en charge arrive à entrer en contact socle à socle, les unités doivent être alignées l'une avec l'autre, de manière à ce que leurs deux côtés soient parallèles et en contact. Le Joueur Actif tourne l'unité en charge autour du point de contact des deux unités, vers l'ennemi (voir figure \ref{figure/chargefrontage}). Si cela ne suffit pas à amener les deux unités en contact complet, parce que d'autres unités ou des Décors bloquent la rotation, tournez l'unité chargée à la place, vers l'unité en charge, ou encore une combinaison des deux, en tournant l'unité chargée le moins possible. L'unité chargée ne doit être déplacée que si c'est le seul moyen d'aligner les unités et ne peut jamais être tournée si elle était déjà engagée au Corps à Corps. Ces mouvements d'alignement font partie du mouvement de charge de l'unité et, de ce fait, ignorent la Règle du Pouce d'Écart. Une unité chargée ne fait jamais de test de Terrain Dangereux lors d'un mouvement d'alignement.

\newpage
\paragraph{Maximiser le contact}

Les mouvements de charge doivent être effectués de manière à satisfaire les conditions suivantes, par ordre décroissant de priorité :

\begin{itemize}[label={-}]
\item Première priorité : Ne charger qu'une seule unité ennemi à la fois. S'il n'est pas possible de réussir une charge sans impacter plusieurs unités, toutes ces dernières peuvent déclarer une réaction à la charge.
\item Deuxième priorité : L'unité chargée ne doit pas effectuer de rotation (voir le paragraphe ci-dessus). Si la rotation de l'unité chargée est inévitable, faire pivoter cette dernière aussi peu que possible. Les unités engagées au Corps à Corps ne peuvent jamais pivoter dans ce contexte.
\item Troisième priorité : Le nombre total d'unités alliées dans le combat doit être maximisé. Cette condition n'est applicable que si plusieurs unités chargent la même unité.
\item Quatrième priorité : Le maximum de figurines des deux camps doit être en contact socle à socle avec au moins une figurine ennemie, en comptant celles qui combattent à travers un vide.
\end{itemize}

S'il n'est pas possible de satisfaire à au moins une priorité, vous devez essayer de respecter les priorités les plus hautes dans la liste, même si cela signifie que plus de priorités ne seront pas satisfaites. Du moment que toutes les conditions ci-dessus sont remplies du mieux possible, le Joueur Actif peut placer ses unités en charge comme il lui plaît en suivant les règles de mouvement de charge.

\paragraph{Charger une unité en fuite}

Quand vous chargez une unité en fuite, suivez les mêmes règles que pour un mouvement de charge ordinaire, sauf que vous pouvez entrer au contact de n'importe quel côté de l'unité chargée et qu'aucun alignement ni maximisation ne doivent être faits. Si l'unité en charge entre en contact avec une unité en fuite, cette dernière est retirée comme perte. L'unité en charge peut alors passer un test de Commandement. S'il est réussi, l'unité peut effectuer un Pivot Post-Combat.

\paragraph{Charges multiples}

Si plusieurs unités déclarent une charge contre un même ennemi, les mouvements de charge sont faits d'une manière légèrement différente. Déterminez les distances de charge de toutes les unités concernées avant de les déplacer. Une fois établi quelles unités vont atteindre la cible, faites les déplacements de charges réussies et ratées en respectant l'ordre des priorités expliqué dans le paragraphe Maximiser le contact.

\paragraph{Charge impossible}

Quand les mouvements de charge sont effectués, une unité peut quelquefois en empêcher une autre de réussir sa charge. Quelquefois, il n'y a pas assez de place pour faire tenir toutes les unités en charge. Quand cela arrive, les unités ne pouvant plus atteindre le combat font un mouvement de charge ratée.

\newpage
\paragraph{Chemin Bloqué}
\label{blocked_path}

Pour éviter certaines situations d'abus où une unité ne peut pas charger une unité ennemie pourtant bien à portée et dans son champ de vision à cause d'un positionnement alambiqué des unités ennemies, appliquez la règle suivante. Si une unité ne peut pas réussir une charge seulement à cause d'unités ennemies non engagées dans des Corps à Corps qu'elle ne pourrait pas charger normalement, elle peut faire un mouvement spécial de charge. Bougez l'unité droit devant elle, jusqu'à sa distance de charge obtenue aux dés. Si cela amène l'unité en contact avec un ou plusieurs ennemis, ils comptent comme étant chargés. Au lieu de procéder à l'alignement normal, l'ennemi fait une Reformation de Combat de manière à ce que les unités soient alignées l'une avec l'autre. La Reformation doit être effectuée de façon à ce que le côté par lequel l'unité est chargée reste le même, que l'unité chargée conserve le même nombre de rangs et de colonnes, et que le nombre maximal de figurines dans les deux camps est en contact avec un ennemi. S'il n'est pas possible d'aligner les unités sans changer le nombre de rangs et de colonnes, vous pouvez le modifier et vous n'avez alors pas à maximiser le nombre de figurines au contact. Si l'unité ennemie ne peut pas faire la Reformation de Combat de manière à s'aligner, ce mouvement spécial de charge ne peut pas être utilisé.

La figure \ref{figure/blockedpath} présente un cas où la règle Chemin Bloqué est applicable.

\hypertarget{failedchargemove}{\paragraph{Charge ratée}}

Si une unité obtient une portée de charge insuffisante, ou si la charge est un échec pour une autre raison, l'unité fait un mouvement de charge ratée. La longueur de ce déplacement est la valeur la plus grande parmi les résultats des dés lancés pour déterminer la portée de charge, en pouces. Faites faire une roue à l'unité, de manière à ce qu'un déplacement droit devant se fasse dans l'axe passant par le centre de l'unité et celui de sa cible, puis faites-la avancer. Ce n'est pas un mouvement de charge, donc la Règle du Pouce d'Écart ne doit pas être ignorée. Si l'unité chargée a été détruite avant de déplacer l'unité en charge, marquez le dernier emplacement du centre de l'unité disparue et faites le mouvement vers ce point. Une unité qui a raté une charge ne peut plus bouger durant cette Phase de Mouvement et ne peut pas tirer à la Phase de Tir qui suit.

\newcommand{\morethanhalfoffrontageisontheorangeunitsfrontarc}{\normalfontsize{Au moins la moitié du front est dans l'arc frontal de l'unité orange}}
\newcommand{\chargefrontageCharge}{\normalfontsize{\flufffont{Charge !}}}

\begin{figure}[!htbp]
\centering
\def\svgwidth{0.6\textwidth}
\input{pics/charge_frontage.pdf_tex}
\caption{Avant ou flanc ?\vspace*{10pt}\newline
La majorité de l'avant de l'unité verte, qui charge, est dans l'arc frontal de l'unité ennemie, en orange. L'unité verte doit donc entrer en contact avec la face avant de l'unité orange.}
\label{figure/chargefrontage}
\end{figure}

\newcommand{\blockedpathCharge}{\normalfontsize{\flufffont{Charge !}}}

\begin{figure}[!htbp]
\centering
\def\svgwidth{0.6\textwidth}
\input{pics/blocked_path.pdf_tex}
\caption{Cas d'application de la règle Chemin Bloqué.\vspace*{10pt}\newline
L'unité bleue charge l'unité verte à gauche mais aucune des deux unités ne peut s'aligner, et ce uniquement à cause de l'unité vert clair à droite. L'unité bleue fait un mouvement de charge de Chemin Bloqué : elle avance droit devant, jusqu'à entrer en contact avec l'unité verte qui effectue ensuite une Reformation de Combat afin de s'aligner.}
\label{figure/blockedpath}
\end{figure}

\newcommand{\chargealignmentA}{a)}
\newcommand{\chargealignmentB}{b)}
\newcommand{\chargealignmentOne}{1}
\newcommand{\chargealignmentTwo}{2}
\newcommand{\chargealignmentCharge}{\normalfontsize{\flufffont{Charge !}}}

\begin{figure}[!htbp]
\centering
\hypertarget{chargealignmentfigure}{
\def\svgwidth{0.7\textwidth}
\input{pics/maximized_charge_models.pdf_tex}}
\caption{Maximiser le contact.\vspace*{10pt}\newline
a) L'unité violette, en charge, essaye de maximiser le nombre de figurines en contact. Cependant, les unités ne peuvent pas être alignées sans bouger l'unité verte, chargée.\vspace*{10pt}\newline
b) Comme l'unité violette pourrait charger de manière à ce que l'unité verte ne soit pas déplacée, elle doit plutôt effectuer ce mouvement.}
\label{figure/chargealignment}
\end{figure}

\newcommand{\multiplechargesCharge}{\flufffont{Charge !}}
\newcommand{\multiplechargesOne}{1)}
\newcommand{\multiplechargesTwoA}{2.a)}
\newcommand{\multiplechargesTwoB}{2.b)}
\newcommand{\multiplechargesTwoC}{2.c)}
\newcommand{\multiplechargesTwoD}{2.d)}

\begin{figure}[!htbp]
\begin{minipage}{0.5\textwidth}
\def\svgwidth{\textwidth}
\input{pics/multiplecharges.pdf_tex}
\end{minipage}\hfill\begin{minipage}{0.47\textwidth}
\caption{Charges Multiples.}
\label{figure/multiplecharges}

\vspace*{10pt}
\noindent 1) Des charges multiples sont déclarées contre une unité. Suivons les priorités données dans le paragraphe Maximiser le contact :
\begin{enumerate}
\item Pas de rotation.
\item Maximiser le nombre d'unités au combat.
\item Maximiser le nombre de figurines en contact socle à socle avec un ennemi.
\end{enumerate}

\vspace*{10pt}
\noindent 2.a) Le nombre d'unités est maximisé (4). Une fois cette priorité respectée, le nombre de figurines en contact socle à socle est maximisé (7 contre 4 = 11 figurines).

\vspace*{10pt}
\noindent 2.b) Autre possibilité : le nombre d'unités est maximisé (4). Une fois cette priorité respectée, le nombre de figurines en contact socle à socle est maximisé (7 contre 4 = 11 figurines).

\vspace*{10pt}
\noindent 2.c) Le nombre d'unités est maximisé (4). Cependant, le nombre de figurines en contact socle à socle n'est pas maximisé (6 contre 4 = 10 figurines).

\vspace*{10pt}
\noindent 2.d) Le nombre d'unités n'est pas maximisé (3).
\end{minipage}
\end{figure}

\newpage
\hypertarget{compulsorymoves}{\section{Mouvements Obligatoires}}

Pendant cette étape de la Phase de Mouvement, toutes les unités qui ne choisissent pas si elles vont bouger ou non doivent bouger. Les unités en fuite, celles avec la règle \randommovement{}, ou encore celles ayant raté un test de \stupidity{} en font partie. Commencez par faire les Tests de Ralliement pour toutes les unités alliées en fuite. Effectuez les déplacements appropriés suivant la réussite ou non de ces tests. Enfin, déplacez les autres unités qui bougent pendant cette étape dans l'ordre de votre choix.

\hypertarget{rallytest}{\paragraph{Test de Ralliement}}

Au début de l'étape des Mouvements obligatoires, toutes les unités alliées en fuite doivent passer un test de Commandement, dans l'ordre souhaité par le Joueur Actif. Les unités qui tombent à un quart ou moins de leur effectif initial, l'effectif inscrit sur la liste d'armée, en prenant en compte les Personnages ayant rejoint l'unité, doivent passer leur test de Ralliement avec une valeur de Commandement divisée par deux, en arrondissant au supérieur. Par exemple, prenons une unité qui a commencé la partie avec 40 figurines. S'il en reste 9, mais que 2 Personnages ont rejoint l'unité, elle peut tout juste tenter un test de Ralliement avec son Commandement non divisé. Une unité qui réussit ce test n'est plus en fuite et peut immédiatement effectuer une Reformation. Une unité qui vient de se rallier ne peut plus bouger durant cette Phase de Mouvement, ni tirer durant la Phase de Tir qui suit. Si le test de Ralliement est raté, l'unité effectue immédiatement un mouvement de fuite.

\hypertarget{fleemove}{\paragraph{Mouvement de fuite}}

La distance de fuite est normalement de \distance{2D6}. Déplacez l'unité en fuite droit devant de la distance obtenue. Si ce mouvement devait faire terminer l'unité à moins d'\distance{1} d'une autre unité ou d'un Terrain Infranchissable, augmentez la distance de fuite du minimum nécessaire pour passer au-delà des obstacles. Si des figurines d'une unité en fuite traversent des figurines ennemies ou un Terrain Infranchissable, elles doivent passer un test de Terrain Dangereux (3). Si ce mouvement de fuite amène l'unité en contact d'un bord de table, l'unité est immédiatement détruite. Cela provoque éventuellement des tests de Panique aux unités aux alentours. Notez que ce mouvement de fuite est souvent précédé d'un Pivot, qui suit les mêmes règles que le mouvement de fuite. Les mouvements de fuite ignorent tout obstacle.

\subsection{Unités en fuite}

Quand une unité fuit, elle ne peut effectuer aucune action volontaire. Cela signifie que si une unité a normalement la possibilité de ne pas effectuer une action, alors elle ne peut pas effectuer cette action lors d'une fuite. Cela inclut, par exemple, déclarer des charges, déclarer une réaction à une charge autre que la fuite, se déplacer d'une autre façon qu'avec un mouvement de fuite, tirer, canaliser un Dé de Magie, lancer des sorts, dissiper des sorts ou encore activer des objets à Usage Unique. Enfin, une figurine en fuite ne peut pas faire profiter les unités proches de ses règles \inspiringpresence{} ou \holdyourground{}.

\newpage
\hypertarget{remainingmoves}{\section{Autres Mouvements}}

Les unités qui n'ont pas encore bougé durant cette phase ont l'occasion de le faire pendant l'étape des Autres Mouvements :

\hspace*{0.3cm}
\begin{tabular}{c|m{12cm}}
1 & Début de l'étape. Les Renforts arrivent.\tabularnewline
2 & Choisissez une unité à déplacer, et un type de mouvement parmi Mouvement Simple, Marche Forcée et Reformation, puis bougez-la.\tabularnewline
3 & Repassez au point 2 s'il reste des unités qui n'ont pas encore bougé durant la phase et que vous voulez les déplacer.\tabularnewline
4 & Si toutes les unités qui pouvaient bouger et que vous vouliez déplacer l'ont fait, l'étape est terminée.\tabularnewline
\end{tabular}

\hypertarget{advancemove}{\subsection{Mouvement Simple}}

Pendant un Mouvement Simple, une unité peut avancer, reculer ou faire des pas de côté. Elle ne peut cependant pas combiner ces directions. Les unités composées d'une seule figurine peuvent faire autant de Pivots qu'elles le souhaitent pendant un Mouvement Simple. Aucune figurine d'une unité effectuant un Mouvement Simple ne peut se déplacer sur une distance supérieure à sa valeur de Mouvement, en comparant la position initiale et la position finale. Si le déplacement fait suite à une Reformation Rapide, la distance est mesurée depuis la position après la Reformation.

\noindent\textbf{En Avant :} L'unité avance droit devant sur une distance maximale de la valeur de son Mouvement, en pouces. Elle peut faire autant de Roues que vous le souhaitez.

\noindent\textbf{En Arrière :} L'unité recule droit sur une distance maximale de la moitié de la valeur de son Mouvement. Par exemple, une unité avec une Caractéristique de Mouvement de 5 peut reculer de 2,5\distance{}.

\noindent\textbf{Pas de Côté :} L'unité se déplace d'un des deux côtés sur une distance maximale de la moitié de sa valeur de Mouvement.

\hypertarget{marchmove}{\subsection{Marche Forcée}}
\label{march_move}

Pendant une Marche Forcée, une unité ne peut qu'avancer, et sur une distance maximale de deux fois sa caractéristique de Mouvement, en pouces. Elle peut faire autant de Roues que vous le souhaitez. Aucune figurine d'une unité effectuant une Marche Forcée ne peut se déplacer au delà du double de sa valeur de Mouvement, en comparant la position initiale et la position finale.

Si des unités ennemies se trouvent à moins de \distance{8} de l'unité voulant effectuer une Marche Forcée, cette dernière doit passer un test de Marche Forcée avant de bouger. Il s'agit d'un test de Commandement. S'il est réussi, l'unité peut faire une Marche Forcée normalement. Sinon, l'unité doit fait une Marche Forcée avec une pénalité de mouvement : la distance maximale est sa caractéristique de Mouvement au lieu du double. Une unité qui a effectué une Marche Forcée ne peut pas tirer lors de la Phase de Tir qui suit. Une unité composée d'une figurine seule peut faire autant de pivots que vous le souhaitez durant sa Marche Forcée.

\hypertarget{reform}{\subsection{Reformation}}

Repérez le centre de l'unité, puis retirez l'unité du champ de bataille. Vous pouvez la replacer dans n'importe quelle formation autorisée et l'orienter dans n'importe quelle direction, en suivant la Règle du Pouce d'Écart, tant que son centre reste à l'emplacement repéré. Après la Reformation, aucune figurine ne peut se retrouver à plus de deux fois la valeur de son Mouvement de sa position initiale. Une unité qui s'est reformée ne peut pas tirer lors de la Phase de Tir qui suit.

\newpage
\section{Pivots et Roues}

Un Pivot est pratiqué principalement par les figurines seules. Pour effectuer un Pivot, repérez le centre de l'unité et retirez-la du champ de bataille. Replacez-la ensuite dans la même formation, orientée dans la direction de votre choix, tout en maintenant son centre à l'emplacement repéré. Suivez normalement la Règle du Pouce d'Écart.

Quand une unité effectue une Roue, elle fait une rotation autour d'un de ses coins avant, vers l'avant. La distance parcourue par l'unité est égale à celle parcourue par la figurine du coin avant opposé. On considère que toutes les figurines de l'unité ont parcouru cette distance.

\newcommand{\wheelsA}{a)}
\newcommand{\wheelsB}{b)}
\newcommand{\wheelsC}{c)}

\begin{figure}[!htbp]
\centering
\hypertarget{pivotsandwheelsfigure}{
\def\svgwidth{\textwidth}
\input{pics/wheels.pdf_tex}}
\caption{Exemples de Roues.\vspace*{10pt}\newline
Les unités de cet exemple ont chacune un Mouvement de 5.\vspace*{10pt}\newline
a) L'unité verte fait deux Roues pendant une Marche Forcée. Elle compte comme ayant bougé de \distance{10}, puisque la mesure de distance est prise au niveau du coin opposé au coin qui sert de point de rotation.\vspace*{10pt}\newline
b) L'unité turquoise fait une seule Roue pendant une Marche Forcée. Cependant, ce mouvement est non réglementaire, car même si la figurine du coin opposé n'a bougé que de \distance{9}, certaines figurines de l'unité sont à plus de \distance{10} de leur position initiale (voir le paragraphe \ref{march_move}, page \pageref{march_move}).\vspace*{10pt}\newline
c) L'unité jaune fait une seule Roue pendant une Marche Forcée. Elle compte comme ayant bougé de \distance{10}, puisque la mesure de distance est prise au niveau du coin opposé au coin qui sert de point de rotation. Remarquez qu'aucune figurine n'a bougé de plus de \distance{10} entre sa position initiale et sa position finale.}
\label{figure/wheels}
\end{figure}
