
\part{Équipement Standard}
\label{mundane_equipment}

Les armes et armures qui ne sont pas des Objets Magiques sont des équipements standard.

\hypertarget{closecombatweapons}{\section{Armes de Corps à Corps}}
\label{close_combat_weapons}

Les armes listées ici sont utilisées au Corps à Corps et peuvent apporter avantages comme inconvénients. Les règles des armes n'ont d'effet que lorsque la figurine utilise l'arme en question. Elles ne s'appliquent donc pas aux Attaques Spéciales telles que le \stomp{} ou quand la figurine se bat avec une autre arme.

\renewcommand{\arraystretch}{2}
\begin{center}
\begin{tabular}{>{\raggedleft\bfseries}p{2.5cm}p{12.5cm}}
\hline
\textnormal{Arme} & Règles \tabularnewline
\hw{} & Toutes les figurines sont équipées d'une \hw{}. Une \hw{} ne peut jamais être perdue, détruite ni neutralisée. Si une figurine possède n'importe quelle autre Arme de Corps à Corps, elle ne peut pas choisir d'utiliser son \hw{}, à moins que le contraire ne soit précisé. Une \hw{} maniée par une figurine à pied peut être utilisée avec un \shield{} pour obtenir la règle \parry{}.\vspace*{5pt}\newline
\newfromWHB{\parry{} : Les Attaques de Corps à Corps portées par des ennemis sur le front de la figurine ne peuvent jamais toucher leur cible plus facilement que sur 4+. Ceci n'est pas un modificateur pour toucher. Appliquez cette règle avant tout modificateur pour toucher.} \tabularnewline
\gw{} & \requirestwohands{}. Les attaques portées avec une \gw{} ont +2 en Force\newfromWHB{, mais sont faites à Initiative 0, quelle que soit l'Initiative du porteur}. \tabularnewline
\flail{} & \requirestwohands{}. \newfromWHB{Les attaques portées avec un \flail{} ont +2 en Force. Les Attaques de Corps à Corps allouées contre le porteur ont un bonus de +1 pour toucher.} \tabularnewline
\halberd{} & \requirestwohands{}. Les attaques portées avec une \halberd{} ont +1 en Force. \tabularnewline
\spear{} & Quand il utilise cette arme, le porteur gagne la règle \fightinextrarank{}. Les attaques portées avec une \spear{} ont la règle \armourpiercing{1}, plus la règle \lethalstrike{} quand elles sont allouées à de la \cavalry{}, des \chariots{} ou de la \monstrouscavalry{} engagés sur le front du porteur. Les figurines montées ne peuvent pas utiliser de \spear{}. \tabularnewline
\lance{} & Les attaques portées avec une \lance{} ont +2 en Force pendant la Manche de Corps à Corps suivant directement une charge du porteur. Ce bonus ne peut être utilisé que pour les attaques allouées à l'unité chargée. Seule une figurine montée, \newfromWHB{une \warbeast{} ou une \monstrousbeast{} peuvent utiliser une \lance{}}. \tabularnewline
\newfromWHB{\lightlance} & \newfromWHB{Suit les mêmes règles qu'une \lance{}, mais ne donne qu'un bonus de +1 en Force.} \tabularnewline
\newfromWHB{\pw} & \requirestwohands{}. Quand il utilise cette arme, le porteur a +1 Attaque \newfromWHB{et +1 en Initiative}. \tabularnewline
\hline
\end{tabular}
\end{center}
\renewcommand{\arraystretch}{1.5}

\subsection{Choix d'Arme}

Quand une figurine possède plusieurs Armes de Corps à Corps, elle doit choisir laquelle utiliser au début de chaque combat et doit continuer à l'utiliser pour toute la durée du combat. Quand une figurine est attaquée par une Attaque à Distance, elle doit utiliser les armes et armures qui lui confèrent un bonus de Sauvegarde, même si elle est en train d'utiliser une autre arme au Corps à Corps. Par exemple, une figurine qui combat avec une \gw{} et qui est ciblée par une Attaque à Distance doit utiliser son \shield{} contre cette attaque. Elle doit ensuite continuer à se battre avec l'\gw{}. Toutes les figurines ordinaires d'une même unité doivent choisir la même arme. À moins que le contraire ne soit précisé, les montures ne bénéficient jamais des effets des armes.


\hypertarget{shootingweapons}{\section{Armes de Tir}}
\label{shooting_weapons}

Les armes listées ici sont utilisées pour faire des Attaques de Tir. Chaque figurine ne peut normalement utiliser qu'une seule Arme de Tir par phase, même si elle en possède plusieurs, et toutes les figurines ordinaires non Champion d'une même unité doivent choisir la même Arme de Tir. Chaque arme dispose d'une portée maximale, d'une valeur de Force et de règles spéciales. Les règles spéciales d'une Arme de Tir ne s'appliquent qu'aux Attaques de Tir effectuées avec cette arme.

\renewcommand{\arraystretch}{2}
\begin{center}
\begin{tabular}{>{\raggedleft\bfseries}p{2.5cm}>{\centering}p{1.5cm}>{\centering}p{2cm}p{8.8cm}}
\hline
\textnormal{Arme} & Portée & Force & Règles Spéciales \tabularnewline
\crossbow{} & \distance{30} & 4 & \newfromWHB{\unwieldy} \tabularnewline
\shortbow{} & \distance{18} & 3 & \newfromWHB{\volleyfire} \tabularnewline
\bow{} & \distance{24} & 3 & \newfromWHB{\volleyfire} \tabularnewline
\longbow{} & \distance{30} & 3 & \newfromWHB{\volleyfire} \tabularnewline
\throwingweapons{} & \newfromWHB{\distance{12}} & \newfromWHB{Utilisateur} & \newfromWHB{\multipleshots{2}, \quicktofire} \tabularnewline
\handgun{} & \distance{24} & 4 & \newfromWHB{\unwieldy}, \armourpiercing{1} \tabularnewline
\pistol{} & \distance{12} & 4 & \armourpiercing{1}, \quicktofire{}\newline Compte comme une \pw{} au Corps à Corps. \tabularnewline
\braceofpistols{} & \distance{12} & 4 & \armourpiercing{1}, \multipleshots{2}, \quicktofire{}\newline Compte comme une \pw{} au Corps à Corps. \tabularnewline
\hline
\end{tabular}
\end{center}
\renewcommand{\arraystretch}{1.5}

\newpage
\hypertarget{artilleryweapons}{\section{Armes d'Artillerie}}
\label{artillery_weapons}

Les armes listées ici sont des Armes de Tir particulières. Ces armes sont quelquefois montées sur des figurines de type \warmachine{}, mais elles peuvent aussi être fixées sur le châssis d'un \chariot{}, être portées par un Monstre ou correspondre à un Objet Magique. Les Armes d'Artillerie sont des Armes de Tir et ont toujours la règle \og \reload{} \fg{}. Chaque Arme d'Artillerie a son propre profil avec une portée, une Force et des règles spéciales, que vous trouverez avec sa description. Une Attaque de Tir d'une Arme d'Artillerie n'est pas résolue comme une Attaque de Tir classique. Suivez les règles ci-dessous pour déterminer les touches qu'elle inflige.

\paragraph{\boltthrower}

Une \boltthrower{} suit les règles normales de tir, sauf qu'elle peut pénétrer les rangs ou les colonnes de l'unité ciblée, provoquant des touches supplémentaires. \newfromWHB{Déterminez le nombre maximal de touches possibles en regardant dans quel arc de la cible se trouve la \boltthrower{}. S'il s'agit du front ou de l'arrière, le nombre de touches maximal est égal au nombre de rangs de l'unité ciblée. S'il s'agit d'un flanc, ce sont les colonnes qui sont prises en compte. Si la \boltthrower{} touche sa cible, faites un jet pour blesser et tentez les sauvegardes comme d'habitude. Si une figurine est retirée comme perte, alors le carreau pénètre dans l'unité et provoque une nouvelle touche avec un malus de -1 en Force. Continuez d'ajouter des touches aussi longtemps qu'une figurine est retirée comme perte, avec un malus supplémentaire de -1 en Force à chaque fois, jusqu'à un minimum de 1. Le nombre de touches ne peut pas dépasser le nombre maximal de touches déterminé plus tôt.}

{\normalfontsize
\begin{center}
\begin{tabular}{M{2.7cm}M{2.7cm}M{2.7cm}M{1cm}M{4.8cm}}
\hline
Force de la 1\iere{} touche & Force de la 2\ieme{} touche & Force de la 3\ieme{} touche & ... & Nombre maximal de touches\tabularnewline
\textbf{F} & \textbf{F-1} & \textbf{F-2} & \textbf{etc.} & Nombre initial de rangs ou colonnes \tabularnewline
\hline
\end{tabular}
\end{center}
}

\paragraph{\volleygun}

\newfromWHB{Une \volleygun{} suit les règles normales de tir avec les exceptions suivantes : elle dispose de la règle \multipleshots{} et ne subit jamais la pénalité pour toucher associée. Si le nombre de tirs est un nombre prédéterminé (\multipleshots{6} par exemple), la \volleygun{} ne peut pas subir d'Incident de Tir. Cependant, si le nombre de tirs est aléatoire (\multipleshots{2D6} par exemple), la \volleygun{} peut subir un Incident de Tir. Si un seul \result{6} naturel est obtenu, après relances, sur le jet du nombre de tirs, la \volleygun{} subit un malus de -1 pour toucher. Si au moins deux \result{6} naturels sont obtenus, après relances, sur le jet du nombre de tirs, elle subit un Incident de Tir : tous les tirs sont annulés et la \volleygun{} doit effectuer un jet sur la Table des Incidents de Tir.}

\paragraph{\flamethrower}

\newfromWHB{Placez le centre d'un Gabarit de \distance{3} en Ligne de Vue et à portée. Déplacez ensuite le Gabarit de \distance{1D6} tout droit à l'opposé du \flamethrower{}, selon l'axe passant par le centre de la figurine et le centre du Gabarit. Si un \result{6} est obtenu sur le jet de la distance, un Incident de Tir est survenu. Le tir est annulé et le \flamethrower{} doit effectuer un jet sur la Table des Incidents de Tir avec un malus de -1. Sinon, toutes les figurines touchées par le Gabarit au cours de son déplacement subissent une touche, en utilisant la Force et les règles spéciales données dans le profil de l'arme. Toute unité qui pourrait être touchée par le Gabarit, entre sa position initiale et \distance{5} devant, est considérée comme cible potentielle de l'attaque. Cela ne peut pas être une unité alliée, ni une unité engagée au Corps à Corps.}

\newpage
\paragraph{\cannon{} (\distance{X})}

\newfromWHB{Ciblez une unité, puis désignez un point du socle d'une figurine de la cible en Ligne de Vue du \cannon{}. Lancez les dés pour toucher normalement, en ignorant les malus dûs aux Couverts, Légers ou Lourds, et en ajoutant un bonus de +1 pour toucher si la figurine sous le point choisi a la règle \largetarget{}. Si le jet pour toucher est raté, il ne peut jamais être relancé. Si un \result{1} naturel est obtenu, le \cannon{} subit un Incident de Tir : le tir est annulé, et le \cannon{} doit effectuer un jet sur la Table des Incidents de Tir.}

\newfromWHB{Si le jet pour toucher est réussi, tracez une ligne de \distance{X} (X étant la valeur entre parenthèses) depuis le point désigné et directement à l'opposé du centre de la figurine portant le \cannon{}. Cette ligne est immédiatement arrêtée si elle rencontre un Mur ou un Terrain Infranchissable.}

\newfromWHB{La figurine sous le point d'impact initial du boulet subit une touche avec la Force et les règles données dans le profil du \cannon{}. Les autres figurines sous la ligne peuvent subir une touche avec les mêmes règles, mais avec une Force divisée par 2.} En commençant par la figurine qui a été touchée la première, c'est-à-dire sous le point désigné, lancez le jet pour blesser et tentez les sauvegardes. Si la figurine est retirée comme perte, faites le jet pour blesser puis les sauvegardes pour la figurine suivante sur la ligne, et ainsi de suite. \newfromWHB{Si une figurine survit, le boulet de canon s'arrête et les figurines suivantes ne sont pas touchées.}

{\normalfontsize
\begin{center}
\begin{tabular}{M{2.7cm}M{2.7cm}M{2.7cm}M{1cm}M{4.8cm}}
\hline
Force de la 1\iere{} touche & Force de la 2\ieme{} touche & Force de la 3\ieme{} touche & ... & Conditions pour toucher\tabularnewline
\textbf{F} & \textbf{\newfromWHB{F/2}} & \textbf{\newfromWHB{F/2}} & \textbf{\newfromWHB{F/2}} & Ligne gabarit \newfromWHB{et Ligne de Vue} \tabularnewline
\hline
\end{tabular}
\end{center}
}

\paragraph{\catapult{} (\distance{X})}
\label{catapult}

Placez le centre d'un Gabarit de taille \distance{X} au-dessus d'une figurine ennemie, à portée et dans la Ligne de Vue de la figurine portant la \catapult{}. Les unités sous la position initiale du Gabarit sont considérées comme étant les cibles de l'Attaque de Tir. Aucune partie du gabarit ne peut être placée sur une figurine alliée ou sur des unités engagées au Corps à Corps. \newfromWHB{Faites ensuite dévier le Gabarit de \distance{1D6x2}. Si un \result{6} naturel est obtenu pour la distance de déviation, avant la multiplication par 2, un Incident de Tir est survenu. Le tir est alors annulé et la \catapult{} doit effectuer un jet sur la Table des Incidents de Tir}.

Le Centre du Gabarit peut également être placé en dehors de la Ligne de Vue de la \catapult{}, mais toujours au-dessus d'une figurine ennemie, à portée et de manière à ce qu'aucune partie du Gabarit ne touche une figurine alliée ou une unité engagée au Corps à Corps. Dans ce cas, le gabarit ne reste pas en place si un \og Touché ! \fg{} est obtenu sur le Dé de Déviation. Au lieu de cela, déplacez le Gabarit dans une direction aléatoire, mais retranchez la Capacité de Tir à la distance sur laquelle le Gabarit bouge (soit \distance{1D6x2 - CT}). Si la distance obtenue est de 0 ou moins, l'unité est touchée et le Gabarit n'est pas dévié.

\begin{center}
\begin{tabular}{c c c}
\hline
 & \textbf{Ligne de Vue} & \textbf{Pas de Ligne de Vue} \tabularnewline
\textbf{Flèche de Déviation} & \distance{1D6x2} & \distance{1D6x2} \tabularnewline
\textbf{Touché !} & Touche directe & \distance{1D6x2 - CT} \tabularnewline
\hline
\end{tabular}
\end{center}

Une fois que la position finale du Gabarit est déterminée, toutes les figurines sous le Gabarit subissent une touche, avec la Force et les règles spéciales précisées dans le profil de l'arme. Certaines Catapultes ont une Force plus grande ou des règles spéciales indiquées entre crochets (comme Force 3 [9]). \newfromWHB{L'effet entre crochets affecte seulement la figurine située sous le centre du Gabarit.}

\newpage
\hypertarget{themisfiretable}{\subsection{Table des Incidents de Tir}}
\label{the_misfire_table}

Quand une Arme d'Artillerie subit un Incident de Tir, lancez 1D6 et consultez l'effet correspondant au résultat dans la table \ref{table/misfire_table}. Les résultats de 0 ou moins surviennent lorsque l'Arme d'Artillerie subit un malus sur son jet, comme par exemple pour le \flamethrower{}.

\begin{table}[!htbp]
\centering
\begin{tabular}{M{2cm}m{12cm}}
\textbf{Résultat} & \centering\newfromWHB{\textbf{Effet}} \tabularnewline
\hline
\textbf{0 ou moins} & \textbf{Explosion !}\vspace*{3pt}\newline 
Toutes les figurines à moins de \distance{1D6} de l'Arme d'Artillerie subissent une touche de Force 5. L'Arme d'Artillerie est détruite, retirez-la comme perte. \tabularnewline
\textbf{1 à 2} & \textbf{Défaillance Critique}\vspace*{3pt}\newline 
Le mécanisme de tir est endommagé. La figurine ne peut plus tirer avec cette arme pour le restant de la partie. \tabularnewline
\textbf{3 à 4} & \textbf{Enrayé}\vspace*{3pt}\newline
Cette Arme d'Artillerie ne peut pas être utilisée au prochain Tour de Joueur du propriétaire. \tabularnewline
\textbf{5+} & \textbf{Dysfonctionnement}\vspace*{3pt}\newline
La figurine subit une blessure sans sauvegarde d'aucune sorte possible. \tabularnewline
\hline
\end{tabular}
\caption{Effets d'un Incident de Tir.}
\label{table/misfire_table}
\end{table}

\newpage
\hypertarget{armourtypes}{\section{Types d'Armure}}
\label{armour_types}

La Sauvegarde d'Armure d'une figurine ou élément de figurine est déterminée par son Armure, parfois modifiée par des règles spéciales et des sorts. La Sauvegarde d'Armure est calculée en combinant toutes les pièces d'Armure. Chaque pièce d'Armure ajoute un bonus au jet de sauvegarde, pour atteindre un maximum de +6. Si le jet de sauvegarde, en incluant les modificateurs, est supérieur ou égal à 7, le jet de Sauvegarde d'Armure est réussi. Un résultat non modifié de \result{1} sur le dé est toujours un échec.

Il existe 5 types différents d'Armure.

\begin{multicols}{2}
\vspace{3.25ex plus 1ex minus .2ex}
\begin{center}\noindent\textbf{Armure Complète}\end{center}
\vspace{1.5ex plus .2ex}

Un élément de figurine ne peut porter qu'une seule Armure Complète.

\noindent\begin{itemize}[label={-}, topsep=0cm, itemsep=0pt]
\item \la{} : +1
\item \ha{} : +2
\item \platearmour{} : +3
\end{itemize}

\vspace{3.25ex plus 1ex minus .2ex}
\begin{center}\noindent\textbf{Montures}\end{center}
\vspace{1.5ex plus .2ex}

\noindent\begin{itemize}[label={-}, topsep=0cm, itemsep=0pt]
\item \newfromWHB{\mountsprotection{6} : +1}
\item \newfromWHB{\mountsprotection{5} : +2}
\end{itemize}

Peu importe le nombre de montures qu'une figurine peut avoir, elle ne peut bénéficier que d'une seule instance de la règle ci-dessus.

\noindent\begin{itemize}[label={-}, topsep=0cm, itemsep=0pt]
\item \barding{} : +1. La monture subit un malus de -1 en Mouvement.
\end{itemize}

\columnbreak

\vspace{3.25ex plus 1ex minus .2ex}
\begin{center}\noindent\textbf{Boucliers}\end{center}
\vspace{1.5ex plus .2ex}

Un élément de figurine ne peut porter qu'un seul Bouclier. Au Corps à Corps, un Bouclier ne peut pas être utilisé avec une arme qui possède la règle \requirestwohands{}.

\noindent\begin{itemize}[label={-}, topsep=0cm, itemsep=0pt]
\item \shield{} : +1
\end{itemize}

\vspace{3.25ex plus 1ex minus .2ex}
\begin{center}\noindent\textbf{\newfromWHB{\innatedefence{}}}\end{center}
\vspace{1.5ex plus .2ex}

Un élément de figurine ne peut bénéficier que d'une instance de cette règle : utilisez la meilleure disponible.

\noindent\begin{itemize}[label={-}, topsep=0cm, itemsep=0pt]
\item \innatedefence{6} : +1
\item \innatedefence{5} : +2
\item \innatedefence{4} : +3
\item et ainsi de suite.
\end{itemize}

\end{multicols}

\paragraph{Autres}

Il existe d'autres moyens d'augmenter la Sauvegarde d'Armure : des équipements spéciaux, des Objets Magiques (un heaume par exemple), des règles spéciales, certains sorts, etc. mais seulement jusqu'à un maximum de +6.

Par exemple, si une figurine s'équipe d'une \la{} (+1), d'un \shield{} (+1), d'un Heaume (+1) et qu'elle monte un destrier avec \mountsprotection{6} (+1) avec un \barding{} (+1), elle totalise un bonus de +5 à ses jets de Sauvegarde d'Armure. Cela veut dire qu'un jet de \result{2} ou plus donnera un résultat supérieur ou égal à \result{7} et sera réussi. Il est d'usage d'appeler une telle sauvegarde \og Sauvegarde d'Armure de 2+ \fg{}. Remarquez qu'avec une limite de +6 à la Sauvegarde d'Armure, aucune figurine ne peut avoir de meilleure Sauvegarde d'Armure que 1+. Rappelons que même avec une telle sauvegarde, un jet non modifié de \result{1} est toujours un échec.
