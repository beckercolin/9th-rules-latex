
\part{Équipement Standard}
\label{mundane_equipment}

Les armes et armures qui ne sont pas des Objets Magiques sont des équipements standard.

\hypertarget{closecombatweapons}{\section{Armes de Corps à Corps}}
\label{close_combat_weapons}

Les armes listées ici sont utilisées au Corps à Corps et peuvent apporter avantages comme inconvénients. Les règles des armes n'ont d'effet que lorsque la figurine utilise l'arme en question. Elles ne s'appliquent donc pas aux Attaques Spéciales telles que le \stomp{} ou quand la figurine se bat avec une autre arme.

\vspace*{10pt}
\renewcommand{\arraystretch}{2}
\begin{center}
\begin{tabular}{>{\raggedleft\bfseries}p{2.5cm}p{12.5cm}}
\hline
\textnormal{Arme} & Règles \tabularnewline
\hw{} & Toutes les figurines sont équipées d'une \hw{}. Une \hw{} ne peut jamais être perdue, détruite ni neutralisée. Si une figurine possède n'importe quelle autre Arme de Corps à Corps, elle ne peut pas choisir d'utiliser son \hw{}, à moins que le contraire ne soit précisé. Une \hw{} maniée par une figurine à pied peut être utilisée avec un \shield{} pour obtenir la règle \parry{}.\tabularnewline
\gw{} & \requirestwohands{}. Les attaques portées avec une \gw{} ont +2 en Force, mais sont faites à Initiative 0, quelle que soit l'Initiative du porteur. \tabularnewline
\flail{} & \requirestwohands{}. Les attaques portées avec un \flail{} ont +2 en Force. Les Attaques de Corps à Corps allouées contre le porteur ont un bonus de +1 pour toucher. \tabularnewline
\halberd{} & \requirestwohands{}. Les attaques portées avec une \halberd{} ont +1 en Force. \tabularnewline
\spear{} & Quand il utilise cette arme, le porteur gagne la règle \fightinextrarank{}. Les attaques portées avec une \spear{} ont la règle \armourpiercing{1}, plus la règle \lethalstrike{} quand elles sont allouées à de la \cavalry{}, des \chariots{} ou de la \monstrouscavalry{} engagés sur le front du porteur. Les figurines montées ne peuvent pas utiliser de \spear{}. \tabularnewline
\lance{} & Les attaques portées avec une \lance{} ont +2 en Force pendant la Manche de Corps à Corps suivant directement une charge du porteur. Ce bonus ne peut être utilisé que pour les attaques allouées à l'unité chargée. Seule une figurine montée, une \warbeast{} ou une \monstrousbeast{} peuvent utiliser une \lance{}. \tabularnewline
\lightlance{} & Suit les mêmes règles qu'une \lance{}, mais ne donne qu'un bonus de +1 en Force. \tabularnewline
\pw{} & \requirestwohands{}. Quand il utilise cette arme, le porteur a +1 Attaque et +1 en Initiative. \tabularnewline
\hline
\end{tabular}
\end{center}
\renewcommand{\arraystretch}{1.5}

\newpage
\subsection{Choix d'Arme}

Quand une figurine possède plusieurs Armes de Corps à Corps, elle doit choisir laquelle utiliser au début de chaque combat et doit continuer à l'utiliser pour toute la durée du combat. Quand une figurine est attaquée par une Attaque à Distance, elle doit utiliser les armes et armures qui lui confèrent un bonus de Sauvegarde, même si elle est en train d'utiliser une autre arme au Corps à Corps. Par exemple, une figurine qui combat avec une \gw{} et qui est ciblée par une Attaque à Distance doit utiliser son \shield{} contre cette attaque. Elle doit ensuite continuer à se battre avec l'\gw{}. Toutes les figurines ordinaires d'une même unité doivent choisir la même arme. À moins que le contraire ne soit précisé, les montures ne bénéficient jamais des effets des armes.


\hypertarget{shootingweapons}{\section{Armes de Tir}}
\label{shooting_weapons}

Les armes listées ici sont utilisées pour faire des Attaques de Tir. Chaque figurine ne peut normalement utiliser qu'une seule Arme de Tir par phase, même si elle en possède plusieurs, et toutes les figurines ordinaires non Champion d'une même unité doivent choisir la même Arme de Tir. Chaque arme dispose d'une portée maximale, d'une valeur de Force et de règles spéciales. Les règles spéciales d'une Arme de Tir ne s'appliquent qu'aux Attaques de Tir effectuées avec cette arme.

\vspace*{10pt}
\renewcommand{\arraystretch}{2}
\begin{center}
\begin{tabular}{>{\raggedleft\bfseries}p{2.5cm}>{\centering}p{1.5cm}>{\centering}p{2cm}p{8.8cm}}
\hline
\textnormal{Arme} & Portée & Force & Règles Spéciales \tabularnewline
\crossbow{} & \distance{30} & 4 & \unwieldy{} \tabularnewline
\bow{} & \distance{24} & 3 & \volleyfire{} \tabularnewline
\longbow{} & \distance{30} & 3 & \volleyfire{} \tabularnewline
\throwingweapons{} & \distance{12} & Utilisateur & \multipleshots{2}, \quicktofire{} \tabularnewline
\handgun{} & \distance{24} & 4 & \unwieldy{}, \armourpiercing{1} \tabularnewline
\pistol{} & \distance{12} & 4 & \armourpiercing{1}, \quicktofire{}\newline Compte comme une \pw{} au Corps à Corps. \tabularnewline
\hline
\end{tabular}
\end{center}
\renewcommand{\arraystretch}{1.5}

\newpage
\hypertarget{artilleryweapons}{\section{Armes d'Artillerie}}
\label{artillery_weapons}

Les armes listées ici sont des Armes de Tir particulières. Ces armes sont quelquefois montées sur des figurines de type \warmachine{}, mais elles peuvent aussi être fixées sur le châssis d'un \chariot{}, être portées par un Monstre ou correspondre à un Objet Magique. Les Armes d'Artillerie sont des Armes de Tir et ont toujours la règle \og \reload{} \fg{}. Chaque Arme d'Artillerie a son propre profil avec une portée, une Force et des règles spéciales, que vous trouverez avec sa description.

\paragraph{\boltthrower}

Les attaques d'une \boltthrower{} disposent de la règle \penetrating{}.

\paragraph{\volleygun{X}}

Les attaques d'une \volleygun{} disposent de la règle \multipleshots{X} et ne subissent pas la pénalité pour toucher associée. Si le nombre de tirs X est aléatoire (\volleygun{2D6} par exemple), suivez la règle suivante. Si un seul \result{6} naturel est obtenu, après relances, sur le jet du nombre de tirs, la ces attaques subissent un malus de -1 pour toucher. Si au moins deux \result{6} naturels sont obtenus, après relances, sur le jet du nombre de tirs, les tirs sont annulés et la \volleygun{} subit un Incident de Tir : effectuez un jet sur la Table des Incidents de Tir et appliquez l'effet correspondant.

\paragraph{\cannon{} (\distance{X})}

Les attaques de \cannon{} disposent de la règle \penetrating{} et ignorent les malus pour toucher dûs aux Couverts, Légers ou Lourds. Elles gagnent un bonus de +1 pour toucher si l'unité ciblée comprend uniquement des figurines avec la règle \toweringpresence{}. Si un \result{1} naturel est obtenu sur le jet pour toucher, le \cannon{} subit un Incident de Tir : le tir est annulé, effectuez un jet sur la Table des Incidents de Tir et appliquez l'effet correspondant. Dans ce cas, le jet ne peut pas être relancé.

\paragraph{\catapult{} (\distance{X})}
\label{catapult}

Les attaques de \catapult{} ignorent les malus pour toucher dûs aux Couverts, Légers ou Lourds. Effectuez le jet pour toucher et suivez la procédure suivante :

\begin{itemize}[label={-}]
\item Sur un résultat naturel de \result{1}, la \catapult{} subit un Incident de Tir : le tir est annulé, effectuez un jet sur la Table des Incidents de Tir et appliquez l'effet correspondant. Dans ce cas, le jet ne peut pas être relancé.
\item Si le jet pour toucher est réussi, l'attaque gagne la règle \areaattack{X}. L'attaque est résolu avec la Force précisée dans la description de la \catapult{}.
\item Sur tout autre résultat, faites un jet pour toucher pour une nouvelle attaque de \catapult{}, appelée Touche Partielle, qui ignore tout Incident de Tir. Si elle touche, elle gagne la règle \areaattack{X-1}, toutes les touches sont réalisées avec une Force divisée par deux en arrondissant au supérieur et perdent les éventuelles règles et Force indiquées entre crochets. Si l'attaque ne touche pas, aucune autre attaque n'est générée.
\end{itemize}

\paragraph{\flamethrower}

Il n'y a pas de jet pour toucher pour le \flamethrower{}. Lancez à la place 1D6 ; ce n'est pas un jet pour toucher. Sur un résultat naturel de \result{1}, le \flamethrower{} subit un Incident de Tir : effectuez un jet sur la Table des Incidents de Tir avec un malus de -1 et appliquez l'effet correspondant. Sur tous les autres résultats, l'attaque est réussie. Utilisez la règle \penetrating{} avec l'exception suivante : ajoutez 1D3 touches par rang (ou colonne) plutôt qu'une seule, avec un maximum de touches par rang (ou colonne) égal au nombre de figurine dans ce rang (ou cette colonne). Certains \flamethrowers{} ont des règles supplémentaires précisées entre accolades, comme \Strength{} 4 \{5\}, \{\multiplewounds{1D3}{}\}. Dans ce cas, utilisez la Force et les règles entre accolades uniquement quand la cible du tir est à Courte Portée.


\newpage
\hypertarget{themisfiretable}{\subsection{Table des Incidents de Tir}}
\label{the_misfire_table}

Un jet pour toucher provoquant un Incident de Tir ne peut pas être relancé. Quand une Arme d'Artillerie subit un Incident de Tir, lancez 1D6 et consultez l'effet correspondant au résultat dans la table \ref{table/misfire_table}. Les résultats de 0 ou moins surviennent lorsque l'Arme d'Artillerie subit un malus sur son jet, comme par exemple pour le \flamethrower{}.

\vspace*{10pt}
\begin{table}[!htbp]
\centering
\begin{tabular}{M{2cm}m{12cm}}
\textbf{Résultat} & \centering\textbf{Effet} \tabularnewline
\hline
\textbf{0 ou moins} & \textbf{Explosion !}\vspace*{3pt}\newline 
Toutes les figurines à moins de \distance{1D6} de l'Arme d'Artillerie subissent une touche de Force 5. L'Arme d'Artillerie est détruite, retirez-la comme perte. \tabularnewline
\textbf{1 à 2} & \textbf{Défaillance Critique}\vspace*{3pt}\newline 
Le mécanisme de tir est endommagé. La figurine ne peut plus tirer avec cette arme pour le restant de la partie. \tabularnewline
\textbf{3 à 4} & \textbf{Enrayé}\vspace*{3pt}\newline
Cette Arme d'Artillerie ne peut pas être utilisée au prochain Tour de Joueur du propriétaire. \tabularnewline
\textbf{5+} & \textbf{Dysfonctionnement}\vspace*{3pt}\newline
La figurine subit une blessure sans sauvegarde d'aucune sorte possible. \tabularnewline
\hline
\end{tabular}
\caption{Effets d'un Incident de Tir.}
\label{table/misfire_table}
\end{table}

\newpage
\hypertarget{armourtypes}{\section{Types d'Armure}}
\label{armour_types}

La Sauvegarde d'Armure d'une figurine ou élément de figurine est déterminée par son Armure, parfois modifiée par des règles spéciales et des sorts. La Sauvegarde d'Armure est calculée en combinant toutes les pièces d'Armure. Chaque pièce d'Armure ajoute un bonus au jet de sauvegarde, pour atteindre un maximum de +6. Si le jet de sauvegarde, en incluant les modificateurs, est supérieur ou égal à 7, le jet de Sauvegarde d'Armure est réussi. Un résultat non modifié de \result{1} sur le dé est toujours un échec.

Il existe 5 types différents d'Armure.

\begin{multicols}{2}
\vspace{3.25ex plus 1ex minus .2ex}
\begin{center}\noindent\textbf{Armure Complète}\end{center}
\vspace{1.5ex plus .2ex}

Un élément de figurine ne peut porter qu'une seule Armure Complète.

\noindent\begin{itemize}[label={-}, topsep=0cm, itemsep=0pt]
\item \la{} : +1
\item \ha{} : +2
\item \platearmour{} : +3
\end{itemize}

\vspace{3.25ex plus 1ex minus .2ex}
\begin{center}\noindent\textbf{Montures}\end{center}
\vspace{1.5ex plus .2ex}

\noindent\begin{itemize}[label={-}, topsep=0cm, itemsep=0pt]
\item \mountsprotection{6} : +1
\item \mountsprotection{5} : +2
\end{itemize}

Peu importe le nombre de montures qu'une figurine peut avoir, elle ne peut bénéficier que d'une seule instance de la règle ci-dessus.

\noindent\begin{itemize}[label={-}, topsep=0cm, itemsep=0pt]
\item \barding{} : +1. La monture subit un malus de -1 en Mouvement.
\end{itemize}

\columnbreak

\vspace{3.25ex plus 1ex minus .2ex}
\begin{center}\noindent\textbf{Boucliers}\end{center}
\vspace{1.5ex plus .2ex}

Un élément de figurine ne peut porter qu'un seul Bouclier. Au Corps à Corps, un Bouclier ne peut pas être utilisé avec une arme qui possède la règle \requirestwohands{}.

\noindent\begin{itemize}[label={-}, topsep=0cm, itemsep=0pt]
\item \shield{} : +1
\end{itemize}

\vspace{3.25ex plus 1ex minus .2ex}
\begin{center}\noindent\textbf{\innatedefence{}}\end{center}
\vspace{1.5ex plus .2ex}

Un élément de figurine ne peut bénéficier que d'une instance de cette règle : utilisez la meilleure disponible.

\noindent\begin{itemize}[label={-}, topsep=0cm, itemsep=0pt]
\item \innatedefence{6} : +1
\item \innatedefence{5} : +2
\item \innatedefence{4} : +3
\item et ainsi de suite.
\end{itemize}

\end{multicols}

\paragraph{Autres}

Il existe d'autres moyens d'augmenter la Sauvegarde d'Armure : des équipements spéciaux, des Objets Magiques (un heaume par exemple), des règles spéciales, certains sorts, etc. mais seulement jusqu'à un maximum de +6.

Par exemple, si une figurine s'équipe d'une \la{} (+1), d'un \shield{} (+1), d'un Heaume (+1) et qu'elle monte un destrier avec \mountsprotection{6} (+1) avec un \barding{} (+1), elle totalise un bonus de +5 à ses jets de Sauvegarde d'Armure. Cela veut dire qu'un jet de \result{2} ou plus donnera un résultat supérieur ou égal à \result{7} et sera réussi. Il est d'usage d'appeler une telle sauvegarde \og Sauvegarde d'Armure de 2+ \fg{}. Remarquez qu'avec une limite de +6 à la Sauvegarde d'Armure, aucune figurine ne peut avoir de meilleure Sauvegarde d'Armure que 1+. Rappelons que même avec une telle sauvegarde, un jet non modifié de \result{1} est toujours un échec.
