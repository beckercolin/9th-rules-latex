% Base sur la VO 0.11.9
% Relecture technique: 
% Relecture syntaxique: 

\part{Figurines, unités et formations}

\section{Figurines}

Les figurines représentent des soldats, monstres et lanceurs de sort. Tous les éléments liés à un même socle sont considérés comme étant une même figurine, comme par exemple un dragon et son cavalier, ou un canon et ses trois servants.

\subsection{Socles et contact socle à socle}

Toutes les figurines sont fixées sur des socles. L'envergure d'une figurine est associée à son socle. Les socles sont normalement rectangulaires, et leurs mesures sont données en millimètres, dans l'ordre face avant puis côté, comme 25x50 {\milli\meter} pour une base de cavalerie ordinaire. Il existe quelques cas rares de socles ronds, par exemple pour les \emph{Machines de Guerre}. Dans ce cas, on donne le diamètre : 60 {\milli\meter} pour le socle de \emph{Machines de Guerre} ordinaires.

On considère que deux figurines sont en contact socle à socle si leurs socles sont en contact physique, contact coin à coin inclus.

\subsection{Figurines en plusieurs éléments}

Certaines figurines ont plusieurs profils de caractéristiques. C'est le cas pour un chevalier et sa monture. Chaque partie d'une figurine qui possède son propre profil de caractéristiques est appelé un élément de figurine. Par exemple, un Char est une figurine unique composée de 5 éléments : 2 destriers, 2 membres d'équipage et un char. Un soldat à pied classique est une figurine unique composée d'un seul élément. \nouveau{Chaque partie d'une telle figurine a son propre profil et constitue un "élément de figurine"}.

\section{Unités}

Toute figurine fait partie d'une unité. Les unités peuvent être un groupe de figurines déployées en rangs et colonnes ou une figurine seule. Quand une règle, une capacité, un sort, etc., affecte une unité, toutes les figurines de l'unité sont concernées. Quand elles forment une unité, toutes les figurines doivent être parfaitement alignées, en contact les unes des autres, et face à la même direction. Tous les rangs, à l'exception du dernier, doivent avoir la même largeur. Le dernier rang pouvant être plus court, on l'appelle quand c'est le cas un rang arrière incomplet. \nouveau{Un dernier rang peut être incomplet du moment que chaque figurine est alignée avec celles des autres rangs}. 

\subsection{Figurines ordinaires}

Les figurines génériques dans une unité sont appelées par la suite \emph{Figurines ordinaires}. Les \emph{Personnages} n'en sont pas.

\subsection{Rangs complets}

Un rang complet est un rang qui contient au moins 5 figurines. Certaines unités, comme l'\emph{Infanterie Monstrueuse}, requièrent moins de 5 figurines pour former un rang complet, voir la partie \ref{types_de_troupe} à la page \pageref{types_de_troupe}.

\subsection{Formation en horde}
\label{horde}

Les unités formées en rang de 10 figurines au moins sont considérées comme étant en formation de \emph{Horde}. Ce nombre est abaissé à 6 figurines pour des \emph{Rangs Monstrueux}.

\subsection{Empreinte au sol}

L'empreinte au sol d'une unité correspond à la surface occupée par l'ensemble des socles de toutes les figurines de l'unité.

\subsection{Centre de l'unité}
\label{centre_unite}

Le centre d'une unité est défini en dessinant un rectangle imaginaire le plus petit possible contenant son empreinte au sol. Le centre de l'unité est le centre de ce rectangle, au croisement des deux diagonales (voir les diagonales en rouge dans la figure \ref{figure/arcs}).

\subsection{Avant, arrière, flancs et arcs de vision}

Une unité possède quatre arcs de vision : l'avant, l'arrière et les deux flancs. Chaque arc est défini en tirant une ligne droite depuis chaque coin du rectangle imaginaire défini dans le paragraphe \ref{centre_unite} ci-dessus, avec un angle de 135{\text{\degree}} par rapport aux côtés de ce rectangle (voir la figure \ref{figure/arcs}). Les unités constituées de socles ronds n'ont pas d'arc de vision, elles ont un champ de vision à 360{\text{\degree}}.

\begin{figure}[!htbp]
\centering
\def\svgwidth{12cm}
\input{arcs.pdf_tex}
\caption{Cette unité possède 3 rangs et 5 colonnes. Le socle sur le côté droit est un \emph{Personnage} avec un socle incompatible avec ceux de l'unité qu'il a rejointe. Le dernier rang est incomplet, ne contenant que 3 figurines. L'empreinte au sol de l'unité est la zone verte. Le centre de l'unité est le centre du rectangle dessiné à partir des bords extérieurs de l'empreinte au sol, dont les diagonales sont en rouge. Les arcs avant, arrière et des flancs sont définis par un angle de 135{\text{\degree}} avec les côtés de ce rectangle.}
\label{figure/arcs}
\end{figure}
