
\part{Figurines, Unités et Formations}

\section{Figurines}

Les figurines représentent des guerriers, des monstres ou encore des lanceurs de sort. Tous les éléments liés à un même socle sont considérés comme étant une même figurine, comme par exemple un dragon et son cavalier, ou un canon et ses trois servants.

\subsection{Socles et contact socle à socle}

Toutes les figurines sont fixées sur des socles qui peuvent être carrés, rectangulaires ou ronds. L'envergure d'une figurine est associée à son socle. Les dimensions des socles sont données en millimètres, dans l'ordre face avant puis côté, comme par exemple \unit{25x50}{\milli\meter} pour un socle de cavalerie ordinaire. Dans le cas peu fréquent d'un socle rond, on donne une seule dimension, le diamètre : \unit{60}{\milli\meter} par exemple pour le socle d'une Machine de Guerre ordinaire.

Deux unités sont considérées comme étant en contact socle à socle si les rectangles correspondant aux bords extérieurs de leur Empreinte au Sol (appelés Rectangles Limites) se touchent l'un l'autre, y-compris par un contact coin à coin. Deux figurines sont considérées comme étant en contact socle à socle si leurs socles se touchent, même coin à coin. Quand des vides sont créés par des rangs incomplets ou des Personnages avec un Socle Incompatible, les figurines de part et d'autre de ces vides sont considérées comme étant en contact socle à socle à travers les vides.

\subsection{Figurines en plusieurs éléments}

Certaines figurines, appelées Figurines en Plusieurs Éléments, ont plusieurs Profils de Caractéristiques. Une figurine de cavalerie, par exemple, est composée de deux éléments (cavalier et monture), tandis qu'une figurine de Char peut posséder cinq éléments (deux chevaux, deux membres d'équipage et un châssis). Un soldat classique à pied est une figurine composée d'un seul élément. Chaque partie d'une telle figurine a son propre Profil de Caractéristiques et constitue un \og élément de figurine \fg{}. Quand un sort, une règle, une capacité ou équivalent affecte une figurine, tous les éléments de la figurine sont affectés à moins que la règle ne précise qu'elle n'affecte qu'une élément de figurine.

\section{Unités}

Toute figurine fait partie d'une unité. Une unité peut être un groupe de figurines déployées en rangs et en colonnes ou une figurine seule. Quand une règle, une capacité, un sort, etc. affecte une unité, toutes les figurines de l'unité sont concernées. Quand elles forment une unité, toutes les figurines doivent être parfaitement alignées, en contact les unes des autres, et face à la même direction. Tous les rangs, à l'exception du dernier, doivent avoir la même largeur. Le dernier rang pouvant être plus court, on l'appelle quand c'est le cas un rang arrière incomplet. Il peut tout à fait y avoir des vides dans ce rang arrière, du moment que chaque figurine est alignée avec celle de devant.

\subsection{Figurines ordinaires}

Les figurines génériques dans une unité sont appelées par la suite Figurines Ordinaires. Seuls les Personnages ne sont pas des figurines ordinaires.

\subsection{Rangs Complets}

Un Rang Complet est un rang qui contient au moins 5 figurines. Certaines unités, comme celles d'Infanterie Monstrueuse, requièrent moins de 5 figurines pour former un Rang Complet. Voir le chapitre \ref{troop_types} à la page \pageref{troop_types} pour plus de détails.

\subsection{Formation de Horde}
\label{horde}

Les unités formées en rangs de 10 figurines au moins sont considérées comme étant en formation de Horde. Ce nombre est abaissé à 6 figurines dans le cas de Rangs Monstrueux.

\subsection{Empreinte au Sol}

L'Empreinte au Sol d'une unité correspond à la surface occupée par l'ensemble des socles de toutes les figurines de l'unité.

\subsection{Centre de l'unité}
\label{centre_unite}

Le centre d'une unité est défini en dessinant un rectangle imaginaire le plus petit possible contenant son Empreinte au Sol. Le centre de l'unité est le centre de ce rectangle, au croisement des deux diagonales (voir les diagonales en rouge dans la figure \ref{figure/arcs}).

\subsection{Avant, arrière, flancs et arcs de vision}

Une unité possède quatre arcs de vision : frontal, arrière et deux latéraux. Chaque arc est défini en tirant une ligne droite depuis chaque coin du rectangle imaginaire défini dans le paragraphe \ref{centre_unite} ci-dessus, avec un angle de 135{\text{\degree}} par rapport aux côtés de ce rectangle (voir la figure \ref{figure/arcs}). Les unités constituées de socles ronds n'ont pas d'arc de vision, des lignes de vue peuvent être tracées depuis n'importe quel point de leur socle et dans n'importe quelle direction.

\newcommand{\frontarc}{Arc frontal}
\newcommand{\leftsidearc}{Arc latéral}
\newcommand{\rightsidearc}{Arc latéral}
\newcommand{\reararc}{Arc arrière}
\newcommand{\centreofunit}{\normalfontsize Centre de l'unité}
\newcommand{\firstangle}{90 \text{\degree}}
\newcommand{\secondangle}{135 \text{\degree}}
\newcommand{\FIGfootprint}{Empreinte au Sol}
\newcommand{\FIGboundaryrectangle}{Rectangle Limite}

\begin{figure}[!htbp]
\centering
\def\svgwidth{14cm}
\input{pics/arcs.pdf_tex}
\caption{Cette unité possède 3 rangs et 5 colonnes. Le socle sur le côté droit est un Personnage dont le socle est incompatible avec ceux de l'unité qu'il a rejointe. Le dernier rang est incomplet, ne contenant que 3 figurines. L'Empreinte au Sol de l'unité est la zone verte. Le centre de l'unité est le centre du rectangle dessiné à partir des bords extérieurs de l'Empreinte au Sol, dont les diagonales sont en rouge. Les arcs avant, arrière et des flancs sont définis par un angle de 135{\text{\degree}} avec les côtés de ce rectangle.\vspace*{10pt}\newline
L'Empreinte au Sol est la zone verte couverte par les socles des figurines, tandis que le Rectangle Limite est la zone grise rectangulaire tracée à partir des limites extérieures de l'Empreinte au Sol.
}
\label{figure/arcs}
\end{figure}
