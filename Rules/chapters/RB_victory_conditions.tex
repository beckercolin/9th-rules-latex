
\hypertarget{victorypoints}{\part{Conditions de Victoire}}
\label{scoring_and_victory_conditions}

\section{Gagner des Points de Victoire}

À la fin de la partie, faites le total de vos points de victoire en suivant ces quelques règles :

\noindent\begin{tabular}{>{\bfseries\raggedleft}p{2.2cm}p{13.5cm}}
Morts ou Déserteurs & Pour chaque unité ennemie qui a été détruite ou qui a fui en dehors du champ de bataille, vous gagnez \textbf{autant de Points de Victoire que sa valeur en points}. \tabularnewline
Effrayés & Pour chaque unité ennemie en fuite sur le champ de bataille, vous gagnez \textbf{autant de Points de Victoire que la moitié de sa valeur en points} (arrondie au supérieur). \tabularnewline
Décimés & Pour chaque unité réduite à 25 \% ou moins de son effectif de PVs de départ, vous gagnez \textbf{autant de Points de Victoire que la moitié de sa valeur en points} (arrondie au supérieur). Les Personnages sont comptés séparément des unités qu'ils ont rejointes.

Notez que si une unité ennemie est à la fois Effrayée et Décimée, vous gagnez autant de Points de Victoire que sa valeur en points. \tabularnewline
Leur Roi est Mort & Si le Général ennemi a été tué ou a fui le champ de bataille, vous gagnez \textbf{200 Points de Victoire}. \tabularnewline
Leur Bannière est à Terre & Si le Porteur de la Grande Bannière ennemi a été tué au Corps à Corps ou a perdu un combat et a raté son test de Moral, vous gagnez \textbf{200 Points de Victoire}. \tabularnewline
\end{tabular}

\hypertarget{secondaryobjectives}{\subsection{Objectifs Secondaires}}
\label{secondary_objectives}

L'Objectif Secondaire choisi en début de partie peut faire gagner des Points de Bataille supplémentaires (voir plus bas le paragraphe \ref{who_is_the_winner}).

\noindent\begin{tabular}{>{\bfseries\raggedleft}p{2.2cm}p{13.5cm}}
Tenez la Ligne & Le joueur avec le plus d'\scoringunits{} à moins de \distance{6} du centre du champ de bataille à la fin de la partie gagne cet Objectif Secondaire. \tabularnewline
Percée & Le joueur qui a le plus d'\scoringunits{} dans la zone de déploiement de son adversaire à la fin de la partie remporte cet Objectif Secondaire. \tabularnewline
Capturez les Étendards & Le joueur qui possède le plus grand nombre d'\scoringunits{} en vie, parmi ceux qui avaient été désignés au début de la partie, remporte cet Objectif Secondaire. \tabularnewline
Sécurisez la Cible & À la fin de la partie, le joueur qui contrôle le plus de marqueurs remporte cet Objectif Secondaire. Pour contrôler un marqueur, il faut posséder plus d'\scoringunits{} que son adversaire à moins de \distance{6} du marqueur. Si une même unité est à moins de \distance{6} des deux marqueurs, elle ne compte que pour le marqueur le plus proche de son centre. Déterminez-le aléatoirement si les deux marqueurs sont à égale distance. \tabularnewline
\end{tabular}

\subsection{\scoringunits}

Une unité dont au moins une figurine possède la règle \scoring{} est considérée comme une \scoringunit{} est sert à capturer les Objectifs Secondaires. Rappelons qu'une unité peut perdre la règle \scoring{} si elle est en fuite, si elle vient de faire une Reformation Post-Combat ou si elle a utilisé sa règle \ambush{} et est entrée sur le champ de bataille au Tour 4 ou plus tard. Voir le chapitre \ref{special_rules} des règles spéciales à la page \pageref{special_rules} pour plus de détails.


\section{Qui a gagné ?}
\label{who_is_the_winner}

Une fois que tous les Points de Victoire ont été comptabilisés, un total de 20 Points de Bataille est divisé entre les joueurs, en fonction de l'écart entre leurs Points de Victoire. Calculez cette différence et référez-vous à la table ci-dessous pour la convertir en Points de Bataille.

Si un Objectif Secondaire a été remporté par un joueur, il gagne 3 Points de Bataille supplémentaires, tandis que son adversaire en perd 3. Si plus d'un Objectif Secondaire a été utilisé, regardez plutôt le joueur qui en a gagné le plus.

\hypertarget{victorypointstable}{\subsection{Table des Points de Bataille}}
\label{victory_points_table}

\begin{center}
\noindent\begin{tabular}{M{4.3cm}M{3cm}M{2cm}M{2cm}}
\hline
\multicolumn{2}{c}{\textbf{Différence de Points de Victoire}} & \multicolumn{2}{c}{\textbf{Points de Bataille}} \tabularnewline
\textbf{Pourcentage du Coût Total de l'Armée} & (à 4500 pts) & \textbf{Gagnant} & \textbf{Perdant} \tabularnewline
0 - 5\% & 0 - 225 & 10 & 10 \tabularnewline
>5 - 10\% & 226 - 450 & 11 & 9 \tabularnewline
>10 - 20\% & 451 - 900 & 12 & 8 \tabularnewline
>20 - 30\% & 901 - 1350 & 13 & 7 \tabularnewline
>30 - 40\% & 1351 - 1800 & 14 & 6 \tabularnewline
>40 - 50\% & 1801 - 2250 & 15 & 5 \tabularnewline
>50 - 70\% & 2251 - 3150 & 16 & 4 \tabularnewline
>70\% & >3150 & 17 & 3 \tabularnewline
\multicolumn{2}{c}{Remporter un Objectif Secondaire} & +3 & -3 \tabularnewline
\hline
\end{tabular}
\end{center}

\subsection{Règles optionnelles simplifiées pour déterminer le vainqueur}

Remporter un Objectif Secondaire apporte un nombre de Points de Victoire égal à 20\% de la taille de la partie. Une fois que tous les points de victoire ont été comptabilisés, comparez les résultats.
\begin{itemize}[label={\textbullet}]
\item Si la différence entre les totaux est de moins de 10\% de la taille de la partie, il s'agit d'une \textbf{Égalité}.
\item Si la différence entre les totaux est entre 10 et 50\%, il s'agit d'une \textbf{Victoire} pour le joueur qui a le plus de Points de Victoire.
\item Si la différence entre les totaux est de plus de 50\%, il s'agit d'un \textbf{Massacre}.
\end{itemize}
