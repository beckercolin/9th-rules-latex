
% Magical Weapons

\newcommand{\giantsword}{Épée de Géant}
\newcommand{\bladeofstrife}{Lame de Conflit}
\newcommand{\ogresword}{Épée Ogre}
\newcommand{\skullsplitter}{Trancheuse de Crâne}
\newcommand{\obsidiansword}{Épée d'Obsidienne}
\newcommand{\fencersswords}{Épées d'Escrimeur}
\newcommand{\kingslayer}{Tueuse de Rois}
\newcommand{\axeofbattle}{Hache de Bataille}
\newcommand{\beastbanehalberd}{Hallebarde Fléau-des-Bêtes}
\newcommand{\swordofhaste}{Épée de Célérité}
\newcommand{\herossword}{Épée des Héros}
\newcommand{\fleshrender}{L'Arracheur de Chair}
\newcommand{\blessedsword}{Épée Sacrée}
\newcommand{\swordofstrength}{Épée de Force}
\newcommand{\swordofskill}{Lame d'Habileté}
\newcommand{\flaminglance}{Lance Enflammée}
\newcommand{\screamingswords}{Épées Hurlantes}
\newcommand{\razorblade}{Épée Tranchante}

% Magical Armour

\newcommand{\forceshield}{Bouclier à Champ de Force}
\newcommand{\armourofdestiny}{Armure du Destin}
\newcommand{\bluffershelm}{Heaume de Tromperie}
\newcommand{\mithrilmail}{Cotte de Maille en Mithril}
\newcommand{\glitteringcuirass}{Cuirasse Étincelante}
\newcommand{\armouroffortune}{Armure de Bonne Fortune}
\newcommand{\bronzebreastplate}{Plastron de Bronze}
\newcommand{\dragonmantle}{Cape en Cuir de Dragon}
\newcommand{\gamblersarmour}{Armure du Parieur}
\newcommand{\dragonscalehelm}{Heaume en Écailles de Dragon}
\newcommand{\hardenedshield}{Bouclier Renforcé}
\newcommand{\luckyshield}{Bouclier de Chance}

% Talismans

\newcommand{\talismanofsupremeshielding}{Talisman de Protection Suprême}
\newcommand{\sproutofrebirth}{Graine de Renaissance}
\newcommand{\obsidiannullstone}{Pierre d'Annulation d'Obsidienne}
\newcommand{\duskstone}{Pierre du Crépuscule}
\newcommand{\obsidianrock}{Gemme d'Obsidienne}
\newcommand{\talismanofgreatershielding}{Talisman de Protection Majeure}
\newcommand{\gemstoneamulet}{Amulette de Pierres Précieuses}
\newcommand{\obsidianpebble}{Éclat d'Obsidienne}
\newcommand{\luckycharm}{Charme de Chance}
\newcommand{\dragonfiregem}{Gemme de Feu de Dragon}
\newcommand{\talismanofshielding}{Talisman de Protection}

% Enchanted Items

\newcommand{\flyingcarpet}{Tapis Volant}
\newcommand{\wizardshood}{Capuche de Magicien}
\newcommand{\crownofscorn}{Couronne de Moquerie}
\newcommand{\crystalball}{Boule de Cristal}
\newcommand{\sceptreofdominion}{Sceptre de Domination}
\newcommand{\gemoffortune}{Gemme de Chance}
\newcommand{\ringoffire}{Anneau de Feu}
\newcommand{\potionofstrength}{Potion de Force}
\newcommand{\charmofcursediron}{Icône de Fer Maudit}
\newcommand{\divineicon}{Icône Divine}
\newcommand{\potionofswiftness}{Potion de Rapidité}

% Arcane Items

\newcommand{\backlashscroll}{Parchemin de Retour de Flamme}
\newcommand{\bookofarcanepower}{Grimoire de Puissance Cabalistique}
\newcommand{\essenceofafreemind}{Essence de Libre Pensée}
\newcommand{\dispelscroll}{Parchemin de Dissipation}
\newcommand{\tomeofarcanelore}{Grimoire de Connaissance Mystique}
\newcommand{\groundingrod}{Baguette Tellurique}
\newcommand{\sceptreofpower}{Sceptre de Pouvoir}
\newcommand{\bindingscroll}{Parchemin d'Entrave}
\newcommand{\shieldingscroll}{Parchemin de Protection}
\newcommand{\wandofstability}{Baguette de Stabilité}

% Magical Standards

\newcommand{\rendingbanner}{Bannière de Rasoirs}
\newcommand{\stalkersstandard}{Étendard du Pisteur}
\newcommand{\aethericon}{Icône d'Æther}
\newcommand{\holyicon}{Icône Sacrée}
\newcommand{\bannerofdiscipline}{Bannière de Discipline}
\newcommand{\bannerofspeed}{Bannière de Vitesse}
\newcommand{\flamingstandard}{Étendard Flamboyant}
\newcommand{\warstandard}{Étendard de Guerre}
\newcommand{\iconoftherelentlesscompany}{Icône de la Compagnie Implacable}
\newcommand{\gleamingicon}{Icône Étincelante}
\newcommand{\bannerofcourage}{Bannière de Courage}



\part{Objets Magiques}

\section{Catégories d'Objets Magiques}

Les Objets Magiques sont répartis en 6 catégories :
\begin{itemize}[label={\textbullet}]
\item Armes Magiques
\item Armures Magiques
\item Talismans
\item Objets Enchantés
\item Objets Cabalistiques
\item Bannières Magiques
\end{itemize}

De plus, chaque type d'Objet Magique possède les règles spécifiques suivantes.

\paragraph{Armes Magiques}

Une figurine équipée d'une Arme Magique (même une \hw{} Magique) doit l'utiliser, y-compris si elle possède plusieurs autres armes. Une \hw{} Magique ne peut pas être utilisée pour la règle Parade. Si une Arme Magique est détruite, elle devient son équivalent standard (si elle en a un).

\paragraph{Armures Magiques}

La plupart des Armures Magiques ont un type qui correspond à leur équivalent standard. Elles suivent les règles associées à cet équivalent standard en plus des règles précisées dans leur description. Si une Armure Magique est détruite, elle devient son équivalent standard (si elle en a un). Si une figurine possède une armure standard du même type que son Armure Magique, cette dernière la remplace. Par exemple, si une figurine possède une \ha{} et qu'elle achète une \la{} Magique, l'\ha{} standard est perdue. Un Bouclier Magique ne peut pas être utilisé avec une \hw{} pour la règle Parade.

Les Sorciers ne peuvent pas prendre d'Armure Magique, à moins qu'ils ne possèdent déjà ou aient la possibilité d'acheter une armure standard autre qu'un \barding{}, une \innatedefence{} ou une \mountsprotection{}. Ignorez les armures des autres cavaliers ou membres d'équipage montés sur la même monture que le Sorcier.

\paragraph{Talismans et Objets Enchantés}

Pas de règle additionnelle.

\paragraph{Objets Cabalistiques}

Seuls les Sorciers peuvent porter des Objets Cabalistiques.

\paragraph{Bannières Magiques}

Seuls les Porte-Étendards ont accès aux Bannières Magiques : a priori le Porte-Étendard d'une unité ou le Porteur de la Grande Bannière.

\subsection{Types d'Objets Magiques}

Chaque Objet Magique a un type qui correspond à celui de son équivalent standard. Il suit donc toutes ses règles et restrictions. Ainsi, une \lance{} Magique donne un bonus de +2 en Force en charge et ne peut être portée que par une figurine montée, une Bête de Guerre ou une Bête Monstrueuse, comme une \lance{} standard. Si un Objet Magique est détruit, il devient son équivalent standard (s'il existe).

\subsection{Restrictions}

\begin{itemize}[label={\textbullet}]
\item Tous les Objets Magiques ont la règle \oneperarmy{} : ils ne peuvent pas être dupliqués.
\item Une figurine ne peut pas posséder deux Objets Magiques de la même catégorie.
\item Un Objet Magique peut avoir des restrictions supplémentaires précisées dans sa description.
\end{itemize}

\subsection{Coût en Points}

La plupart des Objets Magiques ont un Coût en Points précisé après leur nom. \newfromWHB{Certains Objets Magiques ont deux coûts en points : par exemple, \textbf{\giantsword} ... \pts{60/50}. La première valeur s'applique pour un Seigneur, la seconde pour un Héros ou un Champion.}

\subsection{Qui est affecté ?}

Les Objets Magiques font souvent référence à leur \og Porteur \fg{} dans leurs règles. Ce terme correspond à l'élément de figurine pour lequel l'objet a été acheté, \textbf{sans jamais inclure la monture}. D'autres Objets magiques font référence à la \og Figurine \fg{} ou \newfromWHB{la \og Figurine du Porteur \fg{}. Ces termes correspondent à l'intégralité de la figurine, monture comprise. Remarquez que cela prend le pas sur les restrictions du Profil de Monstre Monté (page \pageref{ridden_monster_profile}).} Enfin, une troisième catégorie d'objets comprend dans ses règles le mot-clé \og Unité du Porteur \fg{}. Ces objets affectent toutes les figurines de l'unité, porteur et montures inclus.

\subsection{Usage Unique}

Les effets ne peuvent être appliqués qu'une seule fois par partie.

\section{Liste des Objets Magiques Communs}

Les Objets Magiques listés dans les paragraphes suivants sont considérés comme des Objets Magiques Communs et sont disponibles à l'achat pour toutes les figurines ou unités qui ont accès aux Objets Magiques. Il existe aussi souvent des Objets Magiques spécifiques à chaque armée.

\newpage
\hypertarget{magicalweapons}{\subsection{Armes Magiques}}
\label{magical_weapons}

\startpricelist

\pricelistitem{\giantsword}{60/50}Type : \hw{}.  Les attaques portées avec cette arme ont +3 en Force.

\pricelistitem{\bladeofstrife}{45}Type : \hw{}. Le porteur gagne +3 Attaques quand il combat avec cette arme.

\pricelistitem{\ogresword}{40}Type : \hw{}. Les attaques portées avec cette arme ont +2 en Force.

\pricelistitem{\skullsplitter}{40}Type : Arme de Tir. \range{24}, \Strength{} 4, \armourpiercing{1}, \multipleshots{4}, touche toujours sur 4+.

\pricelistitem{\fencersswords}{35}Figurine à pied uniquement.

Type : \pw{}. Le porteur obtient une Capacité de Combat de 10 quand il combat avec cette arme.

\pricelistitem{\obsidiansword}{35}Type : \hw{}. Les attaques portées avec cette arme ont la règle \armourpiercing{6}.

\pricelistitem{\kingslayer}{30}Type : \hw{}. Le porteur gagne +1 en Force et +1 Attaque quand il combat avec cette arme pour chaque Personnage ennemi en contact socle à socle avec l'unité du porteur. Ce bonus est calculé et prend effet au palier d'Initiative auquel sont faites les attaques.

\pricelistitem{\herossword}{30/20}Personnage uniquement.

Type : \hw{}. Les Attaques portées avec cette arme ont +1 en Force. Le porteur gagne +1 Attaque quand il combat avec cette arme. Quand il attaque avec cette arme, le porteur ne peut excéder 4 Attaques et une Force de 5, quels que soient les modificateurs.

\pricelistitem{\swordofhaste}{25}Type : \hw{}. Le porteur frappe à Initiative 10 quand il combat avec cette arme. Les attaques portées avec cette arme blessent toujours sur 4+ (ou mieux).

\columnbreak

\pricelistitem{\axeofbattle}{25}Type : \hw{}. Le porteur gagne +2 Attaques quand il combat avec cette arme.

\pricelistitem{\beastbanehalberd}{25}Type : \halberd{}. Les attaques portées avec cette arme ont toujours Force 5 (quels que soient les modificateurs) et la règle \multiplewounds{2}{\chariot{}, \monstrousinfantry{}, \monstrousbeast{}, \monstrouscavalry{}, \monster{}, \riddenmonster{}}.

\pricelistitem{\fleshrender}{20}Type : \gw{}. Les attaques portées avec cette arme ont la règle \armourpiercing{1}.

\pricelistitem{\blessedsword}{20}Type : \hw{}. Les attaques portées avec cette arme ont la règle \divineattacks{} et leurs jets pour blesser ratés peuvent être relancés.

\pricelistitem{\swordofstrength}{15}Type : \hw{}. Les attaques portées avec cette arme ont +1 en Force.

\pricelistitem{\swordofskill}{15}Type : \hw{}. Les attaques portées avec cette arme ont un bonus de +1 pour toucher.

\pricelistitem{\screamingswords}{10}Type : \pw{}. Le porteur gagne la règle \fear{}.

\pricelistitem{\flaminglance}{10}Type : \lance{}. Les attaques portées avec cette arme ont la règle \flamingattacks{}.

\pricelistitem{\razorblade}{5}Type : \hw{}. Les attaques portées avec cette arme ont la règle \armourpiercing{1}.

\endpricelist

\newpage
\hypertarget{magicalarmour}{\subsection{Armures Magiques}}
\label{magical_armour}

\startpricelist

\pricelistitem{\forceshield}{70}Type : \shield{}. La figurine du porteur gagne une \wardsave{5} contre les Attaques à Distance.

\pricelistitem{\armourofdestiny}{50}Type : \ha{}. Le porteur gagne une \wardsave{4}.

\pricelistitem{\bluffershelm}{35}Ne peut pas être pris par une figurine avec la règle \largetarget{}.

Type : Aucun (Sauvegarde d'Armure 6+). Les jets pour blesser réussis contre le porteur doivent être relancés.

\pricelistitem{\mithrilmail}{35/25}Figurine à pied uniquement.

Type : \ha{} (Sauvegarde d'Armure 2+). Cette Sauvegarde d'Armure ne peut être en aucun cas améliorée.

\pricelistitem{\glitteringcuirass}{30}Type : \ha{}. Le porteur gagne la règle \distracting{}.

\pricelistitem{\armouroffortune}{25}Type : \ha{}. Le porteur gagne une \wardsave{5}.

\pricelistitem{\dragonmantle}{25}Type : Aucun. Le porteur gagne la règle \innatedefence{5}.

\columnbreak

\pricelistitem{\bronzebreastplate}{25}Type : \ha{}. Usage Unique. Vous pouvez activer l'objet lorsque la figurine du porteur subit une touche. Pour la durée de la Phase, la figurine du porteur gagne alors une Sauvegarde d'Armure de 1+. Si la figurine du porteur est une \largetarget{}, cette Sauvegarde d'Armure est limitée à 2+.

\pricelistitem{\gamblersarmour}{15}Type : \ha{}. Le porteur gagne une \wardsave{6}.

\pricelistitem{\dragonscalehelm}{10}Type : Aucun (Sauvegarde d'Armure 6+). Le porteur gagne la règle \fireborn{}.

\pricelistitem{\luckyshield}{5}Type : \shield{}. Usage Unique. Ignorez la première touche subie par la figurine du porteur. Si plusieurs attaques se disputent le statut de première touche, le porteur peut choisir la touche à ignorer.

\pricelistitem{\hardenedshield}{5}Type : \shield{}. Ajoutez un bonus additionnel de +1 à la Sauvegarde d'Armure du porteur quand il utilise ce bouclier, pour un total de +2.

\endpricelist

\newpage
\hypertarget{talismans}{\subsection{Talismans}}
\label{talismans}

\startpricelist

\pricelistitem{\sproutofrebirth}{50}Le porteur gagne une \regeneration{4}.

\pricelistitem{\talismanofsupremeshielding}{50}Le porteur gagne une \wardsave{4}.

\pricelistitem{\obsidiannullstone}{45}Le porteur gagne la règle \magicresistance{3}.

\pricelistitem{\duskstone}{30}Le porteur peut relancer ses jets de Sauvegarde d'Armure ratés.

\pricelistitem{\obsidianrock}{30}Le porteur gagne la règle \magicresistance{2}.

\pricelistitem{\talismanofgreatershielding}{25}Le porteur gagne une \wardsave{5}.

\pricelistitem{\gemstoneamulet}{15}Usage Unique. Peut être activée lorsque la figurine du porteur rate un jet de Sauvegarde d'Armure ou n'a pas l'occasion d'utiliser sa Sauvegarde d'Armure suite à une blessure. La figurine du porteur gagne alors une \wardsave{4} contre cette blessure.

\pricelistitem{\obsidianpebble}{10}Le porteur gagne la règle \magicresistance{1}.

\pricelistitem{\luckycharm}{5}Usage Unique. Peut être activé quand la figurine du porteur rate un jet de Sauvegarde d'Armure. Ce jet peut alors être relancé.

\pricelistitem{\dragonfiregem}{5}Le porteur gagne la règle \fireborn{}.

\pricelistitem{\talismanofshielding}{5}Le porteur gagne une \wardsave{6}.

\endpricelist

\newpage
\hypertarget{enchanteditems}{\subsection{Objets Enchantés}}
\label{enchanted_items}

\startpricelist

\pricelistitem{\wizardshood}{40}Le porteur gagne la règle \stupidity{}. Si le porteur ne possédait pas de Niveau de Magie, il devient un Sorcier Apprenti de Niveau 2. Au lieu de choisir une Voie de Magie normalement, le porteur utilise une Voie de Magie Commune déterminée aléatoirement au début de la partie. Si la \wizardshood{} est détruite, le porteur cesse immédiatement d'être un Sorcier.

\pricelistitem{\flyingcarpet}{40/30}Figurine à pied uniquement.

Le porteur gagne la règle \fly{6}.

\pricelistitem{\crystalball}{35}Le porteur gagne la règle \lightningreflexes{}.

\pricelistitem{\crownofscorn}{35}Usage Unique. Vous pouvez utiliser cet objet au lieu de faire une tentative de dissipation. Le sort est alors automatiquement dissipé. Vous ne pouvez prendre cet objet que si votre armée ne comprend aucun Sorcier.

\pricelistitem{\sceptreofdominion}{30}Usage Unique. Peut être activé au début de n'importe quelle Manche de Corps à Corps. Le porteur gagne la règle \stubborn{} pour la durée de la phase.

\pricelistitem{\ringoffire}{25}\boundspell{3} : \firesignature{} (\Pathof{} \fire{}).

\columnbreak

\pricelistitem{\gemoffortune}{25}Les jets pour blesser réussis d'Attaques à Distance dirigées contre l'unité du porteur ayant donné \result{6} doivent être relancés, à moins que la touche n'ait été distribuée sur une \largetarget{}.

\pricelistitem{\charmofcursediron}{20}L'unité du porteur gagne une \wardsave{5} contre les blessures causées par des Armes d'Artillerie.

\pricelistitem{\potionofstrength}{20}Usage Unique. Peut être activée au début de n'importe quelle Phase ou Manche de Corps à Corps. Le porteur gagne +2 en Force pour la durée du Tour de Joueur.

\pricelistitem{\divineicon}{15}La figurine du porteur gagne la règle \divineattacks{}.

\pricelistitem{\potionofswiftness}{5}Usage Unique. Peut être activée au début de n'importe quelle Phase ou Manche de Corps à Corps. Le porteur gagne +3 en Initiative pour la durée du Tour de Joueur.

\endpricelist

\newpage
\hypertarget{arcaneitems}{\subsection{Objets Cabalistiques}}
\label{arcane_items}

\startpricelist

\pricelistitem{\backlashscroll}{55}Usage Unique. Vous pouvez utiliser ce parchemin au lieu de faire une tentative de dissipation. Une fois que le sort et l'Attribut de Voie ont été résolus, le lanceur doit faire un jet sur la Table des Fiascos. Cet objet n'a pas d'effet sur les \boundspells{}, les sorts lancés avec un seul Dé de Pouvoir et les sorts lancés avec un \overwhelmingpower{}.

\pricelistitem{\bookofarcanepower}{50/35}Le porteur gagne un modificateur de +1 pour lancer et dissiper les sorts.

\pricelistitem{\essenceofafreemind}{40/30}Le Sorcier peut inscrire deux Voies de Magie sur la Liste d'Armée plutôt qu'une, parmi les Voies qui lui sont normalement accessibles. Choisissez entre ces deux Voies au moment de générer les sorts.

\pricelistitem{\dispelscroll}{35}Usage Unique. Vous pouvez utiliser ce parchemin au lieu de faire une tentative de dissipation. Le sort est automatiquement dissipé.

\pricelistitem{\groundingrod}{25/15}Usage Unique. Le porteur peut utiliser cet objet pour relancer son jet sur la Table des Fiascos.

\pricelistitem{\tomeofarcanelore}{25/15}Le porteur génère un sort supplémentaire.

\pricelistitem{\bindingscroll}{20}Usage Unique. Vous pouvez utiliser ce parchemin au lieu de faire une tentative de dissipation. Le lanceur ne peut plus lancer ce même sort au prochain Tour de Jeu.

\pricelistitem{\sceptreofpower}{20/15}Usage Unique. Le porteur peut relancer un unique Dé de Pouvoir d'un jet de lancement de sort.

\pricelistitem{\wandofstability}{15}Usage Unique. Le porteur peut ajouter un modificateur de +1D6 à un de ses jets de dissipation (ce n'est pas un Dé de Dissipation) et ignorer la règle Pas Assez de Puissance. C'est une exception à la règle des Modificateurs Magiques.

\pricelistitem{\shieldingscroll}{15}Usage Unique. Vous pouvez utiliser ce parchemin au lieu de faire une tentative de dissipation. Toutes les figurines affectées gagnent une \wardsave{4} contre les effets de ce sort.

\endpricelist

\newpage
\hypertarget{magicalstandards}{\subsection{Bannières Magiques}}
\label{magical_standards}

\startpricelist

\pricelistitem{\rendingbanner}{45}L'unité du porteur gagne la règle \armourpiercing{1}.

\pricelistitem{\stalkersstandard}{40}L'unité du porteur gagne les règles \swiftstride{} et \strider{}.

\pricelistitem{\aethericon}{30}L'unité du porteur peut faire des tentatives de dissipation comment si elle était un Maître Sorcier.

\pricelistitem{\holyicon}{30}L'unité du porteur gagne la règle \divineattacks{} mais doit aussi relancer ses propres jets de \wardsave{} réussis.

\pricelistitem{\bannerofdiscipline}{25}L'unité du porteur gagne +1 en Commandement.

\pricelistitem{\bannerofspeed}{25}L'unité du porteur gagne +1 en Mouvement.

\pricelistitem{\flamingstandard}{20}Au début de chaque Manche de Corps à Corps, ou avant de tirer avec l'unité du porteur, la bannière peut être activée. Dans ce cas, toutes les Attaques de Tir et de Corps à Corps non spéciales de l'unité gagnent la règle \flamingattacks{} jusqu'à la fin de la phase.

\columnbreak

\pricelistitem{\warstandard}{15}L'unité du porteur ajoute un bonus de Résultat de Combat de +1 dans n'importe quel Corps à Corps dans lequel elle est impliquée.

\pricelistitem{\iconoftherelentlesscompany}{15}Figurines d'Infanterie uniquement.

Usage Unique. Vous pouvez l'activer au début de n'importe laquelle de vos étapes des Autres Mouvements. L'unité du porteur peut alors tripler son Mouvement en Marche Forcée plutôt que le doubler pour ce tour. Le déplacement ne peut alors pas dépasser \distance{15}. Cet effet ne peut pas être activé au premier Tour de Jeu si l'unité a utilisé la règle \vanguard{} ou \scout{}.

\pricelistitem{\bannerofcourage}{5}L'unité du porteur réussit automatiquement ses tests de \terror{} et est immunisée aux effets de la règle \fear{}.

\pricelistitem{\gleamingicon}{5}Usage Unique. Peut être activée la première fois que l'unité du porteur rate un test de Commandement. L'unité peut alors relancer les dés pour ce test raté.

\endpricelist
