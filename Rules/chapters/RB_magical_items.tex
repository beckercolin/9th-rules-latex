
% Magical Weapons

\newcommand{\giantsword}{Épée de Géant}
\newcommand{\ogresword}{Épée Ogre}
\newcommand{\skullsplitter}{Trancheuse de Crâne}
\newcommand{\fencersswords}{Épées d'Escrimeur}
\newcommand{\axeofbattle}{Hache de Bataille}
\newcommand{\beastbanehalberd}{Hallebarde Fléau-des-Bêtes}
\newcommand{\kingslayer}{Tueuse de Rois}
\newcommand{\obsidiansword}{Épée d'Obsidienne}
\newcommand{\blessedsword}{Épée Sacrée}
\newcommand{\fleshrender}{L'Arracheur de Chair}
\newcommand{\herossword}{Épée des Héros}
\newcommand{\swordofstrength}{Épée de Force}
\newcommand{\flaminglance}{Lance Enflammée}
\newcommand{\razorblade}{Épée Tranchante}

% Magical Armour

\newcommand{\armourofdestiny}{Armure du Destin}
\newcommand{\mithrilmail}{Cotte de Maille en Mithril}
\newcommand{\bluffershelm}{Heaume de Tromperie}
\newcommand{\glitteringcuirass}{Cuirasse Étincelante}
\newcommand{\armouroffortune}{Armure de Bonne Fortune}
\newcommand{\dragonmantle}{Cape en Cuir de Dragon}
\newcommand{\bronzebreastplate}{Plastron de Bronze}
\newcommand{\dragonscalehelm}{Heaume en Écailles de Dragon}
\newcommand{\luckyshield}{Bouclier de Chance}
\newcommand{\hardenedshield}{Bouclier Renforcé}

% Talismans

\newcommand{\talismanofsupremeshielding}{Talisman de Protection Suprême}
\newcommand{\obsidiannullstone}{Pierre d'Annulation d'Obsidienne}
\newcommand{\sproutofrebirth}{Graine de Renaissance}
\newcommand{\duskstone}{Pierre du Crépuscule}
\newcommand{\obsidianrock}{Gemme d'Obsidienne}
\newcommand{\talismanofgreatershielding}{Talisman de Protection Majeure}
\newcommand{\dragonfiregem}{Gemme de Feu de Dragon}
\newcommand{\luckycharm}{Charme de Chance}
\newcommand{\talismanofshielding}{Talisman de Protection}

% Enchanted Items

\newcommand{\wizardshood}{Capuche de Magicien}
\newcommand{\crownofautocracy}{Couronne d'Autocratie}
\newcommand{\potionofstrength}{Potion de Force}
\newcommand{\gemoffortune}{Gemme de Chance}
\newcommand{\ringoffire}{Anneau de Feu}
\newcommand{\divineicon}{Icône Divine}
\newcommand{\crownofscorn}{Couronne de Moquerie}
\newcommand{\charmofcursediron}{Icône de Fer Maudit}
\newcommand{\potionofswiftness}{Potion de Rapidité}

% Arcane Items

\newcommand{\bookofarcanepower}{Grimoire de Puissance Cabalistique}
\newcommand{\dispelscroll}{Parchemin de Dissipation}
\newcommand{\essenceofafreemind}{Essence de Libre Pensée}
\newcommand{\wandofstability}{Baguette de Stabilité}
\newcommand{\shieldingscroll}{Parchemin de Protection}
\newcommand{\sceptreofpower}{Sceptre de Pouvoir}

% Magical Standards

\newcommand{\rendingbanner}{Bannière de Rasoirs}
\newcommand{\stalkersstandard}{Étendard du Pisteur}
\newcommand{\bannerofdiscipline}{Bannière de Discipline}
\newcommand{\bannerofspeed}{Bannière de Vitesse}
\newcommand{\aethericon}{Icône d'Æther}
\newcommand{\flamingstandard}{Étendard Flamboyant}
\newcommand{\warstandard}{Étendard de Guerre}
\newcommand{\iconoftherelentlesscompany}{Icône de la Compagnie Implacable}
\newcommand{\gleamingicon}{Icône Étincelante}



\part{Objets Magiques}

\section{Catégories d'Objets Magiques}

Les Objets Magiques sont répartis en 6 catégories :
\begin{itemize}[label={\textbullet}]
\item Armes Magiques
\item Armures Magiques
\item Talismans
\item Objets Enchantés
\item Objets Cabalistiques
\item Bannières Magiques
\end{itemize}

De plus, chaque type d'Objet Magique possède les règles spécifiques suivantes.

\paragraph{Armes Magiques}

Une figurine équipée d'une Arme Magique (même une \hw{} Magique) doit l'utiliser, y-compris si elle possède plusieurs autres armes. Une \hw{} Magique ne peut pas être utilisée pour la règle Parade.

\paragraph{Armures Magiques}

La plupart des Armures Magiques ont un type qui correspond à leur équivalent standard. Elles suivent les règles associées à cet équivalent standard en plus des règles précisées dans leur description. Si une figurine possède une armure standard du même type que son Armure Magique, cette dernière la remplace. Par exemple, si une figurine possède une \ha{} et qu'elle achète une \la{} Magique, l'\ha{} standard est perdue. Un Bouclier Magique ne peut pas être utilisé avec une \hw{} pour la règle Parade.

Les \wizards{} ne peuvent pas prendre d'Armure Magique, à moins qu'ils ne possèdent déjà ou aient la possibilité d'acheter une armure standard autre qu'un \barding{}, une \innatedefence{} ou une \mountsprotection{}. Ignorez les armures des autres cavaliers ou membres d'équipage montés sur la même monture que le \wizard{}.

\paragraph{Talismans et Objets Enchantés}

Pas de règle additionnelle.

\paragraph{Objets Cabalistiques}

Seuls les \wizards{} peuvent porter des Objets Cabalistiques.

\paragraph{Bannières Magiques}

Seuls les Porte-Étendards ont accès aux Bannières Magiques : a priori le Porte-Étendard d'une unité ou le Porteur de la Grande Bannière.

\subsection{Types d'Objets Magiques}

Chaque Objet Magique a un type qui correspond à celui de son équivalent standard. Il suit donc toutes ses règles et restrictions. Ainsi, une \lance{} Magique donne un bonus de +2 en Force en charge et ne peut être portée que par une figurine montée, une Bête de Guerre ou une Bête Monstrueuse, comme une \lance{} standard. Si un Objet Magique est détruit, il devient son équivalent standard (s'il existe).

\subsection{Restrictions}

\begin{itemize}[label={\textbullet}]
\item Tous les Objets Magiques ont la règle \oneperarmy{} : ils ne peuvent pas être dupliqués.
\item Une figurine ne peut pas posséder deux Objets Magiques de la même catégorie.
\item Un Objet Magique peut avoir des restrictions supplémentaires précisées dans sa description.
\end{itemize}

\subsection{Coût en Points}

La plupart des Objets Magiques ont un Coût en Points précisé après leur nom.

\subsection{Qui est affecté ?}

Les Objets Magiques font souvent référence à leur \og Porteur \fg{} dans leurs règles. Ce terme correspond à l'élément de figurine pour lequel l'objet a été acheté, \textbf{sans jamais inclure la monture}. D'autres Objets magiques font référence à la \og Figurine \fg{} ou la \og Figurine du Porteur \fg{}. Ces termes correspondent à l'intégralité de la figurine, monture comprise. Remarquez que cela prend le pas sur les restrictions du Profil de Monstre Monté (page \pageref{ridden_monster_profile}). Enfin, une troisième catégorie d'objets comprend dans ses règles le mot-clé \og Unité du Porteur \fg{}. Ces objets affectent toutes les figurines de l'unité, porteur et montures inclus.

\subsection{Usage Unique}

Les effets ne peuvent être appliqués qu'une seule fois par partie.

\newpage

\section{Liste des Objets Magiques Communs}

Les Objets Magiques listés dans les paragraphes suivants sont considérés comme des Objets Magiques Communs et sont disponibles à l'achat pour toutes les figurines ou unités qui ont accès aux Objets Magiques. Il existe aussi souvent des Objets Magiques spécifiques à chaque armée.

\hypertarget{magicalweapons}{\subsection{Armes Magiques}}
\label{magical_weapons}

\startpricelist

\pricelistitem{\giantsword}{120}Type : \hw{}.  Les attaques portées avec cette arme ont +3 en Force.

\pricelistitem{\ogresword}{80}Type : \hw{}. Les attaques portées avec cette arme ont +2 en Force.

\pricelistitem{\skullsplitter}{80}Type : Arme de Tir. \range{24}, \Strength{} 4, \armourpiercing{1}, \multipleshots{4}. Les Attaques de Tir réalisées avec cette arme ignorent la Capacité de Tir et les modificateurs pour toucher, et touchent à la place toujours sur 4+.

\pricelistitem{\fencersswords}{60}Figurine à pied uniquement.

Type : \pw{}. Le porteur obtient une Capacité de Combat de 10.

\pricelistitem{\axeofbattle}{60}Type : \hw{}. Le porteur a 6 Attaques quand il utilise cette arme mais ne peut jamais blesser sur un meilleur résultat que 3+.

\pricelistitem{\beastbanehalberd}{60}Type : \halberd{}. Les attaques portées avec cette arme ont toujours Force 5 (quels que soient les modificateurs) et la règle \multiplewounds{2}{\chariot{}, \monstrousinfantry{}, \monstrousbeast{}, \monstrouscavalry{}, \monster{}, \riddenmonster{}}.

\pricelistitem{\kingslayer}{60}Type : \hw{}. Le porteur gagne +1 en Force et +1 Attaque quand il attaque avec cette arme pour chaque Personnage ennemi en contact socle à socle avec l'unité du porteur. Ce bonus est calculé et prend effet au palier d'Initiative auquel sont faites les attaques.

\columnbreak

\pricelistitem{\obsidiansword}{50}Type : \hw{}. Les attaques portées avec cette arme ont la règle \armourpiercing{6}.

\pricelistitem{\blessedsword}{50}Type : \hw{}. Les attaques portées avec cette arme ont la règle \divineattacks{} et leurs jets pour blesser ratés peuvent être relancés.

\pricelistitem{\fleshrender}{50}Type : \gw{}. Les attaques portées avec cette arme ont la règle \armourpiercing{1}.

\pricelistitem{\herossword}{40}Personnage uniquement.

Type : \hw{}. Les Attaques portées avec cette arme ont +1 en Force. Le porteur gagne +1 Attaque quand il combat avec cette arme. Quand il attaque avec cette arme, le porteur ne peut excéder 4 Attaques et une Force de 5, quels que soient les modificateurs.

\pricelistitem{\swordofstrength}{30}Type : \hw{}. Les attaques portées avec cette arme ont +1 en Force.

\pricelistitem{\flaminglance}{20}Type : \lance{}. Les attaques portées avec cette arme ont la règle \flamingattacks{}.

\pricelistitem{\razorblade}{10}Type : \hw{}. Les attaques portées avec cette arme ont la règle \armourpiercing{1}.

\endpricelist

\newpage
\hypertarget{magicalarmour}{\subsection{Armures Magiques}}
\label{magical_armour}

\startpricelist

\pricelistitem{\armourofdestiny}{90}\infantry{}, \warbeasts{} et \cavalry{} uniquement.

Type : \ha{}. Le porteur gagne une \wardsave{4}.

\pricelistitem{\mithrilmail}{70}\infantry{} uniquement.

Type : \ha{} (Sauvegarde d'Armure 2+). Cette Sauvegarde d'Armure ne peut être en aucun cas améliorée.

\pricelistitem{\bluffershelm}{70}Ne peut pas être pris par une figurine avec la règle \toweringpresence{}.

Type : Aucun (Sauvegarde d'Armure 6+). Les jets pour blesser réussis contre le porteur doivent être relancés.

\pricelistitem{\glitteringcuirass}{60}Type : \ha{}. Le porteur gagne la règle \distracting{}.

\pricelistitem{\armouroffortune}{50}Type : \ha{}. Le porteur gagne une \wardsave{5}.

\pricelistitem{\dragonmantle}{50}Type : Aucun. Le porteur gagne la règle \innatedefence{5}.

\columnbreak

\pricelistitem{\bronzebreastplate}{50}Type : \ha{}. Usage Unique. Vous pouvez activer l'objet lorsque la figurine du porteur subit une touche. Pour la durée de la Phase, la figurine du porteur gagne alors une Sauvegarde d'Armure de 1+. Si la figurine du porteur a la règle \toweringpresence{}, cette Sauvegarde d'Armure est limitée à 2+.

\pricelistitem{\dragonscalehelm}{30}Type : Aucun (Sauvegarde d'Armure 6+). Le porteur gagne la règle \fireborn{}.

\pricelistitem{\luckyshield}{10}Type : \shield{}. Usage Unique. Ignorez la première touche subie par la figurine du porteur. Si plusieurs attaques se disputent le statut de première touche, le porteur peut choisir la touche à ignorer.

\pricelistitem{\hardenedshield}{10}Type : \shield{}. Ajoutez un bonus additionnel de +1 à la Sauvegarde d'Armure du porteur quand il utilise ce bouclier, pour un total de +2. Le porteur subit un malus de -3 en Initiative, jusqu'à un minimum de 1, lorsqu'il porte des Attaques normales de Corps à Corps.

\endpricelist

\newpage
\hypertarget{talismans}{\subsection{Talismans}}
\label{talismans}

\startpricelist

\pricelistitem{\talismanofsupremeshielding}{100}Le porteur gagne une \wardsave{4}.

\pricelistitem{\obsidiannullstone}{90}Le porteur gagne la règle \magicresistance{3}.

\pricelistitem{\sproutofrebirth}{80}Le porteur gagne une \regeneration{4}.

\pricelistitem{\duskstone}{60}Le porteur peut relancer ses jets de Sauvegarde d'Armure ratés. S'il choisit de le faire, il ne peut pas utiliser de \wardsave{} ou de \regeneration{}.

\pricelistitem{\obsidianrock}{50}Le porteur gagne la règle \magicresistance{2}.

\columnbreak

\pricelistitem{\talismanofgreatershielding}{50}Le porteur gagne une \wardsave{5}.

\pricelistitem{\dragonfiregem}{20}Le porteur gagne la règle \fireborn{}.

\pricelistitem{\luckycharm}{10}Usage Unique. Peut être activé quand la figurine du porteur rate un jet de Sauvegarde d'Armure. Ce jet peut alors être relancé.

\pricelistitem{\talismanofshielding}{10}Le porteur gagne une \wardsave{6}.

\endpricelist

\newpage
\hypertarget{enchanteditems}{\subsection{Objets Enchantés}}
\label{enchanted_items}

\startpricelist

\pricelistitem{\wizardshood}{120}Figurines à pied uniquement.

Le porteur devient un \wizardapprentice{} avec deux \learnedspells{} (peu importe s'il était un \wizard{} avant). Au lieu de choisir une Voie de Magie normalement, le porteur utilise une Voie déterminée aléatoirement au début de la partie. Tirez au sort parmi toutes les Voies de Magie accessibles à n'importe quelle figurine du même Livre d'Armée, sans prendre en compte les \boundspells{}.

\pricelistitem{\crownofautocracy}{80}Le porteur gagne +1 en Commandement. Un \wizard{} ne peut pas utiliser ce bonus pour augmenter son Commandement au delà de 9.

\pricelistitem{\potionofstrength}{60}Usage Unique. Peut être activée au début de n'importe quelle Phase ou Manche de Corps à Corps. Le porteur gagne +2 en Force jusqu'à la fin du Tour de Joueur.

\pricelistitem{\gemoffortune}{50}Les jets pour blesser réussis d'Attaques à Distance dirigées contre l'unité du porteur ayant donné \result{6} doivent être relancés, à moins que la touche n'ait été distribuée sur une figurine avec la règle \toweringpresence{}.

\columnbreak

\pricelistitem{\ringoffire}{40}\boundspell{3} : \pyromancyspellone{} (\pyromancy{}).

\pricelistitem{\divineicon}{40}La figurine du porteur gagne la règle \divineattacks{}.

\pricelistitem{\crownofscorn}{30}Usage Unique. Vous pouvez utiliser cet objet au lieu de faire une tentative de dissipation. Le sort est alors automatiquement dissipé. Vous ne pouvez prendre cet objet que si votre armée ne comprend aucun \wizard{}.

\pricelistitem{\charmofcursediron}{30}L'unité du porteur gagne une \wardsave{5} contre les blessures causées par des Armes d'Artillerie.

\pricelistitem{\potionofswiftness}{10}Usage Unique. Peut être activée au début de n'importe quelle Phase ou Manche de Corps à Corps. Le porteur gagne +3 en Initiative jusqu'à la fin du Tour de Joueur.

\endpricelist

\newpage
\hypertarget{arcaneitems}{\subsection{Objets Cabalistiques}}
\label{arcane_items}

\startpricelist

\pricelistitem{\bookofarcanepower}{100}Le porteur gagne un modificateur de +1 pour lancer et dissiper les sorts.

\pricelistitem{\dispelscroll}{100}Usage Unique. Vous pouvez utiliser ce parchemin au lieu de faire une tentative de dissipation. Le sort est automatiquement dissipé.

\pricelistitem{\essenceofafreemind}{70}Le \wizard{} peut inscrire deux Voies de Magie sur la Liste d'Armée plutôt qu'une, parmi les Voies qui lui sont normalement accessibles. Choisissez entre ces deux Voies au moment de générer les sorts.

\pricelistitem{\wandofstability}{40}Usage Unique. Le porteur peut ajouter un Dé de Magie gratuit à un de ses jets de dissipation après avoir vu le résultat du jet.

\columnbreak

\pricelistitem{\shieldingscroll}{30}Usage Unique. Vous pouvez utiliser ce parchemin au lieu de faire une tentative de dissipation. Toutes les figurines affectées gagnent une \wardsave{4} contre les effets de ce sort.

\pricelistitem{\sceptreofpower}{30}Usage Unique. Le porteur peut ajouter un unique Dé de Magie depuis sa réserve à un jet de lancement de sort après avoir vu le résultat du jet. Cela ne permet pas de dépasser la limite de 5 Dés de Magie pour lancer des sorts.

\endpricelist

\newpage
\hypertarget{magicalstandards}{\subsection{Bannières Magiques}}
\label{magical_standards}

\startpricelist

\pricelistitem{\rendingbanner}{70}Les figurines non Personnage de l'unité du porteur gagnent la règle \armourpiercing{1}.

\pricelistitem{\stalkersstandard}{60}L'unité du porteur gagne les règles \swiftstride{} et \strider{}.

\pricelistitem{\bannerofdiscipline}{50}L'unité du porteur réussit automatiquement ses tests de Panique.

\pricelistitem{\bannerofspeed}{50}L'unité du porteur gagne +1 en Mouvement.

\pricelistitem{\aethericon}{50}L'unité du porteur peut effectuer des tentatives de dissipation comment si elle était un \wizardmaster{}.

\pricelistitem{\flamingstandard}{40}Au début de chaque Manche de Corps à Corps, ou avant de tirer avec l'unité du porteur, la bannière peut être activée. Dans ce cas, toutes les Attaques de Tir et de Corps à Corps non spéciales de l'unité gagnent la règle \flamingattacks{} jusqu'à la fin de la phase.

\columnbreak

\pricelistitem{\warstandard}{30}L'unité du porteur ajoute un bonus de Résultat de Combat de +1 dans n'importe quel Corps à Corps dans lequel elle est impliquée.

\pricelistitem{\iconoftherelentlesscompany}{30}Usage Unique. Vous pouvez l'activer au début de n'importe laquelle de vos étapes des Autres Mouvements. Les figurines d'\infantry{} de l'unité du porteur peuvent alors tripler leur Mouvement en Marche Forcée plutôt que le doubler pour ce tour. Le déplacement ne peut toutefois pas dépasser \distance{15}. Cet effet ne peut pas être activé au premier Tour de Jeu si l'unité a utilisé la règle \vanguard{} ou \scout{}.

\pricelistitem{\gleamingicon}{10}Usage Unique. Peut être activée la première fois que l'unité du porteur rate un test de Commandement. L'unité peut alors relancer les dés pour ce test raté.

\endpricelist
