
\part{Caractéristiques}

\section{Profil de caractéristiques}

Chaque figurine possède un Profil, qui est composé de neuf caractéristiques présentée dans la table \ref{table/characteristics}.

\begin{table}[!htbp]
\centering
\begin{tabular}{M{0.6cm}M{2.8cm}M{12cm}@{}}
\hline
\textbf{M} & Mouvement & \textit{La vitesse à laquelle la figurine se déplace, en pouces par tour.} \tabularnewline
\textbf{CC} & Capacité de Combat & \textit{Détermine les chances de toucher et d'\hspace{0.5pt}éviter d'être touché au Corps à Corps.} \tabularnewline
\textbf{CT} & Capacité de Tir & \textit{Détermine les chances de toucher avec des armes de Tir.} \tabularnewline
\textbf{F} & Force & \textit{Plus la Force est élevée, plus il est facile de blesser et de passer à travers l'armure.} \tabularnewline
\textbf{E} & Endurance & \textit{Une grande Endurance permet d'\hspace{0.5pt}encaisser des coups plus facilement.} \tabularnewline
\textbf{PV} & Points de Vie & \textit{Quand une figurine perd tous ses Points de Vie, elle est retirée comme perte.} \tabularnewline
\textbf{I} & Initiative & \textit{Une plus grande Initiative permet de frapper en premier.} \tabularnewline
\textbf{A} & Attaques & \textit{Le nombre d'\hspace{0.5pt}attaques qu'\hspace{0.7pt}une figurine peut porter au Corps à Corps en une Manche.} \tabularnewline
\textbf{Cd} & Commandement & \textit{Mesure de la discipline et de la capacité à rester au combat dans des situations dangereuses.} \tabularnewline
\hline
\end{tabular}
\caption{\label{table/characteristics}Les Caractéristiques d'une figurine.}
\end{table}


Toutes les caractéristiques vont de 0 à 10 et ne peuvent jamais dépasser ces valeurs.

\noindent\textbf{Caractéristique à 0}

Quand la valeur d'une Caractéristique est égale à 0, cela peut être représenté par un tiret \og - \fg{} ou par une étoile \og \starsymbol{} \fg{}.

\noindent CC à 0 : La figurine est touchée automatiquement au Corps à Corps et ne peut toucher au Corps à Corps que sur un \result{6}.

\noindent CT à 0 : La figurine ne peut pas utiliser d'Arme de Tir.

\noindent F à 0 : Les attaques de la figurine ne peuvent pas blesser.

\noindent E à 0 : Les jets pour blesser des attaques dirigées contre la figurine réussissent sur 2+.

\noindent PV à 0 : La figurine est retirée comme perte.

\noindent A à 0 : Un élément de figurine dont la Caractéristique d'Attaque non modifiée est à 0 ne peut jamais porter d'attaques au Corps à Corps.

\section{Test de caractéristique}

Quand une figurine doit passer un Test de Caractéristique, lancez 1D6. Si le résultat est \result{6} ou s'il est strictement plus grand que la Caractéristique testée, le test est raté. Sinon, le test est réussi. Ainsi, les figurines avec une Caractéristique à 0 échoueront automatiquement tout Test lié à cette Caractéristique.

Quand une figurine possédant plusieurs valeurs pour une même Caractéristique, comme un cheval et son cavalier, doit passer un Test de Caractéristique, faites un test unique pour la figurine entière en utilisant la Caractéristique la plus haute. Quand un Test de Caractéristique doit être passé pour toute une unité, prenez la valeur la plus grande de l'unité.

\subsection{Caractéristique non modifiée}

Une Caractéristique non modifiée est la valeur de la Caractéristique qui peut être lue sur le profil de la figurine, en ignorant tous les équipements, sorts et règles l'affectant. Les seules exceptions sont les changements de Caractéristique appliqués lors de la construction de l'armée, comme par exemple en améliorant une figurine en \og Vétéran \fg{}, lui donnant +1 en Force. Ce type de modification est considéré comme faisant partie de la Caractéristique non modifiée de la figurine.

\subsection{Caractéristique empruntée}

Dans certains cas,  une figurine peut emprunter ou utiliser la Caractéristique d'une autre figurine. La valeur de la Caractéristique est empruntée après toute modification venant de l'équipement, de sorts ou de règles spéciales affectant la figurine propriétaire. Les modifications liées à la figurine qui emprunte la caractéristique sont ensuite appliquées, en suivant les règles de priorité des modificateurs du paragraphe \ref{priority_of_modifiers} ci-dessous.

\subsection{Test de Commandement}

Un test de Commandement se passe en lançant 2D6 et en comparant le résultat avec la valeur de la Caractéristique de Commandement de la figurine. Si le résultat est strictement plus grand que le Commandement de la figurine, le test est raté. Sinon, le test est réussi. \newfromWHB{Si une unité dispose de plusieurs valeurs de Commandement quand elle doit passer un test, par exemple si un Personnage a rejoint l'unité, le propriétaire peut choisir la valeur à utiliser.}

Il y a beaucoup d'occasions différentes de tester le Commandement, comme les tests de Panique ou de Moral. Ces tests restent des tests de Commandement, peu importe les modificateurs et règles spéciales associées.

\subsection[Priorité des modificateurs]{\newfromWHB{Priorité des modificateurs}}
\label{priority_of_modifiers}

Quand une Caractéristique est modifiée, les modificateurs s'appliquent dans un ordre précis :
\begin{enumerate}
\item Caractéristique empruntée* ou fixée à une certaine valeur, comme par la \inspiringpresence{}, ou après un test de \fear{} raté.
\item Multiplicateurs, comme une division par deux, une Caractéristique doublée ou triplée. À moins que le contraire ne soit précisé, arrondissez toujours au supérieur.
\item Additions et soustractions, comme -1 ou +3.
\end{enumerate}
\noindent * Si la Caractéristique à emprunter est modifiée, appliquer ces modifications avant d'emprunter la Caractéristique. 

Si plusieurs modificateurs de la même étape doivent être appliqués, commencez par appliquer les modificateurs sans précision de valeur minimale ou maximale, puis ceux avec de telles précisions, comme \og -1 en Initiative jusqu'à un minimum de 1 \fg{}. Ensuite, appliquez les modificateurs dans l'ordre chronologique. Rappelez-vous qu'une Caractéristique ne peut jamais être modifiée, même temporairement, de façon à dépasser 10 ou tomber en dessous de 0.

