
\part{Phase de Déploiement}

\subsection{Séquence de la Phase de Déploiement}

\hspace*{0.3cm}
\begin{tabular}{c|m{14cm}}
1 & Déterminez qui commence à se déployer. \tabularnewline
2 & Déployez des unités chacun votre tour. \tabularnewline
3 & Le joueur qui a fini en premier annonce s'il veut jouer en premier ou second. \tabularnewline
4 & Déployez les unités restantes. \tabularnewline
5 & Déployez les Éclaireurs. \tabularnewline
6 & Déplacez les unités avec la règle \vanguard{}. \tabularnewline
7 & Autres règles et capacités. \tabularnewline
8 & Lancez le dé pour le premier tour. \tabularnewline
\end{tabular}

\subsection{Qui commence à se déployer ?}

\newfromWHB{Le joueur qui n'a pas choisi sa zone de déploiement décide quel joueur commence son déploiement.}

\subsection{Cœur du déploiement}

Les joueurs déploient tour à tour leurs unités entièrement dans leurs zones de déploiement respectives. À chacun de ses tours, un joueur doit déployer au moins une unité \newfromWHB{mais peut en déployer autant  qu'il le désire}. Les unités de type \warmachine{} comptent pour une seule unité en terme de déploiement, et doivent être déployées en même temps. Les Personnages suivent le même traitement. \newfromWHB{Quand un joueur a déployé toutes ses unités à l'exception des unités qui ne sont pas déployées selon les règles normales, comme celles avec la règle \scout{} ou \ambush{}, il doit annoncer s'il va essayer d'avoir le premier ou le second tour.}

\subsection{Déploiement des unités restantes}

L'autre joueur peut maintenant déployer le reste de son armée. Comptez combien d'unités il lui reste, l'ensemble des Machines de Guerre et l'ensemble des Personnages comptant toujours chacun pour une unité. Cela représente le \og Nombre d'Unités non Déployées \fg{} qui sera utilisé à la fin de cette séquence.

\subsection{Autres règles et aptitudes}

Toutes les règles et capacités restantes décrites comme ayant lieu juste avant le début de la partie doivent être déclenchées à cette étape.

\newpage
\subsection{Qui joue en premier ?}

\newfromWHB{Les deux joueurs lancent désormais 1D6. Le joueur qui a fini de déployer en premier ajoute le \og Nombre d'Unités non Déployées \fg{} au résultat de son jet.}
\begin{itemize}[label={-}]
\item Si le joueur qui a fini de se déployer en premier obtient un résultat strictement plus haut, il doit jouer en premier ou en second, suivant son souhait annoncé à la troisième étape.
\item Si le joueur qui a fini de se déployer en second obtient un résultat plus haut \newfromWHB{ou une égalité}, il peut choisir quel joueur aura le premier tour.
\end{itemize}

