
\hypertarget{commandgroup}{\part{État-Major}}
\label{command_group}

Certaines unités ont la possibilité de promouvoir des figurines ordinaires en figurines d'État-Major : Champion, Porte-Étendard et Musicien. Chacune de ces figurines possède la règle \newfromWHB{\frontrank} (voir le paragraphe \ref{front_rank}, page \pageref{front_rank}). Si un Porte-Étendard ou un Musicien devait être retiré comme perte, retirez une autre figurine ordinaire à la place, puis mettez le Porte-Étendard à la place de cette dernière. On suppose qu'un autre soldat a pris son équipement et ses responsabilités. Les Champions ne sont cependant pas aussi aisément remplacés et peuvent être ciblés personnellement, blessés et tués comme des Personnages. Si assez de blessures sont infligées à des figurines ordinaires pour éliminer l'unité entière, toute blessure excédentaire est répercutée sur le Champion, même s'il est en Défi.

\section{Musicien}
\label{musician}

\paragraph{Reformation Rapide}

Une unité contenant un Musicien peut effectuer une Reformation Rapide en passant un test de Commandement. Si le test est raté, l'unité doit faire une Reformation normale. S'il est réussi, l'unité fait une Reformation avec les avantages suivants :
\begin{itemize}[label={-}]
\item L'unité peut tirer pendant la Phase de Tir à venir.
\item L'unité peut faire un Mouvement Simple après la Reformation.
\end{itemize}

\paragraph{Du nerf !}

Dans le cas d'une égalité au Résultat de Combat, le camp qui possède un Musicien l'emporte. Si les deux camps ont un Musicien, le combat reste une égalité. Si un camp gagne le combat grâce à son Musicien, \newfromWHB{aucun modificateur du Commandement n'est appliqué à l'unité qui perd, ni pour le test de Moral, ni pour le test de Reformation de Combat. Les unités avec la règle \unstable{} ne sont pas affectées.}

\paragraph{Regroupement}

Une unité en fuite qui possède un Musicien a un bonus de +1 en Commandement pour passer son test de Ralliement.

\newpage
\section{Porte-Étendard}

\paragraph{Bonus de Combat}

Un camp qui possède au moins un Porte-Étendard gagne un bonus de +1 à son Résultat de Combat.

\paragraph{Étendard capturé}

Quand un Porte-Étendard est retiré comme perte alors qu'il était engagé dans un Corps à Corps, on considère que l'Étendard est capturé par l'adversaire. Si une unité contenant un Porte-Étendard fuit d'un combat après avoir raté son test de Moral, l'Étendard est capturé. \newfromWHB{Remplacez sa figurine par une figurine ordinaire.} L'unité ne profite plus des bénéfices du Porte-Étendard, que ce soit le Bonus de Combat ou une éventuelle Bannière Magique.

\section[Porte-Étendard Vétéran]{\newfromWHB{Porte-Étendard Vétéran}}

Certains Porte-Étendards des unités de Base peuvent être améliorés en Porte-Étendard Vétéran. Ce dernier peut prendre une Bannière Magique dans la limite de 25 pts. Il est \oneofakind{}, donc une seule figurine dans l'armée peut le devenir (ou deux dans une Grande Armée).

\section{Champion}

\paragraph{Premier Parmi ses Pairs}

Tous les Champions ont un bonus de \newfromWHB{+1 Attaque, +1 en Capacité de Combat et +1 en Capacité de Tir} sur leur Profil de Caractéristiques. Si la figurine est composée de plusieurs éléments, seules les caractéristiques du cavalier, ou d'un seul membre d'équipage, sont augmentées.

\paragraph{\newfromWHB{Meneur de Charge}}

Une unité contenant un Champion réussit automatiquement les charges nécessitant un résultat du jet de distance de charge de \result{4} ou moins. Sinon, elle lance les dés normalement.

