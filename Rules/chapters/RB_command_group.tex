
\hypertarget{commandgroup}{\part{État-Major}}
\label{command_group}

Certaines unités ont la possibilité de promouvoir des figurines ordinaires en figurines d'État-Major : Champion, Porte-Étendard et Musicien. Chacune de ces figurines possède la règle \frontrank{} (voir le paragraphe \ref{front_rank}, page \pageref{front_rank}). Si un Porte-Étendard ou un Musicien devait être retiré comme perte, retirez une autre figurine ordinaire à la place, puis mettez le Porte-Étendard à la place de cette dernière. On suppose qu'un autre soldat a pris son équipement et ses responsabilités. Les Champions ne sont cependant pas aussi aisément remplacés et peuvent être ciblés personnellement dans certaines situations. Par exemple, on peut leur allouer des attaques au Corps à Corps, les cibler avec une attaque qui touche une figurine individuelle comme un sort de type \focused{} ou encore le toucher avec des attaques qui ciblent toutes les figurines d'une unité. Si assez de blessures sont infligées à des figurines ordinaires pour éliminer l'unité entière, toute blessure excédentaire est répercutée sur le Champion, même s'il est en Défi.

\section{Musicien}
\label{musician}

\paragraph{Reformation Rapide}

Une unité contenant un Musicien peut effectuer une Reformation Rapide. L'unité fait une Reformation avec les avantages suivants :
\begin{itemize}[label={-}]
\item L'unité peut tirer pendant la Phase de Tir à venir.
\item L'unité peut faire un Mouvement Simple après la Reformation.
\end{itemize}

\paragraph{Suivez le Rythme}

Les Tests de Marche Forcée passés par des unités à moins de \distance{8} d'une unité ennemie avec un Musicien sont faits avec un malus de -1 en Commandement, à moins que l'unité qui passe le test possède aussi un Musicien.


\section{Porte-Étendard}

\paragraph{Bonus de Combat}

Un camp qui possède des Porte-Étendards gagne un bonus de +1 à son Résultat de Combat par Porte-Étendard engagé dans le Corps à Corps.


\section{Porte-Étendard Vétéran}

Certains Porte-Étendards des unités de Base peuvent être améliorés en Porte-Étendard Vétéran. Ce dernier peut prendre une Bannière Magique dans la limite de 50 pts. Il a la règle 0-1 Choix par Armée, donc une seule figurine dans l'armée peut le devenir (ou deux dans une Grande Armée).

\section{Champion}
\label{champion}

\paragraph{Premier Parmi ses Pairs}

Un Champion gagne un bonus de +1 Attaque. Si la figurine est composée de plusieurs éléments, seules les caractéristiques du cavalier, ou d'un seul membre d'équipage, sont augmentées.

\paragraph{Meneur de Charge}

Une unité contenant un Champion réussit automatiquement les charges nécessitant un résultat du jet de distance de charge de \result{4} ou moins. Sinon, elle lance les dés normalement.

