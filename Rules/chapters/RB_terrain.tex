
\part{Terrains et Décors}

\section{Types de Terrain}

\paragraph{Terrain Dangereux (X)}

Lorsqu'une figurine effectue une Marche Forcée, charge \newfromWHB{(à l'exception d'une Charge Ratée)}, fuit, poursuit ou fait une Charge Irrésistible au travers, en entrant ou en sortant d'un Terrain Dangereux, elle doit effectuer un test de Terrain Dangereux. \newfromWHB{Lancez 4D6 pour un Char, un Monstre ou un Monstre Monté, 2D6 pour une figurine de type Infanterie Monstrueuse, Bête Monstrueuse ou Cavalerie Monstrueuse}, et 1D6 pour toute autre figurine. Pour chaque \result{1} obtenu, la figurine subit une blessure avec la règle \armourpiercing{6}.

\newfromWHB{Cette règle apparaît sous la forme Terrain Dangereux (X). Dans ce cas, la figurine subit une blessure pour chaque résultat de dé inférieur ou égal à X.}

Les tests de Terrain Dangereux doivent être passés dès que la figurine touche le Décor en question. Dans la plupart des cas, le moment où une figurine meurt importe peu. Il est donc plus aisé d'effectuer tous les tests de Terrain Dangereux d'une unité en même temps. Les blessures dues aux tests de Terrain Dangereux sont allouées à la réserve de PVs correspondant à la figurine, comme d'habitude. Une figurine ne passe jamais plus d'un test de Terrain Dangereux pour un même élément de Décor durant un mouvement, mais il se peut qu'elle doive passer plusieurs tests en raison de différents éléments de Décor ou capacités.

\paragraph{Décor Occultant}

Aucune Ligne de Vue ne peut être tracée au travers d'un Décor Occultant. Il est possible de se déplacer à travers certains Décors Occultants, comme les Collines. Pour ces derniers, les Lignes de Vue peuvent être tracées depuis ou vers l'intérieur du Décor, mais jamais au travers. 

\paragraph{Décor et Couvert Léger}

Une unité sur, dans ou occultée par un Décor offrant un Couvert Léger bénéficie d'un Couvert Léger si au moins la moitié de son Empreinte au Sol est occultée par ce Décor (vis-à-vis des Lignes de Vue du tireur). \newfromWHB{Les Lignes de Vue d'une figurine ne sont pas affectées par le Décor sur ou dans lequel elle se trouve.}

\paragraph{Décor et Couvert Lourd}

Une unité sur, dans ou occultée par un Décor offrant un Couvert Lourd bénéficie d'un Couvert Lourd si au moins la moitié de son Empreinte au Sol est occultée par ce Décor (vis-à-vis des Lignes de Vue du tireur). \newfromWHB{Les Lignes de Vue d'une figurine ne sont pas affectées par le Décor sur ou dans lequel elle se trouve.}

\newpage
\section{Liste des Terrains particuliers}

Un Terrain ou Décor est une zone topographique du champ de bataille qui peut mélanger Terrain Dangereux, Décor Occultant, Couvert Léger ou Lourd, voire d'autres règles spéciales selon son type.

\subsection{Terrain Découvert}

Les Terrains Découverts n'ont normalement aucun effet sur les Lignes de Vue, les Couverts ou le mouvement. Toute partie du champ de bataille qui n'est pas couverte par un autre élément de Décor est considéré comme Terrain Découvert.

\subsection{Terrain Infranchissable}

\noindent\begin{tabular}{>{\bfseries\raggedleft}p{2.2cm}p{13.5cm}}
Ligne de Vue & Un Terrain Infranchissable est un Décor Occultant. \tabularnewline
Couvert & Un Terrain Infranchissable offre un Couvert Lourd. \tabularnewline
Mouvement & Aucune figurine ne peut se déplacer dans ou à travers un Terrain Infranchissable. Si un mouvement de charge amène une figurine à moins d'\distance{1} d'un Terrain Infranchissable, son unité ignore l'écart d'un pouce jusqu'à ce qu'elle se soit éloignée à plus d'\distance{1}. \tabularnewline
\end{tabular}

\subsection{Champ}

\noindent\begin{tabular}{>{\bfseries\raggedleft}p{2.2cm}p{13.5cm}}
Couvert & Un Champ offre un Couvert Léger à toute unité dont au moins la moitié de l'Empreinte au Sol est dans le Champ, sauf si c'est une \largetarget{}. \tabularnewline
Mouvement & Un Champ est un Terrain Dangereux (1) pour la Cavalerie, les Chars et la Cavalerie Monstrueuse. \tabularnewline
Facile à Embraser & Une unité dont au moins la moitié de l'Empreinte au Sol est dans un Champ gagne la règle \flammable{}. Un Champ est un Terrain Dangereux (1) pour une figurine avec des \flamingattacks{} (sur la figurine ou son arme). \tabularnewline
\end{tabular}

\subsection{Colline}

\noindent\begin{tabular}{>{\bfseries\raggedleft}p{2.2cm}p{13.5cm}}
Ligne de Vue & \newfromWHB{Une Colline est un Décor Occultant.} \tabularnewline
Couvert & \newfromWHB{Une Colline offre un Couvert Léger aux unités partiellement sur elle.\newline
Une Colline offre un Couvert Lourd aux unités complètement hors de la Colline.} \tabularnewline
Position surélevée & \newfromWHB{Une figurine dont au moins la moitié du socle repose sur une Colline est considérée comme étant de Grande Taille pour les Lignes de Vue et le Couvert.} \tabularnewline
Charger d'une Colline & Lors de la Manche de Corps à Corps d'un tour où une unité charge, si au moins la moitié de son Empreinte au Sol était sur une Colline au début de son mouvement de charge, et que moins de la moitié de son Empreinte au Sol est toujours sur la Colline à la fin de ce mouvement, elle gagne un bonus de +1 au Résultat de Combat. \tabularnewline
\end{tabular}

\subsection{Forêt}

\noindent\begin{tabular}{>{\bfseries\raggedleft}p{2.2cm}p{13.5cm}}
Couvert & Une Forêt offre un Couvert Léger. \tabularnewline
Mouvement & Une Forêt est un Terrain Dangereux (1) pour la Cavalerie, les Chars, la Cavalerie Monstrueuse et les unités qui effectuent un mouvement de Vol. \tabularnewline
Rangs Brisés & Une unité avec la majorité de son Empreinte au Sol dans une Forêt ne peut jamais être Indomptable. \tabularnewline
\stubborn{} & Une unité avec la règle \skirmisher{} ou un Personnage d'Infanterie seul dont la majorité de l'Empreinte au Sol est dans une Forêt gagne la règle \stubborn{}. \tabularnewline
\end{tabular}

\subsection{Ruines}

\noindent\begin{tabular}{>{\bfseries\raggedleft}p{2.2cm}p{13.5cm}}
Couvert & Des Ruines offrent un Couvert Lourd à toute unité dont au moins la moitié de l'Empreinte au Sol est à l'intérieur, sauf si c'est une \largetarget{}. \tabularnewline
Mouvement & Les Ruines sont un Terrain Dangereux (1) pour toute unité qui n'a pas la règle \skirmisher{}. Elles sont un Terrain Dangereux (2) pour la Cavalerie, les Chars et la Cavalerie Monstrueuse. \tabularnewline
\end{tabular}

\subsection{\water}

\noindent\begin{tabular}{>{\bfseries\raggedleft}p{2.2cm}p{13.5cm}}
Mouvement & Aucune unité ne peut effectuer de Marche Forcée dans, à travers ou pour sortir d'\water{}. Les \water{} sont un Terrain Dangereux (1) pour la Cavalerie, les Chars et la Cavalerie Monstrueuse. \tabularnewline
Désordre & \newfromWHB{Un rang au moins partiellement immergé ne compte jamais comme un Rang Complet. Si la majorité d'une unité se situe dans des \water{}, elle compte comme n'ayant aucun Rang Complet. Cette règle n'affecte pas les unités avec la règle \strider{} ou \strider{\water}.} \tabularnewline
\end{tabular}

\subsection{Mur}

\noindent\begin{tabular}{>{\bfseries\raggedleft}p{2.2cm}p{13.5cm}}
Couvert & Un Mur offre un Couvert Lourd à toute unité, sauf avec la règle \largetarget{}, dont au moins la moitié de l'avant du premier rang (ou socle si figurine seule) défend le Mur (voir ci-dessous). Si l'unité qui la prend pour cible est dans l'arc arrière ou dans un arc latéral, utilisez le côté correspondant de l'unité pour déterminer si elle bénéficie d'un Couvert. \tabularnewline
Mouvement & Ignorez les Murs pour les déplacements et le positionnement des unités. Un Mur est un Terrain Dangereux (1) pour la Cavalerie, les Chars et la Cavalerie Monstrueuse. \tabularnewline
Défendre un Mur & \newfromWHB{Une figurine alignée et en contact socle à socle avec un Mur le Défend.} \tabularnewline
Combat & Les figurines qui Défendent un Mur gagnent la règle \distracting{} contre les attaques venant d'une unité qui vient de les charger et qui est sur le même côté de l'unité en défense que le Mur. Cet effet ne dure que pour la première Manche de Corps à Corps suivant la charge. \tabularnewline
\end{tabular}

\subsection{Bâtiment}

\noindent\begin{tabular}{>{\bfseries\raggedleft}p{2.2cm}p{13.5cm}}
Ligne de Vue & Un Bâtiment est un Décor Occultant. \tabularnewline
Couvert & Un Bâtiment offre un Couvert Lourd. \tabularnewline
Entrer dans un Bâtiment & Une unité constituée uniquement d'Infanterie, de Bêtes de Guerre, de Nuées, d'Infanterie Monstrueuse ou de Bêtes Monstrueuses peut entrer dans un Bâtiment \textbf{vide} \newfromWHB{en entrant en contact avec lui pendant l'étape des Autres Mouvements. L'unité entière pénètre dans le Bâtiment. Aucune figurine de l'unité ne peut parcourir plus de trois fois la valeur de sa Caractéristique de Mouvement en faisant cela : mesurez la distance entre la position initiale de chaque figurine et le point le plus proche du Bâtiment.} Les unités du Type de Troupe mentionnés plus haut peuvent aussi être déployées directement dans un Bâtiment vide. Quand une unité est dans un Bâtiment, son centre et considéré comme étant le centre du Bâtiment et la portée depuis et jusqu'aux figurines se mesure en partant des bords du Bâtiment. Pour pouvoir déployer une unité sans la règle \scout{} dans un Bâtiment, ce dernier doit se trouver intégralement dans votre zone de déploiement. On considère qu'une unité dans un Bâtiment a exactement un rang et une colonne, et aucun Rang Complet. On dit qu'elle est en Garnison dans le Bâtiment. \tabularnewline
\flammable{} & \newfromWHB{Les figurines en Garnison dans un Bâtiment gagnent la règle \flammable{}.} \tabularnewline
Quitter un Bâtiment & Une unité en Garnison dans un Bâtiment peut l'abandonner pendant l'étape des Autres Mouvements si elle n'y est pas entrée pendant le même Tour de Joueur. Quand elle le quitte, l'unité est déployée dans n'importe quelle formation tant que son rang arrière touche le Bâtiment et qu'aucune figurine n'a parcouru plus de deux fois la valeur de sa Caractéristique de Mouvement depuis un bord du Bâtiment. L'unité ne peut alors plus bouger de la phase. Une unité en Garnison ne peut pas déclarer de charge. Il est aussi possible de quitter un Bâtiment quand l'unité en Garnison rate un test de Panique ou perd un combat et rate un test de Moral. Quand cela arrive, commencez par regarder la direction de la fuite et l'endroit où devrait finir le centre de l'unité. Placez ensuite l'unité avec son centre en ce point, avec son front dans la direction de la fuite et dans n'importe quelle formation autorisée qui respecte la Règle du Pouce d'Écart. Si ce n'est pas possible, déployez l'unité dans une formation autorisée, avec son front dans la direction de la fuite, puis déplacez l'unité jusqu'à ce qu'elle puisse être placée. \tabularnewline
Tirer depuis un Bâtiment & Une unité qui tire depuis un Bâtiment compte comme étant de Grande Taille. \newfromWHB{Pas plus de 15 figurines (ou 5 dans le cas de figurines de Type Monstrueux) ne peuvent tirer à la fois d'un Bâtiment.} \tabularnewline
Attaques à Distance sur une Garnison & Un Bâtiment offre un Couvert Lourd à une unité en Garnison. Si un Gabarit touche le Bâtiment, on considère qu'il a touché 1D6 figurines de la Garnison. Si le Gabarit possède des règles particulières pour sa touche centrale, et que le centre touche le Bâtiment, appliquez ces règles pour toutes les touches. \tabularnewline
Autres mouvements & Toute unité qui n'entre pas, ne quitte pas ou n'attaque pas un Bâtiment le traite exactement comme un Terrain Infranchissable. \tabularnewline
\end{tabular}

\noindent\begin{tabular}{>{\bfseries\raggedleft}p{2.2cm}p{13.5cm}}
Attaquer un Bâtiment & La déclaration et la résolution des charges contre un Bâtiment se fait normalement avec les exceptions suivantes. L'unité en Garnison ne peut pas déclarer une fuite en réaction à la charge. Déplacez l'unité en charge en contact avec le Bâtiment, en alignant et en maximisant le contact (le Bâtiment devrait avoir un socle rectangulaire). Le Bâtiment ne peut pas être déplacé pour maximiser le contact. \tabularnewline
Corps à Corps dans un Bâtiment & Les montures, quel que soit leur type, ne peuvent pas combattre contre l'unité en Garnison et aucune figurine dans le combat ne peut bénéficier de \impacthits{}, \lance{}, \lightlance{}, \mountsprotection{} ou \barding{}. Jusqu'à 10 figurines dans chaque camp qui ne sont pas en contact avec d'autres ennemis peuvent combattre. On les appelle les Assiégeants et les Défenseurs. Les Chars, l'Infanterie Monstrueuse, les Bêtes Monstrueuses et la Cavalerie Monstrueuse comptent pour 3 figurines, et les Monstres et Monstres Montés comptent pour 6. Ces figurines sont choisies au début de chaque Manche de Corps à Corps, en commençant par les Assiégeants.

Chaque figurine peut allouer ses attaques contre n'importe quelle figurine ennemie qui a été choisie. Elles comptent toutes comme étant en contact socle à socle. Toute figurine ordinaire tuée peut être remplacée par une autre figurine ordinaire. Les autres figurines ne peuvent pas être remplacées. Par exemple, si un Personnage a été choisi et qu'il est tué avant d'avoir pu attaquer, il n'est pas remplacé et seules 9 figurines pourront attaquer dans son camp.

Quand le combat se termine, calculez le Résultat de Combat comme d'habitude, \newfromWHB{sauf que l'unité en Garnison compte comme n'ayant aucun rang pour le Résultat de Combat et la règle Indomptable. Les Étendards (communs ou Grande Bannière) ne comptent que si les figurines les portant participent au combat.} Une fois le combat résolu, trois cas sont possibles :
\begin{itemize}[label={\textbullet}]
\item \newfromWHB{Si le combat est une égalité, ou si les Assiégeants ont gagné mais l'unité en Garnison a réussi son test de Moral, chaque unité Assiégeante peut au choix effectuer un Pivot Post-Combat, en ignorant le Bâtiment, puis être poussée à \distance{1} du Bâtiment, ou continuer à assiéger le Bâtiment. Dans ce dernier cas, les unités comptent comme étant engagées au Corps à Corps et l'assaut se poursuivra durant la prochaine Phase de Corps à Corps. Si une unité Assiégeante est engagée au Corps à Corps avec d'autres unités ennemies que l'unité en Garnison, elle doit choisir de poursuivre l'assaut.}
\item \newfromWHB{Si les Assiégeants l'emportent et que toutes les unités ennemies ont raté leur test de Moral et fui ou ont été annihilées, ils peuvent soit choisir de rentrer dans le Bâtiment (s'ils le peuvent), soit effectuer un Pivot Post-Combat, en ignorant le Bâtiment, puis être poussés à \distance{1} du Bâtiment. Dans le cas d'un Combat Multiple, ils peuvent aussi choisir de poursuivre une unité ennemie en fuite \textbf{si ce n'était pas} l'unité en Garnison.}
\item \newfromWHB{Si l'unité en Garnison gagne, les Assiégeants doivent passer un test de Moral, comme d'habitude, et fuir s'ils le ratent. L'unité en Garnison ne peut pas poursuivre. Si les Assiégeants réussissent leur test de Moral, ils sont repoussés à \distance{1} du Bâtiment, ainsi que toute autre unité ayant participé au combat et n'étant pas en Garnison. Les unités ne sont plus engagées au Corps à Corps.}
\end{itemize}\tabularnewline
\end{tabular}
