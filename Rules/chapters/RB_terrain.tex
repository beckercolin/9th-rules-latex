
\part{Terrains et Décors}

\hypertarget{terraintypes}{\section{Types de Terrain}}
\label{terrain_types}

\hypertarget{dangerousterrain}{\paragraph{Terrain Dangereux (X)}}

Lorsqu'une figurine effectue une Marche Forcée, charge (à l'exception d'une Charge Ratée), fuit, poursuit ou fait une Charge Irrésistible en entrant, au travers ou en sortant d'un Terrain Dangereux, elle doit effectuer un test de Terrain Dangereux. Lancez 4D6 pour un Char, un Monstre ou un Monstre Monté, 2D6 pour une figurine de type Infanterie Monstrueuse, Bête Monstrueuse ou Cavalerie Monstrueuse, et 1D6 pour toute autre figurine. Pour chaque \result{1} obtenu, la figurine subit une blessure avec la règle \armourpiercing{6}.

Cette règle apparaît sous la forme Terrain Dangereux (X). Dans ce cas, la figurine subit une blessure pour chaque résultat de dé inférieur ou égal à X.

Les tests de Terrain Dangereux doivent être passés dès que la figurine touche le Décor en question. Dans la plupart des cas, le moment où une figurine meurt importe peu. Il est donc plus aisé d'effectuer tous les tests de Terrain Dangereux d'une unité en même temps. Les blessures dues aux tests de Terrain Dangereux sont allouées à la réserve de PVs correspondant à la figurine, comme d'habitude. Une figurine ne passe jamais plus d'un test de Terrain Dangereux pour un même élément de Décor durant un mouvement, mais il se peut qu'elle doive passer plusieurs tests en raison de différents éléments de Décor ou capacités.

\paragraph{Décor Occultant}

Aucune Ligne de Vue ne peut être tracée au travers d'un Décor Occultant. Il est possible de se déplacer à travers certains Décors Occultants, comme les Collines. Pour ces derniers, les Lignes de Vue peuvent être tracées depuis ou vers l'intérieur du Décor, mais jamais au travers. 

\paragraph{\softterrain}

Une unité sur, dans ou cachée par un \softterrain{} bénéficie d'un Couvert Léger si plus de la moitié de son Empreinte au Sol est cachée par ce Décor, c'est à dire que les Lignes de Vue du tireur passent par le Décor. Les Lignes de Vue d'une figurine ne sont pas affectées par le Décor sur ou dans lequel elle se trouve.

\paragraph{\hardterrain}

Une unité sur, dans ou cachée par un \hardterrain{} bénéficie d'un Couvert Lourd si plus de la moitié de son Empreinte au Sol est cachée par ce Décor, c'est à dire que les Lignes de Vue du tireur passent par le Décor. Les Lignes de Vue d'une figurine ne sont pas affectées par le Décor sur ou dans lequel elle se trouve.

\newpage
\hypertarget{terrainfeatures}{\section{Liste des Décors particuliers}}
\label{terrain_features}

Un Terrain ou Décor est une zone topographique du champ de bataille qui peut mélanger Terrain Dangereux, Décor Occultant, \softterrain{} ou \hardterrain{}, voire d'autres règles spéciales selon son type.

\subsection{Terrain Découvert}

Les Terrains Découverts n'ont normalement aucun effet sur les Lignes de Vue, les Couverts ou le mouvement. Toute partie du champ de bataille qui n'est pas couverte par un autre élément de Décor est considérée comme Terrain Découvert.

\hypertarget{impassableterrain}{\subsection{Terrain Infranchissable}}

\noindent\begin{tabular}{>{\bfseries\raggedleft}p{2.2cm}p{13.5cm}}
Ligne de Vue & Un Terrain Infranchissable est un Décor Occultant. \tabularnewline
Couvert & Un Terrain Infranchissable est un \hardterrain{}. \tabularnewline
Mouvement & Aucune figurine ne peut se déplacer dans ou à travers un Terrain Infranchissable. Si un mouvement de charge amène une figurine à moins d'\distance{1} d'un Terrain Infranchissable, son unité ignore l'écart d'un pouce jusqu'à ce qu'elle se soit éloignée à plus d'\distance{1}. \tabularnewline
\end{tabular}

\hypertarget{fields}{\subsection{Champ}}

\noindent\begin{tabular}{>{\bfseries\raggedleft}p{2.2cm}p{13.5cm}}
Couvert & Un Champ compte comme un \softterrain{} pour toute unité dont plus de la moitié de l'Empreinte au Sol est dans le Champ, sauf s'il s'agit d'une figurine avec la règle \toweringpresence{}. \tabularnewline
Mouvement & Un Champ est un Terrain Dangereux (1) pour la Cavalerie, les Chars et la Cavalerie Monstrueuse. \tabularnewline
Facile à Embraser & Une unité dont plus de la moitié de l'Empreinte au Sol est dans un Champ gagne la règle \flammable{}. Un Champ est un Terrain Dangereux (1) pour une figurine avec des \flamingattacks{} (sur la figurine ou son arme). \tabularnewline
\end{tabular}

\hypertarget{hills}{\subsection{Colline}}

\noindent\begin{tabular}{>{\bfseries\raggedleft}p{2.2cm}p{13.5cm}}
Ligne de Vue & Une Colline est un Décor Occultant. \tabularnewline
Couvert & Une Colline compte comme un \softterrain{} pour toute unité partiellement sur elle : au moins la moitié de l'Empreinte au Sol non située sur la Colline doit être occultée par celle-ci pour obtenir le Couvert Léger.\newline
Une Colline compte comme un \hardterrain{} pour toute unité complètement hors de la Colline. \tabularnewline
Position surélevée & Une figurine dont au moins la moitié du socle repose sur une Colline est considérée comme étant de Taille Gigantesque pour les Lignes de Vue et le Couvert. \tabularnewline
Charger d'une Colline & Quand une unité charge, si plus de la moitié de son Empreinte au Sol était sur une Colline au début de son mouvement de charge, et que plus de la moitié de son Empreinte au Sol est hors de la Colline à la fin de ce mouvement, elle gagne un bonus de +1 au Résultat de Combat lors de la Manche de Corps à Corps qui suit. \tabularnewline
\end{tabular}

\hypertarget{forests}{\subsection{Forêt}}

\noindent\begin{tabular}{>{\bfseries\raggedleft}p{2.2cm}p{13.5cm}}
Couvert & Une Forêt est un \softterrain{}. \tabularnewline
Mouvement & Une Forêt est un Terrain Dangereux (1) pour la Cavalerie, les Chars, la Cavalerie Monstrueuse et les unités qui effectuent un mouvement de Vol. \tabularnewline
Rangs Brisés & Une unité dont plus de la moitié de son Empreinte au Sol est dans une Forêt ne peut jamais être Indomptable. \tabularnewline
\stubborn{} & Une unité dont plus de la moitié de l'Empreinte au Sol est dans une Forêt gagne la règle \stubborn{} si elle est composée uniquement de figurines d'Infanterie et d'Infanterie Monstrueuse avec la règle \lighttroops{} mais sans la règle \fly{}. \tabularnewline
\end{tabular}

\hypertarget{ruins}{\subsection{Ruines}}

\noindent\begin{tabular}{>{\bfseries\raggedleft}p{2.2cm}p{13.5cm}}
Couvert & Des Ruines comptent comme un \hardterrain{} pour toute unité dont plus de la moitié de l'Empreinte au Sol est à l'intérieur, sauf s'il s'agit d'une figurine avec la règle \toweringpresence{}. \tabularnewline
Mouvement & Les Ruines sont un Terrain Dangereux (1) pour toute unité qui n'a pas la règle \skirmisher{}. Elles sont un Terrain Dangereux (2) pour la Cavalerie, les Chars et la Cavalerie Monstrueuse. \tabularnewline
\end{tabular}

\hypertarget{water}{\subsection{\water}}

\noindent\begin{tabular}{>{\bfseries\raggedleft}p{2.2cm}p{13.5cm}}
Mouvement & Aucune unité ne peut effectuer de Marche Forcée dans, à travers ou pour sortir d'\water{}. Les \water{} sont un Terrain Dangereux (1) pour la Cavalerie, les Chars et la Cavalerie Monstrueuse. \tabularnewline
Désordre & Un rang au moins partiellement immergé ne compte jamais comme un Rang Complet. Si plus de la moitié de l'Empreinte au Sol d'une unité se situe dans des \water{}, elle compte comme n'ayant aucun Rang Complet. Cette règle n'affecte pas les unités dont plus de la moitié des figurines ont la règle \strider{} ou \strider{\water}. \tabularnewline
\end{tabular}

\hypertarget{walls}{\subsection{Mur}}

\noindent\begin{tabular}{>{\bfseries\raggedleft}p{2.2cm}p{13.5cm}}
Couvert & Un Mur offre un Couvert Lourd à toute unité dont au moins la moitié des figurines n'a pas la règle \toweringpresence{} et dont plus de la moitié de l'avant du premier rang (ou socle si figurine seule) défend le Mur (voir ci-dessous). Si l'unité qui la prend pour cible est dans l'arc arrière ou dans un arc latéral, utilisez le côté correspondant de l'unité pour déterminer si elle bénéficie d'un Couvert. \tabularnewline
Mouvement & Ignorez les Murs pour les déplacements et le positionnement des unités. Un Mur est un Terrain Dangereux (1) pour la Cavalerie, les Chars et la Cavalerie Monstrueuse. \tabularnewline
Défendre un Mur & Une figurine alignée et en contact socle à socle avec un Mur le Défend. \tabularnewline
Combat & Les figurines qui Défendent un Mur gagnent la règle \distracting{} contre les attaques venant d'une unité qui vient de les charger et qui est sur le même côté de l'unité en défense que le Mur. Cet effet ne dure que pour la première Manche de Corps à Corps suivant la charge. \tabularnewline
\end{tabular}

\hypertarget{buildings}{\subsection{Bâtiment}}
\label{buildings}

\noindent\begin{tabular}{>{\bfseries\raggedleft}p{2.2cm}p{13.5cm}}
Ligne de Vue & Un Bâtiment est un Décor Occultant. \tabularnewline
Couvert & Un Bâtiment est un \hardterrain{}. \tabularnewline
Mouvement & Une unité constituée uniquement d'Infanterie, de Bêtes de Guerre, de Nuées, d'Infanterie Monstrueuse, de Bêtes Monstrueuses ou d'un mélange de ces types traite un Bâtiment comme du Terrain Découvert pendant l'étape des Autres Mouvements, mais elles ne peuvent pas finir leur mouvement dans ou à moins de \distance{1} du Bâtiment. Dans toutes les autres situations, un Bâtiment est traité comme un Terrain Infranchissable. \tabularnewline
À Couvert & Une unité dont plus de la moitié de l'Empreinte au Sol traverse un Bâtiment (sauf mouvement de \fly{}) bénéficie d'un Couvert Lourd jusqu'au début du prochain Tour de Joueur allié, ou jusqu'à ce qu'elle utilise une Attaque à Distance, selon la situation qui arrive en premier. \tabularnewline
\end{tabular}
