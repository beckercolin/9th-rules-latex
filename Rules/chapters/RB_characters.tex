
\part{Personnages}
\label{characters}

Hormis dans le cas où cela est spécifiquement mentionné, toute figurine inscrite dans la section des Héros ou des Seigneurs est un Personnage.

\section{Personnages isolés}

\newfromWHB{Un Personnage que vous souhaitez laisser seul peut être considéré comme une unité constituée d'une seule figurine. En conséquence, les règles normales des unités s'appliquent.}

\section{Rejoindre une unité}

Les Personnages peuvent faire partie intégrante des autres unités en les rejoignant. Cela peut être réalisé en déployant le Personnage dans une unité en début de partie ou en amenant le Personnage au contact de l'unité au cours de l'étape des Autres Mouvements. Les unités engagées au Corps à Corps ou en fuite ne peuvent pas être rejointes.

Les Personnages peuvent aussi rejoindre d'autres Personnages pour constituer une unité composée exclusivement de Personnages.

Quand un Personnage rejoint une unité, il est immédiatement placé dans une position règlementaire pour suivre la règle \newfromWHB{Au Premier Rang} (voir le paragraphe \ref{front_rank}, page \pageref{front_rank}), entraînant éventuellement le déplacement d'autres figurines vers l'arrière. \newfromWHB{Lorsqu'un Personnage rejoint une unité, on peut le positionner librement sur tout emplacement règlementaire qu'il pourrait avoir atteint avec son Mouvement en se déplaçant à travers l'unité qu'il a rejointe. Il peut éventuellement prendre la place d'autres figurines suivant la règle Au Premier Rang. Déplacez ces dernières aussi peu que possible, de manière à maintenir toutes les figurines dans des positions règlementaires. Si le Personnage n'a pas assez de mouvement pour atteindre la position souhaitée, il se déplace de la plus petite distance possible, à partir de sa position initiale, de manière à atteindre une position règlementaire, et ne peut alors faire se déplacer que des figurines sans la règle Au Premier Rang. Quand un Personnage rejoint une unité qui ne possède qu'un seul rang, son propriétaire peut choisir soit de déplacer une figurine au second rang, soit d'élargir le front de l'unité en plaçant la figurine remplacée à l'une des extrémités de l'unité.}

Une unité rejointe par un Personnage ne peut plus se déplacer au cours de la même étape des Autres Mouvements. Elle n'est cependant pas considérée comme ayant bougé (pour tirer, par exemple). Le Personnage lui-même compte quand même comme s'étant déplacé.

Un Personnage qui rejoint une unité est considéré comme faisant partie intégrante de celle-ci et suit toutes les règles auxquelles elle est astreinte.

\paragraph{Unité Combinée détruite}

Si toutes les figurines ordinaires d'une unité combinée sont tuées, laissant au moins un Personnage sans troupes, les Personnages survivants forment toujours une unité, qui est considérée comme étant la même unité qu'auparavant pour tous les effets en cours (comme les sorts de type \lastsoneturn{}) et pour la Panique. En effet, l'unité n'est pas considérée comme détruite, mais il se peut les Personnages aient à passer un test en raison de Lourdes Pertes.

\paragraph{Quitter une unité Combinée}
\label{leaving_a_combined_unit}

Un Personnage peut quitter une unité combinée au cours de l'étape des Autres Mouvements si l'unité est en mesure de se déplacer, c'est-à-dire si elle n'est pas engagée au Corps à Corps, ne s'est pas déjà déplacée au cours de cette étape, n'est pas en train de fuir, etc. Si vous désirez effectuer une Marche Forcée et qu'une unité ennemie est à moins de \distance{8}, passez un test de Marche Forcée pour l'ensemble de l'unité avant de bouger une figurine. Une unité quittée par un ou plusieurs Personnages n'est pas considérée comme ayant elle-même bougé (pour tirer, par exemple). Lorsqu'il quitte son unité, un Personnage peut se déplacer à travers elle et sortir par n'importe quel bord de l'unité. Il peut faire un mouvement de Vol s'il dispose de cette règle. Le Personnage compte comme faisant partie intégrante de l'unité jusqu'à ce qu'il l'ait physiquement quittée. Cela signifie qu'il est affecté par des éventuelles altérations du Mouvement touchant l'unité au cours de sa sortie. Si le Personnage ne possède pas assez de mouvement pour être placé à plus d'\distance{1} de l'unité, il ne peut pas la quitter. Un Personnage ne peut pas quitter et rejoindre la même unité durant la même phase de jeu.

\paragraph{Charger hors d'une unité}

Un Personnage peut aussi quitter une unité combinée en chargeant hors de celle-ci. Pour ceci, déclarez une charge avec le Personnage pendant l'étape de Déclaration des Charges, comme vous le feriez pour n'importe quelle unité. Si une telle manœuvre est réalisée, l'unité elle-même ainsi que d'éventuels autres Personnages dans l'unité ne peuvent pas déclarer de charge dans le même Tour de Joueur. \newfromWHB{Les tirs résultant d'une réaction à une charge de Personnage sortant d'une unité combinée sont tous alloués au Personnage.} Lorsqu'un Personnage charge hors d'une unité, il utilise sa propre valeur de Mouvement, peut utiliser un mouvement de Vol s'il dispose de la règle associée, et est affecté par des éventuelles altérations du Mouvement touchant l'unité pour son mouvement de charge. Si la charge est réussie, sortez le Personnage de l'unité et chargez normalement. Il ignore les figurines de l'unité qu'il quitte pour ce mouvement de charge. Si la charge est ratée, le Personnage réalise un déplacement de charge ratée qui le fait sortir de l'unité. Si ce mouvement est trop court pour positionner le Personnage à plus d'\distance{1} de son unité d'origine, il ne bouge pas, et l'unité est considérée comme ayant raté une charge.

\paragraph{Distribuer les touches sur une Unité Combinée}
\label{distributing_hits_at_combined_units}

Quand une attaque touche une unité combinée, il y a deux possibilités pour distribuer les touches :

\setlength\columnseprule{0.5pt}
\begin{multicols}{2}\raggedcolumns

\noindent
\begin{center}\textbf{Les Personnages sont du même Type de Troupe que l'unité}\end{center}
\begin{center}\textbf{- et -}\end{center}
\begin{center}\textbf{L'unité comprend au moins 5 figurines ordinaires}\end{center}

\noindent Toutes les touches sont distribuées sur les figurines ordinaires. Les Personnages ne peuvent pas subir de touches.

\noindent Quand un Gabarit touche un Personnage de l'unité, \newfromWHB{la touche est transférée à une figurine ordinaire}.

\columnbreak

\noindent
\begin{center}\textbf{Les Personnages sont d'un Type de Troupe différent de l'unité}\end{center}
\begin{center}\textbf{- ou -}\end{center}
\begin{center}\textbf{L'unité comprend au plus 4 figurines ordinaires}\end{center}

\noindent \newfromWHB{Le joueur attaquant distribue les touches sur les figurines ordinaires et les Personnages.} Ces touches doivent être distribuées le plus équitablement possible. Aucune figurine ne peut subir de deuxième touche tant que toutes les autres n'en ont pas subie une, et ainsi de suite.

\noindent Quand un Gabarit touche un Personnage de l'unité, il subit la touche normalement.

\end{multicols}
\setlength\columnseprule{0pt}

Notez que si une unité avec au moins 5 figurines ordinaires contient à la fois des Personnages du même Type de Troupe que l'unité et d'un Type de Troupe différent, les Personnages avec le même Type de Troupe que les figurines ordinaires ne peuvent pas recevoir de touches.

\newpage
\subsection{\newfromWHB{\frontrank}}
\label{front_rank}

Tous les Personnages et les figurines d'État-Major ont la règle \frontrank{}. Les figurines qui suivent cette règle doivent toujours être placées le plus à l'avant possible de leur unité. Normalement, cela signifie qu'elles doivent être placées au premier rang. Si le premier rang est déjà pleinement occupé par d'autres figurines avec cette règle, elles sont placées au second rang, et ainsi de suite.

Lorsqu'une unité avec des figurines qui suivent la règle \frontrank{} se déplace, ces figurines peuvent être réorganisées, tant que le nouvel agencement respecte la règle. Cela peut se faire au cours d'un Mouvement Simple, d'une Marche Forcée, d'une Roue ou d'une Reformation, et compte dans la limite de mouvement de l'unité. Mesurez entre la position initiale et la position d'arrivée de la figurine repositionnée pour déterminer de quelle distance elle s'est déplacée et vérifier qu'elle n'a pas dépassé son quota de mouvement.

Si une figurine avec la règle \frontrank{} quitte une unité ou est retirée comme perte, l'espace inoccupé qu'elle laisse doit être comblé par des figurines des autres rangs, si possible en déplaçant des figurines avec la règle \frontrank{} vers le front de l'unité. Si plus d'une figurine avec la règle \frontrank{} est éligible, le propriétaire des figurines décide laquelle de ces figurines est avancée. Si toutes les figurines avec la règle \frontrank{} sont déjà autant à l'avant que possible, comblez tout espace vide avec des figurines ordinaires des rangs arrières.

Parfois, on vous demandera de réorganiser les figurines avec la règle \frontrank{} de manière à ce qu'elles soient toutes autant à l'avant de l'unité que possible. Quand cela se produit, déplacez aussi peu de figurines que possible pour respecter la règle.

\paragraph{Socles Compatibles}

Si une figurine avec la règle \frontrank{} possède un socle de la même taille que ceux des figurines ordinaires de l'unité dans laquelle elle se trouve, ou que son socle est de la même taille qu'un multiple entier de ces socles, on dit que son socle est Combatible. Par exemple, un socle de \unit{40x40}{\milli\meter} dans une unité de figurines avec des socles de \unit{20x20}{\milli\meter} est un Socle Compatible. Le Personnage est alors placé dans l'unité normalement, en déplaçant le nombre nécessaire de figurines. Le front du socle du Personnage doit être placé le plus à l'avant possible de l'unité pour respecter la règle \frontrank{}. La figurine est considérée comme étant dans tous les rangs que son socle remplit. Pour calculer le nombre de figurines dans chaque rang de l'unité, pour le décompte des Rangs Complets ou pour obtenir la formation en Horde, considérez que la figurine compte comme autant de figurines ordinaires que son socle, plus grand, occupe. Une figurine ne peut pas rejoindre une unité ayant plus d'un rang et une largeur inférieure à celle de son socle. De même, une unité ne peut pas effectuer de Reformation qui lui donnerait des rangs moins larges que le socle d'un Personnage de cette unité.

Si une figurine a un Socle Compatible plus long que ceux des figurines ordinaires de son unité, l'unité est autorisée à avoir plus d'un Rang Incomplet, \textbf{à condition} que tous les Rangs Incomplets après le premier ne soient constitués de figurines avec des socles plus longs. Cela signifie que seules les parties arrières des socles plus longs peuvent former plusieurs Rangs Incomplets.

\paragraph{Socles Incompatibles}

Si, en revanche, une figurine avec la règle \frontrank{} ne possède pas un Socle Compatible, comme par exemple un socle de \unit{50x50}{\milli\meter} dans une unité de figurines sur des socles de \unit{20x20}{\milli\meter}, on parle de Socle Incompatible. Le Personnage doit être placé au contact d'un côté de l'unité, aligné avec le front de celle-ci. Seuls deux Personnages avec des Socles Incompatibles peuvent rejoindre une même unité, en les plaçant de chaque côté. Ces figurines sont considérées comme faisant partie du premier rang de l'unité, mais ne sont pas prises en compte dans le nombre de figurines des rangs, que ce soit pour obtenir un Rang Complet ou une formation en Horde.

La figure \ref{figure/front_rank} illustre un cas complexe d'agencement d'unité avec des Personnages.

\newcommand{\figFRFront}{Avant}
\newcommand{\figFRCharOne}{\normalfontsize$ P_{1} $}
\newcommand{\figFRCharTwo}{\normalfontsize$ P_{2} $}
\newcommand{\figFRCharThree}{\normalfontsize$ P_{3} $}
\newcommand{\figFRCharFour}{\normalfontsize$ P_{4} $}
\newcommand{\figFRChamp}{\normalfontsize\textit{Ch}}
\newcommand{\figFRStand}{\normalfontsize\textit{Ét}}
\newcommand{\figFRMus}{\normalfontsize\textit{Mu}}

\begin{figure}[!htbp]
\hypertarget{frontrankfigure}{
\begin{minipage}{0.3\textwidth}
\def\svgwidth{\textwidth}
\input{pics/front_rank.pdf_tex}
\end{minipage}
\hfill\begin{minipage}{0.57\textwidth}
\caption{Quelques illustrations de la règle \frontrank{}.\vspace*{10pt}\newline
Le Personnage $ P_{1} $ a un Socle Incompatible et est placé à côté de l'unité. Les Personnages $ P_{2} $ et $ P_{3} $ ont des Socles Compatibles et sont intégrés à l'unité, le plus à l'avant possible. Cette unité est considérée comme ayant 3 Rangs Complets : $ P_{1} $ n'est pas pris en compte, tandis que $ P_{2} $ compte comme 2 figurines de large.\vspace*{10pt}\newline
Quand le Personnage $ P_{4} $ rejoint l'unité, le Musicien (\textit{Mu}) doit être déplacé sur le côté de manière à avoir toutes les figurines avec la règle \frontrank{} le plus à l'avant de l'unité possible.}
\label{figure/front_rank}
\end{minipage}}
\end{figure}

\hypertarget{makeway}{\subsection{Faites Place}}
\label{make_way}

\newfromWHB{À la troisième étape d'une Manche de Corps à Corps (voir le paragraphe \ref{combat_round_sequence}, page \pageref{combat_round_sequence}), tout Personnage placé au premier rang et qui n'est pas en contact socle à socle avec une figurine ennemie peut être déplacé de manière à entrer en contact avec une figurine ennemie au contact de l'avant de l'unité du Personnage. Pour ceci, échangez les positions de ce Personnage avec une ou des figurines non-Personnage de son unité. Une figurine avec un Socle Incompatible ne peut jamais utiliser cette règle.}

\newpage
\hypertarget{thegeneral}{\section{Le Général}}
\label{thegeneral}

\subsection{Choisir le Général}
\label{choosing_the_general}

Toutes les armées doivent être dirigées par un Général. Le Général est le Personnage avec le plus haut Commandement de votre armée, sauf Porteur de la Grande Bannière et Personnages avec la règle \notaleader{}. Si plusieurs Personnages se disputent le plus haut Commandement, vous êtes libre de choisir lequel est votre Général\newfromWHB{, mais cela doit être mentionné dans votre Liste d'Armée}. Le Général de l'armée possède la règle \inspiringpresence{}.

\subsection{\inspiringpresence}

La figurine, si elle n'est pas en fuite, donne la possibilité à toutes les unités alliées à moins de \distance{12} d'utiliser son Commandement à la place du leur, si elles le souhaitent. Cette capacité suit les règles normales des Caractéristiques Empruntées (paragraphe \ref{borrowed_characteristics}, page \pageref{borrowed_characteristics}), donc les effets modifiant le Commandement du Général sont pris en compte avant qu'il soit transmis. Ce Commandement peut ensuite être modifié à nouveau par des effets affectant l'unité.

\hypertarget{thebsb}{\section{Le Porteur de la Grande Bannière}}
\label{thebsb}

\subsection{Choisir le Porteur de la Grande Bannière}

Certains Personnages peuvent être promus Porteur de la Grande Bannière. Cette possibilité apparait parmi les options dans leur entrée de Livre d'Armée. Une armée ne peut inclure qu'un seul Porteur de la Grande Bannière. Le Porteur de la Grande Bannière possède la règle \holdyourground{}.

\subsection{\holdyourground}

La figurine, si elle n'est pas en fuite, donne à toutes les unités alliées à moins de \distance{12} la capacité de relancer leurs tests de Commandement ratés. \newfromWHB{Elles n'y sont pas obligées.}

\subsection{Bannière Magique}

Si un Porteur de la Grande Bannière a la possibilité de prendre des Objets Magiques, il lui est permis d'acquérir une Bannière Magique. \newfromWHB{Cette Bannière Magique peut être soit comptabilisée dans sa limite de points d’Objets Magiques (habituellement 50 pts pour un héros, par exemple), soit prise sans limite de points. Dans ce dernier cas, il ne peut pas prendre d'autres Objets Magiques.}

\subsection{Leur Bannière est à Terre}

Quand un Porteur de Grande Bannière est retiré comme perte alors qu'il était engagé au Corps à Corps, on considère que la grande bannière est capturée par l'adversaire. \newfromWHB{Quand un Porteur de la Grande Bannière fuit un Corps à Corps après avoir raté un test de Moral, la grande bannière est perdue, avec toute Bannière Magique associée et la règle \holdyourground{}. Les effets de la grande bannière sont perdus et cette dernière est considérée comme capturée par l'adversaire.}

\hypertarget{challenges}{\section{Défis}}

\subsection{Lancer un Défi}

Les Personnages et les Champions engagés au Corps à Corps peuvent lancer un Défi. À la quatrième étape d'une Manche de Corps à Corps (voir le paragraphe \ref{combat_round_sequence}, page \pageref{combat_round_sequence}), le Joueur Actif peut désigner l'un de ses Personnages ou Champions et lancer un Défi avec ce dernier. S'il ne le fait pas, le Joueur Réactif peut à son tour lancer un Défi avec un Personnage ou Champion.

\subsection{Relever ou refuser un Défi}

Si un Défi a été lancé, le joueur adverse peut choisir l'un de ses propres Personnages ou Champions engagé dans le même Corps à Corps pour relever le Défi et combattre la figurine à l'origine du Défi. La figurine qui relève le Défi doit se trouver dans une unité en contact avec l'unité de la figurine qui l'a lancé.

Si un Défi n'est pas relevé, on le dit refusé. Dans ce cas, le joueur qui a lancé le Défi désigne l'un des Personnages de son adversaire qui aurait pu accepter le Défi, s'il y en a un. Les Champions ne peuvent pas être désignés. \newfromWHB{Le Commandement de ce Personnage est réduit à 0 et ni lui, ni son unité ne peuvent utiliser sa règle \stubborn{} (s'il en disposait) jusqu'à la fin du Tour de Joueur au cours duquel le combat se termine, ou jusqu'à ce que ce qu'il relève ou lance un Défi. Il ne peut par ailleurs porter aucune Attaque de Corps à Corps pendant cette Manche de Corps à Corps. Si c'est le Porteur de la Grande Bannière, il perd la règle \holdyourground{} et n'ajoute pas +1 au Résultat de Combat de son camp pour cette manche.}

\subsection{Relever un Défi}
\label{fighting_a_challenge}

Si un Défi est relevé, la figurine à l'origine du Défi et celle qui l'a relevé \newfromWHB{sont considérées comme étant au contact socle à socle, même si leurs socles ne se touchent pas physiquement}. Elles doivent allouer toutes leurs attaques contre leur vis-à-vis. Cela comprend les Attaques Spéciales réalisées normalement contre les unités, telles que le \stomp{}, l'\breathweapon{}, les \impacthits{} et les \grindingattacks{}, qui sont dirigées intégralement contre l'autre combattant. Dans le cas du \stomp{}, ce dernier doit être d'un Type de Troupe vulnérable au \stomp{}, sinon l'attaque est perdue. Aucune autre figurine ne peut allouer d'attaques contre les deux combattants. Aucune attaque ou touche ne peut être distribuée sur eux. Si un des combattants attaque sur plusieurs paliers d'Initiative (comme par exemple un cavalier et sa monture, ou une figurine qui possède la règle \stomp{}) et élimine son adversaire avant d'avoir pu porter toutes ses attaques, \newfromWHB{les attaques restantes peuvent être dirigées sur la figurine tuée, comme si elle était encore vivante et en contact socle à socle, de façon à obtenir des points de Carnage}.

Si l'une des deux figurines engagées dans le Défi est éliminée, fuit le combat ou si le combat s'achève pour une quelconque raison, le Défi est considéré comme terminé à la fin de la phase. Si aucune des figurines n'est tuée avant la prochaine Manche de Corps à Corps, le Défi continue. Aucun autre Défi ne peut être lancé dans le même combat tant que l'un des deux protagonistes du duel n'a pas été tué.

\subsection{Carnage}

Pendant un Défi, les blessures excédentaires comptent dans le Résultat de Combat, jusqu'à un maximum de +3.
