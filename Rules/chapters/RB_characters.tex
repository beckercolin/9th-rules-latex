
\part{Personnages}
\label{characters}

Hormis dans le cas où cela est spécifiquement mentionné, toute figurine inscrite dans la section des Héros ou des Seigneurs est un Personnage.

\section{Personnages isolés}

\newfromWHB{Un Personnage que vous souhaitez laisser seul peut être considéré comme une unité constituée d'une seule figurine. En conséquence, les règles normales des unités s'appliquent.}

\section{Rejoindre une unité}

Les Personnages peuvent faire partie intégrante des autres unités en les rejoignant. Cela peut être réalisé en déployant le Personnage dans une unité en début de partie ou en amenant le Personnage au contact de l'unité au cours de l'étape des Autres Mouvements. Les unités engagées au Corps à Corps ou en fuite ne peuvent pas être rejointes.

Les Personnages peuvent aussi rejoindre d'autres Personnages pour constituer une unité constituée exclusivement de Personnages.

Quand un Personnage rejoint une unité, il est immédiatement placé dans une position règlementaire pour suivre la règle \newfromWHB{Au Premier Rang} (voir le paragraphe \ref{front_rank}, page \pageref{front_rank}), entraînant éventuellement le déplacement d'autres figurines vers l'arrière. \newfromWHB{Lorsqu'un Personnage rejoint une unité, on peut le positionner librement sur tout emplacement règlementaire qu'il pourrait avoir atteint avec son Mouvement en se déplaçant à travers l'unité qu'il a rejointe. Il peut éventuellement prendre la place d'autres figurines suivant la règle Au Premier Rang. Déplacez ces dernières aussi peu que possible, de manière à maintenir toutes les figurines dans des positions règlementaires. Si le Personnage n'a pas assez de mouvement pour atteindre la position souhaitée, il se déplace de la plus petite distance possible, à partir de sa position initiale, de manière à atteindre une position règlementaire, et ne peut alors faire se déplacer que des figurines sans la règle Au Premier Rang. Quand un Personnage rejoint une unité qui ne possède qu'un seul rang, son propriétaire peut choisir soit de déplacer une figurine au second rang, soit d'élargir le front de l'unité en plaçant la figurine remplacée à l'une des extrémités de l'unité.}

Une unité rejointe par un Personnage ne peut plus se déplacer au cours de la même étape des Autres Mouvements. Elle n'est cependant pas considérée comme ayant bougé (pour tirer, par exemple). Le Personnage lui-même compte quand même comme s'étant déplacé.

Un Personnage qui rejoint une unité est considéré comme faisant partie intégrante de celle-ci et suit toutes les règles auxquelles elle est astreinte.

\noindent\textbf{Unité combinée détruite}

Si toutes les figurines ordinaires d'une unité combinée sont tuées, laissant au moins un Personnage sans troupes, les Personnages survivants forment toujours une unité, qui est considérée comme étant la même unité qu'auparavant pour tous les effets en cours, comme les sorts de type \lastsoneturn{}, et pour la Panique. En effet, l'unité n'est pas considérée comme détruite, mais il se peut les Personnages aient à passer un test en raison de Lourdes Pertes.

\noindent\textbf{Quitter une unité combinée}

Un Personnage peut quitter une unité combinée au cours de l'étape des Autres Mouvements si l'unité est en mesure de se déplacer, c'est-à-dire si elle n'est pas engagée au Corps à Corps, ne s'est pas déjà déplacée au cours de cette étape, n'est pas en train de fuir, etc. Passez un test de Marche Forcée pour l'ensemble de l'unité avant de bouger toute figurine si vous désirez en faire une. Une unité quittée par un ou plusieurs Personnages n'est pas considérée comme ayant elle-même bougé (pour tirer, par exemple). Lorsqu'il quitte son unité, un Personnage peut se déplacer à travers elle et sortir par n'importe quel bord de l'unité. Il peut faire un mouvement de Vol s'il dispose de cette règle. Le Personnage compte comme faisant partie intégrante de l'unité jusqu'à ce qu'il l'ait physiquement quittée. Cela signifie qu'il est affecté par des éventuelles altérations du Mouvement touchant l'unité au cours de sa sortie. Si le Personnage ne possède pas assez de mouvement pour être placé à plus d'\distance{1} de l'unité, il ne peut pas la quitter. Un Personnage ne peut pas quitter et rejoindre la même unité durant la même phase de jeu.

\noindent\textbf{Charger hors d'une unité}

Un Personnage peut aussi quitter une unité combinée en chargeant hors de celle-ci. Pour ceci, déclarez une charge avec le Personnage pendant l'étape de Déclaration des Charges, comme vous le feriez pour n'importe quelle unité. Si une telle manœuvre est réalisée, l'unité elle-même ainsi que d'éventuels autres Personnages dans l'unité ne peuvent pas déclarer de charge dans le même Tour de Joueur. \newfromWHB{Les tirs résultant d'une réaction à une charge de Personnage sortant d'une unité combinée sont tous alloués au Personnage.} Lorsqu'un Personnage charge hors d'une unité, il utilise sa propre valeur de Mouvement, peut utiliser un mouvement de Vol s'il dispose de la règle associée, et est affecté par des éventuelles altérations du Mouvement touchant l'unité pour son mouvement de charge. Si la charge est réussie, sortez le Personnage de l'unité et chargez normalement. Il ignore les figurines de l'unité qu'il quitte pour ce mouvement de charge. Si la charge est ratée, le Personnage réalise un déplacement de charge ratée qui le fait sortir de l'unité. Si ce mouvement est trop court pour positionner le Personnage à plus d'\distance{1} de son unité d'origine, il ne bouge pas, et l'unité est considérée comme ayant raté une charge.

\noindent\textbf{Distribuer les touches sur une unité combinée}

Quand une attaque touche une unité combinée, il y a deux possibilités pour distribuer les touches :

\setlength\columnseprule{0.5pt}
\begin{multicols}{2}\raggedcolumns

\noindent
\begin{center}\textbf{Les Personnages sont du même Type de Troupe que l'unité}\end{center}
\vspace*{-0.8cm}\begin{center}\textbf{- et -}\end{center}
\vspace*{-0.8cm}\begin{center}\textbf{L'unité comprend au moins 5 figurines ordinaires}\end{center}

\noindent Toutes les touches sont distribuées sur les figurines ordinaires. Les Personnages ne peuvent pas subir de touches.

\noindent Quand un Gabarit touche un Personnage de l'unité, \newfromWHB{la touche est transférée à une figurine ordinaire}.

\columnbreak

\noindent
\begin{center}\textbf{Les Personnages sont d'un Type de Troupe différent de l'unité}\end{center}
\vspace*{-0.8cm}\begin{center}\textbf{- ou -}\end{center}
\vspace*{-0.8cm}\begin{center}\textbf{L'unité comprend 4 figurines ordinaires ou moins}\end{center}

\noindent \newfromWHB{Le joueur attaquant distribue les touches sur les figurines ordinaires et les Personnages.} Ces touches doivent être distribuées le plus équitablement possible. Aucune figurine ne peut subir de deuxième touche tant que toutes les autres n'en ont pas subie une, et ainsi de suite.

\noindent Quand un Gabarit touche un Personnage de l'unité, il subit la touche normalement.

\end{multicols}
\setlength\columnseprule{0pt}

Notez que si une unité avec au moins 5 figurines ordinaires contient à la fois des Personnages du même Type de Troupe que l'unité et d'un Type de Troupe différent, les Personnages avec le même Type de Troupe que les figurines ordinaires ne peuvent pas recevoir de touches.

\newpage
\subsection{\newfromWHB{Au premier rang}}
\label{front_rank}

Tous les Personnages et les figurines d'État-Major ont la règle \frontrank{}. Les figurines qui suivent cette règle doivent toujours être placées le plus à l'avant possible de leur unité. Normalement, cela signifie qu'elles doivent être placées au premier rang. Si le premier rang est déjà pleinement occupé par d'autres figurines avec cette règle, elles sont placées au second rang, et ainsi de suite.

Lorsqu'une unité avec des figurines qui suivent la règle \frontrank{} se déplace, ces figurines peuvent être réorganisées, tant que le nouvel agencement respecte la règle. Cela peut se faire au cours d'un Mouvement Simple, d'une Marche Forcée, d'une Roue ou d'une Reformation, et compte dans la limite de mouvement de l'unité. Mesurez entre la position initiale et la position d'arrivée de la figurine repositionnée pour déterminer de quelle distance elle s'est déplacée et vérifier qu'elle n'a pas dépassé son quota de mouvement.

Si une figurine avec la règle \frontrank{} quitte une unité ou est retirée comme perte, l'espace inoccupé qu'elle laisse doit être comblé par des figurines des autres rangs, si possible en déplaçant des figurines avec la règle \frontrank{} vers le front de l'unité. Si plus d'une figurine avec la règle \frontrank{} est éligible, le propriétaire des figurines décide laquelle de ces figurines est avancée. Si toutes les figurines avec la règle \frontrank{} sont déjà autant à l'avant que possible, comblez tout espace vide avec des figurines ordinaires des rangs arrières.

Parfois, on vous demandera de réorganiser les figurines avec la règle \frontrank{} de manière à ce qu'elles soient toutes autant à l'avant de l'unité que possible. Quand cela se produit, déplacez aussi peu de figurines que possible pour respecter la règle. Voir la figure \ref{figure/front_rank} pour un exemple d'organisation d'unité avec des socles compatibles et incompatibles.

\noindent\textbf{Socles Compatibles}

Si une figurine avec la règle \frontrank{} possède un socle de la même taille que ceux des figurines ordinaires de l'unité dans laquelle elle se trouve, ou que son socle est de la même taille qu'un multiple entier de ces socles, on dit que son socle est Combatible. Par exemple, un socle de \unit{40x40}{\milli\meter} dans une unité de figurines avec des socles de \unit{20x20}{\milli\meter} est un Socle Compatible. Le Personnage est alors placé dans l'unité normalement, en déplaçant le nombre nécessaire de figurines. Le front du socle du Personnage doit être placé le plus à l'avant possible de l'unité pour respecter la règle \frontrank{}. La figurine est considérée comme étant dans tous les rangs que son socle remplit. Pour calculer le nombre de figurines dans chaque rang de l'unité, pour le décompte des Rangs Complets ou pour obtenir la formation en Horde, considérez que la figurine compte comme autant de figurines ordinaires que son socle, plus grand, occupe. Une figurine ne peut pas rejoindre une unité ayant plus d'un rang et une largeur inférieure à celle de son socle. De même, une unité ne peut pas effectuer de Reformation qui lui donnerait des rangs moins larges que le socle d'un Personnage de cette unité.

Si une figurine a un Socle Compatible plus long que ceux des figurines ordinaires de son unité, l'unité est autorisée à avoir plus d'un Rang Incomplet, \textbf{à condition} que tous les Rangs Incomplets après le premier ne soient constitués de figurines avec des socles plus longs. Cela signifie que seules les parties arrières des socles plus longs peuvent former plusieurs Rangs Incomplets.

\noindent\textbf{Socles Incompatibles}

%Si, en revanche, une figurine avec la règle \emph{Au Premier Rang} ne possède pas un socle compatible, comme par exemple un socle de 50x50{\milli\meter} dans une unité de figurines sur des socles de 20x20{\milli\meter}, le socle est dit \textbf{incompatible} et le \emph{Personnage} est placé sur le côté de l'unité, son socle en contact avec celui des figurines de l'unité, et aligné avec le front de celle-ci. Un maximum de deux \emph{Personnages} avec des socles incompatibles peut rejoindre une même unité, en les plaçant de chaque côté. Ces figurines sont considérées comme faisant partie du premier rang de l'unité, mais ne sont pas prises en compte lorsqu'on compte le nombre de figurines dans chaque rang, pour le décompte des \emph{Rangs Complets} ou pour obtenir la formation en \emph{Horde}.
%
%\begin{figure}[!htbp]
%\centering
%\def\svgwidth{12cm}
%\input{au_premier_rang.pdf_tex}
%\caption{Le \emph{Personnage} $ P_{1} $ a un socle incompatible et est placé à côté de l'unité. Les \emph{Personnages} $ P_{2} $ et $ P_{3} $ ont des socles compatibles et sont placés à l'intérieur de l'unité, le plus à l'avant possible. Cette unité est considérée comme ayant 3 \emph{Rangs Complets} : $ P_{1} $ n'est pas pris en compte, alors que $ P_{2} $ compte comme 2 figurines de large. \\
%Quand le \emph{Personnage} $ P_{4} $ rejoint l'unité, le musicien (Mu) doit être déplacé sur le côté de manière à avoir toutes les figurines qui suivent la règle spéciale \emph{Au Premier Rang} le plus sur le devant de l'unité possible.}
%\label{figure/front_rank}
%\end{figure}

\subsection{Faites place}

%\nouveau{À la troisième étape d'une manche de corps à corps (voir le paragraphe \ref{etapes_manche_cac}), tout \emph{Personnage} placé au premier rang et qui n'est pas en contact socle à socle avec une unité ennemie peut être déplacé en contact avec un ennemi qui est en contact socle à socle avec l'avant de l'unité du \emph{Personnage}. Pour ceci, échangez les positions de ce \emph{Personnage} avec une ou des figurines non-\emph{Personnage} de son unité. Une figurine avec un socle incompatible ne peut jamais opérer un tel mouvement.}


\section{Le général}

\subsection{Choisir le général de l'armée}

%Toutes les armées doivent avoir un \emph{Général} pour les diriger. Il doit être le \emph{Personnage} avec le plus haut Commandement dans votre armée ayant la possibilité d'être le \emph{Général}, hormis le \emph{Porteur de la Grande Bannière}. Si deux \emph{Personnages} ou plus se disputent le plus haut Commandement, vous êtes libre de choisir lequel des \emph{Personnages} est votre \emph{Général}, \nouveau{mais cela doit être mentionné dans votre liste d'armée}. Le \emph{Général} de l'armée possède la règle \emph{Présence Charismatique}.

\subsection{Présence charismatique}

%Une figurine qui n'est pas en fuite avec la règle \emph{Présence Charismatique} donne son Commandement à toutes les unités alliées dans un rayon de 12{\pouce}. Les unités qui bénéficient de la \emph{Présence Charismatique} peuvent utiliser ce Commandement au lieu de leur propre Commandement, si elles le souhaitent. Les effets modifiant le Commandement du \emph{Général} sont pris en compte au moment où celui-ci transmet son Commandement.

\section{Le porteur de la grande bannière}

\subsection{Choisir le porteur de la grande bannière}

%Certains \emph{Personnages} peuvent être promus \emph{Porteur de la Grande Bannière} parmi les options de leur livre d'armée. Une armée ne peut inclure qu'un seul \emph{Porteur de la Grande Bannière}. Le \emph{Porteur de la Grande Bannière} de l'armée possède la règle \emph{Tenez les Rangs}.

\subsection{Tenez les rangs}

%Un Porteur de la Grande Bannière qui n'est pas en fuite et qui a la règle \emph{Tenez les Rangs} permet aux unités alliées se trouvant dans un rayon de 12{\pouce} de relancer leurs tests de Commandement ratés. \nouveau{Elles n'y sont pas obligées}.

\subsection{Bannière magique}

%Si un \emph{Porteur de la Grande Bannière} a la possibilité de choisir des \emph{Objets Magiques}, il lui est permis d'acquérir une bannière magique. \nouveau{Cette bannière magique peut être soit comptabilisée comme faisant partie des objets magiques acquis par la figurine, dans la limite des points d’\emph{Objets Magiques} qui lui est autorisée (habituellement 50 pts pour un héros, par exemple), soit elle peut être prise sans aucune limitation de points, mais dans ce cas la figurine ne peut prendre aucun autre \emph{Objet Magique}}.

\subsection{Leur bannière est à terre}

%Quand un \emph{Porteur de Grande Bannière} est retiré comme perte alors qu'il était engagé au corps à corps, on considère que la grande bannière est capturée par l'adversaire. \nouveau{Quand un \emph{Porteur de la Grande Bannière} fuit un corps à corps, on considère que la grande bannière est perdue, avec toute bannière magique associée et la règle \emph{Tenez les Rangs}, et la figurine qui était en possession de la grande bannière perd tous les effets bénéfiques engendrés par celle-ci. Elle est alors considérée comme capturée par l'armée adverse}.

\section{Défis}

\subsection{Lancer un défi}

%Les \emph{Personnages} ou \emph{Champions} engagés dans un combat peuvent lancer un \emph{Défi}. À la quatrième étape d'une manche de corps à corps (voir \ref{etapes_manche_cac}), le joueur actif peut désigner l'un de ses \emph{Personnages} ou \emph{Champions} et lancer un \emph{Défi} avec ce dernier. S'il n'a pas relevé de \emph{Défi}, le joueur réactif peut alors désigner l'un de ses \emph{Personnages} ou \emph{Champions} et lancer lui-même un \emph{Défi}.

\subsection{Accepter ou refuser un défi}

%Si un \emph{Défi} est lancé, le joueur adverse peut choisir l'un de ses propres \emph{Personnages} ou \emph{Champions} engagés dans le même combat pour relever le \emph{Défi} et combattre le \emph{Personnage} ou \emph{Champion} qui l'a lancé. La figurine qui accepte le \emph{Défi} doit se trouver dans une unité qui est en contact avec la figurine qui l'a lancé, \newrule{ou son unité}.
%
%Si un \emph{Défi} n'est pas relevé, ce \emph{Défi} est dit refusé. Dans ce cas, le joueur qui a lancé le \emph{Défi} désigne l'un des \emph{Personnages} non-\emph{Champion} de son adversaire qui aurait pu accepter le \emph{Défi} (s'il y en a un, bien sûr). \nouveau{Le Commandement de cette figurine est réduit à 0 et elle ne plus utiliser la règle spéciale \emph{Tenace} (notez que son unité ne peut plus utiliser cette règle non plus si c'est le \emph{Personnage} qui lui transmet) jusqu'à la fin du tour au cours duquel le combat se termine, ou jusqu'à ce que ce \emph{Personnage} relève ou lance un \emph{Défi}. Elle ne peut par ailleurs faire aucune attaque de corps à corps au cours de cette manche de corps à corps. Si c'est le \emph{Porteur de la Grande Bannière}, il perd la règle \emph{Tenez les Rangs} et n'ajoute pas +1 au résultat de combat au cours de cette \emph{Phase de Corps à Corps}}.

\subsection{Relever un défi}

%Si un \emph{Défi} est relevé, le \emph{Personnage} ou \emph{Champion} qui lance le \emph{Défi} et le \emph{Personnage} ou \emph{Champion} qui relève le \emph{Défi} \nouveau{sont considérés comme étant au contact socle à socle, même si leurs socles ne sont pas physiquement en contact}, et doivent diriger toutes leurs attaques contre leur vis-à-vis. Les attaques spéciales réalisées contre une unité, telles que les \emph{Piétinements}, les \emph{Attaques de Souffle}, les \emph{Touches d’Impact} et les \emph{Attaques de Broyage}, sont toujours dirigées contre le \emph{Personnage} ou \emph{Champion} engagé dans le duel. Dans le cas des \emph{Piétinements}, le \emph{Personnage} ou \emph{Champion} ciblé doit bien sûr appartenir à un \emph{Type de Troupe} qui peut  être piétiné et, dans le cas contraire, les attaques de \emph{Piétinements} sont perdues. Aucune autre figurine ne peut diriger des attaques contre les deux duellistes. Aucune attaque ou touche réalisée en dehors du \emph{Défi} ne peut être attribuée à l'une des deux figurines engagées dans un \emph{Défi}. Si un \emph{Personnage} ou \emph{Champion} avec des attaques étalées sur plusieurs paliers d'Initiative, comme par exemple un cavalier et sa monture, ou une figurine qui possède la règle spéciale \emph{Piétinement}, élimine son adversaire avant d'avoir pu faire toutes ses attaques, \nouveau{ces attaques restantes peuvent être dirigées sur la figurine tuée, comme si elle était encore vivante et en contact socle à socle, de façon à obtenir des points de \emph{Carnage}}.
%
%Si l'une des deux figurines engagées dans un \emph{Défi} est éliminée, fuit le combat \newrule{ou si le combat s'achève pour une quelconque raison}, le \emph{Défi} est considéré comme terminé à la fin de la phase. Si aucune des figurines n'est tuée avant la prochaine manche de corps à corps, le \emph{Défi} continue. Aucun autre \emph{Défi} ne peut être lancé dans le même combat jusqu'à ce que l'un des deux protagonistes du duel soit tué.

\subsection{Carnage}

%Pendant un \emph{Défi}, toutes les blessures en excédent sont comptabilisées dans le résultat de combat, jusqu'à un maximum de \nouveau{+3}.
