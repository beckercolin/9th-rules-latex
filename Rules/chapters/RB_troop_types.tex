
\hypertarget{trooptypes}{\part{Types de Troupe}}
\label{troop_types}

Toutes les figurines et unités sont définies par un Type de Troupe. Le Type de Troupe d'une unité est celui de la plus grande fraction de ses figurines ordinaires, ou de l'ensemble de ses figurines s'il n'y a pas de figurines ordinaires. L'adversaire du propriétaire choisit dans le cas de plusieurs fractions égales. Chaque Type de Troupe est associé à un certain nombre de règles décrites ci-dessous.

\renewcommand{\arraystretch}{1.2}
\newcommand{\trooptypestarttable}{\noindent\begin{tabular}{p{4.8cm}p{10.8cm}}}

\trooptypestarttable
\textbf{\infantry} & \tabularnewline
Règles Spéciales : & \lighttroops{} (Personnage uniquement). \tabularnewline
\end{tabular}

\trooptypestarttable
\textbf{\warbeast} & \tabularnewline
Règles Spéciales : & \swiftstride{}. \tabularnewline
\end{tabular}

\trooptypestarttable
\textbf{\cavalry} & \tabularnewline
Règles de Type de Troupe : & \combinedprofile{}, \cavalrysupport{}. \tabularnewline
Règles Spéciales : & \swiftstride{}. \tabularnewline
\end{tabular}

\trooptypestarttable
\textbf{\monstrousinfantry} & \tabularnewline
Règles de Type de Troupe : & \monstrousranks{}, \monstroussupport{}. \tabularnewline
Règles Spéciales : & \stomp{1}. \tabularnewline
\end{tabular}

\trooptypestarttable
\textbf{\monstrousbeast} & \tabularnewline
Règles de Type de Troupe : & \monstrousranks{}, \monstroussupport{}. \tabularnewline
Règles Spéciales : & \swiftstride{}, \stomp{1}. \tabularnewline
\end{tabular}

\trooptypestarttable
\textbf{\monstrouscavalry} & \tabularnewline
Règles de Type de Troupe : & \combinedprofile{}, \monstrousranks{}, \cavalrysupport{}, \monstroussupport{}. \tabularnewline
Règles Spéciales : & \swiftstride{}. \tabularnewline
\end{tabular}

\trooptypestarttable
\textbf{\chariot} & \tabularnewline
Règles de Type de Troupe : & \combinedprofile{}, \monstrousranks{}, \cavalrysupport{}. \tabularnewline
Règles Spéciales : & \swiftstride{}, \cannotmarch{}, \impacthits{1D6}. \tabularnewline
\end{tabular}

\trooptypestarttable
\textbf{\monster} & \tabularnewline
Règles de Type de Troupe : & \monsterranks{}. \tabularnewline
Règles Spéciales : & \stomp{1D6}, \toweringpresence{}, \terror{}. \tabularnewline
\end{tabular}

\trooptypestarttable
\textbf{\riddenmonster} & \tabularnewline
Règles de Type de Troupe : & Profil de \riddenmonster{}, \monsterranks{}. \tabularnewline
Règles Spéciales : & \stomp{1D6}, \toweringpresence{}, \terror{}. \tabularnewline
\end{tabular}

\trooptypestarttable
\textbf{\swarm} & \tabularnewline
Règles Spéciales : & \immunetopsychology{}, \unstable{}, \skirmisher{}. \tabularnewline
\end{tabular}

\trooptypestarttable
\textbf{\warmachine} & \tabularnewline
Règles de Type de Troupe : & Profil de \warmachine{}. \tabularnewline
Règles Spéciales : & \moveorfire{}, \cannotmarch{}, \reload{}. \tabularnewline
\end{tabular}
\renewcommand{\arraystretch}{1.5}

\section{Figurines à pied ou montées}

Certains sorts et règles peuvent affecter différemment les figurines montées et les figurines à pied. L'Infanterie, les Bêtes de Guerre, l'Infanterie Monstrueuse, les Bêtes Monstrueuses, les Monstres, les Nuées et les Machines de Guerre sont considérés comme des figurines \textbf{à pied}, à moins d'être des montures (comme un Palanquin), auquel cas ce sont des figurines montées. La \cavalry{}, la Cavalerie Monstrueuse, les Monstres Montés et les Chars sont considérés comme des figurines \textbf{montées}.

\subsection{Montures de Personnage}
\label{character_mounts}

Beaucoup de Personnages ont la possibilité de choisir une monture dans une liste fournie dans leur Livre d'Armée. Quand une figurine est montée, son Type de Troupe change :
\begin{itemize}[label={-}]
\item Une figurine montée sur une \textbf{\warbeast} devient de la \textbf{\cavalry}.
\item Une figurine montée sur une \textbf{\monstrousbeast} devient de la \textbf{\monstrouscavalry}.
\item Une figurine montée sur un \textbf{\monster} devient un \textbf{\riddenmonster}.
\item Une figurine montée sur n'importe quel autre Type de Troupe prend le Type de Troupe de sa monture. Elle compte alors comme une figurine montée et gagne les règles Profil Combiné et \cavalrysupport{}.
\end{itemize}

Tous les éléments de la figurine combinée suivent les règles des Personnages : Personnage, monture, et éventuellement cavaliers ou membres d'équipage supplémentaires. Les cavaliers ou membres d'équipage appartenant à la monture choisie ne comptent pas comme des \og montures \fg{}, ce qui signifie qu'ils peuvent par exemple utiliser leurs armes ou armures. Ils ont leur propre équipement qu'eux seuls peuvent utiliser. Cependant, la figurine combinée n'a qu'une seule Sauvegarde d'Armure. Il faut choisir d'utiliser l'Armure d'un des cavaliers ou membres d'équipage ou l'Armure du Personnage. Quel que soit le choix, cela inclut le \barding{} et la \mountsprotection{}. Rappel : Dans le cas d'un \riddenmonster{}, vous ne pouvez utiliser que la Sauvegarde d'Armure de la monture.

\section{Règles des Types de Troupe}

\paragraph{\monsterranks}

L'unité n'a besoin que d'une seule figurine pour former un Rang Complet.

\paragraph{\monstrousranks}

L'unité a seulement besoin d'un minimum de trois figurines pour former un Rang Complet et de six figurines par rang pour obtenir une formation de Horde.

\paragraph{\cavalrysupport}

Les montures ne peuvent pas porter d'Attaque de Soutien.

\paragraph{\monstroussupport}

La figurine peut porter jusqu'à trois Attaques de Soutien au lieu d'une seule.

\newpage
\paragraph{\combinedprofile}

La figurine a des profils différents pour chacun de ses éléments. Une figurine de \cavalry{} a généralement un profil pour la monture et un profil pour le cavalier. Une figurine de \chariot{} a généralement un profil pour le châssis, un profil pour les membres d'équipage et un profil pour les montures. Certains Chars n'ont qu'un seul profil pour le châssis et les montures.

\textbf{Quand il attaque ou tire}, chaque élément de la figurine utilise ses propres Caractéristiques. Chaque élément peut faire une Attaque de Tir dans la même phase, mais ils doivent tous tirer sur la même cible. Un membre de l'équipage d'un \chariot{} peut choisir de tirer avec une arme transportée par le \chariot{} à la place de son arme personnelle.

\textbf{Dans toutes les autres situations}, la figurine est considérée comme une seule figurine avec un unique ensemble de Caractéristiques défini ci-dessous. Ignorez les Caractéristiques des autres profils, sauf pour les tests de Caractéristique, pour lesquels on doit utiliser la plus haute valeur parmi tous les profils de la figurine.

\vspace*{10pt}
\renewcommand{\arraystretch}{2.5}
\begin{center}
\begin{tabular}{>{\bfseries\raggedleft}p{3cm}p{12cm}}
\hline
Mouvement & Utilisez le Mouvement de la monture ou de l'attelage. \tabularnewline
Capacité de Combat & Utilisez la CC du cavalier ou la plus haute CC parmi les cavaliers ou membres d'équipage s'il y en a plusieurs. \tabularnewline
Endurance & Utilisez la plus haute Endurance parmi tous les profils. \tabularnewline
Points de Vie & Utilisez la plus haute valeur de PV parmi tous les profils. Quand la figurine tombe à 0 PV, retirez la figurine entière comme perte. \tabularnewline
Commandement & Utilisez le Cd du cavalier ou le plus haut Cd parmi les cavaliers ou membres d'équipage s'il y en a plusieurs. \tabularnewline
Sauvegardes & Utilisez la meilleure valeur de Sauvegarde d'Armure, de \regeneration{} et de \wardsave{} disponible parmi celles des éléments de la figurine. La Sauvegarde d'Armure d'un cavalier peut être améliorée par un \barding{} et la règle \mountsprotection{}. \tabularnewline
\hline
\end{tabular}\end{center}
\renewcommand{\arraystretch}{1.5}

\newpage
\paragraph{Profil de Monstre Monté}
\label{ridden_monster_profile}

La figurine a un profil pour le Monstre et un profil pour le ou les cavaliers ou membres d'équipage. Les attaques qu'elle subit sont toujours résolues contre le Monstre.

\textbf{Quand il attaque ou tire}, chaque élément de la figurine utilise ses propres Caractéristiques. Chaque élément peut faire une Attaque de Tir dans la même phase, mais ils doivent tous tirer sur la même cible. Un membre d'équipage ou un cavalier peut choisir de tirer avec une arme transportée par le Monstre à la place de son arme personnelle.

\textbf{Dans toutes les autres situations}, la figurine est considérée comme une seule figurine avec un unique ensemble de Caractéristiques défini ci-dessous. Ignorez les Caractéristiques des autres profils, sauf pour les tests de Caractéristique, pour lesquels on doit utiliser la plus haute valeur parmi tous les profils de la figurine.

\vspace*{10pt}
\renewcommand{\arraystretch}{2.5}
\begin{center}
\begin{tabular}{>{\bfseries\raggedleft}p{3cm}p{12cm}}
\hline
Mouvement & Utilisez le Mouvement du Monstre. \tabularnewline
Capacité de Combat & Utilisez la CC du Monstre. \tabularnewline
Endurance & Utilisez l'Endurance du Monstre. \tabularnewline
Points de Vie & Utilisez les PVs du Monstre. Quand le Monstre tombe à 0 PV, retirez la figurine entière comme perte. \tabularnewline
Commandement & Utilisez le Cd du cavalier ou le plus haut Cd parmi les cavaliers ou membres d'équipage s'il y en a plusieurs. \tabularnewline
Sauvegardes & Utilisez les sauvegardes du Monstre uniquement. Toute pièce d'Armure ou \regeneration{} et \wardsave{} du cavalier n'a aucun effet, à moins que le contraire ne soit précisé. Les Monstres Montés ne peuvent bénéficier que du type d'armure \innatedefence{}. \tabularnewline
\hline
\end{tabular}\end{center}
\renewcommand{\arraystretch}{1.5}

\newpage
\paragraph{Profil de Machine de Guerre}

Une Machine de Guerre est considérée comme une figurine unique avec un seul ensemble de Caractéristiques défini ci-dessous :

\vspace*{10pt}
\renewcommand{\arraystretch}{2.5}
\begin{center}
\begin{tabular}{>{\bfseries\raggedleft}p{4.5cm}p{10.5cm}}
\hline
Endurance et Points de Vie & Utilisez l'Endurance et les PVs de la Machine. \tabularnewline
Autres Caractéristiques et sauvegardes & Utilisez les Caractéristiques, l'Armure et les autres sauvegardes des servants. \tabularnewline
\hline
\end{tabular}\end{center}
\renewcommand{\arraystretch}{1.5}
\vspace*{10pt}

Les Machines de Guerre ont des profils différents pour la Machine et ses servants. Utilisez l'Endurance et les PVs de la Machine. Prenez les Caractéristiques des servants pour le reste, en utilisant les règles de Profil Combiné dans le cas de plusieurs éléments de figurine pour les servants. Les Machines de Guerre ratent automatiquement tout test de Caractéristique à l'exception des tests de Commandement et ne peuvent pas effectuer de Marche Forcée, déclarer des charges, poursuivre ou choisir de fuir en réaction à une charge. Si une Machine de Guerre rate un test de Panique, elle ne fuit pas, mais ne peut pas tirer lors de sa prochaine Phase de Tir. Les Personnages ne peuvent jamais rejoindre une unité de Machine de Guerre.

Quand une unité charge une Machine de Guerre, son front peut être amené au contact de n'importe quel point du socle de la Machine. Aucun mouvement d'alignement n'est autorisé. Ignorez l'orientation de la Machine de Guerre (elle n'en a pas puisqu'elle a un socle rond) ainsi que la maximisation du nombre de figurines au contact socle à socle.

Au début de chaque Manche de Corps à Corps avec une Machine de Guerre, le joueur attaquant choisit autant de figurines alliées engagées que possible, jusqu'à un maximum de 6, qui ne sont pas en contact avec d'autres ennemis que la Machine. Les Chars, l'Infanterie Monstrueuse, les Bêtes Monstrueuses et la Cavalerie Monstrueuse comptent pour 3 figurines chacun, et les Monstres et Monstres Montés comptent comme 6 figurines chacun. Les figurines choisies sont les seules à être considérées comme étant en contact avec la Machine pour le reste de la Manche de Corps à Corps et touchent automatiquement avec leurs Attaques normales de Corps à Corps. Les autres figurines ne peuvent ni allouer ni répartir d'attaques à la Machine.

Une Machine de Guerre qui rate un test de Moral est détruite automatiquement. Les Machines de Guerre et les unités engagées au Corps à Corps contre elles ne peuvent pas effectuer de Reformation de Combat.

\newpage
\section*{Résumé des Types de Troupe}

\renewcommand{\arraystretch}{2.4}
\begin{center}
\begin{tabular}{@{}>{\bfseries}M{2.5cm}|M{2cm}M{2.4cm}M{2.4cm}M{3cm}M{2cm}@{}}
 & \textbf{Type de Profil} & \textbf{Rang Complet (Horde)} & \textbf{Soutien} & \textbf{Règles Spéciales} & \textbf{Taille**} \tabularnewline
 \hline
\infantry{} & - & 5 (10) & 1 & \lighttroops{} \newline (Personnage uniquement) & Standard \tabularnewline

\warbeast{} & - & 5 (10) & 1 & \swiftstride{} & Standard \tabularnewline

\cavalry{} & \combinedprofile{} & 5 (10) & 1 (cavalier uniquement) & \swiftstride{} & Grande \tabularnewline

\monstrousinfantry{} & - & 3 (6) & jusqu'à 3 & \stomp{1} & Grande \tabularnewline

\monstrousbeast{} & - & 3 (6) & jusqu'à 3 & \swiftstride{}\newline \stomp{1} & Grande \tabularnewline

\monstrouscavalry{} & \combinedprofile{} & 3 (6) & jusqu'à 3 (cavalier uniquement) & \swiftstride{}\newline \stomp{1} & Grande \tabularnewline

\chariot{} & \combinedprofile{} & 3 (6) & 1 (1 membre d'équipage uniquement) & \swiftstride{} \newline \cannotmarch{} \newline \impacthits{1D6} & Grande \tabularnewline

\monster{} & - & 1 & - & \stomp{1D6} \newline \toweringpresence{} \newline \terror & Gigantesque \tabularnewline

\riddenmonster{} & Profil de Monstre Monté & 1 & - & \stomp{1D6} \newline \toweringpresence{} \newline \terror & Gigantesque \tabularnewline

\swarm{} & - & * (10) & 1 & \immunetopsychology{} \newline \unstable{} \newline \skirmisher{} & Standard \tabularnewline

\warmachine{} & Profil de Machine de Guerre & Pas de rangs & - & \moveorfire{} \newline \cannotmarch{} \newline \reload{} & Standard \tabularnewline
\end{tabular}
\end{center}
\noindent * Les Nuées ne peuvent pas avoir de Rang Complet puisqu'elles ont la règle \skirmisher{}.\newline
\noindent ** Une figurine avec la règle \toweringpresence{} est de Taille Gigantesque.
\renewcommand{\arraystretch}{1.5}