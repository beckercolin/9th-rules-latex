
\part{Introduction}

\section{Qu'est-ce que BF : Le 9\ieme{} Âge ?}

Batailles Fantastiques : Le 9\ieme{} Âge, souvent raccourci en Le 9\ieme{} Âge ou T9A pour les anglophones, est un jeu de figurines créé par la communauté des joueurs. Il met en scène deux armées qui s'affrontent dans un fracas épique pour le pouvoir ou la survie. Chaque armée peut être composée de simples soldats à pied, d'archers talentueux, de chevaliers en armure, de puissants sorciers, de héros légendaires, de monstres terrifiants, de gigantesques dragons, et bien plus encore. Le jeu se joue sur un champ de bataille de \unit{4}{\foot} par \unit{6}{\foot} (soit environ 1,20 {\meter} par 1,80 {\meter}) avec des dés à six faces pour déterminer le succès d'actions comme charger dans la mêlée, tirer des flèches ou lancer des sorts.

Toutes les règles du jeu, ainsi que les retours et suggestions peuvent être trouvés et donnés ici :

\begin{center}
\url{http://www.the-ninth-age.com/}
\end{center}

\ifdef{\disablenewfromWHB}{}{%
\newfromWHB{Pour aider les anciens joueurs dans leur transition, les changements les plus importants par rapport à la huitième édition sont colorés comme ce paragraphe.}}

\newpage
\section{Note des traducteurs}

L'unité de mesure dans Batailles Fantastiques : Le 9\ieme{} Âge est le pouce anglais, ou \og pouce technique international \fg{}, et se note \inch{}. Il vaut exactement 2,54 {\centi\meter}. Toutes les distances et portées sont indiquées et mesurées en pouces. Le pied, noté \foot{}, vaut 12 pouces. Ce système d'unités, originaire du Moyen-Âge, est encore utilisé aujourd'hui dans quelques régions du monde comme les États-Unis ou le Royaume-Uni, berceau de ce jeu.

Nous souhaitons remercier chaleureusement l'équipe à l'initiative du 9\ieme{} Âge pour leur motivation et leur travail continu pour faire vivre notre passion. Nous espérons que ce jeu saura développer les qualités pour plaire au plus grand nombre et réunir les joueurs, amateurs comme habitués des tournois, autour de règles amusantes et équilibrées, pour finalement s'imposer comme un standard du jeu de figurines. Une grande ambition qui ne pourra s'accomplir que \textbf{grâce à vous}, la communauté, via des retours constructifs, afin de modeler le jeu selon nos désirs. N'étant \textbf{en aucun cas à but lucratif}, le 9\ieme{} Âge part avec un avantage considérable. Les règles des éventuelles nouvelles sorties ne sont pas dictées par le besoin de vendre ces nouveautés. Vous pouvez choisir et acheter vos figurines où bon vous semble, il n'y a pas un unique revendeur toléré. Enfin, vous pouvez être assurés que tant que le 9\ieme{} Âge sera joué, vous disposerez d'un \textbf{support continu et régulier}, celui-ci étant offert par la communauté.

Concernant la traduction en elle-même, nous avons fait de notre mieux pour vous offrir une version de qualité, dont nous espérons qu'elle surpasse celle de la version originale ! Si vous constatez des coquilles, des erreurs, merci de nous les signaler en nous contactant sur le forum du 9\ieme{} Âge, dans le \textbf{sous-forum français} (\url{http://www.the-ninth-age.com/index.php?board/117-french/}). Vous y trouverez aussi les dernières mises à jour. \textbf{En cas de conflit d'interprétation avec la version originale, la version originale fait référence}.

Que ce jeu vous apporte d'innombrables heures de plaisir partagé !

\vspace*{0.5cm}
\ifdef{\translationteam}{
	\begin{multicols}{3}
	\begin{itemize}[label={-}]
		\translationteam
	\end{itemize}
	\end{multicols}
}{}

\vfill
\noindent \labels@latexcredit


\newpage
\section{L'échelle du jeu}

Jouer à un jeu de figurines est souvent un exercice d'abstraction, en particulier lorsqu'il s'agit d'un jeu de batailles de masse comme Batailles Fantastiques : Le 9\ieme{} Âge. Il n'y a donc pas d'échelle recommandée lors d'une partie : une figurine pourrait représenter un seul, une douzaine, ou même une centaine de guerriers. Même si les joueurs sont encouragés à interpréter cette échelle comme ils le souhaitent, les distances utilisées dans les règles ne semblent pas réalistes en comparaison de la taille des figurines utilisées pour le jeu. L'échelle des figurines utilisées pour Le 9\ieme{} Âge est environ de 1/72. On peut en déduire qu'\distance{1} dans le jeu correspondrait à peu près à 1,5 {\meter} dans la réalité. Une figurine typique de taille humaine a une valeur de mouvement de \distance{4}, ce qui signifie qu'en une phase de mouvement, elle se déplacerait de seulement \unit{6}{\meter} (ou \unit{12}{\meter} en cas de marche forcée). De même, une arme de tir telle qu'un arc long a une portée de \distance{30} dans le jeu, ce qui équivaudrait environ à \unit{45}{\meter}. Cela représente à peine 20 \% de la portée historique de ces armes, d'environ \unit{250}{\meter}.

Par exemple, les joueurs pourraient se servir de la portée réelle d'un arc long pour déterminer approximativement la distance qu'\distance{1} représente dans le jeu. Ainsi, \distance{1} représenterait un peu plus de \unit{8}{\meter}, ce qui serait plus proche des distances prises en considération en écrivant les règles de ce jeu.

De même que nous pouvons imaginer que les combattants du jeu sont en réalité plus petits que les figurines qui les personnifient, une figurine ne représente pas forcément un seul guerrier. Par abstraction, nous pourrions envisager qu'une unité de 10 guerriers d'élite elfes représente exactement 10 elfes, ou davantage : 20, 50, ou même 100. Une unité de 10 gobelins gringalets pourrait représenter seulement 10 gobelins, ou plus probablement un groupe de 100, 200 ou 500 avortons. On peut alors se poser la question des personnages et des monstres. Ces figurines sont faites pour incarner des individus exceptionnels et des créatures puissantes qui valent des régiments entiers à elles seules. Si cela facilite les choses, on peut supposer que la figurine d'un personnage représente non seulement cet individu, mais aussi ses gardes du corps et le personnel qui ne manqueraient pas de l'escorter sur le champ de bataille.

De façon semblable, les éléments de décor peuvent être interprétés comme une représentation exacte de ce qu'ils sont dans le jeu, ou être les représentations visuelles de paysages bien plus vastes. Ainsi, un bosquet d'arbres peut évoquer une forêt entière, un ruisseau correspondre à une large rivière, une maison à un hameau, et une tour à un fort.

L'échelle de temps, quant à elle, est à notre avis encore plus arbitraire que l'échelle matérielle du jeu. Les déplacements durant la phase de mouvement pourraient prendre plusieurs minutes de temps réel, alors que les sorts et les tirs pourraient être des évènements quasi-instantanés. Le combat féroce de deux unités au corps à corps pourrait ne durer que quelques battements de cœur, tandis qu'un défi entre deux puissants personnages pourrait être un combat prolongé sur plusieurs minutes ou davantage. Ainsi, aucune mesure de temps quantitative ne peut réellement être associée à un tour ou à une phase de jeu.

Nous ne voulons pas imposer aux joueurs une façon d'imaginer leurs affrontements, ou combien d'individus chaque figurine est censée représenter ; mais nous pensons que la conversion simple d'\distance{1} en environ \unit{10}{\meter} est une interprétation cohérente de la taille du jeu que nous avons créé. Une partie de taille moyenne sera jouée sur une table de \distance{72} par \distance{48}, soit environ 1,80 {\meter} par 1,20 {\meter}, ce qui représente une zone d'environ \unit{700}{\meter} par \unit{500}{\meter}, l'équivalent de 50 terrains de football. Au Moyen-Âge, la période historique la plus proche de notre monde fantastique, cela correspondrait à un champ de bataille de taille moyenne, où deux armées de quelques centaines à quelques milliers de soldats auraient pu se rencontrer.
