
\hypertarget{closecombatphase}{\part{Phase de Corps à Corps}}

Pendant la Phase de Corps à Corps, toutes les figurines engagées au Corps à Corps peuvent et doivent attaquer.

\section{Séquence de la Phase de Corps à Corps}

Une Phase de Corps à Corps est divisée en cinq étapes :

\hspace*{0.3cm}
\begin{tabular}{c|m{14cm}}
1 & Début de la Phase de Corps à Corps. Appliquez la règle Plus Engagés si nécessaire. \tabularnewline
2 & Le Joueur Actif choisit un combat. \tabularnewline
3 & Résolvez cette Manche de Corps à Corps. \tabularnewline
4 & Répétez les étapes 2 et 3 pour chaque combat qui n'a pas encore eu lieu pendant cette phase. \tabularnewline
5 & Une fois que toutes les unités engagées au Corps à Corps ont combattu, la Phase de Corps à Corps prend fin. \tabularnewline
\end{tabular}

Un combat est défini comme un groupe d'unités de camps différents connectées via des contacts socle à socle. Ce groupe peut être simplement constitué de deux unités opposées, de plusieurs unités contre une seule unité ennemie, ou encore d'une longue chaîne d'unités des deux camps. Exécutez toutes les étapes de la Séquence d'une Manche de Corps à Corps pour toutes les unités impliquées avant de passer à un autre combat.

Une unité est considérée comme \textbf{engagée au Corps à Corps} si au moins une figurine de l'unité est en contact socle à socle avec une unité ennemie. Si une unité est engagée au Corps à Corps, toutes ses figurines comptent comme engagées au Corps à Corps. Les unités qui sont engagées au Corps à Corps ne peuvent pas bouger, sauf mention contraire, comme lors des Reformations de Combat ou en cas de fuite.

\hypertarget{nolongerengaged}{\subsection{Plus engagés}}

Une unité qui était engagée dans un combat auparavant, mais dont tous les adversaires ont bougé ou ont été retirés comme perte entre la Phase de Mouvement précédente et cette Phase de Corps à Corps, et dont le contact socle à socle n'a pas pu être maintenu en décalant légèrement les figurines avec les instructions de la règle Éloigné du Combat (paragraphe \ref{dropping_out_of_combat}, page \pageref{dropping_out_of_combat}), doit suivre la règle Plus d'Ennemis (paragraphe \ref{break_test}, page \pageref{break_test}). \newfromWHB{Avant que tout Corps à Corps ne soit résolu, cette unité peut faire un Pivot Post-Combat ou une Reformation Post-Combat. Si elle venait de charger, elle peut aussi effectuer une Charge Irrésistible. Ces actions ne sont autorisées que si l'unité n'a pas bougé, pas même grâce à un Mouvement Magique, depuis que l'unité ennemie a été retirée du combat.}

\newpage
\section{Étapes d'une Manche de Corps à Corps}
\label{combat_round_sequence}

Chaque Manche de Corps à Corps est divisée de la manière suivante :

\hspace*{0.3cm}
\begin{tabular}{c|m{14cm}}
1 & Début de la Manche de Corps à Corps. \tabularnewline
2 & Choisissez les armes (voir le paragraphe \ref{close_combat_weapons}, page \pageref{close_combat_weapons}). \tabularnewline
3 & Appliquez la règle Faites Place (voir le chapitre \ref{characters}, page \pageref{characters}). \tabularnewline
4 & Lancez et relevez ou refusez les Défis (voir le chapitre \ref{characters}, page \pageref{characters}). \tabularnewline
5 & Exécutez les attaques par palier d'Initiative :
	\begin{enumerate}[parsep=0cm,itemsep=0.05cm, topsep=3pt]
		\item Allouez les attaques.
		\item Lancez les jets pour toucher, pour blesser, les jets de sauvegarde et retirez les pertes.
		\item Recommencez pour le palier d'Initiative suivant.
 	\end{enumerate}\tabularnewline
6 & Déterminez quel camp a gagné cette Manche de Corps à Corps. Le(s) perdant(s) passent un test de Moral. \tabularnewline
7 & En cas d'échec, les unités alliées à moins de \distance{6} passent un test de Panique. \tabularnewline
8 & En cas de fuite, choisissez de poursuivre ou de vous réfréner. \tabularnewline
9 & Jet des distances de fuite. \tabularnewline
10 & Jet des distances de poursuite. \tabularnewline
11 & Déplacement des unités en fuite. \tabularnewline
12 & Déplacement des unités poursuivantes. \tabularnewline
13 & Pivots Post-Combat ou Reformations Post-Combat. \tabularnewline
14 & Reformations de Combat. \tabularnewline
15 & Fin de la manche, passez au prochain Corps à Corps. \tabularnewline
\end{tabular}

\vspace*{0.5cm}
Les combats ont lieu dans un ordre strict, en commençant par les attaques des figurines ayant la plus haute valeur d'Initiative (10), puis en descendant par étapes jusqu'à la plus basse. Les attaques appartenant à un même palier d'Initiative sont résolues simultanément. En situation normale, une figurine attaque à la valeur d'Initiative présente sur son profil. Certaines attaques peuvent cependant être résolues à une autre valeur d'Initiative que celle du profil, comme les \impacthits{} d'un Char qui sont faites à Initiative 10. Pour les figurines ayant plusieurs profils, comme un cavalier et sa monture, chaque élément de la figurine frappe à sa propre Initiative.

\newpage
\subsection{Qui peut frapper}

Les figurines en contact socle à socle avec un adversaire peuvent attaquer au palier d'Initiative qui leur correspond. Les figurines des deux camps attaquent à chaque Phase de Corps à Corps.

\paragraph{Attaques de Soutien}

Les figurines au deuxième rang peuvent faire des Attaques de Soutien par-dessus les figurines du premier rang. Les Attaques de Soutien ne peuvent être effectuées que contre des ennemis engagés avec le front de l'unité.

\paragraph{Formation de Horde}

Les figurines déployées en formation de Horde (paragraphe \ref{horde}, page \pageref{horde}) gagnent la règle \fightinextrarank{}.

\paragraph{Rangs Incomplets, attaquer au-dessus d'un vide}

Parfois, des rangs incomplets ou des Personnages avec un socle incompatible peuvent créer des espaces vides au milieu d'unités engagées au Corps à Corps. Si deux unités sont en contact socle à socle, les figurines qui composent ces unités sont autorisées à attaquer à travers les vides. Cela ne permet toutefois pas d'attaquer à travers d'autres unités ou un Terrain Infranchissable. Ces figurines sont considérées comme étant en contact socle à socle malgré les vides. La figure \ref{figure/empty_gaps} illustre ce paragraphe.

\begin{figure}[!htbp]
\hyperlink{whocanstrikefigure}{
\begin{minipage}{0.53\textwidth}
\def\svgwidth{\textwidth}
\input{pics/empty_gaps.pdf_tex}
\end{minipage}\hfill\begin{minipage}{0.44\textwidth}
\caption{Attaques au-dessus de vides.\vspace*{10pt}\newline
Toutes les figurines dont les couleurs sont dans une nuance plus sombre peuvent attaquer. Remarquez que l'unité en rose est en formation de Horde, et peut donc porter des Attaques de Soutien depuis son troisième rang. Notez aussi que l'unité verte n'est pas engagée via son front, et ne peut donc pas effectuer d'Attaques de Soutien sur le flanc ou l'arrière.\vspace*{10pt}\newline
Les figurines dont les bordures sont en gras comptent comme étant en contact socle à socle avec leurs ennemis, même au-dessus des vides. Les figurines dont les couleurs sont dans une nuance plus claire ne peuvent pas faire d'Attaques de Soutien.}
\label{figure/empty_gaps}
\end{minipage}}
\end{figure}

\clearpage
\subsection{Allouer les attaques}

À chaque palier d'Initiative, allouez toutes les attaques avant de de lancer un seul jet pour toucher. Si une figurine est au contact socle à socle avec plus d'une figurine, elle peut choisir laquelle frapper. Les attaques peuvent être allouées à différentes réserves de PVs, c'est-à-dire à des figurines ordinaires, des Champions ou à des Personnages (voir le chapitre \ref{attacks_and_damage}, page \pageref{attacks_and_damage} pour plus de détails sur les réserves de PVs). Le nombre d'attaques qu'une figurine peut porter est égal à la valeur de sa Caractéristique Attaques. L'équipement, les règles spéciales, les sorts, etc., peuvent augmenter ou diminuer ce nombre. Si une figurine a plusieurs attaques, elle peut les répartir comme elle le souhaite entre les cibles en contact socle à socle avec elle. Une figurine qui fait des Attaques de Soutien choisit ses cibles comme si elle était au premier rang, dans la même colonne. Si une figurine peut à la fois attaquer une figurine en contact socle à socle ou effectuer une Attaque de Soutien, elle doit attaquer en priorité une figurine en contact.

\newfromWHB{Dans le cas où une réserve de PVs (la plupart du temps en des figurines ordinaires non Champion de la même unité) est composée de figurines avec des Caractéristiques ou des règles différentes, comme une Endurance ou des sauvegardes différentes, utilisez la valeur ou les règles que possède la majorité des figurines de l'unité. En cas de répartition équitable, l'attaquant choisit. Appliquez-les à tous les jets de dés : toucher, blesser et sauvegardes.}

\paragraph{\newfromWHB{\swirlingmelee}}

Si une figurine ordinaire peut allouer ses attaques à une figurine d'une unité, elle peut toujours choisir de porter toutes ou une partie de ses attaques contre des figurines ordinaires de la même unité à la place. Les figurines ordinaires ne sont ainsi pas forcées d'attaquer les figurines avec lesquelles elles sont en contact socle à socle et peuvent toujours attaquer plutôt d'autres figurines ordinaires. Suivez les règles normales des Attaques de Corps à Corps. Cependant, comme pour les attaques dirigées contre l'unité entière, les touches réussies sont distribuées, et ne peuvent l'être que sur des figurines ordinaires. Notez que cette règle ne s'applique pas aux Personnages, qui ne peuvent attaquer que les figurines avec lesquelles ils sont en contact socle à socle.

\subsection{Jets pour toucher}

Les jets pour toucher sont effectués en lançant 1D6 pour chaque attaque. Comparez ensuite la Capacité de Combat de la figurine portant l'attaque avec celle de la figurine qui reçoit le coup.
\begin{itemize}[label={-}]
\item Si le défenseur a une CC \textbf{inférieure} à celle de l'attaquant, l'attaque touche sur \textbf{3+}.
\item Si le défenseur a une CC \textbf{égale} ou supérieure (jusqu'au double) à celle de l'attaquant, l'attaque touche sur \textbf{4+}.
\item Si le défenseur a une CC \textbf{strictement supérieure au double} de celle de l'attaquant, l'attaque touche sur \textbf{5+}.
\end{itemize}

\newpage
\subsection{Table des jets pour toucher au Corps à Corps}

La règle précédente est utilisée pour composer la table \ref{table/CCtohit}.

\begin{table}[!htbp]
\centering
\begin{tabular}{c|cccccccccc}
\backslashbox{\textbf{D}}{\textbf{A}} & \textbf{1} & \textbf{2} & \textbf{3} & \textbf{4} & \textbf{5} & \textbf{6} & \textbf{7} & \textbf{8} & \textbf{9} & \textbf{10} \\
\hline
\textbf{1} & \yel 4+ & \lem 3+ & \lem 3+ & \lem 3+ & \lem 3+ & \lem 3+ & \lem 3+ & \lem 3+ & \lem 3+ & \lem 3+ \\
\textbf{2} & \yel 4+ & \yel 4+ & \lem 3+ & \lem 3+ & \lem 3+ & \lem 3+ & \lem 3+ & \lem 3+ & \lem 3+ & \lem 3+ \\
\textbf{3} & \ora 5+ & \yel 4+ & \yel 4+ & \lem 3+ & \lem 3+ & \lem 3+ & \lem 3+ & \lem 3+ & \lem 3+ & \lem 3+ \\
\textbf{4} & \ora 5+ & \yel 4+ & \yel 4+ & \yel 4+ & \lem 3+ & \lem 3+ & \lem 3+ & \lem 3+ & \lem 3+ & \lem 3+ \\
\textbf{5} & \ora 5+ & \ora 5+ & \yel 4+ & \yel 4+ & \yel 4+ & \lem 3+ & \lem 3+ & \lem 3+ & \lem 3+ & \lem 3+ \\
\textbf{6} & \ora 5+ & \ora 5+ & \yel 4+ & \yel 4+ & \yel 4+ & \yel 4+ & \lem 3+ & \lem 3+ & \lem 3+ & \lem 3+ \\
\textbf{7} & \ora 5+ & \ora 5+ & \ora 5+ & \yel 4+ & \yel 4+ & \yel 4+ & \yel 4+ & \lem 3+ & \lem 3+ & \lem 3+ \\
\textbf{8} & \ora 5+ & \ora 5+ & \ora 5+ & \yel 4+ & \yel 4+ & \yel 4+ & \yel 4+ & \yel 4+ & \lem 3+ & \lem 3+ \\
\textbf{9} & \ora 5+ & \ora 5+ & \ora 5+ & \ora 5+ & \yel 4+ & \yel 4+ & \yel 4+ & \yel 4+ & \yel 4+ & \lem 3+ \\
\textbf{10} & \ora 5+ & \ora 5+ & \ora 5+ & \ora 5+ & \yel 4+ & \yel 4+ & \yel 4+ & \yel 4+ & \yel 4+ & \yel 4+ \\
\end{tabular}
\caption{Résultat à obtenir pour toucher selon la CC (A : CC de l'Attaquant, D : CC du Défenseur).}
\label{table/CCtohit}
\end{table}

Des modificateurs pour toucher peuvent affecter ces chiffres. Sauf mention contraire, un modificateur pour toucher s'applique aux jets pour toucher de Tir et de Corps à Corps. Un jet pour toucher au Corps à Corps qui donne un \result{1} est toujours un échec, tandis qu'un \result{6} est toujours un succès.

Par exemple, une figurine avec 2 attaques, une CC de 3 et équipée d'une \pw{} aura 3 attaques en tout. Elle alloue 2 touches à une figurine avec une CC de 2, qu'elle touchera sur 3+, et la dernière attaque à une figurine avec une CC de 7, qu'elle touchera sur 5+. Si un ou plusieurs de ces jets pour toucher sont un succès, suivez la procédure décrite dans le chapitre \ref{attacks_and_damage} à la page \pageref{attacks_and_damage}.

La figure \ref{figure/allocation} illustre qui peut attaquer qui dans un cas complexe.

\newcommand{\figAHCharOne}{$P_{1} $}
\newcommand{\figAHCharTwo}{$P_{2} $}
\newcommand{\figAHCharThree}{$P_{3} $}
\newcommand{\figAHChamp}{Ch}
\newcommand{\figAHMus}{Mu}
\newcommand{\figAHStand}{Ét}

\begin{figure}[!htbp]
\begin{minipage}{0.52\textwidth}
\def\svgwidth{\textwidth}
\input{pics/allocation.pdf_tex}
\end{minipage}\hfill\begin{minipage}{0.45\textwidth}
\caption{Exemple d'allocation d'attaques.\vspace*{10pt}\newline
Le Champion de l'unité violette (Ch) et le Personnage $ P_{2} $ sont en Duel, indiqué par le damier en fond, ce qui signifie qu'ils ne peuvent que s'attaquer mutuellement. Les figurines roses et vertes peuvent attaquer les figurines ordinaires de l'unité ennemie. Les figurines avec des bords en gras peuvent attaquer un Personnage ou un Champion. Les figurines en jaune sans contour gras et en violet clair ne peuvent pas attaquer. Le Personnage $ P_{1} $ ne peut pas attaquer parce que la seule figurine en contact socle à socle avec lui est le Champion Ch rose qui est déjà en duel.}
\label{figure/allocation}
\end{minipage}
\end{figure}

\newpage
\subsection{Éloigné du Combat}
\label{dropping_out_of_combat}

Quand on retire des pertes, des unités engagées dans un Corps à Corps perdent parfois le contact socle à socle avec leur adversaire. Quand cela arrive, les unités sont rapprochées pour retrouver le contact de la manière suivante :
\begin{enumerate}
\item L'unité qui est en train de quitter le corps à corps sans avoir subi de pertes est déplacée d'une distance minimale pour rétablir le contact socle à socle.
\item Si cela ne permet pas de rétablir le contact socle à socle, c'est l'unité qui a subi des pertes qui est déplacée à la place, de la même manière.
\end{enumerate}

Une unité qui a perdu le contact socle à socle avec un adversaire est poussée vers l'avant, l'arrière ou sur un côté, voire une combinaison de deux directions (d'abord l'une, puis l'autre). Une unité en contact socle à socle avec d'autres unités ennemies ne peut jamais effectuer cette manœuvre. Ce rapprochement ne peut pas être fait à travers d'autres unités ou un Terrain Infranchissable, mais il est possible de se retrouver à moins de \distance{1} d'autres unités appartenant au même combat. Les unités doivent rester engagées par les mêmes côtés : une unité attaquée sur le flanc avant le mouvement doit le rester. Si plusieurs unités se retrouvent éloignées d'un même combat en même temps, déplacez-les dans l'ordre qui permet au plus grand nombre d'unités de retrouver le corps à corps. S'il reste plusieurs possibilités malgré cette instruction, le Joueur Actif choisit entre celles-ci.

Si aucun déplacement suivant les règles précédentes ne permet de ramener une unité au combat, alors l'unité quitte le Corps à Corps et la règle Plus d'Ennemis (paragraphe \ref{break_test}, page \pageref{break_test}) s'applique.

\newpage
\section{Gagner une Manche de Corps à Corps}

Une fois que tous les paliers d'Initiative sont passés et que toutes les figurines ont eu la possibilité d'attaquer, le Résultat de Combat de chaque camp est calculé pour déterminer le vainqueur de la Manche de Corps à Corps. Additionnez simplement tous les bonus de Résultat de Combat. Le camp avec le Résultat de Combat le plus élevé gagne la manche, tandis que l'autre camp la perd. S'il y a égalité, les deux camps sont considérés comme vainqueurs de cette manche. Un Musicien peut faire pencher la balance en faveur d'un camp (voir le paragraphe \ref{musician}, page \pageref{musician}). Les bonus sont explicités ci-dessous et sont résumés dans la table \ref{table/combat_score}.

\paragraph{Blessures Infligées : +1 par blessure}

Chaque joueur additionne toutes les blessures non sauvegardées causées aux unités ennemies engagées dans le même Corps à Corps pendant cette manche. Cela inclut les ennemis qui étaient engagés dans ce Corps à Corps mais qui en sont sortis ou ont été exterminés pendant cette manche.

\paragraph{Massacre : +1 par blessure (max. \newfromWHB{3})}

Lors d'un Défi, les blessures excédentaires après avoir tué l'adversaire comptent aussi, jusqu'à un maximum de \newfromWHB{+3}. Remarquez que c'est le seul cas où les blessures en trop sont comptabilisées. Dans tous les autres cas, elles sont simplement perdues.

\paragraph{Charge : +1 ou +2}

Si c'est la première Manche de Corps à Corps pour l'unité qui charge, alors son camp reçoit un bonus de +1 à son Résultat de Combat. Si au moins la moitié de l'Empreinte au Sol de l'unité était sur une Colline quand elle a déclaré la charge, et que désormais au moins la moitié de celle-ci est hors de la Colline, elle reçoit un bonus de +2 à la place. Chaque camp ne peut utiliser le bonus que d'une seule unité pour un même combat lors d'un calcul.

\paragraph{Bonus de Rang : +1 par rang (max. +3)}

Chaque camp gagne un bonus de +1 au Résultat de Combat pour chaque Rang Complet après le premier rang d'une même unité, jusqu'à un maximum de +3. Ne comptez ce bonus que pour une seule unité par camp et par Corps à Corps. Prenez simplement l'unité donnant le meilleur Bonus de Rang.

\paragraph{Porte-Étendard : +1}

Chaque camp gagne un bonus de +1 s'il possède au moins un Porte-Étendard engagé dans le combat.

\paragraph{Porteur de la Grande Bannière : +1}

Chaque camp gagne un bonus de +1 si son Porteur de la Grande Bannière est engagé dans le combat.

\paragraph{Attaque de Flanc : +1 \newfromWHB{ou +2}}

Chaque camp gagne un bonus de +1 à son Résultat de Combat si au moins une de ses unités est engagée sur le flanc d'une unité ennemie dans le Corps à Corps. Si l'une de ces unités possède au moins un Rang Complet, le bonus passe à +2.

\paragraph{Attaque de l'Arrière : +2 \newfromWHB{ou +3}}

Chaque camp gagne un bonus de +1 à son Résultat de Combat si au moins une de ses unités est engagée sur l'arrière d'une unité ennemie dans le Corps à Corps. Si l'une de ces unités possède au moins un Rang Complet, le bonus passe à +3.

\begin{table}[!htbp]
\centering
\begin{tabular}{rl}
\hline
Blessures Infligées & \textbf{+1} par blessure \tabularnewline
Massacre & \textbf{+1} par blessure (max. \newfromWHB{\textbf{+3}}) \tabularnewline
Charge & \textbf{+1} (\textbf{+2} depuis une Colline) \tabularnewline
Bonus de Rang & \textbf{+1} par rang (max. \textbf{+3}) \tabularnewline
Porte-Étendard & \textbf{+1} \tabularnewline
Grande Bannière & \textbf{+1} \tabularnewline
Attaque de Flanc & \textbf{+1} ou \newfromWHB{\textbf{+2}} \tabularnewline
Attaque de l'Arrière & \textbf{+2} ou \newfromWHB{\textbf{+3}} \tabularnewline
\hline
\end{tabular}
\caption{Résumé des bonus au Résultat de Combat.}
\label{table/combat_score}
\end{table}

\hypertarget{breaktest}{\section{Test de Moral}}
\label{break_test}

Chaque unité du camp qui a perdu la Manche de Corps à Corps doit passer un test de Moral. L'ordre des tests est choisi par le propriétaire des unités. Un test de Moral est un test de Commandement avec un malus correspondant à la différence entre les deux Résultats de Combat. Par exemple, si les Résultats de Combat sont de 7 contre 4, les unités du camp perdant doivent passer un test de Moral avec un malus de -3. Si le test est raté, l'unité fuit. Si le test est réussi, l'unité reste engagée au Corps à Corps.

\paragraph{Indomptable}

Toute unité possédant plus de Rangs Complets que chaque unité ennemie engagée dans le même Corps à Corps est Indomptable et \newfromWHB{ignore le malus sur le test de Commandement issu des Résultats de Combat}.

\paragraph{Désorganisée}

\newfromWHB{Si une unité est engagée dans un Corps à Corps avec une unité ennemie qui est sur son flanc ou son arrière, et qui possède au moins deux Rangs complets, alors elle est Désorganisée et ne peut pas bénéficier de la règle Indomptable}.

\paragraph{Plus d'Ennemis}

Parfois, une unité extermine toutes les unités ennemies au contact socle à socle avec elle et se retrouve en dehors du Corps à Corps. Elle ne peut alors pas fournir de bonus sur le Résultat de Combat, comme son Porte-Étendard ou via une Attaque de Flanc. Cette unité compte toujours comme ayant gagné le combat, et peut soit faire une Charge Irrésistible si elle y est autorisée, soit faire un \newfromWHB{Pivot Post-Combat} ou une Reformation Post-Combat.

Si cette situation survient dans un combat multiple, les blessures infligées à et par l'unité en question comptent pour le Résultat de Combat, mais les autres bonus sont ignorés. Souvenez-vous que l'unité elle-même ne doit pas passer de test de Moral, puisqu'elle compte toujours comme ayant gagné le combat.

\newpage
\section{Poursuites et Charges Irrésistibles}

Avant de déplacer les unités qui ont raté leur test de Moral, les unités ennemies au contact socle à socle avec elles peuvent choisir de poursuivre une des unités en fuite. Pour pouvoir poursuivre, une unité ne doit pas être engagée avec une unité ennemie qui ne va pas fuir, et doit être au contact de celle qu'elle veut poursuivre. Une unité peut décider de ne pas poursuivre des ennemis en fuite, à condition de réussir un test de Commandement. Si le test de Commandement est raté, l'unité devra poursuivre une des unités ennemies en fuite. Si le test de Commandement est réussi, l'unité peut faire au choix un \newfromWHB{Pivot Post-Combat} ou une Reformation Post-Combat. 

\hypertarget{postcombatpivots}{\paragraph{\newfromWHB{Pivot Post-Combat}}}

\newfromWHB{L'unité pivote sur son centre et peut réorganiser ses figurines avec la règle Au Premier Rang, lesquelles doivent bien sûr rester dans des positions réglementaires.} Ce mouvement est effectué après les mouvements de fuite et de poursuite des autres unités du combat.

\paragraph{Reformation Post-Combat}

L'unité peut effectuer une Reformation. Dans ce cas, \newfromWHB{elle ne pourra pas déclarer de charge durant le Tour de Joueur suivant et ne compte pas comme \scoringunit{} pour les Objectifs Secondaires pour ce Tour de Joueur}. Ce mouvement est effectué après les mouvements de fuite et de poursuite des autres unités du combat.

\paragraph{Charge Irrésistible}

Si une unité vient de combattre sa première Manche d'un Corps à Corps, alors qu'elle avait chargé, et que toutes les unités en contact avec elle ont été détruites, incluant les unités retirées du jeu suite à l'application de la règle \unstable{} ou à une situation similaire, elle peut effectuer un mouvement de poursuite particulier qui s'appelle Charge Irrésistible. Cela remplace le Pivot Post-Combat. La Charge Irrésistible suit les règles de mouvement de poursuite qui se fait droit devant, sans pivot préalable, et pour lequel il n'y a pas besoin de test de Commandement pour se restreindre.

\subsection{Jet de distance de fuite et de poursuite}

Lancez 2D6 pour déterminer la distance de fuite de chaque unité qui a raté son test de Moral. Le joueur ayant gagné le combat lance ensuite 2D6 pour chaque unité ayant déclaré une poursuite, afin de déterminer la distance de cette poursuite. Si au moins une unité poursuivante obtient un résultat \textbf{supérieur ou égal} à la distance de fuite de l'unité qu'elle poursuit, alors cette dernière est rattrapée et détruite. Retirez l'unité du jeu. Aucune sauvegarde ou règle spéciale ne peut la sauver. Notez cependant où se serait trouvé le centre de l'unité qui vient d'être détruite après le mouvement de fuite si elle n'avait pas été rattrapée (voir ci-dessous).

\subsection{Distance et déplacements de fuite}

Chaque unité en fuite qui n'a pas été rattrapée et détruite doit fuir directement à l'opposé de l'unité ennemie en contact socle à socle qui a le plus de Rangs Complets. \newfromWHB{En cas d'égalité de nombre de Rangs Complets entre plusieurs unités, le vainqueur du combat décide quelle unité ces dernières doivent fuir. Pivotez l'unité en fuite pour qu'elle se trouve dans l'axe passant par son centre et le centre de l'unité qu'elle fuit, puis déplacez-la d'un nombre de pouces égal au jet de distance de fuite.} Suivez les règles des mouvements de fuite, à l'exception du fait que la traversée des unités engagées dans le même corps à corps ne provoque pas de test de Terrain Dangereux. Si plusieurs unités fuient un même combat, les unités se déplacent dans le même ordre que les jets de distance de fuite ont été lancés. Le propriétaire choisit l'ordre de ces jets.

\subsection{Distance et déplacement de poursuite}
\label{pursuit_distance_and_pursuing_units}

Chaque unité poursuivante effectue une rotation autour de son centre, par le sens le plus court, pour faire face au centre de l'unité poursuivie ou à sa position si l'unité n'avait pas été rattrapée. Cette rotation ignore tout obstacle : elle peut être faite à travers des figurines ou des décors.
\begin{itemize}[label={\textbullet}]

\item Si cette rotation fait finir le front de l'unité poursuivante sur une unité ennemie, elle compte comme l'ayant chargée. Retirez l'unité poursuivante du champ de bataille, puis replacez-la avec son front en contact socle à socle avec l'unité chargée, dans l'arc correspondant, et en maximisant le nombre de figurines engagées comme toute autre charge. Cependant, s'il n'y a pas assez de place pour permettre cette charge, alors l'unité qui devait être chargée compte plutôt comme un Terrain Infranchissable (voir le point suivant).
\item Si la rotation ne fait pas finir le front de l'unité poursuivante sur une unité ennemie mais sur une unité alliée ou un Terrain Infranchissable, alors elle doit tourner de manière à faire face à l'unité poursuivie autant que possible, en s'arrêtant à \distance{1} de tout obstacle. Son mouvement de poursuite s'arrête ici.
\item Si aucun des cas ci-dessus n'est rencontré, l'unité ignore tout obstacle pendant sa rotation, puis se déplace droit vers l'unité en fuite. Si, durant ce déplacement, elle ne peut pas éviter un obstacle ignoré pendant la rotation initiale, en faisant attention à la Règle du Pouce d'Écart, à moins que le déplacement mène à une charge, revenez en arrière pour respecter l'écart d'\distance{1}. Dans ce cas, il est probable que l'unité ne bouge finalement pas, mais tourne pour faire face à l'unité poursuivie autant que possible.
\end{itemize}

Si ce mouvement de poursuite devait amener l'unité poursuivante en contact avec un nouvel ennemi, elle déclare automatiquement une charge contre cette unité, en utilisant le jet de poursuite comme jet de distance de charge. Cette charge suit les règles habituelles de charge, sauf que l'unité chargée ne peut déclarer aucune réaction. Si cela crée un nouveau Corps à Corps, il sera résolu au prochain Tour de Joueur. L'unité poursuivante sera toujours considérée comme ayant chargé à ce tour. Cependant, si l'unité poursuivante arrive ainsi dans un Corps à Corps qui existait déjà et n'a pas encore été résolu, elle y participera normalement. Ceci lui permet de combattre à nouveau, voire de poursuivre encore. Cette charge spéciale ne peut pas être faite contre une unité qui vient de fuir le combat dont vient l'unité poursuivante. Traitez une telle unité comme une unité alliée pour le mouvement de poursuite, y-compris pour la rotation initiale.

Si l'unité poursuivante ne se retrouve pas en situation de charge comme décrite dans le paragraphe ci-dessus, elle est déplacée droit devant, en respectant la Règle du Pouce d'Écart. 

Si plusieurs unités venant d'un même Corps à Corps poursuivent, elles seront déplacées dans l'ordre des jets de distance de poursuite. Le Joueur Actif décide quel joueur commence à faire ses jets de distance de poursuite, et chaque joueur choisit l'ordre de mouvement de ses propres unités.

La figure \ref{figure/pursuit} illustre quelques cas de poursuite.

\paragraph{Poursuivre hors de la table}

Si une unité qui poursuit rencontre un bord de table, elle quitte le champ de bataille. Elle reviendra à la prochaine étape des Autres Mouvements du propriétaire, en utilisant la règle d'\ambush{}, sauf qu'elle arrive automatiquement et que son rang arrière doit être centré le plus proche possible du point de la table d'où elle est sortie, sans changer de formation. Les figurines qui sont sorties de la table ne peuvent effectuer aucune action. Leurs objets, capacités et règles spéciales cessent de fonctionner en dehors de la table.

\newcommand{\figPursA}{a)}
\newcommand{\figPursB}{b)}
\newcommand{\figPursC}{c)}
\newcommand{\figPursD}{d)}
\newcommand{\figPursFrontofgreenunitontopofredunit}{Front sur l'unité rouge}
\newcommand{\figPursFrontofgreenunitontopoflightgreenunit}{Front sur l'unité vert clair}
\newcommand{\figPursFrontofgreenunitclearofblueunit}{Front dégagé de l'unité bleue}

\begin{figure}[!htbp]
\hypertarget{pursuitsfigure}{
\begin{minipage}{0.53\textwidth}
\def\svgwidth{\textwidth}
\input{pics/pursuit.pdf_tex}
\end{minipage}\hfill\begin{minipage}{0.44\textwidth}
\caption{Quelques cas de poursuite.\vspace*{10pt}\newline
Dans tous ces exemples, l'unité violette est engagée sur le flanc de l'unité verte. L'unité verte gagne le combat. L'unité violette rate son test de Moral et fuit. L'unité verte la poursuit.\vspace*{10pt}\newline
a) Il n'y a pas d'obstacle. L'unité verte se tourne vers l'unité violette et fait son mouvement de poursuite.\vspace*{10pt}\newline
b) La rotation de l'unité verte fait finir son front sur une unité ennemie, en rouge. L'unité verte doit donc charger l'unité rouge : déplacez-la en contact socle à socle avec celle-ci.\vspace*{10pt}\newline
c) La rotation de l'unité verte fait finir son front sur une unité alliée, en vert clair. Dans ce cas, la rotation doit être arrêtée à \distance{1} de l'unité alliée.\vspace*{10pt}\newline
d) La rotation de l'unité verte fait finir une partie de ses figurines sur une autre unité, en bleu, alliée ou ennemie. Cependant, son front est dégagé, donc elle peut faire son mouvement de poursuite droit devant.}
\label{figure/pursuit}
\end{minipage}}
\end{figure}

\clearpage
\hypertarget{combatreform}{\section{Reformation de Combat}}

Après que toutes les fuites et poursuites ont été résolues, les unités restant dans des Corps à Corps peuvent tenter de faire une Reformation de Combat. Les unités engagées sur plus d'un côté, par exemple de front et de flanc, ne peuvent jamais effectuer de Reformation de Combat. Les unités du perdant de la manche doivent réussir au préalable un test de Commandement pour pouvoir faire une Reformation de Combat, avec le même malus que le test de Moral (c'est-à-dire la différence de Résultat de Combat, à moins que l'unité ne soit Indomptable).

\newfromWHB{Si les deux camps veulent effectuer une Reformation de Combat, le Jouer Actif décide qui commence.} Ce joueur doit alors faire les reformations de toutes ses unités du combat, une par une, dans l'ordre qui lui plaît, avant que l'autre joueur puisse faire les siennes.

Quand une unité fait une Reformation de Combat, retirez-la du champ de bataille puis replacez-la dans une formation réglementaire, en contact socle à socle avec la ou les unités ennemies avec lesquelles elle était au Corps à Corps auparavant. Les unités ennemies doivent rester engagées par le même côté. Vous pouvez ignorer la Règle du Pouce d'Écart pour les unités appartenant à un même Corps à Corps, mais vous ne pouvez pas créer de contact avec une nouvelle unité. Aucune figurine ne doit finir à plus de deux fois sa valeur de Mouvement de sa position de départ.

\newfromWHB{À la fin de toute Reformation de Combat, vous devez avoir au moins autant de figurines en contact socle à socle avec un ennemi qu'il y en avait avant la reformation. Tout Personnage qui était au contact avec un ennemi doit le rester, même si cela peut être avec des figurines différentes.} De plus, toute figurine ennemie qui était au contact d'une de vos figurines avant la reformation doit toujours l'être, même si c'est avec d'autres figurines, voire même une autre unité.

