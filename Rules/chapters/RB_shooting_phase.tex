
\hypertarget{shootingphase}{\part{Phase de Tir}}

Lors de la Phase de Tir, les figurines possédant des Attaques de Tir peuvent les utiliser.

\section{Séquence de la Phase de Tir}

La Phase de Tir est divisée en quatre étapes :

\hspace*{0.3cm}
\begin{tabular}{c|m{14cm}}
1 & Début de la Phase de Tir. \tabularnewline
2 & Choisissez une unité avec laquelle effectuer des Attaques de Tir, puis tirez. \tabularnewline
3 & Répétez l'étape 2 avec une autre unité qui n'a pas encore tiré lors de cette phase. \tabularnewline
4 & Quand toutes les unités pouvant tirer et que vous souhaitez faire tirer l'ont fait, la Phase de Tir prend fin. \tabularnewline
\end{tabular}

\subsection{Tirer avec une unité}

Chaque unité équipée d'Arme de Tir peut tirer une fois par Phase de Tir. Les unités en fuite, engagées au Corps à Corps, ou qui ont fait une Marche Forcée, une Reformation, qui se sont ralliées ou qui ont déclaré une charge lors de la Phase de Mouvement de ce tour ne peuvent pas tirer.

Quand une unité veut tirer, commencez par désigner une cible sur laquelle l'unité a une Ligne de Vue. Il est interdit de tirer sur une unité engagée au Corps à Corps. Toutes les figurines d'une même unité doivent tirer sur la même cible. \textbf{Seules les figurines du premier et du deuxième rang peuvent tirer}. Si elles possèdent plusieurs Armes de Tir différentes, annoncez quelle arme est utilisée. Les figurines ordinaires doivent toutes utiliser la même, tandis que les Champions et Personnages peuvent choisir séparément. Chaque figurine de l'unité est libre de décider de ne pas tirer.

Vérifiez la Ligne de Vue de chaque figurine. Souvenez-vous que la Ligne de Vue est toujours tracée depuis l'avant. Les figurines qui n'ont pas de Ligne de Vue sur la cible ne peuvent pas tirer. Mesurez la portée pour chaque figurine qui tire. La portée est mesurée depuis la position réelle de chaque figurine jusqu'au point le plus proche de l'unité ciblée, même si ce point n'est pas en Ligne de Vue. Les figurines hors de portée ne peuvent pas tirer. Une fois que vous avez déterminé les figurines pouvant tirer, lancez les jets pour toucher pour chacune d'elles.

\newpage
\section{Jets pour toucher de tir}

Quand vous lancez le jet pour toucher des Attaques de Tir, utilisez la Capacité de Tir (CT) de la figurine. Si la figurine a plusieurs profils, comme par exemple un cavalier et sa monture, utilisez la CT de l'élément qui tire. Le propriétaire de la figurine lance 1D6. Un résultat non modifié de \result{1} sur le dé est toujours un échec. Il ajoute ensuite sa Capacité de Tir : si le total est de \textbf{7} ou plus, l'attaque touche. Si au moins un tir a touché, suivez la procédure décrite dans le chapitre \ref{attacks_and_damage} à la page \pageref{attacks_and_damage}.

\subsection{Modificateurs du jet pour toucher}
\label{to_hit_modifiers}

\begin{multicols}{2}\raggedcolumns

Les attaques de tir peuvent subir un ou plusieurs modificateurs sur leurs jets pour toucher. Additionnez simplement les modificateurs à la Capacité de Tir du tireur. La table ci-dessous décrit les modificateurs usuels, mais les sorts et capacités peuvent ajouter d'autres modificateurs. 

La table ci-contre résume le résultat à obtenir sur le jet pour toucher selon la CT et les modificateurs.

\begin{center}
\begin{tabular}{rl}
\hline
Bouger et Tirer & -1 \tabularnewline
Longue Portée & -1 \tabularnewline
Tenir la Position et Tirer & -1 \tabularnewline
Couvert Léger & -1 \tabularnewline
Couvert Lourd & -2 \tabularnewline
\hline
\end{tabular}

\noindent Résumé des Modificateurs pour toucher.
\end{center}

\vspace*{\fill}\columnbreak

\begin{center}
\begin{tabular}{rl}
\hline
\textbf{CT + Modif.} & \textbf{Résultat nécessaire} \tabularnewline
6 ou plus & 2+ \tabularnewline
5 & 2+ \tabularnewline
4 & 3+ \tabularnewline
3 & 4+ \tabularnewline
2 & 5+ \tabularnewline
1 & \result{6} \tabularnewline
0 & \result{6} suivi d'un 4+ \tabularnewline
-1 & \result{6} suivi d'un 5+ \tabularnewline
-2 & \result{6} suivi d'un \result{6} \tabularnewline
-3 ou moins & impossible \tabularnewline
\hline
\end{tabular}

\noindent Jets pour toucher au Tir.
\end{center}
\vspace*{\fill}
\end{multicols}

\paragraph{Bouger et Tirer (-1)}

Si l'unité s'est déplacée lors de ce Tour de Joueur, ses figurines subissent un malus de -1 pour toucher.

\paragraph{Longue Portée (-1)}

Si la cible se trouve au-delà de la moitié de la portée de l'arme, le tireur subit un malus de -1 pour toucher. Souvenez-vous que la portée doit être mesurée individuellement pour chaque figurine.

\paragraph{Tenir la Position et Tirer (-1)}

Les Attaques de Tir réalisées pendant la réaction à la charge Tenir la Position et Tirer subissent un malus de -1 pour toucher.

\paragraph{Couvert}

Le Couvert est déterminé individuellement, pour chaque tireur. Il existe deux types de Couvert, Léger et Lourd. Dans les deux cas, le couvert est déterminé en observant la Ligne de Vue de la figurine qui tire. Tracez toutes les Lignes de Vue depuis un point de votre choix à l'avant du socle du tireur jusqu'à tout point de l'Empreinte au Sol de la cible. Exceptionnellement, ces Lignes de Vues peuvent sortir de l'arc frontal du tireur. Si ces lignes sont interrompues par un Décor ou des figurines, selon la nature et la position de ces derniers, le tir peut subir un malus pour toucher. Un tireur ignore toujours les figurines de sa propre unité et de l'unité ciblée, ainsi que le Décor dans lequel il se trouve, pour la détermination du Couvert. Une unité située dans une Forêt ne subit donc pas de malus de Couvert Léger à cause de la Forêt.

\paragraph{Cible derrière un Couvert Léger (-1)}

Une figurine tirant sur une cible derrière un Couvert Léger subit un malus de -1 pour toucher. Le Couvert Léger s'applique si au moins la moitié de l'Empreinte au Sol de la cible est cachée par un ou plusieurs des éléments suivants (voir figure \ref{figure/soft_cover}) :
\begin{itemize}[label={-}]
\item Décor offrant un Couvert Léger.
\item \newfromWHB{Figurines de n'importe quelle taille. Si vous tirez avec ou sur des figurines de Grande Taille (voir le paragraphe \ref{modelheight}, page \pageref{modelheight}), ignorez les figurines de Petite Taille pour déterminer le Couvert.}
\end{itemize}

\paragraph{Cible derrière un Couvert Lourd (-2)}

Une figurine tirant sur une cible derrière un Couvert Lourd subit un malus de -2 pour toucher. Le Couvert Lourd s'applique si au moins la moitié de l'Empreinte au Sol de la cible est cachée par un ou plusieurs des éléments suivants (voir figure \ref{figure/hard_cover}) :
\begin{itemize}[label={-}]
\item Décor offrant un Couvert Lourd.
\item Figurines \textbf{de même Taille ou plus grand} que le tireur \textbf{et} sa cible (voir le paragraphe \ref{modelheight}, page \pageref{modelheight}).
\end{itemize}

\paragraph{Cible derrière un Couvert Léger et un Couvert Lourd}

Si un tireur subit à la fois des malus de Couvert Léger et de Couvert Lourd, ne comptez que le Couvert Lourd. Si l'Empreinte au Sol de la cible est cachée à la fois par des éléments pouvant offrir un Couvert Léger et un Couvert Lourd, mais pas assez pour obtenir l'un ou l'autre des Couverts, le tireur subit la pénalité de Couvert Léger si au moins la moitié de l'Empreinte au Sol de l'unité visée est cachée par l'ensemble des éléments. Par exemple, si la cible est cachée à 30 \% par des éléments offrant un Couvert Léger et à 30 \% par des éléments offrant un Couvert Lourd, le tireur subit la pénalité de Couvert Léger (voir figure \ref{figure/soft_and_hard_cover}).

\newcommand{\figureSCNotinlightofsight}{\normalfontsize{\textit{Pas de Ligne de Vue}}}
\newcommand{\figureSCForest}{Forêt}
\newcommand{\figureSCFire}{\smallfontsize{\textit{Feu !}}}
\newcommand{\figureSCA}{a)}
\newcommand{\figureSCB}{b)}
\newcommand{\figureSCC}{c)}
\newcommand{\figureSCWithinlightofsight}{\normalfontsize{\textit{Ligne de Vue !}}}
\newcommand{\figureSCLessthanhalfoffootprintobscured}{\normalfontsize{\textit{Moins de 50 \% cachée}}}
\newcommand{\figureSCMorethanhalfoffootprintobscured}{\normalfontsize{\textit{Plus de 50 \% cachée}}}

\vspace*{1.5cm}

\begin{figure}[!htbp]
\centering
\hypertarget{coverfigures}{
\def\svgwidth{\textwidth}
\input{pics/soft_cover.pdf_tex}}
\caption{Illustration d'un Couvert Léger.\vspace*{10pt}\newline
a) Ces figurines ne peuvent pas tirer, car elles n'ont pas de Ligne de Vue sur l'ennemi.\vspace*{10pt}\newline
b) Ces figurines peuvent tirer, car elles ont une Ligne de Vue sur l'ennemi, et il n'y a pas de Couvert puisque moins de la moitié de l'Empreinte au Sol de la cible est cachée par la Forêt.\vspace*{10pt}\newline
c) Ces figurines peuvent tirer, mais souffrent de la pénalité de Couvert Léger, car plus de la moitié de l'Empreinte au Sol de la cible est cachée par la Forêt.}
\label{figure/soft_cover}
\end{figure}

\newcommand{\figureHCImpassableTerrain}{\normalfontsize{Terrain Infranchissable}}

\begin{figure}[!htbp]
\centering
\def\svgwidth{\textwidth}
\input{pics/hard_cover.pdf_tex}
\caption{Illustration d'un Couvert Lourd.\vspace*{10pt}\newline
a) Ces figurines ne peuvent pas tirer, car les Lignes de Vue sont bloquées.\vspace*{10pt}\newline
b) Ces figurines peuvent tirer, car elles ont une Ligne de Vue sur l'ennemi, mais souffrent de la pénalité de Couvert Lourd, puisque plus de la moitié de l'Empreinte au Sol de la cible est cachée par le Terrain Infranchissable.\vspace*{10pt}\newline
c) Ces figurines peuvent tirer, et il n'y a pas de Couvert car moins de la moitié de l'Empreinte au Sol de la cible est cachée par le Terrain Infranchissable.}
\label{figure/hard_cover}
\end{figure}

\newcommand{\figureSHCHill}{Colline}
\newcommand{\figureSHCHeightOne}{Taille 1}
\newcommand{\figureSHCHeightTwo}{Taille 2}
\newcommand{\figureSHCLessthanhalffromhardcover}{\normalfontsize{\textit{Moins de 50 \% en Couvert Lourd}}}
\newcommand{\figureSHCLessthanhalffromsoftcover}{\normalfontsize{\textit{Moins de 50 \% en Couvert Léger}}}
\newcommand{\figureSHCMorethanhalftotal}{\normalfontsize{\textit{Plus de 50 \% en tout}}}

\begin{figure}[!htbp]
\begin{minipage}[t]{0.45\textwidth}
\def\svgwidth{\textwidth}
\input{pics/soft_and_hard_cover.pdf_tex}
\end{minipage}\hfill\begin{minipage}[b]{0.52\textwidth}
\caption{Cas de mélange de Couvert Léger et Lourd.\vspace*{10pt}\newline
Dans cet exemple, moins de la moitié de l'Empreinte au Sol de la cible est cachée par un Couvert Léger ou un Couvert Lourd. Toutefois, la combinaison des deux en cache plus de la moitié, donc la cible compte comme étant derrière un Couvert Léger.}
\label{figure/soft_and_hard_cover}
\end{minipage}
\end{figure}

\clearpage
\paragraph{Lignes de Vue et Couverts}

La figure \ref{figure/line_of_sight_and_cover} résume les cas les plus communs de ligne de vue et de couvert entre un tireur et sa cible quand une figurine se trouve sur la trajectoire.

\newcommand{\figureLoSCSoftcover}{\Largefontsize{Couvert Léger}}
\newcommand{\figureLoSCHardcover}{\Largefontsize{Couvert Lourd}}
\newcommand{\figureLoSCNocover}{\Largefontsize{Pas de Couvert}}
\newcommand{\figureLoSCNolineofsight}{\Largefontsize{Pas de Ligne de Vue}}
\newcommand{\figureLoSCSmall}{Petit}
\newcommand{\figureLoSCMedium}{Moyen}
\newcommand{\figureLoSCLarge}{Grand}

\begin{figure}[!htbp]
\centering
\def\svgwidth{12cm}
\input{pics/line_of_sight_and_cover.pdf_tex}
\caption{Ligne de Vue et Couvert dans des cas communs avec une figurine sur la trajectoire.}
\label{figure/line_of_sight_and_cover}
\end{figure}

