
\part{Attaques et blessures}
\label{attacks_and_damage}

Les attaques sont réparties entre Attaques de Corps à Corps et Attaques à Distance.

\paragraph{Attaques de Corps à Corps}

Toutes les attaques portées d'une entité à une autre au contact socle à socle et lors de la Phase de Corps à Corps, ou comme lors d'une Phase de Corps à Corps, sont considérées comme des Attaques de Corps à Corps.

\paragraph{\newfromWHB{Attaques à Distance}}

Toutes les attaques qui ne sont pas des Attaques de Corps à Corps sont des Attaques à Distance.

Toutes les Attaques à Distance réalisées lors de la Phase de Tir ou en réaction à une charge sont des \textbf{Attaques de Tir}.

\paragraph{\newfromWHB{Attaques Spéciales}}

Certaines figurines sont autorisées à faire des \textbf{Attaques Spéciales} pendant la Phase de Corps à Corps, de Mouvement ou de Tir. Les Attaques Spéciales ne profitent a priori pas des bonus conférés par les armes ou les règles spéciales qui affecteraient normalement les attaques de la même catégorie. C'est d'ailleurs explicitement précisé pour certaines Attaques Spéciales, comme le \stomp{}, les \grindingattacks{}, l'\crushattack{}, les \impacthits{} ou encore l'\breathweapon{}.

\section{Séquence d'une attaque}

Lorsqu'une ou plusieurs attaques touchent, suivez la séquence suivante :

\hspace*{0.3cm}
\begin{tabular}{c|m{14cm}}
1 & L'attaquant répartit les touches. \tabularnewline
2 & L'attaquant lance le ou les dés pour blesser. En cas de succès, passez à l'étape suivante. \tabularnewline 
3 & Le défenseur lance le ou les dés pour les Sauvegardes d'Armure. En cas d'échec, passez à l'étape suivante. \tabularnewline
4 & Le défenseur lance le ou les dés pour les sauvegardes spéciales. En cas d'échec, passez à l'étape suivante. \tabularnewline
5 & Le défenseur retire les PVs ou les pertes. \tabularnewline
6 & Le défenseur passe des tests de Panique si nécessaire. \tabularnewline
\end{tabular}

Réalisez chaque étape pour toutes les attaques qui ont été portées en même temps avant de passer à l'étape suivante. Les Attaques de Tir provenant d'une même unité ou toutes les Attaques de Corps à Corps d'un même palier d'Initiative en sont des exemples.

\newpage
\section{Répartir les touches}

\newfromWHB{Les attaques qui ciblent une unité dans son ensemble, comme la plupart des Attaques à Distance et des Attaques Spéciales de Corps à Corps, sont considérées comme touchant uniquement les figurines ordinaires.} Les touches provoquées par des Gabarits sur des figurines ordinaires, y-compris les Champions, sont réparties sur les figurines ordinaires à cette étape : cela signifie qu'un Gabarit ne peut pas toucher une figurine ordinaire spécifique. Cette répartition peut être modifiée lorsqu'un Personnage a rejoint une unité (voir la section \ref{characters}, page \pageref{characters}).

Les Attaques normales de Corps à Corps ne sont pas réparties mais Allouées avant que les jets pour toucher ne soient lancés. Ignorez donc ce paragraphe pour ces attaques. 

\newfromWHB{Dans le cas où une réserve de PVs (la plupart du temps en des figurines ordinaires non Champion de la même unité) est composée de figurines avec des Caractéristiques ou des règles distinctes, comme une Endurance ou des sauvegardes différentes, utilisez la valeur ou les règles que possède la majorité des figurines de l'unité. En cas de répartition équitable, l'attaquant choisit. Appliquez-les à tous les jets de dés : toucher, blesser et sauvegardes.}

\section{Jets pour blesser}

Si l'attaque a une valeur en Force, elle doit blesser la cible pour avoir une chance de lui infliger des dégâts. Comparez la Force de l'attaque à l'Endurance de la cible. Une attaque avec une Force de 0 ne peut pas blesser. Le joueur qui a infligé les touches lance un dé pour blesser pour chaque attaque qui a touché la cible. Un résultat non modifié de \result{6} est toujours un succès tandis qu'un \result{1} est toujours un échec. Sinon, suivez le paragraphe suivant. Si l'attaque n'a pas de valeur en Force, suivez les règles données pour cette attaque particulière. 

\subsection{La table pour blesser}

Lancez 1D6 pour chaque touche. Pour déterminer quel résultat est nécessaire pour blesser la cible, croisez la colonne de la Force (F) correspondant à l'attaque avec la ligne de l'Endurance (E) de la cible sur la table \ref{table/to_wound}.

\begin{table}[!htbp]
\centering
\begin{tabular}{c|cccccccccc}
\backslashbox{\textbf{E}}{\textbf{F}} & \textbf{1} & \textbf{2} & \textbf{3} & \textbf{4} & \textbf{5} & \textbf{6} & \textbf{7} & \textbf{8} & \textbf{9} & \textbf{10} \tabularnewline
\hline
\textbf{1} & \yel 4+ & \lem 3+ & \gre 2+ & \gre 2+ & \gre 2+ & \gre 2+ & \gre 2+ & \gre 2+ & \gre 2+ & \gre 2+ \tabularnewline
\textbf{2} & \ora 5+ & \yel 4+ & \lem 3+ & \gre 2+ & \gre 2+ & \gre 2+ & \gre 2+ & \gre 2+ & \gre 2+ & \gre 2+ \tabularnewline
\textbf{3} & \red 6+ & \ora 5+ & \yel 4+ & \lem 3+ & \gre 2+ & \gre 2+ & \gre 2+ & \gre 2+ & \gre 2+ & \gre 2+ \tabularnewline
\textbf{4} & \red 6+ & \red 6+ & \ora 5+ & \yel 4+ & \lem 3+ & \gre 2+ & \gre 2+ & \gre 2+ & \gre 2+ & \gre 2+ \tabularnewline
\textbf{5} & \red 6+ & \red 6+ & \red 6+ & \ora 5+ & \yel 4+ & \lem 3+ & \gre 2+ & \gre 2+ & \gre 2+ & \gre 2+ \tabularnewline
\textbf{6} & \red 6+ & \red 6+ & \red 6+ & \red 6+ & \ora 5+ & \yel 4+ & \lem 3+ & \gre 2+ & \gre 2+ & \gre 2+ \tabularnewline
\textbf{7} & \red 6+ & \red 6+ & \red 6+ & \red 6+ & \red 6+ & \ora 5+ & \yel 4+ & \lem 3+ & \gre 2+ & \gre 2+ \tabularnewline
\textbf{8} & \red 6+ & \red 6+ & \red 6+ & \red 6+ & \red 6+ & \red 6+ & \ora 5+ & \yel 4+ & \lem 3+ & \gre 2+ \tabularnewline
\textbf{9} & \red 6+ & \red 6+ & \red 6+ & \red 6+ & \red 6+ & \red 6+ & \red 6+ & \ora 5+ & \yel 4+ & \lem 3+ \tabularnewline
\textbf{10} & \red 6+ & \red 6+ & \red 6+ & \red 6+ & \red 6+ & \red 6+ & \red 6+ & \red 6+ & \ora 5+ & \yel 4+ \tabularnewline
\end{tabular}
\caption{\label{table/to_wound}Résultat à obtenir pour blesser.}
\end{table}

\section{Sauvegarde d'Armure et malus}

Si au moins une blessure a été infligée, le propriétaire de l'unité attaquée peut tenter de se protéger de ces blessures avec une Sauvegarde d'Armure. Pour ce faire, il doit lancer 1D6 pour chaque jet pour blesser réussi et comparer le résultat à la valeur de Sauvegarde d'Armure de la figurine (voir le paragraphe \ref{armour_types}, page \pageref{armour_types}). C'est un jet de Sauvegarde d'Armure.

Si la Force de l'attaque est supérieure ou égale à 4, cette attaque applique un malus sur le jet de Sauvegarde d'Armure de -1 pour chaque point de Force au-delà de 3, jusqu'à un malus de -6, comme illustré dans la table \ref{table/armour}.

\begin{table}[!htbp]
\centering
\begin{tabular}{c@{\hspace{0.5cm}}cccccccccc}
\hline
\textbf{Force} & \textbf{1} & \textbf{2} & \textbf{3} & \textbf{4} & \textbf{5} & \textbf{6} & \textbf{7} & \textbf{8} & \textbf{9} & \textbf{10} \tabularnewline
\textbf{Malus} & 0 & 0 & 0 & \red -1 & \red -2 & \red -3 & \red -4 & \red -5 & \red -6 & \red -6 \tabularnewline
\hline
\end{tabular}
\caption{Malus sur le jet de sauvegarde d'armure par la Force.}
\label{table/armour}
\end{table}

Certaines règles spéciales et capacités peuvent également modifier le jet de Sauvegarde d'Armure, comme la règle \armourpiercing{}. 

\section{Régénération et Sauvegarde Invulnérable}

Si une blessure n'a pas été annulée par un jet de Sauvegarde d'Armure réussi, la figurine attaquée peut avoir une dernière chance d'éviter la blessure en réalisant un jet de \regeneration{} ou de \wardsave{}, si elle possède une de ces deux règles. La \regeneration{} et la \wardsave{} fonctionnent de la même manière que la Sauvegarde d'Armure, sauf qu'elles ne peuvent pas être modifiées ni combinées. Ces règles sont toujours données avec une valeur qui correspond au résultat minimal à obtenir sur 1D6 pour réussir la sauvegarde. Ce chiffre figure généralement entre parenthèses. Par exemple, une \wardsave{4} signifie que la blessure est annulée sur un résultat de \result{4} ou plus sur 1D6.

Faites autant de jets de sauvegarde que de blessures qui n'ont pas été évitées par la Sauvegarde d'Armure.

Si une figurine a plusieurs instances des règles \regeneration{} et \wardsave{}, choisissez laquelle utiliser avant de jeter les dés. Une seule \regeneration{} \textbf{ou} \wardsave{} peut être utilisée contre une attaque.

La \regeneration{} et la \wardsave{} fonctionnent de la même manière, mais certains sorts ou règles spéciales affectent différemment ces deux types de sauvegarde.

\newpage
\section{Retirer les Points de Vie}

Retirez un Point de Vie pour chaque blessure pour laquelle la figurine attaquée a raté toutes ses sauvegardes.

\paragraph{Figurines non ordinaires}

Si une attaque a été répartie ou allouée à une figurine non ordinaire, elle perd un Point de Vie pour chaque sauvegarde ratée. Si elle tombe à 0 PV, elle meurt. Retirez-la comme perte. Sinon, gardez en mémoire les figurines qui ont été blessées et qui n'ont pas perdu tous leurs PVs. Vous pouvez par exemple placer des marqueurs de blessures près des figurines. Ces PVs perdus sont à prendre en compte lors de prochaines attaques. Quand une figurine est tuée, toutes les blessures excédentaires sont ignorées, sauf dans le cas d'un Défi où elles peuvent être comptabilisées comme bonus de Carnage pour le Résultat de Combat. 

\paragraph{Champions}

Même si les Champions sont des figurines ordinaires, chaque Champion a sa propre réserve de PVs et suit les règles des figurines non ordinaires ci-dessus. Si suffisamment de blessures sont subies par les figurines ordinaires pour anéantir l'ensemble de l'unité, les blessures excédentaires sont allouées au Champion, même s'il se bat dans un Défi.

\paragraph{Figurines ordinaires}

Les figurines ordinaires d'une même unité (sauf le Champion) partagent une réserve de PVs commune. Si une attaque a été allouée à une figurine ordinaire, retirez un PV de la réserve commune pour chaque sauvegarde ratée. Si les figurines ordinaires ont 1 PV chacune, retirez une figurine par PV perdu.

Si les figurines ordinaires disposent de plus d'un PV, retirez tous les PVs sur une même figurine, si possible, avant d'en retirer sur une autre. Gardez ensuite en mémoire les figurines blessées dont les PVs ne sont pas tombés à zéro. Ces PVs perdus sont à prendre en compte pour les prochaines attaques. Prenons un exemple : une unité de 10 Trolls, avec 3 PVs chacun, reçoit 7 blessures non sauvegardées. Retirez deux figurines, soit 6 PVs, et posez un marqueur indiquant qu'un autre Troll a perdu un PV. Plus tard, cette unité perd 2 PVs. C'est suffisant pour tuer un Troll, puisqu'un PV a été perdu précédemment. 

Si l'unité est détruite, toutes les blessures excédentaires sont allouées au Champion. S'il n'y a pas de Champion, elles sont perdues.

Si une unité est composée de figurines ordinaires de Types de Troupe différents, chaque type de figurine ordinaire a sa propre réserve de PVs. Parfois, une ou plusieurs figurines ordinaires bénéficient de règles spéciales supplémentaires, par exemple de Caractéristiques augmentées, de nouvelles capacités, etc. venant habituellement de sorts ou d'Objets Magiques. De telles différences sont ignorées lorsque vous retirez les pertes. Ces dernières sont retirées normalement depuis le dernier rang, même si une figurine dans ce rang possède des règles ou équipements différents. En effet, seule une différence de Type de Troupe implique une séparation des réserves de PVs. 

\newpage
\section{Retirer les pertes}

\paragraph{Retirer les figurines ordinaires}

Les pertes en figurines ordinaires sont retirées depuis le rang arrière. Si l'unité ne possède qu'un seul rang, retirez les figurines de chaque côté aussi équitablement que possible. Notez que cette consigne s'applique à chaque groupe d'attaques simultanées, il n'y a pas de \og mémoire \fg{} du côté par lequel les précédentes figurines ont été retirées.

Si une figurine non ordinaire se trouve à l'endroit où une perte devrait normalement être retirée, enlevez plutôt la prochaine figurine ordinaire, puis déplacez la figurine non ordinaire pour prendre sa place.

\paragraph{Retirer les figurines non ordinaires}

Les figurines non ordinaires sont retirées directement depuis leur position quand elles sont perdues. Les figurines ordinaires sont ensuite déplacées pour combler les vides. Prenez les figurines nécessaires de la même manière que vous retireriez des pertes, c'est-à-dire du rang arrière, puis équitablement de chaque côté en cas de rang unique.

\paragraph{Unités engagées au Corps à Corps}

Si l'unité est composée d'un rang unique, \newfromWHB{retirez les pertes de chaque côté, de manière à ce que le nombre d'unités au contact soit maximisé en priorité, et qu'ensuite le nombre de figurines en contact socle à socle soit maximisé.}

\section{Test de Panique}

Les tests de Panique sont des tests de Commandement qui doivent être passés immédiatement après qu'une des situations suivantes se produit :
\begin{itemize}[label={-}]
\item Une unité alliée est détruite à moins de \distance{6}. Cela comprend celles qui fuient hors de table.
\item Une unité alliée rate un test de Moral et fuit à moins de \distance{6}.
\item Une unité alliée a fui au travers de l'unité.
\item L'unité subit, lors d'une même phase, 25\% ou plus de pertes en figurines par rapport au début de cette même phase. Cette règle s'appelle \textbf{Lourdes Pertes}.
\end{itemize}

Une unité qui rate un test de Panique doit fuir à l'opposé de l'unité ennemie la plus proche, dans l'axe passant par les centres des deux unités. S'il n'y a pas d'unités ennemies sur la table, la direction est aléatoire.

Si le test de Panique a été engendré par l'un des cas suivants, l'unité fuit plutôt directement à l'opposé de l'unité ennemie à l'origine du test de Panique :
\begin{itemize}[label={-}]
\item \newfromWHB{Un sort jeté par une figurine ennemie.}
\item \newfromWHB{Une règle d'une figurine ennemie (comme la \terror{}).}
\item L'unité subit de Lourdes Pertes, et les blessures permettant de passer le cap de 25\% de pertes ont été infligées par une unité ennemie.
\end{itemize}

Une unité engagée au Corps à Corps, déjà en fuite ou qui a déjà passé un test de Panique lors de cette phase n'a plus à passer de tests de Panique.
