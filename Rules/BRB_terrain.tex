% Base sur la VO 0.11.9
% Relecture technique: 
% Relecture syntaxique: 

\part{Terrains et décors}

\section{Types de terrain}

\subsubsection*{\nouveau{Terrain dangereux (X)}}

Lorsqu'une figurine effectue une \emph{Marche Forcée}, charge (à l'exception d'une charge ratée), fuit, poursuit ou fait une \emph{Charge Irrésistible}, au travers, en entrant ou en sortant d'un \emph{Terrain Dangereux}, elle doit effectuer un test de \emph{Terrain Dangereux}. \nouveau{Pour ce faire, lancez 2D6 pour une figurine de type \emph{Infanterie Monstrueuse} ou \emph{Cavalerie Monstrueuse}, 4D6 pour un \emph{Char} ou un \emph{Monstre}, monté ou non, ou 1D6 pour une figurine d'un autre type}. Pour chaque \result{1} obtenu, la figurine subit une blessure avec la règle spéciale \emph{Perforant (6)}.

Parfois, la règle est écrite \emph{Terrain Dangereux (X)}. Dans ce cas, la figurine subit une blessure pour chaque dé inférieur ou égal à X.

\newrule{Les tests de terrain dangereux sont effectués dès que la figurine touche le terrain en question. Dans la plupart des cas, le moment où une figurine meurt importe peu, et il est donc plus aisé d'effectuer tous les tests de Terrain dangereux d'une unité en même temps. Les blessures dues aux tests de Terrain dangereux sont réparties sur l'unité de façon normale.}

Une figurine n'effectue jamais plus d'un seul test de Terrain Dangereux par élément de décor durant un mouvement, mais il se peut qu'elle doive effectuer plusieurs tests en raison de différents éléments de décor ou règles spéciales.

\subsubsection*{Décor occultant}

Aucune ligne de vue ne peut être tracée au travers d'un \emph{Décor Occultant}. Certains \emph{Décors Occultants}, tels que les collines, peuvent être occupés ou traversés par des figurines. Dans de tels cas, une ligne de vue peut être tracée depuis ou vers l'intérieur du décor, mais jamais au travers de celui-ci. 

\subsubsection*{Terrain et Couvert Léger}

Lorsqu'une figurine tire, sa cible bénéficie d'un \emph{Couvert Léger} si au moins la moitié de l'empreinte au sol de l'unité ciblée est occultée par un terrain offrant un \emph{Couvert Léger} (voir le paragraphe \ref{tir/modificateurs} à la page \pageref{tir/modificateurs}). \nouveau{Pour cette règle, les figurines placées dans un élément de terrain ne le prennent pas en compte.}

\subsubsection*{Terrain et Couvert Lourd}

Lorsqu'une figurine tire, sa cible bénéficie d'un \emph{Couvert Lourd} si au moins la moitié de l'empreinte au sol de l'unité ciblée est occultée par un terrain offrant un \emph{Couvert Lourd} (voir le paragraphe \ref{tir/modificateurs} à la page \pageref{tir/modificateurs}). \nouveau{Pour cette règle, les figurines placées dans un élément de terrain ne le prennent pas en compte.}

\section{Liste des terrains particuliers}

\subsection{Terrain découvert}

Les \emph{Terrains Découverts} n'ont généralement aucun effet sur les lignes de vue, les couverts ou le mouvement. Toute partie du champ de bataille qui n'est pas couverte par un autre élément de décor est considéré comme \emph{Terrain Découvert}.

\subsection{Terrain infranchissable}

\begin{itemize}[label={-}]
\item \textbf{Ligne de vue}. Un \emph{Terrain Infranchissable} est un \emph{Décor Occultant}.
\item \textbf{Couvert}. Un \emph{Terrain Infranchissable} offre un \emph{Couvert Lourd} en cas de couvert.
\item \textbf{Mouvement}. Aucune figurine ne peut se déplacer dans ou au travers d'un \emph{Terrain Infranchissable}. Si une charge amène une figurine à moins de 1{\pouce} d'un \emph{Terrain Infranchissable}, la figurine et son unité sont autorisées à y rester, jusqu'à la première fois où elles s'éloignent de plus de 1{\pouce}.
\end{itemize}

\subsection{\nouveau{Champs}}

\begin{itemize}[label={-}]
\item \textbf{Couvert}. Un \emph{Couvert Léger} est attribué aux unités dont au moins la moitié de l'empreinte au sol est dans un \emph{Champ}.
\item \textbf{Mouvement}. Les \emph{Champs} sont des \emph{Terrains Dangereux (1)} pour la \emph{Cavalerie}, la \emph{Cavalerie Monstrueuse} et les \emph{Chars}.
\item \textbf{Facile à embraser}. Toute unité dont au moins la moitié de l'empreinte au sol est dans un \emph{Champ} est \emph{Inflammable}. De plus, les figurines capables de faire des \emph{Attaques Enflammées}, venant de la figurine ou d'une arme, traitent les \emph{Champs} comme des \emph{Terrains Dangereux (1)}.
\end{itemize}

\subsection{Collines}

\begin{itemize}[label={-}]
\item \textbf{Ligne de vue}. \nouveau{Les \emph{Collines} sont des \emph{Décors Occultants}}.
\item \textbf{Couvert Lourd}. \nouveau{Une \emph{Colline} offre un \emph{Couvert Lourd} aux unités derrière elle. }
\item \textbf{Couvert Léger}. \nouveau{Si au moins la moitié de l'empreinte au sol d'une unité sur la \emph{Colline} est cachée à la ligne de vue du tireur, elle compte comme un \emph{Couvert Léger}.}
\item \textbf{Position élevée}. \nouveau{Toute figurine dont au moins la moitié du socle est sur une \emph{Colline} est considérée comme étant de \emph{Grande Taille} pour la détermination des lignes de vue et des couverts}.
\item \textbf{Position dominante}. Lors de la manche de corps à corps d'un tour où une unité charge, si plus de la moitié de son empreinte au sol était sur une \emph{Colline} au début de son mouvement de charge, et que moins de la moitié de son empreinte au sol est toujours sur la \emph{Colline} à la fin de ce mouvement, elle gagne un +1 au \emph{Résultat de Combat} additionnel.
\end{itemize}

\subsection{Forêts}

\begin{itemize}[label={-}]
\item \textbf{Couvert}. Une \emph{Forêt} offre un \emph{Couvert Léger} en cas de couvert.
\item \textbf{Mouvement}. Les \emph{Forêts} sont des \emph{Terrains Dangereux (1)} pour la \emph{Cavalerie}, la \emph{Cavalerie Monstrueuse}, les \emph{Chars} et les unités se déplaçant en volant.
\item \textbf{Pas de rangs}. Une unité dont au moins la moitié de l'empreinte au sol est dans une \emph{Forêt} ne peut jamais être \emph{Indomptable}.
\item \textbf{Tenaces}. À l'inverse, les \emph{Tirailleurs} et les \emph{Personnages} d'infanterie seuls sont \emph{Tenaces} si la majorité de leur empreinte au sol est dans une \emph{Forêt}.
\end{itemize}

\subsection{\nouveau{Ruines}}

\begin{itemize}[label={-}]
\item \textbf{Couvert}. Des \emph{Ruines} offrent un \emph{Couvert Lourd} en cas de couvert (pour les unités dont au moins la moitié de l'empreinte au sol occupe les Ruines), sauf pour les \emph{Grandes Cibles}.
\item \textbf{Mouvement}. Les \emph{Ruines} sont des \emph{Terrains Dangereux (1)} pour toutes les unités à l'exception des \emph{Tirailleurs}, et des Terrains Dangereux (2) pour la \emph{Cavalerie}, la \emph{Cavalerie Monstrueuse} et les \emph{Chars}.
\end{itemize}

\subsection{Eaux peu profondes}

\begin{itemize}[label={-}]
\item \textbf{Mouvement}. Aucune unité ne peut faire de \emph{Marche Forcée} si au moins une partie de ce mouvement se fait dans des \emph{Eaux peu Profondes}. Les \emph{Eaux peu Profondes} sont des \emph{Terrains Dangereux (1)} pour la \emph{Cavalerie}, la \emph{Cavalerie Monstrueuse} et les \emph{Chars}.
\item \textbf{Désorganisés}. \nouveau{Un rang au moins partiellement immergé ne compte jamais comme un \emph{Rang Complet}. Si la majorité d'une unité est dans des \emph{Eaux peu Profondes}, l'unité compte comme n'ayant aucun \emph{Rang Complet}. La règle \emph{Désorganisés} ne s'applique pas aux unités qui ont la règle spéciale \emph{Guide} ou \emph{Guide (Eaux peu Profondes)}}.
\end{itemize}

\subsection{Murs}

\begin{itemize}[label={-}]
\item \textbf{Couvert}. Un \emph{Mur} offre un \emph{Couvert Lourd} aux unités n'ayant pas la règle \emph{Grande Cible}, si la majorité du front de l'unité est positionné en défense du \emph{Mur}. Si l'unité qui la prend pour cible est dans l'arc arrière ou dans un arc latéral, utilisez le côté correspondant de l'unité pour déterminer si elle bénéficie d'un couvert.
\item \textbf{Mouvement}. Ignorez les \emph{Murs} pour les déplacements et le positionnement des unités. Un \emph{Mur} est un \emph{Terrain Dangereux (1)} pour la \emph{Cavalerie}, la \emph{Cavalerie Monstrueuse} et les \emph{Chars}.
\item \textbf{Combat}. Les figurines qui défendent un \emph{Mur} (voir ci-dessous) gagnent la règle \emph{Distrayant} contre les unités attaquant depuis l'arc de l'unité en défense où se situe le mur, lors de la première manche de corps à corps de l'unité qui charge.
\item \textbf{Défendre un mur}. \nouveau{Pour défendre un \emph{Mur}, une figurine doit être alignée et en contact socle à socle avec ce \emph{Mur}}.
\end{itemize}

\subsection{Bâtiments}

\begin{itemize}[label={-}]
\item \textbf{Ligne de vue}. Les \emph{Bâtiments} sont des \emph{Décors Occultants}.
\item \textbf{Couvert}. Un \emph{Bâtiment} offre un \emph{Couvert Lourd} en cas de couvert.
\item \textbf{Entrer dans un bâtiment}. Une unité composée entièrement d'\emph{Infanterie}, d'\emph{Infanterie Monstrueuse}, de \emph{Bêtes de Guerre}, de \emph{Bêtes Monstrueuses} ou de \emph{Nuées} peut entrer dans un bâtiment vide lors de l'étape des \emph{Autres Mouvements}. L'unité entière doit alors y entrer. Pour ce faire, il suffit qu'elle entre en contact avec le \emph{Bâtiment}, \nouveau{mais aucune figurine de l'unité ne peut parcourir une distance supérieure à trois fois son Mouvement. Mesurez la distance entre la position initiale de la figurine et le point le plus proche du \emph{Bâtiment}}. Une unité du type de troupes mentionnés ci-dessus peut également être déployée dans un \emph{Bâtiment} vide. Lorsqu'une unité occupe un \emph{Bâtiment}, les distances par rapport aux figurines à l'intérieur sont mesurées depuis les bords du \emph{Bâtiment}. Le centre de l'unité est alors le centre du \emph{Bâtiment}. (Notez que pour déployer une unité dans un \emph{Bâtiment}, ce \emph{Bâtiment} doit être complètement dans votre zone de déploiement. De même, un \emph{Bâtiment} doit être à plus de 18{\pouce} de toute unité ennemie pour qu'une unité ayant la règle \emph{Éclaireur} s'y déploie). Une unité dans un \emph{Bâtiment} est toujours considérée comme ayant un rang (mais pas un Rang Complet) et une colonne.
\item \textbf{Inflammables}. \nouveau{Les figurines dans un bâtiment gagnent la règle spéciale \emph{Inflammable} jusqu'à ce qu'elles sortent du bâtiment.}
\item \textbf{Quitter un bâtiment}. Une unité peut quitter un \emph{Bâtiment} lors de l'étape des \emph{Autres Mouvements}, à condition qu'elle n'y soit pas entrée pendant le même tour. Son rang arrière doit alors être placé en contact avec le \emph{Bâtiment}. Toutes ses figurines doivent se retrouver à une distance du \emph{Bâtiment} inférieure à deux fois leur Mouvement (l'unité peut adopter n'importe quelle formation). L'unité ne peut plus se déplacer durant cette phase. Une unité dans un \emph{Bâtiment} ne peut pas déclarer de charge. Un \emph{Bâtiment} peut aussi être quitté lors d'une fuite (à la suite d'un test de \emph{Panique} raté ou d'un combat perdu). Lorsque cela se produit, vérifiez d'abord la direction de fuite et déterminez le point où le centre de l’unité finirait son mouvement. Ensuite, placez le centre de l'unité sur ce point, avec son front dans la direction de la fuite tout en ayant une formation légale (en respectant les règles d'espacement entre les unités). Si cela est impossible, placez l'unité dans une formation légale, toujours avec son front dans la direction de la fuite, et déplacez l'unité dans sa direction de fuite, jusqu'à la limite où elle peut être placée.
\item \textbf{Tirer depuis un bâtiment}. Une unité tirant depuis un \emph{Bâtiment} compte comme étant de \emph{Grande Taille}. \nouveau{Pas plus de 15 figurines, ou 5 dans le cas d'\emph{Infanterie Monstrueuse}, peuvent tirer depuis un \emph{Bâtiment}.}
\item \textbf{Attaques à distance et bâtiments}. Les \emph{Bâtiments} offrent un \emph{Couvert Lourd} aux unités les occupant. Si un gabarit touche un \emph{Bâtiment}, on considère qu'il touche D6 figurines de l'unité qui l'occupe. Si le gabarit possède des règles différentes pour la figurine sous son centre, utiliser les règles du centre pour toutes les touches, sauf si le centre du gabarit n'est pas au dessus du \emph{Bâtiment}.
\item \textbf{Autres mouvements}. Un \emph{Bâtiment} est considéré comme un \emph{Terrain Infranchissable} pour toute unité qui n'y entre pas, ne le quitte pas ou n'y mène pas un assaut.
\item \textbf{Mener un assaut contre un bâtiment}. Les déclarations et résolutions de charge contre un \emph{Bâtiment} se font comme d'ordinaire, avec les exceptions suivantes. L'unité chargée ne peut pas fuir en réponse à la charge. Déplacez l'unité qui charge au contact du bâtiment, en l'alignant et en maximisant le contact comme d'habitude (le bâtiment devrait avoir un socle rectangulaire). Bien entendu, le bâtiment ne peut pas être bougé lors de l'alignement.
\item \textbf{Combat contre un bâtiment}. Les montures, de tout type, ne peuvent pas combattre lors du combat qui s'ensuit, et les figurines qui combattent ne peuvent pas bénéficier de \emph{Touches d'Impact}, de \emph{Lances de Cavalerie}, de \emph{Lances Légères} ou de bonus à leur sauvegarde liés à leur monture, tels que les règles \emph{Protection de la Monture} ou un \emph{Caparaçon}. Jusqu'à 10 figurines qui ne sont pas en contact socle à socle avec d'autres ennemis peuvent combattre, dans chaque camp, et sont qualifiées d'\emph{Attaquants} et de \emph{Défenseurs}. Les \emph{Chars}, l'\emph{Infanterie Monstrueuse}, la \emph{Cavalerie Monstrueuse} et les \emph{Bêtes Monstrueuses} comptent pour 3 figurines chacun pour ceci, tandis que les \emph{Monstres} et \emph{Monstres Montés} comptent pour 6. Ces figurines sont sélectionnées au début de chaque manche de corps à corps, en premier par le joueur \emph{Attaquant}, puis par le joueur \emph{Défenseur}.

Chaque figurine compte comme étant en contact socle à socle avec toutes les figurines ennemies désignées comme \emph{Attaquants} ou \emph{Défenseurs}, et peut répartir ses attaques entre elles. Les figurines ordinaires tuées peuvent être remplacées par d'autres figurines ordinaires, quel que soit leur camp. Cependant, les figurines non ordinaires ne peuvent pas être remplacées. Ainsi, si un \emph{Personnage} fait partie du combat et est tué avant de pouvoir attaquer, il ne peut pas être remplacé et seules 9 figurines de son camp pourront combattre.

À la fin du combat, calculez le \emph{Résultat de Combat} comme d'habitude, avec les exceptions suivantes. \nouveau{L'unité dans le \emph{Bâtiment} compte comme n'ayant aucun rang pour le \emph{Résultat de Combat} ou pour déterminer si elle est \emph{Indomptable}. Les étendards, ou Grande Bannière, ne comptent que si la figurine la portant fait partie des \emph{Attaquants} ou des \emph{Défenseurs} choisis pour cette manche de corps à corps}. À l'issue de la résolution du combat, les situations suivantes peuvent se produire :
\begin{itemize}
\item[$\bullet$] \nouveau{Si le résultat est une égalité ou si les \emph{Attaquants} ont gagné, mais que les \emph{Défenseurs} n'ont pas raté leur test de \emph{Moral} et fuient, les \emph{Attaquants} peuvent choisir de faire un \emph{Pivot Post-Combat} (en ignorant le \emph{Bâtiment}) et d'être repoussés à 1{\pouce} du \emph{Bâtiment}, ou de continuer l'assaut contre le \emph{Bâtiment}, auquel cas les deux unités seront encore engagées au corps à corps, et l'assaut se poursuivra lors de la \emph{Phase de Corps à Corps} suivante. Si les \emph{Attaquants} sont alors engagés par d'autres unités ennemies, ils doivent choisir de poursuivre l'assaut du \emph{Bâtiment}}.
\item[$\bullet$] \nouveau{Si les \emph{Attaquants} ont gagné et que tous leurs ennemis ont fui le combat ou ont été détruits, l'unité peut choisir d'entrer dans le \emph{Bâtiment}, si elle le peut, ou d'être repoussée à 1{\pouce} du \emph{Bâtiment} et d'effectuer un \emph{Pivot Post-Combat} (en ignorant le \emph{Bâtiment}). S'il s'agissait d'un combat multiple, les Attaquants peuvent également poursuivre une unité ennemie qui a fui \textbf{SI} elle n'occupait pas le \emph{Bâtiment} précédemment}.
\item[$\bullet$] \nouveau{Si les \emph{Défenseurs} ont gagné, les \emph{Attaquants} doivent réussir un test de \emph{Moral} comme d'habitude ou fuir le combat. L'unité occupant le \emph{Bâtiment} ne peut pas poursuivre. Si le test de \emph{Moral} est réussi, \newrule{les unités attaquantes (ainsi que toutes les unités engagées dans ce combat et qui ne défendent pas le \emph{Bâtiment})} sont repoussée à 1{\pouce} du \emph{Bâtiment} et les unités ne sont plus considérées comme engagées au corps à corps}.
\end{itemize}
\end{itemize}
