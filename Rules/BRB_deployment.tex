% Base sur la VO 0.11.9
% Relecture technique: 
% Relecture syntaxique: 

\part[Déploiement]{\nouveau{Déploiement}}

\subsection*{Étapes du déploiement}

Le déploiement suit les étapes décrites dans la table \ref{table/etapes_deploiement}.

\begin{table}[!htbp]
\centering
\begin{tabular}{c|l}
\textbf{1} & \nouveau{Déterminez qui commence à se déployer}. \tabularnewline
\textbf{2} & Déployez des unités chacun votre tour. \tabularnewline
\textbf{3} & Déterminez qui va essayer d'avoir le premier tour. \tabularnewline
\textbf{4} & Déployez les unités restantes. \tabularnewline
\textbf{5} & Déployez les Éclaireurs. \tabularnewline
\textbf{6} & Déplacez les unités d'\emph{Avant-Garde}. \tabularnewline
\textbf{7} & Autres règles et capacités. \tabularnewline
\textbf{8} & Lancez le dé pour le premier tour. \tabularnewline
\end{tabular}
\caption{\label{table/etapes_deploiement}Étapes du déploiement.}
\end{table}

\subsection*{Qui commence à se déployer ?}

\nouveau{Le joueur qui n'a pas choisi sa zone de déploiement décide quel joueur commence son déploiement}.

\subsection*{Cœur du déploiement}

Les joueurs alternent alors, déployant tour à tour leurs unités, entièrement dans leurs zones de déploiement respectives. À chacun de ses tours, un joueur doit déployer au moins une unité, \nouveau{mais peut décider d'en déployer autant  qu'il le désire}. Les unités de type \emph{Machine de guerre} comptent pour une seule unité en terme de déploiement, et doivent être déployées en même temps. Les \emph{Personnages} suivent le même traitement. \nouveau{Quand un joueur a déployé toutes ses unités, à l'exception des unités qui ne sont pas déployées selon les règles normales, comme les \emph{Éclaireurs} ou les unités suivant la règle \emph{Embuscade}, le joueur doit annoncer s'il souhaite jouer en premier ou en second}.

\subsection*{Déploiement des unités restantes}

Quand un joueur a déployé toutes ses unités et a décidé s'il tentera de commencer premier ou second, l'autre joueur peut déployer ses unités restantes. Comptez combien d'unités il lui reste, l'ensemble des \emph{Machines de Guerre} et l'ensemble des \emph{Personnages} comptant toujours chacun pour une unité. Cela représente le \og nombre d'unités non déployées \fg , lequel sera utilisé à la fin des préparatifs.

\subsection*{Autres règles et aptitudes}

Toutes les règles et capacités restantes décrites comme ayant lieu juste avant le début de la partie doivent être déclenchées à cette étape.

\subsection*{Qui joue en premier ?}

\nouveau{Une fois toutes les unités déployées, les deux joueurs lancent 1D6. Le joueur qui a fini de déployer en premier ajoute le \og nombre d'unités non déployées \fg{} au résultat de son jet de dé}.
\begin{itemize}[label={-}]
\item Si le joueur qui a fini de se déployer en premier obtient un résultat strictement plus haut, il doit jouer en premier ou en second, en suivant strictement son souhait annoncé auparavant.
\item Si le score est une égalité ou si le joueur qui a fini de se déployer en second obtient un résultat plus haut, il peut choisir quel joueur aura le premier tour.
\end{itemize}

