% Base sur la VO 0.11.9
% Relecture technique: 
% Relecture syntaxique: 

\part{Personnages}
\label{personnages}

Hormis dans le cas où cela est spécifiquement mentionné, toute figurine inscrite dans la section des héros ou des seigneurs est un \emph{Personnage}.

\section{Personnages isolés}

\nouveau{Tout \emph{Personnage} que vous souhaitez laisser seul peut être considéré comme une unité constituée d'une seule figurine. En conséquence, les règles normales des unités s'appliquent}.

\section{Rejoindre une unité}

Les \emph{Personnages} peuvent faire partie intégrante des autres unités en les rejoignant. Cela peut être réalisé en déployant le \emph{Personnage} dans une unité en début de partie ou en amenant le \emph{Personnage} en contact avec l'unité au cours de l'étape des \emph{Autres Mouvements}. Les unités engagées au corps à corps ou en fuite ne peuvent pas être rejointes.

Notez que les \emph{Personnages} peuvent rejoindre d'autres \emph{Personnages}, pour constituer une unité constituée exclusivement de \emph{Personnages}.

Quand un \emph{Personnage} rejoint une unité, il est immédiatement déplacé à une position règlementaire pour suivre la règle \emph{\nouveau{Au Premier Rang}}, entraînant éventuellement le déplacement d'autres figurines vers l'arrière. \nouveau{Lorsqu'on choisit la position d'un \emph{Personnage} à l'intérieur d'une unité, on peut le positionner librement sur tout emplacement règlementaire qu'il pourrait avoir atteint avec son Mouvement, en se déplaçant à travers l'unité qu'il a rejointe, éventuellement en prenant la place de figurines suivant aussi la règle \emph{Au Premier Rang}. Déplacez les figurines qui suivent la règle \emph{Au Premier Rang} aussi peu que possible, de manière à maintenir toutes les figurines dans des positions règlementaires. Si le \emph{Personnage} n'a pas assez de mouvement pour atteindre la position souhaitée, il se déplace de la plus petite distance possible, à partir de sa position initiale, de manière à atteindre une position règlementaire, et ne peut alors faire se déplacer que les figurines sans la règle \emph{Au Premier Rang}. Quand un \emph{Personnage} rejoint une unité qui ne possède qu'un seul rang, son propriétaire peut choisir soit de déplacer une figurine au second rang, soit d'élargir le front de l'unité en plaçant la figurine remplacée à l'une des extrémités du premier et seul rang}.

Si une unité est rejointe par un \emph{Personnage}, elle ne peut plus se déplacer au cours de la même étape des \emph{Autres Mouvements}. Une unité qui a été rejointe par un ou plusieurs \emph{Personnages} n'est cependant pas considérée comme ayant bougé (pour la \emph{Phase de Tir} par exemple). Le \emph{Personnage} qui a rejoint l'unité est quant à lui considéré comme ayant bougé.

Une fois qu'il a rejoint une unité, un \emph{Personnage} est considéré comme faisant partie intégrante de cette unité et suit toutes les règles auxquelles elle est astreinte.

\subsubsection*{Unité combinée détruite}

Si toutes les figurines ordinaires d'une unité combinée sont tués, laissant un ou plusieurs \emph{Personnages} sans troupes, le ou les \emph{Personnages} survivants forment toujours une unité, qui est considérée comme étant la même unité qu'auparavant pour tous les effets qui ont une durée (comme les sorts de type \emph{Dure un tour}), et pour la \emph{Panique} (pour ce dernier point, aucune unité n'est considérée comme détruite, mais il se peut les \emph{Personnages} doivent faire un test en raison de \emph{Lourdes Pertes}).

\subsubsection*{Quitter une unité combinée}

Un \emph{Personnage} peut quitter une unité combinée au cours de l'étape des \emph{Autres Mouvements} si l'unité est en mesure de se déplacer, c'est-à-dire si elle n'est pas engagée au corps à corps, ne s'est pas déjà déplacée au cours de cette étape, n'est pas en train de fuir, etc. Réalisez un test de \emph{Marche Forcée} pour l'ensemble de l'unité avant de bouger toute figurine si vous désirez en faire une. Une unité quittée par un ou plusieurs \emph{Personnages} n'est pas considérée comme ayant elle-même bougé, pour la \emph{Phase de Tir} par exemple. Lorsqu'il quitte son unité, un \emph{Personnage} peut se déplacer à travers celle-ci, et peut ainsi sortir de l'unité par n'importe quel bord et faire un mouvement de \emph{Vol} s'il a cette règle spéciale. Le \emph{Personnage} compte comme faisant partie intégrante de l'unité jusqu'à ce qu'il l'ait physiquement quittée. Cela  signifie qu'il est affecté par des éventuelles altérations du Mouvement touchant l'unité au cours de sa sortie. Si le \emph{Personnage} ne possède pas assez de mouvement pour être placé à moins de 1{\pouce} de l'unité, il ne peut pas quitter celle-ci. \newrule{Un \emph{Personnage} ne peut pas quitter et rejoindre la même unité durant la même phase de jeu.}

\subsubsection*{Charger hors d'une unité}

Un \emph{Personnage} peut aussi quitter une unité combinée en chargeant en dehors de celle-ci. Pour réaliser cela, il faut déclarer une charge avec un \emph{Personnage} de l'unité combinée, pendant l'étape de \emph{Déclaration des Charges}, comme à l'accoutumée. Si une telle manœuvre est réalisée, l'unité elle-même, ainsi que d'éventuels autres \emph{Personnages} de l'unité, ne peuvent pas déclarer de charge dans le même tour de joueur. \nouveau{Les tirs résultant des réactions à une charge de \emph{Personnage} sortant d'une unité combinée, du type \emph{Tenir sa Position et Tirer}, sont tous alloués au \emph{Personnage}}. Lorsqu'un \emph{Personnage} charge à partir d'une unité combinée, il utilise sa propre valeur de Mouvement, peut utiliser un mouvement de \emph{Vol} s'il en possède un, et est affecté par des éventuelles altérations du Mouvement touchant l'unité pour son mouvement de charge. Si la charge est réussie, déplacez le \emph{Personnage} en dehors de l'unité et chargez normalement. Pour ce mouvement de charge, le personnage peut ignorer les figurines de l'unité qu'il quitte. Si la charge est ratée, le \emph{Personnage} réalise un déplacement de \emph{Charge Ratée} qui le fait sortir de l'unité. Si le mouvement de \emph{Charge Ratée} est trop court pour positionner le Personnage à plus de 1{\pouce} de l'unité combinée, laissez le \emph{Personnage} à l'intérieur de l'unité. Celle-ci est considérée comme ayant raté une charge.

\subsubsection*{Attaques sur une unité combinée}

Quand des attaques sont infligées à une unité combinée, il y a deux possibilités pour distribuer les touches :
\begin{itemize}[label={-}]
\item \textbf{Tous les \emph{Personnages} sont du même \emph{Type de Troupe} que les figurines ordinaires de l'unité dans laquelle ils se trouvent, et il y a au moins 5 figurines ordinaires dans cette unité.} Dans ce cas, toutes les touches sont réparties sur les figurines ordinaires, les \emph{Personnages} n'étant pas affectés par les touches. Si le \emph{Personnage} est touché par un gabarit, \nouveau{cette touche est transférée à l'unité}.
\item \textbf{Certains Personnages sont d'un \emph{Type de Troupe} différent de celui des figurines ordinaires de l'unité dans laquelle ils se trouvent, ou il y a moins de 5 figurines ordinaires dans l'unité combinée.} \nouveau{Dans ce cas, le joueur qui réalise les attaques répartit équitablement les touches entre les figurines ordinaires et les \emph{Personnages} appartenant à un autre \emph{Type de Troupe}}. Une figurine ne peut pas subir une deuxième touche tant que toutes les figurines de l'unité n'en ont pas reçu une première, et ainsi de suite. Si un \emph{Personnage} est touché par un gabarit, il en subit les conséquences normalement. Notez que si l'unité a 5 figurines ordinaires ou plus, les \emph{Personnages} du même \emph{Type de Troupe} sont ignorés (aucune touche ne peut leur être allouée, y compris par un gabarit).
\end{itemize}

\subsection{\nouveau{Au premier rang}}
\label{au_premier_rang}

Tous les \emph{Personnages} et les figurines d'\emph{État-Major} ont la règle \emph{Au Premier Rang}. Les figurines qui suivent cette règle doivent \textbf{toujours} être placées le plus à l'avant possible de leur unité. Normalement, cela signifie qu'elles doivent être placées au premier rang, mais si le premier rang est déjà occupé par d'autres figurines suivant la règle \emph{Au Premier Rang}, elles sont placées au second rang, et ainsi de suite.

Lorsqu'on déplace une unité qui inclut des figurines avec la règle spéciale \emph{Au Premier Rang}, ces figurines peuvent être réorganisées dans une nouvelle position, toujours autant au devant que possible. Cela fait partie du mouvement. Cela peut se faire au cours d'un \emph{Mouvement Simple}, d'une \emph{Marche Forcée}, d'une \emph{Roue} ou d'une \emph{Reformation}, et compte dans la limite du mouvement permis par l'unité. Mesurez entre la position initiale de la figurine et sa position d'arrivée pour déterminer de quelle distance elle s'est déplacée.

Si une figurine avec la règle spéciale \emph{Au Premier Rang} quitte une unité ou est enlevée comme perte, l'espace inoccupé qu'elle laisse doit être comblé avec des figurines des autres rangs, si possible en déplaçant des figurines avec la règle \emph{Au Premier Rang} vers le front de l'unité, si cela les déplace bien vers des positions plus en avant. Si plus d'une figurine avec la règle \emph{Au Premier Rang} peut se déplacer vers l'avant, le propriétaire des figurines décide lesquelles de ces figurines sont déplacées vers l'avant. Si toutes les figurines avec la règle \emph{Au Premier Rang} sont déjà autant à l'avant que possible, comblez tout espace vide avec des figurines ordinaires venant de l'arrière.

Parfois, les figurines avec la règle \emph{Au Premier Rang} doivent être redistribuées de manière à ce que toutes les figurines de ce type soient autant à l'avant de l'unité que possible. Quand cela se produit, déplacez aussi peu de figurines que possible pour respecter la règle. Voir la figure \ref{figure/au_premier_rang} pour un exemple d'organisation d'unité avec des socles compatibles et incompatibles.

\subsubsection*{Socle compatible}
Si une figurine suivant la règle \emph{Au Premier Rang} n'a pas la même taille de socle que l'unité dans laquelle elle se trouve, mais que son socle est de la même taille qu'un multiple entier de socles de figurines de l'unité (on parle alors d'un \textbf{socle compatible}), par exemple un socle de 40x40{\milli\meter} dans une unité de figurines sur des socles de 20x20{\milli\meter}, le \emph{Personnage} est placé dans l'unité normalement, en déplaçant le nombre nécessaire de figurines. L'avant du socle du \emph{Personnage} doit être placé le plus à l'avant possible de l'unité pour respecter la règle \emph{Au Premier Rang}). La figurine est considérée comme étant dans tous les rangs que son socle remplit. Pour calculer le nombre de figurines dans chaque rang de l'unité, pour le décompte des \emph{Rangs Complets} ou pour obtenir la formation en \emph{Horde}, considérez que le \emph{Personnage} compte comme autant de figurines ordinaires que son socle, plus grand, occupe. \newrule{Une figurine ne peut pas rejoindre une unité ayant plus d'un rang et une largeur inférieure à celle de son socle. De même, une unité ne peut pas faire de reformation qui serait moins large que le socle d'un \emph{Personnage} de cette unité. Si une figurine a un socle compatible plus long que les figurines ordinaires d'une unité, celle-ci est autorisée à avoir plus d'un rang incomplet, à condition qu'ils ne soient formés que des parties arrières des socles plus longs.}

\subsubsection*{Socle incompatible}
Si, en revanche, une figurine avec la règle \emph{Au Premier Rang} ne possède pas un socle compatible, comme par exemple un socle de 50x50{\milli\meter} dans une unité de figurines sur des socles de 20x20{\milli\meter}, le socle est dit \textbf{incompatible} et le \emph{Personnage} est placé sur le côté de l'unité, son socle en contact avec celui des figurines de l'unité, et aligné avec le front de celle-ci. Un maximum de deux \emph{Personnages} avec des socles incompatibles peut rejoindre une même unité, en les plaçant de chaque côté. Ces figurines sont considérées comme faisant partie du premier rang de l'unité, mais ne sont pas prises en compte lorsqu'on compte le nombre de figurines dans chaque rang, pour le décompte des \emph{Rangs Complets} ou pour obtenir la formation en \emph{Horde}.

\begin{figure}[!htbp]
\centering
\def\svgwidth{12cm}
\input{au_premier_rang.pdf_tex}
\caption{Le \emph{Personnage} $ P_{1} $ a un socle incompatible et est placé à côté de l'unité. Les \emph{Personnages} $ P_{2} $ et $ P_{3} $ ont des socles compatibles et sont placés à l'intérieur de l'unité, le plus à l'avant possible. Cette unité est considérée comme ayant 3 \emph{Rangs Complets} : $ P_{1} $ n'est pas pris en compte, alors que $ P_{2} $ compte comme 2 figurines de large. \\
Quand le \emph{Personnage} $ P_{4} $ rejoint l'unité, le musicien (Mu) doit être déplacé sur le côté de manière à avoir toutes les figurines qui suivent la règle spéciale \emph{Au Premier Rang} le plus sur le devant de l'unité possible.}
\label{figure/au_premier_rang}
\end{figure}

\subsection{Faites place}

\nouveau{À la troisième étape d'une manche de corps à corps (voir le paragraphe \ref{etapes_manche_cac}), tout \emph{Personnage} placé au premier rang et qui n'est pas en contact socle à socle avec une unité ennemie peut être déplacé en contact avec un ennemi qui est en contact socle à socle avec l'avant de l'unité du \emph{Personnage}. Pour ceci, échangez les positions de ce \emph{Personnage} avec une ou des figurines non-\emph{Personnage} de son unité. Une figurine avec un socle incompatible ne peut jamais opérer un tel mouvement.}


\section{Le général}

\subsection{Choisir le général de l'armée}

Toutes les armées doivent avoir un \emph{Général} pour les diriger. Il doit être le \emph{Personnage} avec le plus haut Commandement dans votre armée ayant la possibilité d'être le \emph{Général}, hormis le \emph{Porteur de la Grande Bannière}. Si deux \emph{Personnages} ou plus se disputent le plus haut Commandement, vous êtes libre de choisir lequel des \emph{Personnages} est votre \emph{Général}, \nouveau{mais cela doit être mentionné dans votre liste d'armée}. Le \emph{Général} de l'armée possède la règle \emph{Présence Charismatique}.

\subsection{Présence charismatique}

Une figurine qui n'est pas en fuite avec la règle \emph{Présence Charismatique} donne son Commandement à toutes les unités alliées dans un rayon de 12{\pouce}. Les unités qui bénéficient de la \emph{Présence Charismatique} peuvent utiliser ce Commandement au lieu de leur propre Commandement, si elles le souhaitent. Les effets modifiant le Commandement du \emph{Général} sont pris en compte au moment où celui-ci transmet son Commandement.

\section{Le porteur de la grande bannière}

\subsection{Choisir le porteur de la grande bannière}

Certains \emph{Personnages} peuvent être promus \emph{Porteur de la Grande Bannière} parmi les options de leur livre d'armée. Une armée ne peut inclure qu'un seul \emph{Porteur de la Grande Bannière}. Le \emph{Porteur de la Grande Bannière} de l'armée possède la règle \emph{Tenez les Rangs}.

\subsection{Tenez les rangs}

Un Porteur de la Grande Bannière qui n'est pas en fuite et qui a la règle \emph{Tenez les Rangs} permet aux unités alliées se trouvant dans un rayon de 12{\pouce} de relancer leurs tests de Commandement ratés. \nouveau{Elles n'y sont pas obligées}.

\subsection{Bannière magique}

Si un \emph{Porteur de la Grande Bannière} a la possibilité de choisir des \emph{Objets Magiques}, il lui est permis d'acquérir une bannière magique. \nouveau{Cette bannière magique peut être soit comptabilisée comme faisant partie des objets magiques acquis par la figurine, dans la limite des points d’\emph{Objets Magiques} qui lui est autorisée (habituellement 50 pts pour un héros, par exemple), soit elle peut être prise sans aucune limitation de points, mais dans ce cas la figurine ne peut prendre aucun autre \emph{Objet Magique}}.

\subsection{Leur bannière est à terre}

Quand un \emph{Porteur de Grande Bannière} est retiré comme perte alors qu'il était engagé au corps à corps, on considère que la grande bannière est capturée par l'adversaire. \nouveau{Quand un \emph{Porteur de la Grande Bannière} fuit un corps à corps, on considère que la grande bannière est perdue, avec toute bannière magique associée et la règle \emph{Tenez les Rangs}, et la figurine qui était en possession de la grande bannière perd tous les effets bénéfiques engendrés par celle-ci. Elle est alors considérée comme capturée par l'armée adverse}.

\section{Défis}

\subsection{Lancer un défi}

Les \emph{Personnages} ou \emph{Champions} engagés dans un combat peuvent lancer un \emph{Défi}. À la quatrième étape d'une manche de corps à corps (voir \ref{etapes_manche_cac}), le joueur actif peut désigner l'un de ses \emph{Personnages} ou \emph{Champions} et lancer un \emph{Défi} avec ce dernier. S'il n'a pas relevé de \emph{Défi}, le joueur réactif peut alors désigner l'un de ses \emph{Personnages} ou \emph{Champions} et lancer lui-même un \emph{Défi}.

\subsection{Accepter ou refuser un défi}

Si un \emph{Défi} est lancé, le joueur adverse peut choisir l'un de ses propres \emph{Personnages} ou \emph{Champions} engagés dans le même combat pour relever le \emph{Défi} et combattre le \emph{Personnage} ou \emph{Champion} qui l'a lancé. La figurine qui accepte le \emph{Défi} doit se trouver dans une unité qui est en contact avec la figurine qui l'a lancé, \newrule{ou son unité}.

Si un \emph{Défi} n'est pas relevé, ce \emph{Défi} est dit refusé. Dans ce cas, le joueur qui a lancé le \emph{Défi} désigne l'un des \emph{Personnages} non-\emph{Champion} de son adversaire qui aurait pu accepter le \emph{Défi} (s'il y en a un, bien sûr). \nouveau{Le Commandement de cette figurine est réduit à 0 et elle ne plus utiliser la règle spéciale \emph{Tenace} (notez que son unité ne peut plus utiliser cette règle non plus si c'est le \emph{Personnage} qui lui transmet) jusqu'à la fin du tour au cours duquel le combat se termine, ou jusqu'à ce que ce \emph{Personnage} relève ou lance un \emph{Défi}. Elle ne peut par ailleurs faire aucune attaque de corps à corps au cours de cette manche de corps à corps. Si c'est le \emph{Porteur de la Grande Bannière}, il perd la règle \emph{Tenez les Rangs} et n'ajoute pas +1 au résultat de combat au cours de cette \emph{Phase de Corps à Corps}}.

\subsection{Relever un défi}

Si un \emph{Défi} est relevé, le \emph{Personnage} ou \emph{Champion} qui lance le \emph{Défi} et le \emph{Personnage} ou \emph{Champion} qui relève le \emph{Défi} \nouveau{sont considérés comme étant au contact socle à socle, même si leurs socles ne sont pas physiquement en contact}, et doivent diriger toutes leurs attaques contre leur vis-à-vis. Les attaques spéciales réalisées contre une unité, telles que les \emph{Piétinements}, les \emph{Attaques de Souffle}, les \emph{Touches d’Impact} et les \emph{Attaques de Broyage}, sont toujours dirigées contre le \emph{Personnage} ou \emph{Champion} engagé dans le duel. Dans le cas des \emph{Piétinements}, le \emph{Personnage} ou \emph{Champion} ciblé doit bien sûr appartenir à un \emph{Type de Troupe} qui peut  être piétiné et, dans le cas contraire, les attaques de \emph{Piétinements} sont perdues. Aucune autre figurine ne peut diriger des attaques contre les deux duellistes. Aucune attaque ou touche réalisée en dehors du \emph{Défi} ne peut être attribuée à l'une des deux figurines engagées dans un \emph{Défi}. Si un \emph{Personnage} ou \emph{Champion} avec des attaques étalées sur plusieurs paliers d'Initiative, comme par exemple un cavalier et sa monture, ou une figurine qui possède la règle spéciale \emph{Piétinement}, élimine son adversaire avant d'avoir pu faire toutes ses attaques, \nouveau{ces attaques restantes peuvent être dirigées sur la figurine tuée, comme si elle était encore vivante et en contact socle à socle, de façon à obtenir des points de \emph{Carnage}}.

Si l'une des deux figurines engagées dans un \emph{Défi} est éliminée, fuit le combat \newrule{ou si le combat s'achève pour une quelconque raison}, le \emph{Défi} est considéré comme terminé à la fin de la phase. Si aucune des figurines n'est tuée avant la prochaine manche de corps à corps, le \emph{Défi} continue. Aucun autre \emph{Défi} ne peut être lancé dans le même combat jusqu'à ce que l'un des deux protagonistes du duel soit tué.

\subsection{Carnage}

Pendant un \emph{Défi}, toutes les blessures en excédent sont comptabilisées dans le résultat de combat, jusqu'à un maximum de \nouveau{+3}.
