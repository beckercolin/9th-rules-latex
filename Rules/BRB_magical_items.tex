% Base sur la VO 0.11.9
% Relecture technique: 
% Relecture syntaxique: 


\part{Objets magiques}

\section{Catégories d'objets magiques}

Les Objets magiques sont répartis en 6 catégories :
\begin{itemize}[label={-}, itemsep=0.2cm]
\item \textbf{Armes magiques.}
\item \textbf{Armures magiques.}
\item \textbf{Talismans.}
\item \textbf{Objets enchantés.}
\item \textbf{Objets cabalistiques.}
\item \textbf{Bannières magiques.}
\end{itemize}

De plus, chaque type d'Objet magique possède les règles spécifiques suivantes.

\subsubsection*{Armes magiques}

Une figurine qui possède une Arme magique (comprenant les Armes de base magiques) \textbf{doit} l'utiliser, même si elle possède plusieurs autres armes. Une Arme de base magique ne peut pas être utilisée pour la règle \emph{Parade}. Si une Arme magique est détruite, elle devient son équivalent ordinaire.

\subsubsection*{Armures magiques}

Si une figurine possède une Armure du même type que son Armure magique, l'Armure magique la remplace. Par exemple, si une figurine possède une Armure lourde et qu'elle achète une Armure légère magique, l'Armure lourde standard est perdue. Un Bouclier magique ne peut pas être utilisé avec une Arme à une main pour la règle \emph{Parade}.

Les \emph{Sorciers} ne peuvent pas prendre d'Armure magique, à moins qu'ils ne possèdent déjà une Armure standard, ou l'option pour en avoir. Ils ont tout de même accès à la \emph{Protection innée}, à l'\emph{Armure naturelle} et aux montures et leur Caparaçon.

\subsubsection*{Talismans et objets enchantés}

Pas de règles additionnelles.

\subsubsection*{Objets cabalistiques}

Les Objets cabalistiques ne peuvent être choisis que par des \emph{Sorciers}.

\subsubsection*{Bannières magiques}

Seuls les Porte-étendards peuvent avoir accès aux Bannières magiques, a priori le \emph{Porte-étendard} d'une unité, le \emph{Porte-étendard Vétéran} d'une unité et le \emph{Porteur de la Grande Bannière}.

\subsection{Types d'objets magiques}
Tous les Objets magiques sont d'un type, leur équivalent ordinaire, et suivent donc toutes les règles et restrictions de ces objets standards en plus de leurs règles propres. Ainsi, une Lance de cavalerie magique donne un bonus de +2 en Force lors d'une charge et elle ne peut être portée que par une figurine montée, une Bête de Guerre ou une Bête Monstrueuse, comme une Lance de cavalerie standard. 

\subsection{Restrictions}

Tous les Objets magiques sont \emph{Uniques}, donc aucun Objet magique ne peut être possédé en double au sein d'une même armée. Aucune figurine ne peut porter plus d'un Objet magique d'une même catégorie, les catégories étant : \emph{Armes magiques}, \emph{Armures magiques}, \emph{Talismans}, \emph{Objets enchantés}, \emph{Objets cabalistiques} et \emph{Bannières magiques}. De plus, chaque type d'Objet magique peut obéir à des restrictions supplémentaires.

\subsection{Coût en points}

Le coût en points des Objets magiques est donné après leur nom. \nouveau{Certains Objets magiques ont deux coûts en points (par exemple, \emph{Épée de Géant} (\unit{60}{pts} / \unit{50}{pts})). La première valeur est le coût pour un \emph{Seigneur}, et la seconde pour un \emph{Héros} ou un Champion d'unité}.

\subsection{Qui est affecté ?}

Les Objets magiques font souvent référence à leur \emph{Porteur} dans leurs règles. Ces règles s'appliquent alors à la figurine pour laquelle l'objet a été sélectionné, \textbf{sans jamais inclure la monture}. D'autres Objets magiques font référence à la \emph{Figurine}, ou \nouveau{la \emph{Figurine} du porteur. Cela signifie que l'objet fonctionne pour l'intégralité de la figurine, y compris la monture. Remarquez que cette règle prend le pas sur les restrictions du \emph{Profil de Monstre Monté} à la page \pageref{profil_monstre_monte}}. Enfin une troisième catégorie d'objet comprend dans ses règles le mot-clé \emph{Unité}. Ces objets affectent toutes les figurines de l'unité, y compris les montures et le porteur.

\subsection{Usage Unique}

Les objets qui portent la mention \emph{Usage Unique} ne peuvent être utilisés qu'une fois \textbf{dans toute la partie}. Une fois utilisés, ils n'existent plus au vu du jeu.


\subsection{Objets magiques communs}

Les Objets magiques listés dans les paragraphes suivants sont considérés comme des Objets magiques communs et sont disponibles pour toutes les figurines ou unités qui ont la possibilité d'avoir accès aux Objets magiques. Il existe aussi souvent des Objets magiques spécifiques à chaque armée.

\newpage

\subsection{Armes magiques}

\begin{multicols}{2}
\begin{itemize}[label={-}]
\item \textbf{Épée de Géant} \dotfill \unit{60}{pts} / \unit{50}{pts} \\
\textit{Arme de base}. Les attaques effectuées avec cette arme ont un bonus de +3 en Force.

\item \textbf{Lame de Conflit} \dotfill \unit{45}{pts} \\
\textit{Arme de base}. Le porteur possède +3 Attaques.

\item \textbf{Trancheuse de Crânes} \dotfill \unit{40}{pts} \\
\textit{Arme de tir}. Portée \unit{24}{\pouce}, Force 4, \emph{Perforant (1)}, \emph{Tirs multiples (4)}, Touche toujours sur 4+.

\item \textbf{Épée Ogre} \dotfill \unit{40}{pts} \\
\textit{Arme de base}. Les attaques effectuées avec cette arme ont un bonus de +2 en Force.

\item \textbf{Épée d'Obsidienne} \dotfill \unit{35}{pts} \\
\textit{Arme de base}. Les attaques effectuées avec cette arme ont la règle spéciale \emph{Perforant (6)}.

\item \textbf{Épées d'Escrimeur} \dotfill \unit{35}{pts} \\
\textit{Paire d'armes}. Le porteur a une Capacité de Combat de 10. Figurine à pied uniquement.

\item \textbf{Tueuse de Rois} \dotfill \unit{30}{pts} \\
\textit{Arme de base}. Le porteur possède +1 en Force et +1 Attaque pour chaque \emph{Personnage} ennemi en contact socle à socle avec l'unité du porteur (le bonus est calculé et prend effet au niveau d'Initiative où sont effectuées les attaques de cette arme).

\item \textbf{Hache de Bataille} \dotfill \unit{25}{pts} \\
\textit{Arme de base}. Le porteur possède +2 Attaques.

\item \textbf{Hallebarde Fléau-des-bêtes} \dotfill \unit{25}{pts} \\
\textit{Hallebarde}. Les attaques effectuées avec cette arme sont de Force 5 (quels que soient les éventuels modificateurs) et ont la règle spéciale \emph{Blessures multiples (2)} contre les \emph{Bêtes monstrueuses}, la \emph{Cavalerie monstrueuse}, les \emph{Chars}, l'\emph{Infanterie monstrueuse}, les \emph{Monstres} et les \emph{Monstres montés}.

\item \textbf{Épée de Hâte} \dotfill \unit{25}{pts} \\
\textit{Arme de base}. Le porteur a Initiative 10. Les attaques effectuées avec cette arme ne peuvent jamais blesser plus difficilement que sur 4+ (un 5+ ou 6+ est ramené à 4+).

\item \textbf{Épée des Héros} \dotfill \unit{20}{pts} \\
\textit{Arme de base}. Les attaques effectuées avec cette arme ont un bonus de +1 en Force. Le porteur gagne aussi +1 Attaque. \newrule{Quand il attaque avec cette arme, le porteur ne peut pas avoir plus de 4 Attaques, ni une Force supérieure à 5 (les éventuels modificateurs ne peuvent plus s'appliquer).}

\item \textbf{L'Arracheur de Chair} \dotfill \unit{20}{pts} \\
\textit{Arme lourde}. Les attaques effectuées avec cette arme ont la règle spéciale \emph{Perforant (1)}.

\item \textbf{Épée Sacrée} \dotfill \unit{20}{pts} \\
\textit{Arme de base}. Les attaques effectuées avec cette arme possèdent la règle spéciale \emph{Attaques Divines} et ont \emph{Relance des jets pour blesser ratés}.

\item \textbf{Épée de Force} \dotfill \unit{15}{pts} \\
\textit{Arme de base}. Les attaques effectuées avec cette arme ont un bonus de +1 en Force.

\item \textbf{Épée de Frappe} \dotfill \unit{15}{pts} \\
\textit{Arme de base}. Les attaques effectuées avec cette arme ont un bonus de +1 pour toucher.

\item \textbf{Lance Enflammée} \dotfill \unit{10}{pts} \\
\textit{Lance de cavalerie}. Les attaques effectuées avec cette arme ont la règle spéciale \emph{Attaques Enflammées}.

\item \textbf{Épées Hurlantes} \dotfill \unit{10}{pts} \\
\textit{Paire d'armes}. Le porteur provoque la \emph{Peur}.

\item \textbf{Épée Tranchante} \dotfill \unit{5}{pts} \\
\textit{Arme de base}. Les attaques effectuées avec cette arme ont la règle spéciale \emph{Perforant (1)}.

\end{itemize}
\end{multicols}

\newpage

\subsection{Armures magiques}

\begin{multicols}{2}
\begin{itemize}[label={-}]
\item \textbf{Bouclier à Champ de Force} \dotfill \unit{70}{pts} \\
\emph{Bouclier}. La figurine du porteur possède une \emph{Sauvegarde Invulnérable (5+)} contre les attaques à distance.

\item \textbf{Armure du Destin} \dotfill \unit{50}{pts} \\
\emph{Armure lourde}. Le porteur possède une \emph{Sauvegarde Invulnérable (4+)}.

\item \textbf{Heaume de Tromperie} \dotfill \unit{35}{pts} \\
\emph{Aucun type}. Sauvegarde d'Armure de 6+. Les jets pour blesser réussis contre le porteur doivent être relancés. Ne peut pas être pris par une \emph{Grande Cible}.

\item \textbf{Cotte de Maille en Mithril} \dotfill \unit{35}{pts} / \unit{25}{pts} \\
\emph{Armure lourde}. Figurine à pied uniquement. Sauvegarde d'Armure de 2+. Cette Sauvegarde d'Armure ne peut pas être améliorée.

\item \textbf{Cuirasse Étincelante} \dotfill \unit{30}{pts} \\
\emph{Armure lourde}. Le porteur gagne la règle spéciale \emph{Distrayant}.

\item \textbf{Armure de Bonne Fortune} \dotfill \unit{25}{pts} \\
\emph{Armure lourde}. Le porteur possède une \emph{Sauvegarde Invulnérable (5+)}.

\item \textbf{Plastron de Bronze} \dotfill \unit{25}{pts} \\
\emph{Armure lourde}. \emph{Usage unique}. À n'importe quel moment où la figurine du porteur reçoit une touche, l'objet peut être activé. Pour le restant de la phase, la figurine du porteur possède une Sauvegarde d'Armure de 1+. Si la figurine du porteur a la règle spéciale \emph{Grande Cible}, elle gagne une Sauvegarde d'Armure de 2+ à la place.

\item \textbf{Cape en Cuir de Dragon} \dotfill \unit{25}{pts} \\
Le porteur a la règle spéciale \emph{Protection innée (5+)}. Figurine à pied uniquement.

\item \textbf{Armure du Parieur} \dotfill \unit{15}{pts} \\
\emph{Armure lourde}. Le porteur possède une \emph{Sauvegarde Invulnérable (6+)}.

\item \textbf{Heaume en Écailles de Dragon} \dotfill \unit{10}{pts} \\
\emph{Aucun type}. Sauvegarde d'Armure de 6+. Le porteur possède la règle spéciale \emph{Né du feu}.

\item \textbf{Bouclier Renforcé} \dotfill \unit{5}{pts} \\
\emph{Bouclier}. Ajoutez +1 en Sauvegarde d'Armure du porteur en plus du bonus de Bouclier, pour un total de +2, \newrule{si le porteur utilise ce bouclier}.

\item \textbf{Bouclier de Chance} \dotfill \unit{5}{pts} \\
\emph{Bouclier}. Ignorez la première touche subie par la figurine du porteur \newrule{s'il utilise ce bouclier} (si le porteur est touché par plusieurs attaques simultanément, il choisit quelle attaque ignorer).

\end{itemize}
\end{multicols}

\newpage

\subsection{Talismans}

\begin{multicols}{2}
\begin{itemize}[label={-}]
\item \textbf{Talisman de Protection Suprême} \\ .\dotfill \unit{50}{pts} \\
Le porteur possède une \emph{Sauvegarde Invulnérable (4+)}.

\item \textbf{Graine de Renaissance} \dotfill \unit{50}{pts} \\
Le porteur a la règle spéciale \emph{Régénération (4+)}.

\item \textbf{Pierre d'Annulation d'Obsidienne} \\ .\dotfill \unit{45}{pts} \\
Le porteur a la règle spéciale \emph{Résistance à la magie (3)}.

\item \textbf{Pierre du Crépuscule} \dotfill \unit{30}{pts} \\
Le porteur peut relancer ses jets de sauvegarde d'armure ratés.

\item \textbf{Gemme d'Obsidienne} \dotfill \unit{30}{pts} \\
Le porteur a la règle spéciale \emph{Résistance à la magie (2)}.

\item \textbf{Talisman de Protection Majeure} \dotfill \unit{25}{pts} \\
Le porteur possède une \emph{Sauvegarde Invulnérable (5+)}.

\bigskip\medskip
\item \textbf{Amulette de Pierres Précieuses} \dotfill \unit{15}{pts} \\
\emph{Usage Unique}. Peut être activé quand la figurine du porteur rate une Sauvegarde d'Armure, quand elle ne peut pas l'utiliser, ou subit une blessure et n'a pas de Sauvegarde d'Armure. Le porteur possède une \emph{Sauvegarde Invulnérable (4+)} contre cette blessure.

\item \textbf{Éclat d'Obsidienne} \dotfill \unit{10}{pts} \\
Le porteur a la règle spéciale \emph{Résistance à la magie (1)}.

\item \textbf{Talisman de Protection} \dotfill \unit{5}{pts} \\
Le porteur possède une \emph{Sauvegarde Invulnérable (6+)}.

\item \textbf{Charme de Chance} \dotfill \unit{5}{pts} \\
\emph{Usage Unique}. Peut être activé quand la figurine du porteur rate une Sauvegarde d'Armure. Le porteur peut relancer cette Sauvegarde d'Armure.

\item \textbf{Gemme de Feu de Dragon} \dotfill \unit{5}{pts} \\
Le porteur possède la règle spéciale \emph{Né du feu}.

\end{itemize}
\end{multicols}

\newpage

\subsection{Objets enchantés}

\begin{multicols}{2}
\begin{itemize}[label={-}]
\item \textbf{Tapis Volant} \dotfill \unit{40}{pts} / \unit{30}{pts} \\
Le porteur possède la règle spéciale \emph{Vol (6)}. Figurine à pied uniquement.

\item \textbf{Capuche de Magicien} \dotfill \unit{40}{pts} \\
Le porteur possède la règle spéciale \emph{Stupide}. Si le porteur n'avait pas de niveau de magie, il devient un \emph{Sorcier} de niveau 2. Au lieu de choisir sa \emph{Discipline Magique} normalement, le porteur tire au sort une des 8 \emph{Disciplines} de base au début de la partie.

\item \textbf{Couronne de Moquerie} \dotfill \unit{35}{pts} \\
\emph{Usage Unique}. Au lieu de tenter un jet de dissipation, vous pouvez utiliser cet objet. Le sort est automatiquement dissipé. Cet objet ne peut être sélectionné que dans une armée ne contenant aucun \emph{Sorcier}.

\item \textbf{Boule de Cristal} \dotfill \unit{35}{pts} \\
Le porteur possède la règle spéciale \emph{Réflexes Foudroyants}.

\item \textbf{Sceptre de Domination} \dotfill \unit{30}{pts} \\
\emph{Usage Unique}. Peut être activé au début de n'importe quelle \emph{Phase de Corps à Corps}. Pour la durée de la phase, le porteur possède la règle spéciale \emph{Tenace}.

\item \textbf{Anneau de Feu} \dotfill \unit{25}{pts} \\
\emph{Objet de sort}. Niveau de pouvoir 3. Contient le sort \emph{Boule de feu} (Discipline Magique du Feu).

\item \textbf{Gemme de Chance} \dotfill \unit{25}{pts} \\
Les attaques à distance contre l'unité du porteur doivent relancer tout \result{6} naturel obtenu lors du jet pour blesser. Cet objet ne fonctionne pas sur une \emph{Grande cible}.

\item \textbf{Potion de Force} \dotfill \unit{20}{pts} \\
\emph{Usage Unique}. Peut être activé au début de n'importe quelle phase \newrule{ou round de combat}. Pour la durée du tour du joueur, le porteur possède +2 en Force.

\item \textbf{Icône de Fer Maudit} \dotfill \unit{20}{pts} \\
Le porteur et son unité possèdent une \emph{Sauvegarde Invulnérable (5+)} contre les blessures causées par des \emph{Armes d'Artillerie}.

\item \textbf{Statuette Divine} \dotfill \unit{15}{pts} \\
La figurine du porteur possède la règle spéciale \emph{Attaques Divines}.

\item \textbf{Potion de Rapidité} \dotfill \unit{5}{pts} \\
\emph{Usage Unique}. Peut être activé au début de n'importe quelle phase \newrule{ou round de combat}. Pour la durée du tour du joueur, le porteur possède +3 en Initiative.

\end{itemize}
\end{multicols}

\newpage

\subsection{Objets cabalistiques}

\begin{multicols}{2}
\begin{itemize}[label={-}]
\item \textbf{Parchemin de Retour de Flamme} \\ .\dotfill \unit{55}{pts} \\
\emph{Usage Unique}. Au lieu de tenter un jet de dissipation, vous pouvez utiliser ce parchemin. Après que les effets du sort et de l'\emph{Attribut de la Discipline} ont été appliqués, et à moins que le sort ait été lancé avec un \emph{Pouvoir irrésistible}, le lanceur du sort doit faire un jet sur la table des \emph{Fiascos}. Cet objet n'a aucun effet contre les \emph{Objets de sort} et les sorts lancés avec un seul dé de pouvoir.

\item \textbf{Grimoire de Puissance Cabalistique} \\ .\dotfill \unit{50}{pts} / \unit{35}{pts} \\
Le porteur reçoit un bonus de +1 pour lancer et dissiper les sorts.

\item \textbf{Essence de Libre Pensée} \dotfill \unit{40}{pts} / \unit{30}{pts} \\
Le porteur peut inscrire 2 \emph{Disciplines Magiques} sur sa liste d'armée au lieu d'une seule \newrule{parmi celles dont il a accès}. Il peut choisir celle qu'il utilisera, au début de la partie, avant la génération des sorts.

\item \textbf{Parchemin de Dissipation} \dotfill \unit{35}{pts} \\
\emph{Usage Unique}. Au lieu de tenter un jet de dissipation, vous pouvez utiliser ce parchemin. Le sort est automatiquement dissipé.

\item \textbf{Grimoire de Connaissance Mystique} \\ .\dotfill \unit{25}{pts} / \unit{15}{pts} \\
Le porteur \newrule{génère} un sort supplémentaire.

\item \textbf{Baguette Tellurique} \dotfill \unit{25}{pts} / \unit{15}{pts} \\
\emph{Usage Unique}. Le porteur peut choisir d'utiliser cet objet pour relancer un jet sur la table des \emph{Fiascos}.

\item \textbf{Parchemin d'Entrave} \dotfill \unit{20}{pts} \\
\emph{Usage Unique}. Au lieu de tenter un jet de dissipation, vous pouvez utiliser ce parchemin. Le \emph{Sorcier} lançant le sort ne peut pas lancer ce sort au prochain tour de jeu.

\item \textbf{Sceptre de Pouvoir} \dotfill \unit{20}{pts} / \unit{15}{pts} \\
\emph{Usage Unique}. Le porteur peut relancer un unique dé de pouvoir lors du lancement d'un sort.

\item \textbf{Parchemin de Protection} \dotfill \unit{15}{pts} \\
\emph{Usage Unique}. Au lieu de tenter un jet de dissipation, vous pouvez utiliser ce parchemin. Toutes les figurines ciblées par ce sort ont une \emph{Sauvegarde Invulnérable (4+)} contre ce sort.

\item \textbf{Baguette de Stabilité} \dotfill \unit{15}{pts} \\
\emph{Usage Unique}. Le porteur peut augmenter un seul de ses jets de dissipation d'un modificateur de +1D6 (ce n'est pas un dé de dissipation) et ignorer la règle \emph{Pas assez de puissance !} (c'est une exception à la limitation des Modificateurs magiques).

\end{itemize}
\end{multicols}

\newpage

\subsection{Bannières magiques}

\begin{multicols}{2}
\begin{itemize}[label={-}]
\item \textbf{Bannière de Rasoirs} \dotfill \unit{45}{pts} \\
L'unité \newrule{du porteur} possède la règle spéciale \emph{Perforant (1)}.

\item \textbf{Bannière du Gardien} \dotfill \unit{40}{pts} \\
L'unité \newrule{du porteur} possède les règles spéciales \emph{Guide} et \emph{Rapide}.

\item \textbf{Icône d'éther} \dotfill \unit{30}{pts} \\
L'unité \newrule{du porteur} peut dissiper des sorts comme si elle était un \emph{Maître sorcier}.

\item \textbf{Icône Divine} \dotfill \unit{30}{pts} \\
L'unité du porteur possède la règle spéciale \emph{Attaques Divines}, mais doit relancer ses propres jets de \emph{Sauvegarde Invulnérable} réussis.

\item \textbf{Bannière de Discipline} \dotfill \unit{25}{pts} \\
L'unité du porteur possède +1 en Commandement.

\item \textbf{Bannière de Vitesse} \dotfill \unit{25}{pts} \\
L'unité du porteur gagne +1 en Mouvement.

\item \textbf{Bannière de Feu} \dotfill \unit{20}{pts} \\
Au début de la phase de tir de chaque joueur, le pouvoir de la bannière peut être activé. Dans ce cas, chaque attaque non spéciale de corps à corps ou de tir de l'unité du porteur gagne la règle spéciale \emph{Attaques Enflammées}.

\item \textbf{Bannière de Guerre} \dotfill \unit{15}{pts} \\
L'unité du porteur ajoute +1 au \emph{Résultat de combat} dans tout corps à corps où elle est impliquée.

\medskip
\item \textbf{Icône de la Compagnie Implacable} \\ .\dotfill \unit{15}{pts} \\
\emph{Usage Unique}. Vous pouvez activer cette bannière pendant n'importe lequel de vos tours, au début de l'étape des \emph{Autres Mouvements}. L'unité du porteur peut tripler son mouvement en \emph{Marche Forcée} au lieu de le doubler pour ce tour (jusqu'à un maximum de \unit{15}{\pouce}). Cette bannière ne peut pas être utilisée au tour 1 si l'unité a utilisé sa règle spéciale \emph{Avant-garde} ou \emph{Éclaireurs}. Figurines d'\emph{Infanterie} uniquement.

\item \textbf{Icône Étincelante} \dotfill \unit{5}{pts} \\
\emph{Usage Unique}. Doit être activée la première fois que l'unité du porteur rate un test de Commandement. L'unité peut alors choisir de relancer ce test raté. L'Icône Étincelante est perdue si le test de Commandement est relancé grâce à une autre capacité, comme celle de la Grande Bannière.

\item \textbf{Bannière de Courage} \dotfill \unit{5}{pts} \\
L'unité du porteur réussit automatiquement les tests de \emph{Terreur} et est immunisée aux effets de la règle spéciale \emph{Peur}.

\end{itemize}
\end{multicols}
