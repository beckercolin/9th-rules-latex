% Base sur la VO 0.11.9
% Relecture technique: 
% Relecture syntaxique: 

\part{Logistique du champ de bataille}

\section{Distances}

L'unité de mesure dans \emph{Batailles Fantastiques : Le 9\ieme Âge} est le pouce anglais, ou \og pouce technique international \fg , et se note \pouce. Il vaut \unit{2,54}{\centi\meter} exactement. Toutes les distances et portées sont indiquées et mesurées en pouces. Pour déterminer la distance entre deux objets sur le champ de bataille (des unités, décors ou autres éléments), vous devez toujours choisir les points au sol les plus proches des deux objets, même si cela traverse un quelconque obstacle : ignorez-le pour la mesure. Entre deux figurines, les mesures se font donc bout de socle à bout de socle.

Les règles se réfèrent souvent à des objets étant à une certaine portée. Dans ce cas, mesurez la distance entre leurs points les plus proches. Si cette distance est inférieure à la portée donnée, les objets sont considérés comme étant à portée. Notez que cela signifie qu'une figurine est toujours à portée d'elle-même et qu'une figurine ou une unité n'a pas besoin d'être \textbf{entièrement} à portée, simplement une partie de celle-ci.
 
Les joueurs sont autorisés à mesurer n'importe quelle distance à tout moment.

\section{Champ de vision}

\nouveau{Une figurine a une ligne de vue sur sa cible (point ou unité) si vous pouvez tracer une ligne droite depuis l'avant de son socle directement jusqu'à la cible, sans sortir de l'arc frontal de la figurine, et sans être bloqué par un \emph{Décor Occultant} ou par le socle d'une figurine qui a une taille \textbf{plus grande} que la figurine et sa cible \textbf{à la fois}}. Les figurines des rangs arrières ont toujours le même champ de vision que si elles étaient au premier rang dans la même colonne. Une unité est considérée comme ayant une ligne de vue sur une cible si une ou plusieurs figurines de l'unité ont une ligne de vue sur cette cible. \nouveau{Les figurines d'une unité ne bloquent jamais le champ de vision des autres figurines de cette unité}.

\subsection{Hauteur des figurines}
\label{hauteur_figs}

\nouveau{Les figurines peuvent avoir trois tailles.
\begin{description}%[label={-}]
\item [Petite :] Figurines ayant comme type de troupe \emph{Bête de guerre}, \emph{Infanterie}, \emph{Machine de guerre} ou \emph{Nuée}.
\item [Moyenne :] Figurines ayant comme type de troupe \emph{Bête Monstrueuse}, \emph{Cavalerie}, \emph{Cavalerie Monstrueuse}, \emph{Char} ou \emph{Infanterie Monstrueuse}.
\item [Grande :] Figurines ayant la règle spéciale \emph{Grande Cible} (peu importe leur type de troupe).
\end{description}
}

\section{Règle du pouce d'écart}

Les unités doivent toujours laisser un espace d'au moins \unit{1}{\pouce} entre elles, et avec les Terrains Infranchissables. Ceci s'applique aux unités alliées comme adverses. \nouveau{Pendant un déplacement, cette distance est réduite à \unit{0.5}{\pouce}, mais à la fin du déplacement, l'unité doit toujours respecter la \emph{Règle du Pouce d'Écart}. Si ce n'est pas le cas, revenez en arrière dans le mouvement jusqu'à ce que l'espace de \unit{1}{\pouce} soit respecté}.

Quelques déplacements spéciaux autorisent une entorse à la \emph{Règle du Pouce d'Écart}, comme les charges. \nouveau{Quand une unité se retrouve à moins de \unit{1}{\pouce} d'une autre unité ou d'un \emph{Terrain Infranchissable} suite à une charge, elle peut ignorer la \emph{Règle du Pouce d'Écart} par rapport à cette unité ou ce \emph{Terrain Infranchissable} aussi longtemps qu'elle reste à moins de \unit{1}{\pouce}. Elle ne peut cependant jamais entrer en contact socle à socle avec une unité ennemie qu'elle n'a pas chargée}.

\section{Bords de table}

Les bords de table correspondent aux limites du champ de bataille. Les figurines peuvent se retrouver temporairement en dehors de cette limite durant un déplacement, du moment que pas plus de la moitié du socle de la figurine n'est en dehors à n'importe quel moment, et tant qu'aucune partie de la figurine ne se retrouve en dehors à la fin du déplacement. Les \emph{Gabarits} et équivalents peuvent être placés partiellement en dehors du champ de bataille, ils comptent toujours comme affectant les figurines avec leur partie dans le champ de bataille.

