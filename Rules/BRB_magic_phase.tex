% Base sur la VO 0.11.9
% Relecture technique: 
% Relecture syntaxique: 

\part{Phase de magie}

Lors de la \emph{Phase de Magie}, vos \emph{Sorciers} peuvent lancer des sorts et votre adversaire peut tenter de les dissiper.

\section{Sorciers}

Les figurines pouvant lancer des sorts (non liés à des Objets de sort) sont appelées \emph{Sorciers}. Tous les \emph{Sorciers} ont un niveau représentant leur maîtrise des arcanes. 

\subsubsection*{Niveau de magie} 

Le niveau de magie d'un \emph{Sorcier} indique le nombre de sorts qu'il connaît. Un \emph{Sorcier} de niveau 1 connaît 1 sort, un \emph{Sorcier} de niveau 2 connaît 2 sorts, etc. Si un \emph{Sorcier} perd des niveaux de magie, il perd un sort par niveau. Sauf mention contraire, les sorts perdus sont choisis aléatoirement. Si le niveau d'un sorcier augmente ou diminue pour une quelconque raison, ses bonus de lancement et de dissipation changent aussi (voir les paragraphes Sorcier apprenti et Maître sorcier). Si le niveau du \emph{Sorcier} tombe en dessous de 1, il est considéré comme un \emph{Sorcier} de niveau 0. Les \emph{Sorciers} de niveau 0 sont toujours des \emph{Sorciers}.

\subsubsection*{Sorcier apprenti} 

\nouveau{Tous les \emph{Sorciers} de niveau 1 et 2 sont appelés des \emph{Sorciers apprentis}. Ils ajoutent 1 à leurs lancements ou dissipations de sorts}.

\subsubsection*{Maître sorcier} 

\nouveau{Tous les \emph{Sorciers} de niveau 3 et 4 sont appelés des \emph{Maîtres sorciers}. Ils ajoutent 2 à leurs lancements ou dissipations de sorts.}


\section{Sorts}

Les sorts peuvent être lancés durant la \emph{Phase de Magie}. Les sorts connus par un \emph{Sorcier} sont normalement déterminés avant le début de la partie, de façon  aléatoire, en utilisant les règles de génération des sorts (voir \ref{generation_sorts} à la page \pageref{generation_sorts}). La plupart des sorts font partie des \emph{Disciplines Magiques} décrites dans \emph{Batailles Fantastiques : Le 9\ieme Âge, Disciplines Magiques}. Chacun de vos \emph{Sorciers} peut choisir une des \emph{Disciplines} auxquelles il a accès et générer ses sorts à partir de là. La \emph{Discipline} choisie doit être inscrite dans la liste d'armée. Tous les sorts sont définis par les cinq critères suivants.

\subsubsection*{Nom} 

C'est le nom du sort, à utiliser pour annoncer le sort que vous allez tenter de lancer.

\subsubsection*{Valeur de lancement} 

C'est la valeur minimale à obtenir pour lancer le sort avec succès. Les sorts peuvent avoir différentes valeurs de lancement (voir la partie \ref{amelioration_sort} Amélioration de sort à la page \pageref{amelioration_sort}).

\subsubsection*{Type}

Le \emph{Type} du sort détermine quelles cibles peuvent être choisies. Un sort peut avoir plusieurs types, et leurs restrictions se cumulent. Par exemple, un sort de type \emph{Portée 12{\pouce}}, \emph{Malédiction} et \emph{Direct} ne peut cibler que les unités ennemies à moins de 12{\pouce} dans l'arc frontal du \emph{Sorcier}. Sauf mention contraire, un sort ne peut viser qu'une cible.

\subsubsection*{Durée}

La \emph{Durée} d'un sort indique le temps pendant lequel ses effets restent actifs.

\subsubsection*{Effet}

L'\emph{Effet} d'un sort décrit ce qui doit être appliqué dans le cas où le sort est lancé avec succès et non dissipé. \nouveau{L'\emph{Effet} d'un sort n'est jamais affecté par des règles spéciales, des objets magiques, d'autres effets de sorts ou tout type de règles donnant des avantages au lanceur, sauf indication contraire.}

\subsection{Amélioration de sort}
\label{amelioration_sort}
Certains sorts ont plus d'une Valeur de lancement. La ou les valeurs plus élevées correspondent à des versions améliorées du sort. Une version améliorée d'un sort peut en changer les effets, ou en modifier les restrictions, comme la portée ou les cibles. Ces modifications seront clairement indiquées pour chaque sort qui possède une ou des versions améliorées. Avant de jeter les dés pour tenter de lancer un sort, déclarez si vous souhaitez utiliser une version améliorée du sort, et laquelle. Sans déclaration de votre part, on considère toujours que le sort n'est pas amélioré.

\subsection{Types de sorts}

Le \emph{Type} du sort détermine les restrictions sur la ou les cibles du sort.

\subsubsection*{\nouveau{Amélioration}}

\nouveau{Les sorts d'\emph{Amélioration} ne peuvent cibler que des unités amies (ou une figurine amie (y compris dans une unité) si le sort est aussi \emph{Focalisé})}.

\subsubsection*{\nouveau{Aura}}

\nouveau{Les sorts d'\emph{Aura} ont un effet de zone. Si vous lancez un sort d'\emph{Aura}, toutes les cibles réglementairement possibles sont touchées par les sorts. Par exemple, un sort d'\emph{Aura} de type \emph{Amélioration} avec une \emph{Portée} de 12{\pouce} affectera toutes les unités amies à moins de 12{\pouce}}.

\subsubsection*{\nouveau{Dégâts}}

\nouveau{Les sorts de type \emph{Dégâts} ne peuvent cibler que des figurines qui ne sont pas engagées au corps à corps}.

\subsubsection*{\nouveau{Direct}}

\nouveau{Les sorts de type \emph{Direct} ne peuvent cibler que des figurines se trouvant dans l'arc frontal du lanceur}.

\subsubsection*{\nouveau{Focalisé}}

\nouveau{Les sorts de type \emph{Focalisé} ne peuvent cibler qu'une seule figurine, y compris un \emph{Personnage} à l'intérieur d'une unité. Si la cible est une figurine composée de plusieurs éléments (comme un \emph{Char} avec 2 auriges et 2 chevaux de trait ou un chevalier et sa monture), seul un des éléments peut être pris pour cible}.

\subsubsection*{Gabarit linéaire}

Pour définir qui est touché par un sort de type \emph{Gabarit linéaire}, tracez une ligne droite depuis le centre de l'avant du socle du lanceur, dans la direction de votre choix. \nouveau{Toutes les figurines touchées par cette ligne sont affectées par le sort}. Cette ligne droite compte comme un \emph{Gabarit}.

\subsubsection*{Lanceur}

Les sorts de type \emph{Lanceur} ne ciblent que la figurine qui lance le sort.

\subsubsection*{\nouveau{Malédiction}}

\nouveau{Les sorts de type \emph{Malédiction} ne peuvent cibler que des unités ennemies (ou une figurine ennemie (y compris dans une unité) si le sort est aussi \emph{Focalisé})}.

\subsubsection*{\nouveau{Marqueur}}

\nouveau{Les sorts de type \emph{Marqueur} ne ciblent pas des unités ou des figurines. La cible du sort est un point sur le champ de bataille, choisi par le joueur lançant le sort}.

\subsubsection*{Personnage uniquement}

Les sorts de type \emph{Personnage Uniquement} ne peuvent cibler que des figurines de \emph{Personnage} (ce qui inclut leurs montures).

\subsubsection*{Portée X{\pouce}}

Les sorts ont normalement une portée maximale indiquée par \emph{Portée X{\pouce}}. Seules les figurines ou unités situées à moins de X{\pouce} du lanceur peuvent être prises pour cible.

\subsubsection*{Projectile}

Les sorts de type \emph{Projectile} ne peuvent cibler que des unités dans le champ de vision du lanceur. Ils ne peuvent être lancés si le lanceur, ou son unité, sont engagés au corps à corps.

\subsubsection*{Unité du lanceur}

Les sorts de type \emph{Unité du Lanceur} ne ciblent que l'unité dans laquelle se trouve la figurine qui lance le sort.

\subsubsection*{Universel}

Les sorts de type \emph{Universel} peuvent cibler des unités amies ou ennemies (ou une figurine amie ou ennemie si le sort est aussi \emph{Focalisé}).

\subsubsection*{Vortex (Portée X{\pouce}, Gabarit Y{\pouce})}

Pour définir qui est touché par un sort de type \emph{Vortex}, placez un gabarit de taille Y{\pouce} en contact avec le socle du lanceur, le centre du gabarit dans l'arc frontal du lanceur, et lancez un D6.
\begin{itemize}[label={-}]
\item \emph{1 à 5}. \nouveau{Multiplier le résultat du dé par la valeur X{\pouce}}. Le résultat est la distance sur laquelle vous devez déplacer le gabarit, dans la direction de la cible, \nouveau{qui est toujours un point sur le champ de bataille, puisque tous les sorts de \emph{Vortex} sont aussi de type \emph{Marqueur}}.
\item \emph{6}. Centrez le gabarit sur le lanceur et déplacez-le de 1D6{\pouce} dans une direction aléatoire.
\end{itemize}

Toutes les figurines sur le passage du gabarit, de sa position originelle à sa position finale, sont affectées par le sort. \nouveau{Une fois déterminé les figurines touchées et les effets appliqués, retirez le gabarit du jeu, et le sort prend fin immédiatement}.

\subsection{Durée des sorts}

La durée d'un sort indique pendant combien de temps ses effets persistent. Le sort peut être de type \emph{Immédiat}, \emph{Dure un Tour}, \emph{Permanent} ou \emph{Reste en Jeu}, décrits ci-dessous.

\subsubsection*{\nouveau{Dure un tour}}

\nouveau{L'effet dure jusqu'au début de la prochaine phase de magie du lanceur. Si une unité affectée se sépare en plusieurs parties, par exemple un \emph{Personnage} quitte l'unité, chacune des parties continue d'être affectée par le sort. Les \emph{Personnages} qui rejoignent une unité déjà affectée par un sort qui \emph{Dure un Tour} ne sont pas affectés par le sort}.

\subsubsection*{Immédiat}

Le sort n'a pas de durée. Les effets sont résolus puis le sort prend immédiatement fin.

\subsubsection*{\nouveau{Permanent}}

\nouveau{Dure jusqu'à la fin de la partie ou jusqu'à ce qu'une condition décrite par le sort soit atteinte. Un sort permanent ne peut être supprimé par aucun autre moyen que cette condition. Si une unité affectée se sépare en plusieurs parties, par exemple un \emph{Personnage} quitte l'unité, chacune des parties continue d'être affectée par le sort. Les \emph{Personnages} qui rejoignent une unité déjà affectée par un sort \emph{Permanent} ne sont pas affectés par le sort}.

\subsubsection*{Reste en jeu}

Les effets du sort durent jusqu'à ce que le sort soit dissipé ou que le lanceur soit tué. Un sort qui reste en jeu peut être dissipé lors d'une phase de magie suivante (voir la partie Dissipation d'un sort restant en jeu). \nouveau{Si une unité affectée se sépare en plusieurs parties, par exemple un \emph{Personnage} quitte l'unité, chacune des parties continue d'être affectée par le sort. Dans ce cas de figure, une dissipation réussie permet de dissiper le sort sur toutes les unités qui en étaient affectées.} Les \emph{Personnages} qui rejoignent une unité déjà affectée par un sort \emph{Reste en Jeu} ne sont pas affectés par le sort. Tant qu'un sort \emph{Reste en Jeu} est actif, son lanceur ne peut le relancer. \nouveau{Si le lanceur est tué, le sort est automatiquement dissipé à la première occasion où il aurait pu être normalement dissipé.}

\section{Étapes de la phase de magie}
\label{etapes_phase_magie}

Chaque \emph{Phase de Magie} est divisée en \emph{Étapes}. Suivez l'ordre de la table \ref{table/etapes_magie}.

\begin{table}[!htbp]
\centering
\begin{tabular}{c|m{12cm}}
\textbf{1} & Début de la \emph{Phase de Magie}. Déterminez les flux magiques et faites les tentatives de canalisation. \tabularnewline
\textbf{2} & \nouveau{Le joueur réactif peut tenter de dissiper des sorts \emph{Reste en Jeu} lancés lors des phases de magie précédentes.} \tabularnewline
\textbf{3} & Le joueur actif peut tenter de dissiper des sorts \emph{Reste en Jeu} lancés lors des phases de magie précédentes. \tabularnewline
\textbf{4} & Le joueur actif peut tenter de lancer un sort (voir \ref{lancement_sort} à la page \pageref{lancement_sort}). \tabularnewline
\textbf{5} & Retournez à l'étape 2, à moins qu'aucun joueur n'ait fait d'action pendant les étapes 2 à 4. Dans ce cas, passez à l'étape 6. \tabularnewline
\textbf{6} & Fin de la \emph{Phase de Magie}. Les capacités se passant à la fin de la phase sont déclenchées. \tabularnewline
\end{tabular}
\caption{\label{table/etapes_magie}Étapes de la phase de magie.}
\end{table}

\subsection{Dés de magie : flux magiques et canalisation}

Pendant la phase de magie, les sorts sont lancés et dissipés avec des dés de magie, appelés différemment selon le joueur. Le joueur actif a des dés de pouvoir, le joueur réactif a des dés de dissipation. Chaque joueur garde ses dés dans une réserve et peut piocher dedans pour lancer ou dissiper des sorts. \nouveau{Aucun des joueurs ne peut posséder ou utiliser plus de 12 dés de magie par phase. De plus, ils ne peuvent ajouter que 2 dés de magie maximum aux dés générés par les \emph{Vents de Magie} lors d'une \emph{Phase de Magie}}.

Au début de la phase de magie, le joueur actif lance 2D6 pour déterminer la force des \emph{Flux Magiques}. Le nombre de dés de pouvoir est égal à la somme des résultats des deux dés, tandis que le nombre de dés de dissipation est égal au résultat le plus élevé des deux dés. Le joueur actif et le joueur réactif peuvent alors faire une tentative de \emph{Canalisation}. \nouveau{Pour tenter de canaliser, lancez 1D6. Ajoutez 1 au résultat pour chaque figurine non en fuite ayant la règle spéciale \emph{Canalisation} en votre possession. Si le résultat est de 7 ou plus, vous pouvez ajouter un dé de magie à votre réserve}.

\subsection{Dissipation d'un sort restant en jeu}

\nouveau{Un joueur peut dissiper un sort \emph{Reste en Jeu} lancé lors d'une phase de magie précédente à l'étape dédiée de la \emph{Phase de Magie}, en utilisant des dés de magie, pouvoir ou de dissipation, en tant que dés de dissipation et en suivant les étapes 4 et 5 du lancement d'un sort. Le lanceur du sort peut dissiper ses propres sorts \emph{Reste en Jeu} automatiquement et sans utiliser de dés de magie, tandis que son adversaire devra faire une tentative de dissipation.} Pour dissiper un sort \emph{Reste en Jeu}, le résultat de la tentative doit être supérieur ou égal à la plus petite valeur de lancement de la version du sort qui \emph{Reste en Jeu} (ignorez les valeurs des versions améliorées).
Le joueur réactif ne sait pas si le joueur actif va dissiper son sort \emph{Reste en Jeu} ou pas. C'est donc à lui de prendre la décision en premier. Il doit choisir s'il prend le risque de dépenser des dés de dissipation (qu'il aurait pu garder si le joueur actif avait pour intention de dissiper) ou bien s'il garde ses dés de dissipation, laissant ainsi le sort en jeu, en prenant le risque que le joueur actif ne le dissipe pas de lui même.


\section{Étapes de lancement d'un sort}
\label{lancement_sort}

Chacun de vos \emph{Sorciers} non en fuite, ou chaque figurine avec un objet de sort, peut tenter de jeter une fois chacun de ses sorts par phase de magie. Si le sort est lancé avec succès, le joueur adversaire peut tenter de le dissiper. Si la dissipation échoue ou n'est pas tentée, appliquez les effets du sort, puis l'\emph{Attribut de la Discipline}.

Pour lancer un sort, suivez la procédure de la table \ref{table/etapes_lancement_sort} (page \pageref{table/etapes_lancement_sort}).

\begin{table}[!htbp]
\centering
\begin{tabular}{c|m{12cm}}
\textbf{1} & Le joueur actif indique quel \emph{Sorcier} tente de lancer quel sort, et le nombre de dés de pouvoir utilisés. Il doit préciser s'il opte pour une version améliorée du sort, \nouveau{ainsi que la cible du sort et de celle de l'attribut si nécessaire}. \nouveau{Le joueur peut lancer entre 1 et 5 dés de pouvoir, dans la limite de sa réserve}. \tabularnewline
\textbf{2} & Le joueur actif lance le nombre de dés de pouvoir annoncé, en les retirant de sa réserve. Additionnez les résultats des dés avec tous les modificateurs de lancer (voir la partie Modificateur magiques). \tabularnewline
\textbf{3} & Le sort est lancé avec succès si le total de lancer est supérieur ou égal à la valeur de lancement. Sinon, le lancement de sort échoue, le lanceur subit l'effet \emph{Perte de Concentration}. Passez au point 8. \tabularnewline
\textbf{4} & Le joueur réactif peut tenter de dissiper le sort. Dans ce cas, il doit indiquer, s'il en a, lequel de ses \emph{Sorciers} n'étant pas en fuite va tenter la dissipation. Il doit ensuite annoncer combien de dés de dissipation il va utiliser, avec un minimum de un et dans la limite de sa réserve. Il est possible de tenter une dissipation même sans avoir de \emph{Sorcier}. Si aucune tentative n'est faite, passez au point 6. \tabularnewline
\textbf{5} & Le joueur réactif lance le nombre de dés de dissipation annoncé, en les retirant de sa réserve. Additionnez les résultats des dés avec les modificateurs de dissipation (pour un \emph{Pouvoir Irrésistible}, par exemple). Si le résultat est supérieur ou égal au total de lancement, le sort est dissipé, et passez au point 8. Sinon, le \emph{Sorcier} ayant tenté la dissipation subit l'effet \emph{Perte de Concentration}, et passez au point 6. \tabularnewline
\textbf{6} & Appliquez les effets du sort, puis ceux de l'\emph{Attribut de la Discipline}. \tabularnewline
\textbf{7} & Si le sort a été lancé avec un \emph{Pouvoir Irrésistible}, appliquez les effets du \emph{Fiasco}. \tabularnewline
\textbf{8} & La tentative de lancer est terminée, retournez à l'étape 4 de la \emph{Phase de Magie} (voir \ref{etapes_phase_magie} à la page \pageref{etapes_phase_magie}). \tabularnewline
\end{tabular}
\caption{\label{table/etapes_lancement_sort}Étapes de lancement d'un sort.}
\end{table}

\subsection{Modificateurs magiques}
Lors d'une tentative de lancement ou de dissipation d'un sort, ajouter le bonus donné par le niveau de magie du sorcier et tout autre modificateur pour obtenir le total de jet de lancement ou de dissipation. \nouveau{Les bonus obtenus de cette manière ne peuvent \newrule{ni dépasser un total de +3 ni descendre en dessous de -3}. Il y a quelques exceptions à cette règle, notamment lors d'un pouvoir irrésistible. Dans ce cas, ajouter d'abord les modificateurs normaux jusqu'à un maximum de +3, puis ajouter les modificateurs exceptionnels (qui, eux, peuvent dépasser +3).}

\subsection{Perte de concentration (voir la table \ref{table/etapes_lancement_sort}, page \pageref{table/etapes_lancement_sort})}

\nouveau{Un \emph{Sorcier} qui subit l'effet \emph{Perte de Concentration} ne peut ajouter aucun bonus (que ce soit son niveau de magie ou un \emph{Pouvoir Irrésistible}) à des jets de lancer ou de dissipation de sort pour le reste de la phase. Un \emph{Sorcier} qui subit l'effet \emph{Perte de Concentration} ne peut plus bénéficier de ses bonus s'il fait un \emph{Pouvoir Irrésistible} et doit en subir les effets négatifs (\emph{Fiasco} et retrait de dés de Pouvoir).}

\subsection{Pas assez de puissance !}

Quand vous tentez de lancer ou de dissiper un sort \nouveau{avec un seul dé de magie}, un résultat sans modificateur de \result{1} ou \result{2} est toujours un échec.


\subsection{Pouvoir irrésistible}

\nouveau{Le nombre de dés de pouvoir utilisés pour lancer le sort est noté NDP. Quand vous lancez ou dissipez un sort, et qu'au moins deux dés sont des \result{6} sans modificateurs, la tentative bénéficie d'un \emph{Pouvoir Irrésistible}. Dans ce cas, ajoutez immédiatement \result{1D3 + NDP} à votre total de lancer ou de dissipation. C'est une exception à la règle des Modificateurs magiques (ce modificateur peut dépasser +3). Si un sort est lancé avec un \emph{Pouvoir Irrésistible} et qu'il n'est pas dissipé, le lanceur du sort subit l'effet d'un \emph{Fiasco}}.

\subsection{Fiasco}

\nouveau{Lancez 2D6 et appliquez les effets décrits dans la ligne correspondant au résultat du tableau \ref{table/fiasco}. Le nombre de dés de pouvoir utilisés pour lancer le sort est noté NDP. La Force des touches d'un \emph{Fiasco} est égale à \result{NDP + 2}. Toutes ces touches suivent les règles spéciales \emph{Attaques magiques} et \emph{Perforant (1)}. La figurine du \emph{Sorcier} à l'origine du \emph{Fiasco} ne peut bénéficier d'aucune sauvegarde contre les effets du Fiasco. De plus, finissez par retirer de la réserve NDP dés de pouvoir.}

\begin{table}[!htbp]
\centering
\begin{tabular}{cm{12cm}}
\hline
\textbf{2 à 4} & Centrez le gabarit de 5{\pouce} sur le lanceur. Toute figurine recouverte par le gabarit, même partiellement, subit une touche. De plus, \nouveau{si NDP = \textbf{5}, retirez le lanceur de la partie. Si NDP = 4, lancez un D6. Sur un résultat de 1 à 3, le lanceur est retiré de la partie.} \tabularnewline
\textbf{5 à 6} & Centrez le gabarit de 3{\pouce} sur le lanceur. Toute figurine recouverte par le gabarit subit une touche. Le lanceur doit subir une touche. \tabularnewline
\textbf{7} & \nouveau{L'unité du lanceur subit NDP touches, réparties comme des tirs. Le lanceur ne peut cependant subir qu'une seule touche au maximum.} \tabularnewline
\textbf{8 à 9} & Le lanceur et tout \emph{Sorcier} ami subissent une touche. \tabularnewline
\textbf{10 à 12} & \nouveau{Le niveau de magie du lanceur est diminué de NDP - 2.} Il perd un sort pour chaque niveau de magie perdu, en commençant par le sort ayant causé le \emph{Fiasco} et en tirant les autres au hasard. \tabularnewline
\hline
\end{tabular}
\caption{\label{table/fiasco}Effets d'un \emph{Fiasco}.}
\end{table}
 
\section{Attributs de discipline}

Les \emph{Attributs de Discipline} sont des sorts qui ne peuvent pas être lancés indépendamment. Ils sont lancés automatiquement, du moment qu'il y a une cible possible, à chaque fois qu'un sort de la \emph{Discipline} est lancé avec succès et n'est pas dissipé. Les \emph{Attributs} ne peuvent pas être dissipés et n'ont pas de valeur de lancement.

\section{Objets de sort}

Certains sorts sont liés à des \emph{Objets de Sort}. Ce sont des sorts qui sont généralement contenus dans des objets magiques. Ces sorts peuvent être lancés par des figurines ou unités n'étant pas des \emph{Sorciers}. Posséder un \emph{Objet de Sort} ne fait pas de la figurine ou de l'unité un \emph{Sorcier}. Les sorts des \emph{Objets de Sort} ne peuvent jamais être lancés en version améliorée. 

\subsubsection*{Lancer le sort d'un objet de sort}
Lancer un sort lié à un \emph{Objet de Sort} se fait de la même manière qu'un sort normal, sauf qu'il est impossible d'ajouter des modificateurs positifs. La figurine lançant un sort depuis un \emph{Objet de Sort} ne souffre jamais de \emph{Perte de Concentration}. Pour lancer un sort lié, vous devez obtenir un total de lancer supérieur ou égal au niveau de puissance de l'objet. Le niveau de puissance correspond à la valeur de lancement. Si le sort lié a aussi une valeur de lancement normale, le niveau de puissance la remplace. Si le sort est lancé avec succès, appliquez aussi les effets de l'\emph{Attribut de la Discipline}.

\subsubsection*{Pouvoir irrésistible et Fiasco}
\nouveau{Si vous lancez un sort lié avec un \emph{Pouvoir Irrésistible}, n'ajoutez aucun bonus au lancement. En outre, ne faites pas de jet dans la table des \emph{Fiascos}. Retirez NDP dés de pouvoir de la réserve. De plus, si le sort a été lancé avec 3 dés ou moins, rien d'autre ne se passe. Si le sort a été lancé avec 4 dés ou plus, l'objet est perdu et le sort ne peut plus être lancé de la partie}.

\subsubsection*{Dissipation}
\nouveau{Quand un joueur dissipe un sort lié, il ajoute un bonus de +1 à son jet de dissipation.} C'est une exception à la règle des Modificateurs magiques.

\section{Effets magiques particuliers}

\subsection{Mouvement magique}

Tout mouvement réalisé pendant la phase de magie est un \emph{Mouvement Magique}. Ce mouvement est effectué comme lors de l'étape des \emph{Autres Mouvements} et suit les même restrictions. Les unités en fuite ou au corps à corps ne peuvent donc pas se déplacer. Toute action autorisée pendant cette étape peut être faite lors d'un mouvement magique, comme une \emph{Roue}, une \emph{Reformation}, quitter ou rejoindre une unité pour un \emph{Personnage}, à l'exception d'une marche forcée. Un \emph{Mouvement Magique} a toujours une distance limite précisée (par exemple, la cible peut faire un \emph{Mouvement Magique} de 12{\pouce}). Cette valeur est utilisée à la place de la valeur de Mouvement de la cible. Une unité ne peut faire qu'un seul \emph{Mouvement Magique} par phase de magie.

\subsection{Récupérer des PV et ressusciter des figurines}

Certains sorts ou capacités permettent de récupérer des points de vie perdus pendant la bataille. La quantité de points de vie récupérés est notée dans la capacité (\emph{Récupérer [X] PVs}). Si une unité contient plusieurs figurines, chaque figurine doit récupérer tous ses PVs perdus avant qu'une autre figurine puisse en récupérer à son tour. Les \emph{Personnages} à l'intérieur des unités ne peuvent jamais récupérer de PVs grâce à une capacité qui fait récupérer des PVs à leur unité (les \emph{Personnages} ne récupèrent des PVs que lorsqu'ils sont la seule cible du sort ou de la capacité). \emph{Récupérer des PVs} ne peut jamais ramener à la vie des figurines retirées comme perte, et ne peut pas, à moins que le contraire ne soit précisé, permettre à une figurine de dépasser ses PVs initiaux. Tout PV soigné au-delà est perdu.

\emph{Ressusciter des Figurines} utilise les mêmes règles que \emph{Récupérer des PVs}, sauf que \emph{Ressusciter des Figurines} peut ramener à la vie des figurines retirées comme perte. D'abord, les figurines de l'unité (à l'exception des \emph{Personnages}) récupèrent tous leurs PVs perdus, puis les figurines sont ramenées à la vie dans l'ordre suivant : \emph{Champion}, \emph{Porte-étendard}, \emph{Musicien} et enfin figurines ordinaires. \nouveau{Un \emph{Porte-étendard} capturé ne peut pas être ramené à la vie}. Chaque figurine relevée doit récupérer la totalité de ses PVs initiaux avant qu'une autre figurine ne soit relevée. Cela ne peut pas, à moins que le contraire ne soit précisé, permettre à une unité de dépasser ses effectifs initiaux. Tout PV ressuscité au-delà est perdu.

\subsection{Unités invoquées}

Les unités invoquées sont des unités créées pendant la partie. Toutes les figurines d'une unité nouvellement invoquée doivent être déployées à portée du sort ou de la capacité. Si l'unité est invoquée grâce à un sort de type \emph{Marqueur}, au moins une des figurines doit être déployée sur le point ciblé et toutes les autres figurines doivent être déployées à portée du sort. De plus les figurines invoquées doivent être placées au moins à 1{\pouce} des autres unités et des \emph{Terrains Infranchissables}. 
Si l'unité ne peut pas être déployée en entier, alors aucune figurine ne doit être déployée. Une fois invoquées, ces unités comptent comme des unités ordinaires, dans le camp du lanceur du sort ou de la capacité. Cependant, les unités invoquées ne peuvent donner aucun point de victoire à l'adversaire.
