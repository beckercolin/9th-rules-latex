% Base sur la VO 0.11.9
% Relecture technique: 
% Relecture syntaxique: 

\part{Équipement standard}

\section{Armes de corps à corps}
\label{equipement/cac}

Les armes listées dans cette section sont utilisées au corps à corps et chacune apporte avantages et inconvénients. Les règles de ces armes s'appliquent uniquement aux attaques effectuées avec ces armes. Elles ne s'appliquent jamais aux attaques spéciales telles que le \emph{Piétinement} ou quand la figurine se bat avec une arme différente. Quand une figurine possède plusieurs Armes de corps à corps, elle doit choisir laquelle utiliser au début de chaque combat et doit continuer à utiliser la même arme pour toute la durée du combat. Toutes les figurines ordinaires d'une même unité doivent choisir la même arme. À moins que le contraire ne soit précisé, les montures ne bénéficient jamais des effets des armes.

\begin{table}[H]
\begin{tabular}{c m{12.5cm}}
\hline
\textbf{Arme de base} & Toutes les figurines sont équipées d'une Arme de base. Les Armes de base ne peuvent jamais être perdues ou détruites. Si une figurine possède n'importe quelle Arme de corps à corps autre qu'une Arme de base, elle ne peut pas choisir d'utiliser son Arme de base, à moins que le contraire ne soit précisé. Les Armes de base maniées par des figurines à pied peuvent être utilisées avec un Bouclier pour obtenir la règle \emph{Parade}.

\nouveau{\emph{Parade}: les figurines attaquant le front d'une unité qui utilise la \emph{Parade} ne peuvent jamais toucher leur cible sur un jet plus facile que 4+. Appliquez cette règle avant tout modificateur pour toucher.} Ceci n'est pas considéré comme un modificateur pour toucher. \tabularnewline
\textbf{Fléau} & \emph{Arme à Deux Mains}. \nouveau{Les attaques portées avec un fléau ont +2 en Force. Les attaques de corps à corps allouées contre une figurine portant un Fléau ont un bonus de +1 pour toucher}. \tabularnewline
\textbf{Arme lourde} & \emph{Arme à Deux Mains}. Les attaques portées avec une arme lourde ont +2 en Force, mais \nouveau{sont faites à Initiative 0} (sans changer l'Initiative du porteur). \tabularnewline
\textbf{Hallebarde} & \emph{Arme à Deux Mains}. Les attaques portées avec une hallebarde ont +1 en Force. \tabularnewline
\textbf{Paire d'armes} & \emph{Arme à Deux Mains}. +1 Attaque \nouveau{et +1 en Initiative}. \tabularnewline
\textbf{Lance de cavalerie} & Les attaques portées avec une lance de cavalerie ont +2 en Force lors du round de combat suivant une charge. Ce bonus ne peut être utilisé que pour les attaques visant l'unité chargée. Seules une figurine montée, \nouveau{une \emph{Bête de Guerre} ou une \emph{Bête Monstrueuse}} peuvent utiliser une lance de cavalerie. \tabularnewline
\textbf{Lance légère} & \nouveau{Suit les mêmes règles qu'une lance de cavalerie, mais ne donne qu'un bonus de +1 en Force.} Seules une figurine montée, \nouveau{une \emph{Bête de Guerre} ou une \emph{Bête Monstrueuse}} peuvent utiliser une lance légère. \tabularnewline
\textbf{Lance} & \emph{Combat avec un Rang Supplémentaire}, \emph{Perforant (1)}. \nouveau{\emph{Coup Fatal} contre la \emph{Cavalerie}, la \emph{Cavalerie Monstrueuse} et les \emph{Chars} ennemis engagés avec le front de l'unité qui utilise les lances. Les figurines montées ne peuvent pas utiliser de lances.} \tabularnewline
\hline
\end{tabular}
\end{table}

\section{Armes de tir}

Les armes listées dans cette section sont utilisées pour faire des attaques à distance. Chaque figurine ne peut normalement utiliser qu'une seule Arme de tir par phase, même si elle en possède plusieurs, et toutes les figurines ordinaires d'une même unité doivent choisir la même Arme de tir. Chaque Arme de tir a une portée maximale, une valeur de Force et peut avoir des règles spéciales. Les règles spéciales listées pour une Arme de tir ne s'appliquent qu'aux attaques de tir effectuées avec cette arme.

\begin{table}[H]
\centering
\begin{tabular}{r c c m{8cm}}
\hline
Arme & Portée & Force & Règles spéciales \tabularnewline
\textbf{Arc court} & 18{\pouce} & 3 & \nouveau{\emph{Tir de Volée}}. \tabularnewline
\textbf{Arc} & 24{\pouce} & 3 & \nouveau{\emph{Tir de Volée}}. \tabularnewline
\textbf{Arc long} & 30{\pouce} & 3 & \nouveau{\emph{Tir de Volée}}. \tabularnewline
\textbf{Arbalète} & 30{\pouce} & 4 & \nouveau{\emph{Encombrant}}. \tabularnewline
\textbf{Arquebuse} & 24{\pouce} & 4 & \nouveau{\emph{Encombrant}}, \emph{Perforant (1)}. \tabularnewline
\textbf{Pistolet} & 12{\pouce} & 4 & \emph{Perforant (1)}, \emph{Tir Rapide}, compte comme une paire d'armes au corps à corps. \tabularnewline
\textbf{Paire de pistolets} & 12{\pouce} & 4 & \emph{Perforant (1)}, \emph{Tirs Multiples (2)}, \emph{Tir Rapide}, compte comme une paire d'armes au corps à corps. \tabularnewline
\textbf{Armes de jet} & \nouveau{12{\pouce}} & \nouveau{Force du tireur} & \nouveau{\emph{Tirs Multiples (2)}, \emph{Tir Rapide}}. \tabularnewline
\hline
\end{tabular}
\end{table}

\section{Armes d'artillerie}

Les armes listées dans cette section sont des armes de tir particulières. Ces armes sont quelquefois montées sur des figurines de type \emph{Machines de Guerre}, mais elles peuvent aussi être montées sur des \emph{Chars}, portées par des \emph{Monstres} ou contenues dans des \emph{Objets Magiques}. Les \emph{Armes d'Artillerie} sont des armes de tir et ont toujours la règle spéciale \emph{Rechargez!}. Chaque \emph{Armes d'Artillerie} son propre profil avec sa portée, une Force pour ses touches et ses règles spéciales, que vous trouverez avec sa description. Une attaque de tir d'une \emph{Arme d'Artillerie} n'est pas résolue comme une attaque de tir classique: suivez les règles ci-dessous pour déterminer les dommages qu'elle cause.

\subsubsection*{Baliste}

Les \emph{Balistes} suivent les règles normales de tir, mais elles également peuvent pénétrer les rangs ou les colonnes, provoquant ainsi des dommages supplémentaires. \nouveau{Déterminez le nombre maximal de touches possibles en regardant dans quel arc de la cible est située la \emph{Baliste}. S'il s'agit du front ou de l'arrière, le nombre de touches maximal est égal au nombre de rangs de l'unité ciblée. S'il s'agit d'un flanc, le nombre de touches maximal est égal au nombre de colonnes de l'unité ciblée. Si la \emph{Baliste} touche sa cible, faites un jet pour blesser et tentez les sauvegardes comme d'habitude. Si une figurine est retirée comme perte, alors le carreau pénètre dans l'unité et provoque une nouvelle touche avec un malus de -1 à sa Force. Continuez d'ajouter des touches aussi longtemps qu'une figurine est retirée comme perte dans la limite du nombre maximal de touches déterminé plus tôt, en ajoutant un malus supplémentaire de -1 à la Force à chaque fois, jusqu'à un minimum de 1}.
\begin{table}[H]
\centering
\begin{tabular}{m{2.5cm}m{2.5cm}m{2.5cm}m{1cm}m{3cm}}
\hline
\centering Force de la 1\iere touche & \centering Force de la \newline 2\ieme touche & \centering Force de la 3\ieme touche & \centering ... & Nombre maximal de touches\tabularnewline
\centering F & \centering F-1 & \centering F-2 & \centering etc. & \raggedright Nombre initial de rangs ou de colonnes \tabularnewline
\hline
\end{tabular}
\caption{\label{table/tir_baliste}Force d'un tir de \emph{Baliste}.}
\end{table}

\subsubsection*{Batterie de tir}

\nouveau{Une \emph{Batterie de Tir} suit les règles normales de tir avec les exceptions suivantes : toutes les \emph{Batteries de Tir} ont la règle spéciale \emph{Tirs Multiples}. Toutefois, cette règle ne leur inflige pas la pénalité pour toucher usuelle. Si le nombre de tirs est un nombre prédéterminé, comme \emph{Tirs Multiples (6)}, la \emph{Batterie de Tir} ne peut pas subir d'\emph{Incident de Tir}. Cependant, si le nombre de tirs est déterminé aléatoirement (par exemple \emph{Tirs Multiples (2D6)}), la \emph{Batterie de Tir} peut subir un \emph{Incident de Tir}. Si un seul \result{6} naturel, après relances, est obtenu dans le résultat du nombre de tirs, la \emph{Batterie de Tir} a un malus de -1 pour toucher pour ces tirs. Si au moins deux \result{6} naturels, après relances, sont obtenus dans le résultat du nombre de tirs, la \emph{Batterie de Tir} subit un \emph{Incident de Tir}. Tous les tirs sont annulés, et la \emph{Batterie de Tir} doit effectuer un jet sur le tableau des \emph{Incidents de Tir}}.

\subsubsection*{Canon (X)}

\nouveau{Choisissez une cible normalement, puis désignez un point du socle de cette cible en ligne de vue du canon. Lancez les dés pour toucher normalement, en ignorant les malus dûs aux couverts (lourds ou légers) et en ajoutant un bonus de +1 pour toucher si la figurine sous le point choisi possède la règle spéciale \emph{Grande Cible}. Si le jet pour toucher est raté, il ne peut jamais être relancé. Si un '1' naturel est obtenu, le canon subit un \emph{Incident de Tir} : le tir est annulé, et le \emph{Canon} doit effectuer un jet sur le tableau des \emph{Incidents de Tir}}.

\nouveau{Si le canon réussit son jet pour toucher, tracez une ligne droite de X{\pouce} (X étant la valeur entre parenthèses) depuis le point d'impact du boulet de \emph{Canon}, en s'éloignant du centre de la figurine portant le \emph{Canon}. Cette ligne est immédiatement stoppée si elle rencontre un mur ou un élément de terrain infranchissable.}

\nouveau{La figurine sous le point d'impact initial du \emph{Canon} subit une touche avec la Force et les règles spéciales données dans le profil du \emph{Canon}. Les autres figurines sous la ligne peuvent subir une touche avec les mêmes règles spéciales, mais avec une Force divisée par 2}. En commençant par la figurine qui a été touchée la première, la plus proche du point d'impact du boulet de \emph{Canon}, lancez le jet pour blesser et les sauvegardes. Si la figurine est retirée comme perte, faites le jet pour blesser puis les sauvegardes pour la figurine suivante sur la ligne, et ainsi de suite. \nouveau{Si une figurine survit, le boulet de canon s'arrête et les figurines suivantes ne sont pas touchées}.
\begin{table}[H]
\centering
\begin{tabular}{m{2.5cm}m{2.5cm}m{2.5cm}m{1cm}m{3cm}}
\hline
\centering Force de la 1\iere touche & \centering Force de la \newline 2\ieme touche & \centering Force de la 3\ieme touche & \centering ... & Conditions pour toucher \tabularnewline
\centering F & \centering F/2 & \centering F/2 & \centering etc. & \raggedright Gabarit de Ligne et \nouveau{en Ligne de Vue}\tabularnewline
\hline
\end{tabular}
\caption{\label{table/tir_canon}Force d'un tir de \emph{Canon}.}
\end{table}

\subsubsection*{Canon à flammes}

\nouveau{Placez le gabarit de 3{\pouce} avec son centre dans la ligne de vue et à portée. Puis éloignez le gabarit de 1D6{\pouce} du \emph{Canon à Flammes}, tout droit, selon l'axe passant par le centre de la figurine et le centre du gabarit. Si un \result{6} naturel est obtenu pour déterminer cette distance, un \emph{Incident de Tir} est survenu. Le tir est annulé et le \emph{Canon à Flammes} doit effectuer un jet sur le tableau des \emph{Incidents de Tir} avec un modificateur de -1. Autrement, toutes les figurines touchées par le gabarit au cours de son déplacement de 1D6{\pouce} subissent une touche, en utilisant la Force et les règles spéciales données dans le profil de l'\emph{Arme d'Artillerie}. Toute unité qui peut être potentiellement touchée par le gabarit, entre sa position initiale et 5{\pouce} devant est considérée comme cible potentielle de l'attaque. Cela ne peut pas être une unité amie ou une unité engagée au corps à corps}.

\subsubsection*{Catapulte (X)}

Placez un gabarit de la taille indiquée dans le profil de l'arme, son centre étant au-dessus d'une figurine ennemie, dans la ligne de vue et à portée. Aucune partie du gabarit ne peut être placée sur une figurine amie ou sur des unités engagées au corps à corps. \nouveau{Faites ensuite dévier le gabarit de 1D6x2{\pouce}. Si un \result{6} naturel est obtenu pour la distance de déviation, avant la multiplication par 2, un \emph{Incident de Tir} est survenu. Le tir est alors annulé et la \emph{Catapulte} doit effectuer un jet sur le tableau des \emph{Incidents de Tir}}.

Le centre du gabarit peut également être placé en dehors de la ligne de vue de la \emph{Catapulte} (mais toujours au-dessus d'une figurine ennemie, à portée et dont aucune partie du gabarit n'est placée sur une figurine amie ou sur des unités engagées au corps à corps). Si c'est le cas, le gabarit ne reste pas en place si un \og Touché ! \fg{} est obtenu. Au lieu de cela, déplacez le gabarit dans une direction aléatoire, mais retranchez la Capacité de Tir à la distance sur laquelle le gabarit bouge, pour obtenir 1D6x2 - CT{\pouce}. Si la distance est de 0 ou moins, l'unité est touchée et le gabarit n'est pas dévié.
\begin{table}[H]
\centering
\begin{tabular}{c c c}
\hline
 & \textbf{En ligne de vue} & \textbf{Hors de la ligne de vue} \tabularnewline
\textbf{Flèche de déviation} & 1D6x2{\pouce} & 1D6x2{\pouce} \tabularnewline
\textbf{Touché !} & Touche directe & 1D6x2 - CT{\pouce} \tabularnewline
\hline
\end{tabular}
\caption{\label{table/tir_catapulte}Distance de déviation d'un tir de \emph{Catapulte}.}
\end{table}
Une fois que la position finale du gabarit est déterminée, toutes les figurines sous le gabarit sont touchées, en utilisant la Force et les règles spéciales données dans le profil de l'\emph{Arme d'Artillerie}. Quelques \emph{Catapultes} ont une Force plus grande ou des règles spéciales indiquées entre crochets (comme Force 3 [9]). \nouveau{L'effet entre crochets affecte seulement la figurine située au centre exact du gabarit. S'il y a un doute entre plusieurs figurines, tirez au sort}.

\subsection{Tableau des Incidents}

Quand une \emph{Arme d'Artillerie} subit un \emph{Incident de Tir}, lancez 1D6 et consultez l'effet correspondant au résultat dans le tableau \ref{table/incident_tir}.

\begin{table}[H]
\centering
\begin{tabular}{c m{12.5cm}}
\hline
Résultat & Effet \tabularnewline
\result{0} ou moins & \textbf{Explosion!} \newline Toutes les figurines à moins de 1D6{\pouce} de la figurine qui subit un \emph{Incident de Tir} reçoivent une touche de Force 5. La figurine qui tire est détruite, retirez-la comme perte. \tabularnewline
1 à 2 & \textbf{Défaillance critique} \newline Le mécanisme de tir est endommagé. La figurine ne peut plus tirer avec cette arme pour le restant de la partie. \tabularnewline
3 à 4 & \textbf{Enrayé} \newline Cette \emph{Arme d'Artillerie} ne peut pas être utilisée au prochain tour de son propriétaire. \tabularnewline
5+ & \textbf{Dysfonctionnement} \newline La figurine subit une blessure sans sauvegarde d'aucune sorte autorisée. \tabularnewline
\hline
\end{tabular}
\caption{\label{table/incident_tir}Effets d'un \emph{Incident de Tir}.}
\end{table}

\section{Armures}
\label{equipement_armure}

La Sauvegarde d'Armure d'une figurine ou élément de figurine est déterminée par son Armure, parfois modifiée par des règles spéciales et des sorts. La Sauvegarde d'Armure d'une figurine est calculée en combinant toutes ses pièces d'Armure. Chaque pièce d'Armure ajoute un bonus au jet de sauvegarde, pour atteindre un maximum de +6. Si le jet de sauvegarde, en incluant les modificateurs, est égal à 7 ou plus, la Sauvegarde d'Armure est réussie. Un jet naturel de \result{1} est toujours un échec.

Il existe 5 différents types d'armure.
\begin{table}[H]
\centering
\begin{tabular}{m{7.5cm} m{7.5cm}}
\hline
\textbf{Armure complète}
\newline Une figurine ne peut porter qu'une seule \emph{Armure Complète}.
\newline \emph{Armure légère}: +1
\newline \emph{Armure lourde}: +2
\newline \emph{Armure de plates}: +3
&
\textbf{Bouclier}
\newline Une figurine ne peut porter qu'un Bouclier. Au corps à corps, il est impossible d'utiliser un Bouclier en même temps qu'une arme ayant la règle \emph{Arme à Deux Mains}.
\newline \emph{Bouclier}: +1
\tabularnewline
\textbf{Montures}
\newline \nouveau{\emph{Protection de monture (6+)}: +1}
\newline \nouveau{\emph{Protection de monture (5+)}: +2}
\newline Peu importe le nombre de montures qu'a la figurine, elle ne gagne ce bonus qu'une fois.
\newline \emph{Caparaçon}: +1
\newline Le caparaçon donne également -1 en Mouvement à la monture.
&
\textbf{\nouveau{Protection innée}}
Une figurine ne peut profiter que d'une seule instance de \emph{Protection innée}. Utilisez la meilleure disponible.
\newline \emph{Protection innée (6+)}: +1
\newline \emph{Protection innée (5+)}: +2
\newline \emph{Protection innée (4+)}: +3
\tabularnewline
\multicolumn{2}{m{15,5cm}}{
\textbf{Autres}
\newline Il existe d'autres moyens d'augmenter la Sauvegarde d'Armure : des équipements spéciaux, des Objets magiques, comme des heaumes, des règles spéciales, certains sorts, et d'autres règles, mais seulement jusqu'à un maximum de +6.

Par exemple, si une figurine s'équipe d'une Armure légère (+1), d'un Bouclier (+1), d'un Heaume (+1) et qu'elle monte un cheval avec \emph{Protection de monture (6+)} (+1) caparaçonné (+1), elle totalise un bonus de +5 à ses jets de Sauvegarde d'Armure. Cela veut dire qu'un jet de 2+ donnera un résultat de \result{7} ou plus, et sera une Sauvegarde d'Armure réussie. Normalement, une telle Sauvegarde d'Armure sera évoquée comme une Sauvegarde d'Armure de 2+. Remarquez qu'avec un total de +6 à la Sauvegarde d'Armure, aucune figurine ne peut avoir de meilleure Sauvegarde d'Armure que 1+ (et même avec ça, un jet de \result{1} naturel est toujours un échec).
}
\tabularnewline
\hline
\end{tabular}
\caption{\label{table/armures}Types d'Armures}
\end{table}
