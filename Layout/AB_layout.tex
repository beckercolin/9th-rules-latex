
\documentclass[a4paper,8pt]{extarticle} % extarticle allows to use font size of 8pt.

\usepackage[a4paper, top=1.6cm, bottom=2cm, left=1.6cm, right=1.6cm]{geometry} % Marge reduction.

%% Language specific package
\usepackage[french]{babel}
\frenchbsetup{StandardLists=true} % Necessary to use enumitem with babel/french.

%% Font and typing packages
\usepackage{fontspec}
\setmainfont[
	Ligatures=TeX,
	ItalicFont={Dancing Script},
	BoldItalicFont={Dancing Script}
	]{PT Serif} % default is Latin Modern
\newfontfamily\antiquefont[Ligatures=TeX]{Caslon Antique} % fancy font
\usepackage{microtype}			% Greatly improves general appearance of the text.
\usepackage{SIunits}			% Unit appearance.
\usepackage{xspace}				% Define commands that appear not to eat spaces.
\usepackage{ulem}				% To cross words out. Use \sout{}.

%% Array utilities
\usepackage{array}				% Additionnal options for arrays.
\usepackage{colortbl}			% Additionnal options for coloring arrays.
\usepackage[table]{xcolor}		% Auto alternate grey-white rows.

%% List utilities
\usepackage[inline]{enumitem}   % Display inline lists.
\usepackage{etoolbox}           % General utility. Good for lists for instance.
\usepackage{xparse}             % List utilities.
\usepackage{datatool}	% Handling alphabetical order.

%% Frames
\usepackage{framed}				% Boxes.
\usepackage[framemethod=TikZ]{mdframed}% For fancy frames.
\usepackage{tikz}				% For fancy frames.
\usepackage{wrapfig}			% Fancy insertion of pics in text.

%% Page utilities
\usepackage{multicol}			% Allows to divide a part of the page in multiple columns.
	
%% Others
\usepackage{keyval}             % Used to create maps of commands/labels/objects.
	\makeatletter                  % Mandatory for the usage of keyval.
\usepackage{xstring}            % String parsing, cutting, etc.
\usepackage{hyperref} % Links in PDF.


%%% Update of the dotfill command to always get dots

\newcommand{\predotfill}{\penalty0\hbox{}\nobreak}%


%%% Command to avoid typing \xspace when creating a new name macro

\newcommand{\newnamemacro}[2]{\newcommand{#1}{#2}} % \xspace removed for compatibility with alphabetical ordering

%%% Language specific stuff


\newcommand{\translationteam}{\item \og AEnoriel \fg \item \og Anglachel \fg \item \og Astadriel \fg \item \og Batcat \fg \item \og Bigfish \fg \item \og Eru \fg  \item \og Gandarin \fg \item \og Groumbahk \fg \item \og Iluvatar \fg \item \og Mammstein \fg \item \og Shlagrabak \fg \item et beaucoup d'autres...}

\hypersetup{
	pdfauthor={Équipe de traduction française de T9A},
	pdfsubject={Règles pour le jeu Batailles Fantastiques : Le 9\ieme{} Âge},
}

%%% Commands %%%

\ifdef{\isitanAB}{

\newcommand{\addtosortedlist}[1]{%
	\protected@edef\textarg{#1}%
	\protected@edef\textwithoutspaces{\expandafter\removespaces\expandafter{\textarg}}%
	\substitute\textwithoutspaces{É}{E}% Most used special characters of the language, and equivalent for alphabetical ordering
	\substitute\textwithoutspaces{È}{E}%
	\substitute\textwithoutspaces{Ê}{E}%
	\substitute\textwithoutspaces{é}{e}%
	\substitute\textwithoutspaces{è}{e}%
	\substitute\textwithoutspaces{ê}{e}%
	\substitute\textwithoutspaces{À}{A}%
	\substitute\textwithoutspaces{à}{a}%
	\substitute\textwithoutspaces{ù}{u}%
	\expandafter\sortitem\expandafter[\textwithoutspaces]{#1}%
}%

\newcommand{\pts}[1]{% First step is to remove spaces if there are some
	\def\numberwithoutspaces{\expandafter\removespaces\expandafter{#1}}%
	% Next step is getting rid of formatting if there are any (bold, color, ...)
	\pdfstringdef\cleannumber{\numberwithoutspaces}%
	% Now we can try if it is 1 or not
	\expandafter\ifstrequal\expandafter{\cleannumber}{1}{#1~\labels@point}{%
	\expandafter\ifstrequal\expandafter{\cleannumber}{0.5}{0,5~\labels@point}{%
	\expandafter\ifstrequal\expandafter{\cleannumber}{1.5}{1,5~\labels@point}{%
	#1~\labels@points}}}%
}

}{}

% Dark gods
\newcommand{\dchange}{Changement}
\newcommand{\dlust}{Luxure}
\newcommand{\pestilence}{Pestilence}
\newcommand{\wrath}{Courroux}
\newcommand{\truechaos}{Chaos Primordial}


% Nothing to edit here

\ifdef{\isitanAB}{

\newcommand{\alliancepts}[1]{
\ifsubstring{#1}{\free}{%
		\free{}%
	}{%
	\ifsubstring{#1}{\permodel}{%
		\splitatinf{#1}\myoption\myvalue%
		\pts{\myvalue}\permodel{}%
	}{%
	\pts{#1}
	}}
}

% You might wanna change the order of the gods - advanced user
\newcommand{\allianceoptions}[1]{%
	\defallianceoptions{#1}%
	\unitentryformat{\labels@allianceoptions\spacebeforecolon{}:}\newline
	\expandafter\ifblank\expandafter{\allianceoptions@introsentence}{}{\noindent\allianceoptions@introsentence{}\spacebeforecolon{}:}
	
	\expandafter\ifblank\expandafter{\allianceoptions@wrath}{
		\setlength{\columnseprule}{0.5pt}
		\renewcommand{\columnseprulecolor}{\color{black!30}}
		\vspace*{-0.2cm}\begin{multicols}{3}\raggedcolumns
		
			\begin{center}
			\noindent\dchange{}
			
			\noindent\alliancepts{\allianceoptions@change}
			\vspace*{-0.3cm}
			\end{center}
		
		\columnbreak
		
			\begin{center}
			\noindent\dlust{}
			
			\noindent\alliancepts{\allianceoptions@lust}
			\vspace*{-0.3cm}
			\end{center}
		
		\columnbreak

			\begin{center}
			\noindent\pestilence{}
			
			\noindent\alliancepts{\allianceoptions@pestilence}
			\vspace*{-0.3cm}
			\end{center}
			
		\end{multicols}
		\setlength{\columnseprule}{0pt}	
	}{
		\setlength{\columnseprule}{0.5pt}
		\renewcommand{\columnseprulecolor}{\color{black!30}}
		\vspace*{-0.2cm}\begin{multicols}{4}\raggedcolumns
		
			\begin{center}
			\noindent\dchange{}
			
			\noindent\alliancepts{\allianceoptions@change}
			\vspace*{-0.3cm}
			\end{center}
		
		\columnbreak

			\begin{center}
			\noindent\wrath{}
			
			\noindent\alliancepts{\allianceoptions@wrath}
			\vspace*{-0.3cm}
			\end{center}
		
		\columnbreak
		
			\begin{center}
			\noindent\dlust{}
			
			\noindent\alliancepts{\allianceoptions@lust}
			\vspace*{-0.3cm}
			\end{center}
		
		\columnbreak

			\begin{center}
			\noindent\pestilence{}
			
			\noindent\alliancepts{\allianceoptions@pestilence}
			\vspace*{-0.3cm}
			\end{center}
			
		\end{multicols}
		\setlength{\columnseprule}{0pt}
	}
}

}{}


%%% Labels %%%

% Profile

\newcommand{\labels@M}{M}
\newcommand{\labels@WS}{CC}
\newcommand{\labels@BS}{CT}
\newcommand{\labels@S}{F}
\newcommand{\labels@T}{E}
\newcommand{\labels@W}{PV}
\newcommand{\labels@I}{I}
\newcommand{\labels@A}{A}
\newcommand{\labels@Ld}{Cd}
\newcommand{\labels@Invocation}{Invocation} % For Vampire Covenant profiles
\newcommand{\labels@roundbase}{rond} % printed after XX mm for round bases

\newcommand{\Strength}{Force}

% Technical

\newcommand{\labels@range}{Portée}
\newcommand{\labels@point}{pt}
\newcommand{\labels@points}{pts}
\newcommand{\labels@only}{uniquement}
\newcommand{\labels@magic}{Magie}
\newcommand{\labels@pathsused}{Génère ses sorts dans la Discipline}
\newcommand{\labels@model}{figurine}
\newcommand{\labels@models}{figurines}
\newcommand{\labels@Singlemodel}{Figurine \textbf{seule}}

% Unit entry labels

\newcommand{\labels@basesize}{Socle}
\newcommand{\labels@trooptype}{Type de troupe}
\newcommand{\labels@specialrules}{Règles Spéciales}
\newcommand{\labels@alignment}{Allégeance}
\newcommand{\labels@alliance}{Allégeance}
\newcommand{\labels@allianceoptions}{Options d'Allégeance}
\newcommand{\labels@greenhiderace}{Race de Peaux Vertes}
\newcommand{\labels@equipment}{Équipement}
\newcommand{\labels@weapons}{Armes}
\newcommand{\labels@armour}{Armure}
\newcommand{\labels@options}{Options}
\newcommand{\labels@commandgroup}{État-Major}
\newcommand{\labels@charactermounts}{Montures de Personnages}
\newcommand{\labels@mounts}{Montures}
\newcommand{\labels@mount}{Monture}
\newcommand{\labels@specialequipment}{Équipement Spécial}

% Command groups

\newcommand{\labels@champion}{Champion}
\newcommand{\labels@standardbearer}{Porte-Étendard}
\newcommand{\labels@musician}{Musicien}
\newcommand{\labels@singlebannerallowance}{Une seule unité de ce type peut prendre une Bannière Magique}
\newcommand{\labels@condsinglebannerallowance}{Une seule unité de ce type peut prendre une Bannière Magique si}
\newcommand{\labels@bannerallowance}{Bannière Magique}
\newcommand{\labels@veteranstandardbearer}{Peut devenir Porte-Étendard Vétéran}
\newcommand{\labels@championallowance}{Arme Magique}

% Titles

\newcommand{\labels@armylist}{Liste des Troupes}
\newcommand{\labels@lords}{Seigneurs}
\newcommand{\labels@heroes}{Héros}
\newcommand{\labels@coreunits}{Unités de Base}
\newcommand{\labels@specialunits}{Unités Spéciales}
\newcommand{\labels@rareunits}{Unités Rares}
\newcommand{\labels@armywiderules}{Règles Communes de l'Armée}
\newcommand{\labels@armyspecialrules}{Règles Spéciales de l'Armée}
\newcommand{\labels@armoury}{Armurerie}
\newcommand{\labels@magicalitems}{Objets Magiques}
\newcommand{\labels@magicalweapons}{Armes Magiques}
\newcommand{\labels@magicalarmour}{Armures Magiques}
\newcommand{\labels@talismans}{Talismans}
\newcommand{\labels@enchanteditems}{Objets Enchantés}
\newcommand{\labels@arcaneitems}{Objets Cabalistiques}
\newcommand{\labels@magicalbanners}{Bannières Magiques}
\newcommand{\labels@quickrefsheet}{Fiche de Référence}
\newcommand{\labels@changelog}{Change Log}

\newcommand{\labels@lordsInitial}{S}
\newcommand{\labels@heroesInitial}{H}
\newcommand{\labels@coreunitsInitial}{B}
\newcommand{\labels@specialunitsInitial}{S}
\newcommand{\labels@rareunitsInitial}{R}
\newcommand{\labels@mountsInitial}{M}


% Titlepage

\newcommand{\labels@fantasybattles}{Batailles Fantastiques}
\newcommand{\labels@NinthAge}{Le 9\ieme{} Âge}
\newcommand{\labels@armyrules}{Règles de l'Armée}
\newcommand{\labels@frontpagecredits}{%
\labels@fantasybattles{} : \labels@NinthAge{} est un jeu créé et entretenu par la communauté qui met en scène des affrontements de figurines.\newline
Toutes les règles sont disponibles gratuitement sur le site suivant. Vos retours et suggestions sont les bienvenus : \url{http://www.the-ninth-age.com/}
}
\newcommand{\labels@license}{Copyright Creative Commons license : \url{the-ninth-age.com/license.html}}
\newcommand{\labels@tableofcontents}{Sommaire}
\newcommand{\labels@introduction}{%
\begin{center}\noindent{\Largerfontsize\textbf{Note des traducteurs}}\end{center}
\vspace{0.5cm}

Nous souhaitons remercier chaleureusement l'équipe à l'initiative du 9\ieme{} Âge pour leur motivation et leur travail continu pour faire vivre notre passion. Nous espérons que ce jeu saura développer les qualités pour plaire au plus grand nombre et réunir les joueurs, amateurs comme habitués des tournois, autour de règles amusantes et équilibrées, pour finalement s'imposer comme un standard du jeu de figurines. Une grande ambition qui ne pourra s'accomplir que \textbf{grâce à vous}, la communauté, via des retours constructifs, afin de modeler le jeu selon nos désirs. N'étant \textbf{en aucun cas à but lucratif}, le 9\ieme{} Âge part avec un avantage considérable. Les règles des éventuelles nouvelles sorties ne sont pas dictées par le besoin de vendre ces nouveautés. Vous pouvez choisir et acheter vos figurines où bon vous semble, il n'y a pas un unique revendeur toléré. Enfin, vous pouvez être assurés que tant que le 9\ieme{} Âge sera joué, vous disposerez d'un \textbf{support continu et régulier}, celui-ci étant offert par la communauté.

Concernant la traduction en elle-même, nous avons fait de notre mieux pour vous offrir une version de qualité, dont nous espérons qu'elle surpasse celle de la version originale ! Si vous constatez des coquilles, des erreurs, merci de nous les signaler en nous contactant sur le forum du 9\ieme{} Âge, dans le \textbf{sous-forum français} (\url{http://www.the-ninth-age.com/index.php?board/117-french/}). Vous y trouverez aussi les dernières mises à jour. \textbf{En cas de conflit d'interprétation avec la version originale, la version originale fait référence}.

\vspace{0.5cm}
Que ce jeu vous apporte d'innombrables heures de plaisir partagé !

\vspace{1cm}

\ifdef{\translationteam}{
	\begin{multicols}{3}
	\begin{itemize}
		\translationteam
	\end{itemize}
	\end{multicols}
}{}
}
\newcommand{\labels@rulechanges}{% blank ATM
}
\newcommand{\labels@latexcredit}{Document réalisé à l'aide de \LaTeX .}


%%% Technical commands

\newcommand{\only}[1]{(#1 uniquement)}
\newcommand{\free}{gratuit}
\newcommand{\upto}{jusqu'à}
\newcommand{\Upto}{Jusqu'à}
\newcommand{\unlimited}{pas de limite}
\newcommand{\permodel}{/fig.}
\newcommand{\listlastchoice}{ ou}
\newcommand{\notif}[1]{(pas #1)}
\newcommand{\wordand}{et}
\newcommand{\wordwith}{avec}
\newcommand{\ifNmodelsorless}[1]{(#1 figurines ou moins)}
\newcommand{\unitwith}{unité avec}
\newcommand{\From}{De} % From ... to ... models
\newcommand{\wordto}{à}
\newcommand{\wordAll}{Tous}
\newcommand{\spacebeforecolon}{ } % French put a space before colons
\newcommand{\minprice}{Coût min. :}
\newcommand{\mincostfor}{Coût min. pour}
\newcommand{\maxunitsize}{Taille max.}
\newcommand{\additionalfigscost}{Les figurines additionnelles coûtent}


%%% Special rules %%%

\newcommand{\ambush}{Embuscade}
\newcommand{\armourpiercing}[1]{Perforant\ifblank{#1}{}{ (#1)}}
\newcommand{\bodyguard}[1]{Garde du Corps\ifblank{#1}{}{ (#1)}}
\newcommand{\breathweapon}[1]{Attaque de Souffle\ifblank{#1}{}{ (#1)}}
\newcommand{\channel}{Canalisation}
\newcommand{\crushattack}{Attaque Écrasante}
\newcommand{\daemonicinstability}{Instabilité Démoniaque}
\newcommand{\devastatingcharge}{Charge Dévastatrice}
\newcommand{\distracting}{Distrayant}
\newcommand{\divineattacks}{Attaques Divines}
\newcommand{\engineer}{Ingénieur}
\newcommand{\ethereal}{Éthéré}
\newcommand{\fastcavalry}{Cavalerie Légère}
\newcommand{\fear}{Peur}
\newcommand{\fightinextrarank}{Combat avec un Rang Supplémentaire}
\newcommand{\fireborn}{Né du Feu}
\newcommand{\flamingattacks}{Attaques Enflammées}
\newcommand{\flammable}{Inflammable}
\newcommand{\frenzy}{Frénésie}
\newcommand{\fly}[1]{Vol\ifblank{#1}{}{ (#1)}}
\newcommand{\grindingattacks}[1]{Attaques de Broyage\ifblank{#1}{}{ (#1)}}
\newcommand{\hardtarget}{Camouflé}
\newcommand{\hatred}{Haine}
\newcommand{\hellfire}{Feu Démoniaque}
\newcommand{\hidden}{Caché}
\newcommand{\holyattacks}{Attaques Divines} % deprecated, still has to be filled. same as Divine Attacks.
\newcommand{\immunetopsychology}{Immunisé à la Psychologie}
\newcommand{\impacthits}[1]{Touches d'Impact\ifblank{#1}{}{ (#1)}}
\newcommand{\insignificant}{Insignifiant}
\newcommand{\largetarget}{Grande Cible}
\newcommand{\lethalstrike}{Coup Fatal}
\newcommand{\lightningattacks}{Attaques Foudroyantes}
\newcommand{\lightningreflexes}{Réflexes Foudroyants}
\newcommand{\lighttroops}{Troupe Légère}
\newcommand{\magicresistance}[1]{Résistance à la Magie\ifblank{#1}{}{ (#1)}}
\newcommand{\magicalattacks}{Attaques Magiques}
\newcommand{\metalshifting}{Fusion du Métal}
\newcommand{\moveorfire}{Mouvement ou Tir}
\newcommand{\multipleshots}[1]{Tirs Multiples\ifblank{#1}{}{ (#1)}}
\newcommand{\multiplewounds}[2]{Blessures Multiples\ifblank{#1}{}{ (#1\ifblank{#2}{)}{, #2)}}}
\newcommand{\notaleader}{Pas un Meneur}
\newcommand{\otherworldly}{D'Outre-Monde}
\newcommand{\pathmaster}[1]{Maître de la Voie\ifblank{#1}{}{ (#1)}}
\newcommand{\poisonedattacks}{Attaques Empoisonnées}
\newcommand{\quicktofire}{Tir Rapide}
\newcommand{\randommovement}[1]{Mouvement Aléatoire\ifblank{#1}{}{ (#1)}}
\newcommand{\randomattacks}[1]{Attaques Aléatoires\ifblank{#1}{}{ (#1)}}
\newcommand{\regeneration}[1]{Régénération\ifblank{#1}{}{ (#1+)}}
\newcommand{\reload}{Rechargez !}
\newcommand{\requirestwohands}{Arme à deux Mains}
\newcommand{\scythes}{Faux}
\newcommand{\scout}{Éclaireur}
\newcommand{\scouts}{Éclaireurs}
\newcommand{\stomp}[1]{Piétinement\ifblank{#1}{}{ (#1)}}
\newcommand{\strider}[1]{Guide\ifblank{#1}{}{ (#1)}}
\newcommand{\stubborn}{Tenace}
\newcommand{\stupidity}{Stupidité}
\newcommand{\skirmisher}{Tirailleur}
\newcommand{\skirmishers}{Tirailleurs}
\newcommand{\sweepingattack}{Attaque au Passage}
\newcommand{\swiftstride}{Course Rapide}
\newcommand{\thunderouscharge}{Charge Tonitruante}
\newcommand{\terror}{Terreur}
\newcommand{\toxicattacks}{Attaques Toxiques}
\newcommand{\unbreakable}{Indémoralisable}
\newcommand{\undead}{Mort-Vivant}
\newcommand{\unstable}{Instable}
\newcommand{\unwieldy}{Encombrant}
\newcommand{\vanguard}{Avant-Garde}
\newcommand{\volleyfire}{Tir de Volée}
\newcommand{\warplatform}{Plateforme de Guerre}
\newcommand{\wardsave}[1]{Sauvegarde Invulnérable\ifblank{#1}{}{ (#1+)}}
\newcommand{\weaponmaster}{Maître d'Ar\-mes}
\newcommand{\wizardconclave}[1]{Conclave de Sorciers\ifblank{#1}{}{ (#1)}}


%%% Magic %%%


% General

\newcommand{\Pathof}{Voie}

\newcommand{\battle}{Commune}

\newcommand{\anyofthebattlemagic}{dans n'importe laquelle des Voies Communes}
\newcommand{\ONLYanyofthebattlemagic}{Commune de votre choix}

\newcommand{\magiclevel}[1]{\ifnumcomp{#1}{<}{3}{Apprenti Magicien}{Maître Magicien} Niveau #1}
\newcommand{\Level}{Niveau}

\newcommand{\wizard}{Magicien}
\newcommand{\wizards}{Magiciens}

\newcommand{\learnedspell}{Sort Appris}
\newcommand{\learnedspells}{Sorts Appris}
\newcommand{\attributespell}{Attribut de la Voie}
\newcommand{\attributespells}{Attributs de la Voie}
\newcommand{\attributespellnumber}{A}
\newcommand{\traitspell}{Sort Caractéristique}
\newcommand{\traitspells}{Sorts Caractéristiques}
\newcommand{\traitspellnumber}{C}


\newcommand{\boundspell}[1]{Objet de Sort\ifblank{#1}{}{, Puissance #1}}
\newcommand{\boundspells}[1]{Objets de Sort\ifblank{#1}{}{, Puissance #1}}

% Casting Vocabulary

\newcommand{\lostfocus}{Perte de Concentration}
\newcommand{\miscast}{Fiasco}
\newcommand{\miscasts}{Fiascos}
\newcommand{\overwhelmingpower}{Pouvoir Irrésistible}

\newcommand{\breachintheveil}{Brèche dans le Voile}
\newcommand{\catastrophicdetonation}{Explosion Catastrophique}
\newcommand{\witchfire}{Feu de Sorcières}
\newcommand{\sorcerousbacklash}{Contrecoup Magique}
\newcommand{\amnesia}{Amnésie}

% Spell Types

\newcommand{\augment}{Amélioration}
\newcommand{\hex}{Malédiction}
\newcommand{\universal}{Universel}
\newcommand{\missile}{Projectile}
\newcommand{\damage}{Dégâts}
\newcommand{\direct}{Direct}
\newcommand{\focused}{Focalisé}
\newcommand{\vortex}{Vortex}
\newcommand{\ground}{Marqueur}
\newcommand{\linetemplate}{Gabarit de Ligne}
\newcommand{\specialTYPE}{Spécial}
\newcommand{\aura}{Aura}
\newcommand{\castersunit}{Unité du Lanceur}
\newcommand{\caster}{Lanceur}

\newcommand{\template}{Gabarit}

% Spell Durations

\newcommand{\lastsoneturn}{Dure un Tour}
\newcommand{\instant}{Immédiat}
\newcommand{\permanent}{Permanent}
\newcommand{\remainsinplay}{Reste en Jeu}


% Battle Magic

\newcommand{\alchemy}{de l'Alchimie}
\newcommand{\alchemyattribute}{Édit de Fer}
\newcommand{\alchemysignature}{Métal Fondu}
\newcommand{\alchemyspellone}{Lames Enchantées}
\newcommand{\alchemyspelltwo}{Corrosion Rampante}
\newcommand{\alchemyspellthree}{Manteau de Vif-Argent}
\newcommand{\alchemyspellfour}{Pieu d'Argent}
\newcommand{\alchemyspellfive}{Fléau de l'Acier}
\newcommand{\alchemyspellsix}{Transmutation en Or}

\newcommand{\death}{de la Mort}
\newcommand{\deathattribute}{Nuage de Désespoir}
\newcommand{\deathsignature}{Le Baiser de la Faucheuse}
\newcommand{\deathspellone}{Malédiction du Mortel}
\newcommand{\deathspelltwo}{Esprits Dévorants}
\newcommand{\deathspellthree}{Sangsue Psychique}
\newcommand{\deathspellfour}{Moisson d’Âmes}
\newcommand{\deathspellfive}{L’Abîme aussi te Regarde...}
\newcommand{\deathspellsix}{Maelström d’Âmes}

\newcommand{\fire}{du Feu}
\newcommand{\fireattribute}{Feu Déchaîné}
\newcommand{\firesignature}{Boule de Feu}
\newcommand{\firespellone}{Cascade Ardente}
\newcommand{\firespelltwo}{Épées Flamboyantes}
\newcommand{\firespellthree}{Jet de Flammes}
\newcommand{\firespellfour}{Traits Enflammés}
\newcommand{\firespellfive}{Remparts Incandescents}
\newcommand{\firespellsix}{Souffler sur les Braises}

\newcommand{\heavens}{des Cieux}
\newcommand{\heavensattribute}{Second Sceau}
\newcommand{\heavenssignature}{Aquilon}
\newcommand{\heavensspellone}{Bourrasque}
\newcommand{\heavensspelltwo}{Choc Foudroyant}
\newcommand{\heavensspellthree}{Conjonction Astrale}
\newcommand{\heavensspellfour}{Fléau du Ponant}
\newcommand{\heavensspellfive}{Déluge d'Éclairs}
\newcommand{\heavensspellsix}{Appel de la Comète}

\newcommand{\light}{de la Lumière}
\newcommand{\lightattribute}{Lumière Gardienne}
\newcommand{\lightsignature}{Éclat Brûlant}
\newcommand{\lightspellone}{Bouclier Protecteur}
\newcommand{\lightspelltwo}{Étincelle de Courage}
\newcommand{\lightspellthree}{Vitesse Fulgurante}
\newcommand{\lightspellfour}{Toile Scintillante}
\newcommand{\lightspellfive}{Distorsion Temporelle}
\newcommand{\lightspellsix}{Bannissement Divin}

\newcommand{\nature}{de la Nature}
\newcommand{\natureattribute}{Souffle de Vie}
\newcommand{\naturesignature}{Eaux Vivifiantes}
\newcommand{\naturespellone}{Maître de la Terre}
\newcommand{\naturespelltwo}{Le Trône de Chêne}
\newcommand{\naturespellthree}{Esprits des Bois}
\newcommand{\naturespellfour}{Croissance Estivale}
\newcommand{\naturespellfive}{Peau Rocailleuse}
\newcommand{\naturespellsix}{Créatures Souterraines}

\newcommand{\shadows}{des Ombres}
\newcommand{\shadowsattribute}{Course Parmi les Ombres}
\newcommand{\shadowssignature}{Miasmes Obscurs}
\newcommand{\shadowsspellone}{Orbe de Noirceur}
\newcommand{\shadowsspelltwo}{Partir en Fumée}
\newcommand{\shadowsspellthree}{Expérience de Mort Imminente}
\newcommand{\shadowsspellfour}{Char Vaporeux}
\newcommand{\shadowsspellfive}{Ombres Dévorantes}
\newcommand{\shadowsspellsix}{Scalpel Psychique}

\newcommand{\wilderness}{de la Sauvagerie}
\newcommand{\wildernessattribute}{La Chasse Sauvage}
\newcommand{\wildernesssignature}{La Bête qui Sommeille}
\newcommand{\wildernessspellone}{Essaim d’Insectes}
\newcommand{\wildernessspelltwo}{Rage Intérieure}
\newcommand{\wildernessspellthree}{Pieu de Rougebois}
\newcommand{\wildernessspellfour}{Calamité des Bois Sauvages}
\newcommand{\wildernessspellfive}{Tempête Furieuse}
\newcommand{\wildernessspellsix}{Métamorphose
Monstrueuse}

\newcommand{\eightpaths}{Octuple}



% Army Specific Magic

\newcommand{\butchery}{de la Boucherie}
\newcommand{\butcheryattribute}{Sang de Kholag}
\newcommand{\butcherysignature}{Briseur de Dents}
\newcommand{\butcheryspellone}{Buveur de Moelle}
\newcommand{\butcheryspelltwo}{Festin de Tripaille}
\newcommand{\butcheryspellthree}{Concasseur d’Os}
\newcommand{\butcheryspellfour}{Gobeur de Cervelle}
\newcommand{\butcheryspellfive}{Cœur de Troll}
\newcommand{\butcheryspellsix}{Gosier de Géant}

\newcommand{\change}{du Changement}
\newcommand{\changeattribute}{Vent du Changement}
\newcommand{\changesignature}{Feu Azur}
\newcommand{\changespellone}{Feu Rose}
\newcommand{\changespelltwo}{Vague du Changement}
\newcommand{\changespellthree}{Secrets Volés}
\newcommand{\changespellfour}{Règne de la Confusion}
\newcommand{\changespellfive}{Inéluctable Trahison}
\newcommand{\changespellsix}{Portail Éternel}

\newcommand{\thebiggreengods}{des Grands Dieux Verts}
\newcommand{\thebiggreengodsattribute}{Chopez-les !}
\newcommand{\thebiggreengodssignature}{L'Heure de la Raclée}
\newcommand{\thebiggreengodsspellone}{Coup de Boule}
\newcommand{\thebiggreengodsspelltwo}{Poings Bastonneurs}
\newcommand{\thebiggreengodsspellthree}{Même Pas Mal !}
\newcommand{\thebiggreengodsspellfour}{Grande Main Verte}
\newcommand{\thebiggreengodsspellfive}{Boum !}
\newcommand{\thebiggreengodsspellsix}{Le Gros Piétinement}

\newcommand{\thelittlegreengods}{des Petits Dieux Verts}
\newcommand{\thelittlegreengodsattribute}{Fourbe Larcin}
\newcommand{\thelittlegreengodssignature}{Œil Mauvais}
\newcommand{\thelittlegreengodsspellone}{Taillades Sournoises}
\newcommand{\thelittlegreengodsspelltwo}{Bénédiction de la Mère-Araignée}
\newcommand{\thelittlegreengodsspellthree}{Ça Démange ?}
\newcommand{\thelittlegreengodsspellfour}{Chut ! Pas un Bruit...}
\newcommand{\thelittlegreengodsspellfive}{J’vous Arrange Ça}
\newcommand{\thelittlegreengodsspellsix}{Malédiction de la Lune Verte}

\newcommand{\blackmagic}{de la Magie Noire}
\newcommand{\blackmagicattribute}{Soif d’Âmes}
\newcommand{\blackmagicsignature}{Furie de Moraec}
\newcommand{\blackmagicspellone}{Rafale Glaciale}
\newcommand{\blackmagicspelltwo}{Tourbillon de Lames}
\newcommand{\blackmagicspellthree}{Agonie Paralysante}
\newcommand{\blackmagicspellfour}{Marque de la Peur}
\newcommand{\blackmagicspellfive}{Trait d’Énergie Noire}
\newcommand{\blackmagicspellsix}{Terreur Noire}

\newcommand{\disease}{de la Maladie}
\newcommand{\diseaseattribute}{Bénédiction Nécrotique}
\newcommand{\diseasesignature}{Relents de Pestilence}
\newcommand{\diseasespellone}{Haleine Corruptrice}
\newcommand{\diseasespelltwo}{Toucher Putréfiant}
\newcommand{\diseasespellthree}{Excroissance Adipeuse}
\newcommand{\diseasespellfour}{Purge Parasitaire}
\newcommand{\diseasespellfive}{Malédiction du Lépreux}
\newcommand{\diseasespellsix}{Tourbillon Fétide}

\newcommand{\lust}{de la Luxure}
\newcommand{\lustattribute}{Masochisme}
\newcommand{\lustsignature}{Flagellation Démoniaque}
\newcommand{\lustspellone}{Grâce Hypnotique}
\newcommand{\lustspelltwo}{Valse Irrésistible}
\newcommand{\lustspellthree}{Hystérie}
\newcommand{\lustspellfour}{Fantasmagorie}
\newcommand{\lustspellfive}{Déchirement psychique}
\newcommand{\lustspellsix}{Chœur Dissonant}

\newcommand{\necromancy}{de la Nécromancie}
\newcommand{\necromancyattribute}{Tromper la Faucheuse}
\newcommand{\necromancysignature}{Adjuration des Morts}
\newcommand{\necromancyspellone}{Parodie de Vie}
\newcommand{\necromancyspelltwo}{Convocation Profanatoire}
\newcommand{\necromancyspellthree}{Sarabande Macabre}
\newcommand{\necromancyspellfour}{Regard de Setesh}
\newcommand{\necromancyspellfive}{Vol de Jeunesse}
\newcommand{\necromancyspellsix}{Malédiction des Morts}

\newcommand{\ruin}{de la Ruine}
\newcommand{\ruinattribute}{Hordes Sans Fin}
\newcommand{\ruinsignature}{Éclair Noir}
\newcommand{\ruinspellone}{Nourrissons-les...}
\newcommand{\ruinspelltwo}{Souiller le Sol}
\newcommand{\ruinspellthree}{La Faim}
\newcommand{\ruinspellfour}{Appel de la Tempête}
\newcommand{\ruinspellfive}{Rupture Sismique}
\newcommand{\ruinspellsix}{Pour Qui Sonne le Glas}

\newcommand{\forge}{de la Forge}
\newcommand{\forgeattribute}{Fournaise Haineuse}
\newcommand{\forgesignature}{Bouclier de Sombrefeu}
\newcommand{\forgespellone}{Rage Incendiaire}
\newcommand{\forgespelltwo}{Subjugation}
\newcommand{\forgespellthree}{Souffle de Haine}
\newcommand{\forgespellfour}{Anathème de Noirceur}
\newcommand{\forgespellfive}{Cendres Asphyxiantes}
\newcommand{\forgespellsix}{Flammes de la Forge}

\newcommand{\sands}{des Sables}
\newcommand{\sandsattribute}{Les Morts sans Repos}
\newcommand{\sandssignature}{Sirocco}
\newcommand{\sandsspellone}{Lames Maudites}
\newcommand{\sandsspelltwo}{Dessiccation Mortelle}
\newcommand{\sandsspellthree}{Frappes Vengeresses}
\newcommand{\sandsspellfour}{Jugement Divin}
\newcommand{\sandsspellfive}{Sables Mouvants}
\newcommand{\sandsspellsix}{Écho des Gloires
Passées}

\newcommand{\whitemagic}{de la Magie Blanche}
\newcommand{\whitemagicattribute}{Bouclier des Anciens}
\newcommand{\whitemagicsignature}{Traits de Lumière}
\newcommand{\whitemagicspellone}{Résurrection du Phénix}
\newcommand{\whitemagicspelltwo}{Volonté Inspirante}
\newcommand{\whitemagicspellthree}{Sentier Secret}
\newcommand{\whitemagicspellfour}{Bénédiction d’Amhar}
\newcommand{\whitemagicspellfive}{Fusion d’Artefact}
\newcommand{\whitemagicspellsix}{Cataclysme}

% Paths Initials

\newcommand{\alchemyInitials}{A}
\newcommand{\deathInitials}{M}
\newcommand{\fireInitials}{F}
\newcommand{\heavensInitials}{C}
\newcommand{\lightInitials}{L}
\newcommand{\natureInitials}{N}
\newcommand{\shadowsInitials}{O}
\newcommand{\wildernessInitials}{S}

\newcommand{\eightfoldInitials}{8}

\newcommand{\whitemagicInitials}{MB}
\newcommand{\blackmagicInitials}{MN}
\newcommand{\necromancyInitials}{N}
\newcommand{\sandsInitials}{S}
\newcommand{\forgeInitials}{F}
\newcommand{\biggreengodsInitials}{GDV}
\newcommand{\littlegreengodsInitials}{PDV}
\newcommand{\butcheryInitials}{B}
\newcommand{\ruinInitials}{R}
\newcommand{\diseaseInitials}{M}
\newcommand{\lustInitials}{L}
\newcommand{\changeInitials}{C}


%%% Other rules %%%

% Troop types rules

\newcommand{\combinedprofile}{Profil Combiné}
\newcommand{\cavalrysupport}{Soutien de Cavalerie}
\newcommand{\monstrousranks}{Rangs Monstrueux}
\newcommand{\monstroussupport}{Soutien Monstrueux}
\newcommand{\monsterranks}{Rang de Monstre}

\newcommand{\armoursave}{Sauvegarde d'Armure}
\newcommand{\frontrank}{Au Premier Rang}
\newcommand{\hardcover}{Couvert Lourd}
\newcommand{\holdyourground}{Tenez les Rangs}
\newcommand{\inspiringpresence}{Présence Charismatique}
\newcommand{\lightcover}{Couvert Léger}
\newcommand{\ordnance}{Artillerie}
\newcommand{\parry}{Parade}
\newcommand{\raisewounds}{Ressusciter des Figurines}
\newcommand{\recoverwounds}{Récupérer des PVs}
\newcommand{\rnf}{ordinaires}
\newcommand{\general}{Général}
\newcommand{\bsb}{Porteur de la Grande Bannière}
\newcommand{\cannotmarch}{Pas de Marche Forcée}
\newcommand{\veteranstandardbearer}{Porte-Étendard Vétéran}
\newcommand{\swirlingmelee}{Mêlée Tourbillonnante}
\newcommand{\scoringunit}{Unité de Capture}
\newcommand{\scoringunits}{Unités de Capture}


%%% Equipment %%%

\newcommand{\hw}{Arme de Base}
\newcommand{\pw}{Paire d'Armes}
\newcommand{\spear}{Lance}
\newcommand{\halberd}{Hallebarde}
\newcommand{\gw}{Arme Lourde}
\newcommand{\lance}{Lance de Cavalerie}
\newcommand{\lightlance}{Lance Légère}
\newcommand{\flail}{Fléau}

\newcommand{\throwingweapons}{Armes de Jet}
\newcommand{\shortbow}{Arc Court}
\newcommand{\bow}{Arc}
\newcommand{\longbow}{Arc Long}
\newcommand{\handgun}{Arquebuse}
\newcommand{\crossbow}{Arbalète}
\newcommand{\pistol}{Pistolet}
\newcommand{\braceofpistols}{Paire de Pistolets}	

\newcommand{\innatedefence}[1]{Protection Innée\ifblank{#1}{}{~(#1+)}}
\newcommand{\mountsprotection}[1]{Protection de Monture\ifblank{#1}{}{~(#1+)}}
\newcommand{\la}{Armure Légère}
\newcommand{\ha}{Armure Lourde}
\newcommand{\platearmour}{Armure de Plates}
\newcommand{\shield}{Bouclier}
\newcommand{\barding}{Caparaçon}

\newcommand{\cannon}{Canon}
\newcommand{\cannons}{Canons}
\newcommand{\catapult}{Catapulte}
\newcommand{\catapults}{Catapultes}
\newcommand{\volleygun}{Batterie de Tir}
\newcommand{\boltthrower}{Baliste}
\newcommand{\flamethrower}{Lance-Flammes}
\newcommand{\artilleryweapon}{Arme d'Artillerie}


%%% Troop types %%%

\newcommand{\characters}{Personnages}
\newcommand{\infantry}{Infanterie}
\newcommand{\monstrousinfantry}{Infanterie Monstrueuse}
\newcommand{\cavalry}{Cavalerie}
\newcommand{\monstrouscavalry}{Cavalerie Monstrueuse}
\newcommand{\swarm}{Nuée}
\newcommand{\swarms}{Nuées}
\newcommand{\warbeast}{Bête de Guerre}
\newcommand{\warbeasts}{Bêtes de Guerre}
\newcommand{\monster}{Monstre}
\newcommand{\monsters}{Monstres}
\newcommand{\monstrousbeast}{Bête Monstrueuse}
\newcommand{\monstrousbeasts}{Bêtes Monstrueuses}
\newcommand{\chariot}{Char}
\newcommand{\chariots}{Chars}
\newcommand{\riddenmonster}{Monstre Monté}
\newcommand{\riddenmonsters}{Monstres Montés}
\newcommand{\warmachine}{Machine de Guerre}
\newcommand{\warmachines}{Machines de Guerre}


%%% Terrain %%%

\newcommand{\water}{Eaux Peu Profondes}
\newcommand{\forest}{Forêt}
\newcommand{\impassableterrain}{Terrain Infranchissable}


%%% Profile wording

\newcommand{\oneperarmy}{Un par Armée}
\newcommand{\oneofakind}{Uni\-que}
\newcommand{\zerotoXchoice}[1]{0-#1 Choix}
\newcommand{\onechoiceonlyNOC}{(un seul choix)}
\newcommand{\onfootonly}{(à pied uniquement)}
\newcommand{\closecombatonly}{seulement au Corps à Corps}
\newcommand{\Xmodelsorless}[1]{(max. #1 figurines)}
\newcommand{\magicalitemsallowance}{Objets Magiques}
\newcommand{\magicalweaponallowance}{Arme Magique}
\newcommand{\notmagicalarmour}{(pas d'Armure Magique)}
\newcommand{\weapononechoice}{\optionschoice{Arme \onechoiceonlyNOC{} :}}
\newcommand{\weaponschoice}{\optionschoice{Armes :}}
\newcommand{\shootingweapononechoice}{\optionschoice{Arme de Tir \onechoiceonlyNOC{} :}}
\newcommand{\combatweapononechoice}{\optionschoice{Arme de Corps à Corps \onechoiceonlyNOC{} :}}
\newcommand{\combatweapononechoiceTWOCOL}{\optionschoiceTWOCOL{Arme de Corps à Corps \onechoiceonlyNOC{} :}}
\newcommand{\armouronechoice}{\optionschoice{Armure \onechoiceonlyNOC{} :}}
\newcommand{\magiclevelchoice}{\optionschoice{Magie \onechoiceonlyNOC{} :}}
\newcommand{\mustbecomeoneofthefollowing}{\optionschoice{\textbf{Doit} devenir au choix :}}
\newcommand{\mustbecomeoneofthefollowingNOC}{Doit devenir au choix :}
\newcommand{\musttakeoneormoreofthefollowing}{\optionschoice{\textbf{Doit} prendre au moins un choix :}}
\newcommand{\musttakeoneofthefollowing}{\optionschoice{\textbf{Doit} prendre un et un seul choix :}}
\newcommand{\musttakeoneofthefollowingNOC}{Doit choisir entre :}
\newcommand{\uptotwoofthefollowing}{\optionschoice{Jusqu'à deux choix :}}
\newcommand{\uptotwoofthefollowingTWOCOL}{\optionschoiceTWOCOL{Jusqu'à deux choix :}}

\newcommand{\onechoiceonly}{\optionschoice{Un seul choix :}}
\newcommand{\onechoiceonlyTWOCOL}{\optionschoiceTWOCOL{Un seul choix :}}

\newcommand{\maytake}{Peut prendre}




%%% Orcs N Goblins debug, let it as it is

\newcommand{\pershadygit}{debug}
\newcommand{\permadgit}{debug}

%%% Dwarven Holds debug, let it as it is

\newcommand{\perrune}{debug}


%%% Commands to handle strings, better than xstring to handle commands inside the strings %%%

\newcommand{\substitute}[3]{%
  \protected@edef\sub@temp{#1}%
  \saveexpandmode
  \expandarg\StrSubstitute{\sub@temp}{#2}{#3}[#1]%
  \restoreexpandmode
}

\newcommand{\splitatstar}[3]{%
  \protected@edef\split@temp{#1}%
  \saveexpandmode
  \expandarg\StrCut{\split@temp}{*}#2#3%
  \restoreexpandmode
}

\newcommand{\splitatinf}[3]{%
  \protected@edef\split@temp{#1}%
  \saveexpandmode
  \expandarg\StrCut{\split@temp}{<}#2#3%
  \restoreexpandmode
}

\newcommand{\splitatequal}[3]{%
  \protected@edef\split@temp{#1}%
  \saveexpandmode
  \expandarg\StrCut{\split@temp}{=}#2#3%
  \restoreexpandmode
}

\newcommand{\ifsubstring}[4]{%
  \protected@edef\split@temp{#1}%
  \protected@edef\split@tempbis{#2}%
  \saveexpandmode
  \expandarg\IfSubStr{\split@temp}{\split@tempbis}{#3}{#4}%
  \restoreexpandmode
}

\def\removespaces#1{\zap@space#1 \@empty}

%%% Commands for alphabetical ordering %%%

\newcommand{\sortitem}[2][\relax]{%
	\DTLnewrow{list}% Create a new entry
	\ifx#1\relax%
		\DTLnewdbentry{list}{sortlabel}{#2}% Add entry sortlabel (no optional argument)
	\else%
		\DTLnewdbentry{list}{sortlabel}{#1}% Add entry sortlabel (optional argument)
	\fi%
		\DTLnewdbentry{list}{description}{#2}% Add entry description
}
\newenvironment{sortedlist}{%
	\DTLifdbexists{list}{\DTLcleardb{list}}{\DTLnewdb{list}}% Create new/discard old list
}{%
	\DTLsort{sortlabel}{list}% Sort list
	\begin{itemize*}[label={}, itemjoin={,}]%
		\DTLforeach*{list}{\theDesc=description}{%
		\item\theDesc}% Print each item
	\end{itemize*}%
}

\pdfstringdefDisableCommands{\def\textcolor#1{}}

% See language specific file for \addtosortedlist

%%% Database for automatic Quick Ref Sheet %%%

\DTLnewdb{profiles} % Database containing name, category, multiprofile number, profilename (if multi), caraclist, trooptype, invocation for CV.
\newcommand{\profilecategory}{\labels@lords} % Will be updated in relevant categories

\newcommand{\profiledtbfillname}[1]{\DTLnewdbentry{profiles}{name}{#1}}
\newcommand{\profiledtbfillcategory}[1]{\DTLnewdbentry{profiles}{category}{#1}}
\newcommand{\profiledtbfilltrooptype}[1]{\DTLnewdbentry{profiles}{trooptype}{#1}}
\newcommand{\profiledtbfillinvocation}[1]{\DTLnewdbentry{profiles}{invocation}{#1}}
\newcommand{\profiledtbfillprofile}[1]{\DTLnewdbentry{profiles}{profile}{#1}}
\newcommand{\profiledtbfillmultipleprofile}[1]{\DTLnewdbentry{profiles}{multipleprofile}{#1}}

\newcommand{\void}[1]{}
\newcounter{multiprofilecounter}

\newcommand{\profiledtbfillcarac}[1]{%
	\profiledtbfillprofile{#1}
	\parselist{#1}{\locallists@profileslist}% Split of the different profiles in the case of a multiprofile.
	\setcounter{multiprofilecounter}{0}%
	\forlistloop{\stepcounter{multiprofilecounter}\void}{\locallists@profileslist}%
	\expandafter\profiledtbfillmultipleprofile\expandafter{\number\value{multiprofilecounter}}
}


%%% Technical commands %%%

\newcommand{\newrule}{\textcolor{green!50!black}}
\newcommand{\removedrule}[1]{\textcolor{green!50!black}{\sout{#1}}}
\newcommand{\starsymbol}{$\star$}
\newcommand{\refsymbol}{$^\star$}

\newcommand{\inch}{\arcsecond}
\newcommand{\foot}{\arcminute}
\newcommand{\range}[1] {\labels@range~\unit{#1}{\inch}}
\newcommand{\distance}[1] {\unit{#1}{\inch}}
\newcommand{\result}[1] {\texttt{'}$ #1 $\texttt{'}}
\newcommand{\pts}[1]{% First step is to remove spaces if there are some
	\def\numberwithoutspaces{\expandafter\removespaces\expandafter{#1}}%
	% Next step is getting rid of formatting if there are any (bold, color, ...)
	\pdfstringdef\cleannumber{\numberwithoutspaces}%
	% Now we can try if it is 1 or not
	\expandafter\ifstrequal\expandafter{\cleannumber}{1}{#1~\labels@point}{%
	\expandafter\ifstrequal\expandafter{\cleannumber}{0.5}{#1~\labels@point}{%
	#1~\labels@points}}%
}


%%% Fonts and sizes %%%

\newcommand{\bigtitle}[1]{\vspace*{-1.5cm}\section*{}\noindent\begin{center}\Hugefontsize\textbf{\antiquefont\expandafter\uppercase\expandafter{#1}}\end{center}}

\newcommand{\subtitle}[1]{\subsection*{}\noindent{\hugefontsize\antiquefont #1}}

\newcommand{\subsubtitle}[1]{\subsubsection*{}\noindent{\Largerfontsize\antiquefont #1}}

\newcommand{\verysmallfontsize}{\fontsize{4}{4.8}\selectfont}
\newcommand{\smallfontsize}{\fontsize{6}{7.2}\selectfont}
\newcommand{\normalfontsize}{\fontsize{8}{9.6}\selectfont}
\newcommand{\largefontsize}{\fontsize{10}{12}\selectfont}
\newcommand{\largerfontsize}{\fontsize{12}{14.4}\selectfont}
\newcommand{\Largefontsize}{\fontsize{14}{16.8}\selectfont}
\newcommand{\Largerfontsize}{\fontsize{15}{18}\selectfont}
\newcommand{\hugefontsize}{\fontsize{18}{21.6}\selectfont}
\newcommand{\Hugefontsize}{\fontsize{25}{30}\selectfont}

\newcommand{\unitentryformat}[1]{\textit{\largefontsize{#1}}}
\newcommand{\textIT}[1]{\textit{\largefontsize{#1}}}


%%% Titles %%%

\newcommand{\lordstitle}{\def\logolocalpath{../Layout/pics/logo_lord.png}\bigtitle{\labels@lords}}
\newcommand{\heroestitle}{%
\def\logolocalpath{../Layout/pics/logo_hero.png}%
\clearpage\bigtitle{\labels@heroes}%
\renewcommand{\profilecategory}{\labels@heroes}%
}
\newcommand{\coreunitstitle}{%
\def\logolocalpath{../Layout/pics/logo_core.png}%
\clearpage\bigtitle{\labels@coreunits}%
\renewcommand{\profilecategory}{\labels@coreunits}%
}
\newcommand{\specialunitstitle}{%
\def\logolocalpath{../Layout/pics/logo_special.png}%
\clearpage\bigtitle{\labels@specialunits}%
\renewcommand{\profilecategory}{\labels@specialunits}%
}
\newcommand{\rareunitstitle}{%
\def\logolocalpath{../Layout/pics/logo_rare.png}%
\clearpage\bigtitle{\labels@rareunits}%
\renewcommand{\profilecategory}{\labels@rareunits}%
}
\newcommand{\mountstitle}{%
\def\logolocalpath{../Layout/pics/logo_mount.png}%
\clearpage\bigtitle{\labels@charactermounts}%
\renewcommand{\profilecategory}{\labels@mounts}%
}

\newcommand{\startarmywiderules}{\newpage\bigtitle{\labels@armywiderules}\largefontsize}
\newcommand{\closearmywiderules}{\normalfontsize}
\newcommand{\armywideruleentry}[1]{\subtitle{#1}\vspace{5pt}}

\newcommand{\startarmyspecialrules}{\bigtitle{\labels@armyspecialrules}\largefontsize}
\newcommand{\closearmyspecialrules}{\normalfontsize}
\newcommand{\armyspecialruleentry}[1]{\subtitle{#1}\vspace{5pt}}

\newcommand{\startarmyarmoury}{\bigtitle{\labels@armoury}\largefontsize\subtitle{}}
\newcommand{\closearmyarmoury}{\normalfontsize}

\newcommand{\startarmymagicalitems}{\newpage\largefontsize\bigtitle{\labels@magicalitems}\begin{multicols}{2}\raggedcolumns}
\newcommand{\closearmymagicalitems}{\end{multicols}\normalfontsize}

\newcommand{\armymagicalweapons}{\subtitle{\labels@magicalweapons}}
\newcommand{\armymagicalarmour}{\subtitle{\labels@magicalarmour}}
\newcommand{\armytalismans}{\subtitle{\labels@talismans}}
\newcommand{\armyenchanteditems}{\subtitle{\labels@enchanteditems}}
\newcommand{\armyarcaneitems}{\subtitle{\labels@arcaneitems}}
\newcommand{\armymagicalbanners}{\subtitle{\labels@magicalbanners}}

\newcommand{\startarmynewsection}[1]{\newpage\bigtitle{#1}\largefontsize}
\newcommand{\closearmynewsection}{\normalfontsize}

\newcommand{\armynewsubsection}[1]{\subtitle{#1}\vspace{5pt}}
\newcommand{\armynewsubsubsection}[1]{\subsubtitle{#1}\vspace{3pt}}

\newcommand{\armylist}{\clearpage}

\newcommand{\quickrefsheettitle}{\clearpage\newgeometry{top=1.6cm, bottom=2cm, left=1cm, right=1cm}\bigtitle{\labels@quickrefsheet}\vspace*{0.4cm}}
\newcommand{\changelogtitle}{\clearpage\bigtitle{\labels@changelog}\spaceaftersection{}}

\newcommand{\spaceaftersection}{\vspace{0.8cm}}

\newcommand{\separator}{\noindent\begin{center}\textcolor{black!30}{\rule{0.7\columnwidth}{2pt}}\end{center}}


%%% Custom lists and description for first sections of the army books

\newcommand{\startpricelist}{\begin{samepage}\begin{description}[leftmargin=0.3cm, labelindent=0cm, labelsep=0.1cm]}
\def\endpricelist{\end{description}\end{samepage}}
\newcommand{\pricelistitem}[2]{\item \option{\textbf{#1}}{#2}\newline}

\newcommand{\startpricelistNSP}{\begin{description}[leftmargin=0.3cm, labelindent=0cm, labelsep=0.1cm]}
\def\endpricelistNSP{\end{description}}

\newcommand{\startitemlist}{\begin{multicols}{2}\raggedcolumns\begin{description}[leftmargin=0.3cm, labelindent=0cm, labelsep=0.1cm]}
\def\enditemlist{\end{description}\end{multicols}}
\newcommand{\listitem}[1]{\item[#1\spacebeforecolon{}:]}

\newcommand{\startitemlistonecol}{\begin{description}[leftmargin=0.3cm, labelindent=0cm, labelsep=0.1cm]}
\def\enditemlistonecol{\end{description}}
\newcommand{\listitemonecol}[1]{\item \textbf{#1\spacebeforecolon{}:}\newline}

\newenvironment{customitemize}{\begin{description}[leftmargin=0.3cm, labelindent=0cm, labelsep=0cm]}{\end{description}}
\newenvironment{customsubitemize}{\begin{itemize}[label={-}, labelsep=0.1cm, topsep=0cm, parsep=0cm, itemsep=0cm, leftmargin=0.4cm, labelindent=0cm]}{\end{itemize}}

%%% Table parameters %%%

\newcolumntype{M}[1]{>{\centering\let\newline\\\arraybackslash\hspace{0pt}}m{#1}}


%%%  Lists handling %%%

\newcommand{\addlocallist}{\listadd\locallists@dummy}%
\NewDocumentCommand{\parsespacelist}{>{\SplitList{ }} m }{%
	\ProcessList{#1}{\addlocallist}%
}%
\NewDocumentCommand{\parsecommalist}{>{\SplitList{,}} m }{%
	\ProcessList{#1}{\addlocallist}%
}%
\newcommand{\parselist}[3][,]{%
	\renewcommand\addlocallist{\listadd#3}%
  	\undef#3%
  	\ifstrequal{#1}{ }{\parsespacelist{#2}}{\parsecommalist{#2}}%
}


%%% Profiles handling %%%

% Element of a table that contains the characteristics of a model (or part of a model)
\newcommand\caraclist[1]{
	\parselist[ ]{#1}{\locallists@caraclist}%
	\forlistloop{&}{\locallists@caraclist}%
}

\newcommand\caraclistbold[1]{
	\parselist[ ]{#1}{\locallists@caraclist}%
	\forlistloop{&\bfseries}{\locallists@caraclist}%
}

% Line of a profile table, including bottom line. It is meant to contain the name of the model (or part), its characteristics (preferably, the second argument should contain the \carac macro), troop type and base size.
\newcommand{\profilefirstline}[4]{#1 & #2 &   & #3 & #4 }

% Start of a profile table. Includes the table commands, and the column labels. \profilecellsize is the size of the characteristics cells in the profile.
\newcommand{\profilecellsize}{0.56cm}
\newcommand{\profilestart}{%
	\noindent %
	\begin{tabular}{@{}p{3cm}@{}M{\profilecellsize}@{}M{\profilecellsize}@{}M{\profilecellsize}@{}M{\profilecellsize}@{}M{\profilecellsize}@{}M{\profilecellsize}@{}M{\profilecellsize}@{}M{\profilecellsize}@{}M{\profilecellsize}@{}p{2.7cm}@{}p{3.3cm}@{}p{2cm}@{}}%
	 &% \textbf{\labels@profile}
	\labels@M & \labels@WS & \labels@BS & \labels@S & \labels@T & \labels@W & \labels@I & \labels@A & \labels@Ld &%
	&%
	{\unitentryformat{\labels@trooptype}} &%
	{\unitentryformat{\labels@basesize}}%
}

% End of a profile table.
\newcommand{\profileend}{\end{tabular}}

% Algorithm to automatically use and fill previous command, with coherence check.
\providebool{profilefirst}
\newcommand{\profileitem}[1]{%
	\tabularnewline%
	\splitatinf{#1}\local@unitname\local@unitprofile%
	\local@unitname \expandafter\caraclistbold\expandafter{\local@unitprofile}%
	&%
	& \ifbool{profilefirst}{\unit@type}{}%
	& \ifbool{profilefirst}{%
		\ifsubstring{\unit@basesize}{x}{% Rectangular base
			\unit{\unit@basesize}{\milli\meter}%
		}{% Circular base
			\unit{\unit@basesize}{\milli\meter} \labels@roundbase%
		}%
	}{}%
	\global\boolfalse{profilefirst}%
}
\newcommand{\profile}[1]{%
	\parselist{#1}{\locallists@profileslist}%
	\profilestart%
	\global\booltrue{profilefirst}%
	\forlistloop{\profileitem}{\locallists@profileslist}%
	\profileend%
}


%%% Profiles handling in case of invocation %%%

\newcommand{\invocprofilestart}{%
	\noindent %
	\begin{tabular}{@{}p{3cm}@{}M{\profilecellsize}@{}M{\profilecellsize}@{}M{\profilecellsize}@{}M{\profilecellsize}@{}M{\profilecellsize}@{}M{\profilecellsize}@{}M{\profilecellsize}@{}M{\profilecellsize}@{}M{\profilecellsize}@{}M{2.2cm}@{}p{0.5cm}@{}p{3.3cm}@{}p{2cm}@{}}%
	 &% \textbf{\labels@profile}
	\labels@M & \labels@WS & \labels@BS & \labels@S & \labels@T & \labels@W & \labels@I & \labels@A & \labels@Ld & \unitentryformat{\labels@Invocation} &%
	&%
	{\unitentryformat{\labels@trooptype}} &%
	{\unitentryformat{\labels@basesize}}%
}

\newcommand{\invocprofileitem}[1]{%
	\tabularnewline%
	\splitatinf{#1}\local@unitname\local@unitprofile%
	\local@unitname \expandafter\caraclistbold\expandafter{\local@unitprofile}%
	& \ifbool{profilefirst}{\unit@invocation}{} &%
	& \ifbool{profilefirst}{\unit@type}{}%
	& \ifbool{profilefirst}{\unit{\unit@basesize}{\milli\meter}}{}%
	\global\boolfalse{profilefirst}%
}

\newcommand{\invocprofile}[1]{%
	\parselist{#1}{\locallists@profileslist}%
	\invocprofilestart%
	\global\booltrue{profilefirst}%
	\forlistloop{\invocprofileitem}{\locallists@profileslist}%
	\profileend%
}


%%%%%%%%%%%%%%%%%%
%%% Unit rules %%%
%%%%%%%%%%%%%%%%%%

%%% Entry title command %%%

\newcommand{\unitentry}[2]{\ifdefempty{#1}{}{\noindent #2}}


%%% Special rules %%%

% Special rules listing for a unit, with alphabetical order.
\newcommand{\ruleslist}[1]{%
	\parselist[,]{#1}{\locallists@ruleslist}%
	\begin{sortedlist}%
		\forlistloop{\addtosortedlist}{\locallists@ruleslist}%
	\end{sortedlist}%
}

% Special rules entry.
\newcommand{\specialrules}[1]{\unitentry{#1}{\unitentryformat{\labels@specialrules\spacebeforecolon{}:}\newline\hspace*{-\fontdimen2\font}\expandafter\ruleslist\expandafter{#1}.}}
\newcommand{\commonspecialrules}[2]{\unitentry{#2}{\unitentryformat{#1\spacebeforecolon{}:}\newline\hspace*{-\fontdimen2\font}\expandafter\ruleslist\expandafter{#2}.}}


%%% Magical abilities %%%

% Paths listing for a unit.
\newcommand{\pathslist}[1]{%
	\parselist[,]{#1}{\locallists@pathslist}%
	\begin{itemize*}[label={}, itemjoin={,}, itemjoin*={\listlastchoice}]%
		\forlistloop{\item}{\locallists@pathslist}%
	\end{itemize*}%
}

% Magic entry.
\newcommand{\magic}[2]{\unitentry{#2}{\unitentryformat{\labels@magic\spacebeforecolon{}: }\newline\ifdefempty{#1}{}{\textbf{\magiclevel{#1}}. }\labels@pathsused\expandafter\pathslist\expandafter{#2}.}}

% Wizard Conclave.
\newcommand{\magicwizardconclave}[1]{\unitentry{#1}{\unitentryformat{\labels@magic\spacebeforecolon{}: }\newline\textbf{\wizardconclave{}}\spacebeforecolon{}: #1.}}


%%% Equipment %%%

% Equipment listing.
\newcommand{\equipmentlist}[1]{%
	\parselist[,]{#1}{\locallists@equipmentlist}%
	\begin{sortedlist}%
		\forlistloop{\addtosortedlist}{\locallists@equipmentlist}%
	\end{sortedlist}%
}

% Equipment entry.
\newcommand{\weapons}[1]{\unitentry{#1}{\unitentryformat{\labels@weapons\spacebeforecolon{}:}\newline\hspace*{-\fontdimen2\font}\expandafter\equipmentlist\expandafter{#1}.}}

\newcommand{\armour}[1]{\unitentry{#1}{\unitentryformat{\labels@armour\spacebeforecolon{}:}\newline\hspace*{-\fontdimen2\font}\expandafter\equipmentlist\expandafter{#1}.}}


%%% Alignment %%%

\newcommand{\alignment}[1]{\unitentry{#1}{\unitentryformat{\labels@alignment\spacebeforecolon{}:}\newline\textbf{#1}.}}

%%% Green Hide Race %%%

\newcommand{\greenhideraceentry}[1]{\unitentry{#1}{\unitentryformat{\labels@greenhiderace\spacebeforecolon{}:}\newline\textbf{#1}.}}


%%% Options %%%

% Frame commands.
\newcommand{\optionsframestart}{\begin{innerframe}[\labels@options]}
\newcommand{\optionsframeend}{\end{innerframe}}

% Options listing.
\newcommand{\optionslist}[1]{%
	\parselist[,]{#1}{\locallists@optionslist}%
	\begin{description}[leftmargin=0.3cm, labelindent=0cm, labelsep=0cm, itemsep=0cm, parsep=0cm]%
		\forlistloop{\item\setoption}{\locallists@optionslist}%
	\end{description}%
}

% Options entry.
\newcommand{\options}[1]{\ifdefempty{#1}{}{\optionsframestart\vspace*{-0.4cm}\unitentry{#1}{\expandafter\optionslist\expandafter{#1}}\optionsframeend}}

% Option specific commands.
\newcommand{\setoption}[1]{%
	\noexpandarg\StrCut{#1}{=}\optiontext\optionvalue%
	\expandafter\ifstrequal\expandafter{\optionvalue}{}{%
		\optiontext%
	}{%
	\ifsubstring{\optionvalue}{\free}{%
		\option[\free]{\optiontext}{\optionvalue}%
	}{%
	\ifsubstring{\optionvalue}{\unlimited}{%
		\option[\unlimited]{\optiontext}{\optionvalue}%
	}{%
	\ifsubstring{\optionvalue}{\upto}{%
		\splitatinf{\optionvalue}\myoption\myvalue%
		\option[\upto]{\optiontext}{\myvalue}%
	}{%
	\ifsubstring{\optionvalue}{\permodel}{%
		\splitatinf{\optionvalue}\myoption\myvalue%
		\option[\permodel]{\optiontext}{\myvalue}%
	}{%
	\ifsubstring{\optionvalue}{\pershadygit}{% For Orcs N Goblins
		\splitatinf{\optionvalue}\myoption\myvalue%
		\option[\pershadygit]{\optiontext}{\myvalue}%
	}{%
	\ifsubstring{\optionvalue}{\permadgit}{% For Orcs N Goblins
		\splitatinf{\optionvalue}\myoption\myvalue%
		\option[\permadgit]{\optiontext}{\myvalue}%
	}{%	
	\ifsubstring{\optionvalue}{\perrune}{% For Dwarven Holds
		\splitatinf{\optionvalue}\myoption\myvalue%
		\option[\perrune]{\optiontext}{\myvalue}%
	}{%	
		\option{\optiontext}{\optionvalue}%
	}}}}}}}}%
}

\newcommand{\option}[3][]{#2\predotfill\dotfill\nobreak%
	% Add \upto token if necessary.
	\ifstrequal{#1}{\upto}{\upto~}{}%
	% The option can be free, have an unlimited cost, or have a points cost.
	\ifstrequal{#1}{\free}{\free}{\ifstrequal{#1}{\unlimited}{\unlimited}{\pts{#3}}}%
	% Add \permodel if necessary.
	\ifstrequal{#1}{\permodel}{\nobreak\permodel}{}%
	% Add \persomething if necessary.
	\ifstrequal{#1}{\pershadygit}{\nobreak\pershadygit}{}% For Orcs N Goblins
	\ifstrequal{#1}{\permadgit}{\nobreak\permadgit}{}% For Orcs N Goblins
	\ifstrequal{#1}{\perrune}{\nobreak\perrune}{}% For Dwarven Holds
}
\newcommand\optionschoice[2]{%
	\parselist[,]{#2}{\locallists@optionschoice}%
	#1%
	\begin{itemize}[label={}, parsep=0cm, labelindent=0cm, labelwidth=0cm, noitemsep, topsep=0em, leftmargin=0.3cm]%
	\forlistloop{\item\setoption}{\locallists@optionschoice}%
	\end{itemize}%
}

% Option description in army desc.
\newcommand{\optiondef}[3]{\option{\textbf{#1}}{#2}\ifblank{#3}{}{\\{#3}}}


%%% Mount options %%%

% Frame commands.
\newcommand{\mountsframestart}{\begin{innerframe}[\labels@mounts]}
\newcommand{\mountsframeend}{\end{innerframe}}

% Mount listing.
\newcommand{\mountslist}[1]{%
	\parselist[,]{#1}{\locallists@mountslist}%
	\begin{description}[leftmargin=0.3cm, labelindent=0cm, labelsep=0cm, itemsep=0cm, parsep=0cm]%
		\forlistloop{\item\setoption}{\locallists@mountslist}%
	\end{description}%
}

% Mount entry.
\newcommand{\mounts}[1]{\ifdefempty{#1}{}{\mountsframestart\vspace*{-0.4cm}\unitentry{#1}{\expandafter\mountslist\expandafter{#1}}\mountsframeend}}


%%% Command group %%%

% Command group specific commands.
\define@key{commandgroup}{restriction}            {\def\commandgroup@restriction{#1}}
\define@key{commandgroup}{champion}               {\def\commandgroup@champion{#1}}
\define@key{commandgroup}{championallowance}      {\def\commandgroup@championallowance{#1}}
\define@key{commandgroup}{championoption}         {\def\commandgroup@championoption{#1}}
\define@key{commandgroup}{championprerestriction} {\def\commandgroup@championprerestriction{#1}}
\define@key{commandgroup}{championrestriction}    {\def\commandgroup@championrestriction{#1}}
\define@key{commandgroup}{banner}                 {\def\commandgroup@banner{#1}}
\define@key{commandgroup}{bannerallowance}        {\def\commandgroup@bannerallowance{#1}}
\define@key{commandgroup}{veteranstandardbearer}  {\def\commandgroup@veteranstandardbearer{#1}}
\define@key{commandgroup}{singlebannerallowance}  {\def\commandgroup@singlebannerallowance{#1}}
\define@key{commandgroup}{condsinglebannerallowance}  {\def\commandgroup@condsinglebannerallowance{#1}}
\define@key{commandgroup}{banneroption}           {\def\commandgroup@banneroption{#1}}
\define@key{commandgroup}{bannerrestriction}      {\def\commandgroup@bannerrestriction{#1}}
\define@key{commandgroup}{musician}               {\def\commandgroup@musician{#1}}
\define@key{commandgroup}{musicianrestriction}    {\def\commandgroup@musicianrestriction{#1}}
\newcommand{\defcommandgroup}{%
	\setkeys{commandgroup}{restriction=,
	                       champion=, championallowance=, championoption=, championprerestriction=, 
	                       championrestriction=, banner=, bannerallowance=, veteranstandardbearer=, 
	                       singlebannerallowance=, condsinglebannerallowance=, banneroption=, 
	                       bannerrestriction=, musician=, musicianrestriction=}%
	\setkeys{commandgroup}%
}

% Frame commands.
\newcommand{\commandgroupframestart}{\begin{innerframe}[\labels@commandgroup]}
\newcommand{\commandgroupframeend}{\end{innerframe}}

% Command group entry.
\newcommand{\commandgroup}[1]{%
	\defcommandgroup{#1}%
	\ifstrempty{#1}{}{\commandgroupframestart\vspace*{-0.2cm}%
		\begin{description}[leftmargin=0.3cm, labelindent=0cm, labelsep=0cm, itemsep=0cm, parsep=0cm]%
			% Command group title, including restrictions applying to all the command group
			\item \textbf{\expandafter\ifblank\expandafter{\commandgroup@restriction}{}{ \only{\commandgroup@restriction}\spacebeforecolon{}: }} 
			% Champion handling.
			\ifdefempty{\commandgroup@champion}{}{% We have a champion!
			\ifdefempty{\commandgroup@championprerestriction}{% There is no prerestriction to have a champion
				\item \hspace*{-0.04cm}\option{\labels@champion%
					% Possible restrictions to taking a champion
				    \expandafter\ifblank\expandafter{\commandgroup@championrestriction}{}{ \only{\commandgroup@championrestriction}}%
				    % Cost of a champion
				    }{\commandgroup@champion}%
				    % Magical allowance of the champion. Should probably not be used, champion option can do it as well and is more flexible.
					\ifdefempty{\commandgroup@championallowance}{}{\par\option[\upto]{\hspace*{0.3cm}- \labels@championallowance}{\commandgroup@championallowance}}%
					% Any option available to the champion, in the form option:cost
					\ifdefempty{\commandgroup@championoption}{}{%
						\splitatinf{\commandgroup@championoption}\local@option\local@cost%
						\par\option{\hspace*{0.3cm}- \local@option}{\local@cost}}%
			}{% There is a pre-restriction to have a champion
				\item \hspace*{-0.04cm}\commandgroup@championprerestriction	\newline%
				\option{\labels@champion}{\commandgroup@champion}%
				% Magical allowance of the champion. Should probably not be used, champion option can do it as well and is more flexible.
				\ifdefempty{\commandgroup@championallowance}{}{\par\option[\upto]{\hspace*{0.3cm}- \labels@championallowance}{\commandgroup@championallowance}}%
				% Any option available to the champion, in the form option:cost
				\ifdefempty{\commandgroup@championoption}{}{%
					\splitatinf{\commandgroup@championoption}\local@option\local@cost%
					\par\option{\hspace*{0.3cm}- \local@option}{\local@cost}}%
			} %End of the prerestriction of not condition
			}% End of champion handling
			\ifdefempty{\commandgroup@banner}{}{% We have a banner!
				\item \hspace*{-0.04cm}\option{\labels@standardbearer%
					% Possible restrictions to taking a banner
				    \expandafter\ifblank\expandafter{\commandgroup@bannerrestriction}{}{ \only{\commandgroup@bannerrestriction}}%
				    % Cost of a banner
				    }{\commandgroup@banner}%
				    % Magical banner, if all units of this type can take one.
					\ifdefempty{\commandgroup@bannerallowance}{}{\par\option[\upto]{\hspace*{0.3cm}- \labels@bannerallowance}{\commandgroup@bannerallowance}}%
					% Magical banner, if Veteran.
					\ifdefempty{\commandgroup@veteranstandardbearer}{}{\par\hspace*{0.3cm}- \labels@veteranstandardbearer%
					\expandafter\ifstrequal\expandafter{\commandgroup@veteranstandardbearer}{*}{*}{}%
					}%
					% Magical banner, if only one unit of this type can take one.
					\ifdefempty{\commandgroup@singlebannerallowance}{}{\par\option[\upto]{\hspace*{0.3cm}- \labels@singlebannerallowance}{\commandgroup@singlebannerallowance}}%
					% Magical banner, if only one unit of this type can take one, but with condtions.
					\ifdefempty{\commandgroup@condsinglebannerallowance}{}{%
						\splitatinf{\commandgroup@condsinglebannerallowance}\local@option\local@cost%
						\par\option[\upto]{\hspace*{0.3cm}- \labels@condsinglebannerallowance \local@option}{\local@cost}}%
					% Additional option for the banner, such as Hill Goblin Lookouts for Ogres
					\ifdefempty{\commandgroup@banneroption}{}{%
						\splitatinf{\commandgroup@banneroption}{\local@option}{\local@cost}%
						\par\option{\hspace*{0.3cm}- \local@option}{\local@cost}%
					}%
			}%
			\ifdefempty{\commandgroup@musician}{}{% We have a musician!
				\item \hspace*{-0.04cm}\option{\labels@musician%
					% Possible restrictions to taking a musician
				    \expandafter\ifblank\expandafter{\commandgroup@musicianrestriction}{}{ \only{\commandgroup@musicianrestriction}}%
				    % Cost of a musician
				    }{\commandgroup@musician}%
			}%
		\end{description}%
	\commandgroupframeend%
	 }%
}


%%% Unit rules %%%

% Frame commands.
\newcommand{\unitrulesframestart}{\begin{innerframe}[\labels@specialrules]}
\newcommand{\unitrulesframeend}{\end{innerframe}}

% Unit rules specific commands.
\newcommand{\unitrule}[2]{\item[#1\spacebeforecolon{}:]#2}

% Unit rule entry.
\newcommand{\unitrules}[1]{\ifdefempty{#1}{}{\unitrulesframestart\vspace*{-0.05cm}\begin{description}[leftmargin=0.3cm, labelindent=0cm, labelsep=0.1cm, itemsep=0.2cm, parsep=0cm]#1\end{description}\unitrulesframeend}}


%%% Special equipment %%%

% Frame commands.
\newcommand{\unitequipmentframestart}{\begin{innerframe}[\labels@specialequipment]}
\newcommand{\unitequipmentframeend}{\end{innerframe}}

% Special equipment specific commands.
\newcommand{\equipmentdef}[2]{\item[#1\spacebeforecolon{}:]#2}

% Special equipment entry.
\newcommand{\unitequipment}[1]{\ifdefempty{#1}{}{\unitequipmentframestart\vspace*{-0.05cm}\begin{description}[leftmargin=0.3cm, labelindent=0cm, labelsep=0.1cm, itemsep=0.2cm, parsep=0cm]#1\end{description}\unitequipmentframeend}}






%%%%%%%%%%%%%%%%%%%%%%%%%%%%%%%%
%%% Profile input and layout %%%
%%%%%%%%%%%%%%%%%%%%%%%%%%%%%%%%

%%% Input parameters %%%

\define@key{unit}{notinQRS}{\def\unit@notinQRS{#1}}
\define@key{unit}{name}{\def\unit@name{#1}}
\define@key{unit}{QRSname}{\def\unit@QRSname{#1}}
\define@key{unit}{profile}{\def\unit@profile{#1}}
\define@key{unit}{cost}{\def\unit@cost{#1}}
\define@key{unit}{invocation}{\def\unit@invocation{#1}}
\define@key{unit}{costpermodel}{\def\unit@costpermodel{#1}}
\define@key{unit}{maxmodels}{\def\unit@maxmodels{#1}}
\define@key{unit}{type}{\def\unit@type{#1}}
\define@key{unit}{unitsize}{\def\unit@unitsize{#1}}
\define@key{unit}{basesize}{\def\unit@basesize{#1}}
\define@key{unit}{commonspecialrules}{\def\unit@commonspecialrules{#1}}
\define@key{unit}{commontype}{\def\unit@commontype{#1}}
\define@key{unit}{commonspecialrulesB}{\def\unit@commonspecialrulesB{#1}}
\define@key{unit}{commontypeB}{\def\unit@commontypeB{#1}}
\define@key{unit}{specialrules}{\def\unit@specialrules{#1}}
\define@key{unit}{magiclevel}{\def\unit@magiclevel{#1}}
\define@key{unit}{magicpaths}{\def\unit@magicpaths{#1}}
\define@key{unit}{equipment}{\def\unit@equipment{#1}}
\define@key{unit}{alignment}{\def\unit@alignment{#1}}
\define@key{unit}{greenhiderace}{\def\unit@greenhiderace{#1}}
\define@key{unit}{weapons}{\def\unit@weapons{#1}}
\define@key{unit}{armour}{\def\unit@armour{#1}}
\define@key{unit}{wizardconclave}{\def\unit@wizardconclave{#1}}
\define@key{unit}{unitequipment}{\def\unit@unitequipment{#1}}
\define@key{unit}{options}{\def\unit@options{#1}}
\define@key{unit}{mounts}{\def\unit@mounts{#1}}
\define@key{unit}{commandgroup}{\def\unit@commandgroup{#1}}
\define@key{unit}{unitrules}{\def\unit@unitrules{#1}}
\define@key{unit}{additional}{\def\unit@additional{#1}}


%%% Frames definition %%%

% Unit's big frame.
\tikzset{unitprice/.style={draw=white, fill=white, rectangle, rounded corners, right, minimum height=0.7cm}}
\tikzset{unittitle/.style={draw=white, fill=white, rectangle, rounded corners, right, minimum height=0.7cm, font=\bfseries}}
\tikzset{unitlogo/.style={draw=white, fill=white, rectangle, right, minimum height=0.7cm}}

\newenvironment{unitframe}[2][]{%
	\mdfsetup{%
		nobreak=true,%
		linewidth=1pt,%
		linecolor=black!30,%
		roundcorner=5pt,%
		backgroundcolor=white,%
		innertopmargin=1.2\baselineskip,
		innerbottommargin=1.2\baselineskip,
		singleextra={
			\expandafter\ifblank\expandafter{\unit@cost}{}{%
				\node[unitprice,anchor=east,xshift=-0.5cm] at (P)%
					{%
						{{\smallfontsize\minprice} \Largefontsize\pts{\textbf{\unit@cost}}}%
					};
				}%
				\node[unittitle,xshift=0.5cm] at (P-|O)%
					{\Largefontsize\antiquefont\uppercase\expandafter\expandafter\expandafter{\unit@name}};
				\node[unitlogo, xshift=8.1cm, yshift=0.1cm] at (P-|O)%
					{\includegraphics[width=1.2cm]{\logolocalpath}};
		}
	}%
	\begin{mdframed}[]\relax%
}%
{%
\end{mdframed}%
}

% Inner small frames for options, special rules definition, ...
\tikzset{innertitle/.style={fill=white, rectangle, rounded corners, right, minimum height=8pt, xshift=0.5cm}}

\newenvironment{innerframe}[1][]{%
	\mdfsetup{%
		innerleftmargin=5pt,%
		innerrightmargin=5pt,%
		linecolor=black!30,%
		linewidth=0.5pt,%
		roundcorner=5pt,%
		backgroundcolor=white,%
		innertopmargin=1.1\baselineskip,
		singleextra={
		\node[innertitle] at (P-|O)%
			{\unitentryformat{#1}};
		}
	}%
	\vspace*{-0.2cm}\begin{mdframed}[]\relax%
}%
{%
\end{mdframed}%
}

%%% Command to add a new unit definition %%%

\newcommand{\defunit}{
	\setkeys{unit}{%
		notinQRS=, name=, QRSname=, profile=, cost=, invocation=, costpermodel=, maxmodels=, type=, unitsize=, basesize=, commonspecialrules=, commontype=, commonspecialrulesB=, commontypeB=, specialrules=, magiclevel=, magicpaths=, alignment=, greenhiderace=, equipment=, weapons=, armour=, wizardconclave=, unitequipment=, options=, mounts=, commandgroup=, unitrules=, additional=%
	}%
	\setkeys{unit}%
}

\newcommand{\showunit}[1]{
	\defunit{#1}
	\begin{unitframe}[\unit@name]{\unit@cost}
	\mdfsetup{style=defaultoptions}
	\expandafter\ifblank\expandafter{\unit@unitsize}{}{%
	\expandafter\ifstrequal\expandafter{\unit@unitsize}{1}{% single model
		% Can you add model to this single model ?
		\expandafter\ifblank\expandafter{\unit@maxmodels}{% no		
			{\hspace*{0.25cm}\labels@Singlemodel}%
		}{% yes
			{\hspace*{0.25cm}\mincostfor{} \textbf{1} \labels@model{}. \maxunitsize{}\spacebeforecolon{}: \textbf{\unit@maxmodels} \labels@models{}.\hfill \additionalfigscost{} {\largefontsize\pts{\textbf{\unit@costpermodel{}}}\permodel}\hspace*{0.1cm}}%
		}%
	}{% not single model
		% Test if we wanna print a sentence instead of unit number
		\ifsubstring{\unit@unitsize}{SPECIAL-}{%
			\hspace*{0.25cm}\StrDel{\unit@unitsize}{SPECIAL-}%
		}{%	
			{\hspace*{0.25cm}\mincostfor{} \textbf{\unit@unitsize} \labels@models{}. \maxunitsize{}\spacebeforecolon{}: \textbf{\unit@maxmodels} \labels@models{}.\hfill \additionalfigscost{} {\largefontsize\pts{\textbf{\unit@costpermodel{}}}\permodel}\hspace*{0.1cm}}%
		}%
	}%
	}%
	\vspace*{-0.1cm}
	\noindent\begin{center}\textcolor{black!30}{\rule{\columnwidth}{1pt}}\end{center}
		\expandafter\ifblank\expandafter{\unit@invocation}{%
			\expandafter\profile\expandafter{\unit@profile}
		}{%
			\expandafter\invocprofile\expandafter{\unit@profile}
		}
	\noindent\begin{center}\textcolor{black!30}{\rule{\columnwidth}{1pt}}\end{center}
	\vspace*{-0.2cm}
	\setlength\multicolsep{0pt}
	\begin{multicols}{2}
		\raggedcolumns
		\vspace*{-0.3cm}{\setlength{\parskip}{0.3cm}
		\expandafter\ifblank\expandafter{\unit@alignment}{}{\noindent\parbox{\columnwidth}{\alignment{\unit@alignment}}}
		
		\expandafter\ifblank\expandafter{\unit@greenhiderace}{}{\noindent\parbox{\columnwidth}{\greenhideraceentry{\unit@greenhiderace}}}
		
		\expandafter\ifblank\expandafter{\unit@equipment}{}{\noindent\parbox{\columnwidth}{\equipment{\unit@equipment}}}
				
		\expandafter\ifblank\expandafter{\unit@weapons}{}{\noindent\parbox{\columnwidth}{\weapons{\unit@weapons}}}
		
		\expandafter\ifblank\expandafter{\unit@armour}{}{\noindent\parbox{\columnwidth}{\armour{\unit@armour}}}
		
		\expandafter\ifblank\expandafter{\unit@commonspecialrules}{}{\noindent\parbox{\columnwidth}{\commonspecialrules{\unit@commontype}{\unit@commonspecialrules}}}
		
		\expandafter\ifblank\expandafter{\unit@commonspecialrulesB}{}{\noindent\parbox{\columnwidth}{\commonspecialrules{\unit@commontypeB}{\unit@commonspecialrulesB}}}
		
		\expandafter\ifblank\expandafter{\unit@specialrules}{}{\noindent\parbox{\columnwidth}{\specialrules{\unit@specialrules}}}
		
		\expandafter\ifblank\expandafter{\unit@magiclevel}{}{\noindent\parbox{\columnwidth}{\magic{\unit@magiclevel}{\unit@magicpaths}}}
		
		\expandafter\ifblank\expandafter{\unit@wizardconclave}{}{\noindent\parbox{\columnwidth}{\magicwizardconclave{\unit@wizardconclave}}}
		}
		\vspace{0.1cm}
		\mounts{\unit@mounts}
		\options{\unit@options}
		\expandafter\ifblank\expandafter{\unit@commandgroup}{}{\expandafter\commandgroup\expandafter{\unit@commandgroup}}
		\unitrules{\unit@unitrules}
		\unitequipment{\unit@unitequipment}
	\end{multicols}
	\vspace*{0.1cm}\unit@additional
	\end{unitframe}
	% Database filling for auto QRS
	\expandafter\ifblank\expandafter{\unit@notinQRS}{%
	\DTLnewrow{profiles}%
	\expandafter\ifblank\expandafter{\unit@QRSname}{%
		\expandafter\profiledtbfillname\expandafter{\unit@name}%
	}{%
		\expandafter\profiledtbfillname\expandafter{\unit@QRSname}%
	}
	\expandafter\profiledtbfillcategory\expandafter{\profilecategory}%
	\expandafter\profiledtbfilltrooptype\expandafter{\unit@type}%
	\expandafter\ifblank\expandafter{\unit@invocation}{}{\expandafter\profiledtbfillinvocation\expandafter{\unit@invocation}}%
	\expandafter\profiledtbfillcarac\expandafter{\unit@profile}
	}{}%
}


%%% Changelog commands %%%

\newcommand{\newlog}[2]{%
\vspace*{0.2cm}\noindent{\antiquefont\Large\textbf{V#1}}
\parselist[,]{#2}{\locallists@changelist}%
\begin{itemize}[itemsep=0pt]%
\forlistloop{\item[-]}{\locallists@changelist}%
\end{itemize}%
}

\newcommand{\startchangelog}{\begin{multicols}{2}\vspace*{-0.2cm}}
\def\endchangelog{\end{multicols}}
