
\newcommand{\translationteam}{\item \og AEnoriel \fg \item \og Anglachel \fg \item \og Astadriel \fg \item \og Batcat \fg \item \og Eru \fg  \item \og Gandarin \fg \item \og Groumbahk \fg \item \og Iluvatar \fg \item \og Kingcheese \fg \item \og Lamronchak \fg \item \og Mammstein \fg}

%%% Commands %%%

\newcommand{\addtosortedlist}[1]{%
	\protected@edef\textarg{#1}%
	\protected@edef\textwithoutspaces{\expandafter\removespaces\expandafter{\textarg}}%
	\substitute\textwithoutspaces{É}{e}% Most used special characters of the language, and equivalent for alphabetical ordering
	\substitute\textwithoutspaces{È}{e}%
	\substitute\textwithoutspaces{Ê}{e}%
	\substitute\textwithoutspaces{é}{e}%
	\substitute\textwithoutspaces{è}{e}%
	\substitute\textwithoutspaces{ê}{e}%
	\substitute\textwithoutspaces{À}{a}%
	\substitute\textwithoutspaces{à}{a}%
	\substitute\textwithoutspaces{ù}{u}%
	\expandafter\sortitem\expandafter[\textwithoutspaces]{#1}%
}%

\newcommand{\pts}[1]{% First step is to remove spaces if there are some
	\def\numberwithoutspaces{\expandafter\removespaces\expandafter{#1}}%
	% Next step is getting rid of formatting if there are any (bold, color, ...)
	\pdfstringdef\cleannumber{\numberwithoutspaces}%
	% Now we can try if it is 1 or not
	\expandafter\ifstrequal\expandafter{\cleannumber}{1}{#1~\labels@point}{%
	\expandafter\ifstrequal\expandafter{\cleannumber}{0.5}{0,5~\labels@point}{%
	\expandafter\ifstrequal\expandafter{\cleannumber}{1.5}{1,5~\labels@point}{%
	#1~\labels@points}}}%
}

%%% Labels %%%

% Profile

\newcommand{\labels@M}{M}
\newcommand{\labels@WS}{CC}
\newcommand{\labels@BS}{CT}
\newcommand{\labels@S}{F}
\newcommand{\labels@T}{E}
\newcommand{\labels@W}{PV}
\newcommand{\labels@I}{I}
\newcommand{\labels@A}{A}
\newcommand{\labels@Ld}{Cd}
\newcommand{\labels@Invocation}{Invocation} % For Vampire Covenant profiles
\newcommand{\labels@roundbase}{rond} % printed after XX mm for round bases

\newcommand{\Strength}{Force}

% Technical

\newcommand{\labels@range}{Portée}
\newcommand{\labels@point}{pt}
\newcommand{\labels@points}{pts}
\newcommand{\labels@only}{uniquement}
\newcommand{\labels@magic}{Magie}
\newcommand{\labels@pathsused}{Génère ses sorts dans la Discipline}
\newcommand{\labels@model}{figurine}
\newcommand{\labels@models}{figurines}
\newcommand{\labels@Singlemodel}{Figurine \textbf{seule}}

% Unit entry labels

\newcommand{\labels@basesize}{Socle}
\newcommand{\labels@trooptype}{Type de troupe}
\newcommand{\labels@specialrules}{Règles spéciales}
\newcommand{\labels@alignment}{Allégeance}
\newcommand{\labels@greenhiderace}{Race de Peaux Vertes}
\newcommand{\labels@equipment}{Équipement}
\newcommand{\labels@weapons}{Armes}
\newcommand{\labels@armour}{Armure}
\newcommand{\labels@options}{Options}
\newcommand{\labels@commandgroup}{État-Major}
\newcommand{\labels@charactermounts}{Montures de Personnages}
\newcommand{\labels@mounts}{Montures}
\newcommand{\labels@mount}{Monture}
\newcommand{\labels@specialequipment}{Équipement spécial}

% Command groups

\newcommand{\labels@champion}{Champion}
\newcommand{\labels@standardbearer}{Porte-étendard}
\newcommand{\labels@musician}{Musicien}
\newcommand{\labels@singlebannerallowance}{Une seule unité de ce type peut prendre une Bannière magique}
\newcommand{\labels@condsinglebannerallowance}{Une seule unité de ce type peut prendre une Bannière magique si}
\newcommand{\labels@bannerallowance}{Bannière Magique}
\newcommand{\labels@veteranstandardbearer}{Peut devenir Porte-étendard Vétéran}
\newcommand{\labels@championallowance}{Arme Magique}

% Titles

\newcommand{\labels@lords}{Seigneurs}
\newcommand{\labels@heroes}{Héros}
\newcommand{\labels@coreunits}{Unités de base}
\newcommand{\labels@specialunits}{Unités spéciales}
\newcommand{\labels@rareunits}{Unités rares}
\newcommand{\labels@armywiderules}{Règles communes de l'armée}
\newcommand{\labels@armyspecialrules}{Règles spéciales de l'armée}
\newcommand{\labels@armoury}{Armurerie}
\newcommand{\labels@magicalitems}{Objets magiques}
\newcommand{\labels@magicalweapons}{Armes magiques}
\newcommand{\labels@magicalarmour}{Armures magiques}
\newcommand{\labels@talismans}{Talismans}
\newcommand{\labels@enchanteditems}{Objets enchantés}
\newcommand{\labels@arcaneitems}{Objets cabalistiques}
\newcommand{\labels@magicalbanners}{Bannières magiques}
\newcommand{\labels@quickrefsheet}{Fiche de référence}
\newcommand{\labels@changelog}{Change Log}

\newcommand{\labels@lordsInitial}{S}
\newcommand{\labels@heroesInitial}{H}
\newcommand{\labels@coreunitsInitial}{B}
\newcommand{\labels@specialunitsInitial}{S}
\newcommand{\labels@rareunitsInitial}{R}
\newcommand{\labels@mountsInitial}{M}


% Titlepage

\newcommand{\labels@fantasybattles}{Batailles Fantastiques}
\newcommand{\labels@NinthAge}{Le 9\ieme Âge}
\newcommand{\labels@armyrules}{Règles de l'Armée}
\newcommand{\labels@frontpagecredits}{%
\labels@fantasybattles{} : \labels@NinthAge{} est un jeu créé et entretenu par la communauté qui met en scène des affrontements de figurines.\newline
Toutes les règles sont disponibles gratuitement sur le site suivant. Vos retours et suggestions sont les bienvenus : \url{http://www.the-ninth-age.com/}
}
\newcommand{\labels@license}{Copyright Creative Commons license : \url{the-ninth-age.com/license.html}}
\newcommand{\labels@introduction}{%
\noindent {\Largerfontsize\textbf{Note des traducteurs}}
\vspace{0.5cm}

Nous souhaitons remercier chaleureusement l'équipe à l'initiative du 9\ieme Âge pour leur motivation et leur travail continu pour faire vivre notre passion. Nous espérons que ce jeu saura développer les qualités pour plaire au plus grand nombre et réunir les joueurs, amateurs comme habitués des tournois, autour de règles amusantes et équilibrées, pour finalement s'imposer comme un standard du jeu de figurines. Une grande ambition qui ne pourra s'accomplir que \textbf{grâce à vous}, la communauté, via des retours constructifs, afin de modeler le jeu selon nos désirs. N'étant \textbf{en aucun cas à but lucratif}, le 9\ieme Âge part avec un avantage considérable. Les règles des éventuelles nouvelles sorties ne sont pas dictées par le besoin de vendre ces nouveautés. Vous pouvez choisir et acheter vos figurines où bon vous semble, il n'y a pas un unique revendeur toléré. Enfin, vous pouvez être assurés que tant que le 9\ieme Âge sera joué, vous disposerez d'un \textbf{support continu et régulier}, celui-ci étant offert par la communauté.

Nous attirons votre attention sur le fait que ce jeu arrive à la fin de son \textbf{stade de développement}. Ce document correspond à une version de brouillon \textbf{\og{} beta \fg{}} finale, dont le but est de repérer toute erreur qui a pu nous échapper. N'hésitez pas à nous aider à les trouver ! Ce livre d'armée n'est utilisable qu'en compagnie du livre de Règles et du livre de Magie.

Concernant la traduction en elle-même, nous avons fait de notre mieux pour vous offrir une version de qualité, dont nous espérons qu'elle surpasse celle de la version originale ! Si vous constatez des coquilles, des erreurs, merci de nous les signaler en nous contactant sur le forum du 9\ieme Âge, dans le \textbf{sous-forum français} (\url{http://www.the-ninth-age.com/index.php?board/117-french/}). Vous y trouverez aussi les dernières mises à jour. \textbf{En cas de conflit d'interprétation avec la version originale, la version originale fait référence}.

\vspace{0.5cm}
Que ce jeu vous apporte d'innombrables heures de plaisir partagé !

\vspace{0.7cm}
\noindent {\Largerfontsize\textbf{Les traducteurs}}
\vspace{0.1cm}

\ifdef{\translationteam}{
	\begin{multicols}{3}
	\begin{itemize}
		\translationteam
	\end{itemize}
	\end{multicols}
}{}
}
\newcommand{\labels@rulechanges}{% blank ATM
}
\newcommand{\labels@latexcredit}{Document réalisé à l'aide de \LaTeX .}


%%% Technical commands

\newcommand{\only}[1]{(#1 uniquement)}
\newcommand{\free}{gratuit}
\newcommand{\upto}{jusqu'à}
\newcommand{\Upto}{Jusqu'à}
\newcommand{\unlimited}{pas de limite}
\newcommand{\permodel}{/fig.}
\newcommand{\listlastchoice}{ ou}
\newcommand{\notif}[1]{(pas #1)}
\newcommand{\wordand}{et}
\newcommand{\wordwith}{avec}
\newcommand{\ifNmodelsorless}[1]{(#1 figurines ou moins)}
\newcommand{\unitwith}{unité avec}
\newcommand{\From}{De} % From ... to ... models
\newcommand{\wordto}{à}
\newcommand{\wordAll}{Tous}
\newcommand{\spacebeforecolon}{ } % French put a space before colons
\newcommand{\minprice}{Coût min. :}
\newcommand{\mincostfor}{Coût min. pour}
\newcommand{\maxunitsize}{Taille max.}
\newcommand{\additionalfigscost}{Les figurines additionnelles coûtent}


%%% Special rules %%%

\newcommand{\ambush}{Embuscade}
\newcommand{\armourpiercing}[1]{Perforant\ifblank{#1}{}{ (#1)}}
\newcommand{\bodyguard}[1]{Garde du Corps\ifblank{#1}{}{ (#1)}}
\newcommand{\breathweapon}[1]{Attaque de Souffle\ifblank{#1}{}{ (#1)}}
\newcommand{\channel}{Canalisation}
\newcommand{\crushattack}{Attaque Écrasante}
\newcommand{\devastatingcharge}{Charge Dévastatrice}
\newcommand{\distracting}{Distrayant}
\newcommand{\divineattacks}{Attaques Divines}
\newcommand{\engineer}{Ingénieur}
\newcommand{\ethereal}{Éthéré}
\newcommand{\fastcavalry}{Cavalerie Légère}
\newcommand{\fear}{Peur}
\newcommand{\fightinextrarank}{Combat avec un Rang Supplémentaire}
\newcommand{\fireborn}{Né du Feu}
\newcommand{\flamingattacks}{Attaques Enflammées}
\newcommand{\flammable}{Inflammable}
\newcommand{\lighttroops}{Troupes Légères}
\newcommand{\frenzy}{Frénésie}
\newcommand{\fly}[1]{Vol\ifblank{#1}{}{ (#1)}}
\newcommand{\grindingattacks}[1]{Attaques de Broyage\ifblank{#1}{}{ (#1)}}
\newcommand{\hardtarget}{Camouflé}
\newcommand{\hatred}{Haine}
\newcommand{\hellfire}{Flammes de l'Enfer}
\newcommand{\hidden}{Caché}
\newcommand{\holyattacks}{Attaques Divines} % deprecated, still has to be filled. same as Divine Attacks.
\newcommand{\immunetopsychology}{Immunisé à la Psychologie}
\newcommand{\impacthits}[1]{Touches d'Impact\ifblank{#1}{}{ (#1)}}
\newcommand{\insignificant}{Insignifiant}
\newcommand{\largetarget}{Grande Cible}
\newcommand{\lethalstrike}{Coup Fatal}
\newcommand{\lightningattacks}{Attaques Foudroyantes}
\newcommand{\lightningreflexes}{Réflexes Foudroyants}
\newcommand{\magicresistance}[1]{Résistance à la Magie\ifblank{#1}{}{ (#1)}}
\newcommand{\magicalattacks}{Attaques Magiques}
\newcommand{\metalshifting}{Fusion du Métal}
\newcommand{\moveorfire}{Mouvement ou Tir}
\newcommand{\multipleshots}[1]{Tirs Multiples\ifblank{#1}{}{ (#1)}}
\newcommand{\multiplewounds}[2]{Blessures Multiples\ifblank{#1}{}{ (#1\ifblank{#2}{)}{, #2)}}}
\newcommand{\notaleader}{Pas un Meneur}
\newcommand{\otherworldly}{D'Outre-Monde}
\newcommand{\pathmaster}[1]{Maître de la Discipline\ifblank{#1}{}{ (#1)}}
\newcommand{\poisonedattacks}{Attaques Empoisonnées}
\newcommand{\quicktofire}{Tir Rapide}
\newcommand{\randommovement}[1]{Mouvement Aléatoire\ifblank{#1}{}{ (#1)}}
\newcommand{\randomattacks}[1]{Attaques Aléatoires\ifblank{#1}{}{ (#1)}}
\newcommand{\regeneration}[1]{Régénération\ifblank{#1}{}{ (#1+)}}
\newcommand{\reload}{Rechargez !}
\newcommand{\requirestwohands}{Arme à deux Mains}
\newcommand{\scythes}{Faux}
\newcommand{\scout}{Éclaireur}
\newcommand{\scouts}{Éclaireurs}
\newcommand{\stomp}[1]{Piétinement\ifblank{#1}{}{ (#1)}}
\newcommand{\strider}[1]{Guide\ifblank{#1}{}{ (#1)}}
\newcommand{\stubborn}{Tenace}
\newcommand{\stupidity}{Stupidité}
\newcommand{\skirmisher}{Tirailleur}
\newcommand{\skirmishers}{Tirailleurs}
\newcommand{\sweepingattack}{Attaque au Passage}
\newcommand{\swiftstride}{Rapide}
\newcommand{\thunderouscharge}{Charge Tonitruante}
\newcommand{\terror}{Terreur}
\newcommand{\toxicattacks}{Attaques Toxiques}
\newcommand{\unbreakable}{Indémoralisable}
\newcommand{\undead}{Mort-Vivant}
\newcommand{\unstable}{Instable}
\newcommand{\unwieldy}{Encombrant}
\newcommand{\vanguard}{Avant-Garde}
\newcommand{\volleyfire}{Tir de Volée}
\newcommand{\warplatform}{Plateforme de Guerre}
\newcommand{\wardsave}[1]{Sauvegarde Invulnérable\ifblank{#1}{}{ (#1+)}}
\newcommand{\weaponmaster}{Maître d'Ar\-mes}
\newcommand{\wizardconclave}[1]{Conclave de Sorciers\ifblank{#1}{}{ (#1)}}


%%% Magic %%%

\newnamemacro{\Pathof}{Discipline}

\newcommand{\battle}{Commune}
\newcommand{\alchemy}{de l'Alchimie}
\newcommand{\death}{de la Mort}
\newcommand{\fire}{du Feu}
\newcommand{\heavens}{des Cieux}
\newcommand{\light}{de la Lumière}
\newcommand{\nature}{de la Nature}
\newcommand{\shadows}{des Ombres}
\newcommand{\wilderness}{de la Sauvagerie Bestiale}
\newcommand{\butchery}{de la Boucherie}
\newcommand{\change}{du Changement}
\newcommand{\thebiggreengods}{des Grands Dieux Verts}
\newcommand{\thelittlegreengods}{des Petits Dieux Verts}
\newcommand{\blackmagic}{de la Magie Noire}
\newcommand{\disease}{de la Maladie}
\newcommand{\lust}{de la Luxure}
\newcommand{\necromancy}{de la Nécromancie}
\newcommand{\ruin}{de la Ruine}
\newcommand{\forge}{de la Forge}
\newcommand{\sands}{des Sables}
\newcommand{\whitemagic}{de la Magie Blanche}

\newcommand{\anyofthebattlemagic}{dans n'importe laquelle des Disciplines Communes}

\newcommand{\magiclevel}[1]{\ifnumcomp{#1}{<}{3}{Sorcier Apprenti}{Maître Sorcier} Niveau #1}
\newcommand{\Level}{Niveau}

\newcommand{\wizard}{Sorcier}
\newcommand{\wizards}{Sorciers}

\newcommand{\boundspell}[1]{Objet de Sort, Puissance #1}


%%% Other rules %%%

\newcommand{\armoursave}{Sauvegarde d'Armure}
\newcommand{\firstinrank}{Au Premier Rang}
\newcommand{\hardcover}{Couvert Lourd}
\newcommand{\holdyourground}{Tenez les Rangs}
\newcommand{\inspiringpresence}{Présence Charismatique}
\newcommand{\lightcover}{Couvert Léger}
\newcommand{\monstrousrank}{Rang Monstrueux}
\newcommand{\ordnance}{Artillerie}
\newcommand{\parry}{Parade}
\newcommand{\raisewounds}{Ressusciter des Figurines}
\newcommand{\recoverwounds}{Récupérer des PVs}
\newcommand{\aideddispel}{Dissipation Assistée}
\newcommand{\rnf}{ordinaires}
\newcommand{\general}{Général}
\newcommand{\bsb}{Porteur de la Grande Bannière}
\newcommand{\cannotmarch}{Pas de Marche Forcée}


%%% Equipment %%%

\newcommand{\hw}{Arme de Base}
\newcommand{\pw}{Paire d'Armes}
\newcommand{\spear}{Lance}
\newcommand{\halberd}{Hallebarde}
\newcommand{\gw}{Arme Lourde}
\newcommand{\lance}{Lance de Cavalerie}
\newcommand{\lightlance}{Lance Légère}
\newcommand{\flail}{Fléau}

\newcommand{\throwingweapons}{Armes de Jet}
\newcommand{\shortbow}{Arc Court}
\newcommand{\bow}{Arc}
\newcommand{\longbow}{Arc Long}
\newcommand{\handgun}{Arquebuse}
\newcommand{\crossbow}{Arbalète}
\newcommand{\pistol}{Pistolet}
\newcommand{\braceofpistols}{Paire de Pistolets}	

\newcommand{\innatedefence}[1]{Protection Innée\ifblank{#1}{}{~(#1+)}}
\newcommand{\mountsprotection}[1]{Protection de Monture\ifblank{#1}{}{~(#1+)}}
\newcommand{\la}{Armure Légère}
\newcommand{\ha}{Armure Lourde}
\newcommand{\platearmour}{Armure de Plates}
\newcommand{\shield}{Bouclier}
\newcommand{\barding}{Caparaçon}

\newcommand{\cannon}{Canon}
\newcommand{\cannons}{Canons}
\newcommand{\catapult}{Catapulte}
\newcommand{\catapults}{Catapultes}
\newcommand{\volleygun}{Batterie de Tir}
\newcommand{\boltthrower}{Baliste}
\newcommand{\flamethrower}{Canon à Flammes}
\newcommand{\artilleryweapon}{Arme d'Artillerie}


%%% Troop types %%%

\newcommand{\characters}{Personnages}
\newcommand{\infantry}{Infanterie}
\newcommand{\monstrousinfantry}{Infanterie Monstrueuse}
\newcommand{\cavalry}{Cavalerie}
\newcommand{\monstrouscavalry}{Cavalerie Monstrueuse}
\newcommand{\swarm}{Nuée}
\newcommand{\swarms}{Nuées}
\newcommand{\warbeast}{Bête de Guerre}
\newcommand{\warbeasts}{Bêtes de Guerre}
\newcommand{\monster}{Monstre}
\newcommand{\monsters}{Monstres}
\newcommand{\monstrousbeast}{Bête Monstrueuse}
\newcommand{\monstrousbeasts}{Bêtes Monstrueuses}
\newcommand{\chariot}{Char}
\newcommand{\chariots}{Chars}
\newcommand{\riddenmonster}{Monstre Monté}
\newcommand{\riddenmonsters}{Monstres Montés}
\newcommand{\warmachine}{Machine de Guerre}
\newcommand{\warmachines}{Machines de Guerre}


%%% Terrain %%%

\newcommand{\water}{Eaux peu profondes}
\newcommand{\forest}{Forêt}
\newcommand{\impassableterrain}{Terrain Infranchissable}


%%% Profile wording

\newcommand{\oneperarmy}{Un par Armée}
\newcommand{\oneofakind}{Uni\-que}
\newcommand{\zerotoXchoice}[1]{0-#1 Choix}
\newcommand{\onechoiceonlyNOC}{(un seul choix)}
\newcommand{\onfootonly}{(à pied seulement)}
\newcommand{\closecombatonly}{seulement au Corps à Corps}
\newcommand{\Xmodelsorless}[1]{(#1 figurines ou moins)}
\newcommand{\magicalitemsallowance}{Objets Magiques}
\newcommand{\magicalweaponallowance}{Arme Magique}
\newcommand{\notmagicalarmour}{(pas d'Armure Magique)}
\newcommand{\weapononechoice}{\optionschoice{Arme \onechoiceonlyNOC{} :}}
\newcommand{\weaponschoice}{\optionschoice{Armes :}}
\newcommand{\shootingweapononechoice}{\optionschoice{Arme de Tir \onechoiceonlyNOC{} :}}
\newcommand{\combatweapononechoice}{\optionschoice{Arme de Corps à Corps \onechoiceonlyNOC{} :}}
\newcommand{\combatweapononechoiceTWOCOL}{\optionschoiceTWOCOL{Arme de Corps à Corps \onechoiceonlyNOC{} :}}
\newcommand{\armouronechoice}{\optionschoice{Armure \onechoiceonlyNOC{} :}}
\newcommand{\magiclevelchoice}{\optionschoice{Magie \onechoiceonlyNOC{} :}}
\newcommand{\mustbecomeoneofthefollowing}{\optionschoice{\textbf{Doit} devenir au choix :}}
\newcommand{\mustbecomeoneofthefollowingNOC}{Doit devenir au choix :}
\newcommand{\musttakeoneormoreofthefollowing}{\optionschoice{\textbf{Doit} prendre au moins un choix :}}
\newcommand{\musttakeoneofthefollowingNOC}{Doit choisir entre :}
\newcommand{\uptotwoofthefollowing}{\optionschoice{Jusqu'à deux choix :}}
\newcommand{\uptotwoofthefollowingTWOCOL}{\optionschoiceTWOCOL{Jusqu'à deux choix :}}

\newcommand{\onechoiceonly}{\optionschoice{Un seul choix :}}
\newcommand{\onechoiceonlyTWOCOL}{\optionschoiceTWOCOL{Un seul choix :}}

\newcommand{\maytake}{Peut prendre}




%%% Orcs N Goblins debug, let it as it is

\newcommand{\pershadygit}{debug}
\newcommand{\permadgit}{debug}

%%% Dwarven Holds debug, let it as it is

\newcommand{\perrune}{debug}
