%%% Labels %%%

% Profile

\newcommand{\labels@M}{M}
\newcommand{\labels@WS}{CC}
\newcommand{\labels@BS}{CT}
\newcommand{\labels@S}{F}
\newcommand{\labels@T}{E}
\newcommand{\labels@W}{PV}
\newcommand{\labels@I}{I}
\newcommand{\labels@A}{A}
\newcommand{\labels@Ld}{Cd}
\newcommand{\labels@Invocation}{Invocation} % For Vampire Covenant profiles

\newcommand{\Strength}{Force}

% Technical

\newcommand{\labels@range}{Portée}
\newcommand{\labels@point}{pt}
\newcommand{\labels@points}{pts}
\newcommand{\labels@only}{uniquement}
\newcommand{\labels@magic}{Magie}
\newcommand{\labels@pathsused}{Génère ses sorts dans la Discipline}
\newcommand{\labels@model}{figurine}
\newcommand{\labels@models}{figurines}
\newcommand{\labels@Singlemodel}{Figurine \textbf{seule}}

% Unit entry labels

\newcommand{\labels@basesize}{Socle}
\newcommand{\labels@trooptype}{Type de troupe}
\newcommand{\labels@specialrules}{Règles spéciales}
\newcommand{\labels@equipment}{Équipement}
\newcommand{\labels@weapons}{Armes}
\newcommand{\labels@armour}{Armure}
\newcommand{\labels@options}{Options}
\newcommand{\labels@commandgroup}{État-Major}
\newcommand{\labels@mounts}{Montures}
\newcommand{\labels@specialequipment}{Équipement spécial}

% Command groups

\newcommand{\labels@champion}{Champion}
\newcommand{\labels@standardbearer}{Porte-étendard}
\newcommand{\labels@musician}{Musicien}
\newcommand{\labels@singlebannerallowance}{Une seule unité de ce type peut prendre une Bannière magique}
\newcommand{\labels@condsinglebannerallowance}{Une seule unité de ce type peut prendre une Bannière magique si}
\newcommand{\labels@bannerallowance}{Peut prendre une Bannière magique}
\newcommand{\labels@veteranstandardbearer}{Peut devenir Porte-étendard Vétéran}
\newcommand{\labels@championallowance}{Peut prendre des objets magiques}

% Titles

\newcommand{\labels@lords}{Seigneurs}
\newcommand{\labels@heroes}{Héros}
\newcommand{\labels@coreunits}{Unités de base}
\newcommand{\labels@specialunits}{Unités spéciales}
\newcommand{\labels@rareunits}{Unités rares}
\newcommand{\labels@armyspecialrules}{Règles spéciales de l'armée}
\newcommand{\labels@armoury}{Armurerie}
\newcommand{\labels@magicalitems}{Objets magiques}
\newcommand{\labels@magicalweapons}{Armes magiques}
\newcommand{\labels@magicalarmour}{Armures magiques}
\newcommand{\labels@talismans}{Talismans}
\newcommand{\labels@enchanteditems}{Objets enchantés}
\newcommand{\labels@arcaneitems}{Objets cabalistiques}
\newcommand{\labels@magicalbanners}{Bannières magiques}
\newcommand{\labels@quickrefsheet}{Fiche de référence}
\newcommand{\labels@changelog}{Change Log}

\newcommand{\labels@lordsInitial}{S}
\newcommand{\labels@heroesInitial}{H}
\newcommand{\labels@coreunitsInitial}{B}
\newcommand{\labels@specialunitsInitial}{S}
\newcommand{\labels@rareunitsInitial}{R}
\newcommand{\labels@mountsInitial}{M}


% Titlepage

\newcommand{\labels@fantasybattles}{Batailles Fantastiques}
\newcommand{\labels@NinthAge}{Le 9\ieme Âge}
\newcommand{\labels@creators}{Une collaboration des créateurs de l'ETC et du Swedish Comp System}
\newcommand{\labels@introduction}{%
\noindent {\fontsize{16}{19.2}\selectfont\textbf{Note des traducteurs}}
\vspace{0.5cm}

Nous souhaitons remercier chaleureusement l'équipe à l'initiative du 9\ieme Âge pour leur motivation et leur travail continu pour faire vivre notre passion. Nous espérons que ce jeu saura développer les qualités pour plaire au plus grand nombre et réunir les joueurs, amateurs comme habitués des tournois, autour de règles amusantes et équilibrées, pour finalement s'imposer comme un standard du jeu de figurines. Une grande ambition qui ne pourra s'accomplir que \textbf{grâce à vous}, la communauté, via des retours constructifs, afin de modeler le jeu selon nos désirs. N'étant \textbf{en aucun cas à but lucratif}, le 9\ieme Âge part avec un avantage considérable. Les règles des éventuelles nouvelles sorties ne seront pas dictées par le besoin de vendre ces nouveautés. Vous pouvez choisir et acheter vos figurines où bon vous semble, il n'y a pas un unique revendeur toléré. Vous n'êtes pas bloqués dans une spirale infernale où pour continuer à jouer à un jeu, dans lequel vous vous êtes tant investis, vous devez payer toujours plus cher pour entretenir votre collection. Enfin, vous pouvez être assurés que tant que 9\ieme Âge sera joué, vous disposerez d'un \textbf{support continu et régulier}, celui-ci étant offert par la communauté.

Nous attirons votre attention sur le fait que ce jeu en est encore à ses débuts et dans un \textbf{stade de développement}. Ce document correspond à une version de brouillon \textbf{\og{} beta \fg{}}, dont le but et de tester le jeu et le modifier jusqu'à atteindre une version satisfaisante. Attendez-vous donc à trouver des déséquilibres, des incohérences, et à obtenir des mises à jour régulières avec éventuellement des changements importants. N'hésitez pas à nous donner vos avis ! Ce livre d'armée n'est utilisable qu'en compagnie du livre de Règles et du livre de Magie.

Concernant la traduction en elle-même, nous avons fait de notre mieux pour vous offrir une version de qualité, dont nous espérons qu'elle surpasse celle de la version originale ! Si vous constatez des coquilles, des erreurs, merci de nous les signaler en nous contactant sur le forum du 9\ieme Âge, dans le \textbf{sous-forum français} (\url{http://www.the-ninth-age.com/index.php?board/117-french/}). Vous y trouverez aussi les dernières mises à jour. \textbf{En cas de conflit d'interprétation avec la version originale, la version originale fait référence}.

\vspace{0.5cm}
Que ce jeu vous apporte d'innombrables heures de plaisir partagé !

\vspace{0.7cm}
\noindent {\fontsize{16}{19.2}\selectfont \textbf{Les traducteurs}}
\vspace{0.1cm}

\ifdef{\translationteam}{
	\begin{multicols}{2}
	\begin{itemize}
		\translationteam
	\end{itemize}
	\end{multicols}
}{}
}
\newcommand{\labels@secondpageannouncement}{%
	\labels@fantasybattles{} : \labels@NinthAge{} est un jeu créé et entretenu par la communauté qui met en scène des affrontements de figurines. Toutes les règles ainsi que les retours et suggestions peuvent être trouvées ou donnés sur le site :
	\newline\url{http://www.the-ninth-age.com/}
}
\newcommand{\labels@rulechanges}{%
	Les changements de règles entre versions sont colorés comme ce paragraphe. Une liste en anglais de ces changements par version est ajoutée à la fin de cet ouvrage.
}
\newcommand{\labels@latexcredit}{Document réalisé à l'aide de \LaTeX .}


%%% Technical commands

\newcommand{\free}{gratuit}
\newcommand{\upto}{jusqu'à}
\newcommand{\Upto}{Jusqu'à}
\newcommand{\unlimited}{sans limite de pts}
\newcommand{\permodel}{/fig.}
\newcommand{\listlastchoice}{, ou}
\newcommand{\notif}[1]{(pas #1)}
\newcommand{\wordand}{et}
\newcommand{\wordwith}{avec}
\newcommand{\ifNmodelsorless}[1]{(#1 figurines ou moins)}
\newcommand{\unitwith}{unité avec}
\newcommand{\From}{De} % From ... to ... models
\newcommand{\wordto}{à}
\newcommand{\wordAll}{Tous}
\newcommand{\spacebeforecolon}{ } % French put a space before colons
\newcommand{\pricepermodelabovemin}{Prix par fig. au delà de la taille minimale :}
\newcommand{\minprice}{Prix min. :}


%%% Special rules %%%

\newnamemacro{\ambush}{Embuscade}
\newcommand{\armourpiercing}[1]{Perforant\ifblank{#1}{}{ (#1)}}
\newcommand{\bodyguard}[1]{Garde du Corps\ifblank{#1}{}{ (#1)}}
\newcommand{\breathweapon}[1]{Attaque de Souffle\ifblank{#1}{}{ (#1)}}
\newnamemacro{\channel}{Canalisation}
\newnamemacro{\crushattack}{Attaque Écrasante}
\newnamemacro{\cumbersome}{Difficile à Manier}
\newnamemacro{\devastatingcharge}{Charge Dévastatrice}
\newnamemacro{\distracting}{Distrayant}
\newnamemacro{\engineer}{Ingénieur}
\newnamemacro{\ethereal}{Éthéré}
\newnamemacro{\fastcavalry}{Cavalerie Légère}
\newnamemacro{\fear}{Peur}
\newnamemacro{\fightinextrarank}{Combat avec un Rang Supplémentaire}
\newnamemacro{\fireborn}{Né du Feu}
\newnamemacro{\flamingattacks}{Attaques Enflammées}
\newnamemacro{\flammable}{Inflammable}
\newnamemacro{\lighttroops}{Troupes Légères}
\newnamemacro{\frenzy}{Frénésie}
\newcommand{\fly}[1]{Vol\ifblank{#1}{}{ (#1)}}
\newcommand{\grindingattacks}[1]{Attaques de Broyage\ifblank{#1}{}{ (#1)}}
\newnamemacro{\hardtarget}{Camouflé}
\newnamemacro{\hatred}{Haine}
\newnamemacro{\hellfire}{Flammes de l'Enfer}
\newnamemacro{\hidden}{Caché}
\newnamemacro{\holyattacks}{Attaques Divines}
\newnamemacro{\immunetopsychology}{Immunisé à la Psychologie}
\newcommand{\impacthits}[1]{Touches d'Impact\ifblank{#1}{}{ (#1)}}
\newnamemacro{\insignificant}{Insignifiant}
\newnamemacro{\largetarget}{Grande Cible}
\newnamemacro{\lethalstrike}{Coup Fatal}
\newnamemacro{\lightningattacks}{Attaques Foudroyantes}
\newnamemacro{\lightningreflexes}{Réflexes Foudroyants}
\newcommand{\magicresistance}[1]{Résistance à la Magie\ifblank{#1}{}{ (#1)}}
\newnamemacro{\magicalattacks}{Attaques Magiques}
\newnamemacro{\metalshifting}{Fusion du Métal}
\newnamemacro{\moveorfire}{Mouvement ou Tir}
\newcommand{\multipleshots}[1]{Tirs Multiples\ifblank{#1}{}{ (#1)}}
\newcommand{\multiplewounds}[2]{Blessures Multiples\ifblank{#1}{}{ (#1\ifblank{#2}{)}{, #2)}}}
\newnamemacro{\notaleader}{Pas un Meneur}
\newnamemacro{\otherworldly}{D'Outre-Monde}
\newcommand{\pathmaster}[1]{Maître de la Discipline\ifblank{#1}{}{ (#1)}}
\newnamemacro{\poisonedattacks}{Attaques Empoisonnées}
\newnamemacro{\quicktofire}{Tir Rapide}
\newcommand{\randommovement}[1]{Mouvement Aléatoire\ifblank{#1}{}{ (#1)}}
\newcommand{\randomattacks}[1]{Attaques Aléatoires\ifblank{#1}{}{ (#1)}}
\newcommand{\regeneration}[1]{Régénération\ifblank{#1}{}{ (#1+)}}
\newnamemacro{\requirestwohands}{Arme à deux Mains}
\newnamemacro{\scythes}{Faux}
\newnamemacro{\scout}{Éclaireur}
\newnamemacro{\scouts}{Éclaireurs}
\newcommand{\stomp}[1]{Piétinement\ifblank{#1}{}{ (#1)}}
\newcommand{\strider}[1]{Guide\ifblank{#1}{}{ (#1)}}
\newnamemacro{\stubborn}{Tenace}
\newnamemacro{\stupidity}{Stupidité}
\newnamemacro{\skirmisher}{Tirailleur}
\newnamemacro{\skirmishers}{Tirailleurs}
\newcommand{\sweepingattack}{Attaque au Passage}
\newnamemacro{\swiftstride}{Rapide}
\newnamemacro{\thunderouscharge}{Charge Tonitruante}
\newnamemacro{\terror}{Terreur}
\newnamemacro{\toxicattacks}{Attaques Toxiques}
\newnamemacro{\unbreakable}{Indémoralisable}
\newnamemacro{\undead}{Mort-Vivant}
\newnamemacro{\unstable}{Instable}
\newnamemacro{\unwieldy}{Encombrant}
\newnamemacro{\vanguard}{Avant-Garde}
\newnamemacro{\volleyfire}{Tir de Volée}
\newnamemacro{\warplatform}{Plateforme de Guerre}
\newcommand{\wardsave}[1]{Sauvegarde Invulnérable\ifblank{#1}{}{ (#1+)}}
\newnamemacro{\weaponmaster}{Maître d'Ar\-mes}
\newcommand{\wizardconclave}[1]{Conclave de Sorciers\ifblank{#1}{}{ (#1)}}


%%% Magic %%%

\newnamemacro\Pathof{Discipline}

\newnamemacro\battle{de Bataille}
\newnamemacro\alchemy{de l'Alchimie}
\newnamemacro\death{de la Mort}
\newnamemacro\fire{du Feu}
\newnamemacro\heavens{des Cieux}
\newnamemacro\light{de la Lumière}
\newnamemacro\nature{de la Nature}
\newnamemacro\shadows{des Ombres}
\newnamemacro\wilderness{de la Sauvagerie Bestiale}
\newnamemacro\butchery{de la Boucherie}
\newnamemacro\change{du Changement}
\newnamemacro\thebiggreengods{des Grands Dieux Verts}
\newnamemacro\thelittlegreengods{des Petits Dieux Verts}
\newnamemacro\blackmagic{de la Magie Noire}
\newnamemacro\disease{de la Maladie}
\newnamemacro\lust{de la Luxure}
\newnamemacro\necromancy{de la Nécromancie}
\newnamemacro\ruin{de la Ruine}
\newnamemacro\forge{de la Forge}
\newnamemacro\sands{des Sables}
\newnamemacro\whitemagic{de la Magie Blanche}

\newcommand{\magiclevel}[1]{\ifnumcomp{#1}{<}{3}{Apprenti}{Maître} Sorcier Niveau #1}
\newcommand{\Level}{Niveau}

\newnamemacro{\wizard}{Sorcier}
\newnamemacro{\wizards}{Sorciers}

\newcommand{\boundspell}[1]{Objet de Sort, Puissance #1}


%%% Other rules %%%

\newnamemacro{\armoursave}{Sauvegarde d'Armure}
\newnamemacro{\firstinrank}{Au Premier Rang}
\newnamemacro{\hardcover}{Couvert Lourd}
\newnamemacro{\holdyourground}{Tenez les Rangs}
\newnamemacro{\inspiringpresence}{Présence Charismatique}
\newnamemacro{\lightcover}{Couvert Léger}
\newnamemacro{\monstrousrank}{Rang Monstrueux}
\newnamemacro{\ordnance}{Artillerie}
\newnamemacro{\parry}{Parade}
\newnamemacro{\raisewounds}{Ressusciter des Figurines}
\newnamemacro{\recoverwounds}{Récupérer des PVs}
\newnamemacro{\aideddispel}{Dissipation Assistée}
\newnamemacro{\rnf}{ordinaires}
\newnamemacro{\general}{Général}


%%% Equipment %%%

\newcommand{\innatedefence}[1]{Protection Innée\ifblank{#1}{}{~(#1+)}}
\newcommand{\mountsprotection}[1]{Protection de Monture\ifblank{#1}{}{~(#1+)}}
\newnamemacro{\la}{Armure Légère}
\newnamemacro{\ha}{Armure Lourde}
\newnamemacro{\platearmour}{Armure de Plates}
\newnamemacro{\hw}{Arme de Base}
\newnamemacro{\pw}{Paire d'Armes}
\newnamemacro{\spear}{Lance}
\newnamemacro{\halberd}{Hallebarde}
\newnamemacro{\gw}{Arme Lourde}
\newnamemacro{\lance}{Lance de Cavalerie}
\newnamemacro{\lightlance}{Lance Légère}
\newnamemacro{\shield}{Bouclier}
\newnamemacro{\barding}{Caparaçon}
\newnamemacro{\throwingweapons}{Armes de Jet}
\newnamemacro{\shortbow}{Arc Court}

\newnamemacro{\cannon}{Canon}
\newnamemacro{\catapult}{Catapulte}
\newnamemacro{\volleygun}{Batterie de Tir}
\newnamemacro{\boltthrower}{Baliste}
\newnamemacro{\artilleryweapon}{Arme d'Artillerie}


%%% Troop types %%%

\newnamemacro{\characters}{Personnages}
\newnamemacro{\infantry}{Infanterie}
\newnamemacro{\monstrousinfantry}{Infanterie Monstrueuse}
\newnamemacro{\cavalry}{Cavalerie}
\newnamemacro{\monstrouscavalry}{Cavalerie Monstrueuse}
\newnamemacro{\swarm}{Nuée}
\newnamemacro{\swarms}{Nuées}
\newnamemacro{\warbeast}{Bête de Guerre}
\newnamemacro{\warbeasts}{Bêtes de Guerre}
\newnamemacro{\monster}{Monstre}
\newnamemacro{\monsters}{Monstres}
\newnamemacro{\monstrousbeast}{Bête Monstrueuse}
\newnamemacro{\monstrousbeasts}{Bêtes Monstrueuses}
\newnamemacro{\chariot}{Char}
\newnamemacro{\chariots}{Chars}
\newnamemacro{\riddenmonster}{Monstre Monté}
\newnamemacro{\riddenmonsters}{Monstres Montés}
\newnamemacro{\warmachine}{Machine de Guerre}
\newnamemacro{\warmachines}{Machines de Guerre}


%%% Profile wording

\newnamemacro{\oneofakind}{(Uni\-que)}
\newcommand{\onechoiceonly}{(un seul choix)}
\newcommand{\onfootonly}{(à pied seulement)}
\newcommand{\Xmodelsorless}[1]{(#1 figurines ou moins)}
\newcommand{\magicalitemsallowance}{Peut prendre des Objets Magiques}
\newcommand{\magicalweaponallowance}{Peut prendre une Arme Magique}
\newcommand{\anyofthefollowing}{\optionschoice{Peut prendre :}}
\newcommand{\weapononechoice}{\optionschoice{Peut prendre une arme \onechoiceonly{} :}}
\newcommand{\weaponschoice}{\optionschoice{Peut prendre des armes :}}
\newcommand{\shootingweapononechoice}{\optionschoice{Peut prendre une arme de tir \onechoiceonly{} :}}
\newcommand{\combatweapononechoice}{\optionschoice{Peut prendre une arme de corps à corps \onechoiceonly{} :}}
\newcommand{\armouronechoice}{\optionschoice{Peut prendre une armure \onechoiceonly{} :}}
\newcommand{\magiclevelchoice}{\optionschoice{Peut devenir, au choix :}}
\newcommand{\bsboption}{Peut devenir Porteur de la Grande Bannière}
\newcommand{\mayupgradeto}{Peut être amélioré en}
\newcommand{\mustbecomeoneofthefollowing}{\optionschoice{Doit devenir un choix parmi :}}
\newcommand{\maytakeoneofthefollowing}{\optionschoice{Peut prendre un choix parmi :}}
\newcommand{\maytakeuptotwoofthefollowing}{\optionschoice{Peut prendre jusqu'à deux choix parmi :}}
\newcommand{\maygain}{Peut gagner la règle}
\newcommand{\maytakeashield}{Peut prendre un Bouclier}
\newcommand{\maytakela}{Peut prendre une Armure Légère}
\newcommand{\maytakeha}{Peut prendre une Armure Lourde}
\newcommand{\maytakemountsprotectionX}[1]{Peut prendre une \mountsprotection{#1}}
\newcommand{\maytakeagw}{Peut prendre une Arme Lourde}
\newcommand{\replaceshieldwithhalberd}{Remplacer le Bouclier par une Hallebarde}

\newcommand{\veteranstandardbearernote}{%
Un Porte-étendard Vétéran est Unique (un seul Vétéran dans toute l'armée) et peut choisir une Bannière Magique dans un budget de 25 pts.
}

\newcommand{\mountssectionannouncement}{%
La section Montures concerne les montures de Personnages. Les montures pour non-Personnages suivent les règles données dans leur description d'unité.
}