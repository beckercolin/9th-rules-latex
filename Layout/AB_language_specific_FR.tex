
\newcommand{\translationteam}{\item \og AEnoriel \fg \item \og Anglachel \fg \item \og Astadriel \fg \item \og Batcat \fg \item \og Bigfish \fg \item \og Eru \fg  \item \og Gandarin \fg \item \og Groumbahk \fg \item \og Iluvatar \fg \item \og Mammstein \fg \item \og Shlagrabak \fg \item et beaucoup d'autres...}

\hypersetup{
	pdfauthor={Équipe de traduction française de T9A},
	pdfsubject={Règles pour le jeu Batailles Fantastiques : Le 9\ieme{} Âge},
}

%%% Commands %%%

\ifdef{\isitanAB}{

\newcommand{\addtosortedlist}[1]{%
	\protected@edef\textarg{#1}%
	\protected@edef\textwithoutspaces{\expandafter\removespaces\expandafter{\textarg}}%
	\substitute\textwithoutspaces{É}{E}% Most used special characters of the language, and equivalent for alphabetical ordering
	\substitute\textwithoutspaces{È}{E}%
	\substitute\textwithoutspaces{Ê}{E}%
	\substitute\textwithoutspaces{é}{e}%
	\substitute\textwithoutspaces{è}{e}%
	\substitute\textwithoutspaces{ê}{e}%
	\substitute\textwithoutspaces{À}{A}%
	\substitute\textwithoutspaces{à}{a}%
	\substitute\textwithoutspaces{ù}{u}%
	\expandafter\sortitem\expandafter[\textwithoutspaces]{#1}%
}%

\newcommand{\pts}[1]{% First step is to remove spaces if there are some
	\def\numberwithoutspaces{\expandafter\removespaces\expandafter{#1}}%
	% Next step is getting rid of formatting if there are any (bold, color, ...)
	\pdfstringdef\cleannumber{\numberwithoutspaces}%
	% Now we can try if it is 1 or not
	\expandafter\ifstrequal\expandafter{\cleannumber}{1}{#1~\labels@point}{%
	\expandafter\ifstrequal\expandafter{\cleannumber}{0.5}{0,5~\labels@point}{%
	\expandafter\ifstrequal\expandafter{\cleannumber}{1.5}{1,5~\labels@point}{%
	#1~\labels@points}}}%
}

}{}

% Dark gods
\newcommand{\dchange}{Changement}
\newcommand{\dlust}{Luxure}
\newcommand{\pestilence}{Pestilence}
\newcommand{\wrath}{Courroux}
\newcommand{\truechaos}{Chaos Intégral}


% Nothing to edit here

\ifdef{\isitanAB}{

\newcommand{\alliancepts}[1]{
\ifsubstring{#1}{\free}{%
		\free{}%
	}{%
	\ifsubstring{#1}{\permodel}{%
		\splitatinf{#1}\myoption\myvalue%
		\pts{\myvalue}\permodel{}%
	}{%
	\pts{#1}
	}}
}

% You might wanna change the order of the gods - advanced user
\newcommand{\allianceoptions}[1]{%
	\defallianceoptions{#1}%
	\unitentryformat{\labels@allianceoptions\spacebeforecolon{}:}\newline
	\expandafter\ifblank\expandafter{\allianceoptions@introsentence}{}{\noindent\allianceoptions@introsentence{}\spacebeforecolon{}:}
	
	\expandafter\ifblank\expandafter{\allianceoptions@wrath}{
		\setlength{\columnseprule}{0.5pt}
		\renewcommand{\columnseprulecolor}{\color{black!30}}
		\vspace*{-0.2cm}\begin{multicols}{3}\raggedcolumns
		
			\begin{center}
			\noindent\dchange{}
			
			\noindent\alliancepts{\allianceoptions@change}
			\vspace*{-0.3cm}
			\end{center}
		
		\columnbreak
		
			\begin{center}
			\noindent\dlust{}
			
			\noindent\alliancepts{\allianceoptions@lust}
			\vspace*{-0.3cm}
			\end{center}
		
		\columnbreak

			\begin{center}
			\noindent\pestilence{}
			
			\noindent\alliancepts{\allianceoptions@pestilence}
			\vspace*{-0.3cm}
			\end{center}
			
		\end{multicols}
		\setlength{\columnseprule}{0pt}	
	}{
		\setlength{\columnseprule}{0.5pt}
		\renewcommand{\columnseprulecolor}{\color{black!30}}
		\vspace*{-0.2cm}\begin{multicols}{4}\raggedcolumns
		
			\begin{center}
			\noindent\dchange{}
			
			\noindent\alliancepts{\allianceoptions@change}
			\vspace*{-0.3cm}
			\end{center}
		
		\columnbreak

			\begin{center}
			\noindent\wrath{}
			
			\noindent\alliancepts{\allianceoptions@wrath}
			\vspace*{-0.3cm}
			\end{center}
		
		\columnbreak
		
			\begin{center}
			\noindent\dlust{}
			
			\noindent\alliancepts{\allianceoptions@lust}
			\vspace*{-0.3cm}
			\end{center}
		
		\columnbreak

			\begin{center}
			\noindent\pestilence{}
			
			\noindent\alliancepts{\allianceoptions@pestilence}
			\vspace*{-0.3cm}
			\end{center}
			
		\end{multicols}
		\setlength{\columnseprule}{0pt}
	}
}

}{}


%%% Labels %%%

% Profile

\newcommand{\labels@M}{M}
\newcommand{\labels@WS}{CC}
\newcommand{\labels@BS}{CT}
\newcommand{\labels@S}{F}
\newcommand{\labels@T}{E}
\newcommand{\labels@W}{PV}
\newcommand{\labels@I}{I}
\newcommand{\labels@A}{A}
\newcommand{\labels@Ld}{Cd}
\newcommand{\labels@Invocation}{Invocation} % For Vampire Covenant profiles
\newcommand{\labels@roundbase}{rond} % printed after XX mm for round bases

\newcommand{\Strength}{Force}

% Technical

\newcommand{\labels@range}{Portée}
\newcommand{\labels@point}{pt}
\newcommand{\labels@points}{pts}
\newcommand{\labels@only}{uniquement}
\newcommand{\labels@magic}{Magie}
\newcommand{\labels@pathsused}{Génère ses sorts dans la Discipline}
\newcommand{\labels@model}{figurine}
\newcommand{\labels@models}{figurines}
\newcommand{\labels@Singlemodel}{Figurine \textbf{seule}}

% Unit entry labels

\newcommand{\labels@basesize}{Socle}
\newcommand{\labels@trooptype}{Type de troupe}
\newcommand{\labels@specialrules}{Règles Spéciales}
\newcommand{\labels@alignment}{Allégeance}
\newcommand{\labels@alliance}{Allégeance}
\newcommand{\labels@allianceoptions}{Options d'Allégeance}
\newcommand{\labels@greenhiderace}{Race de Peaux Vertes}
\newcommand{\labels@equipment}{Équipement}
\newcommand{\labels@weapons}{Armes}
\newcommand{\labels@armour}{Armure}
\newcommand{\labels@options}{Options}
\newcommand{\labels@commandgroup}{État-Major}
\newcommand{\labels@charactermounts}{Montures de personnages}
\newcommand{\labels@mounts}{Montures}
\newcommand{\labels@mount}{Monture}
\newcommand{\labels@specialequipment}{Équipement Spécial}

% Command groups

\newcommand{\labels@champion}{Champion}
\newcommand{\labels@standardbearer}{Porte-étendard}
\newcommand{\labels@musician}{Musicien}
\newcommand{\labels@singlebannerallowance}{Une seule unité de ce type peut prendre une Bannière magique}
\newcommand{\labels@condsinglebannerallowance}{Une seule unité de ce type peut prendre une Bannière magique si}
\newcommand{\labels@bannerallowance}{Bannière Magique}
\newcommand{\labels@veteranstandardbearer}{Peut devenir Porte-étendard Vétéran}
\newcommand{\labels@championallowance}{Arme Magique}

% Titles

\newcommand{\labels@armylist}{Liste des troupes}
\newcommand{\labels@lords}{Seigneurs}
\newcommand{\labels@heroes}{Héros}
\newcommand{\labels@coreunits}{Unités de base}
\newcommand{\labels@specialunits}{Unités spéciales}
\newcommand{\labels@rareunits}{Unités rares}
\newcommand{\labels@armywiderules}{Règles communes de l'armée}
\newcommand{\labels@armyspecialrules}{Règles spéciales de l'armée}
\newcommand{\labels@armoury}{Armurerie}
\newcommand{\labels@magicalitems}{Objets magiques}
\newcommand{\labels@magicalweapons}{Armes magiques}
\newcommand{\labels@magicalarmour}{Armures magiques}
\newcommand{\labels@talismans}{Talismans}
\newcommand{\labels@enchanteditems}{Objets enchantés}
\newcommand{\labels@arcaneitems}{Objets cabalistiques}
\newcommand{\labels@magicalbanners}{Bannières magiques}
\newcommand{\labels@quickrefsheet}{Fiche de référence}
\newcommand{\labels@changelog}{Change Log}

\newcommand{\labels@lordsInitial}{S}
\newcommand{\labels@heroesInitial}{H}
\newcommand{\labels@coreunitsInitial}{B}
\newcommand{\labels@specialunitsInitial}{S}
\newcommand{\labels@rareunitsInitial}{R}
\newcommand{\labels@mountsInitial}{M}


% Titlepage

\newcommand{\labels@fantasybattles}{Batailles Fantastiques}
\newcommand{\labels@NinthAge}{Le 9\ieme{} Âge}
\newcommand{\labels@armyrules}{Règles de l'Armée}
\newcommand{\labels@frontpagecredits}{%
\labels@fantasybattles{} : \labels@NinthAge{} est un jeu créé et entretenu par la communauté qui met en scène des affrontements de figurines.\newline
Toutes les règles sont disponibles gratuitement sur le site suivant. Vos retours et suggestions sont les bienvenus : \url{http://www.the-ninth-age.com/}
}
\newcommand{\labels@license}{Copyright Creative Commons license : \url{the-ninth-age.com/license.html}}
\newcommand{\labels@tableofcontents}{Sommaire}
\newcommand{\labels@introduction}{%
\begin{center}\noindent{\Largerfontsize\textbf{Note des traducteurs}}\end{center}
\vspace{0.5cm}

Nous souhaitons remercier chaleureusement l'équipe à l'initiative du 9\ieme{} Âge pour leur motivation et leur travail continu pour faire vivre notre passion. Nous espérons que ce jeu saura développer les qualités pour plaire au plus grand nombre et réunir les joueurs, amateurs comme habitués des tournois, autour de règles amusantes et équilibrées, pour finalement s'imposer comme un standard du jeu de figurines. Une grande ambition qui ne pourra s'accomplir que \textbf{grâce à vous}, la communauté, via des retours constructifs, afin de modeler le jeu selon nos désirs. N'étant \textbf{en aucun cas à but lucratif}, le 9\ieme{} Âge part avec un avantage considérable. Les règles des éventuelles nouvelles sorties ne sont pas dictées par le besoin de vendre ces nouveautés. Vous pouvez choisir et acheter vos figurines où bon vous semble, il n'y a pas un unique revendeur toléré. Enfin, vous pouvez être assurés que tant que le 9\ieme{} Âge sera joué, vous disposerez d'un \textbf{support continu et régulier}, celui-ci étant offert par la communauté.

Concernant la traduction en elle-même, nous avons fait de notre mieux pour vous offrir une version de qualité, dont nous espérons qu'elle surpasse celle de la version originale ! Si vous constatez des coquilles, des erreurs, merci de nous les signaler en nous contactant sur le forum du 9\ieme{} Âge, dans le \textbf{sous-forum français} (\url{http://www.the-ninth-age.com/index.php?board/117-french/}). Vous y trouverez aussi les dernières mises à jour. \textbf{En cas de conflit d'interprétation avec la version originale, la version originale fait référence}.

\vspace{0.5cm}
Que ce jeu vous apporte d'innombrables heures de plaisir partagé !

\vspace{1cm}

\ifdef{\translationteam}{
	\begin{multicols}{3}
	\begin{itemize}
		\translationteam
	\end{itemize}
	\end{multicols}
}{}
}
\newcommand{\labels@rulechanges}{% blank ATM
}
\newcommand{\labels@latexcredit}{Document réalisé à l'aide de \LaTeX .}


%%% Technical commands

\newcommand{\only}[1]{(#1 uniquement)}
\newcommand{\free}{gratuit}
\newcommand{\upto}{jusqu'à}
\newcommand{\Upto}{Jusqu'à}
\newcommand{\unlimited}{pas de limite}
\newcommand{\permodel}{/fig.}
\newcommand{\listlastchoice}{ ou}
\newcommand{\notif}[1]{(pas #1)}
\newcommand{\wordand}{et}
\newcommand{\wordwith}{avec}
\newcommand{\ifNmodelsorless}[1]{(#1 figurines ou moins)}
\newcommand{\unitwith}{unité avec}
\newcommand{\From}{De} % From ... to ... models
\newcommand{\wordto}{à}
\newcommand{\wordAll}{Tous}
\newcommand{\spacebeforecolon}{ } % French put a space before colons
\newcommand{\minprice}{Coût min. :}
\newcommand{\mincostfor}{Coût min. pour}
\newcommand{\maxunitsize}{Taille max.}
\newcommand{\additionalfigscost}{Les figurines additionnelles coûtent}


%%% Special rules %%%

\newcommand{\ambush}{Embuscade}
\newcommand{\armourpiercing}[1]{Perforant\ifblank{#1}{}{ (#1)}}
\newcommand{\bodyguard}[1]{Garde du Corps\ifblank{#1}{}{ (#1)}}
\newcommand{\breathweapon}[1]{Attaque de Souffle\ifblank{#1}{}{ (#1)}}
\newcommand{\channel}{Canalisation}
\newcommand{\crushattack}{Attaque Écrasante}
\newcommand{\daemonicinstability}{Instabilité Démoniaque}
\newcommand{\devastatingcharge}{Charge Dévastatrice}
\newcommand{\distracting}{Distrayant}
\newcommand{\divineattacks}{Attaques Divines}
\newcommand{\engineer}{Ingénieur}
\newcommand{\ethereal}{Éthéré}
\newcommand{\fastcavalry}{Cavalerie Légère}
\newcommand{\fear}{Peur}
\newcommand{\fightinextrarank}{Combat avec un Rang Supplémentaire}
\newcommand{\fireborn}{Né du Feu}
\newcommand{\flamingattacks}{Attaques Enflammées}
\newcommand{\flammable}{Inflammable}
\newcommand{\frenzy}{Frénésie}
\newcommand{\fly}[1]{Vol\ifblank{#1}{}{ (#1)}}
\newcommand{\grindingattacks}[1]{Attaques de Broyage\ifblank{#1}{}{ (#1)}}
\newcommand{\hardtarget}{Camouflé}
\newcommand{\hatred}{Haine}
\newcommand{\hellfire}{Feu Démoniaque}
\newcommand{\hidden}{Caché}
\newcommand{\holyattacks}{Attaques Divines} % deprecated, still has to be filled. same as Divine Attacks.
\newcommand{\immunetopsychology}{Immunisé à la Psychologie}
\newcommand{\impacthits}[1]{Touches d'Impact\ifblank{#1}{}{ (#1)}}
\newcommand{\insignificant}{Insignifiant}
\newcommand{\largetarget}{Grande Cible}
\newcommand{\lethalstrike}{Coup Fatal}
\newcommand{\lightningattacks}{Attaques Foudroyantes}
\newcommand{\lightningreflexes}{Réflexes Foudroyants}
\newcommand{\lighttroops}{Troupe Légère}
\newcommand{\magicresistance}[1]{Résistance à la Magie\ifblank{#1}{}{ (#1)}}
\newcommand{\magicalattacks}{Attaques Magiques}
\newcommand{\metalshifting}{Fusion du Métal}
\newcommand{\moveorfire}{Mouvement ou Tir}
\newcommand{\multipleshots}[1]{Tirs Multiples\ifblank{#1}{}{ (#1)}}
\newcommand{\multiplewounds}[2]{Blessures Multiples\ifblank{#1}{}{ (#1\ifblank{#2}{)}{, #2)}}}
\newcommand{\notaleader}{Pas un Meneur}
\newcommand{\otherworldly}{D'Outre-Monde}
\newcommand{\pathmaster}[1]{Maître de la Voie\ifblank{#1}{}{ (#1)}}
\newcommand{\poisonedattacks}{Attaques Empoisonnées}
\newcommand{\quicktofire}{Tir Rapide}
\newcommand{\randommovement}[1]{Mouvement Aléatoire\ifblank{#1}{}{ (#1)}}
\newcommand{\randomattacks}[1]{Attaques Aléatoires\ifblank{#1}{}{ (#1)}}
\newcommand{\regeneration}[1]{Régénération\ifblank{#1}{}{ (#1+)}}
\newcommand{\reload}{Rechargez !}
\newcommand{\requirestwohands}{Arme à deux Mains}
\newcommand{\scythes}{Faux}
\newcommand{\scout}{Éclaireur}
\newcommand{\scouts}{Éclaireurs}
\newcommand{\stomp}[1]{Piétinement\ifblank{#1}{}{ (#1)}}
\newcommand{\strider}[1]{Guide\ifblank{#1}{}{ (#1)}}
\newcommand{\stubborn}{Tenace}
\newcommand{\stupidity}{Stupidité}
\newcommand{\skirmisher}{Tirailleur}
\newcommand{\skirmishers}{Tirailleurs}
\newcommand{\sweepingattack}{Attaque au Passage}
\newcommand{\swiftstride}{Course Rapide}
\newcommand{\thunderouscharge}{Charge Tonitruante}
\newcommand{\terror}{Terreur}
\newcommand{\toxicattacks}{Attaques Toxiques}
\newcommand{\unbreakable}{Indémoralisable}
\newcommand{\undead}{Mort-Vivant}
\newcommand{\unstable}{Instable}
\newcommand{\unwieldy}{Encombrant}
\newcommand{\vanguard}{Avant-Garde}
\newcommand{\volleyfire}{Tir de Volée}
\newcommand{\warplatform}{Plateforme de Guerre}
\newcommand{\wardsave}[1]{Sauvegarde Invulnérable\ifblank{#1}{}{ (#1+)}}
\newcommand{\weaponmaster}{Maître d'Ar\-mes}
\newcommand{\wizardconclave}[1]{Conclave de Sorciers\ifblank{#1}{}{ (#1)}}


%%% Magic %%%


% General

\newcommand{\Pathof}{Voie}

\newcommand{\battle}{Commune}

\newcommand{\anyofthebattlemagic}{dans n'importe laquelle des Voies Communes}
\newcommand{\ONLYanyofthebattlemagic}{Commune de votre choix}

\newcommand{\magiclevel}[1]{\ifnumcomp{#1}{<}{3}{Apprenti Magicien}{Maître Magicien} Niveau #1}
\newcommand{\Level}{Niveau}

\newcommand{\wizard}{Magicien}
\newcommand{\wizards}{Magiciens}

\newcommand{\learnedspell}{Sort Appris}
\newcommand{\learnedspells}{Sorts Appris}
\newcommand{\attributespell}{Attribut de la Voie}
\newcommand{\attributespells}{Attributs de la Voie}
\newcommand{\attributespellnumber}{A}
\newcommand{\traitspell}{Sort Caractéristique}
\newcommand{\traitspells}{Sorts Caractéristiques}
\newcommand{\traitspellnumber}{C}


\newcommand{\boundspell}[1]{Objet de Sort\ifblank{#1}{}{, Puissance #1}}
\newcommand{\boundspells}[1]{Objets de Sort\ifblank{#1}{}{, Puissance #1}}

% Casting Vocabulary

\newcommand{\lostfocus}{Perte de Concentration}
\newcommand{\miscast}{Fiasco}
\newcommand{\miscasts}{Fiascos}
\newcommand{\overwhelmingpower}{Pouvoir Irrésistible}

\newcommand{\breachintheveil}{Brèche dans le Voile}
\newcommand{\catastrophicdetonation}{Explosion Catastrophique}
\newcommand{\witchfire}{Feu de Sorcières}
\newcommand{\sorcerousbacklash}{Contrecoup Magique}
\newcommand{\amnesia}{Amnésie}

% Spell Types

\newcommand{\augment}{Amélioration}
\newcommand{\hex}{Malédiction}
\newcommand{\universal}{Universel}
\newcommand{\missile}{Projectile}
\newcommand{\damage}{Dégâts}
\newcommand{\direct}{Direct}
\newcommand{\focused}{Focalisé}
\newcommand{\vortex}{Vortex}
\newcommand{\ground}{Marqueur}
\newcommand{\linetemplate}{Gabarit de Ligne}
\newcommand{\specialTYPE}{Spécial}
\newcommand{\aura}{Aura}
\newcommand{\castersunit}{Unité du Lanceur}
\newcommand{\caster}{Lanceur}

\newcommand{\template}{Gabarit}

% Spell Durations

\newcommand{\lastsoneturn}{Dure un Tour}
\newcommand{\instant}{Immédiat}
\newcommand{\permanent}{Permanent}
\newcommand{\remainsinplay}{Reste en Jeu}


% Battle Magic

\newcommand{\alchemy}{de l'Alchimie}
\newcommand{\alchemyattribute}{Édit de Fer}
\newcommand{\alchemysignature}{Métal Fondu}
\newcommand{\alchemyspellone}{Lames Enchantées}
\newcommand{\alchemyspelltwo}{Corrosion Rampante}
\newcommand{\alchemyspellthree}{Manteau de Vif-Argent}
\newcommand{\alchemyspellfour}{Pieu d'Argent}
\newcommand{\alchemyspellfive}{Fléau de l'Acier}
\newcommand{\alchemyspellsix}{Transmutation en Or}

\newcommand{\death}{de la Mort}
\newcommand{\deathattribute}{Nuage de Désespoir}
\newcommand{\deathsignature}{Le Baiser de la Faucheuse}
\newcommand{\deathspellone}{Malédiction du Mortel}
\newcommand{\deathspelltwo}{Esprits Dévorants}
\newcommand{\deathspellthree}{Sangsue Psychique}
\newcommand{\deathspellfour}{Moisson d’Âmes}
\newcommand{\deathspellfive}{L’Abîme aussi te Regarde...}
\newcommand{\deathspellsix}{Maelström d’Âmes}

\newcommand{\fire}{du Feu}
\newcommand{\fireattribute}{Feu Déchaîné}
\newcommand{\firesignature}{Boule de Feu}
\newcommand{\firespellone}{Cascade Ardente}
\newcommand{\firespelltwo}{Épées Flamboyantes}
\newcommand{\firespellthree}{Jet de Flammes}
\newcommand{\firespellfour}{Traits Enflammés}
\newcommand{\firespellfive}{Remparts Incandescents}
\newcommand{\firespellsix}{Souffler sur les Braises}

\newcommand{\heavens}{des Cieux}
\newcommand{\heavensattribute}{Second Sceau}
\newcommand{\heavenssignature}{Aquilon}
\newcommand{\heavensspellone}{Bourrasque}
\newcommand{\heavensspelltwo}{Choc Foudroyant}
\newcommand{\heavensspellthree}{Conjonction Astrale}
\newcommand{\heavensspellfour}{Fléau du Ponant}
\newcommand{\heavensspellfive}{Déluge d'Éclairs}
\newcommand{\heavensspellsix}{Appel de la Comète}

\newcommand{\light}{de la Lumière}
\newcommand{\lightattribute}{Lumière Gardienne}
\newcommand{\lightsignature}{Éclat Brûlant}
\newcommand{\lightspellone}{Bouclier Protecteur}
\newcommand{\lightspelltwo}{Étincelle de Courage}
\newcommand{\lightspellthree}{Vitesse Fulgurante}
\newcommand{\lightspellfour}{Toile Scintillante}
\newcommand{\lightspellfive}{Distorsion Temporelle}
\newcommand{\lightspellsix}{Bannissement Divin}

\newcommand{\nature}{de la Nature}
\newcommand{\natureattribute}{Souffle de Vie}
\newcommand{\naturesignature}{Eaux Vivifiantes}
\newcommand{\naturespellone}{Maître de la Terre}
\newcommand{\naturespelltwo}{Le Trône de Chêne}
\newcommand{\naturespellthree}{Esprits des Bois}
\newcommand{\naturespellfour}{Croissance Estivale}
\newcommand{\naturespellfive}{Peau Rocailleuse}
\newcommand{\naturespellsix}{Créatures Souterraines}

\newcommand{\shadows}{des Ombres}
\newcommand{\shadowsattribute}{Course Parmi les Ombres}
\newcommand{\shadowssignature}{Miasmes Obscurs}
\newcommand{\shadowsspellone}{Orbe de Noirceur}
\newcommand{\shadowsspelltwo}{Partir en Fumée}
\newcommand{\shadowsspellthree}{Expérience de Mort Imminente}
\newcommand{\shadowsspellfour}{Char Vaporeux}
\newcommand{\shadowsspellfive}{Ombres Dévorantes}
\newcommand{\shadowsspellsix}{Scalpel Psychique}

\newcommand{\wilderness}{de la Sauvagerie}
\newcommand{\wildernessattribute}{La Chasse Sauvage}
\newcommand{\wildernesssignature}{La Bête qui Sommeille}
\newcommand{\wildernessspellone}{Essaim d’Insectes}
\newcommand{\wildernessspelltwo}{Rage Intérieure}
\newcommand{\wildernessspellthree}{Pieu de Rougebois}
\newcommand{\wildernessspellfour}{Calamité des Bois Sauvages}
\newcommand{\wildernessspellfive}{Tempête Furieuse}
\newcommand{\wildernessspellsix}{Métamorphose
Monstrueuse}

\newcommand{\eightpaths}{Octuple}



% Army Specific Magic

\newcommand{\butchery}{de la Boucherie}
\newcommand{\butcheryattribute}{Sang de Kholag}
\newcommand{\butcherysignature}{Briseur de Dents}
\newcommand{\butcheryspellone}{Buveur de Moelle}
\newcommand{\butcheryspelltwo}{Festin de Tripaille}
\newcommand{\butcheryspellthree}{Concasseur d’Os}
\newcommand{\butcheryspellfour}{Gobeur de Cervelle}
\newcommand{\butcheryspellfive}{Cœur de Troll}
\newcommand{\butcheryspellsix}{Gosier de Géant}

\newcommand{\change}{du Changement}
\newcommand{\changeattribute}{Vent du Changement}
\newcommand{\changesignature}{Feu Azur}
\newcommand{\changespellone}{Feu Rose}
\newcommand{\changespelltwo}{Vague du Changement}
\newcommand{\changespellthree}{Secrets Volés}
\newcommand{\changespellfour}{Règne de la Confusion}
\newcommand{\changespellfive}{Inéluctable Trahison}
\newcommand{\changespellsix}{Portail Éternel}

\newcommand{\thebiggreengods}{des Grands Dieux Verts}
\newcommand{\thebiggreengodsattribute}{Chopez-les !}
\newcommand{\thebiggreengodssignature}{L'Heure de la Raclée}
\newcommand{\thebiggreengodsspellone}{Coup de Boule}
\newcommand{\thebiggreengodsspelltwo}{Poings Bastonneurs}
\newcommand{\thebiggreengodsspellthree}{Même Pas Mal !}
\newcommand{\thebiggreengodsspellfour}{Grande Main Verte}
\newcommand{\thebiggreengodsspellfive}{Boum !}
\newcommand{\thebiggreengodsspellsix}{Le Gros Piétinement}

\newcommand{\thelittlegreengods}{des Petits Dieux Verts}
\newcommand{\thelittlegreengodsattribute}{Fourbe Larcin}
\newcommand{\thelittlegreengodssignature}{Œil Mauvais}
\newcommand{\thelittlegreengodsspellone}{Taillades Sournoises}
\newcommand{\thelittlegreengodsspelltwo}{Bénédiction de la Mère-Araignée}
\newcommand{\thelittlegreengodsspellthree}{Ça Démange ?}
\newcommand{\thelittlegreengodsspellfour}{Chut ! Pas un Bruit...}
\newcommand{\thelittlegreengodsspellfive}{J’vous Arrange Ça}
\newcommand{\thelittlegreengodsspellsix}{Malédiction de la Lune Verte}

\newcommand{\blackmagic}{de la Magie Noire}
\newcommand{\blackmagicattribute}{Soif d’Âmes}
\newcommand{\blackmagicsignature}{Furie de Moraec}
\newcommand{\blackmagicspellone}{Rafale Glaciale}
\newcommand{\blackmagicspelltwo}{Tourbillon de Lames}
\newcommand{\blackmagicspellthree}{Agonie Paralysante}
\newcommand{\blackmagicspellfour}{Marque de la Peur}
\newcommand{\blackmagicspellfive}{Trait d’Énergie Noire}
\newcommand{\blackmagicspellsix}{Terreur Noire}

\newcommand{\disease}{de la Maladie}
\newcommand{\diseaseattribute}{Bénédiction Nécrotique}
\newcommand{\diseasesignature}{Relents de Pestilence}
\newcommand{\diseasespellone}{Haleine Corruptrice}
\newcommand{\diseasespelltwo}{Toucher Putréfiant}
\newcommand{\diseasespellthree}{Excroissance Adipeuse}
\newcommand{\diseasespellfour}{Purge Parasitaire}
\newcommand{\diseasespellfive}{Malédiction du Lépreux}
\newcommand{\diseasespellsix}{Tourbillon Fétide}

\newcommand{\lust}{de la Luxure}
\newcommand{\lustattribute}{Masochisme}
\newcommand{\lustsignature}{Flagellation Démoniaque}
\newcommand{\lustspellone}{Grâce Hypnotique}
\newcommand{\lustspelltwo}{Valse Irrésistible}
\newcommand{\lustspellthree}{Hystérie}
\newcommand{\lustspellfour}{Fantasmagorie}
\newcommand{\lustspellfive}{Déchirement psychique}
\newcommand{\lustspellsix}{Chœur Dissonant}

\newcommand{\necromancy}{de la Nécromancie}
\newcommand{\necromancyattribute}{Tromper la Faucheuse}
\newcommand{\necromancysignature}{Adjuration des Morts}
\newcommand{\necromancyspellone}{Parodie de Vie}
\newcommand{\necromancyspelltwo}{Convocation Profanatoire}
\newcommand{\necromancyspellthree}{Sarabande Macabre}
\newcommand{\necromancyspellfour}{Regard de Setesh}
\newcommand{\necromancyspellfive}{Vol de Jeunesse}
\newcommand{\necromancyspellsix}{Malédiction des Morts}

\newcommand{\ruin}{de la Ruine}
\newcommand{\ruinattribute}{Hordes Sans Fin}
\newcommand{\ruinsignature}{Éclair Noir}
\newcommand{\ruinspellone}{Nourrissons-les...}
\newcommand{\ruinspelltwo}{Souiller le Sol}
\newcommand{\ruinspellthree}{La Faim}
\newcommand{\ruinspellfour}{Appel de la Tempête}
\newcommand{\ruinspellfive}{Rupture Sismique}
\newcommand{\ruinspellsix}{Pour Qui Sonne le Glas}

\newcommand{\forge}{de la Forge}
\newcommand{\forgeattribute}{Fournaise Haineuse}
\newcommand{\forgesignature}{Bouclier de Sombrefeu}
\newcommand{\forgespellone}{Rage Incendiaire}
\newcommand{\forgespelltwo}{Subjugation}
\newcommand{\forgespellthree}{Souffle de Haine}
\newcommand{\forgespellfour}{Anathème de Noirceur}
\newcommand{\forgespellfive}{Cendres Asphyxiantes}
\newcommand{\forgespellsix}{Flammes de la Forge}

\newcommand{\sands}{des Sables}
\newcommand{\sandsattribute}{Les Morts sans Repos}
\newcommand{\sandssignature}{Sirocco}
\newcommand{\sandsspellone}{Lames Maudites}
\newcommand{\sandsspelltwo}{Dessiccation Mortelle}
\newcommand{\sandsspellthree}{Frappes Vengeresses}
\newcommand{\sandsspellfour}{Jugement Divin}
\newcommand{\sandsspellfive}{Sables Mouvants}
\newcommand{\sandsspellsix}{Écho des Gloires
Passées}

\newcommand{\whitemagic}{de la Magie Blanche}
\newcommand{\whitemagicattribute}{Bouclier des Anciens}
\newcommand{\whitemagicsignature}{Traits de Lumière}
\newcommand{\whitemagicspellone}{Résurrection du Phénix}
\newcommand{\whitemagicspelltwo}{Volonté Inspirante}
\newcommand{\whitemagicspellthree}{Sentier Secret}
\newcommand{\whitemagicspellfour}{Bénédiction d’Amhar}
\newcommand{\whitemagicspellfive}{Fusion d’Artefact}
\newcommand{\whitemagicspellsix}{Cataclysme}

% Paths Initials

\newcommand{\alchemyInitials}{A}
\newcommand{\deathInitials}{M}
\newcommand{\fireInitials}{F}
\newcommand{\heavensInitials}{C}
\newcommand{\lightInitials}{L}
\newcommand{\natureInitials}{N}
\newcommand{\shadowsInitials}{O}
\newcommand{\wildernessInitials}{S}

\newcommand{\eightfoldInitials}{8}

\newcommand{\whitemagicInitials}{MB}
\newcommand{\blackmagicInitials}{MN}
\newcommand{\necromancyInitials}{N}
\newcommand{\sandsInitials}{S}
\newcommand{\forgeInitials}{F}
\newcommand{\biggreengodsInitials}{GDV}
\newcommand{\littlegreengodsInitials}{PDV}
\newcommand{\butcheryInitials}{B}
\newcommand{\ruinInitials}{R}
\newcommand{\diseaseInitials}{M}
\newcommand{\lustInitials}{L}
\newcommand{\changeInitials}{C}


%%% Other rules %%%

\newcommand{\armoursave}{Sauvegarde d'Armure}
\newcommand{\firstinrank}{Au Premier Rang}
\newcommand{\hardcover}{Couvert Lourd}
\newcommand{\holdyourground}{Tenez les Rangs}
\newcommand{\inspiringpresence}{Présence Charismatique}
\newcommand{\lightcover}{Couvert Léger}
\newcommand{\monstrousranks}{Rangs Monstrueux}
\newcommand{\monstroussupport}{Soutien Monstrueux}
\newcommand{\ordnance}{Artillerie}
\newcommand{\parry}{Parade}
\newcommand{\raisewounds}{Ressusciter des Figurines}
\newcommand{\recoverwounds}{Récupérer des PVs}
\newcommand{\aideddispel}{Dissipation Assistée}
\newcommand{\rnf}{ordinaires}
\newcommand{\general}{Général}
\newcommand{\bsb}{Porteur de la Grande Bannière}
\newcommand{\cannotmarch}{Pas de Marche Forcée}
\newcommand{\veteranstandardbearer}{Porte-Étendard Vétéran}
\newcommand{\swirlingmelee}{Mêlée Tourbillonnante}
\newcommand{\scoringunit}{Unité de Capture}


%%% Equipment %%%

\newcommand{\hw}{Arme de Base}
\newcommand{\pw}{Paire d'Armes}
\newcommand{\spear}{Lance}
\newcommand{\halberd}{Hallebarde}
\newcommand{\gw}{Arme Lourde}
\newcommand{\lance}{Lance de Cavalerie}
\newcommand{\lightlance}{Lance Légère}
\newcommand{\flail}{Fléau}

\newcommand{\throwingweapons}{Armes de Jet}
\newcommand{\shortbow}{Arc Court}
\newcommand{\bow}{Arc}
\newcommand{\longbow}{Arc Long}
\newcommand{\handgun}{Arquebuse}
\newcommand{\crossbow}{Arbalète}
\newcommand{\pistol}{Pistolet}
\newcommand{\braceofpistols}{Paire de Pistolets}	

\newcommand{\innatedefence}[1]{Protection Innée\ifblank{#1}{}{~(#1+)}}
\newcommand{\mountsprotection}[1]{Protection de Monture\ifblank{#1}{}{~(#1+)}}
\newcommand{\la}{Armure Légère}
\newcommand{\ha}{Armure Lourde}
\newcommand{\platearmour}{Armure de Plates}
\newcommand{\shield}{Bouclier}
\newcommand{\barding}{Caparaçon}

\newcommand{\cannon}{Canon}
\newcommand{\cannons}{Canons}
\newcommand{\catapult}{Catapulte}
\newcommand{\catapults}{Catapultes}
\newcommand{\volleygun}{Batterie de Tir}
\newcommand{\boltthrower}{Baliste}
\newcommand{\flamethrower}{Lance-Flammes}
\newcommand{\artilleryweapon}{Arme d'Artillerie}


%%% Troop types %%%

\newcommand{\characters}{Personnages}
\newcommand{\infantry}{Infanterie}
\newcommand{\monstrousinfantry}{Infanterie Monstrueuse}
\newcommand{\cavalry}{Cavalerie}
\newcommand{\monstrouscavalry}{Cavalerie Monstrueuse}
\newcommand{\swarm}{Nuée}
\newcommand{\swarms}{Nuées}
\newcommand{\warbeast}{Bête de Guerre}
\newcommand{\warbeasts}{Bêtes de Guerre}
\newcommand{\monster}{Monstre}
\newcommand{\monsters}{Monstres}
\newcommand{\monstrousbeast}{Bête Monstrueuse}
\newcommand{\monstrousbeasts}{Bêtes Monstrueuses}
\newcommand{\chariot}{Char}
\newcommand{\chariots}{Chars}
\newcommand{\riddenmonster}{Monstre Monté}
\newcommand{\riddenmonsters}{Monstres Montés}
\newcommand{\warmachine}{Machine de Guerre}
\newcommand{\warmachines}{Machines de Guerre}


%%% Terrain %%%

\newcommand{\water}{Eaux Peu Profondes}
\newcommand{\forest}{Forêt}
\newcommand{\impassableterrain}{Terrain Infranchissable}


%%% Profile wording

\newcommand{\oneperarmy}{Un par Armée}
\newcommand{\oneofakind}{Uni\-que}
\newcommand{\zerotoXchoice}[1]{0-#1 Choix}
\newcommand{\onechoiceonlyNOC}{(un seul choix)}
\newcommand{\onfootonly}{(à pied uniquement)}
\newcommand{\closecombatonly}{seulement au Corps à Corps}
\newcommand{\Xmodelsorless}[1]{(max. #1 figurines)}
\newcommand{\magicalitemsallowance}{Objets Magiques}
\newcommand{\magicalweaponallowance}{Arme Magique}
\newcommand{\notmagicalarmour}{(pas d'Armure Magique)}
\newcommand{\weapononechoice}{\optionschoice{Arme \onechoiceonlyNOC{} :}}
\newcommand{\weaponschoice}{\optionschoice{Armes :}}
\newcommand{\shootingweapononechoice}{\optionschoice{Arme de Tir \onechoiceonlyNOC{} :}}
\newcommand{\combatweapononechoice}{\optionschoice{Arme de Corps à Corps \onechoiceonlyNOC{} :}}
\newcommand{\combatweapononechoiceTWOCOL}{\optionschoiceTWOCOL{Arme de Corps à Corps \onechoiceonlyNOC{} :}}
\newcommand{\armouronechoice}{\optionschoice{Armure \onechoiceonlyNOC{} :}}
\newcommand{\magiclevelchoice}{\optionschoice{Magie \onechoiceonlyNOC{} :}}
\newcommand{\mustbecomeoneofthefollowing}{\optionschoice{\textbf{Doit} devenir au choix :}}
\newcommand{\mustbecomeoneofthefollowingNOC}{Doit devenir au choix :}
\newcommand{\musttakeoneormoreofthefollowing}{\optionschoice{\textbf{Doit} prendre au moins un choix :}}
\newcommand{\musttakeoneofthefollowing}{\optionschoice{\textbf{Doit} prendre un et un seul choix :}}
\newcommand{\musttakeoneofthefollowingNOC}{Doit choisir entre :}
\newcommand{\uptotwoofthefollowing}{\optionschoice{Jusqu'à deux choix :}}
\newcommand{\uptotwoofthefollowingTWOCOL}{\optionschoiceTWOCOL{Jusqu'à deux choix :}}

\newcommand{\onechoiceonly}{\optionschoice{Un seul choix :}}
\newcommand{\onechoiceonlyTWOCOL}{\optionschoiceTWOCOL{Un seul choix :}}

\newcommand{\maytake}{Peut prendre}




%%% Orcs N Goblins debug, let it as it is

\newcommand{\pershadygit}{debug}
\newcommand{\permadgit}{debug}

%%% Dwarven Holds debug, let it as it is

\newcommand{\perrune}{debug}
