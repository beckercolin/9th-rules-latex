
\documentclass[a4paper,8pt]{extarticle} % extarticle allows to use font size of 8pt.

\usepackage[a4paper, top=1.6cm, bottom=2cm, left=1.6cm, right=1.6cm]{geometry} % Marge reduction.

%% Language specific package
\usepackage[french]{babel}
\frenchbsetup{StandardLists=true} % Necessary to use enumitem with babel/french.

%% Font and typing packages
\usepackage{fontspec}
\setmainfont[
	Ligatures=TeX,
	ItalicFont={Dancing Script},
	BoldItalicFont={Dancing Script}
	]{PT Serif} % default is Latin Modern
\newfontfamily\antiquefont[Ligatures=TeX]{Caslon Antique} % fancy font
\usepackage{microtype}			% Greatly improves general appearance of the text.
\usepackage{SIunits}			% Unit appearance.
\usepackage{xspace}				% Define commands that appear not to eat spaces.
\usepackage{ulem}				% To cross words out. Use \sout{}.

%% Array utilities
\usepackage{array}				% Additionnal options for arrays.
\usepackage{colortbl}			% Additionnal options for coloring arrays.
\usepackage[table]{xcolor}		% Auto alternate grey-white rows.
\usepackage[export]{adjustbox}		% Centered pics in tables

%% List utilities
\usepackage[inline]{enumitem}   % Display inline lists.
\usepackage{etoolbox}           % General utility. Good for lists for instance.
\usepackage{xparse}             % List utilities.
\usepackage{datatool}	% Handling alphabetical order.

%% Frames
\usepackage{framed}				% Boxes.
\usepackage[framemethod=TikZ]{mdframed}% For fancy frames.
\usepackage{tikz}				% For fancy frames.
\usepackage{wrapfig}			% Fancy insertion of pics in text.

%% Page utilities
\usepackage{multicol}			% Allows to divide a part of the page in multiple columns.
	
%% Others
\usepackage{keyval}             % Used to create maps of commands/labels/objects.
	\makeatletter                  % Mandatory for the usage of keyval.
\usepackage{xstring}            % String parsing, cutting, etc.
\usepackage{hyperref} % Links in PDF.


%%% Update of the dotfill command to always get dots

\newcommand{\predotfill}{\penalty0\hbox{}\nobreak}%


%%% Command to avoid typing \xspace when creating a new name macro

\newcommand{\newnamemacro}[2]{\newcommand{#1}{#2}} % \xspace removed for compatibility with alphabetical ordering

%%% Language specific stuff


%%% Commands %%%

\newcommand{\addtosortedlist}[1]{%
	\protected@edef\textarg{#1}%
	\protected@edef\textwithoutspaces{\expandafter\removespaces\expandafter{\textarg}}%
	\substitute\textwithoutspaces{É}{e}% Most used special characters of the language, and equivalent for alphabetical ordering
	\substitute\textwithoutspaces{È}{e}%
	\substitute\textwithoutspaces{Ê}{e}%
	\substitute\textwithoutspaces{é}{e}%
	\substitute\textwithoutspaces{è}{e}%
	\substitute\textwithoutspaces{ê}{e}%
	\substitute\textwithoutspaces{À}{a}%
	\substitute\textwithoutspaces{à}{a}%
	\substitute\textwithoutspaces{ù}{u}%
	\expandafter\sortitem\expandafter[\textwithoutspaces]{#1}%
}%


%%% Labels %%%

% Profile

\newcommand{\labels@M}{M}
\newcommand{\labels@WS}{CC}
\newcommand{\labels@BS}{CT}
\newcommand{\labels@S}{F}
\newcommand{\labels@T}{E}
\newcommand{\labels@W}{PV}
\newcommand{\labels@I}{I}
\newcommand{\labels@A}{A}
\newcommand{\labels@Ld}{Cd}
\newcommand{\labels@Invocation}{Invocation} % For Vampire Covenant profiles

\newcommand{\Strength}{Force}

% Technical

\newcommand{\labels@range}{Portée}
\newcommand{\labels@point}{pt}
\newcommand{\labels@points}{pts}
\newcommand{\labels@only}{uniquement}
\newcommand{\labels@magic}{Magie}
\newcommand{\labels@pathsused}{Génère ses sorts dans la Discipline}
\newcommand{\labels@model}{figurine}
\newcommand{\labels@models}{figurines}
\newcommand{\labels@Singlemodel}{Figurine \textbf{seule}}

% Unit entry labels

\newcommand{\labels@basesize}{Socle}
\newcommand{\labels@trooptype}{Type de troupe}
\newcommand{\labels@specialrules}{Règles spéciales}
\newcommand{\labels@alignment}{Allégeance}
\newcommand{\labels@equipment}{Équipement}
\newcommand{\labels@weapons}{Armes}
\newcommand{\labels@armour}{Armure}
\newcommand{\labels@options}{Options}
\newcommand{\labels@commandgroup}{État-Major}
\newcommand{\labels@mounts}{Montures}
\newcommand{\labels@specialequipment}{Équipement spécial}

% Command groups

\newcommand{\labels@champion}{Champion}
\newcommand{\labels@standardbearer}{Porte-étendard}
\newcommand{\labels@musician}{Musicien}
\newcommand{\labels@singlebannerallowance}{Une seule unité de ce type peut prendre une Bannière magique}
\newcommand{\labels@condsinglebannerallowance}{Une seule unité de ce type peut prendre une Bannière magique si}
\newcommand{\labels@bannerallowance}{Peut prendre une Bannière Magique}
\newcommand{\labels@veteranstandardbearer}{Peut devenir Porte-étendard Vétéran}
\newcommand{\labels@championallowance}{Peut prendre une Arme Magique}

% Titles

\newcommand{\labels@lords}{Seigneurs}
\newcommand{\labels@heroes}{Héros}
\newcommand{\labels@coreunits}{Unités de base}
\newcommand{\labels@specialunits}{Unités spéciales}
\newcommand{\labels@rareunits}{Unités rares}
\newcommand{\labels@armywiderules}{Règles communes de l'armée}
\newcommand{\labels@armyspecialrules}{Règles spéciales de l'armée}
\newcommand{\labels@armoury}{Armurerie}
\newcommand{\labels@magicalitems}{Objets magiques}
\newcommand{\labels@magicalweapons}{Armes magiques}
\newcommand{\labels@magicalarmour}{Armures magiques}
\newcommand{\labels@talismans}{Talismans}
\newcommand{\labels@enchanteditems}{Objets enchantés}
\newcommand{\labels@arcaneitems}{Objets cabalistiques}
\newcommand{\labels@magicalbanners}{Bannières magiques}
\newcommand{\labels@quickrefsheet}{Fiche de référence}
\newcommand{\labels@changelog}{Change Log}

\newcommand{\labels@lordsInitial}{S}
\newcommand{\labels@heroesInitial}{H}
\newcommand{\labels@coreunitsInitial}{B}
\newcommand{\labels@specialunitsInitial}{S}
\newcommand{\labels@rareunitsInitial}{R}
\newcommand{\labels@mountsInitial}{M}


% Titlepage

\newcommand{\labels@fantasybattles}{Batailles Fantastiques}
\newcommand{\labels@NinthAge}{Le 9\ieme Âge}
\newcommand{\labels@creators}{Une collaboration des créateurs de l'ETC et du Swedish Comp System}
\newcommand{\labels@introduction}{%
\noindent {\Largerfontsize\textbf{Note des traducteurs}}
\vspace{0.5cm}

Nous souhaitons remercier chaleureusement l'équipe à l'initiative du 9\ieme Âge pour leur motivation et leur travail continu pour faire vivre notre passion. Nous espérons que ce jeu saura développer les qualités pour plaire au plus grand nombre et réunir les joueurs, amateurs comme habitués des tournois, autour de règles amusantes et équilibrées, pour finalement s'imposer comme un standard du jeu de figurines. Une grande ambition qui ne pourra s'accomplir que \textbf{grâce à vous}, la communauté, via des retours constructifs, afin de modeler le jeu selon nos désirs. N'étant \textbf{en aucun cas à but lucratif}, le 9\ieme Âge part avec un avantage considérable. Les règles des éventuelles nouvelles sorties ne seront pas dictées par le besoin de vendre ces nouveautés. Vous pouvez choisir et acheter vos figurines où bon vous semble, il n'y a pas un unique revendeur toléré. Vous n'êtes pas bloqués dans une spirale infernale où pour continuer à jouer à un jeu, dans lequel vous vous êtes tant investis, vous devez payer toujours plus cher pour entretenir votre collection. Enfin, vous pouvez être assurés que tant que 9\ieme Âge sera joué, vous disposerez d'un \textbf{support continu et régulier}, celui-ci étant offert par la communauté.

Nous attirons votre attention sur le fait que ce jeu en est encore à ses débuts et dans un \textbf{stade de développement}. Ce document correspond à une version de brouillon \textbf{\og{} beta \fg{}}, dont le but et de tester le jeu et le modifier jusqu'à atteindre une version satisfaisante. Attendez-vous donc à trouver des déséquilibres, des incohérences, et à obtenir des mises à jour régulières avec éventuellement des changements importants. N'hésitez pas à nous donner vos avis ! Ce livre d'armée n'est utilisable qu'en compagnie du livre de Règles et du livre de Magie.

Concernant la traduction en elle-même, nous avons fait de notre mieux pour vous offrir une version de qualité, dont nous espérons qu'elle surpasse celle de la version originale ! Si vous constatez des coquilles, des erreurs, merci de nous les signaler en nous contactant sur le forum du 9\ieme Âge, dans le \textbf{sous-forum français} (\url{http://www.the-ninth-age.com/index.php?board/117-french/}). Vous y trouverez aussi les dernières mises à jour. \textbf{En cas de conflit d'interprétation avec la version originale, la version originale fait référence}.

\vspace{0.5cm}
Que ce jeu vous apporte d'innombrables heures de plaisir partagé !

\vspace{0.7cm}
\noindent {\Largerfontsize\textbf{Les traducteurs}}
\vspace{0.1cm}

\ifdef{\translationteam}{
	\begin{multicols}{3}
	\begin{itemize}
		\translationteam
	\end{itemize}
	\end{multicols}
}{}
}
\newcommand{\labels@secondpageannouncement}{%
	\labels@fantasybattles{} : \labels@NinthAge{} est un jeu créé et entretenu par la communauté qui met en scène des affrontements de figurines. Toutes les règles sont disponibles gratuitement sur le site suivant. Vos retours et suggestions sont les bienvenus.
	\newline\url{http://www.the-ninth-age.com/}
}
\newcommand{\labels@rulechanges}{%
	Les changements de règles entre versions sont colorés comme ce paragraphe. Une liste en anglais de ces changements par version est ajoutée à la fin de cet ouvrage.
}
\newcommand{\labels@latexcredit}{Document réalisé à l'aide de \LaTeX .}


%%% Technical commands

\newcommand{\only}[1]{(#1 uniquement)}
\newcommand{\free}{gratuit}
\newcommand{\upto}{jusqu'à}
\newcommand{\Upto}{Jusqu'à}
\newcommand{\unlimited}{sans limite de pts}
\newcommand{\permodel}{/fig.}
\newcommand{\listlastchoice}{ ou}
\newcommand{\notif}[1]{(pas #1)}
\newcommand{\wordand}{et}
\newcommand{\wordwith}{avec}
\newcommand{\ifNmodelsorless}[1]{(#1 figurines ou moins)}
\newcommand{\unitwith}{unité avec}
\newcommand{\From}{De} % From ... to ... models
\newcommand{\wordto}{à}
\newcommand{\wordAll}{Tous}
\newcommand{\spacebeforecolon}{ } % French put a space before colons
\newcommand{\minprice}{Coût min. :}
\newcommand{\mincostfor}{Coût min. pour}
\newcommand{\maxunitsize}{Taille max.}
\newcommand{\additionalfigscost}{Les figurines additionnelles coûtent}


%%% Special rules %%%

\newcommand{\ambush}{Embuscade}
\newcommand{\armourpiercing}[1]{Perforant\ifblank{#1}{}{ (#1)}}
\newcommand{\bodyguard}[1]{Garde du Corps\ifblank{#1}{}{ (#1)}}
\newcommand{\breathweapon}[1]{Attaque de Souffle\ifblank{#1}{}{ (#1)}}
\newcommand{\channel}{Canalisation}
\newcommand{\crushattack}{Attaque Écrasante}
\newcommand{\devastatingcharge}{Charge Dévastatrice}
\newcommand{\distracting}{Distrayant}
\newcommand{\engineer}{Ingénieur}
\newcommand{\ethereal}{Éthéré}
\newcommand{\fastcavalry}{Cavalerie Légère}
\newcommand{\fear}{Peur}
\newcommand{\fightinextrarank}{Combat avec un Rang Supplémentaire}
\newcommand{\fireborn}{Né du Feu}
\newcommand{\flamingattacks}{Attaques Enflammées}
\newcommand{\flammable}{Inflammable}
\newcommand{\lighttroops}{Troupes Légères}
\newcommand{\frenzy}{Frénésie}
\newcommand{\fly}[1]{Vol\ifblank{#1}{}{ (#1)}}
\newcommand{\grindingattacks}[1]{Attaques de Broyage\ifblank{#1}{}{ (#1)}}
\newcommand{\hardtarget}{Camouflé}
\newcommand{\hatred}{Haine}
\newcommand{\hellfire}{Flammes de l'Enfer}
\newcommand{\hidden}{Caché}
\newcommand{\holyattacks}{Attaques Divines}
\newcommand{\immunetopsychology}{Immunisé à la Psychologie}
\newcommand{\impacthits}[1]{Touches d'Impact\ifblank{#1}{}{ (#1)}}
\newcommand{\insignificant}{Insignifiant}
\newcommand{\largetarget}{Grande Cible}
\newcommand{\lethalstrike}{Coup Fatal}
\newcommand{\lightningattacks}{Attaques Foudroyantes}
\newcommand{\lightningreflexes}{Réflexes Foudroyants}
\newcommand{\magicresistance}[1]{Résistance à la Magie\ifblank{#1}{}{ (#1)}}
\newcommand{\magicalattacks}{Attaques Magiques}
\newcommand{\metalshifting}{Fusion du Métal}
\newcommand{\moveorfire}{Mouvement ou Tir}
\newcommand{\multipleshots}[1]{Tirs Multiples\ifblank{#1}{}{ (#1)}}
\newcommand{\multiplewounds}[2]{Blessures Multiples\ifblank{#1}{}{ (#1\ifblank{#2}{)}{, #2)}}}
\newcommand{\notaleader}{Pas un Meneur}
\newcommand{\otherworldly}{D'Outre-Monde}
\newcommand{\pathmaster}[1]{Maître de la Discipline\ifblank{#1}{}{ (#1)}}
\newcommand{\poisonedattacks}{Attaques Empoisonnées}
\newcommand{\quicktofire}{Tir Rapide}
\newcommand{\randommovement}[1]{Mouvement Aléatoire\ifblank{#1}{}{ (#1)}}
\newcommand{\randomattacks}[1]{Attaques Aléatoires\ifblank{#1}{}{ (#1)}}
\newcommand{\regeneration}[1]{Régénération\ifblank{#1}{}{ (#1+)}}
\newcommand{\reload}{Rechargez !}
\newcommand{\requirestwohands}{Arme à deux Mains}
\newcommand{\scythes}{Faux}
\newcommand{\scout}{Éclaireur}
\newcommand{\scouts}{Éclaireurs}
\newcommand{\stomp}[1]{Piétinement\ifblank{#1}{}{ (#1)}}
\newcommand{\strider}[1]{Guide\ifblank{#1}{}{ (#1)}}
\newcommand{\stubborn}{Tenace}
\newcommand{\stupidity}{Stupidité}
\newcommand{\skirmisher}{Tirailleur}
\newcommand{\skirmishers}{Tirailleurs}
\newcommand{\sweepingattack}{Attaque au Passage}
\newcommand{\swiftstride}{Rapide}
\newcommand{\thunderouscharge}{Charge Tonitruante}
\newcommand{\terror}{Terreur}
\newcommand{\toxicattacks}{Attaques Toxiques}
\newcommand{\unbreakable}{Indémoralisable}
\newcommand{\undead}{Mort-Vivant}
\newcommand{\unstable}{Instable}
\newcommand{\unwieldy}{Encombrant}
\newcommand{\vanguard}{Avant-Garde}
\newcommand{\volleyfire}{Tir de Volée}
\newcommand{\warplatform}{Plateforme de Guerre}
\newcommand{\wardsave}[1]{Sauvegarde Invulnérable\ifblank{#1}{}{ (#1+)}}
\newcommand{\weaponmaster}{Maître d'Ar\-mes}
\newcommand{\wizardconclave}[1]{Conclave de Sorciers\ifblank{#1}{}{ (#1)}}


%%% Magic %%%

\newnamemacro{\Pathof}{Discipline}

\newcommand{\battle}{Commune}
\newcommand{\alchemy}{de l'Alchimie}
\newcommand{\death}{de la Mort}
\newcommand{\fire}{du Feu}
\newcommand{\heavens}{des Cieux}
\newcommand{\light}{de la Lumière}
\newcommand{\nature}{de la Nature}
\newcommand{\shadows}{des Ombres}
\newcommand{\wilderness}{de la Sauvagerie Bestiale}
\newcommand{\butchery}{de la Boucherie}
\newcommand{\change}{du Changement}
\newcommand{\thebiggreengods}{des Grands Dieux Verts}
\newcommand{\thelittlegreengods}{des Petits Dieux Verts}
\newcommand{\blackmagic}{de la Magie Noire}
\newcommand{\disease}{de la Maladie}
\newcommand{\lust}{de la Luxure}
\newcommand{\necromancy}{de la Nécromancie}
\newcommand{\ruin}{de la Ruine}
\newcommand{\forge}{de la Forge}
\newcommand{\sands}{des Sables}
\newcommand{\whitemagic}{de la Magie Blanche}

\newcommand{\anyofthebattlemagic}{dans n'importe laquelle des Disciplines Communes}

\newcommand{\magiclevel}[1]{\ifnumcomp{#1}{<}{3}{Sorcier Apprenti}{Maître Sorcier} Niveau #1}
\newcommand{\Level}{Niveau}

\newcommand{\wizard}{Sorcier}
\newcommand{\wizards}{Sorciers}

\newcommand{\boundspell}[1]{Objet de Sort, Puissance #1}


%%% Other rules %%%

\newcommand{\armoursave}{Sauvegarde d'Armure}
\newcommand{\firstinrank}{Au Premier Rang}
\newcommand{\hardcover}{Couvert Lourd}
\newcommand{\holdyourground}{Tenez les Rangs}
\newcommand{\inspiringpresence}{Présence Charismatique}
\newcommand{\lightcover}{Couvert Léger}
\newcommand{\monstrousrank}{Rang Monstrueux}
\newcommand{\ordnance}{Artillerie}
\newcommand{\parry}{Parade}
\newcommand{\raisewounds}{Ressusciter des Figurines}
\newcommand{\recoverwounds}{Récupérer des PVs}
\newcommand{\aideddispel}{Dissipation Assistée}
\newcommand{\rnf}{ordinaires}
\newcommand{\general}{Général}


%%% Equipment %%%

\newcommand{\innatedefence}[1]{Protection Innée\ifblank{#1}{}{~(#1+)}}
\newcommand{\mountsprotection}[1]{Protection de Monture\ifblank{#1}{}{~(#1+)}}
\newcommand{\la}{Armure Légère}
\newcommand{\ha}{Armure Lourde}
\newcommand{\platearmour}{Armure de Plates}
\newcommand{\hw}{Arme de Base}
\newcommand{\pw}{Paire d'Armes}
\newcommand{\spear}{Lance}
\newcommand{\halberd}{Hallebarde}
\newcommand{\gw}{Arme Lourde}
\newcommand{\lance}{Lance de Cavalerie}
\newcommand{\lightlance}{Lance Légère}
\newcommand{\shield}{Bouclier}
\newcommand{\barding}{Caparaçon}
\newcommand{\throwingweapons}{Armes de Jet}
\newcommand{\shortbow}{Arc Court}
\newcommand{\flail}{Fléau}

\newcommand{\cannon}{Canon}
\newcommand{\catapult}{Catapulte}
\newcommand{\volleygun}{Batterie de Tir}
\newcommand{\boltthrower}{Baliste}
\newcommand{\artilleryweapon}{Arme d'Artillerie}


%%% Troop types %%%

\newcommand{\characters}{Personnages}
\newcommand{\infantry}{Infanterie}
\newcommand{\monstrousinfantry}{Infanterie Monstrueuse}
\newcommand{\cavalry}{Cavalerie}
\newcommand{\monstrouscavalry}{Cavalerie Monstrueuse}
\newcommand{\swarm}{Nuée}
\newcommand{\swarms}{Nuées}
\newcommand{\warbeast}{Bête de Guerre}
\newcommand{\warbeasts}{Bêtes de Guerre}
\newcommand{\monster}{Monstre}
\newcommand{\monsters}{Monstres}
\newcommand{\monstrousbeast}{Bête Monstrueuse}
\newcommand{\monstrousbeasts}{Bêtes Monstrueuses}
\newcommand{\chariot}{Char}
\newcommand{\chariots}{Chars}
\newcommand{\riddenmonster}{Monstre Monté}
\newcommand{\riddenmonsters}{Monstres Montés}
\newcommand{\warmachine}{Machine de Guerre}
\newcommand{\warmachines}{Machines de Guerre}


%%% Terrain %%%

\newcommand{\water}{Eaux peu profondes}


%%% Profile wording

\newcommand{\oneofakind}{Uni\-que}
\newcommand{\onechoiceonly}{(un seul choix)}
\newcommand{\onfootonly}{(à pied seulement)}
\newcommand{\closecombatonly}{seulement au Corps à Corps}
\newcommand{\Xmodelsorless}[1]{(#1 figurines ou moins)}
\newcommand{\magicalitemsallowance}{Peut prendre des Objets Magiques}
\newcommand{\magicalweaponallowance}{Peut prendre une Arme Magique}
\newcommand{\notmagicalarmour}{(mais pas d'Armure Magique)}
\newcommand{\anyofthefollowing}{\optionschoice{Peut prendre :}}
\newcommand{\weapononechoice}{\optionschoice{Peut prendre une arme \onechoiceonly{} :}}
\newcommand{\weaponschoice}{\optionschoice{Peut prendre des armes :}}
\newcommand{\shootingweapononechoice}{\optionschoice{Peut prendre une arme de tir \onechoiceonly{} :}}
\newcommand{\combatweapononechoice}{\optionschoice{Peut prendre une arme de corps à corps \onechoiceonly{} :}}
\newcommand{\armouronechoice}{\optionschoice{Peut prendre une armure \onechoiceonly{} :}}
\newcommand{\magiclevelchoice}{\optionschoice{Peut devenir au choix :}}
\newcommand{\bsboption}{Peut devenir Porteur de la Grande Bannière}
\newcommand{\mayupgradeto}{Peut être amélioré en}
\newcommand{\mustbecomeoneofthefollowing}{\optionschoice{Doit devenir un choix parmi :}}
\newcommand{\maybecomeoneofthefollowing}{\optionschoice{Peut devenir un choix parmi :}}
\newcommand{\maytakeoneofthefollowing}{\optionschoice{Peut prendre un choix parmi :}}
\newcommand{\maytakeuptotwoofthefollowing}{\optionschoice{Peut prendre jusqu'à deux choix parmi :}}
\newcommand{\maygain}{Peut gagner la règle}
\newcommand{\maytake}{Peut prendre}
\newcommand{\maytakeashield}{Peut prendre un Bouclier}
\newcommand{\maytakela}{Peut prendre une Armure Légère}
\newcommand{\maytakeha}{Peut prendre une Armure Lourde}
\newcommand{\maytakemountsprotectionX}[1]{Peut prendre une \mountsprotection{#1}}
\newcommand{\maytakeagw}{Peut prendre une Arme Lourde}
\newcommand{\maytakeaspear}{Peut prendre une Lance}
\newcommand{\maytakepw}{Peut prendre une Paire d'Armes}
\newcommand{\maytakethrowingweapons}{Peut prendre des Armes de Jet}
\newcommand{\maytakebarding}{Peut prendre un Caparaçon}
\newcommand{\replaceshieldwithhalberd}{Remplacer le Bouclier par une Hallebarde}
\newcommand{\maybecome}{Peut devenir}

\newcommand{\maytakeonechoiceonly}{\optionschoice{\maytake{} \onechoiceonly{}\spacebeforecolon{}:}}

\newcommand{\mountssectionannouncement}{%
La section Montures concerne les montures de Personnages. Les montures pour non-Personnages suivent les règles données dans leur description d'unité.
}

%%% Commands to handle strings, better than xstring to handle commands inside the strings %%%

\newcommand{\substitute}[3]{%
  \protected@edef\sub@temp{#1}%
  \saveexpandmode
  \expandarg\StrSubstitute{\sub@temp}{#2}{#3}[#1]%
  \restoreexpandmode
}

\newcommand{\splitatstar}[3]{%
  \protected@edef\split@temp{#1}%
  \saveexpandmode
  \expandarg\StrCut{\split@temp}{*}#2#3%
  \restoreexpandmode
}

\newcommand{\splitatinf}[3]{%
  \protected@edef\split@temp{#1}%
  \saveexpandmode
  \expandarg\StrCut{\split@temp}{<}#2#3%
  \restoreexpandmode
}

\newcommand{\splitatequal}[3]{%
  \protected@edef\split@temp{#1}%
  \saveexpandmode
  \expandarg\StrCut{\split@temp}{=}#2#3%
  \restoreexpandmode
}

\newcommand{\ifsubstring}[4]{%
  \protected@edef\split@temp{#1}%
  \protected@edef\split@tempbis{#2}%
  \saveexpandmode
  \expandarg\IfSubStr{\split@temp}{\split@tempbis}{#3}{#4}%
  \restoreexpandmode
}

\def\removespaces#1{\zap@space#1 \@empty}

%%% Commands for alphabetical ordering %%%

\newcommand{\sortitem}[2][\relax]{%
	\DTLnewrow{list}% Create a new entry
	\ifx#1\relax%
		\DTLnewdbentry{list}{sortlabel}{#2}% Add entry sortlabel (no optional argument)
	\else%
		\DTLnewdbentry{list}{sortlabel}{#1}% Add entry sortlabel (optional argument)
	\fi%
		\DTLnewdbentry{list}{description}{#2}% Add entry description
}
\newenvironment{sortedlist}{%
	\DTLifdbexists{list}{\DTLcleardb{list}}{\DTLnewdb{list}}% Create new/discard old list
}{%
	\DTLsort{sortlabel}{list}% Sort list
	\begin{itemize*}[label={}, itemjoin={,}]%
		\DTLforeach*{list}{\theDesc=description}{%
		\item\theDesc}% Print each item
	\end{itemize*}%
}

\pdfstringdefDisableCommands{\def\textcolor#1{}}

% See language specific file for \addtosortedlist

%%% Database for automatic Quick Ref Sheet %%%

\DTLnewdb{profiles} % Database containing name, category, multiprofile number, profilename (if multi), caraclist, trooptype, invocation for CV.
\newcommand{\profilecategory}{\labels@lords} % Will be updated in relevant categories

\newcommand{\profiledtbfillname}[1]{\DTLnewdbentry{profiles}{name}{#1}}
\newcommand{\profiledtbfillcategory}[1]{\DTLnewdbentry{profiles}{category}{#1}}
\newcommand{\profiledtbfilltrooptype}[1]{\DTLnewdbentry{profiles}{trooptype}{#1}}
\newcommand{\profiledtbfillinvocation}[1]{\DTLnewdbentry{profiles}{invocation}{#1}}
\newcommand{\profiledtbfillprofile}[1]{\DTLnewdbentry{profiles}{profile}{#1}}
\newcommand{\profiledtbfillmultipleprofile}[1]{\DTLnewdbentry{profiles}{multipleprofile}{#1}}

\newcommand{\void}[1]{}
\newcounter{multiprofilecounter}

\newcommand{\profiledtbfillcarac}[1]{%
	\profiledtbfillprofile{#1}
	\parselist{#1}{\locallists@profileslist}% Split of the different profiles in the case of a multiprofile.
	\setcounter{multiprofilecounter}{0}%
	\forlistloop{\stepcounter{multiprofilecounter}\void}{\locallists@profileslist}%
	\expandafter\profiledtbfillmultipleprofile\expandafter{\number\value{multiprofilecounter}}
}


%%% Technical commands %%%

\newcommand{\newrule}{\textcolor{green!50!black}}
\newcommand{\removedrule}[1]{\textcolor{green!50!black}{\sout{#1}}}
\newcommand{\starsymbol}{$\star$}
\newcommand{\refsymbol}{$^\star$}

\newcommand{\inch}{\arcsecond}
\newcommand{\foot}{\arcminute}
\newcommand{\range}[1] {\labels@range~\unit{#1}{\inch}}
\newcommand{\distance}[1] {\unit{#1}{\inch}}
\newcommand{\result}[1] {\texttt{'}#1\texttt{'}}


%%% Fonts and sizes %%%

\newcommand{\bigtitle}[1]{\vspace*{-1.5cm}\section*{}\noindent\begin{center}\Hugefontsize\textbf{\antiquefont\expandafter\uppercase\expandafter{#1}}\end{center}}

\newcommand{\subtitle}[1]{\subsection*{}\noindent{\hugefontsize\antiquefont #1}}

\newcommand{\subsubtitle}[1]{\subsubsection*{}\noindent{\Largerfontsize\antiquefont #1}}

\newcommand{\verysmallfontsize}{\fontsize{4}{4.8}\selectfont}
\newcommand{\smallfontsize}{\fontsize{6}{7.2}\selectfont}
\newcommand{\normalfontsize}{\fontsize{8}{9.6}\selectfont}
\newcommand{\largefontsize}{\fontsize{10}{12}\selectfont}
\newcommand{\largerfontsize}{\fontsize{12}{14.4}\selectfont}
\newcommand{\Largefontsize}{\fontsize{14}{16.8}\selectfont}
\newcommand{\Largerfontsize}{\fontsize{15}{18}\selectfont}
\newcommand{\hugefontsize}{\fontsize{18}{21.6}\selectfont}
\newcommand{\Hugefontsize}{\fontsize{25}{30}\selectfont}

\newcommand{\unitentryformat}[1]{\textit{\largefontsize{#1}}}
\newcommand{\textIT}[1]{\textit{\largefontsize{#1}}}


%%% Titles %%%

\newcommand{\lordstitle}{\def\logolocalpath{../Layout/pics/logo_lord.png}\bigtitle{\labels@lords}}
\newcommand{\heroestitle}{%
\def\logolocalpath{../Layout/pics/logo_hero.png}%
\clearpage\bigtitle{\labels@heroes}%
\renewcommand{\profilecategory}{\labels@heroes}%
}
\newcommand{\coreunitstitle}{%
\def\logolocalpath{../Layout/pics/logo_core.png}%
\clearpage\bigtitle{\labels@coreunits}%
\renewcommand{\profilecategory}{\labels@coreunits}%
}
\newcommand{\specialunitstitle}{%
\def\logolocalpath{../Layout/pics/logo_special.png}%
\clearpage\bigtitle{\labels@specialunits}%
\renewcommand{\profilecategory}{\labels@specialunits}%
}
\newcommand{\rareunitstitle}{%
\def\logolocalpath{../Layout/pics/logo_rare.png}%
\clearpage\bigtitle{\labels@rareunits}%
\renewcommand{\profilecategory}{\labels@rareunits}%
}
\newcommand{\mountstitle}{%
\def\logolocalpath{../Layout/pics/logo_mount.png}%
\clearpage\bigtitle{\labels@charactermounts}%
\renewcommand{\profilecategory}{\labels@mounts}%
}

\newcommand{\startarmywiderules}{\newpage\bigtitle{\labels@armywiderules}\largefontsize}
\newcommand{\closearmywiderules}{\normalfontsize}
\newcommand{\armywideruleentry}[1]{\subtitle{#1}\vspace{5pt}}

\newcommand{\startarmyspecialrules}{\bigtitle{\labels@armyspecialrules}\largefontsize}
\newcommand{\closearmyspecialrules}{\normalfontsize}
\newcommand{\armyspecialruleentry}[1]{\subtitle{#1}\vspace{5pt}}

\newcommand{\startarmyarmoury}{\bigtitle{\labels@armoury}\largefontsize\subtitle{}}
\newcommand{\closearmyarmoury}{\normalfontsize}

\newcommand{\startarmymagicalitems}{\newpage\largefontsize\bigtitle{\labels@magicalitems}\begin{multicols}{2}\raggedcolumns}
\newcommand{\closearmymagicalitems}{\end{multicols}\normalfontsize}

\newcommand{\armymagicalweapons}{\subtitle{\labels@magicalweapons}}
\newcommand{\armymagicalarmour}{\subtitle{\labels@magicalarmour}}
\newcommand{\armytalismans}{\subtitle{\labels@talismans}}
\newcommand{\armyenchanteditems}{\subtitle{\labels@enchanteditems}}
\newcommand{\armyarcaneitems}{\subtitle{\labels@arcaneitems}}
\newcommand{\armymagicalbanners}{\subtitle{\labels@magicalbanners}}

\newcommand{\startarmynewsection}[1]{\newpage\bigtitle{#1}\largefontsize}
\newcommand{\startarmynewsectionSP}[1]{\vspace{1.5cm}\bigtitle{#1}\largefontsize}
\newcommand{\closearmynewsection}{\normalfontsize}

\newcommand{\armynewsubsection}[1]{\subtitle{#1}\vspace{5pt}}
\newcommand{\armynewsubsubsection}[1]{\subsubtitle{#1}\vspace{3pt}}

\newcommand{\armylist}{\clearpage}

\newcommand{\quickrefsheettitle}{\clearpage\newgeometry{top=1.6cm, bottom=2cm, left=1cm, right=1cm}\bigtitle{\labels@quickrefsheet}\vspace*{0.4cm}}
\newcommand{\changelogtitle}{\clearpage\bigtitle{\labels@changelog}\spaceaftersection{}}

\newcommand{\spaceaftersection}{\vspace{0.8cm}}

\newcommand{\separator}{\noindent\begin{center}\textcolor{black!30}{\rule{0.7\columnwidth}{2pt}}\end{center}}


%%% Custom lists and description for first sections of the army books

\newcommand{\startpricelist}{\begin{samepage}\begin{description}[leftmargin=0.3cm, labelindent=0cm, labelsep=0.1cm]}
\def\endpricelist{\end{description}\end{samepage}}
\newcommand{\pricelistitem}[2]{\item \option{\textbf{#1}}{#2}\newline}

\newcommand{\startpricelistNSP}{\begin{description}[leftmargin=0.3cm, labelindent=0cm, labelsep=0.1cm]}
\def\endpricelistNSP{\end{description}}

\newcommand{\startitemlist}{\begin{multicols}{2}\raggedcolumns\begin{description}[leftmargin=0.3cm, labelindent=0cm, labelsep=0.1cm]}
\def\enditemlist{\end{description}\end{multicols}}
\newcommand{\listitem}[1]{\item[#1\spacebeforecolon{}:]}

\newcommand{\startitemlistonecol}{\begin{description}[leftmargin=0.3cm, labelindent=0cm, labelsep=0.1cm]}
\def\enditemlistonecol{\end{description}}
\newcommand{\listitemonecol}[1]{\item \textbf{#1\spacebeforecolon{}:}\newline}

\newenvironment{customitemize}{\begin{description}[leftmargin=0.3cm, labelindent=0cm, labelsep=0cm]}{\end{description}}
\newenvironment{customsubitemize}{\begin{itemize}[label={-}, labelsep=0.1cm, topsep=0cm, parsep=0cm, itemsep=0cm, leftmargin=0.4cm, labelindent=0cm]}{\end{itemize}}

%%% Table parameters %%%

\newcolumntype{M}[1]{>{\centering\let\newline\\\arraybackslash\hspace{0pt}}m{#1}}


%%%  Lists handling %%%

\newcommand{\addlocallist}{\listadd\locallists@dummy}%
\NewDocumentCommand{\parsespacelist}{>{\SplitList{ }} m }{%
	\ProcessList{#1}{\addlocallist}%
}%
\NewDocumentCommand{\parsecommalist}{>{\SplitList{,}} m }{%
	\ProcessList{#1}{\addlocallist}%
}%
\newcommand{\parselist}[3][,]{%
	\renewcommand\addlocallist{\listadd#3}%
  	\undef#3%
  	\ifstrequal{#1}{ }{\parsespacelist{#2}}{\parsecommalist{#2}}%
}


%%% Profiles handling %%%

% Element of a table that contains the characteristics of a model (or part of a model)
\newcommand\caraclist[1]{
	\parselist[ ]{#1}{\locallists@caraclist}%
	\forlistloop{&}{\locallists@caraclist}%
}

\newcommand\caraclistbold[1]{
	\parselist[ ]{#1}{\locallists@caraclist}%
	\forlistloop{&\bfseries}{\locallists@caraclist}%
}

% Line of a profile table, including bottom line. It is meant to contain the name of the model (or part), its characteristics (preferably, the second argument should contain the \carac macro), troop type and base size.
\newcommand{\profilefirstline}[4]{#1 & #2 &   & #3 & #4 }

% Start of a profile table. Includes the table commands, and the column labels. \profilecellsize is the size of the characteristics cells in the profile.
\newcommand{\profilecellsize}{0.56cm}
\newcommand{\profilestart}{%
	\noindent %
	\begin{tabular}{@{}p{3cm}@{}M{\profilecellsize}@{}M{\profilecellsize}@{}M{\profilecellsize}@{}M{\profilecellsize}@{}M{\profilecellsize}@{}M{\profilecellsize}@{}M{\profilecellsize}@{}M{\profilecellsize}@{}M{\profilecellsize}@{}p{2.7cm}@{}p{3.3cm}@{}p{2cm}@{}}%
	 &% \textbf{\labels@profile}
	\labels@M & \labels@WS & \labels@BS & \labels@S & \labels@T & \labels@W & \labels@I & \labels@A & \labels@Ld &%
	&%
	{\unitentryformat{\labels@trooptype}} &%
	{\unitentryformat{\labels@basesize}}%
}

% End of a profile table.
\newcommand{\profileend}{\end{tabular}}

% Algorithm to automatically use and fill previous command, with coherence check.
\providebool{profilefirst}
\newcommand{\profileitem}[1]{%
	\tabularnewline%
	\splitatinf{#1}\local@unitname\local@unitprofile%
	\local@unitname \expandafter\caraclistbold\expandafter{\local@unitprofile}%
	&%
	& \ifbool{profilefirst}{\unit@type}{}%
	& \ifbool{profilefirst}{%
		\ifsubstring{\unit@basesize}{x}{% Rectangular base
			\unit{\unit@basesize}{\milli\meter}%
		}{% Circular base
			\unit{\unit@basesize}{\milli\meter} \labels@roundbase%
		}%
	}{}%
	\global\boolfalse{profilefirst}%
}
\newcommand{\profile}[1]{%
	\parselist{#1}{\locallists@profileslist}%
	\profilestart%
	\global\booltrue{profilefirst}%
	\forlistloop{\profileitem}{\locallists@profileslist}%
	\profileend%
}


%%% Profiles handling in case of invocation %%%

\newcommand{\invocprofilestart}{%
	\noindent %
	\begin{tabular}{@{}p{3cm}@{}M{\profilecellsize}@{}M{\profilecellsize}@{}M{\profilecellsize}@{}M{\profilecellsize}@{}M{\profilecellsize}@{}M{\profilecellsize}@{}M{\profilecellsize}@{}M{\profilecellsize}@{}M{\profilecellsize}@{}M{2.2cm}@{}p{0.5cm}@{}p{3.3cm}@{}p{2cm}@{}}%
	 &% \textbf{\labels@profile}
	\labels@M & \labels@WS & \labels@BS & \labels@S & \labels@T & \labels@W & \labels@I & \labels@A & \labels@Ld & \unitentryformat{\labels@Invocation} &%
	&%
	{\unitentryformat{\labels@trooptype}} &%
	{\unitentryformat{\labels@basesize}}%
}

\newcommand{\invocprofileitem}[1]{%
	\tabularnewline%
	\splitatinf{#1}\local@unitname\local@unitprofile%
	\local@unitname \expandafter\caraclistbold\expandafter{\local@unitprofile}%
	& \ifbool{profilefirst}{\unit@invocation}{} &%
	& \ifbool{profilefirst}{\unit@type}{}%
	& \ifbool{profilefirst}{\unit{\unit@basesize}{\milli\meter}}{}%
	\global\boolfalse{profilefirst}%
}

\newcommand{\invocprofile}[1]{%
	\parselist{#1}{\locallists@profileslist}%
	\invocprofilestart%
	\global\booltrue{profilefirst}%
	\forlistloop{\invocprofileitem}{\locallists@profileslist}%
	\profileend%
}


%%%%%%%%%%%%%%%%%%
%%% Unit rules %%%
%%%%%%%%%%%%%%%%%%

%%% Entry title command %%%

\newcommand{\unitentry}[2]{\ifdefempty{#1}{}{\noindent #2}}


%%% Special rules %%%

% Special rules listing for a unit, with alphabetical order.
\newcommand{\ruleslist}[1]{%
	\parselist[,]{#1}{\locallists@ruleslist}%
	\begin{sortedlist}%
		\forlistloop{\addtosortedlist}{\locallists@ruleslist}%
	\end{sortedlist}%
}

% Special rules entry.
\newcommand{\specialrules}[1]{\unitentry{#1}{\unitentryformat{\labels@specialrules\spacebeforecolon{}:}\newline\hspace*{-\fontdimen2\font}\expandafter\ruleslist\expandafter{#1}.}}
\newcommand{\commonspecialrules}[2]{\unitentry{#2}{\unitentryformat{#1\spacebeforecolon{}:}\newline\hspace*{-\fontdimen2\font}\expandafter\ruleslist\expandafter{#2}.}}


%%% Magical abilities %%%

% Paths listing for a unit.
\newcommand{\pathslist}[1]{%
	\parselist[,]{#1}{\locallists@pathslist}%
	\begin{itemize*}[label={}, itemjoin={,}, itemjoin*={\listlastchoice}]%
		\forlistloop{\item}{\locallists@pathslist}%
	\end{itemize*}%
}

% Magic entry.
\newcommand{\magic}[2]{\unitentry{#2}{\unitentryformat{\labels@magic\spacebeforecolon{}: }\newline\ifdefempty{#1}{}{\textbf{\magiclevel{#1}}. }\labels@pathsused\expandafter\pathslist\expandafter{#2}.}}

% Wizard Conclave.
\newcommand{\magicwizardconclave}[1]{\unitentry{#1}{\unitentryformat{\labels@magic\spacebeforecolon{}: }\newline\textbf{\wizardconclave{}}\spacebeforecolon{}: #1.}}


%%% Equipment %%%

% Equipment listing.
\newcommand{\equipmentlist}[1]{%
	\parselist[,]{#1}{\locallists@equipmentlist}%
	\begin{sortedlist}%
		\forlistloop{\addtosortedlist}{\locallists@equipmentlist}%
	\end{sortedlist}%
}

% Equipment entry.
\newcommand{\weapons}[1]{\unitentry{#1}{\unitentryformat{\labels@weapons\spacebeforecolon{}:}\newline\hspace*{-\fontdimen2\font}\expandafter\equipmentlist\expandafter{#1}.}}

\newcommand{\armour}[1]{\unitentry{#1}{\unitentryformat{\labels@armour\spacebeforecolon{}:}\newline\hspace*{-\fontdimen2\font}\expandafter\equipmentlist\expandafter{#1}.}}


%%% Alignment %%%

\newcommand{\alignment}[1]{\unitentry{#1}{\unitentryformat{\labels@alignment\spacebeforecolon{}:}\newline\textbf{#1}.}}

%%% Green Hide Race %%%

\newcommand{\greenhideraceentry}[1]{\unitentry{#1}{\unitentryformat{\labels@greenhiderace\spacebeforecolon{}:}\newline\textbf{#1}.}}


%%% Options %%%

% Frame commands.
\newcommand{\optionsframestart}{\begin{innerframe}[\labels@options]}
\newcommand{\optionsframeend}{\end{innerframe}}

% Options listing.
\newcommand{\optionslist}[1]{%
	\parselist[,]{#1}{\locallists@optionslist}%
	\begin{description}[leftmargin=0.3cm, labelindent=0cm, labelsep=0cm, itemsep=0cm, parsep=0cm]%
		\forlistloop{\item\setoption}{\locallists@optionslist}%
	\end{description}%
}

% Options entry.
\newcommand{\options}[1]{\ifdefempty{#1}{}{\optionsframestart\vspace*{-0.4cm}\unitentry{#1}{\expandafter\optionslist\expandafter{#1}}\optionsframeend}}

% Option specific commands.
\newcommand{\setoption}[1]{%
	\noexpandarg\StrCut{#1}{=}\optiontext\optionvalue%
	\expandafter\ifstrequal\expandafter{\optionvalue}{}{%
		\optiontext%
	}{%
	\ifsubstring{\optionvalue}{\free}{%
		\option[\free]{\optiontext}{\optionvalue}%
	}{%
	\ifsubstring{\optionvalue}{\unlimited}{%
		\option[\unlimited]{\optiontext}{\optionvalue}%
	}{%
	\ifsubstring{\optionvalue}{\upto}{%
		\splitatinf{\optionvalue}\myoption\myvalue%
		\option[\upto]{\optiontext}{\myvalue}%
	}{%
	\ifsubstring{\optionvalue}{\permodel}{%
		\splitatinf{\optionvalue}\myoption\myvalue%
		\option[\permodel]{\optiontext}{\myvalue}%
	}{%
	\ifsubstring{\optionvalue}{\pershadygit}{% For Orcs N Goblins
		\splitatinf{\optionvalue}\myoption\myvalue%
		\option[\pershadygit]{\optiontext}{\myvalue}%
	}{%
	\ifsubstring{\optionvalue}{\permadgit}{% For Orcs N Goblins
		\splitatinf{\optionvalue}\myoption\myvalue%
		\option[\permadgit]{\optiontext}{\myvalue}%
	}{%	
	\ifsubstring{\optionvalue}{\perrune}{% For Dwarven Holds
		\splitatinf{\optionvalue}\myoption\myvalue%
		\option[\perrune]{\optiontext}{\myvalue}%
	}{%	
		\option{\optiontext}{\optionvalue}%
	}}}}}}}}%
}

\newcommand{\option}[3][]{#2\predotfill\dotfill\nobreak%
	% Add \upto token if necessary.
	\ifstrequal{#1}{\upto}{\upto~}{}%
	% The option can be free, have an unlimited cost, or have a points cost.
	\ifstrequal{#1}{\free}{\free}{\ifstrequal{#1}{\unlimited}{\unlimited}{\pts{#3}}}%
	% Add \permodel if necessary.
	\ifstrequal{#1}{\permodel}{\nobreak\permodel}{}%
	% Add \persomething if necessary.
	\ifstrequal{#1}{\pershadygit}{\nobreak\pershadygit}{}% For Orcs N Goblins
	\ifstrequal{#1}{\permadgit}{\nobreak\permadgit}{}% For Orcs N Goblins
	\ifstrequal{#1}{\perrune}{\nobreak\perrune}{}% For Dwarven Holds
}

\newcommand\optionschoice[2]{%
	\parselist[,]{#2}{\locallists@optionschoice}%
	#1%
	\begin{itemize}[label={}, parsep=0cm, labelindent=0cm, labelwidth=0cm, noitemsep, topsep=0em, leftmargin=0.3cm]%
	\forlistloop{\item\setoption}{\locallists@optionschoice}%
	\end{itemize}%
}

\newcommand\optionschoiceTWOCOL[2]{%
	\parselist[,]{#2}{\locallists@optionschoice}%
	#1%
	\begin{itemize}[label={}, parsep=0cm, labelindent=0cm, labelwidth=0cm, noitemsep, topsep=0em, leftmargin=0.3cm]%
	\setlength{\columnseprule}{0.5pt}
	\renewcommand{\columnseprulecolor}{\color{black!30}}
	\vspace*{-5pt}\begin{multicols}{2}\raggedcolumns
	\forlistloop{\item\setoption}{\locallists@optionschoice}%
	\end{multicols}\setlength{\columnseprule}{0pt}
	\end{itemize}%
}

% Option description in army desc.
\newcommand{\optiondef}[3]{\option{\textbf{#1}}{#2}\ifblank{#3}{}{\\{#3}}}


%%% Mount options %%%

% Frame commands.
\newcommand{\mountsframestart}{\begin{innerframe}[\labels@mounts]}
\newcommand{\mountsframeend}{\end{innerframe}}

% Mount listing.
\newcommand{\mountslist}[1]{%
	\parselist[,]{#1}{\locallists@mountslist}%
	\begin{description}[leftmargin=0.3cm, labelindent=0cm, labelsep=0cm, itemsep=0cm, parsep=0cm]%
		\forlistloop{\item\setoption}{\locallists@mountslist}%
	\end{description}%
}

% Mount entry.
\newcommand{\mounts}[1]{\ifdefempty{#1}{}{\mountsframestart\vspace*{-0.4cm}\unitentry{#1}{\expandafter\mountslist\expandafter{#1}}\mountsframeend}}


%%% Command group %%%

% Command group specific commands.
\define@key{commandgroup}{restriction}            {\def\commandgroup@restriction{#1}}
\define@key{commandgroup}{champion}               {\def\commandgroup@champion{#1}}
\define@key{commandgroup}{championallowance}      {\def\commandgroup@championallowance{#1}}
\define@key{commandgroup}{championoption}         {\def\commandgroup@championoption{#1}}
\define@key{commandgroup}{championprerestriction} {\def\commandgroup@championprerestriction{#1}}
\define@key{commandgroup}{championrestriction}    {\def\commandgroup@championrestriction{#1}}
\define@key{commandgroup}{banner}                 {\def\commandgroup@banner{#1}}
\define@key{commandgroup}{bannerallowance}        {\def\commandgroup@bannerallowance{#1}}
\define@key{commandgroup}{veteranstandardbearer}  {\def\commandgroup@veteranstandardbearer{#1}}
\define@key{commandgroup}{singlebannerallowance}  {\def\commandgroup@singlebannerallowance{#1}}
\define@key{commandgroup}{condsinglebannerallowance}  {\def\commandgroup@condsinglebannerallowance{#1}}
\define@key{commandgroup}{banneroption}           {\def\commandgroup@banneroption{#1}}
\define@key{commandgroup}{bannerrestriction}      {\def\commandgroup@bannerrestriction{#1}}
\define@key{commandgroup}{musician}               {\def\commandgroup@musician{#1}}
\define@key{commandgroup}{musicianrestriction}    {\def\commandgroup@musicianrestriction{#1}}
\newcommand{\defcommandgroup}{%
	\setkeys{commandgroup}{restriction=,
	                       champion=, championallowance=, championoption=, championprerestriction=, 
	                       championrestriction=, banner=, bannerallowance=, veteranstandardbearer=, 
	                       singlebannerallowance=, condsinglebannerallowance=, banneroption=, 
	                       bannerrestriction=, musician=, musicianrestriction=}%
	\setkeys{commandgroup}%
}

% Frame commands.
\newcommand{\commandgroupframestart}{\begin{innerframe}[\labels@commandgroup]}
\newcommand{\commandgroupframeend}{\end{innerframe}}

% Command group entry.
\newcommand{\commandgroup}[1]{%
	\defcommandgroup{#1}%
	\ifstrempty{#1}{}{\commandgroupframestart\vspace*{-0.2cm}%
		\begin{description}[leftmargin=0.3cm, labelindent=0cm, labelsep=0cm, itemsep=0cm, parsep=0cm]%
			% Command group title, including restrictions applying to all the command group
			\item \textbf{\expandafter\ifblank\expandafter{\commandgroup@restriction}{}{ \only{\commandgroup@restriction}\spacebeforecolon{}: }} 
			% Champion handling.
			\ifdefempty{\commandgroup@champion}{}{% We have a champion!
			\ifdefempty{\commandgroup@championprerestriction}{% There is no prerestriction to have a champion
				\item \hspace*{-0.04cm}\option{\labels@champion%
					% Possible restrictions to taking a champion
				    \expandafter\ifblank\expandafter{\commandgroup@championrestriction}{}{ \only{\commandgroup@championrestriction}}%
				    % Cost of a champion
				    }{\commandgroup@champion}%
				    % Magical allowance of the champion. Should probably not be used, champion option can do it as well and is more flexible.
					\ifdefempty{\commandgroup@championallowance}{}{\par\option[\upto]{\hspace*{0.3cm}- \labels@championallowance}{\commandgroup@championallowance}}%
					% Any option available to the champion, in the form option:cost
					\ifdefempty{\commandgroup@championoption}{}{%
						\splitatinf{\commandgroup@championoption}\local@option\local@cost%
						\par\option{\hspace*{0.3cm}- \local@option}{\local@cost}}%
			}{% There is a pre-restriction to have a champion
				\item \hspace*{-0.04cm}\commandgroup@championprerestriction	\newline%
				\option{\labels@champion}{\commandgroup@champion}%
				% Magical allowance of the champion. Should probably not be used, champion option can do it as well and is more flexible.
				\ifdefempty{\commandgroup@championallowance}{}{\par\option[\upto]{\hspace*{0.3cm}- \labels@championallowance}{\commandgroup@championallowance}}%
				% Any option available to the champion, in the form option:cost
				\ifdefempty{\commandgroup@championoption}{}{%
					\splitatinf{\commandgroup@championoption}\local@option\local@cost%
					\par\option{\hspace*{0.3cm}- \local@option}{\local@cost}}%
			} %End of the prerestriction of not condition
			}% End of champion handling
			\ifdefempty{\commandgroup@musician}{}{% We have a musician!
				\item \hspace*{-0.04cm}\option{\labels@musician%
					% Possible restrictions to taking a musician
				    \expandafter\ifblank\expandafter{\commandgroup@musicianrestriction}{}{ \only{\commandgroup@musicianrestriction}}%
				    % Cost of a musician
				    }{\commandgroup@musician}%
			}%
			\ifdefempty{\commandgroup@banner}{}{% We have a banner!
				\item \hspace*{-0.04cm}\option{\labels@standardbearer%
					% Possible restrictions to taking a banner
				    \expandafter\ifblank\expandafter{\commandgroup@bannerrestriction}{}{ \only{\commandgroup@bannerrestriction}}%
				    % Cost of a banner
				    }{\commandgroup@banner}%
				    % Magical banner, if all units of this type can take one.
					\ifdefempty{\commandgroup@bannerallowance}{}{\par\option[\upto]{\hspace*{0.3cm}- \labels@bannerallowance}{\commandgroup@bannerallowance}}%
					% Magical banner, if Veteran.
					\ifdefempty{\commandgroup@veteranstandardbearer}{}{\par\hspace*{0.3cm}- \labels@veteranstandardbearer%
					\expandafter\ifstrequal\expandafter{\commandgroup@veteranstandardbearer}{*}{*}{}%
					}%
					% Magical banner, if only one unit of this type can take one.
					\ifdefempty{\commandgroup@singlebannerallowance}{}{\par\option[\upto]{\hspace*{0.3cm}- \labels@singlebannerallowance}{\commandgroup@singlebannerallowance}}%
					% Magical banner, if only one unit of this type can take one, but with condtions.
					\ifdefempty{\commandgroup@condsinglebannerallowance}{}{%
						\splitatinf{\commandgroup@condsinglebannerallowance}\local@option\local@cost%
						\par\option[\upto]{\hspace*{0.3cm}- \labels@condsinglebannerallowance \local@option}{\local@cost}}%
					% Additional option for the banner, such as Hill Goblin Lookouts for Ogres
					\ifdefempty{\commandgroup@banneroption}{}{%
						\splitatinf{\commandgroup@banneroption}{\local@option}{\local@cost}%
						\par\option{\hspace*{0.3cm}- \local@option}{\local@cost}%
					}%
			}%
		\end{description}%
	\commandgroupframeend%
	 }%
}


%%% Unit rules %%%

% Frame commands.
\newcommand{\unitrulesframestart}{\begin{innerframe}[\labels@specialrules]}
\newcommand{\unitrulesframeend}{\end{innerframe}}

% Unit rules specific commands.
\newcommand{\unitrule}[2]{\item[#1\spacebeforecolon{}:]#2}

% Unit rule entry.
\newcommand{\unitrules}[1]{\ifdefempty{#1}{}{\unitrulesframestart\vspace*{-0.05cm}\begin{description}[leftmargin=0.3cm, labelindent=0cm, labelsep=0.1cm, itemsep=0.2cm, parsep=0cm]#1\end{description}\unitrulesframeend}}


%%% Special equipment %%%

% Frame commands.
\newcommand{\unitequipmentframestart}{\begin{innerframe}[\labels@specialequipment]}
\newcommand{\unitequipmentframeend}{\end{innerframe}}

% Special equipment specific commands.
\newcommand{\equipmentdef}[2]{\item[#1\spacebeforecolon{}:]#2}

% Special equipment entry.
\newcommand{\unitequipment}[1]{\ifdefempty{#1}{}{\unitequipmentframestart\vspace*{-0.05cm}\begin{description}[leftmargin=0.3cm, labelindent=0cm, labelsep=0.1cm, itemsep=0.2cm, parsep=0cm]#1\end{description}\unitequipmentframeend}}






%%%%%%%%%%%%%%%%%%%%%%%%%%%%%%%%
%%% Profile input and layout %%%
%%%%%%%%%%%%%%%%%%%%%%%%%%%%%%%%

%%% Input parameters %%%

\define@key{unit}{notinQRS}{\def\unit@notinQRS{#1}}
\define@key{unit}{name}{\def\unit@name{#1}}
\define@key{unit}{QRSname}{\def\unit@QRSname{#1}}
\define@key{unit}{profile}{\def\unit@profile{#1}}
\define@key{unit}{cost}{\def\unit@cost{#1}}
\define@key{unit}{invocation}{\def\unit@invocation{#1}}
\define@key{unit}{costpermodel}{\def\unit@costpermodel{#1}}
\define@key{unit}{maxmodels}{\def\unit@maxmodels{#1}}
\define@key{unit}{type}{\def\unit@type{#1}}
\define@key{unit}{unitsize}{\def\unit@unitsize{#1}}
\define@key{unit}{basesize}{\def\unit@basesize{#1}}
\define@key{unit}{commonspecialrules}{\def\unit@commonspecialrules{#1}}
\define@key{unit}{commontype}{\def\unit@commontype{#1}}
\define@key{unit}{commonspecialrulesB}{\def\unit@commonspecialrulesB{#1}}
\define@key{unit}{commontypeB}{\def\unit@commontypeB{#1}}
\define@key{unit}{specialrules}{\def\unit@specialrules{#1}}
\define@key{unit}{magiclevel}{\def\unit@magiclevel{#1}}
\define@key{unit}{magicpaths}{\def\unit@magicpaths{#1}}
\define@key{unit}{equipment}{\def\unit@equipment{#1}}
\define@key{unit}{alignment}{\def\unit@alignment{#1}}
\define@key{unit}{greenhiderace}{\def\unit@greenhiderace{#1}}
\define@key{unit}{weapons}{\def\unit@weapons{#1}}
\define@key{unit}{armour}{\def\unit@armour{#1}}
\define@key{unit}{wizardconclave}{\def\unit@wizardconclave{#1}}
\define@key{unit}{unitequipment}{\def\unit@unitequipment{#1}}
\define@key{unit}{options}{\def\unit@options{#1}}
\define@key{unit}{mounts}{\def\unit@mounts{#1}}
\define@key{unit}{commandgroup}{\def\unit@commandgroup{#1}}
\define@key{unit}{unitrules}{\def\unit@unitrules{#1}}
\define@key{unit}{additional}{\def\unit@additional{#1}}


%%% Frames definition %%%

% Unit's big frame.
\tikzset{unitprice/.style={draw=white, fill=white, rectangle, rounded corners, right, minimum height=0.7cm}}
\tikzset{unittitle/.style={draw=white, fill=white, rectangle, rounded corners, right, minimum height=0.7cm, font=\bfseries}}
\tikzset{unitlogo/.style={draw=white, fill=white, rectangle, right, minimum height=0.7cm}}

\newenvironment{unitframe}[2][]{%
	\mdfsetup{%
		nobreak=true,%
		linewidth=1pt,%
		linecolor=black!30,%
		roundcorner=5pt,%
		backgroundcolor=white,%
		innertopmargin=1.2\baselineskip,
		innerbottommargin=1.2\baselineskip,
		singleextra={
			\expandafter\ifblank\expandafter{\unit@cost}{}{%
				\node[unitprice,anchor=east,xshift=-0.5cm] at (P)%
					{%
						{{\smallfontsize\minprice} \Largefontsize\pts{\textbf{\unit@cost}}}%
					};
				}%
				\node[unittitle,xshift=0.5cm] at (P-|O)%
					{\Largefontsize\antiquefont\uppercase\expandafter\expandafter\expandafter{\unit@name}};
				\node[unitlogo, xshift=8.1cm, yshift=0.1cm] at (P-|O)%
					{\includegraphics[width=1.2cm]{\logolocalpath}};
		}
	}%
	\begin{mdframed}[]\relax%
}%
{%
\end{mdframed}%
}

% Inner small frames for options, special rules definition, ...
\tikzset{innertitle/.style={fill=white, rectangle, rounded corners, right, minimum height=8pt, xshift=0.5cm}}

\newenvironment{innerframe}[1][]{%
	\mdfsetup{%
		innerleftmargin=5pt,%
		innerrightmargin=5pt,%
		linecolor=black!30,%
		linewidth=0.5pt,%
		roundcorner=5pt,%
		backgroundcolor=white,%
		innertopmargin=1.1\baselineskip,
		singleextra={
		\node[innertitle] at (P-|O)%
			{\unitentryformat{#1}};
		}
	}%
	\vspace*{-0.2cm}\begin{mdframed}[]\relax%
}%
{%
\end{mdframed}%
}

%%% Command to add a new unit definition %%%

\newcommand{\defunit}{
	\setkeys{unit}{%
		notinQRS=, name=, QRSname=, profile=, cost=, invocation=, costpermodel=, maxmodels=, type=, unitsize=, basesize=, commonspecialrules=, commontype=, commonspecialrulesB=, commontypeB=, specialrules=, magiclevel=, magicpaths=, alignment=, greenhiderace=, equipment=, weapons=, armour=, wizardconclave=, unitequipment=, options=, mounts=, commandgroup=, unitrules=, additional=%
	}%
	\setkeys{unit}%
}

\newcommand{\showunit}[1]{
	\defunit{#1}
	\begin{unitframe}[\unit@name]{\unit@cost}
	\mdfsetup{style=defaultoptions}
	\expandafter\ifblank\expandafter{\unit@unitsize}{}{%
	\expandafter\ifstrequal\expandafter{\unit@unitsize}{1}{% single model
		% Can you add model to this single model ?
		\expandafter\ifblank\expandafter{\unit@maxmodels}{% no		
			{\hspace*{0.25cm}\labels@Singlemodel}%
		}{% yes
			{\hspace*{0.25cm}\mincostfor{} \textbf{1} \labels@model{}. \maxunitsize{}\spacebeforecolon{}: \textbf{\unit@maxmodels} \labels@models{}.\hfill \additionalfigscost{} {\largefontsize\pts{\textbf{\unit@costpermodel{}}}\permodel}\hspace*{0.1cm}}%
		}%
	}{% not single model
		% Test if we wanna print a sentence instead of unit number
		\ifsubstring{\unit@unitsize}{SPECIAL-}{%
			\hspace*{0.25cm}\StrDel{\unit@unitsize}{SPECIAL-}%
		}{%	
			{\hspace*{0.25cm}\mincostfor{} \textbf{\unit@unitsize} \labels@models{}. \maxunitsize{}\spacebeforecolon{}: \textbf{\unit@maxmodels} \labels@models{}.\hfill \additionalfigscost{} {\largefontsize\pts{\textbf{\unit@costpermodel{}}}\permodel}\hspace*{0.1cm}}%
		}%
	}%
	}%
	\vspace*{-0.1cm}
	\noindent\begin{center}\textcolor{black!30}{\rule{\columnwidth}{1pt}}\end{center}
		\expandafter\ifblank\expandafter{\unit@invocation}{%
			\expandafter\profile\expandafter{\unit@profile}
		}{%
			\expandafter\invocprofile\expandafter{\unit@profile}
		}
	\noindent\begin{center}\textcolor{black!30}{\rule{\columnwidth}{1pt}}\end{center}
	\vspace*{-0.2cm}
	\setlength\multicolsep{0pt}
	\begin{multicols}{2}
		\raggedcolumns
		\vspace*{-0.3cm}{\setlength{\parskip}{0.3cm}
		\expandafter\ifblank\expandafter{\unit@alignment}{}{\noindent\parbox{\columnwidth}{\alignment{\unit@alignment}}}
		
		\expandafter\ifblank\expandafter{\unit@greenhiderace}{}{\noindent\parbox{\columnwidth}{\greenhideraceentry{\unit@greenhiderace}}}
		
		\expandafter\ifblank\expandafter{\unit@equipment}{}{\noindent\parbox{\columnwidth}{\equipment{\unit@equipment}}}
				
		\expandafter\ifblank\expandafter{\unit@weapons}{}{\noindent\parbox{\columnwidth}{\weapons{\unit@weapons}}}
		
		\expandafter\ifblank\expandafter{\unit@armour}{}{\noindent\parbox{\columnwidth}{\armour{\unit@armour}}}
		
		\expandafter\ifblank\expandafter{\unit@commonspecialrules}{}{\noindent\parbox{\columnwidth}{\commonspecialrules{\unit@commontype}{\unit@commonspecialrules}}}
		
		\expandafter\ifblank\expandafter{\unit@commonspecialrulesB}{}{\noindent\parbox{\columnwidth}{\commonspecialrules{\unit@commontypeB}{\unit@commonspecialrulesB}}}
		
		\expandafter\ifblank\expandafter{\unit@specialrules}{}{\noindent\parbox{\columnwidth}{\specialrules{\unit@specialrules}}}
		
		\expandafter\ifblank\expandafter{\unit@magicpaths}{}{\noindent\parbox{\columnwidth}{\magic{\unit@magiclevel}{\unit@magicpaths}}}
		
		\expandafter\ifblank\expandafter{\unit@wizardconclave}{}{\noindent\parbox{\columnwidth}{\magicwizardconclave{\unit@wizardconclave}}}
		}
		\vspace{0.1cm}
		\mounts{\unit@mounts}
		\options{\unit@options}
		\expandafter\ifblank\expandafter{\unit@commandgroup}{}{\expandafter\commandgroup\expandafter{\unit@commandgroup}}
		\unitrules{\unit@unitrules}
		\unitequipment{\unit@unitequipment}
	\end{multicols}
	\vspace*{0.1cm}\unit@additional
	\end{unitframe}
	% Database filling for auto QRS
	\expandafter\ifblank\expandafter{\unit@notinQRS}{%
	\DTLnewrow{profiles}%
	\expandafter\ifblank\expandafter{\unit@QRSname}{%
		\expandafter\profiledtbfillname\expandafter{\unit@name}%
	}{%
		\expandafter\profiledtbfillname\expandafter{\unit@QRSname}%
	}
	\expandafter\profiledtbfillcategory\expandafter{\profilecategory}%
	\expandafter\profiledtbfilltrooptype\expandafter{\unit@type}%
	\expandafter\ifblank\expandafter{\unit@invocation}{}{\expandafter\profiledtbfillinvocation\expandafter{\unit@invocation}}%
	\expandafter\profiledtbfillcarac\expandafter{\unit@profile}
	}{}%
}


%%% Changelog commands %%%

\newcommand{\newlog}[2]{%
\vspace*{0.2cm}\noindent{\antiquefont\Large\textbf{V#1}}
\parselist[,]{#2}{\locallists@changelist}%
\begin{itemize}[itemsep=0pt]%
\forlistloop{\item[-]}{\locallists@changelist}%
\end{itemize}%
}

\newcommand{\startchangelog}{\begin{multicols}{2}\vspace*{-0.2cm}}
\def\endchangelog{\end{multicols}}


\newcommand{\booktitle}{Hautes Lignées Elfes}
\newcommand{\version}{0.99.9}
\newcommand{\frenchversion}{2.0}
\newcommand{\booklogo}{pics/logo_big_HBE.png}

% Army special rules

\newcommand{\martialdiscipline}{Discipline Martiale}
\newcommand{\masterofbalance}{Maître de l'Équilibre}
\newcommand{\valiant}{Intrépide}
\newcommand{\lastoftheirkind}{Derniers de leur Espèce}

% Army common type special rules

\newcommand{\elvencommonrules}{Règles spéciales des Elfes}

% Armoury

\newcommand{\moonlightarrows}{Flèches du Clair de Lune}
\newcommand{\bitterarrows}{Flèches Cinglantes}
\newcommand{\dragonforgedarmour}{Armure du Feu-Dragon}

% Honours

\newcommand{\honour}{Honneur}
\newcommand{\honours}{Honneurs}

% Spells

\newcommand{\drainmagicspell}{Drain Magique}
\newcommand{\thunderboltspell}{Choc Foudroyant}

% Other rules

\newcommand{\swordsworn}{Serment de l'Épée}
\newcommand{\lionsfur}{Fourrure de Lion}
\newcommand{\steadyaim}{Visée Stable}
\newcommand{\elvenboltthrower}{Faucheuse Garde-Mer}
\newcommand{\repeatingshots}{Tir à Répétition}
\newcommand{\skyreaper}{Faucheuse des Airs}
\newcommand{\aldanwarhorn}{Cor d'Aldan}
\newcommand{\stormpennant}{Fanion de Tempête}
\newcommand{\flameswoop}{Par les Flammes du Phénix!} 
\newcommand{\rebirth}{Renaissance}
\newcommand{\chillaura}{Aura Glaciale}



%%% Names

% Characters

\newcommand{\highprince}{Haut Prince}
\newcommand{\archmage}{Archimage}
\newcommand{\commander}{Commandant}
\newcommand{\mage}{Mage}
\newcommand{\highprinces}{Hauts Princes}
\newcommand{\archmages}{Archimages}
\newcommand{\commanders}{Commandants}
\newcommand{\mages}{Mages}

% Core

\newcommand{\citizenspears}{Lanciers Plébéiens}
\newcommand{\citizenspear}{Lancier Plébéien}
\newcommand{\citizenarchers}{Archers Plébéiens}
\newcommand{\citizenarcher}{Archer Plébéien}
\newcommand{\seaguards}{Gardes-Mer}
\newcommand{\seaguard}{Garde-Mer}
\newcommand{\highbornlancers}{Lance Patricienne}
\newcommand{\highbornlancer}{Lance Patricienne}
\newcommand{\eleinreavers}{Patrouilleurs d'Elein} 
\newcommand{\eleinreaver}{Patrouilleur d'Elein}

% Special

\newcommand{\swordmasters}{Maîtres de l'Épée}
\newcommand{\swordmaster}{Maître de l'Épée}
\newcommand{\lionguards}{Gardes-Lion}
\newcommand{\lionguard}{Garde-Lion}
\newcommand{\flamewardens}{Protecteurs de la Flamme}
\newcommand{\flamewarden}{Protecteur de la Flamme}
\newcommand{\greywatchers}{Veilleurs de l'Ombre}
\newcommand{\greywatcher}{Veilleur de l'Ombre}
\newcommand{\knightsofryma}{Chevaliers de Ryma}
\newcommand{\knightofryma}{Chevalier de Ryma}
\newcommand{\reaverchariot}{Char Patrouilleur}
\newcommand{\lionchariot}{Char à Lions}
\newcommand{\seaguardreaper}{Faucheuse Garde-Mer}

% Rare

\newcommand{\queensguard}{Gardes de la Reine}
\newcommand{\skysloop}{Char Rapace}
\newcommand{\gianteagles}{Aigles Géants}
\newcommand{\firephoenix}{Phénix de Feu}
\newcommand{\firephoenixes}{Phénix de Feu}
\newcommand{\frostphoenix}{Phénix de Glace}
\newcommand{\frostphoenixes}{Phénix de Glace}


% Mounts

\newcommand{\gianteagle}{Aigle Géant}
\newcommand{\griffon}{Griffon}
\newcommand{\elvenhorse}{Cheval Elfique}
\newcommand{\youngdragon}{Jeune Dragon}
\newcommand{\youngdragons}{Jeunes Dragons}
\newcommand{\dragon}{Dragon}
\newcommand{\dragons}{Dragons}
\newcommand{\ancientdragon}{Dragon Ancien}

% Short names and multiprofiles names

\newcommand{\rider}{Cavalier}
\newcommand{\knight}{Chevalier}
\newcommand{\crew}{Équipage}
\newcommand{\reaper}{Faucheuse}
\newcommand{\lion}{Lion}
\newcommand{\hawk}{Faucon}

% Honours layout

\newcommand{\honoursidepic}[1]{%
\begin{minipage}[t][][b]{0.07\textwidth}
\includegraphics[width=\textwidth]{pics/#1}
\end{minipage}%
}
\newcommand{\honourstartsidetext}[2]{%
\begin{minipage}[t]{0.87\textwidth}
\noindent\option{\textbf{#1}}{#2}\vspace{5pt}\newline
}

\newcommand{\honourmulticols}{%
\normalfontsize
\vspace{-0.2cm}
\begin{multicols}{2}\raggedcolumns
}

\newcommand{\honourclosesidetext}{%
\end{multicols}
\largefontsize
\end{minipage}
}

\newcommand{\spacebetweenhonours}{\vspace{0.8cm}}

% Profile wording

\newcommand{\honourchoice}{Peut choisir un seul \honour}
\newcommand{\anybattlepaths}{la Discipline Commune de son choix}
\newcommand{\ambushseaguardoption}{\ambush{} (max. 20 figs., \oneofakind)}
\newcommand{\skirmishlionguardoption}{\skirmishers{} (max. 15 figs., \oneofakind)}
\newcommand{\maytakealdanwarhorn}{Peut avoir \aldanwarhorn}
\newcommand{\maytakestormpennant}{Peut avoir \stormpennant}
\newcommand{\firephoenixmountmention}{Voir l'unité Rare Phénix de Feu pour le détail des règles.}
\newcommand{\frostphoenixmountmention}{Voir l'unité Rare Phénix de Glace pour le détail des règles.}
\newcommand{\ifmountedbyhighprinceupgradetoancientdragon}{Si monté par un \highprince{}{,}\newline peut devenir un \ancientdragon}
\newcommand{\ancientdragonbasesizenote}{La taille de son socle passe alors à \unit{100x150}{\milli\meter}.}

% Profile rules

\newcommand{\steadyaimrule}{%
Les figurines comptent toujours comme n'ayant pas bougé pour ce qui concerne la règle \volleyfire{}. Ne peut pas être utilisé en même temps que la règle \quicktofire{}. Si une figurine dispose des deux règles, choisissez avant de tirer.
}

\newcommand{\swordswornrule}{%
Si la figurine est à pied et combat avec une Arme lourde, elle ignore les règles \parry{} et \distracting{} des ennemis.
}

\newcommand{\lionsfurrule}{%
Le porteur gagne une \innatedefence{6} qui passe à (5+) contre les Attaques de Tir.
}

\newcommand{\elvenboltthrowerrule}{%
\textbf{\artilleryweapon} de type \textbf{\boltthrower}.\newline
\range{48}, \Strength{} 6, \multiplewounds{1D3}{}, \armourpiercing{6}.
}

\newcommand{\repeatingshotsrule}{%
La \elvenboltthrower{} peut aussi être utilisée comme une \textbf{\artilleryweapon} de type \textbf{\volleygun}.\newline
\range{48}, \Strength{} 4, \armourpiercing{1}, \multipleshots{6}.
}

\newcommand{\skyreaperrule}{%
\textbf{\artilleryweapon} de type \textbf{\volleygun}.\newline
\range{24}, \Strength{} 5, \armourpiercing{1}, \multipleshots{4}, \quicktofire{}.
}

\newcommand{\aldanwarhornrule}{%
Les unités ennemies à moins de \distance{12} d'au moins une figurine avec cette règle subissent un malus de -1 en Capacité de Combat.
}

\newcommand{\stormpennantrule}{%
Les Attaques de Corps à Corps et les \impacthits{} de la figurine gagnent la règle \lightningattacks{}. La figurine gagne un \boundspell{4} : \thunderboltspell{} (Discipline \heavens{}).
}

\newcommand{\flameswooprule}{%
\sweepingattack{}. L'unité ciblée subit 1D6 touches, plus 1D3 touches supplémentaires pour chacun de ses rangs après le premier. Ces touches sont résolues à Force 4 avec la règle \flamingattacks{}.
}

\newcommand{\rebirthrule}{%
La première fois qu'un \firephoenix{} perd son dernier Point de Vie, lancez 1D6. Sur 5+ (ou 3+ s'il est monté), placez un marqueur centré sur la dernière position de la figurine. Si le jet est raté, la figurine est retirée comme perte. Au début de l'étape des Autres Mouvements du prochain Tour de Joueur, replacez le \firephoenix{} et son Personnage (s'il en portait un) sur la table. Le centre de la figurine doit être placé à \distance{3} au plus du marqueur, et à plus d'\distance{1} de toute autre figurine ou Terrain Infranchissable. Il peut être orienté dans la direction de votre choix. Si vous ne pouvez pas le placer, la figurine est considérée comme perdue et ne revient plus en jeu. Si elle revient en jeu, il s'agit de la même figurine que celle qui a quitté le jeu, avec tous les effets de jeu la ciblant précédemment, tels que les sorts Restant en Jeu, avec l'exception qu'il ne lui reste plus qu'un seul Point de Vie.
}

\newcommand{\chillaurarule}{%
Les unités ennemies en contact avec au moins une figurine avec cette règle subissent des malus de -3 en Initiative et -1 en Force, jusqu'à un minimum de 1.
}






\begin{document}

\newgeometry{margin=1in}

% Table options
\arrayrulecolor{black!30}
\setlength{\arrayrulewidth}{0.5pt}
\renewcommand{\arraystretch}{1.2}

\begin{titlepage}
\begin{center}

\ifdef{\booktitle}{}{\newcommand{\booktitle}{Missing title}}
\ifdef{\version}{}{\newcommand{\version}{Missing version}}

{\antiquefont\fontsize{40}{48}\selectfont\noindent\labels@fantasybattles

\labels@NinthAge}

\vspace*{0.5cm}
\ifdef{\booklogo}{%
\includegraphics[height=10cm]{\booklogo}%
}{%
\includegraphics[height=10cm]{../Layout/pics/logo_9th.png}%
}

\vspace*{-1cm}
{\antiquefont\fontsize{50}{60}\selectfont \booktitle
\vspace{0.4cm}

\fontsize{14}{16.8}\selectfont \labels@armyrules{}

Beta v\version{} - \today{}}

\ifdef{\frenchversion}{{\fontsize{14}{16.8}\selectfont \vspace{0.2cm}\noindent\texttt{VF \frenchversion}}}{}
\vfill

\begin{tabular}{@{}m{2cm}@{\hskip 20pt}m{13cm}@{}}
\includegraphics[width=2cm]{../Layout/pics/seal_9th.png} &
{\fontsize{10}{12}\selectfont \textcolor{black!50}{\noindent\labels@frontpagecredits}}

\ifdef{\frontpageaddstuff}{{\fontsize{10}{12}\selectfont \noindent\textcolor{black!50}{\frontpageaddstuff}}}{}

\vspace*{10pt}
\noindent{\fontsize{10}{12}\selectfont \textcolor{black!50}{\labels@license}}
\tabularnewline
\end{tabular}


\end{center}

\newpage

\thispagestyle{empty}

{\fontsize{10}{12}\selectfont

\begin{center}\noindent{\Largerfontsize\textbf{\labels@tableofcontents}}\end{center}

\vspace*{0.2cm}\begin{multicols}{2}

\tocfirstcolumn

\vspace*{\fill}\columnbreak

\tocentry{lordtitle}{\labels@lords}

\tocentry{herotitle}{\labels@heroes}

\ifdef{\tocmounts}{\tocentry{mountstitle}{\tocmounts}}{}

\tocentry{coretitle}{\labels@coreunits}

\tocentry{specialtitle}{\labels@specialunits}

\tocentry{raretitle}{\labels@rareunits}

\vspace*{\fill}\end{multicols}

\ifdef{\labels@introduction}{\vspace{0.7cm}\labels@introduction}{\vphantom{1pt}}
\vfill

\noindent\newrule{\labels@rulechanges}

\bigskip
\noindent \labels@latexcredit
}


\end{titlepage}

\restoregeometry

\startarmyspecialrules

\armyspecialruleentry{\martialdiscipline}

Si une unité qui est composée d'une majorité de figurines avec cette règle effectue un test de Commandement, lancez 1D6 supplémentaire, puis retirez le dé ayant donné le résultat le plus élevé. Cette règle ne s'applique pas pour les tests de Panique ou de Moral.

\armyspecialruleentry{\masterofbalance}

Si votre armée contient au moins une figurine avec cette règle, ajoutez +1 à vos tentatives de Canalisation pendant la Phase de Magie adverse.

\armyspecialruleentry{\valiant}

La figurine gagne la règle \stubborn{} si elle est en contact avec un ennemi qui provoque la \fear{}. Les figurines ordinaires gagnent aussi \bodyguard{\highprince{}, \commander{}}, mais seulement pour un Personnage avec l’Honneur Chasseur Royal ou n'ayant pas choisi d'Honneur.

\armyspecialruleentry{\lastoftheirkind}

Votre armée ne peut pas inclure plus de deux figurines parmi cette liste : \youngdragon{}, \dragon{}, \firephoenixes{} et \frostphoenixes{}. Cela inclut les montures de Personnage. Cette limite passe à 4 pour une Grande Armée et à 1 pour une Patrouille.

\closearmyspecialrules

\vspace*{1.5cm}
\startarmyarmoury

\startitemlistonecol

\listitemonecol{\moonlightarrows}Lorsqu'elles sont tirées avec un \bow{} ou un \longbow{} ordinaire, ces flèches gagnent Force 4, \flamingattacks{} et \magicalattacks{}.

\listitemonecol{\bitterarrows}Lorsqu'elles sont tirées avec un \bow{} ou un \longbow{} ordinaire, vous pouvez choisir d'échanger la règle \volleyfire{} contre des \poisonedattacks{}. 

\listitemonecol{\dragonforgedarmour}Type : \ha{}. Le porteur gagne \fireborn{} et \wardsave{6}. Non cumulable avec une \lionsfur{}.

\enditemlistonecol

\closearmyarmoury








\startarmynewsection{\honours}

\armynewsubsection{\honours{} des \highprinces{} et des \commanders{}}
\vspace{0.5cm}

\honoursidepic{logo_tower.png}
\hfill
\honourstartsidetext{Maître de la Tour de Canreig}{150/75}\oneofakind{}.\vspace{5pt}

La figurine gagne les règles \swordsworn{} (voir l'unité spéciale \swordmasters{}) et \masterofbalance{}. Elle devient un \textbf{\magiclevel{1}}. Les \commanders{} choisissent \textbf{deux} sorts de la Discipline des Huit Vents, qui réunit les sorts primaires des huit Disciplines Communes. Les sorts choisis doivent être inscrits sur la feuille d'armée. Les \highprinces{} connaissent tous les sorts de la Discipline des Huit Vents.\newline
Le nombre de sorts connus ne peut alors jamais être augmenté.
\honourmulticols{}
\def\tempmounts{À pied uniquement=,}
\vspace*{-0.3cm}\mounts{\tempmounts}\columnbreak\phantom{debug}
\honourclosesidetext{}

\spacebetweenhonours{}

\honourstartsidetext{Prince de Ryma}{30}\oneofakind{}. \highprince{} uniquement.\vspace{5pt}

La figurine gagne la règle \devastatingcharge{} et sa monture n'est pas comptée pour la règle \lastoftheirkind{}. Elle \textbf{doit} sélectionner une des trois montures suivantes :
\honourmulticols{}
\def\tempmounts{\elvenhorse{}=\free{},\youngdragon{}=230,\dragon{}=295,}
\vspace*{-0.3cm}\mounts{\tempmounts}\columnbreak\phantom{debug}
\honourclosesidetext{}
\hfill
\honoursidepic{logo_ryma.png}

\spacebetweenhonours{}

\honoursidepic{logo_fleet.png}
\hfill
\honourstartsidetext{Commodore}{20}La figurine gagne les règles \weaponmaster{} et \steadyaim{} (voir l'unité de base \seaguards{}). Elle peut acheter n'importe quel nombre d'armes standard. Si la figurine est à pied, son unité peut effectuer une Reformation de Combat lorsqu'elle est chargée avec succès, immédiatement après que toutes les unités ennemies ont été amenées au contact. Cette Reformation suit les règles habituelles de Reformation de Combat. Le Commodore n'a accès qu'aux montures de la liste ci-dessous.
\honourmulticols{}
\def\tempmounts{\gianteagle{}=40,\skysloop{}=120,}
\def\tempoptions{\ambush{} \only{\commander{} à pied}=15,}
\vspace*{-0.3cm}\mounts{\tempmounts}\columnbreak\vspace*{-0.4cm}\options{\tempoptions}
\honourclosesidetext{}

\spacebetweenhonours{}

\honourstartsidetext{Haut Gardien de la Flamme}{85/70}La figurine gagne les règles \divineattacks{}, \immunetopsychology{}, \magicresistance{1} et \wardsave{4}. Son unité gagne la règle \magicalattacks{}. Les montures de la figurine sont restreintes aux choix suivants :
\honourmulticols{}
\def\tempmounts{\firephoenix{}=160/190,\frostphoenix{}\refsymbol{}=200,\refsymbol{} \highprince{} \labels@only{}=,}
\vspace*{-0.3cm}\mounts{\tempmounts}\columnbreak\phantom{debug}
\honourclosesidetext{}
\hfill
\honoursidepic{logo_warden.png}

\spacebetweenhonours{}

\honoursidepic{logo_queen.png}
\hfill
\honourstartsidetext{Suivante de la Reine}{20}La figurine gagne la règle \multipleshots{3}. Son unité gagne la règle \quicktofire{}.
\honourmulticols{}
\def\tempmounts{À pied uniquement=,}
\def\tempoptions{\maytakeonechoiceonly{%
	\moonlightarrows{}=5,
	\scout{}{,} \bitterarrows{}=20,
	},
}
\vspace*{-0.3cm}\mounts{\tempmounts}\columnbreak\vspace*{-0.4cm}\options{\tempoptions}
\honourclosesidetext{}

\spacebetweenhonours{}

\honourstartsidetext{Chasseur Royal}{60/45}La figurine gagne la règle \valiant{} et une \lionsfur{} (voir l'unité spéciale \lionguards{}). Elle ne peut pas prendre d'\dragonforgedarmour{}. Lorsqu'il se bat avec une \gw{}, il gagne la règle \multiplewounds{2}{\monstrouscavalry{}, \monstrousbeast{}, \monster{}, \riddenmonster{}}. Son unité est immunisée aux effets de la \fear{} et de la \terror{}. Il n'a accès qu'à la monture suivante :
\honourmulticols{}
\def\tempmounts{\lionchariot{}=35/60,}
\vspace*{-0.3cm}\mounts{\tempmounts}\columnbreak\phantom{debug}
\end{multicols}
\largefontsize
\end{minipage}
\hfill
\honoursidepic{logo_huntsman.png}

\armynewsubsection{\honours{} des \archmages{} et des \mages{}}
\vspace{0.5cm}

\honoursidepic{logo_scholar.png}
\hfill
\honourstartsidetext{Érudit d'Asfad}{40/30}Tout sort lancé par le Sorcier voit sa portée augmenter de \distance{6}. L'effet n'est que de \distance{3} pour les sorts de type Aura, et il n'a pas d'effet sur les sorts de type Vortex, les Objets de Sort et les sorts sans portée. De plus, le Sorcier gagne l'\boundspell{4} suivant :\vspace{5pt}\newline
\textbf{Drain Magique}. Type : Universel, \range{18}. Durée : Immédiat.\newline Tous les sorts affectant la cible dont la Durée est Dure un Tour ou Reste en Jeu disparaissent immédiatement.\vspace{5pt}\newline
La figurine a accès à ses montures habituelles.
\end{minipage}

\spacebetweenhonours{}

\honourstartsidetext{Ordre du C\oe{}ur de Feu}{220/205}Le Sorcier doit choisir la Discipline \fire{}, et connaît toujours le sort Épées Flamboyantes (Discipline \fire{}) en plus de ses sorts normaux. Si ce sort est tiré lors de la génération des sorts, relancez le dé correspondant jusqu'à obtenir un autre sort. Cela n'empêche pas d'autres Sorciers d'avoir accès au sort. De plus, la figurine ignore les mots-clés Projectile et Dégâts lorsqu'elle lance un sort contre une unité avec laquelle elle est engagée au Corps à Corps. Elle \textbf{doit} sélectionner une des deux montures suivantes :
\honourmulticols{}
\def\tempmounts{\youngdragon{}=\free{},\dragon{}\refsymbol{}=130,\refsymbol{} \archmage{} \labels@only{}=,}
\def\tempoptions{\dragonforgedarmour{}=15/12,}
\vspace*{-0.3cm}\mounts{\tempmounts}\columnbreak\vspace*{-0.4cm}\options{\tempoptions}
\honourclosesidetext{}
\hfill
\honoursidepic{logo_fieryheart.png}

\closearmynewsection








\startarmymagicalitems

\armymagicalweapons

\startpricelist

\pricelistitem{Grand Arc d'Azur}{35}Type : \longbow{}. \range{30}, Force de l'utilisateur +1, \armourpiercing{1}, \multipleshots{3}.


\pricelistitem{Lance de l'Aube Éclatante}{30/20}Personnages uniquement.

Type : \spear{}. La Force des attaques effectuées avec cette arme est supérieure d'un point à l'Endurance de la cible. Si la Force était déjà supérieure, ignorez la règle.

\endpricelist

\armymagicalarmour

\startpricelist

\pricelistitem{Heaume du Chasseur de Démon}{35/25}Ne peut pas être pris par une \largetarget{}.

Type : Aucun (Sauvegarde d'Armure de 6+). Le porteur gagne une \wardsave{3} contre les \magicalattacks{}.

\endpricelist

\armytalismans

\startpricelist

\pricelistitem{Robes Étincelantes}{45}\archmage{} ou \mage{} à pied uniquement.

Le porteur gagne la règle \ethereal{}. Il ne peut pas porter d'attaques, y compris spéciales.

\endpricelist

\armyenchanteditems

\startpricelist

\pricelistitem{Cape des Étoiles}{20}Les sorts ciblant l'unité du lanceur subissent un malus de -1 sur leur jet de lancement.

\pricelistitem{Éclat de Cenyrn}{10}Usage Unique. Activez-le au début de n'importe quelle Phase de Corps à Corps. Les effets durent jusqu'à la fin de cette phase. Après avoir lancé les dés pour toucher, le porteur peut choisir de relancer \textbf{tous} ces dés. Après avoir lancé les dés pour blesser, le porteur peut choisir de relancer \textbf{tous} ces dés. Les jets de sauvegarde de toute sorte réussis contre des blessures infligées par le porteur doivent être relancés. À la fin de cette phase, le porteur subit une blessure sans aucune sauvegarde autorisée avec la règle \multiplewounds{2}{\riddenmonster}.

\endpricelist

\armyarcaneitems

\startpricelist

\pricelistitem{Cristal d'Améthyste}{55}Pendant le tour de votre adversaire, après avoir déterminé les Flux de Magie, vous pouvez retirer un Dé de Pouvoir de sa réserve. Ajoutez +1 à votre jet de Canalisation pour cette phase.

\pricelistitem{Livre de Meladys}{45/25}Une fois par Phase de Magie, le porteur peut relancer un Dé de Pouvoir utilisé pour une tentative de lancement de sort, à condition que ce dé n'ait pas donné un \result{6} naturel.

\endpricelist

\armymagicalbanners

\startpricelist

\pricelistitem{Bannière d'Apaisement}{35}Le porteur gagne la règle \channel{}. Au début de chaque Phase de Magie adverse, choisissez un Sorcier allié à moins de \distance{12} du porteur. Ce Sorcier gagne un bonus de +1 pour dissiper lors de cette phase.

\pricelistitem{Bannière de Bataille de Ryma}{30}L'unité du porteur gagne la règle \thunderouscharge{}. Les montures ne sont pas affectées.

\endpricelist

\closearmymagicalitems



%%% START OF THE ARMYLIST - Translators shouldn't have to edit it %%%


%%% v0.99.1

\armylist

\lordstitle

\showunit{
	name={\highprince},
	cost={135},
	profile={ < 5 7 7 4 3 3 8 4 10},
	type=\infantry ,
	basesize=20x20,
	unitsize=1,
	commontype=\elvencommonrules ,
	commonspecialrules={\martialdiscipline, \lightningreflexes},
	armour={\la},
	options={
		\honourchoice{}=\unlimited,
		\magicalitemsallowance{}=\upto{}<100,
		\anyofthefollowing{
			\shield{}=5,
			\ha{}=8,
			\dragonforgedarmour{}=15,
		},
		\longbow{}=3,
		\combatweapononechoice{
			\pw{}=5,
			\spear{}=5,
			\lightlance{}=5,
			\gw{}=10,
			\halberd{}=10,
			\lance{}=15,
		},
	},
	mounts={
		\elvenhorse{}=20,
		\reaverchariot{}=25,
		\gianteagle{}=50,
		\griffon{}=120,
		\youngdragon{}=170,
		\dragon{}=250,
		},
}

\showunit{
	name={\archmage},
	cost={185},
	profile={ < 5 4 4 3 3 3 5 1 9},
	type= \infantry ,
	basesize=20x20,
	unitsize=1,
	commontype=\elvencommonrules ,
	commonspecialrules={\martialdiscipline, \lightningreflexes},
	specialrules={\masterofbalance},
	magiclevel=3,
	magicpaths={\battle, \whitemagic},
	options={
		\honourchoice{}=\unlimited,
		\magiclevel{4}=30,
		\magicalitemsallowance{}=\upto{}<100,
	},
	mounts={
		\elvenhorse{}=20,
		\reaverchariot{}=25,
		\gianteagle{}=45,
		\griffon{}=100,
		\youngdragon{}=170,
		\dragon{}=300,
		},
}

\heroestitle


\showunit{
	name={\commander},
	cost={70},
	profile={ < 5 6 6 4 3 2 7 3 9},
	type=\infantry ,
	basesize=20x20,
	unitsize=1,
	commontype=\elvencommonrules ,
	commonspecialrules={\martialdiscipline, \lightningreflexes},
	armour={\la},
	options={
		\bsboption{}=25,
		\honourchoice{}=\unlimited,
		\magicalitemsallowance{}=\upto{}<50,
		\anyofthefollowing{
			\shield{}=2,
			\ha{}=5,
			\dragonforgedarmour{}=12,
		},
		\longbow{}=2,
		\combatweapononechoice{
			\pw{}=5,
			\spear{}=5,
			\lightlance{}=5,
			\gw{}=8,
			\halberd{}=8,
			\lance{}=10,
		},
	},
	mounts={
		\elvenhorse{}=15,
		\reaverchariot{}=40,
		\gianteagle{}=50,
		\griffon{}=150,
		},
}


\showunit{
	name={\mage},
	cost={70},
	profile={ < 5 4 4 3 3 2 5 1 8},
	type= \infantry ,
	basesize=20x20,
	unitsize=1,
	commontype=\elvencommonrules ,
	commonspecialrules={\martialdiscipline, \lightningreflexes},
	specialrules={\masterofbalance},
	magiclevel=1,
	magicpaths={\battle, \whitemagic},
	options={
		\honourchoice{}=\unlimited,
		\magiclevel{2}=25,
		\magicalitemsallowance{}=\upto{}<50,
	},
	mounts={
		\elvenhorse{}=15,
		\gianteagle{}=35,
		\reaverchariot{}=40,
		},
}



\coreunitstitle

\showunit{
	name={\citizenspears},
	cost=105,
	profile={ < 5 4 4 3 3 1 5 1 8},
	type=\infantry ,
	basesize=20x20,
	unitsize=15,
	maxmodels=50,
	costpermodel=9,
	commontype=\elvencommonrules ,
	commonspecialrules={\martialdiscipline, \lightiningreflexes},
	specialrules={\fightinextrarank{}},
	weapons={\spear},
	armour={\la, \shield},
	options={
		\maytakeha{}=\permodel{}<2,
	},
	commandgroup={champion=10, musician=10, banner=10, veteranstandardbearer=yessir}
}

\showunit{
	name={\citizenarchers},
	cost=90,
	profile={ < 5 4 4 3 3 1 5 1 8},
	type=\infantry ,
	basesize=20x20,
	unitsize=10,
	maxmodels=30,
	costpermodel=9,
	commontype=\elvencommonrules ,
	commonspecialrules={\martialdiscipline, \lightiningreflexes},
	weapons={\longbow},
	armour={\la},
	commandgroup={champion=10, musician=10, banner=10, veteranstandardbearer=yessir}
}

\showunit{
	name={\seaguard},
	cost=120,
	profile={ < 5 4 4 3 3 1 5 1 8},
	type=\infantry ,
	basesize=20x20,
	unitsize=10,
	maxmodels=40,
	costpermodel=12,
	commontype=\elvencommonrules ,
	commonspecialrules={\martialdiscipline, \lightiningreflexes},
	specialrules={\fightinextrarank{}, \weaponmaster{}, \steadyaim},
	weapons={\bow, \spear},
	armour={\la, \shield},
	options={
		\maytakeha{}=\permodel{}<2,
		\maytakeambushseaguardoption{}=\permodel{}<3,
	},
	unitrules={
		\unitrule{\steadyaim}{\steadyaimrule}
	},
	commandgroup={champion=10, musician=10, banner=10, veteranstandardbearer=yessir}
}

\showunit{
	name={\highbornlancers},
	cost=95,
	profile={ 
		\rider < 5 4 4 3 3 1 5 1 8,
		\elvenhorse < 9 3 - 3 3 1 4 1 3,
		},
	type=\cavalry ,
	basesize=25x50,
	unitsize=5,
	maxmodels=15,
	costpermodel=18,
	commontype=\elvencommonrules ,
	commonspecialrules={\martialdiscipline, \lightiningreflexes \only{\rider{}}},
	weapons={\lance},
	armour={\ha, \shield, \mountsprotection{6}},
	options={
		\maytakemountsprotectionX{5}=\permodel{}<4,
	},
	commandgroup={champion=10, musician=10, banner=10, veteranstandardbearer=yessir}
}

\showunit{
	name={\eleinreavers},
	cost=85,
	profile={ 
		\rider < 5 4 4 3 3 1 5 1 8,
		\elvenhorse < 9 3 - 3 3 1 4 1 3,
		},
	type=\cavalry ,
	basesize=25x50,
	unitsize=5,
	maxmodels=10,
	costpermodel=14,
	commontype=\elvencommonrules ,
	commonspecialrules={\martialdiscipline, \lightiningreflexes \only{\rider{}}},
	specialrules={\fastcavalry{}},
	weapons={\lightlance},
	armour={\la, \mountsprotection{6}},
	options={
		\maytakebow{}=\permodel{}<1,
		\maytakemountsprotectionX{5}=\permodel{}<3,
	},
	commandgroup={champion=10, musician=10, banner=10}
}


\specialunitstitle

\showunit{
	name={\swordmasters},
	cost=65,
	profile={ < 5 6 4 3 3 1 6 2 8},
	type=\infantry ,
	basesize=20x20,
	unitsize=5,
	maxmodels=30,
	costpermodel=13,
	commontype=\elvencommonrules ,
	commonspecialrules={\martialdiscipline, \lightiningreflexes},
	specialrules={\swordsworn},
	weapons={\gw},
	armour={\ha},
	unitrules={
		\unitrule{\swordsworn}{\swordswornrule}
	},
	commandgroup={champion=10, championallowance=25, musician=10, banner=10, bannerallowance=50}
}

\showunit{
	name={\lionguards},
	cost=120,
	profile={ < 5 5 4 4 3 1 5 1 8},
	type=\infantry ,
	basesize=20x20,
	unitsize=10,
	maxmodels=30,
	costpermodel=14,
	commontype=\elvencommonrules ,
	commonspecialrules={\martialdiscipline, \lightiningreflexes},
	specialrules={\multiplewounds{2}{\monstrousbeast, \monstrouscavalry, \monster et \riddenmonster}, \strider{\forest{}}, \valiant},
	weapons={\gw},
	armour={\ha, \lionsfur},
	options={
		\maytakeskirmishlionguardoption{}=\permodel{}<3,
	},
	unitrules={
		\unitrule{\lionsfur}{\lionsfurrule}
	},
	commandgroup={champion=10, championallowance=25, musician=10, banner=10, bannerallowance=50}
}

\showunit{
	name={\flamewardens},
	cost=135,
	profile={ < 5 5 4 3 3 1 6 1 9},
	type=\infantry ,
	basesize=20x20,
	unitsize=10,
	maxmodels=25,
	costpermodel=16,
	commontype=\elvencommonrules ,
	commonspecialrules={\martialdiscipline, \lightiningreflexes},
	specialrules={\fightinextrarank{}, \immunetopsychology{}, \wardsave{4}},
	weapons={\halberd},
	armour={\ha},
	commandgroup={champion=10, championallowance=25, musician=10, banner=10, bannerallowance=50}
}

\showunit{
	name={\greywatchers},
	cost=80,
	profile={ < 5 5 5 3 3 1 5 1 8},
	type=\infantry ,
	basesize=20x20,
	unitsize=5,
	maxmodels=10,
	costpermodel=16,
	commontype=\elvencommonrules ,
	commonspecialrules={\martialdiscipline, \lightiningreflexes},
	specialrules={\skirmishers{}, \scout{}},
	weapons={\bow, \bitterarrows},
	armour={\la},
	options={
		\maytakelongbow{}=\permodel{}<1,
		\maytakepw{}=\permodel{}<1,
	},
	commandgroup={champion=10}
}

\showunit{
	name={\knightsofryma},
	cost=135,
	profile={ 
		\knight < 5 5 4 4 3 1 6 1 9,
		\elvenhorse < 9 3 - 3 3 1 4 1 3,
		},
	type=\cavalry ,
	basesize=25x50,
	unitsize=5,
	maxmodels=12,
	costpermodel=27,
	commontype=\elvencommonrules ,
	commonspecialrules={\martialdiscipline, \lightiningreflexes \only{\rider{}}},
	weapons={\lance},
	armour={\dragonforgedarmour, \shield, \mountsprotection{5}},
	options={
		\maytakedevastatingcharge{}=\permodel{}<5,
	},
	commandgroup={champion=10, championallowance=25, musician=10, banner=10, bannerallowance=50}
}

\showunit{
	name={\reaverchariot},
	cost=65,
	profile={
		\chariot < - - - 5 4 4 - - -,
		\crew (2) < - 4 4 3 - - 5 1 8,
		\elvenhorse (2) < 9 3 - 3 - - 4 1 3,
		},
	type=\chariot ,
	basesize=50x100,
	unitsize=1,
	maxmodels=4,
	costpermodel=60,
	commontype=\elvencommonrules ,
	commonspecialrules={\martialdiscipline, \lightiningreflexes \only{\crew{}}},
	specialrules={\lighttroops},
	weapons={\lightlance \only{\crew{}}, \longbow \only{\crew{}}},
	armour={\la, \mountsprotection{6}},
	options={
		\maytakevanguard{}=\permodel{}<15,
	},
}

\showunit{
	name={\lionchariot},
	cost=100,
	profile={
		\chariot < - - - 5 4 4 - - -,
		\crew (2) < - 5 4 5 - - 5 1 8,
		\lion (2) < 8 5 - 5 - - 4 2 8,
		},
	type=\chariot ,
	basesize=50x100,
	unitsize=1,
	commontype=\elvencommonrules ,
	commonspecialrules={\martialdiscipline, \lightiningreflexes \only{\crew{}}},
	specialrules={\multiplewounds{2}{\monstrousbeast, \monstrouscavalry, \monster et \riddenmonster} \only{\crew{}}, \impacthits{+1}, \valiant},
	weapons={\gw \only{\crew{}}},
	armour={\ha, \mountsprotection{5}},
}

\showunit{
	name={\seaguardreaper},
	cost=60,
	profile={
		\reaper < - - - - 7 2 - - -,
		\crew (2) < 5 4 4 3 3 - 5 1 8,
		},
	type=\warmachine ,
	basesize=60,
	unitsize=1,
	commontype=\elvencommonrules ,
	commonspecialrules={\martialdiscipline, \lightiningreflexes},
	weapons={\elvenboltthrower},
	unitequipment={
		\equipmentdef{\elvenboltthrower}{\elvenboltthrowerrule}
		\equipmentdef{\repeatingshots}{\repeatingshotsrule}
	},
	armour={\la},
	options={
		\maytakerepeatingshots{}=20,
	},
}


\rareunitstitle


\showunit{
	name={\queensguards},
	cost=65,
	profile={ < 5 5 5 3 3 1 5 1 8},
	type=\infantry ,
	basesize=20x20,
	unitsize=5,
	maxmodels=20,
	costpermodel=13,
	commontype=\elvencommonrules ,
	commonspecialrules={\martialdiscipline, \lightiningreflexes},
	weapons={\bow, \moonlightarrows},
	armour={\la},
	options={
		\maytakespear{}=\permodel{}<1,
		\maytakelongbow{}=\permodel{}<2,
	},
	commandgroup={champion=10, musician=10, banner=10, bannerallowance=25}
}

\showunit{
	name={\skysloop},
	cost=120,
	profile={
		\chariot < - - - 5 4 4 - - -,
		\crew (2) < - 4 4 3 - - 5 1 8,
		\hawk (1) < 2 4 - 4 - - 4 2 8,
		},
	type=\chariot ,
	basesize=50x100,
	unitsize=1,
	commontype=\elvencommonrules ,
	commonspecialrules={\martialdiscipline, \lightiningreflexes \only{\crew{}}},
	specialrules={\hardtarget, \fly{9}},
	unitrules={
		\unitrule{\aldanwarhorn}{\aldanwarhornrule}
		\unitrule{\stormpennant}{\stormpennantrule}
	},
	weapons={\lightlance \only{\crew{}}, \skyreaper \only{\chariot{}}},
	unitequipment={
		\equipmentdef{\skyreaper}{\skyreaperrule}
	},
	armour={\la, \mountsprotection{6}},
	options={
		\maytakeoneofthefollowing{
			\aldanwarhorn{}=40,
			\stormpennant{}=40,
			},
	},
}


\showunit{
	name={\gianteagle},
	cost=50,
	profile={ < 2 5 - 4 4 3 4 2 8},
	type=\monstrousbeast ,
	basesize=50x50,
	unitsize=1,
	maxmodels=4,
	costpermodel=30,
	specialrules={\fly{9}},
	options={
		\maytakearmourpiercing{}=\permodel{}<5,
		\maytakelightningreflexes{}=\permodel{}<5,
	},
}


\showunit{
	name={\firephoenix},
	cost=160,
	profile={ < 2 5 - 5 5 5 4 3 8},
	type=\monster ,
	basesize=50x100,
	unitsize=1,
	specialrules={\fly{9}, \magicalattacks, \wardsave{5}, \fireborn, \flamingattacks, \flameswoop, \rebirth, \lastoftheirkind},
	unitrules={
		\unitrule{\flameswoop}{\flameswooprule}
		\unitrule{\rebirth}{\rebirthrule}
	},
}

\showunit{
	name={\frostphoenix},
	cost=200,
	profile={ < 2 5 - 5 5 5 3 5 8},
	type=\monster ,
	basesize=50x100,
	unitsize=1,
	specialrules={\fly{8}, \magicalattacks, \wardsave{5}, \chillaura, \lastoftheirkind},
	unitrules={
		\unitrule{\chillaura}{\chillaurarule}
	},
	armour={\innatedefence{5}},
}


\mountstitle

\mountssectionannouncement

\showunit{
	name={\gianteagle},
	profile={ < 2 5 - 4 4 3 4 2 8},
	type=\monstrousbeast ,
	basesize=50x50,
	specialrules={\fly{9}},
	armour={\mountsprotection{6}},
	options={
		\maytakearmourpiercing{}=5,
		\maytakelightningreflexes{}=<5,
	},
}

\showunit{
	name={\griffon},
	profile={ < 6 5 - 5 5 4 5 4 5},
	type=\monstrousbeast ,
	basesize=50x50,
	specialrules={\fly{8}, \largetarget, \fear},
	options={
		\anyofthefollowing{
			\armourpiercing{1}=5,
			\lightningreflexes{}=10,
			\devastatingchargeandthunderouscharge{}=25,
		},
	},
}

\showunit{
	name={\elvenhorse},
	profile={ < 9 3 - 3 3 1 4 1 3},
	type=\warbeast ,
	basesize=25x50,
	armour={\mountsprotection{6}},
	options={
		\maytakemountsprotectionX{5}=10,
	},
}

\showunit{
	name={\reaverchariot},
	profile={
		\chariot < - - - 5 4 4 - - -,
		\crew (2) < - 4 4 3 - - 5 1 8,
		\elvenhorse (2) < 9 3 - 3 - - 4 1 3,
		},
	type=\chariot ,
	basesize=50x100,
	commontype=\elvencommonrules ,
	commonspecialrules={\martialdiscipline, \lightiningreflexes \only{\crew{}}},
	specialrules={\lighttroops},
	weapons={\lightlance \only{\crew{}}, \longbow \only{\crew{}}},
	armour={\la, \mountsprotection{6}},
	options={
		\maytakevanguard{}=10,
	},
}

\showunit{
	name={\lionchariot},
	profile={
		\chariot < - - - 5 4 4 - - -,
		\crew (2) < - 5 4 5 - - 5 1 8,
		\lion (2) < 8 5 - 5 - - 4 2 -,
		},
	type=\chariot ,
	basesize=50x100,
	commontype=\elvencommonrules ,
	commonspecialrules={\martialdiscipline, \lightiningreflexes \only{\crew{}}},
	specialrules={\multiplewounds{2}{\monstrousbeast, \monstrouscavalry, \monster et \riddenmonster} \only{\crew{}}, \impacthits{+1}, \valiant},
	weapons={\gw \only{\crew{}}},
	armour={\ha, \mountsprotection{5}},
}

\showunit{
	name={\skysloop},
	profile={
		\chariot < - - - 5 4 4 - - -,
		\crew (2) < - 4 4 3 - - 5 1 8,
		\hawk (1) < 2 4 - 4 - - 4 2 8,
		},
	type=\chariot ,
	basesize=50x100,
	commontype=\elvencommonrules ,
	commonspecialrules={\martialdiscipline, \lightiningreflexes \only{\crew{}}},
	specialrules={\hardtarget, \fly{9}},
	unitrules={
		\unitrule{\aldanwarhorn}{\aldanwarhornrule}
		\unitrule{\stormpennant}{\stormpennantrule}
	},
	weapons={\lightlance \only{\crew{}}, \skyreaper \only{\chariot{}}},
	unitequipment={
		\equipmentdef{\skyreaper}{\skyreaperrule}
	},
	armour={\la, \mountsprotection{6}},
	options={
		\maytakeoneofthefollowing{
			\aldanwarhorn{}=40,
			\stormpennant{}=40,
			},
	},
}

\showunit{
	name={\firephoenix},
	profile={ < 2 5 - 5 5 5 4 3 8},
	type=\monster ,
	basesize=50x100,
	unitsize=1,
	specialrules={\fly{9}, \magicalattacks, \wardsave{4}, \fireborn, \flamingattacks, \flameswoop, \rebirth, \lastoftheirkind},
	unitrules={
		\unitrule{\flameswoop}{\seephoenixrareunit}
		\unitrule{\rebirth}{\seephoenixrareunit}
	},
}

\showunit{
	name={\frostphoenix},
	profile={ < 2 5 - 5 5 5 3 5 8},
	type=\monster ,
	basesize=50x100,
	specialrules={\fly{8}, \magicalattacks, \wardsave{4}, \chillaura, \lastoftheirkind},
	unitrules={
		\unitrule{\chillaura}{\seephoenixrareunit}
	},
	armour={\innatedefence{5}},
}

\showunit{
	name={\youngdragon},
	profile={ < 6 5 1 5 5 4 3 4 9},
	type=\monstrousbeast ,
	basesize=50x100,
	specialrules={\fly{7}, \fear, \largetarget, \stomp{1D3}, \breathweaponyoungdragon, \lastoftheirkind},
	armour={\innatedefence{6}},
}

\showunit{
	name={\dragon},
	profile={ 
		\dragon < 6 5 1 6 6 6 3 5 9,
		\ancientdragon < 6 6 1 7 7 7 3 6 9,
	},
	type=\monster ,
	basesize=50x100,
	unitsize=\oneofakind,
	specialrules={\fly{7}, \breathweapondragon},
	armour={\innatedefence{3}},
	options={
		\ifmounterdbyhighprinceupgradetoancientdragon{}=100,
	},
}
























%%% Quick Reference Sheet - AB_qrs.tex is automatic and shouldn't be edited %%%

\quickrefsheettitle

\input{../Layout/AB_qrs.tex}
\bigskip
\begin{center}
\noindent{\antiquefont\Largefontsize\textbf{Artillerie Elfique}}
\medskip

\rowcolors{1}{white}{black!10}
\noindent\begin{tabular}{lcccccc}
\textbf{Nom} & \textbf{Artillerie} & \textbf{Portée} & \textbf{\labels@S{}} & \textbf{\multipleshots{}} & \textbf{\multiplewounds{}} & \textbf{\armourpiercing{}} \tabularnewline
\seaguardreaper{} & \boltthrower{} & \distance{48} & 6 & - & 1D3 & 6 \tabularnewline
\seaguardreaper{} (\repeatingshots{}) & \volleygun{} & \distance{48} & 4 & 6 & - & 1 \tabularnewline
\skyreaper{} & \volleygun{} & \distance{24} & 5 & 4 & - & 1 \tabularnewline
\end{tabular}

\newcommand{\finallogosize}{1cm}
\vfill\begin{tabular}{@{}cccccccc@{}}
\rowcolor{white}
\includegraphics[width=\finallogosize]{pics/logo_tower.png} &
\includegraphics[width=\finallogosize]{pics/logo_ryma.png} &
\includegraphics[width=\finallogosize]{pics/logo_fleet.png} &
\includegraphics[width=\finallogosize]{pics/logo_warden.png} &
\includegraphics[width=\finallogosize]{pics/logo_queen.png} &
\includegraphics[width=\finallogosize]{pics/logo_huntsman.png} &
\includegraphics[width=\finallogosize]{pics/logo_scholar.png} &
\includegraphics[width=\finallogosize]{pics/logo_fieryheart.png} \tabularnewline
\end{tabular}
\end{center}

\restoregeometry

\end{document}