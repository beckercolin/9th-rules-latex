\def\languageisfrench{}
\documentclass[a4paper,8pt]{extarticle} % extarticle allows to use font size of 8pt.

\usepackage[a4paper, top=1.6cm, bottom=2cm, left=1.6cm, right=1.6cm]{geometry} % Marge reduction.

%% Language specific package
\usepackage[french]{babel}
\frenchbsetup{StandardLists=true} % Necessary to use enumitem with babel/french.

%% Font and typing packages
\usepackage{fontspec}
\setmainfont[
	Ligatures=TeX,
	ItalicFont={Dancing Script},
	BoldItalicFont={Dancing Script}
	]{PT Serif} % default is Latin Modern
\newfontfamily\antiquefont[Ligatures=TeX]{Caslon Antique} % fancy font
\usepackage{microtype}			% Greatly improves general appearance of the text.
\usepackage{SIunits}			% Unit appearance.
\usepackage{ulem}				% To cross words out. Use \sout{}.

%% Array utilities
\usepackage{array}				% Additionnal options for arrays.
\usepackage{colortbl}			% Additionnal options for coloring arrays.
\usepackage{multirow}
\usepackage[table]{xcolor}		% Auto alternate grey-white rows.

%% List utilities
\usepackage[inline]{enumitem}   % Display inline lists.
\usepackage{etoolbox}           % General utility. Good for lists for instance.
\usepackage{xparse}             % List utilities.

%% Page utilities
\usepackage{multicol}			% Allows to divide a part of the page in multiple columns.
\usepackage{fancyhdr}		% For custom headers and foot texts
\pagestyle{fancy}
	
%% Others
\usepackage{xstring}            % String parsing, cutting, etc.
\usepackage[hidelinks, bookmarks=false, pdfdisplaydoctitle=true, pdfstartview=FitH, pdfpagelabels=false]{hyperref} % Links in PDF.

\makeatletter

%%% Language specific stuff


%%% Commands %%%

\newcommand{\addtosortedlist}[1]{%
	\protected@edef\textarg{#1}%
	\protected@edef\textwithoutspaces{\expandafter\removespaces\expandafter{\textarg}}%
	\substitute\textwithoutspaces{É}{e}% Most used special characters of the language, and equivalent for alphabetical ordering
	\substitute\textwithoutspaces{È}{e}%
	\substitute\textwithoutspaces{Ê}{e}%
	\substitute\textwithoutspaces{é}{e}%
	\substitute\textwithoutspaces{è}{e}%
	\substitute\textwithoutspaces{ê}{e}%
	\substitute\textwithoutspaces{À}{a}%
	\substitute\textwithoutspaces{à}{a}%
	\substitute\textwithoutspaces{ù}{u}%
	\expandafter\sortitem\expandafter[\textwithoutspaces]{#1}%
}%


%%% Labels %%%

% Profile

\newcommand{\labels@M}{M}
\newcommand{\labels@WS}{CC}
\newcommand{\labels@BS}{CT}
\newcommand{\labels@S}{F}
\newcommand{\labels@T}{E}
\newcommand{\labels@W}{PV}
\newcommand{\labels@I}{I}
\newcommand{\labels@A}{A}
\newcommand{\labels@Ld}{Cd}
\newcommand{\labels@Invocation}{Invocation} % For Vampire Covenant profiles

\newcommand{\Strength}{Force}

% Technical

\newcommand{\labels@range}{Portée}
\newcommand{\labels@point}{pt}
\newcommand{\labels@points}{pts}
\newcommand{\labels@only}{uniquement}
\newcommand{\labels@magic}{Magie}
\newcommand{\labels@pathsused}{Génère ses sorts dans la Discipline}
\newcommand{\labels@model}{figurine}
\newcommand{\labels@models}{figurines}
\newcommand{\labels@Singlemodel}{Figurine \textbf{seule}}

% Unit entry labels

\newcommand{\labels@basesize}{Socle}
\newcommand{\labels@trooptype}{Type de troupe}
\newcommand{\labels@specialrules}{Règles spéciales}
\newcommand{\labels@alignment}{Allégeance}
\newcommand{\labels@equipment}{Équipement}
\newcommand{\labels@weapons}{Armes}
\newcommand{\labels@armour}{Armure}
\newcommand{\labels@options}{Options}
\newcommand{\labels@commandgroup}{État-Major}
\newcommand{\labels@mounts}{Montures}
\newcommand{\labels@specialequipment}{Équipement spécial}

% Command groups

\newcommand{\labels@champion}{Champion}
\newcommand{\labels@standardbearer}{Porte-étendard}
\newcommand{\labels@musician}{Musicien}
\newcommand{\labels@singlebannerallowance}{Une seule unité de ce type peut prendre une Bannière magique}
\newcommand{\labels@condsinglebannerallowance}{Une seule unité de ce type peut prendre une Bannière magique si}
\newcommand{\labels@bannerallowance}{Peut prendre une Bannière Magique}
\newcommand{\labels@veteranstandardbearer}{Peut devenir Porte-étendard Vétéran}
\newcommand{\labels@championallowance}{Peut prendre une Arme Magique}

% Titles

\newcommand{\labels@lords}{Seigneurs}
\newcommand{\labels@heroes}{Héros}
\newcommand{\labels@coreunits}{Unités de base}
\newcommand{\labels@specialunits}{Unités spéciales}
\newcommand{\labels@rareunits}{Unités rares}
\newcommand{\labels@armywiderules}{Règles communes de l'armée}
\newcommand{\labels@armyspecialrules}{Règles spéciales de l'armée}
\newcommand{\labels@armoury}{Armurerie}
\newcommand{\labels@magicalitems}{Objets magiques}
\newcommand{\labels@magicalweapons}{Armes magiques}
\newcommand{\labels@magicalarmour}{Armures magiques}
\newcommand{\labels@talismans}{Talismans}
\newcommand{\labels@enchanteditems}{Objets enchantés}
\newcommand{\labels@arcaneitems}{Objets cabalistiques}
\newcommand{\labels@magicalbanners}{Bannières magiques}
\newcommand{\labels@quickrefsheet}{Fiche de référence}
\newcommand{\labels@changelog}{Change Log}

\newcommand{\labels@lordsInitial}{S}
\newcommand{\labels@heroesInitial}{H}
\newcommand{\labels@coreunitsInitial}{B}
\newcommand{\labels@specialunitsInitial}{S}
\newcommand{\labels@rareunitsInitial}{R}
\newcommand{\labels@mountsInitial}{M}


% Titlepage

\newcommand{\labels@fantasybattles}{Batailles Fantastiques}
\newcommand{\labels@NinthAge}{Le 9\ieme Âge}
\newcommand{\labels@creators}{Une collaboration des créateurs de l'ETC et du Swedish Comp System}
\newcommand{\labels@introduction}{%
\noindent {\Largerfontsize\textbf{Note des traducteurs}}
\vspace{0.5cm}

Nous souhaitons remercier chaleureusement l'équipe à l'initiative du 9\ieme Âge pour leur motivation et leur travail continu pour faire vivre notre passion. Nous espérons que ce jeu saura développer les qualités pour plaire au plus grand nombre et réunir les joueurs, amateurs comme habitués des tournois, autour de règles amusantes et équilibrées, pour finalement s'imposer comme un standard du jeu de figurines. Une grande ambition qui ne pourra s'accomplir que \textbf{grâce à vous}, la communauté, via des retours constructifs, afin de modeler le jeu selon nos désirs. N'étant \textbf{en aucun cas à but lucratif}, le 9\ieme Âge part avec un avantage considérable. Les règles des éventuelles nouvelles sorties ne seront pas dictées par le besoin de vendre ces nouveautés. Vous pouvez choisir et acheter vos figurines où bon vous semble, il n'y a pas un unique revendeur toléré. Vous n'êtes pas bloqués dans une spirale infernale où pour continuer à jouer à un jeu, dans lequel vous vous êtes tant investis, vous devez payer toujours plus cher pour entretenir votre collection. Enfin, vous pouvez être assurés que tant que 9\ieme Âge sera joué, vous disposerez d'un \textbf{support continu et régulier}, celui-ci étant offert par la communauté.

Nous attirons votre attention sur le fait que ce jeu en est encore à ses débuts et dans un \textbf{stade de développement}. Ce document correspond à une version de brouillon \textbf{\og{} beta \fg{}}, dont le but et de tester le jeu et le modifier jusqu'à atteindre une version satisfaisante. Attendez-vous donc à trouver des déséquilibres, des incohérences, et à obtenir des mises à jour régulières avec éventuellement des changements importants. N'hésitez pas à nous donner vos avis ! Ce livre d'armée n'est utilisable qu'en compagnie du livre de Règles et du livre de Magie.

Concernant la traduction en elle-même, nous avons fait de notre mieux pour vous offrir une version de qualité, dont nous espérons qu'elle surpasse celle de la version originale ! Si vous constatez des coquilles, des erreurs, merci de nous les signaler en nous contactant sur le forum du 9\ieme Âge, dans le \textbf{sous-forum français} (\url{http://www.the-ninth-age.com/index.php?board/117-french/}). Vous y trouverez aussi les dernières mises à jour. \textbf{En cas de conflit d'interprétation avec la version originale, la version originale fait référence}.

\vspace{0.5cm}
Que ce jeu vous apporte d'innombrables heures de plaisir partagé !

\vspace{0.7cm}
\noindent {\Largerfontsize\textbf{Les traducteurs}}
\vspace{0.1cm}

\ifdef{\translationteam}{
	\begin{multicols}{3}
	\begin{itemize}
		\translationteam
	\end{itemize}
	\end{multicols}
}{}
}
\newcommand{\labels@secondpageannouncement}{%
	\labels@fantasybattles{} : \labels@NinthAge{} est un jeu créé et entretenu par la communauté qui met en scène des affrontements de figurines. Toutes les règles sont disponibles gratuitement sur le site suivant. Vos retours et suggestions sont les bienvenus.
	\newline\url{http://www.the-ninth-age.com/}
}
\newcommand{\labels@rulechanges}{%
	Les changements de règles entre versions sont colorés comme ce paragraphe. Une liste en anglais de ces changements par version est ajoutée à la fin de cet ouvrage.
}
\newcommand{\labels@latexcredit}{Document réalisé à l'aide de \LaTeX .}


%%% Technical commands

\newcommand{\only}[1]{(#1 uniquement)}
\newcommand{\free}{gratuit}
\newcommand{\upto}{jusqu'à}
\newcommand{\Upto}{Jusqu'à}
\newcommand{\unlimited}{sans limite de pts}
\newcommand{\permodel}{/fig.}
\newcommand{\listlastchoice}{ ou}
\newcommand{\notif}[1]{(pas #1)}
\newcommand{\wordand}{et}
\newcommand{\wordwith}{avec}
\newcommand{\ifNmodelsorless}[1]{(#1 figurines ou moins)}
\newcommand{\unitwith}{unité avec}
\newcommand{\From}{De} % From ... to ... models
\newcommand{\wordto}{à}
\newcommand{\wordAll}{Tous}
\newcommand{\spacebeforecolon}{ } % French put a space before colons
\newcommand{\minprice}{Coût min. :}
\newcommand{\mincostfor}{Coût min. pour}
\newcommand{\maxunitsize}{Taille max.}
\newcommand{\additionalfigscost}{Les figurines additionnelles coûtent}


%%% Special rules %%%

\newcommand{\ambush}{Embuscade}
\newcommand{\armourpiercing}[1]{Perforant\ifblank{#1}{}{ (#1)}}
\newcommand{\bodyguard}[1]{Garde du Corps\ifblank{#1}{}{ (#1)}}
\newcommand{\breathweapon}[1]{Attaque de Souffle\ifblank{#1}{}{ (#1)}}
\newcommand{\channel}{Canalisation}
\newcommand{\crushattack}{Attaque Écrasante}
\newcommand{\devastatingcharge}{Charge Dévastatrice}
\newcommand{\distracting}{Distrayant}
\newcommand{\engineer}{Ingénieur}
\newcommand{\ethereal}{Éthéré}
\newcommand{\fastcavalry}{Cavalerie Légère}
\newcommand{\fear}{Peur}
\newcommand{\fightinextrarank}{Combat avec un Rang Supplémentaire}
\newcommand{\fireborn}{Né du Feu}
\newcommand{\flamingattacks}{Attaques Enflammées}
\newcommand{\flammable}{Inflammable}
\newcommand{\lighttroops}{Troupes Légères}
\newcommand{\frenzy}{Frénésie}
\newcommand{\fly}[1]{Vol\ifblank{#1}{}{ (#1)}}
\newcommand{\grindingattacks}[1]{Attaques de Broyage\ifblank{#1}{}{ (#1)}}
\newcommand{\hardtarget}{Camouflé}
\newcommand{\hatred}{Haine}
\newcommand{\hellfire}{Flammes de l'Enfer}
\newcommand{\hidden}{Caché}
\newcommand{\holyattacks}{Attaques Divines}
\newcommand{\immunetopsychology}{Immunisé à la Psychologie}
\newcommand{\impacthits}[1]{Touches d'Impact\ifblank{#1}{}{ (#1)}}
\newcommand{\insignificant}{Insignifiant}
\newcommand{\largetarget}{Grande Cible}
\newcommand{\lethalstrike}{Coup Fatal}
\newcommand{\lightningattacks}{Attaques Foudroyantes}
\newcommand{\lightningreflexes}{Réflexes Foudroyants}
\newcommand{\magicresistance}[1]{Résistance à la Magie\ifblank{#1}{}{ (#1)}}
\newcommand{\magicalattacks}{Attaques Magiques}
\newcommand{\metalshifting}{Fusion du Métal}
\newcommand{\moveorfire}{Mouvement ou Tir}
\newcommand{\multipleshots}[1]{Tirs Multiples\ifblank{#1}{}{ (#1)}}
\newcommand{\multiplewounds}[2]{Blessures Multiples\ifblank{#1}{}{ (#1\ifblank{#2}{)}{, #2)}}}
\newcommand{\notaleader}{Pas un Meneur}
\newcommand{\otherworldly}{D'Outre-Monde}
\newcommand{\pathmaster}[1]{Maître de la Discipline\ifblank{#1}{}{ (#1)}}
\newcommand{\poisonedattacks}{Attaques Empoisonnées}
\newcommand{\quicktofire}{Tir Rapide}
\newcommand{\randommovement}[1]{Mouvement Aléatoire\ifblank{#1}{}{ (#1)}}
\newcommand{\randomattacks}[1]{Attaques Aléatoires\ifblank{#1}{}{ (#1)}}
\newcommand{\regeneration}[1]{Régénération\ifblank{#1}{}{ (#1+)}}
\newcommand{\reload}{Rechargez !}
\newcommand{\requirestwohands}{Arme à deux Mains}
\newcommand{\scythes}{Faux}
\newcommand{\scout}{Éclaireur}
\newcommand{\scouts}{Éclaireurs}
\newcommand{\stomp}[1]{Piétinement\ifblank{#1}{}{ (#1)}}
\newcommand{\strider}[1]{Guide\ifblank{#1}{}{ (#1)}}
\newcommand{\stubborn}{Tenace}
\newcommand{\stupidity}{Stupidité}
\newcommand{\skirmisher}{Tirailleur}
\newcommand{\skirmishers}{Tirailleurs}
\newcommand{\sweepingattack}{Attaque au Passage}
\newcommand{\swiftstride}{Rapide}
\newcommand{\thunderouscharge}{Charge Tonitruante}
\newcommand{\terror}{Terreur}
\newcommand{\toxicattacks}{Attaques Toxiques}
\newcommand{\unbreakable}{Indémoralisable}
\newcommand{\undead}{Mort-Vivant}
\newcommand{\unstable}{Instable}
\newcommand{\unwieldy}{Encombrant}
\newcommand{\vanguard}{Avant-Garde}
\newcommand{\volleyfire}{Tir de Volée}
\newcommand{\warplatform}{Plateforme de Guerre}
\newcommand{\wardsave}[1]{Sauvegarde Invulnérable\ifblank{#1}{}{ (#1+)}}
\newcommand{\weaponmaster}{Maître d'Ar\-mes}
\newcommand{\wizardconclave}[1]{Conclave de Sorciers\ifblank{#1}{}{ (#1)}}


%%% Magic %%%

\newnamemacro{\Pathof}{Discipline}

\newcommand{\battle}{Commune}
\newcommand{\alchemy}{de l'Alchimie}
\newcommand{\death}{de la Mort}
\newcommand{\fire}{du Feu}
\newcommand{\heavens}{des Cieux}
\newcommand{\light}{de la Lumière}
\newcommand{\nature}{de la Nature}
\newcommand{\shadows}{des Ombres}
\newcommand{\wilderness}{de la Sauvagerie Bestiale}
\newcommand{\butchery}{de la Boucherie}
\newcommand{\change}{du Changement}
\newcommand{\thebiggreengods}{des Grands Dieux Verts}
\newcommand{\thelittlegreengods}{des Petits Dieux Verts}
\newcommand{\blackmagic}{de la Magie Noire}
\newcommand{\disease}{de la Maladie}
\newcommand{\lust}{de la Luxure}
\newcommand{\necromancy}{de la Nécromancie}
\newcommand{\ruin}{de la Ruine}
\newcommand{\forge}{de la Forge}
\newcommand{\sands}{des Sables}
\newcommand{\whitemagic}{de la Magie Blanche}

\newcommand{\anyofthebattlemagic}{dans n'importe laquelle des Disciplines Communes}

\newcommand{\magiclevel}[1]{\ifnumcomp{#1}{<}{3}{Sorcier Apprenti}{Maître Sorcier} Niveau #1}
\newcommand{\Level}{Niveau}

\newcommand{\wizard}{Sorcier}
\newcommand{\wizards}{Sorciers}

\newcommand{\boundspell}[1]{Objet de Sort, Puissance #1}


%%% Other rules %%%

\newcommand{\armoursave}{Sauvegarde d'Armure}
\newcommand{\firstinrank}{Au Premier Rang}
\newcommand{\hardcover}{Couvert Lourd}
\newcommand{\holdyourground}{Tenez les Rangs}
\newcommand{\inspiringpresence}{Présence Charismatique}
\newcommand{\lightcover}{Couvert Léger}
\newcommand{\monstrousrank}{Rang Monstrueux}
\newcommand{\ordnance}{Artillerie}
\newcommand{\parry}{Parade}
\newcommand{\raisewounds}{Ressusciter des Figurines}
\newcommand{\recoverwounds}{Récupérer des PVs}
\newcommand{\aideddispel}{Dissipation Assistée}
\newcommand{\rnf}{ordinaires}
\newcommand{\general}{Général}


%%% Equipment %%%

\newcommand{\innatedefence}[1]{Protection Innée\ifblank{#1}{}{~(#1+)}}
\newcommand{\mountsprotection}[1]{Protection de Monture\ifblank{#1}{}{~(#1+)}}
\newcommand{\la}{Armure Légère}
\newcommand{\ha}{Armure Lourde}
\newcommand{\platearmour}{Armure de Plates}
\newcommand{\hw}{Arme de Base}
\newcommand{\pw}{Paire d'Armes}
\newcommand{\spear}{Lance}
\newcommand{\halberd}{Hallebarde}
\newcommand{\gw}{Arme Lourde}
\newcommand{\lance}{Lance de Cavalerie}
\newcommand{\lightlance}{Lance Légère}
\newcommand{\shield}{Bouclier}
\newcommand{\barding}{Caparaçon}
\newcommand{\throwingweapons}{Armes de Jet}
\newcommand{\shortbow}{Arc Court}
\newcommand{\flail}{Fléau}

\newcommand{\cannon}{Canon}
\newcommand{\catapult}{Catapulte}
\newcommand{\volleygun}{Batterie de Tir}
\newcommand{\boltthrower}{Baliste}
\newcommand{\artilleryweapon}{Arme d'Artillerie}


%%% Troop types %%%

\newcommand{\characters}{Personnages}
\newcommand{\infantry}{Infanterie}
\newcommand{\monstrousinfantry}{Infanterie Monstrueuse}
\newcommand{\cavalry}{Cavalerie}
\newcommand{\monstrouscavalry}{Cavalerie Monstrueuse}
\newcommand{\swarm}{Nuée}
\newcommand{\swarms}{Nuées}
\newcommand{\warbeast}{Bête de Guerre}
\newcommand{\warbeasts}{Bêtes de Guerre}
\newcommand{\monster}{Monstre}
\newcommand{\monsters}{Monstres}
\newcommand{\monstrousbeast}{Bête Monstrueuse}
\newcommand{\monstrousbeasts}{Bêtes Monstrueuses}
\newcommand{\chariot}{Char}
\newcommand{\chariots}{Chars}
\newcommand{\riddenmonster}{Monstre Monté}
\newcommand{\riddenmonsters}{Monstres Montés}
\newcommand{\warmachine}{Machine de Guerre}
\newcommand{\warmachines}{Machines de Guerre}


%%% Terrain %%%

\newcommand{\water}{Eaux peu profondes}


%%% Profile wording

\newcommand{\oneofakind}{Uni\-que}
\newcommand{\onechoiceonly}{(un seul choix)}
\newcommand{\onfootonly}{(à pied seulement)}
\newcommand{\closecombatonly}{seulement au Corps à Corps}
\newcommand{\Xmodelsorless}[1]{(#1 figurines ou moins)}
\newcommand{\magicalitemsallowance}{Peut prendre des Objets Magiques}
\newcommand{\magicalweaponallowance}{Peut prendre une Arme Magique}
\newcommand{\notmagicalarmour}{(mais pas d'Armure Magique)}
\newcommand{\anyofthefollowing}{\optionschoice{Peut prendre :}}
\newcommand{\weapononechoice}{\optionschoice{Peut prendre une arme \onechoiceonly{} :}}
\newcommand{\weaponschoice}{\optionschoice{Peut prendre des armes :}}
\newcommand{\shootingweapononechoice}{\optionschoice{Peut prendre une arme de tir \onechoiceonly{} :}}
\newcommand{\combatweapononechoice}{\optionschoice{Peut prendre une arme de corps à corps \onechoiceonly{} :}}
\newcommand{\armouronechoice}{\optionschoice{Peut prendre une armure \onechoiceonly{} :}}
\newcommand{\magiclevelchoice}{\optionschoice{Peut devenir au choix :}}
\newcommand{\bsboption}{Peut devenir Porteur de la Grande Bannière}
\newcommand{\mayupgradeto}{Peut être amélioré en}
\newcommand{\mustbecomeoneofthefollowing}{\optionschoice{Doit devenir un choix parmi :}}
\newcommand{\maybecomeoneofthefollowing}{\optionschoice{Peut devenir un choix parmi :}}
\newcommand{\maytakeoneofthefollowing}{\optionschoice{Peut prendre un choix parmi :}}
\newcommand{\maytakeuptotwoofthefollowing}{\optionschoice{Peut prendre jusqu'à deux choix parmi :}}
\newcommand{\maygain}{Peut gagner la règle}
\newcommand{\maytake}{Peut prendre}
\newcommand{\maytakeashield}{Peut prendre un Bouclier}
\newcommand{\maytakela}{Peut prendre une Armure Légère}
\newcommand{\maytakeha}{Peut prendre une Armure Lourde}
\newcommand{\maytakemountsprotectionX}[1]{Peut prendre une \mountsprotection{#1}}
\newcommand{\maytakeagw}{Peut prendre une Arme Lourde}
\newcommand{\maytakeaspear}{Peut prendre une Lance}
\newcommand{\maytakepw}{Peut prendre une Paire d'Armes}
\newcommand{\maytakethrowingweapons}{Peut prendre des Armes de Jet}
\newcommand{\maytakebarding}{Peut prendre un Caparaçon}
\newcommand{\replaceshieldwithhalberd}{Remplacer le Bouclier par une Hallebarde}
\newcommand{\maybecome}{Peut devenir}

\newcommand{\maytakeonechoiceonly}{\optionschoice{\maytake{} \onechoiceonly{}\spacebeforecolon{}:}}

\newcommand{\mountssectionannouncement}{%
La section Montures concerne les montures de Personnages. Les montures pour non-Personnages suivent les règles données dans leur description d'unité.
}


%%% Technical commands %%%

\newcommand{\amel}[1]{\textcolor{blue}{[#1]}}
\newcommand{\base}{\textcolor{red}}
\newcommand{\amelbis}[1]{\textcolor{olive}{[[#1]]}}

\newcommand{\newrule}{\textcolor{green!50!black}}
\newcommand{\removedrule}[1]{\textcolor{green!50!black}{\sout{#1}}}
\newcommand{\starsymbol}{$\star$}
\newcommand{\refsymbol}{$^\star$}

\newcommand{\inch}{\arcsecond}
\newcommand{\foot}{\arcminute}
\newcommand{\range}[1] {\labels@range~\unit{#1}{\inch}}
\newcommand{\distance}[1] {\unit{#1}{\inch}}
\newcommand{\result}[1] {\texttt{'}#1\texttt{'}}
\newcommand{\plusone}{+1}

\newcommand{\verysmallfontsize}{\fontsize{4}{4.8}\selectfont}
\newcommand{\smallfontsize}{\fontsize{6}{7.2}\selectfont}
\newcommand{\normalfontsize}{\fontsize{8}{9.6}\selectfont}
\newcommand{\largefontsize}{\fontsize{10}{12}\selectfont}
\newcommand{\largerfontsize}{\fontsize{12}{14.4}\selectfont}
\newcommand{\Largefontsize}{\fontsize{14}{16.8}\selectfont}
\newcommand{\Largerfontsize}{\fontsize{15}{18}\selectfont}
\newcommand{\hugefontsize}{\fontsize{18}{21.6}\selectfont}
\newcommand{\Hugefontsize}{\fontsize{25}{30}\selectfont}

%%% Table of Contents %%%

\newcommand{\toctarget}[1]{%
\phantomsection\label{#1}%
\hypertarget{#1}%
}

\newcommand{\tocentry}[2]{%
\noindent\hyperlink{#1}{#2}\dotfill\pageref{#1}%
}


%%% Headers %%%

\renewcommand{\headrulewidth}{0pt}
\fancyfoot[L]{\textcolor{black!30}{%
%\hyperlink{lordtitle}{\labels@lordsInitial}\hspace*{0.4cm}
%\hyperlink{herotitle}{\labels@heroesInitial}\hspace*{0.4cm}
%\hyperlink{coretitle}{\labels@coreunitsInitial}\hspace*{0.4cm}
%\hyperlink{specialtitle}{\labels@specialunitsInitial}\hspace*{0.4cm}
%\hyperlink{raretitle}{\labels@rareunitsInitial}\hspace*{0.4cm}
}}
\fancyfoot[R]{\textcolor{black!30}{%
%\hyperlink{lordtitle}{\labels@lordsInitial}\hspace*{0.4cm}
%\hyperlink{herotitle}{\labels@heroesInitial}\hspace*{0.4cm}
%\hyperlink{coretitle}{\labels@coreunitsInitial}\hspace*{0.4cm}
%\hyperlink{specialtitle}{\labels@specialunitsInitial}\hspace*{0.4cm}
%\hyperlink{raretitle}{\labels@rareunitsInitial}\hspace*{0.4cm}
}}


\setlength{\columnsep}{1cm}

%%% Table parameters %%%

\newcolumntype{M}[1]{>{\centering\let\newline\\\arraybackslash\hspace{0pt}}m{#1}}

\renewcommand{\arraystretch}{3.2}

\arrayrulecolor{black!30}
\setlength{\arrayrulewidth}{2pt}

\newcommand{\starttable}[2][black]{%
\vspace{0.3cm}
\begin{center}
\begin{tabular}{@{}>{\bf\LARGE}M{0.7cm}>{\raggedright}m{3cm}M{0.8cm}M{1.8cm}M{1.4cm}m{6.5cm}@{}}
\rowcolor[HTML]{#2} &
\textcolor{#1}{\textbf{Nom}} &
\textcolor{#1}{\textbf{Lancement}} &
\textcolor{#1}{\textbf{Type}} &
\textcolor{#1}{\textbf{Durée}} &
\centering\textcolor{#1}{\textbf{Effet}}
\tabularnewline
}

\newcommand{\closetable}{%
\end{tabular}
\end{center}
}

\def\colors@alchemy{FFD966}
\def\colors@death{434343}
\def\colors@fire{FF0000}
\def\colors@heavens{C9DAF8}
\def\colors@light{FFF2CC}
\def\colors@nature{274E13}
\def\colors@shadows{999999}
\def\colors@wilderness{7F6000}
\def\colors@butchery{85200C}
\def\colors@change{9900FF}
\def\colors@forge{5B0F00}
\def\colors@biggreengods{38761D}
\def\colors@littlegreengods{93C47D}
\def\colors@lust{D5A6BD}
\def\colors@whitemagic{CFE2F3}
\def\colors@blackmagic{20124D}
\def\colors@disease{7F6000}
\def\colors@necromancy{000000}
\def\colors@ruin{69431C}
\def\colors@sand{FFD966}



\newcommand{\booktitle}{Livre de Magie}
\newcommand{\version}{1.2.1}
\newcommand{\frenchversion}{0.1}

\hypersetup{pdftitle={T9A - \booktitle}}

% Document Titles

\newcommand{\howtousethisdoc}{Comment utiliser ce livre ?}
\newcommand{\magicphasesummary}{Résumé de la Phase de Magie}

% Table Titles

\newcommand{\spellsCastingValue}{Lancement}
\newcommand{\spellsType}{Type}
\newcommand{\spellsDuration}{Durée}
\newcommand{\spellsEffect}{Effet}

% Paths TOC

\newcommand{\alchemyTOC}{Alchimie}
\newcommand{\shamanismTOC}{Chamanisme}
\newcommand{\cosmologyTOC}{Cosmologie}
\newcommand{\divinationTOC}{Divination}
\newcommand{\druidismTOC}{Druidisme}
\newcommand{\evocationTOC}{Évocation}
\newcommand{\occultismTOC}{Occultisme}
\newcommand{\pyromancyTOC}{Pyromancie}
\newcommand{\witchcraftTOC}{Witchcraft}
\newcommand{\thaumaturgyTOC}{Thaumaturgie}

\newcommand{\characteronly}{Personnage uniquement}







\begin{document}

\newgeometry{margin=1in}

\begin{titlepage}
\begin{center}

\ifdef{\booktitle}{}{\newcommand{\booktitle}{Missing title}}
\ifdef{\version}{}{\newcommand{\version}{Missing version}}

{\titlefont\fontsize{40}{48}\selectfont\noindent\labels@fantasybattles

\labels@NinthAge}

\vspace*{0.7cm}

\newcommand{\iconheight}{1.6cm}
\newcommand{\minipagewidth}{6cm}
\newcommand{\spacebetweeniconrows}{0.3cm}

\hfill\begin{minipage}[c]{\minipagewidth}
\begin{center}
\includegraphics[height=\iconheight]{\alchemyicon}

\vspace*{\spacebetweeniconrows}

\includegraphics[height=\iconheight]{\shamanismicon}

\vspace*{\spacebetweeniconrows}

\includegraphics[height=\iconheight]{\cosmologyicon}

\vspace*{\spacebetweeniconrows}

\includegraphics[height=\iconheight]{\divinationicon}

\vspace*{\spacebetweeniconrows}

\includegraphics[height=\iconheight]{\druidismicon}

\end{center}
\end{minipage}\begin{minipage}[c]{\minipagewidth}
\begin{center}
\includegraphics[height=\iconheight]{\evocationicon}

\vspace*{\spacebetweeniconrows}

\includegraphics[height=\iconheight]{\occultismicon}

\vspace*{\spacebetweeniconrows}

\includegraphics[height=\iconheight]{\pyromancyicon}

\vspace*{\spacebetweeniconrows}

\includegraphics[height=\iconheight]{\witchcrafticon}

\vspace*{\spacebetweeniconrows}

\includegraphics[height=\iconheight]{\thaumaturgyicon}
\end{center}
\end{minipage}\hspace*{\fill}


\vspace*{0.7cm}

{\titlefont\fontsize{50}{60}\selectfont \booktitle
\vspace{0.4cm}

\fontsize{14}{16.8}\selectfont Beta v\version{} - \today{}}

\ifdef{\frenchversion}{{\fontsize{14}{16.8}\selectfont \vspace{0.2cm}\noindent\texttt{VF \frenchversion}}}{}
\vfill

\begin{tabular}{@{}m{2cm}@{\hskip 20pt}m{13cm}@{}}
\includegraphics[width=2cm]{../Layout/pics/seal_9th.png} &
{\fontsize{10}{12}\selectfont \textcolor{black!50}{\noindent\labels@frontpagecredits}}

\ifdef{\frontpageaddstuff}{{\fontsize{10}{12}\selectfont \noindent\textcolor{black!50}{\frontpageaddstuff}}}{}

\vspace*{10pt}
\noindent{\fontsize{10}{12}\selectfont \textcolor{black!50}{\labels@license}}
\tabularnewline
\end{tabular}


\end{center}

\newpage

\thispagestyle{empty}

{\fontsize{10}{12}\selectfont

\begin{center}\hypertarget{tableofcontents}{\noindent{\Largerfontsize\textbf{\labels@tableofcontents}}}\end{center}

\vspace*{0.2cm}

\begin{center}
\noindent\hyperlink{howtousethisdoc}{\howtousethisdoc}
\end{center}

\begin{multicols}{2}

\tocentry{alchemy}{\alchemyTOC}

\tocentry{shamanism}{\shamanismTOC}

\tocentry{cosmology}{\cosmologyTOC}

\tocentry{divination}{\divinationTOC}

\tocentry{druidism}{\druidismTOC}

\tocentry{evocation}{\evocationTOC}

\tocentry{occultism}{\occultismTOC}

\tocentry{pyromancy}{\pyromancyTOC}

\tocentry{witchcraft}{\witchcraftTOC}

\tocentry{thaumaturgy}{\thaumaturgyTOC}

\end{multicols}

\begin{center}
\noindent\hyperlink{magicphasesummary}{\magicphasesummary}
\end{center}

\ifdef{\labels@introduction}{\vspace{0.7cm}\labels@introduction}{\vphantom{1pt}}
\vfill

\noindent\newrule{\labels@rulechanges}

\bigskip
\noindent \labels@latexcredit
}


\end{titlepage}

\restoregeometry

\basictitle{howtousethisdoc}{\howtousethisdoc}

\spaceaftersection{}

Dans ce livre, toutes les Voies de Magie sont présentées. Elles comprennent les 8 \textbf{Voies Communes} et les \textbf{Voies Spécifiques} des différentes armées. Pour plus d'informations sur les règles du jeu et en particulier la Phase de Magie, merci de vous référer au Livre de Règles.

\vspace*{10pt}
Chaque Voie comprend 8 sorts. Le premier est appelé l’\textbf{Attribut} de la Voie, qui est indiqué par la mention \og A \fg{} à la place du numéro du sort. C'est un sort particulier qui est automatiquement lancé quand un autre sort de la même Voie est lancé avec succès. Les sept sorts suivants sont numérotés de 0 à 6. Le sort 0 est le sort \textbf{Primaire} de la Voie. Un Sorcier peut toujours choisir d'échanger l'un de ses sorts générés aléatoirement par le sort Primaire, et plusieurs Sorciers peuvent connaître ce sort en même temps.

\vspace*{10pt}
Certains sorts disposent de plusieurs valeurs de lancement. La ou les plus grandes sont pour la  ou les versions \textbf{améliorée(s)} du sort. Le joueur doit choisir la version du sort à lancer, sachant que plus la valeur de lancement est grande, plus le sort sera difficile à lancer, mais aussi plus il sera puissant. Parfois, les versions améliorées voient leur portée et le nombre de cibles augmenter, tandis que pour d'autres sorts, les effets peuvent être modifiés. Les différences entre versions améliorée(s) et basique suivent un code couleur. La partie en noir s'applique à toutes les versions du sort. Les textes en \base{rouge} s'appliquent à la version basique, tandis que les textes en \amel{bleu et entre crochets} s'appliquent à la version améliorée. Dans certains cas, il existe une deuxième \augment{} au sort, marquée en \amelbis{vert olive entre accolades}. 



%
%\newbattlepath{alchemy}{\alchemyTOC}
%
%\starttable{\colors@alchemy}
%A &
%\alchemyattribute &
%&
%\range{12} \newline
%\augment{} &
%\lastsoneturn{} &
%La cible gagne +1 en Sauvegarde d'Armure. Aucune figurine ne peut obtenir mieux que 3+ en Sauvegarde d'Armure grâce à ce sort avant d'appliquer d'éventuels malus.
%\tabularnewline
%\hline
%0 &
%\alchemysignature &
%\base{9+}\newline  \amel{17+} &
%\range{24} \newline
%\hex{} \newline
%\missile{} \newline
%\damage{} &
%\instant{} &
%La cible subit \base{1D6} \amel{2D6} touches avec la règle \metalshifting{}.
%\tabularnewline
%\hline
%1 &
%\alchemyspellone{} &
%\base{7+} \newline
%\amel{11+} &
%\base{\range{18}} \newline
%\amel{\range{36}} \newline
%\augment{} &
%\lastsoneturn{} &
%Les Attaques de Tir et de Corps à Corps de la cible ont +1 pour toucher et gagnent les règles \magicalattacks{} et \armourpiercing{+1}.
%\tabularnewline
%\hline
%2 &
%\alchemyspelltwo{} &
%\base{7+} \newline
%\amel{10+} &
%\base{\range{24}} \newline
%\amel{\range{48}} \newline
%\hex{} &
%\permanent{} &
%La cible subit un malus de -1 sur sa Sauvegarde d'Armure.
%\tabularnewline
%\hline
%3 &
%\alchemyspellthree{} &
%\base{8+} \newline
%\amel{11+} &
%\base{\range{12}} \newline
%\amel{\range{24}} \newline
%\augment{} &
%\lastsoneturn{} &
%La cible gagne les règles \hardtarget{} et \distracting{}.
%
%\vspace*{5pt}De plus, les Attaques de Corps à Corps contre la cible subissent également une pénalité de -1 sur leur règle \armourpiercing{}.
%\tabularnewline
%\hline
%4 &
%\alchemyspellfour{} &
%\base{8+} \newline
%\amel{11+} &
%\base{\range{18}} \newline
%\amel{\range{36}} \newline
%\hex{} \newline 
%\missile{} \newline 
%\damage{} &
%\instant{} &
%La cible subit une touche avec les règles \multiplewounds{1D3}{} et \metalshifting{}. Les rangs de l'unité ciblée sont pénétrés de la même façon que par une \boltthrower{}, mais l'attaque subit un malus de -1 pour blesser au lieu de -1 en Force pour chaque rang pénétré. 
%\tabularnewline
%\hline
%5 &
%\alchemyspellfive{}  &
%\base{9+} \newline
%\amel{12+} &
%\base{\range{24}} \newline
%\amel{\range{48}} \newline
%\hex{} &
%\lastsoneturn{} &
%Les Armes de Tir standard portées par l'unité ciblée subissent un malus de -1 en Force. Ce sort n'affecte que les équipements standard et leur Force, aucune règle spéciale n'est altérée par ce sort.
%
%\vspace*{5pt}
%Les figurines de l'unité ciblée ne peuvent pas recevoir de bonus de Force de leurs Armes de Corps à Corps standard.
%\tabularnewline
%\hline
%6 &
%\alchemyspellsix{} &
%\base{15+} \newline
%\amel{18+} &
%\base{\range{12}} \newline
%\amel{\range{24}} \newline
%\hex{} \newline
%\direct{} \newline
%\damage{} &
%\instant{} \newline \lastsoneturn{} &
%Le propriétaire de l'unité ciblée lance un dé pour chaque figurine de l'unité, dans l'ordre de son choix. Ignorez le premier 5+ obtenu. Pour les figurines suivantes, sur 5+, la figurine subit une blessure avec la règle \multiplewounds{10}{} sans aucune sauvegarde autorisée.
%
%\vspace*{5pt}
%Les unités ennemies à moins de \distance{12} de la cible gagnent la règle \stupidity{}.
%\tabularnewline
%\closetable
%
%
%
%\newbattlepath{heavens}{\heavensTOC}
%
%\starttable{\colors@heavens}
%A &
%\heavensattribute{} &
%&
%\specialTYPE{} &
%\lastsoneturn{} &
%L'armée du lanceur gagne un marqueur \og \heavensattribute{} \fg{}. Ce marqueur peut être dépensé pour relancer un unique D6 d'un jet pour toucher, pour blesser ou de Sauvegarde d'Armure.
%\tabularnewline
%\hline
%0 &
%\heavenssignature{} &
%\base{7+}\newline
%\amel{10+} &
%\base{\range{12}} \newline
%\amel{\range{36}} \newline
%\hex{} &
%\lastsoneturn{} &
%La cible subit un malus de -1 pour toucher et de -1 en Commandement. Les Attaques de Tir ne nécessitant pas l'utilisation de la CT doivent obtenir un 4+ sur 1D6 pour pouvoir être utilisées.
%\tabularnewline
%\hline
%1 &
%\heavensspellone{} &
%\base{6+} \newline
%\amel{9+} &
%\base{\range{18}} \newline
%\amel{\range{36}} \newline
%\hex{} &
%\lastsoneturn{} &
%L'unité ciblée ne peut pas se déplacer de plus de \distance{10} durant l'Étape des Autres Mouvements. Toutes les unités ennemies à moins de \distance{6} de la cible lorsque le sort est lancé subissent un malus de -1 en Capacité de Tir.
%\tabularnewline
%\hline
%2 &
%\heavensspelltwo{} &
%\base{8+}\newline
%\amel{11+} &
%\base{\range{24}} \newline
%\amel{\range{48}} \newline
%\hex{} \newline
%\missile{} \newline
%\damage{} &
%\instant{} &
%La cible subit 1D6 touches de Force 6 avec la règle \lightningattacks{}.
%\tabularnewline
%\hline
%3 &
%\heavensspellthree{} &
%\base{9+}\newline
%\amel{12+} &
%\base{\range{12}} \newline
%\amel{\range{24}} \newline
%\augment{} &
%\lastsoneturn{} &
%La cible peut relancer au choix les jets ratés pour toucher, pour blesser ou de Sauvegarde d’Armure. Déclarez votre choix avant de jeter le sort.
%\tabularnewline
%\hline
%4 &
%\heavensspellfour{} &
%\base{9+}\newline
%\amel{12+} &
%\base{\range{12}} \newline
%\amel{\range{24}} \newline
%\hex{} &
%\lastsoneturn{} &
%La cible doit relancer au choix les jets réussis pour toucher, pour blesser ou de Sauvegarde d'Armure. Déclarez votre choix avant de jeter le sort.
%\tabularnewline
%\hline
%5 &
%\heavensspellfive{} &
%13+ &
%\range{24} \newline
%\hex{} \newline
%\direct{} \newline
%\damage{} &
%\instant{} &
%Ciblez une unité, puis jetez 1D6. Sur 3+, choisissez une nouvelle cible située à moins de \distance{6} de la première. Continuez ainsi, de cible en cible à \distance{6} d'écart, jusqu'à ce que vous obteniez un \result{1} ou un \result{2}, ou qu'aucune cible ne soit éligible. Aucune unité ne peut être ciblée deux fois.
%
%\vspace*{5pt}
%Chaque cible subit 1D6 touches de Force 6 avec la règle \lightningattacks{}.
%\tabularnewline
%\hline
%6 &
%\heavensspellsix{} &
%\base{13+}\newline
%\amel{16+} &
%\ground{} &
%\permanent{} &
%Posez un marqueur où vous voulez sur le champ de bataille.
%
%\vspace*{5pt}
%\base{À la fin de chaque Phase de Magie suivante, jetez un dé. Si le résultat est de 4+, la comète arrive.}
%
%\vspace*{5pt}
%\amel{Le lanceur choisit le Tour de Joueur, autre que celui en cours, et hors premier Tour de Jeu, où la comète arrive et l'écrit secrètement. La comète arrivera à la fin de la Phase de Magie de ce tour de joueur.}
%
%\vspace*{5pt}
%À la fin de chaque Phase de Magie pendant laquelle la comète n'est pas arrivée, ajoutez un marqueur au même endroit. Quand la comète s'écrase, toutes les unités situées dans un rayon de \distance{2D6+X} subissent 2D6 touches de Force 4+X, où X est égal au nombre de marqueurs. Enlevez ensuite tous les marqueurs, et le sort prend fin.
%\tabularnewline
%\closetable
%
%
%
%\newbattlepath{fire}{\fireTOC}
%
%\starttable[white]{\colors@fire}
%A &
%\fireattribute{} &
%&
%\range{24} \newline
%\hex{} \newline
%\missile{} \newline
%\damage{} &
%\instant{} &
%La cible subit 1D3 touches de Force 4 avec la règle \flamingattacks{}.
%\tabularnewline
%\hline
%0 &
%\firesignature{} &
%\base{5+}\newline
%\amel{10+} \newline
%\amelbis{14+} &
%\base{\range{24}} \newline
%\amel{\range{36}} \newline
%\amelbis{\range{48}} \newline
%\hex{} \newline
%\missile{} \newline
%\damage{} &
%\instant{} &
%La cible subit \base{1D6} \amel{2D6} \amelbis{3D6} touches de Force 4 avec la règle \flamingattacks{}.
%\tabularnewline
%\hline
%1 &
%\firespellone{} &
%\base{7+}\newline
%\amel{10+} &
%\base{\range{18}} \newline
%\amel{\range{36}} \newline
%\hex{} &
%\remainsinplay{} &
%À la fin de chaque Phase de Magie, la cible subit 1D6 touches de Force 4 avec la règle \flamingattacks{}.
%\tabularnewline
%\hline
%2 &
%\firespelltwo{} &
%\base{7+}\newline
%\amel{13+} &
%\base{\range{24}} \newline
%\amel{\range{6}} \newline
%\amel{\aura{}} \newline
%\augment{} &
%\lastsoneturn{} &
%Les Attaques de Tir et de Corps à Corps de la cible ont +1 pour blesser et gagnent les règles \flamingattacks{} et \magicalattacks{}.
%\tabularnewline
%\hline
%3 &
%\firespellthree{} &
%\base{9+}\newline
%\amel{12+} &
%\base{\range{18}} \newline
%\amel{\range{36}} \newline
%\ground{} \newline
%\direct{} \newline
%\linetemplate{} &
%\instant{} &
%Chaque figurine sous le gabarit subit une touche de Force 4 avec la règle \flamingattacks{}. Les unités ennemies touchées par le gabarit doivent effectuer un test de Panique.
%\tabularnewline
%\hline
%4 &
%\firespellfour{} &
%\base{9+}\newline
%\amel{12+} &
%\base{\range{24}} \newline
%\amel{\range{36}} \newline
%\hex{} \newline
%\missile{} \newline
%\damage{} &
%\instant{} &
%La cible subit 1 touche de Force 4 avec la règle \flamingattacks{} pour chaque rang ou colonne, au choix du lanceur, dans l'unité ciblée.
%\tabularnewline
%\hline
%5 &
%\firespellfive{} &
%\base{10+}\newline
%\amel{13+} &
%\base{\range{24}} \newline
%\amel{\range{48}} \newline
%\hex{} &
%\remainsinplay{} &
%La cible subit 1D6 touches de Force 4 avec la règle \flamingattacks{}.
%
%\vspace*{5pt}
%À la fin de chaque phase, chaque figurine de l'unité subit une touche de Force 4 avec la règle \flamingattacks{} si l'unité a effectué une ou plusieurs des actions suivantes durant la phase :
%
%Charge (réussie ou non), Charge Irrésistible, Fuite, Marche Forcée, Mouvement Simple, Pivot, Poursuite, Reformation ou Reformation de Combat.
%
%Une même unité ne peut subir ces touches qu'une fois par Phase.
%\tabularnewline
%\hline
%6 &
%\firespellsix{} &
%\base{11+}\newline
%\amel{14+} &
%\base{\range{24}} \newline
%\amel{\range{48}} \newline
%\augment{} &
%\lastsoneturn{} &
%La cible gagne +1 en Endurance, une Sauvegarde Invulnérable (5+) et la règle \fireborn{}.
%\tabularnewline
%\closetable
%
%
%
%\newbattlepath{light}{\lightTOC}
%
%\starttable{\colors@light}
%A &
%\lightattribute{} &
%&
%\range{48} \newline
%\augment{} &
%\lastsoneturn{} &
%La cible gagne +1 en Commandement. Aucune figurine ne peut être affectée par ce sort plus d'une fois par Phase de Magie.
%\tabularnewline
%\hline
%0 &
%\lightsignature{} &
%\base{5+}\newline
%\amel{14+} &
%\base{\range{24}} \newline
%\amel{\range{48}} \newline
%\hex{} \newline
%\missile{} \newline
%\damage{} &
%\instant{} &
%La cible subit 1D6 touches de Force \base{4} \amel{6} avec la règle \flamingattacks{}.
%
%\vspace*{5pt}
%Si la cible dispose de la règle \otherworldly{} ou \undead{}, elle subit 2D6 touches à la place.
%\tabularnewline
%\hline
%1 &
%\lightspellone{} &
%\base{7+}\newline
%\amel{10+} &
%\base{\range{24}} \newline
%\amel{\range{6}} \newline
%\amel{\aura{}} \newline
%\augment{} &
%\lastsoneturn{} &
%La cible gagne les règles \hardtarget{} et \distracting{}. Les Attaques de Tir ciblant l'unité ne nécessitant pas l'utilisation de la CT doivent obtenir un 4+ sur 1D6 pour pouvoir être utilisées.
%\tabularnewline
%\hline
%2 &
%\lightspelltwo{} &
%7+ &
%\range{24} \newline
%\augment{} &
%\instant{}\newline
%\lastsoneturn{} &
%Si la cible n'est ni engagée dans un combat ni en fuite, elle peut effectuer immédiatement une Reformation. Si elle est engagée au corps à corps, elle peut effectuer une Reformation de Combat.
%
%\vspace*{5pt}
%De plus, la cible apporte un bonus de +2 au Résultat de Combat de son camp.
%\tabularnewline
%\hline
%3 &
%\lightspellthree{} &
%\base{8+}\newline
%\amel{12+} &
%\base{\range{24}} \newline
%\amel{\range{12}} \newline
%\amel{\aura{}} \newline
%\augment{} &
%\lastsoneturn{} &
%La cible gagne +3 en Capacité de Combat et en Initiative.
%\tabularnewline
%\hline
%4 &
%\lightspellfour{} &
%\base{9+}\newline
%\amel{12+} &
%\base{\range{24}} \newline
%\amel{\range{48}} \newline
%\hex{} &
%\lastsoneturn{} &
%Au début de chaque phase, jetez un dé. Sur 5+, la cible ne peut pas effectuer les actions suivantes correspondant à la phase :
%
%\vspace*{5pt}
%\textbf{Phase de Mouvement} : Déclarer une charge.
%
%\textbf{Phase de Magie} : Lancer des sorts.
%
%\textbf{Phase de Tir} : Tirer.
%
%\textbf{Phase de Corps à Corps} : Poursuivre et effectuer une Charge Irrésistible.
%\tabularnewline
%\hline
%5 &
%\lightspellfive{} &
%\base{10+}\newline
%\amel{15+} &
%\range{12} \newline
%\augment{} \newline
%\amel{\aura} &
%\lastsoneturn{} &
%La cible gagne +1 Attaque, la règle \divineattacks{} et double son Mouvement jusqu'à un maximum de 10.
%\tabularnewline
%\hline
%6 &
%\lightspellsix{} &
%\base{10+}\newline
%\amel{13+} &
%\range{24} \newline
%\hex{} \newline
%\missile{} \newline
%\damage{} &
%\instant{} &
%La cible subit 2D6 touches de Force 4 avec la règle \divineattacks{}. Pour chaque autre Sorcier connaissant au moins un sort (non Objet de Sort) de la Voie \light{} à moins de \distance{12} du lanceur, ajoutez +1 \base{pour blesser} \amel{en Force}.
%
%\vspace*{5pt}
%Si la cible dispose de la règle \otherworldly{} ou \undead{}, elle subit 3D6 touches à la place.
%\tabularnewline
%\closetable
%
%
%
%
%
%\newbattlepath{death}{\deathTOC}
%
%\starttable[white]{\colors@death}
%A &
%\deathattribute{} &
% &
%\range{24} \newline
%\hex{} &
%\lastsoneturn{} &
%La cible subit un malus de -1 en Commandement. Aucune figurine ne peut être affectée par ce sort plus d'une fois par Phase de Magie.
%\tabularnewline
%\hline
%0 &
%\deathsignature{} &
%\base{8+}\newline
%\amel{11+} &
%\range{18} \newline
%\focused{} \newline
%\hex{} \newline
%\direct{} \newline
%\damage{} &
%\instant{} &
%La cible subit 1 blessure avec la règle \armourpiercing{6}.
%
%\vspace*{5pt}
%\amel{Si la blessure n'est pas sauvegardée, le lanceur Récupère 1 PV}.
%\tabularnewline
%\hline
%1 &
%\deathspellone{} &
%\base{6+}\newline
%\amel{9+} &
%\range{24} \newline
%\hex{} &
%\lastsoneturn{} &
%La cible subit un malus de -1 en Force \amel{et en Endurance}, jusqu'à un minimum de 1.
%\tabularnewline
%\hline
%2 &
%\deathspelltwo{} &
%\base{7+}\newline
%\amel{9+} &
%\range{24} \newline
%\hex{} \newline
%\missile{} \newline
%\damage{} &
%\instant{} &
%La cible subit \base{2D6} \amel{3D6} touches de Force 2 avec la règle \armourpiercing{6}.
%\tabularnewline
%\hline
%3 &
%\deathspellthree{} &
%\base{7+}\newline
%\amel{10+} &
%\base{\range{12}} \newline
%\amel{\range{24}} \newline
%\focused{} \newline
%\hex{} \newline
%\direct{} \newline
%\damage{} &
%\instant{} &
%Le lanceur et la cible lancent chacun 1D6 et ajoutent leur Commandement actuel au résultat. Si le total du lanceur est le plus élevé, la cible subit un nombre de blessures avec la règle \armourpiercing{6} égal à la différence entre leurs totaux respectifs.
%\tabularnewline
%\hline
%4 &
%\deathspellfour{} &
%\base{8+}\newline
%\amel{12+} &
%\range{24} \newline
%\augment{} &
%\lastsoneturn{} &
%Les Attaques de Corps à Corps de la cible gagnent la règle \divineattacks{}.
%
%\vspace*{5pt}
%\amel{Les jets de Sauvegarde d'Armure réussis contre ces attaques doivent être relancés}.
%\tabularnewline
%\hline
%5 &
%\deathspellfive{} &
%10+ &
%\range{18} \newline
%\augment{} &
%\lastsoneturn{} &
%La cible gagne les règles \lethalstrike{} et \fear{}.
%\tabularnewline
%\hline
%6 &
%\deathspellsix{} &
%14+ &
%\vortex{} \newline
%(\range{6}, \template{} \distance{1}) \newline \ground{} &
%\instant{} &
%Les figurines touchées par le gabarit doivent effectuer un test d'Initiative. Chaque figurine échouant subit une blessure avec les règles \multiplewounds{\ordnance}{} et \armourpiercing{6} ne permettant pas la \regeneration{}.
%\tabularnewline
%\closetable
%
%
%
%
%
%\newbattlepath{nature}{\natureTOC}
%
%\starttable[white]{\colors@nature}
%A &
%\natureattribute{} &
%&
%\range{12} \newline
%\focused{} \newline
%\augment{} &
%\instant{} &
%La cible Récupère un PV. Aucune figurine ne peut Récupérer plus d'un PV par Phase de Magie avec ce sort.
%\tabularnewline
%\hline
%0 &
%\naturesignature{} &
%\base{4+}\newline
%\amel{8+} &
%\base{\castersunit} \newline
%\amel{\range{12}} \newline
%\augment{} &
%\lastsoneturn{} &
%La cible gagne la règle \regeneration{5} \amelbis{\regeneration{4} si \naturespelltwo{} est actif}. 
%\tabularnewline
%\hline
%1 &
%\naturespellone{} &
%\base{6+}\newline
%\amel{11+} &
%\range{18} \newline
%\hex{} \newline
%\direct{} \newline
%\damage{} &
%\instant{} &
%La portée de ce sort peut être mesurée à partir du lanceur ou de tout Terrain Infranchissable ou Colline du champ de bataille.
%
%\vspace*{5pt}
%La cible subit \base{1D6} \amel{2D6} touches de Force 4 \amelbis{Force 5 si \naturespelltwo{} est actif}.
%\tabularnewline
%\hline
%2 &
%\naturespelltwo{} &
%7+ &
%\caster{} &
%\remainsinplay{} &
%Si le lanceur subit un Fiasco lorsqu'il lance un sort autre que \naturespelltwo{}, il compte comme ayant lancé un Dé de Pouvoir de moins, jusqu'à un minimum de deux.
%
%\vspace*{5pt}
%Lorsque ce sort est en jeu, le lanceur dispose de versions \amelbis{améliorées} des autres sorts.
%\tabularnewline
%\hline
%3 &
%\naturespellthree{} &
%\base{9+}\newline
%\amel{13+} &
%\range{12} \newline
%\base{\augment} \newline
%\amel{\hex} &
%\lastsoneturn{} &
%La portée de ce sort peut être mesurée à partir du lanceur ou de n'importe quelle Forêt sur le champ de bataille.
%
%\vspace*{5pt}
%Toutes les figurines de l'unité ciblée sont considérées comme étant dans une Forêt.
%\tabularnewline
%\hline
%4 &
%\naturespellfour{} &
%\base{10+}\newline
%\amel{15+} &
%\base{\range{24}} \newline
%\amel{\range{48}} \newline
%\augment{} &
%\instant{} &
%Ressuscitez 1D3+1 PVs \amelbis{1D6+1 PVs si \naturespelltwo{} est actif} dans l'unité ciblée. La quantité de PVs Ressuscités est divisée par deux en arrondissant au supérieur pour les unités de figurines de taille Moyenne et Grande.
%\tabularnewline
%\hline
%5 &
%\naturespellfive{} &
%11+ &
%\range{24} \newline
%\augment{} &
%\lastsoneturn{} &
%La cible gagne +2 en Endurance \amelbis{+4 si \naturespelltwo{} est actif}.
%\tabularnewline
%\hline
%6 &
%\naturespellsix{} &
%\base{15+}\newline
%\amel{18+} &
%\base{\range{12}} \newline
%\amel{\range{24}} \newline
%\hex{} \newline
%\direct{} \newline
%\damage{} &
%\instant{} &
%Chaque figurine de l'unité ciblée doit effectuer un test de Force, dans l'ordre du choix du propriétaire. Ignorez le premier test raté. Toute autre figurine ratant son test subit une blessure sans aucune sauvegarde autorisée avec la règle \multiplewounds{10}{}.
%\tabularnewline
%\closetable
%
%
%
%
%\newbattlepath{shadows}{\shadowsTOC}
%
%\starttable[white]{\colors@shadows}
%A &
%\shadowsattribute{} &
%&
%\range{12} \newline
%\focused{} \newline
%\augment{} &
%\instant{} &
%Ce sort ne peut être lancé que sur un Personnage ou sur une unité constituée d'une seule figurine. La cible peut effectuer un Mouvement Magique de \distance{10} avec la règle \fly{}.
%
%\vspace*{5pt}
%Si la cible dispose de la règle \largetarget{}, ce mouvement est limité à \distance{2}.
%\tabularnewline
%\hline
%0 &
%\shadowssignature{} &
%\base{4+}\newline
%\amel{7+} &
%\range{48} \newline
%\hex{} &
%\lastsoneturn{} &
%La cible subit un malus de \base{-1} \amel{-1D3} à une caractéristique parmi M, CC, CT ou I, jusqu'à un minimum de 1. Le lanceur doit préciser laquelle avant de lancer le sort.
%\tabularnewline
%\hline
%1 &
%\shadowsspellone{} &
%\base{5+}\newline
%\amel{8+} &
%\range{24} \newline
%\ground{} &
%\lastsoneturn{} &
%Placez un gabarit de \distance{3} sur le point ciblé, à plus de \distance{1} de toute unité. La zone couverte par le gabarit compte pour toute figurine comme étant un Terrain Dangereux (1), et offre un Couvert Lourd.
%
%\vspace*{5pt}
%\amel{Cette zone compte également comme un Décor Occultant.}
%\tabularnewline
%\hline
%2 &
%\shadowsspelltwo{} &
%\base{7+}\newline
%\amel{13+} &
%\base{\range{18}} \newline
%\amel{\range{36}} \newline
%\hex{} &
%\remainsinplay{} &
%La cible perd \base{1} \amel{1D3} en Force, jusqu'à un minimum de 1.
%\tabularnewline
%\hline
%3 &
%\shadowsspellthree{} &
%\base{9+}\newline
%\amel{15+} &
%\base{\range{18}} \newline
%\amel{\range{36}} \newline
%\hex{} &
%\remainsinplay{} &
%La cible perd \base{1} \amel{1D3} en Endurance, jusqu'à un minimum de 1.
%\tabularnewline
%\hline
%4 &
%\shadowsspellfour{} &
%\base{11+}\newline
%\amel{14+} &
%\base{\range{12}} \newline
%\amel{\range{24}} \newline
%\augment{} &
%\instant{} &
%La cible peut effectuer un Mouvement Magique de \distance{8} avec la règle \fly{}. À l'issue de ce mouvement, elle peut effectuer une Reformation, ce qui ne l'empêche pas de tirer.
%\tabularnewline
%\hline
%5 &
%\shadowsspellfive{} &
%12+ &
%\range{24} \newline
%\hex{} \newline
%\direct{} \newline
%\damage{} &
%\instant{} &
%Placez le centre d'un gabarit de \distance{3} sur l'unité ciblée, et à portée du sort, puis faites une Déviation de \distance{1D6}. Toutes les figurines sous le gabarit doivent effectuer un test d'Initiative ou subir une blessure avec les règles \multiplewounds{\ordnance}{} et \armourpiercing{6} ne permettant pas la \regeneration{}.
%\tabularnewline
%\hline
%6 &
%\shadowsspellsix{} &
%\base{15+}\newline
%\amel{18+} &
%\base{\range{18}} \newline
%\amel{\range{36}} \newline
%\augment{} &
%\lastsoneturn{} &
%Les Attaques au Corps à Corps non spéciales de la cible blessent automatiquement (aucun jet nécessaire) et gagnent la règle \armourpiercing{1}.
%\tabularnewline
%\closetable
%
%
%
%
%\newbattlepath{wilderness}{\wildernessTOC}
%
%\starttable[white]{\colors@wilderness}
%A &
%\wildernessattribute{} &
%&
%\range{12} \newline
%\augment{} &
%\instant{} &
%Ce sort ne peut affecter que les unités composées uniquement, au choix, de :
%\begin{itemize}[label={-}]
%\item \cavalry{}, \warbeast{}, \monstrouscavalry{}, \monstrousbeast{}, \chariot{}, \monster{} et \riddenmonster{}.
%\item toute unité du Livre d'Armée Hardes Bestiales.
%\item l'unité du lanceur.
%\end{itemize}
%
%La cible peut effectuer un Mouvement Magique de \distance{1D3+2}.
%\tabularnewline
%\hline
%0 &
%\wildernesssignature{} &
%\base{9+} \newline
%\amel{12+} &
%\base{\range{12}} \newline
%\amel{\range{24}} \newline
%\augment{} &
%\lastsoneturn{} &
%La cible gagne +1 en Force et en Endurance.
%\tabularnewline
%\hline
%1 &
%\wildernessspellone{} &
%\base{5+}\newline
%\amel{8+} &
%\base{\range{24}} \newline
%\amel{\range{48}} \newline
%\hex{} \newline
%\missile{} \newline
%\damage{} &
%\instant{} &
%La cible subit 5D6 touches de Force 1.
%\tabularnewline
%\hline
%2 &
%\wildernessspelltwo{} &
%\base{5+} \newline
%\amel{8+} &
%\base{\range{6}} \newline
%\amel{\range{18}} \newline
%\universal{} &
%\lastsoneturn{} &
%La cible gagne la règle \frenzy{}. 
%\tabularnewline
%\hline
%3 &
%\wildernessspellthree{} &
%\base{8+} \newline
%\amel{14+} &
%\range{24} \newline
%\hex{} \newline
%\missile{} \newline
%\damage{} &
%\instant{} &
%La cible subit une touche de Force \base{6} \amel{10} avec les règles \multiplewounds{\base{1D3} \amel{\ordnance}}{} et \armourpiercing{6}. Les rangs de l'unité ciblée sont pénétrés de la même façon que par une \boltthrower{}. 
%\tabularnewline
%\hline
%4 &
%\wildernessspellfour{} &
%\base{9+} \newline
%\amel{12+} &
%\base{\range{36}} \newline
%\amel{\range{72}} \newline
%\hex{} &
%\lastsoneturn{} &
%La cible subit un malus de -1 pour toucher et traite tous les terrains, y compris les Terrains Découverts, comme des Terrains Dangereux (2).
%\tabularnewline
%\hline
%5 &
%\wildernessspellfive{} &
%10+ &
%\range{24} \newline
%\hex{} &
%\lastsoneturn{} &
%La cible ne peut pas utiliser la règle \fly{} ni effectuer d'Attaques de Tir.
%\tabularnewline
%\hline
%6 &
%\wildernessspellsix{} &
%\base{11+} \newline
%\amel{14+} &
%\base{\range{6}} \newline
%\amel{\range{12}} \newline
%\universal{} \newline
%\focused{} \newline
%\characteronly{} &
%\lastsoneturn{} &
%La cible reçoit de nouvelles règles et un changement de caractéristiques correspondant à un des Aspects listés dans la table ci dessous. Choisissez avant le lancement du sort.
%
%\tabularnewline
%\closetable
%
%\vspace*{0.5cm}
%\renewcommand{\arraystretch}{1.5}
%\begin{center}
%\begin{tabular}{rccccl}
%\hline
%& \labels@WS{} & \labels@S{} & \labels@T{} & \labels@A{} & \labels@specialrules{} \tabularnewline
%\textbf{\aspectofhydra} & \textbf{6} & \textbf{5} & \textbf{5} & \textbf{6} & \regeneration{4} \tabularnewline
%\textbf{\aspectofmanticore} & \textbf{6} & \textbf{5} & \textbf{5} & \textbf{4} & \multiplewounds{1D3}{}, \lethalstrike{} \tabularnewline
%\textbf{\aspectofdragon} & \textbf{6} & \textbf{6} & \textbf{6} & \textbf{3} & \breathweapon{\Strength{} 4, \flamingattacks} \tabularnewline
%\hline
%\end{tabular}
%\end{center}
%\renewcommand{\arraystretch}{3.2}
%
%
%
%
%\eightfoldpathtitle{eightfoldpath}{\Pathof{} \eightpaths}
%
%\spaceaftersection{}
%
%La Voie \eightpaths{} est accessible à de rares Sorciers. Elle regroupe les sorts Primaires des huit Voies Communes, et chacun de ces sorts déclenche l'Attribut de sa Voie lorsqu'il est lancé avec succès.
%
%\vspace*{1cm}
%\begin{center}
%\begin{tabular}{>{\bf}M{1.75cm}>{\raggedright}m{2.1cm}M{1cm}M{1.9cm}M{1.5cm}m{7cm}}
%\rowcolor{black!30} &
%\textbf{\spellsName} &
%\textbf{\spellsCastingValue} &
%\textbf{\spellsType} &
%\textbf{\spellsDuration} &
%\centering\textbf{\spellsEffect}
%\tabularnewline
%\cellcolor[HTML]{\colors@alchemy} \alchemyTOC{} &
%\alchemyattribute{} &
%\textbf{A} &
%\range{12} \newline
%\augment{} &
%\lastsoneturn{} &
%La cible gagne +1 en Sauvegarde d'Armure. Aucune figurine ne peut obtenir mieux que 3+ en Sauvegarde d'Armure grâce à ce sort avant d'appliquer d'éventuels malus. \tabularnewline
%\cellcolor[HTML]{\colors@alchemy} \includegraphics[width=1cm]{pics/alchemy.png}&
%\alchemysignature &
%\base{9+}\newline
%\amel{17+} &
%\range{24} \newline
%\hex{} \newline
%\missile{} \newline
%\damage{} &
%\instant{} &
%La cible subit \base{1D6} \amel{2D6} touches avec la règle \metalshifting{}.
%\tabularnewline
%\hline
%\cellcolor[HTML]{\colors@heavens} \heavensTOC{} &
%\heavensattribute{} &
%\textbf{A} &
%\specialTYPE{} &
%\lastsoneturn{} &
%L'armée du lanceur gagne un marqueur \og \heavensattribute{} \fg{}. Ce marqueur peut être dépensé pour relancer un unique D6 d'un jet pour toucher, pour blesser ou de Sauvegarde d'Armure.
%\tabularnewline
%\cellcolor[HTML]{\colors@heavens} \includegraphics[width=1cm]{pics/heavens.png} &
%\heavenssignature{} &
%\base{7+}\newline
%\amel{10+} &
%\base{\range{12}} \newline
%\amel{\range{36}} \newline
%\hex{} &
%\lastsoneturn{} &
%La cible subit un malus de -1 pour toucher et de -1 en Commandement. Les Attaques de Tir ne nécessitant pas l'utilisation de la CT doivent obtenir un 4+ sur 1D6 pour pouvoir être utilisées.
%\tabularnewline
%\hline
%\cellcolor[HTML]{\colors@fire} \textcolor{white}{\fireTOC} &
%\fireattribute{} &
%\textbf{A} &
%\range{24} \newline
%\hex{} \newline
%\missile{} \newline
%\damage{} &
%\instant{} &
%La cible subit 1D3 touches de Force 4 avec la règle \flamingattacks{}.
%\tabularnewline
%\cellcolor[HTML]{\colors@fire} \includegraphics[width=1cm]{pics/fire.png} &
%\firesignature{} &
%\base{5+}\newline
%\amel{10+} \newline
%\amelbis{14+} &
%\base{\range{24}} \newline
%\amel{\range{36}} \newline
%\amelbis{\range{48}} \newline
%\hex{} \newline
%\missile{} \newline
%\damage{} &
%\instant{} &
%La cible subit \base{1D6} \amel{2D6} \amelbis{3D6} touches de Force 4 avec la règle \flamingattacks{}.
%\tabularnewline
%\hline
%\cellcolor[HTML]{\colors@light} \lightTOC{} &
%\lightattribute{} &
%\textbf{A} &
%\range{48} \newline
%\augment{} &
%\lastsoneturn{} &
%La cible gagne +1 en Commandement. Aucune figurine ne peut être affectée par ce sort plus d'une fois par Phase de Magie.
%\tabularnewline
%\cellcolor[HTML]{\colors@light} \includegraphics[width=1cm]{pics/light.png} &
%\lightsignature{} &
%\base{5+}\newline
%\amel{14+} &
%\base{\range{24}} \newline
%\amel{\range{48}} \newline
%\hex{} \newline
%\missile{} \newline
%\damage{} &
%\instant{} &
%La cible subit 1D6 touches de Force \base{4} \amel{6} avec la règle \flamingattacks{}.
%
%\vspace*{5pt}
%Si la cible dispose de la règle \otherworldly{} ou \undead{}, elle subit 2D6 touches à la place.
%\tabularnewline
%\end{tabular}
%\end{center}
%
%
%
%\newpage
%
%\vspace*{2cm}
%\begin{center}
%\begin{tabular}{>{\bf}M{1.75cm}>{\raggedright}m{2.1cm}M{1cm}M{1.9cm}M{1.5cm}m{7cm}}
%\rowcolor{black!30} &
%\textbf{\spellsName} &
%\textbf{\spellsCastingValue} &
%\textbf{\spellsType} &
%\textbf{\spellsDuration} &
%\centering\textbf{\spellsEffect}
%\tabularnewline
%\cellcolor[HTML]{\colors@death} \textcolor{white}{\deathTOC} &
%\deathattribute{} &
% \textbf{A} &
%\range{24} \newline
%\hex{} &
%\lastsoneturn{} &
%La cible subit un malus de -1 en Commandement. Aucune figurine ne peut être affectée par ce sort plus d'une fois par Phase de Magie.
%\tabularnewline
%\cellcolor[HTML]{\colors@death} \includegraphics[width=1cm]{pics/death.png} &
%\deathsignature{} &
%\base{8+}\newline
%\amel{11+} &
%\range{18} \newline
%\focused{} \newline
%\hex{} \newline
%\direct{} \newline
%\damage{} &
%\instant{} &
%La cible subit 1 blessure avec la règle \armourpiercing{6}.
%
%\vspace*{5pt}
%\amel{Si la blessure n'est pas sauvegardée, le lanceur Récupère 1 PV}.
%\tabularnewline
%\hline
%\cellcolor[HTML]{\colors@nature} \textcolor{white}{\natureTOC} &
%\natureattribute{} &
%\textbf{A} &
%\range{12} \newline
%\focused{} \newline
%\augment{} &
%\instant{} &
%La cible Récupère un PV. Aucune figurine ne peut Récupérer plus d'un PV par Phase de Magie avec ce sort.\tabularnewline
%\cellcolor[HTML]{\colors@nature} \includegraphics[width=1cm]{pics/nature.png} &
%\naturesignature{} &
%\base{4+}\newline
%\amel{8+} &
%\base{\castersunit} \newline
%\amel{\range{12}} \newline
%\augment{} &
%\lastsoneturn{} &
%La cible gagne la règle \regeneration{5}. 
%\tabularnewline
%\hline
%\cellcolor[HTML]{\colors@shadows} \textcolor{white}{\shadowsTOC} &
%\shadowsattribute{} &
%\textbf{A} &
%\range{12} \newline
%\focused{} \newline
%\augment{} &
%\instant{} &
%Ce sort ne peut être lancé que sur une unité constituée d'une seule figurine ou sur un Personnage. La cible peut effectuer un Mouvement Magique de \distance{10} avec la règle \fly{}.
%
%\vspace*{5pt}
%Si la cible dispose de la règle \largetarget{}, ce mouvement est limité à \distance{2}.
%\tabularnewline
%\cellcolor[HTML]{\colors@shadows} \includegraphics[width=1cm]{pics/shadows.png} &
%\shadowssignature{} &
%\base{4+}\newline
%\amel{7+} &
%\range{48} \newline
%\hex{} &
%\lastsoneturn{} &
%La cible subit un malus de \base{-1} \amel{-1D3} à une caractéristique parmi M, CC, CT ou I, jusqu'à un minimum de 1. Le lanceur doit préciser laquelle avant de lancer le sort.
%\tabularnewline
%\hline
%\cellcolor[HTML]{\colors@wilderness} \textcolor{white}{\wildernessTOC} &
%\wildernessattribute{} &
%\textbf{A} &
%\range{12} \newline
%\augment{} &
%\instant{} &
%Ce sort ne peut affecter que les unités composées uniquement, au choix, de :
%\begin{itemize}[label={-}]
%\item \cavalry{}, \warbeast{}, \monstrouscavalry{}, \monstrousbeast{}, \chariot{}, \monster{} et Monstre Mon\-té.
%\item toute unité du Livre d'Armée Hardes Bestiales.
%\item l'unité du lanceur.
%\end{itemize}
%
%La cible peut effectuer un Mouvement Magique de\newline\distance{1D3+2}.
%\tabularnewline
%\cellcolor[HTML]{\colors@wilderness} \includegraphics[width=1cm]{pics/wilderness.png} &
%\wildernesssignature{} &
%\base{9+} \newline
%\amel{12+} &
%\base{\range{12}} \newline
%\amel{\range{24}} \newline
%\augment{} &
%\lastsoneturn{} &
%La cible gagne +1 en Force et en Endurance.
%\tabularnewline
%\end{tabular}
%\end{center}
%
%
%
%
%
%
%\newspecificpath{butchery}{\butchery}
%
%\starttable[white]{\colors@butchery}
%A &
%\butcheryattribute{} &
%&
%\caster{} &
%\instant{}\newline \lastsoneturn{} &
%La cible Récupère un PV précédemment perdu et gagne +1 en Endurance. De plus, les attaques avec \poisonedattacks{} dirigées contre la cible perdent cette règle.
%\tabularnewline
%\hline
%0 &
%\butcherysignature{} &
%\base{7+}\newline \amel{11+} &
%\base{\range{18}} \newline
%\amel{\range{12}} \newline
%\amel{\aura{}} \newline
%\augment{} &
%\lastsoneturn{} &
%La cible gagne +1 en Endurance. 
%\tabularnewline
%\hline
%1 &
%\butcheryspellone{} &
%\base{6+}\newline
%\amel{9+} &
%\base{\range{12}} \newline
%\amel{\range{24}} \newline
%\augment{} &
%\lastsoneturn{} &
%La cible gagne la règle \stubborn{}.
%\tabularnewline
%\hline
%2 &
%\butcheryspelltwo{} &
%\base{6+}\newline
%\amel{10+} &
%\base{\range{18}} \newline
%\amel{\range{12}} \newline
%\amel{\aura{}} \newline
%\augment{} &
%\lastsoneturn{} &
%La cible gagne +1 en Force.
%\tabularnewline
%\hline
%3 &
%\butcheryspellthree{} &
%\base{7+}\newline
%\amel{12+} &
%\base{\range{24}} \newline
%\amel{\range{36}} \newline
%\hex{} \newline
%\missile{} \newline
%\damage{} &
%\instant{} &
%La cible subit 2D6 touches de Force \base{2} \amel{3} avec la règle \armourpiercing{6}.
%\tabularnewline
%\hline
%4 &
%\butcheryspellfour{} &
%\base{7+}\newline
%\amel{10+} &
%\base{\range{36}} \newline
%\amel{\range{72}} \newline
%\hex{} &
%\instant{}\newline
%\lastsoneturn{} &
%La cible doit immédiatement effectuer un test de Panique.
%
%\vspace*{5pt}
%\amel{Toutes les unités gagnent la règle \hatred{} contre la cible.}
%\tabularnewline
%\hline
%5 &
%\butcheryspellfive{} &
%\base{11+}\newline
%\amel{14+} &
%\base{\range{12}} \newline
%\amel{\range{24}} \newline
%\augment{} &
%\lastsoneturn{} &
%La cible gagne la \regeneration{4}.
%\tabularnewline
%\hline
%6 &
%\butcheryspellsix{} &
%\base{12+} \newline
%\amel{14+} &
%\base{\range{18}} \newline
%\amel{\range{24}} \newline
%\hex{} \newline
%\direct{} \newline
%\damage{} &
%\instant{}\newline
%\lastsoneturn{} &
%La cible subit 1D6 touches de Force 5 avec la règle \armourpiercing{6}.
%
%\vspace*{5pt}
%Si au moins une blessure est infligée, la cible ne peut pas faire de Marche Forcée et jette un dé de moins pour ses jets de distance de Charge, Charge Irrésistible, Fuite et Poursuite.
%\tabularnewline
%\closetable
%
%
%
%
%\newspecificpath{change}{\change}
%
%\starttable[white]{\colors@change}
%A &
%\changeattribute{} &
%&
%\universal{} &
%\permanent{} &
%L'armée gagne un marqueur \changeattribute{}. Lorsque l'armée lance un sort de la Voie \change{} non lié à un \boundspell{}, tout Dé de Pouvoir ayant donné un \result{1} naturel peut être relancé en échange d'un marqueur.
%\tabularnewline
%\hline
%0 &
%\changesignature{} &
%\base{5+}\newline
%\amel{8+} \newline
%\amelbis{11+} &
%\base{\range{24}} \newline
%\amel{\range{48}} \newline
%\amelbis{\range{48}} \newline
%\hex{} \newline
%\missile{} \newline
%\damage{} &
%\instant{} &
%La cible subit \base{1D6} \amel{1D6} \amelbis{1D6+1} touches de Force 1D6 \amel{1D6} \amelbis{1D6+1} avec la règle \hellfire{}.
%\tabularnewline
%\hline
%1 &
%\changespellone{} &
%\base{6+}\newline
%\amel{11+} &
%\base{\caster} \newline
%\amel{\range{24}} \newline
%\amel{\focused} \newline
%\amel{\augment} &
%\lastsoneturn{} &
%La cible gagne une \breathweapon{\Strength{} 1D3+2, \hellfire{}}.
%\tabularnewline
%\hline
%2 &
%\changespelltwo{} &
%7+ &
%\range{24} \newline
%\hex{} \newline
%\missile{} \newline
%\damage{} &
%\instant{} &
%La cible subit une touche de Force 1D6+4 avec les règles \multiplewounds{1D3}{}, \hellfire{} et \armourpiercing{6}. Les rangs de l'unité ciblée sont pénétrés de la même façon que par une \boltthrower{}. 
%\tabularnewline
%\hline
%3 &
%\changespellthree{} &
%7+ &
%\range{18} \newline
%\hex{} \newline
%\direct{} \newline
%\damage{} \newline
%\focused{} &
%\instant{} &
%Le lanceur et la cible lancent chacun 1D6 et ajoutent leur Niveau de Magie au résultat. Si le total du lanceur est plus élevé, la cible subit une touche de Force 3 avec la règle \armourpiercing{6} et perd un Niveau de Magie, si elle en avait.
%\tabularnewline
%\hline
%4 &
%\changespellfour{} &
%9+ &
%\range{24} \newline
%\hex{} &
%\lastsoneturn{} &
%La cible ne peut pas bénéficier des règles \inspiringpresence{} et \holdyourground{}.
%\tabularnewline
%\hline
%5 &
%\changespellfive{} &
%9+ &
%\range{24} \newline
%\hex{} &
%\lastsoneturn{} &
%Lorsque la cible effectue une Attaque de Tir ou de Corps à Corps, tout jet pour toucher ayant donné \result{1}, après une éventuelle relance, est mis de côté. Pour chaque jet ainsi mis de côté, la cible subit immédiatement une touche avec la même Force et les mêmes règles que l'attaque initiale.
%
%\vspace*{5pt}
%Les Attaques de Tir comptent toujours comme des Attaques de Tir, et les Attaques de Corps à Corps comptent toujours comme des Attaques de Corps à Corps, participant ainsi au Résultat de Combat du camp du lanceur du sort.
%
%\vspace*{5pt}
%Les unités composées d'une seule figurine ne peuvent pas être affectées par ce sort.
%\tabularnewline
%\hline
%6 &
%\changespellsix{} &
%15+ &
%\range{24} \newline
%\hex{} \newline
%\missile{} \newline
%\damage{} &
%\instant{} &
%La cible subit 2D6 touches de Force 2D6 (limitée à 10) avec la règle \hellfire{}.
%\tabularnewline
%\closetable
%
%
%
%
%\newspecificpath{biggreengods}{\thebiggreengods}
%
%\starttable[white]{\colors@biggreengods}
%A &
%\thebiggreengodsattribute{} &
%&
%\range{24} \newline
%\augment{} &
%\lastsoneturn{} &
%La cible peut relancer ses jets pour blesser ayant donné \result{1} au Corps à Corps. 
%\tabularnewline
%\hline
%0 &
%\thebiggreengodssignature{} &
%10+ &
%\range{18} \newline
%\hex{} &
%\lastsoneturn{} &
%Toutes les unités peuvent relancer leurs jets pour toucher ratés contre la cible au Corps à Corps.
%\tabularnewline
%\hline
%1 &
%\thebiggreengodsspellone{} &
%\base{6+} \newline
%\amel{13+} &
%\base{\range{24}} \newline
%\amel{\range{18}} \newline
%\amel{\aura{}} \newline
%\hex{} \newline
%\direct{} \newline
%\damage{} \newline
%\focused{} &
%\instant{} &
%Affecte uniquement les Sorciers.
%
%\vspace*{5pt}
%La cible subit une touche de Force 5 avec la règle \armourpiercing{6}.
%\tabularnewline
%\hline
%2 &
%\thebiggreengodsspelltwo{} &
%\base{6+} \newline
%\amel{11+} &
%\base{\caster} \newline
%\amel{\range{12}} \newline
%\amel{\focused} \newline
%\amel{\augment} \newline
%\amel{\characteronly} &
%\remainsinplay{} &
%La cible gagne +3 en Force, +3 Attaques et la règle \magicalattacks{}. 
%\tabularnewline
%\hline
%3 &
%\thebiggreengodsspellthree{} &
%\base{8+} \newline
%\amel{11+} &
%\base{\range{12}} \newline
%\amel{\range{24}} \newline
%\augment{} &
%\lastsoneturn{} &
%La cible gagne une \wardsave{5}.
%\tabularnewline
%\hline
%4 &
%\thebiggreengodsspellfour{} &
%\base{11+} \newline
%\amel{14+} &
%\range{24} \newline
%\augment{} &
%\instant{} &
%La cible peut faire un Pivot, en ignorant tout obstacle, puis effectuer un Mouvement Magique de \base{\distance{3D6}} \amel{\distance{5D6}} avec la règle \fly{}. Jetez les dés pour la distance avant de pivoter.
%
%\vspace*{5pt}
%À la fin de ce mouvement, l'unité peut encore faire un Pivot en ignorant les obstacles. L'unité ne peut pas finir positionnée à moins de \distance{1} d'une autre unité ou d'un Terrain Infranchissable. 
%\tabularnewline
%\hline
%5 &
%\thebiggreengodsspellfive{} &
%11+ &
%\range{18} \newline
%\hex{} \newline
%\direct{} \newline
%\damage{} &
%\instant{} &
%La cible subit 2D6 touches de Force 5.
%\tabularnewline
%\hline
%6 &
%\thebiggreengodsspellsix{} &
%\base{13+} \newline
%\amel{16+} &
%\range{36} \newline
%\hex{} \newline
%\direct{} \newline
%\damage{} &
%\instant{} &
%Placez le centre d'un gabarit de \distance{3} sur l'unité ciblée et à portée du sort, puis faites une Déviation de \distance{1D6}. Toutes les figurines sous le gabarit subissent chacune une touche de Force \base{6} \amel{8} avec la règle \multiplewounds{1D3}{}.
%\tabularnewline
%\closetable
%
%
%
%
%\newspecificpath{littlegreengods}{\thelittlegreengods}
%
%\spaceaftersection{}
%
%Après le déploiement, tout Sorcier connaissant au moins un sort de cette Voie doit choisir un sort non lié à un \boundspell{} connu par n'importe quel Sorcier sur le champ de bataille. Il gagne l'Attribut de Voie de ce sort pour ses propres lancements.
%
%\starttable{\colors@littlegreengods}
%A &
%\thelittlegreengodsattribute{} &
%&
%\specialTYPE{} &
%\specialTYPE{} &
%Utilisez l'Attribut de Voie du sort choisi en début de partie. Remplacez toutes les références à la Voie de l'Attribut par la Voie \thelittlegreengods{}. Si le sort choisi appartient à la Voie \thelittlegreengods{}, l'Attribut n'a pas d'effet.
%\tabularnewline
%\hline
%0 &
%\thelittlegreengodssignature{} &
%\base{4+} \newline
%\amel{9+} &
%\range{24} \newline
%\hex{} \newline
%\missile{} \newline
%\damage{} &
%\instant{} &
%La cible subit \base{2D6} \amel{3D6} touches de Force 3.
%\tabularnewline
%\hline
%1 &
%\thelittlegreengodsspellone{} &
%\base{5+}\newline
%\amel{8+} &
%\base{\range{12}} \newline
%\amel{\range{24}} \newline
%\augment{} &
%\lastsoneturn{} &
%Les Attaques de Tir et de Corps à Corps de la cible gagnent la règle \armourpiercing{1}.
%
%\vspace*{5pt}
%Si la cible attaque une unité sur son flanc ou son arrière, la cible peut relancer ses jets pour toucher et pour blesser ratés au Corps à Corps.
%\tabularnewline
%\hline
%2 &
%\thelittlegreengodsspelltwo{} &
%\base{6+}\newline
%\amel{9+} &
%\base{\range{12}} \newline
%\amel{\range{24}} \newline
%\augment{} &
%\lastsoneturn{} &
%La cible gagne la règle \poisonedattacks{}. Les attaques qui disposaient déjà de cette règle blessent automatiquement sur 5+.
%\tabularnewline
%\hline
%3 &
%\thelittlegreengodsspellthree{} &
%6+ &
%\range{24} \newline
%\hex{} &
%\lastsoneturn{} &
%La cible subit un malus de -1D3 en Mouvement et en Initiative, jusqu'à un minimum de 1. 
%\tabularnewline
%\hline
%4 &
%\thelittlegreengodsspellfour{} &
%\base{7+} \newline
%\amel{10+} &
%\base{\range{12}} \newline
%\amel{\range{6}} \newline
%\amel{\aura{}} \newline
%\augment{} &
%\lastsoneturn{} &
%La cible gagne les règles \hardtarget{} et \distracting{}. Toute figurine d'une unité chargeant avec succès la cible doit effectuer un test de Terrain Dangereux (1).
%\tabularnewline
%\hline
%5 &
%\thelittlegreengodsspellfive{} &
%9+ &
%\range{24} \newline
%\hex{} &
%\lastsoneturn{} &
%La cible doit relancer ses jets pour toucher, pour blesser et de Sauvegarde de tout type ayant donné \result{6}.
%\tabularnewline
%\hline
%6 &
%\thelittlegreengodsspellsix{} &
%13+ &
%\vortex{} \newline
%(\range{4}, \template{} \distance{5}) \newline
%\ground{} &
%\instant{}\newline
%\lastsoneturn{} &
%Toutes les figurines touchées par le gabarit subissent chacune une touche de Force 3 avec la règle \armourpiercing{6}.
%
%\vspace*{5pt}
%Toute unité touchée par le gabarit subit un malus de -1 en Capacité de Combat.
%\tabularnewline
%\closetable
%
%
%
%
%\newspecificpath{forge}{\forge}
%
%\starttable[white]{\colors@forge}
%A &
%\forgeattribute{} &
%&
%\range{18} \newline
%\hex{} &
%\lastsoneturn{} &
%La cible gagne la règle \flammable{}. 
%\tabularnewline
%\hline
%0 &
%\forgesignature{} &
%\base{8+} \newline
%\amel{11+} &
%\range{12} \newline
%\augment{} &
%\remainsinplay{} &
%La cible gagne les règles \flamingattacks{} et \magicalattacks{}, \base{ou} \amel{et} les attaques dirigées contre la cible subissent un malus de -1 pour blesser.\newline
%\base{Choisissez avant de lancer le sort.}
%\tabularnewline
%\hline
%1 &
%\forgespellone{} &
%\base{5+} \newline
%\amel{11+} &
%\range{12} \newline
%\hex{} \newline
%\missile{} \newline
%\damage{} &
%\instant{} &
%La cible subit \base{1D6} \amel{2D6} touches de Force 6 avec la règle \flamingattacks{}.
%\tabularnewline
%\hline
%2 &
%\forgespelltwo{} &
%7+ &
%\range{24} \newline
%\hex{} &
%\permanent{} &
%La cible subit un malus de -1 en Commandement.
%\tabularnewline
%\hline
%3 &
%\forgespellthree{} &
%7+ &
%\range{12} \newline
%\augment{} &
%\lastsoneturn{} &
%La cible peut relancer ses jets pour toucher au Corps à Corps.
%\tabularnewline
%\hline
%4 &
%\forgespellfour{} &
%9+ &
%\range{18} \newline
%\focused{} \newline
%\hex{} \newline
%\direct{} \newline
%\damage{} &
%\instant{} &
%La cible subit un nombre de touches égal à 2D6 moins son Endurance. Ces touches blessent sur 4+ et ont la règle \armourpiercing{6}.
%\tabularnewline
%\hline
%5 &
%\forgespellfive{} &
%\base{11+} \newline
%\amel{14+} &
%\base{\range{24}} \newline
%\amel{\range{48}} \newline
%\hex{} &
%\lastsoneturn{} &
%La cible subit un malus de -1 pour toucher au Corps à Corps et de -2 pour toucher avec ses Attaques de Tir.
%
%\vspace*{5pt}
%Elle ne peut pas effectuer de Marche Forcée durant l'Étape des Autres Mouvements. Si l'unité veut charger, elle divise par deux le résultat de son jet de distance de charge, en arrondissant au supérieur.
%
%\vspace*{5pt}
%La portée de tous les sorts connus par la cible est limitée à un maximum de \distance{12}.
%\tabularnewline
%\hline
%6 &
%\forgespellsix{} &
%\base{14+} \newline
%\amel{17+} &
%\range{36} \newline
%\hex{} \newline
%\missile{} \newline
%\damage{} &
%\instant{} &
%Placez le centre d'un gabarit de \distance{3} sur la cible et à portée du sort. Faites une Déviation de \distance{1D6}. Toutes les figurines sous le gabarit subissent une touche de Force \base{5} \amel{7} avec les règles \flamingattacks{} et \multiplewounds{\ordnance}{}.
%\tabularnewline
%\closetable
%
%
%
%
%\newspecificpath{lust}{\lust}
%
%\starttable{\colors@lust}
%A &
%\lustattribute{} &
%&
%\range{12} \newline
%\augment{} \newline
%\focused{} &
%\lastsoneturn &
%La cible gagne +3 en Initiative et +1 Attaque, \textbf{ou} la règle \armourpiercing{+1}.
%
%\vspace*{5pt}
%Choisissez au moment de lancer le sort. Chaque figurine ne peut être affectée par chacun des effets qu'une fois par Phase de Magie.
%\tabularnewline
%\hline
%0 &
%\lustsignature{} &
%6+ &
%\range{24} \newline
%\ground{} \newline
%\direct{} \newline
%\linetemplate{} &
%\instant{} &
%Toutes les figurines sous le gabarit subissent une touche de Force 4 avec la règle \armourpiercing{2}.
%\tabularnewline
%\hline
%1 &
%\lustspellone{} &
%\base{5+} \newline
%\amel{11+} &
%\base{\range{24}} \newline
%\amel{\range{36}} \newline
%\hex{} &
%\lastsoneturn{} &
%La cible porte ses attaques à Initiative 0 \amel{et gagne la règle \randommovement{1D6}}. 
%\tabularnewline
%\hline
%2 &
%\lustspelltwo{} &
%\base{7+}\newline
%\amel{11+} &
%\base{\range{12}} \newline
%\amel{\range{24}} \newline
%\focused{} \newline
%\hex{} \newline
%\direct{} \newline
%\damage{} &
%\instant{} &
%La cible doit effectuer un test de Commandement en jetant un dé supplémentaire. Si le test est raté, la cible subit une blessure avec la règle \armourpiercing{6}, et doit continuer à effectuer des tests de Commandement de la même façon, avec les mêmes effets, jusqu'à ce qu'elle réussisse ou succombe.
%\tabularnewline
%\hline
%3 &
%\lustspellthree{} &
%8+ &
%\range{24} \newline
%\universal{} &
%\remainsinplay{} &
%La cible gagne la règle \frenzy{}. Si elle l'avait déjà, elle gagne encore une attaque supplémentaire, jusqu'à ce qu'elle perde la \frenzy{}.
%
%\vspace*{5pt}
%À la fin de chaque Phase de Magie du lanceur, la cible subit 1D6 touches de Force 3.
%\tabularnewline
%\hline
%4 &
%\lustspellfour{} &
%\base{8+}\newline
%\amel{12+} &
%\base{\range{24}} \newline
%\amel{\range{12}} \newline
%\amel{\aura} \newline
%\hex{} &
%\lastsoneturn{} &
%À chaque fois que la cible effectue un test de Commandement, elle lance un dé supplémentaire et ignore le dé avec le résultat le plus bas.
%\tabularnewline
%\hline
%5 &
%\lustspellfive{} &
%9+ &
%\range{24} \newline
%\hex{} \newline
%\missile{} \newline
%\damage{} &
%\instant{} &
%La cible subit 1D6 touches de Force 4 avec la règle \armourpiercing{1}.
%
%\vspace*{5pt}
%De plus, la cible doit effectuer un test de Commandement sans aucune relance autorisée. En cas d'échec, elle subit à nouveau 1D6 touches, et doit continuer à effectuer des tests de Commandement de la même façon, avec les mêmes effets, jusqu'à ce qu'elle réussisse ou succombe.
%\tabularnewline
%\hline
%6 &
%\lustspellsix{} &
%13+ &
%\range{12} \newline
%\hex{} &
%\instant{}\newline
%\lastsoneturn{} &
%La cible subit immédiatement 2D6 touches blessant sur 4+ avec la règle \armourpiercing{6}.
%
%\vspace*{5pt}
%Si au moins une blessure est infligée ainsi, la cible porte ses attaques à Initiative 0 et gagne la règle \randommovement{1D6}.
%\tabularnewline
%\closetable
%
%
%
%
%\newspecificpath{whitemagic}{\whitemagic}
%\label{white_magic}
%
%\starttable{\colors@whitemagic}
%A &
%\whitemagicattribute{} &
%&
%\range{18} \newline
%\augment{} &
%\permanent{} &
%Placez un marqueur Bouclier sur la cible. Ignorez la prochaine blessure non sauvegardée, après avoir calculé les éventuelles Blessures Multiples, que devrait subir une unité qui possède un marqueur. Retirez alors le marqueur.
%
%\vspace*{5pt}
%Si la cible subit plusieurs blessures simultanées, son propriétaire choisit la blessure à ignorer.
%
%\vspace*{5pt}
%Si une unité devait disposer de plusieurs marqueurs, retirez-les tous sauf un.
%
%\vspace*{5pt}
%Toutes les figurines d'une unité peuvent utiliser le marqueur, y-compris les Personnages ayant rejoint l'unité après le lancement du sort. Si un Personnage quitte une unité avec un marqueur, choisissez qui du Personnage ou de l'unité le garde. Une figurine avec la règle \largetarget{} ne peut jamais profiter de ce sort.
%\tabularnewline
%\hline
%0 &
%\whitemagicsignature{} &
%\base{9+} \newline
%\amel{12+} &
%\base{\range{24}} \newline
%\amel{\range{18}} \newline
%\hex{} \newline
%\missile{} \newline
%\damage{} &
%\instant{} &
%La cible subit \base{2D6} \amel{3D6} touches de Force 4.
%\tabularnewline
%\hline
%1 &
%\whitemagicspellone{} &
%\base{4+}\newline
%\amel{8+} &
%\range{18} \newline
%\focused{} \newline
%\augment{} &
%\instant{} \newline
%\amel{\lastsoneturn}  &
%La cible Récupère 1 PV.
%
%\vspace*{5pt}
%\amel{Elle et son unité gagnent également +1 en Force.}
%\tabularnewline
%\hline
%2 &
%\whitemagicspelltwo{} &
%\base{6+} \newline
%\amel{9+} &
%\range{18} \newline
%\augment{} &
%\lastsoneturn{} &
%La cible gagne +1D3 à \base{une} caractéristique parmi M, CC, CT ou I. Le lanceur doit préciser laquelle avant de lancer le sort.
%
%\vspace*{5pt}
%\amel{Les 4 caractéristiques sont affectées.}
%\tabularnewline
%\hline
%3 &
%\whitemagicspellthree{} &
%\base{7+} \newline
%\amel{15+} &
%\range{24} \newline
%\augment{} &
%\special{} &
%La cible gagne la règle \ethereal{} jusqu'à la fin de la Phase de Magie et peut faire un Mouvement Magique de \base{\distance{8}} \amel{\distance{16}}.
%\tabularnewline
%\hline
%4 &
%\whitemagicspellfour{} &
%\base{9+} \newline
%\amel{12+} &
%\base{\range{12}} \newline
%\amel{\range{24}} \newline
%\augment{} &
%\lastsoneturn{} &
%La cible gagne une \wardsave{5}. Si une figurine affectée dispose déjà d'une \wardsave{}, cette dernière est améliorée d'un point, pour obtenir 3+ au mieux.
%\tabularnewline
%\hline
%5 &
%\whitemagicspellfive{} &
%11+ &
%\range{24} \newline
%\hex{} \newline
%\direct{} \newline
%\focused{} \newline
%\damage{} &
%\instant{} &
%La cible subit une touche avec la règle \metalshifting{}.
%
%\vspace*{5pt}
%Si elle porte un ou plusieurs Objets Magiques, l'un d'entre eux, déterminé au hasard, est détruit. Il perd toutes ses règles et devient son équivalent ordinaire pour le reste de la partie. Les Objets Magiques comportant la mention Une Seule Utilisation ne peuvent pas être choisis.
%\tabularnewline
%\hline
%6 &
%\whitemagicspellsix{} &
%\base{14+} \newline
%\amel{18+} &
%\range{24} \newline
%\hex{} \newline
%\direct{} \newline
%\damage{} &
%\remainsinplay{} &
%À la fin de chaque Phase de Magie, toutes les figurines de l'unité ciblée subissent une touche de Force \base{3} \amel{4} avec la règle \flamingattacks{}.
%\tabularnewline
%\closetable
%
%
%
%
%\newspecificpath{blackmagic}{\blackmagic}
%
%\starttable[white]{\colors@blackmagic}
%A &
%\blackmagicattribute{} &
%&
%\range{18} \newline
%\hex{} \newline
%\missile{} \newline
%\damage{} &
%\instant{} &
%La cible subit une touche de Force 5.
%
%\vspace*{5pt}
%Si cela cause une blessure non sauvegardée, le lanceur Récupère 1 PV. Aucune figurine ne peut Récupérer plus d'un PV par Phase de Magie grâce à ce sort.
%\tabularnewline
%\hline
%0 &
%\blackmagicsignature{} &
%\base{8+} \newline
%\amel{10+} &
%\base{\range{6}} \newline
%\amel{\range{18}} \newline
%\augment{} &
%\lastsoneturn{} &
%La cible gagne +1 en Force \amel{et la règle \armourpiercing{1}}.
%\tabularnewline
%\hline
%1 &
%\blackmagicspellone{} &
%\base{4+} \newline
%\amel{9+} &
%\base{\range{24}} \newline
%\amel{\range{36}} \newline
%\hex{} \newline
%\direct{} \newline
%\damage{} &
%\instant{} \newline
%\amel{\lastsoneturn} &
%La cible subit immédiatement 1D6 touches de Force 4.
%
%\vspace*{5pt}
%\amel{Elle subit aussi un malus de -1D3 en CT, jusqu'à un minimum de 1.}
%\tabularnewline
%\hline
%2 &
%\blackmagicspelltwo{} &
%\base{7+} \newline
%\amel{9+} &
%\range{24} \newline
%\hex{} &
%\lastsoneturn{} &
%La cible subit un malus de -1D3 en CC, jusqu'à un minimum de 1.
%
%\vspace*{5pt}
%\amel{De plus, elle ne peut pas utiliser les règles \distracting{} ou \parry{}.}
%\tabularnewline
%\hline
%3 &
%\blackmagicspellthree{} &
%\base{8+} \newline
%\amel{14+} &
%\base{\range{18}} \newline
%\amel{\range{36}} \newline
%\hex{} &
%\lastsoneturn{} &
%La cible subit un malus de \base{-1} \amel{-2} en Force et en Initiative, jusqu'à un minimum de 1.
%\tabularnewline
%\hline
%4 &
%\blackmagicspellfour{} &
%\base{8+} \newline
%\amel{13+} &
%\base{\range{18}} \newline
%\amel{\range{36}} \newline
%\hex{} &
%\remainsinplay{} &
%La cible ne peut pas bénéficier des règles \inspiringpresence{} ou \holdyourground{}.
%\tabularnewline
%\hline
%5 &
%\blackmagicspellfive{} &
%11+ &
%\range{18} \newline
%\hex{} \newline
%\missile{} \newline
%\damage{} &
%\instant{} &
%La cible subit 2D6 touches de Force 5.
%\tabularnewline
%\hline
%6 &
%\blackmagicspellsix{} &
%12+ &
%\vortex{} \newline
%(\range{6}, \template{} \distance{1}) \newline
%\ground{} &
%\instant{} &
%Les figurines touchées par le gabarit doivent effectuer un test de Force. Toute figurine ratant son test subit une blessure avec les règles \multiplewounds{\ordnance}{} et \armourpiercing{6} ne permettant pas la \regeneration{}.
%\tabularnewline
%\closetable
%
%
%
%
%\newspecificpath{disease}{\disease}
%
%\starttable[white]{\colors@disease}
%A &
%\diseaseattribute{} &
%&
%\range{12} \newline
%\augment{}  \newline
%\focused{} &
%\lastsoneturn{} &
%La cible gagne +1 en Endurance. Une figurine ne peut être affectée par ce sort qu'une fois par Phase de Magie.
%\tabularnewline
%\hline
%0 &
%\diseasesignature{} &
%\base{4+} \newline
%\amel{9+} &
%\range{18} \newline
%\augment{} &
%\lastsoneturn{} &
%Les unités ennemies en contact socle à socle avec la cible ont un malus de \base{-1} \amel{-1D3} en Capacité de Combat et en Initiative, jusqu'à un minimum de 1.
%\tabularnewline
%\hline
%1 &
%\diseasespellone{} &
%\base{6+} \newline
%\amel{9+} &
%\base{\caster} \newline
%\amel{\range{24}} \newline
%\amel{\focused} \newline
%\amel{\augment} &
%Dure un tour &
%La cible gagne une \breathweapon{\toxicattacks}.
%\tabularnewline
%\hline
%2 &
%\diseasespelltwo{} &
%\base{6+} \newline
%\amel{9+} &
%\base{\range{12}} \newline
%\amel{\range{24}} \newline
%\augment{} &
%\lastsoneturn{} &
%La cible gagne la règle \poisonedattacks{}. Les attaques qui disposaient déjà de cette règle blessent automatiquement sur un point de moins : 6+ devient 5+ et 5+ devient 4+.
%\tabularnewline
%\hline
%3 &
%\diseasespellthree{} &
%\base{9+} \newline
%\amel{12+} &
%\base{\range{12}} \newline
%\amel{\range{24}} \newline
%\augment{} &
%\lastsoneturn{} &
%La cible gagne la règle \regeneration{5}. Si une figurine affectée dispose déjà d'une \regeneration{}, cette dernière est améliorée d'un point, pour obtenir 3+ au mieux.
%\tabularnewline
%\hline
%4 &
%\diseasespellfour{} &
%10+ &
%\range{18} \newline
%\hex{} \newline
%\missile{} \newline
%\damage{} &
%\instant{} &
%La cible subit 1D6 touches de Force 5.
%
%\vspace*{5pt}
%De plus, elle doit effectuer un test d'Endurance. En cas d'échec, elle subit à nouveau 1D6 touches, et doit continuer à effectuer des tests d'Endurance, avec les mêmes effets, jusqu'à ce qu'elle réussisse ou succombe.
%\tabularnewline
%\hline
%5 &
%\diseasespellfive{} &
%\base{11+} \newline
%\amel{14+} &
%\base{\range{18}} \newline
%\amel{\range{36}} \newline
%\universal{} &
%\lastsoneturn{} &
%L'Endurance de la cible est, au choix, augmentée ou diminuée de 1D3, jusqu'à un minimum de 1. Choisissez au moment de déterminer la cible.
%\tabularnewline
%\hline
%6 &
%\diseasespellsix{} &
%14+ &
%\vortex{} \newline
%(\range{6}, \template{} \distance{3}) \newline
%\ground{} &
%\instant{} &
%Les figurines touchées par le gabarit subissent chacune une touche avec les règles \toxicattacks{} et \multiplewounds{1D3}{}.
%\tabularnewline
%\closetable
%
%
%
%\newspecificpath{necromancy}{\necromancy}
%
%\starttable[white]{\colors@necromancy}
%A &
%\necromancyattribute{} &
%&
%\range{12} \newline
%\augment{} &
%\instant{} &
%L'unité ciblée, ou un Personnage situé dans l'unité ciblée, Ressuscite 1 PV.
%
%\vspace*{5pt}
%Aucune unité ne peut être ciblée plus de deux fois par Phase de Magie. 
%\tabularnewline
%\hline
%0 &
%\necromancysignature{} &
%\base{5+} \newline
%\amel{5+} \newline
%\amelbis{11+} &
%\base{\range{18}} \newline
%\amel{\range{6}} \newline
%\amel{\aura} \newline
%\amelbis{\range{12}} \newline
%\amelbis{\aura} \newline
%\augment{} &
%\instant{} &
%La cible Ressuscite le nombre de PVs correspondant à la valeur de sa caractéristique Invocation. Une unité ne disposant pas de caractéristique Invocation ne peut pas être ciblée par ce sort.
%\tabularnewline
%\hline
%1 &
%\necromancyspellone{} &
%\base{6+} \newline
%\amel{10+} &
%\range{12} \newline
%\amel{\aura} \newline
%\augment{} &
%\lastsoneturn{} &
%La cible peut relancer ses jets pour blesser ratés et gagne la règle \fear{}.
%\tabularnewline
%\hline
%2 &
%\necromancyspelltwo{} &
%\base{7+} \newline
%\amel{9+} &
%\base{\range{12}} \newline
%\amel{\range{24}} \newline
%\ground{} &
%\instant{} &
%Invoque une unité au choix parmi celles précisées dans la règle Éveil (X) du lanceur. Choisissez avant le lancement du sort. Elle dispose d'un nombre de PVs correspondant à la valeur de sa caractéristique Invocation.
%
%\vspace*{5pt}
%Une figurine de l'unité invoquée doit être placée au niveau du marqueur, et toutes les figurines doivent être à portée du sort. Toutes les options normalement accessibles à l'unité sont autorisées, sauf l'État-Major.
%\tabularnewline
%\hline
%3 &
%\necromancyspellthree{} &
%\base{8+} \newline
%\amel{12+} &
%\range{12} \newline
%\amel{\aura} \newline
%\augment{} &
%\instant{} \newline
%\lastsoneturn{} &
%La cible peut immédiatement effectuer un Mouvement Magique de \distance{8} \base{et} \amel{ou} relancer ses jets pour toucher ratés au Corps à Corps.
%
%\vspace*{5pt}
%\amel{Choisissez pour chaque cible au moment de lancer le sort.}
%\tabularnewline
%\hline
%4 &
%\necromancyspellfour{} &
%\base{9+} \newline
%\amel{11+} &
%\base{\range{24}} \newline
%\amel{\range{48}} \newline
%\hex{} \newline
%\missile{} \newline
%\damage{} &
%\instant{} &
%La cible subit 2D6 touches de Force 4.
%\tabularnewline
%\hline
%5 &
%\necromancyspellfive{} &
%10+ &
%\range{18} \newline
%\hex{} \newline
%\direct{} &
%\remainsinplay{} &
%À la fin de chaque Phase de Magie, toutes les figurines de l'unité ciblée subissent une touche qui blesse sur 6+. Ces touches suivent les règles \multiplewounds{2}{\monstrousinfantry{}, \monstrouscavalry{}, \monstrousbeast{}, \swarm{}, \monster{}, \riddenmonster} et \armourpiercing{6}.
%
%\vspace*{5pt}
%Ces touches ont un bonus de +1 pour blesser pour chaque Tour de Joueur passé depuis que le sort a été lancé.
%\tabularnewline
%\hline
%6 &
%\necromancyspellsix{} &
%12+ &
%\range{18} \newline
%\augment{} &
%\lastsoneturn{} &
%La cible gagne +1 en Force et la règle \regeneration{5}. Si une figurine affectée dispose déjà d'une \regeneration{}, cette dernière est améliorée d'un point, pour obtenir 4+ au mieux.
%\tabularnewline
%\closetable
%
%
%
%\newspecificpath{ruin}{\ruin}
%
%\starttable[white]{\colors@ruin}
%A &
%\ruinattribute{} &
% &
%\range{24} \newline
%\augment{} &
%\lastsoneturn{} &
%La cible peut compter jusqu'à quatre points de Bonus de Rang pour le Résultat de Combat et gagne la règle \fightinextrarank{}.
%\tabularnewline
%\hline
%0 &
%\ruinsignature{} &
%\base{5+} \newline
%\amel{8+} &
%\base{\range{24}} \newline
%\amel{\range{48}} \newline
%\hex{} \newline
%\missile{} \newline
%\damage{} &
%\instant{} &
%La cible subit 1D6 touches de Force 5 avec la règle \lightningattacks{}. Si un \result{6} est obtenu pour le nombre de touches, le lanceur subit aussi une touche de Force 5.
%\tabularnewline
%\hline
%1 &
%\ruinspellone{} &
%\base{6+} \newline
%\amel{10+} &
%\range{24} \newline
%\augment{} &
%\lastsoneturn{} &
%Les Attaques de Corps à Corps de la cible ont +1 pour blesser \amel{et gagnent la règle \lethalstrike{}}.
%\tabularnewline
%\hline
%2 &
%\ruinspelltwo{} &
%8+ &
%\range{18} \newline
%\ground{} &
%\specialTYPE{} &
%Dure jusqu'au début du prochain tour du lanceur. Placez le centre d'un gabarit de \distance{3} sur le marqueur et à au moins \distance{1} de toute unité.
%
%\vspace*{5pt}
%Si une unité entre en contact avec le gabarit lors d'un déplacement sans règle \fly{}, toutes ses figurines traitent n'importe quel terrain, y compris les Terrains Découverts, comme Terrain Dangereux (2).
%\tabularnewline
%\hline
%3 &
%\ruinspellthree{} &
%\base{8+} \newline
%\amel{11+} &
%\base{\range{12}} \newline
%\amel{\range{24}} \newline
%\augment{} &
%\permanent{} &
%La cible gagne +1 Attaque et la règle \frenzy{}.
%
%\vspace*{5pt}
%À la fin de chacune de vos Phases de Corps à Corps, elle subit 1D6 touches de Force 4 avec la règle \armourpiercing{6}. Tous les effets de ce sort prennent fin lorsque la cible perd la \frenzy{}.
%\tabularnewline
%\hline
%4 &
%\ruinspellfour{} &
%\base{9+} \newline
%\amel{10+} &
%\specialTYPE{} &
%\lastsoneturn{} &
%Toutes les unités ennemies subissent un malus de -1 en CT.
%
%\vspace*{5pt}
%\amel{La règle \fly{} ne peut pas être utilisée.}
%\tabularnewline
%\hline
%5 &
%\ruinspellfive{} &
%\base{9+} \newline
%\amel{12+} &
%\range{18} \newline
%\direct{} \newline
%\ground{} \newline
%\linetemplate{} &
%\instant{} &
%Toutes les figurines touchées subissent une touche de Force 6 avec la règle \armourpiercing{6} \amel{et \multiplewounds{1D3}{}}.
%\tabularnewline
%\hline
%6 &
%\ruinspellsix{} &
%\base{11+} \newline
%\amel{14+} &
%\base{\range{12}} \newline
%\amel{\range{24}} \newline
%\direct{} &
%\instant{} &
%Toutes les figurines de l'unité ciblée subissent une touche avec la règle \toxicattacks{}.
%
%\vspace*{5pt}
%Si ce sort est lancé sur une unité engagée au Corps à Corps, toutes les figurines engagées dans le combat subissent une touche avec la règle \toxicattacks{} à la place.
%\tabularnewline
%\closetable
%
%
%
%\newspecificpath{sands}{\sands}
%
%\starttable{\colors@sand}
%A &
%\sandsattribute{} &
%&
%\range{6} \newline
%\augment{} &
%\instant{} &
%Cet Attribut peut cibler une unité normalement à portée, \textbf{ou} une des unités ciblées par le sort qui a déclenché l'Attribut.
%
%\vspace*{5pt}
%L'unité ciblée, ou un Personnage situé dans l'unité ciblée, Ressuscite autant de PVs que la valeur de la caractéristique Résurrection de la cible.
%
%\vspace*{5pt}
%Les Personnages et les figurines avec la règle \largetarget{} ne peuvent pas Ressusciter plus de 2 PVs par Phase de Magie. Une unité qui n'a pas de caractéristique Résurrection ne peut pas être ciblée.
%\tabularnewline
%\hline
%0 &
%\sandssignature{} &
%\base{5+} \newline
%\amel{11+} &
%\base{\range{18}} \newline
%\amel{\range{12}} \newline
%\amel{\aura} \newline
%\augment{} &
%\instant{} &
%La cible peut effectuer un Mouvement Magique de \distance{X}, X étant égal à la valeur de Mouvement ou de \fly{} de la cible.
%\tabularnewline
%\hline
%1 &
%\sandsspellone{} &
%\base{5+} \newline
%\amel{10+} &
%\base{\range{18}} \newline
%\amel{\range{12}} \newline
%\amel{\aura} \newline
%\augment{} &
%\lastsoneturn{} &
%La cible gagne la règle \lethalstrike{}. Pour les attaques qui ont déjà la règle \lethalstrike{}, elle peut relancer ses jets pour blesser ratés au Corps à Corps.
%\tabularnewline
%\hline
%2 &
%\sandsspelltwo{} &
%\base{7+} \newline
%\amel{9+} &
%\range{24} \newline
%\hex{} &
%\lastsoneturn{} &
%La cible subit un malus de -1 en Endurance \amel{et en Force}, jusqu'à un minimum de 1.
%\tabularnewline
%\hline
%3 &
%\sandsspellthree{} &
%\base{7+} \newline
%\amel{12+} &
%\base{\range{18}} \newline
%\amel{\range{12}} \newline
%\amel{\aura} \newline
%\augment{} &
%\lastsoneturn{} &
%La cible gagne au choix +1 Attaque, ou tous ses Arcs, Grands Arcs ou Arcs Géants Aspic gagnent la règle \multipleshots{2}. Choisissez au moment de lancer le sort.
%\tabularnewline
%\hline
%4 &
%\sandsspellfour{} &
%\base{8+} \newline
%\amel{10+} &
%\range{36} \newline
%\hex{} \newline
%\damage{} &
%\instant{} &
%\amel{Ciblez une unité, puis jetez 1D6. Sur 3+, choisissez une nouvelle cible située à moins de \distance{6} de la première. Continuez ainsi, de cible en cible à \distance{6} d'écart, jusqu'à ce que vous obteniez un \result{1} ou un \result{2}, ou qu'aucune cible ne soit éligible. Aucune unité ne peut être ciblée deux fois.}
%
%\vspace*{5pt}
%Chaque cible doit effectuer un test de Commandement avec un dé additionnel. En cas d'échec, pour chaque point d'écart entre le résultat et son Commandement, la cible subit une blessure avec la règle \armourpiercing{6}.
%\tabularnewline
%\hline
%5 &
%\sandsspellfive{} &
%\base{9+} \newline
%\amel{12+} &
%\base{\range{24}} \newline
%\amel{\range{48}} \newline
%\hex{} &
%\lastsoneturn{} &
%La cible subit un malus de -1D3 en Mouvement, jusqu'à un minimum de 1. Elle traite tous les terrains, y compris les Terrains Découverts, comme des Terrains Dangereux (2).
%\tabularnewline
%\hline
%6 &
%\sandsspellsix{} &
%\base{10+} \newline
%\amel{15+} \newline
%\amelbis{17+} &
%\base{\range{18}} \newline
%\amel{\range{9}} \newline
%\amel{\aura} \newline
%\amelbis{\range{15}} \newline
%\amelbis{\aura} \newline
%\augment{} &
%\base{\lastsoneturn} \newline
%\amel{\remainsinplay} \newline
%\amelbis{\remainsinplay} &
%La cible gagne +1 en Capacité de Combat, Force et Initiative.
%\tabularnewline
%\closetable
%
%
%
%\newpage
%\basictitle{magicphasesummary}{\magicphasesummary}
%
%\renewcommand{\arraystretch}{2}
%\setlength{\columnsep}{1cm}
%
%\begin{multicols}{2}\raggedcolumns
%
%\basicsubtitle{Séquence de la Phase de Magie}
%
%\begin{tabular}{c|p{6.8cm}}
%1 & Début de la Phase de Magie. Lancez les dés pour les \textbf{Flux de Magie} et la \textbf{\channel}. \tabularnewline
%2 & Les sorts de type \textbf{\remainsinplay} peuvent être dissipés. \tabularnewline
%3 & Le Joueur Actif peut \textbf{tenter de lancer un sort}. \tabularnewline
%4 & Répétez les étapes 2 et 3 jusqu'à ce qu'aucun joueur ne tente quoique ce soit. \tabularnewline
%5 & Fin de la Phase de Magie. Les capacités prenant effet à la fin de la phase sont déclenchées. \tabularnewline
%\end{tabular}
%
%\basicsubtitle{Tentative de Lancement de Sort}
%
%\begin{tabular}{c|m{6.8cm}}
%1 & Le Joueur Actif indique quel Sorcier tente de lancer quel sort. Il doit préciser s'il opte pour une version améliorée du sort, ainsi que la cible du sort et de celle de l'attribut si nécessaire. Il indique enfin le nombre de Dés de Pouvoir utilisés, entre 1 et 5. \tabularnewline
%2 & Le Joueur Actif lance le nombre de Dés de Pouvoir annoncé, en les retirant de sa réserve. Additionnez les résultats des dés avec les modificateurs de lancer, tels qu'un Pouvoir Irrésistible, pour obtenir le total de lancement. \tabularnewline
%3 & La tentative de lancement réussit si le total de lancement est \textbf{supérieur ou égal} à la valeur de lancement. Sinon, le lancement de sort échoue et le lanceur subit une \lostfocus{}. \tabularnewline
%\end{tabular}
%
%\basicsubtitle{Tentative de Dissipation}
%
%\begin{tabular}{c|m{6.8cm}}
%1 & Le Joueur Réactif indique, s'il le souhaite, un Sorcier n'étant pas en fuite pour tenter la dissipation, et annonce combien de Dés de Dissipation il va utiliser. Il doit utiliser au moins un dé, et jusqu'à la totalité de sa réserve. Il est possible de tenter une dissipation même sans avoir de Sorcier. \tabularnewline
%2 & Le Joueur Réactif lance le nombre de Dés de Dissipation annoncé, en les retirant de sa réserve. Additionnez les résultats des dés avec les modificateurs de dissipation, tels qu'un Pouvoir Irrésistible, pour obtenir le total de dissipation. \tabularnewline
%3 & La tentative de dissipation réussit si le total de dissipation est \textbf{supérieur ou égal} au total de lancement. Le sort est alors dissipé et le lancement échoue. Sinon, la tentative de dissipation échoue et le Sorcier à l'origine de cette tentative subit une \lostfocus{}. \tabularnewline
%\end{tabular}
%
%\vspace*{10pt}
%\begin{framed}
%\vspace*{-17pt}
%\basicsubtitle{Modificateurs Magiques}
%
%\noindent Sorcier Apprenti, Niveau 1 à 2 : +1
%
%\vspace*{3pt}
%\noindent Maître Sorcier, Niveau 3 à 4 : +2
%
%\vspace*{3pt}
%\noindent \overwhelmingpower{} : +1D3 +NDU.
%
%\vspace*{3pt}
%\noindent La somme des modificateurs hors \overwhelmingpower{} ne peut pas dépasser +3.
%
%\end{framed}
%
%\vspace*{\fill}
%\columnbreak
%
%\basicsubtitle{Table des \miscasts{}}
%
%\vspace*{-10pt}
%\begin{center}
%\begin{tabular}{cm{6.75cm}@{}}
%\hline
%\textbf{2 à 4} & \textbf{\breachintheveil}
%
%\vspace*{3pt}
%Centrez le gabarit de \distance{5} sur le lanceur. Toute figurine touchée par le gabarit subit une touche. Le lanceur doit subir une touche.
%
%\vspace*{3pt}
%Si \textbf{4} Dés de Pouvoir ont été utilisés, lancez un dé. Sur un résultat de 1 à 3, retirez le lanceur de la partie.
%
%\vspace*{3pt}
%Si \textbf{5} Dés de Pouvoir ont été utilisés, retirez le lanceur de la partie.\tabularnewline
%\textbf{5 à 6} & \textbf{\catastrophicdetonation}
%
%\vspace*{3pt}
%Centrez le gabarit de \distance{3} sur le lanceur. Toute figurine touchée par le gabarit subit une touche. Le lanceur doit subir une touche.\tabularnewline
%\textbf{7} & \textbf{\witchfire}
%
%\vspace*{3pt}
%L'unité du lanceur subit NDU touches, distribuées comme des tirs. Le lanceur ne peut cependant subir qu'une seule touche au plus.\tabularnewline
%\textbf{8 à 9} & \textbf{\sorcerousbacklash}
%
%\vspace*{3pt}
%Tous les Sorciers alliés subissent une touche. Une figurine en plusieurs éléments comprenant plusieurs Sorciers subit une touche par élément de figurine étant Sorcier. \tabularnewline
%\textbf{10 à 12} & \textbf{\amnesia}
%
%\vspace*{3pt}
%Le Niveau de Magie du lanceur est diminué de NDU-2. Rappel : un Sorcier perd un sort pour chaque Niveau de Magie perdu. Commencez par le sort ayant causé le \miscast{} puis tirez les autres au hasard.\tabularnewline
%\hline
%\end{tabular}
%\end{center}
%
%\vspace*{5pt}
%\noindent Les touches infligées par un \miscast{} ont une Force de NDU+2 et les règles \magicalattacks{} et \armourpiercing{1}. Le Sorcier ayant provoqué le \miscast{} ne peut utiliser aucune sauvegarde.
%
%\vspace*{5pt}
%\noindent Quel que soit le résultat du jet, retirez NDU Dés de Pouvoir de la réserve du propriétaire du lanceur.
%
%\vspace*{10pt}
%\begin{framed}
%\vspace*{-17pt}
%\basicsubtitle{\boundspells{}}
%
%\noindent Pour lancer un sort lié à un Objet de Sort avec succès, le jet de lancement doit être supérieur ou égal à son Niveau de Puissance.
%\begin{itemize}[label={-}, itemsep=3pt]
%\item Aucun modificateur positif ne peut être ajouté au jet de lancement.
%\item Un échec ne provoque pas de \lostfocus{} pour le lanceur.
%\item Un objet de sort ne bénéficie pas du bonus de lancement d'un Pouvoir Irrésistible.
%\item L'Attribut de la Voie est lancé normalement.
%\end{itemize}
%
%\noindent En cas de Pouvoir Irrésistible :
%\begin{itemize}[label={-}, itemsep=3pt]
%\item Si 4 dés ou plus ont été lancés, l'\boundspell{} est perdu et le sort ne peut plus être lancé de la partie.
%\item Retirez NDU Dés de Pouvoir de la réserve du propriétaire du lanceur.
%\end{itemize}
%\end{framed}
%
%\vspace*{\fill}
%\end{multicols}

\end{document}
