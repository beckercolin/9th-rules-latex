\documentclass[a4paper,8pt]{extarticle}

\usepackage[a4paper, top=2cm, bottom=2cm, left=1.5cm, right=1.5cm]{geometry}
\usepackage[T1]{fontenc}
\usepackage[utf8]{inputenc}
\usepackage[french]{babel}
\usepackage{graphicx}
\usepackage{microtype}
\usepackage{SIunits}
\usepackage{palatino}
\usepackage{xcolor}
\usepackage{caption} % pour sauter des lignes dans les légendes
\usepackage{array}
\usepackage{multicol}
\usepackage{colortbl}
\usepackage{textcomp} % pour l'apostrophe droite \textquotesingle
\usepackage{multirow}
\usepackage{array}

\usepackage[colorlinks=true]{hyperref}

\newcommand{\pouce}{\arcsecond}
\newcommand{\pied}{\arcminute}
\newcommand{\amel}[1]{\textcolor{blue}{[#1]}}
\newcommand{\base}{\textcolor{red}}
\newcommand{\amelbis}[1]{\textcolor{olive}{[[#1]]}}
\newcommand{\portee}[1] {Portée \unit{#1}{\pouce}}
\newcommand{\distance}[1] {\unit{#1}{\pouce}}
\newcommand{\result}[1] {\textquotesingle #1\textquotesingle}
\newcommand{\newrule}{\textcolor{green!50!black}}

\newcolumntype{M}[1]{>{\centering\let\newline\\\arraybackslash\hspace{0pt}}m{#1}}

\renewcommand{\arraystretch}{3.2}

\arrayrulecolor{black!30}
\setlength{\arrayrulewidth}{2pt}

\newcommand{\starttable}[2][black]{%
\vspace{0.3cm}
\begin{center}
\begin{tabular}{@{}>{\bf\LARGE}M{0.7cm}>{\raggedright}m{3cm}M{0.8cm}M{1.8cm}M{1.4cm}m{6.5cm}@{}}
\rowcolor[HTML]{#2} &
\textcolor{#1}{\textbf{Nom}} &
\textcolor{#1}{\textbf{Lancement}} &
\textcolor{#1}{\textbf{Type}} &
\textcolor{#1}{\textbf{Durée}} &
\centering\textcolor{#1}{\textbf{Effet}}
\tabularnewline
}

\newcommand{\closetable}{%
\end{tabular}
\end{center}
}

\makeatletter
\def\colors@alchemy{FFD966}
\def\colors@death{434343}
\def\colors@fire{FF0000}
\def\colors@heavens{C9DAF8}
\def\colors@light{FFF2CC}
\def\colors@nature{274E13}
\def\colors@shadows{999999}
\def\colors@wilderness{7F6000}
\def\colors@butchery{85200C}
\def\colors@change{9900FF}
\def\colors@forge{5B0F00}
\def\colors@biggreengods{38761D}
\def\colors@littlegreengods{93C47D}
\def\colors@lust{D5A6BD}
\def\colors@whitemagic{CFE2F3}
\def\colors@blackmagic{20124D}
\def\colors@disease{7F6000}
\def\colors@necromancy{000000}
\def\colors@ruin{69431C}
\def\colors@sand{FFD966}



\begin{document}

\input{titlepage_Paths.tex}

\newpage

{
\Large
\tableofcontents
}
\newpage

{\large
\section{Introduction}

\subsection{BF : Le 9\ieme Âge, qu'est-ce que c'est ?}

\emph{Batailles Fantastiques : Le 9\ieme Âge} est un jeu de figurines créé par une communauté, sans lien avec une quelconque entreprise. Le jeu met en scène deux armées s'affrontant, représentées par des figurines adéquates. Chaque armée est contrôlée par un joueur, bien qu'en certaines occasions, il puisse y avoir plus d'un joueur par camp (comme dans le cas d'une alliance). Le jeu se joue sur une planche de \unit{4}{\pied} par \unit{6}{\pied} (environ \unit{1.20}{\meter} par \unit{1.80}{\meter}), bien que la taille du plateau puisse être adaptée pour des batailles plus petites ou plus grandes. Les deux adversaires déploient leurs armées et jouent tour à tour pour faire agir leurs figurines. Chaque action a une certaine chance de réussite, ce pourquoi nous utilisons des dés. Pendant chaque tour de joueur, un joueur peut faire agir son armée pendant quatre phases de jeu consécutives : Mouvement, Magie, Tir et Corps à Corps. À la fin de la partie, un vainqueur est déterminé (ou un match nul est déclaré).

Toutes les règles du jeu, ainsi que les retours et suggestions peuvent être trouvés et donnés ici :
\begin{center}
\url{http://www.the-ninth-age.com/}
\end{center}

\subsection{Comment utiliser ce livre ?}

Dans ce livre, toutes les Disciplines de Magie sont présentées. Elles comprennent les 8 \textbf{Disciplines Communes} et les \textbf{Disciplines Spécifiques} des différentes armées, sont présentées.

Chaque Discipline comprend 8 sorts. Sept sorts sont numérotés de 0 à 6. Le huitième est appelé l’Attribut de la Discipline, qui est indiqué par la mention \og A \fg{} à la place du numéro du sort. Le sort 0 est le sort \textbf{Primaire} de la Discipline. Un Sorcier peut toujours choisir d'échanger l'un de ses sorts générés aléatoirement par le sort Primaire, plusieurs Sorciers peuvent connaître ce sort en même temps. Certains sorts disposent de plusieurs valeurs de lancement. La plus grande est pour la version \textbf{Améliorée} du sort. Parfois, les versions Améliorées voient leur portée et le nombre de cibles augmentés, tandis que pour d'autres sorts, les effets peuvent être modifiés. Les différences entre versions Améliorée et basique suivent un code couleur. La partie en noir s'applique à toutes les versions du sort. Les textes en \base{rouge} s'appliquent aux sorts basiques, tandis que les textes en \amel{bleu et encadrés de crochets simples} s'appliquent aux sorts Améliorés. Dans certains cas, il existe une deuxième amélioration au sort, marquée en \amelbis{vert olive et encadrée de crochets doubles}. 

%\vspace{1cm}
%\noindent Note: les changements de règles des sorts entre cette version et la version précédente sont généralement signalés avec \newrule{la couleur verte}. Néanmoins, comme il n'est pas toujours possible de colorer ainsi le changement exact (dans le cas où le changement concerne une version particulière d'un sort, par exemple), le numéro et le nom de chaque sort qui a changé sont \newrule{en vert}. Nous espérons que ceci attirera votre attention sur les changements effectués, même lorsqu'il n'est pas possible d'être plus précis.
}
\clearpage

\section{Disciplines Communes}

\subsection{Discipline de l'Alchimie}

\starttable{\colors@alchemy}
A &
Pierre Philosophale &
&
\portee{12} \newline Amélioration &
Dure un tour &
La cible gagne +1 en Sauvegarde d'Armure. Aucune figurine ne peut obtenir mieux que 3+ en Sauvegarde d'Armure grâce à ce sort.
\tabularnewline
\hline
0 & Plomb Fondu &
\base{9+}\newline  \amel{17+} &
\portee{24} \newline Malédiction \newline Projectile \newline Dégâts &
Immédiat &
La cible subit \base{1D6}/\amel{2D6} touches avec la règle \emph{Fusion du Métal}.
\tabularnewline
\hline
1 & Glaives d'Argent &
\base{7+} \newline \amel{11+} &
\base{\portee{18}} \newline \amel{\portee{36}} \newline Amélioration &
Dure un tour &
La cible gagne +1 pour toucher. Toutes les attaques, y compris de tir, gagnent les règles \emph{Attaques Magiques} et \emph{Perforant (+1)}.
\tabularnewline
\hline
2 & Oxydation du Cuivre &
\base{7+} \newline \amel{10+} &
\base{\portee{24}} \newline \amel{\portee{48}} \newline Malédiction &
Permanent &
La cible subit un malus de -1 sur sa Sauvegarde d'Armure.
\tabularnewline
\hline
3 & Pourpoint de Mercure &
\base{8+} \newline \amel{11+} &
\base{\portee{12}} \newline \amel{\portee{24}} \newline Amélioration &
Dure un tour &
La cible gagne les règles \emph{Camouflé} et \emph{Distrayant}. De plus, les attaques de corps à corps contre la cible subissent également une pénalité de -1 sur leur règle \emph{Perforant}.
\tabularnewline
\hline
4 & Lance de Fer &
\base{8+} \newline \amel{11+} &
\base{\portee{18}} \newline \amel{\portee{36}} \newline Malédiction \newline Projectile \newline Dégâts &
Immédiat &
La cible subit une touche avec les règles \emph{Fusion du Métal} et \emph{Blessures Multiples (1D3)}. Les rangs de l'unité ciblée sont pénétrés de la même façon que pour une \emph{Baliste}, mais l'attaque subit un malus de -1 pour blesser au lieu de -1 en Force pour chaque rang pénétré. 
\tabularnewline
\hline
5 & Lames d'Etain  &
\base{9+} \newline \amel{12+} &
\base{\portee{24}} \newline \amel{\portee{48}} \newline Malédiction &
Dure un tour &
Les figurines de l'unité visée ne reçoivent pas de bonus de Force par leurs armes de corps à corps non magiques. Les armes de tirs non magiques portées par l'unité ciblée ont -1 en Force. Ce sort n'affecte que les équipements standard et leur Force, aucune règle spéciale n'est altérée par ce sort.
\tabularnewline
\hline
6 & Transmutation en Or &
\base{15+} \newline \amel{18+} &
\base{\portee{12}} \newline \amel{\portee{24}} \newline Malédiction \newline Direct \newline Dégâts &
Immédiat \newline Dure un tour &
Le propriétaire de l'unité ciblée lance un dé pour chaque figurine de son unité, dans l'ordre qu'il choisit. Ignorez le premier 5+. Les figurines suivantes obtenant 5+ subissent une blessure avec la règle \emph{Blessures Multiples (10)} et n'autorisant aucune sauvegarde d'aucune sorte.\newline Les unités ennemies situées dans un rayon de \distance{12} autour de la cible souffrent de \emph{Stupidité}.
\tabularnewline
\closetable

\newpage

\subsection{Discipline des Cieux}

\starttable{\colors@heavens}
A &
Second Sceau &
&
Spécial &
Dure un tour &
L'armée du lanceur gagne un marqueur \og Second Sceau \fg{}. Ce marqueur peut être dépensé pour relancer un unique D6, s'il s'agit d'un jet pour toucher, d'un jet pour blesser ou d'une Sauvegarde d'Armure.
\tabularnewline
\hline
0 & Aquilon &
\base{7+}\newline \amel{10+} &
\base{\portee{12}} \newline \amel{\portee{36}} \newline Malédiction &
Dure un tour &
La cible subit un malus de -1 pour toucher et de -1 en Commandement. Toute figurine de l'unité cible qui veut employer des armes de tir ne nécessitant pas l'utilisation de la CT doit jeter 1D6. Sur 4+, elle ne peut pas tirer.
\tabularnewline
\hline
1 & Ouragan &
8+ &
\portee{18} \newline Malédiction &
Dure un tour &
La cible ne peut pas faire de Marche Forcée, ni utiliser la règle \emph{Vol}. Elle jette un dé de plus pour ses jets de charge, charge irrésistible, fuite et poursuite, et ignore celui ayant le résultat le plus élevé.
\tabularnewline
\hline
2 & Choc Foudroyant &
\base{8+}\newline \amel{11+} &
\base{\portee{24}} \newline \amel{\portee{48}} \newline Malédiction \newline Projectile \newline Dégâts &
Immédiat &
La cible subit 1D6 touches de Force 6 avec la règle \emph{Attaques Foudroyantes}.
\tabularnewline
\hline
3 & Harmonie des Sphères &
\base{10+}\newline \amel{13+} &
\base{\portee{12}} \newline \amel{\portee{24}} \newline Amélioration &
Dure un tour &
La cible peut relancer au choix les jets ratés pour toucher, pour blesser ou ses Sauvegardes d’Armure. Déclarez votre choix avant de jeter le sort.
\tabularnewline
\hline
4 & Fléau du Ponant &
\base{10+}\newline \amel{13+} &
\base{\portee{12}} \newline \amel{\portee{24}} \newline Malédiction &
Dure un tour &
La cible doit relancer au choix les jets réussis pour toucher, pour blesser ou ses Sauvegardes d'Armure. Déclarez votre choix avant de jeter le sort.
\tabularnewline
\hline
5 & Déluge d'Éclairs &
13+ &
\portee{24} \newline Malédiction \newline Direct \newline Dégâts &
Immédiat &
Ciblez une unité, puis jetez 1D6. Sur 3+, choisissez une nouvelle cible située à moins de \distance{6} de la première. Continuez ainsi, de cible en cible à \distance{6} d'écart, jusqu'à ce que vous obteniez un \result{1} ou un \result{2}, ou qu'aucune cible ne soit éligible. Aucune unité ne peut être ciblée deux fois. Chaque cible subit 1D6 touches de Force 6 avec la règle \emph{Attaques Foudroyantes}.
\tabularnewline
\hline
6 & Appel de la Comète &
\base{13+}\newline \amel{16+} &
Marqueur &
Permanent &
Posez un marqueur où vous voulez sur le champ de bataille.

\base{À la fin de chaque Phase de Magie suivante, jetez un dé. Si le résultat est de 4+, la comète arrive}. 

\amel{Le lanceur choisit le tour de joueur, autre que celui en cours, où la comète arrive et l'écrit secrètement. La comète arrivera à la fin de la phase de Magie de ce tour de joueur.}

À la fin de chaque Phase de Magie pendant laquelle la comète n'est pas arrivée, ajoutez un marqueur au même endroit. Quand la comète s'écrase, toutes les unités situées dans un rayon de \distance{2D6+X} subissent 2D6 touches de Force 4+X, où X est égal au nombre de marqueurs. Enlevez ensuite tous les marqueurs, et le sort prend fin.
\tabularnewline
\closetable



\newpage

\subsection{Discipline du Feu}


\starttable[white]{\colors@fire}
A &
Feu Enragé &
&
\portee{24} \newline Malédiction \newline Projectile \newline Dégâts &
Immédiat &
La cible subit 1D3 touches de Force 4 avec la règle \emph{Attaques Enflammées}.
\tabularnewline
\hline
0 & Boule de Feu &
\base{5+}\newline \amel{10+} \newline \amelbis{14+} &
\base{\portee{24}} \newline \amel{\portee{36}} \newline \amelbis{\portee{48}} \newline Malédiction \newline Projectile \newline Dégâts &
Immédiat &
La cible subit \base{1D6}/\amel{2D6}/\amelbis{3D6} touches de Force 4 avec la règle \emph{Attaques Enflammées}.
\tabularnewline
\hline
1 & Cascade Ardente &
\base{7+}\newline \amel{10+} &
\base{\portee{18}} \newline \amel{\portee{36}} \newline Malédiction &
Reste en jeu &
A la fin de chaque Phase de Magie, la cible subit 1D6 touches de Force 4 avec la règle \emph{Attaques Enflammées}.
\tabularnewline
\hline
2 & Épées Flamboyantes &
\base{7+}\newline \amel{13+} &
\base{\portee{24}} \newline \amel{\portee{6}} \newline \amel{Aura} \newline Amélioration &
Dure un tour &
Les attaques de tir et de corps à corps de la cible ont +1 pour blesser et suivent les règles \emph{Attaques Magiques} et \emph{Attaques Enflammées}.
\tabularnewline
\hline
3 & Jet de Flammes &
\base{9+}\newline \amel{12+} &
\base{\portee{18}} \newline \amel{\portee{36}} \newline Marqueur \newline Direct \newline Gabarit de ligne &
Immédiat &
Chaque figurine sous le gabarit subit une touche de Force 4 suivant la règle \emph{Attaques Enflammées}. Les unités ennemies touchées par le gabarit doivent effectuer un test de Panique.
\tabularnewline
\hline
4 & Traits Enflammés &
\base{9+}\newline \amel{12+} &
\base{\portee{24}} \newline \amel{\portee{36}} Malédiction \newline Projectile \newline Dégâts &
Immédiat &
La cible subit 1 touche de Force 4 suivant la règle \emph{Attaques Enflammées} pour chaque rang ou colonne (le lanceur doit choisir entre les deux) de l'unité ciblée.
\tabularnewline
\hline
5 & Cage Incandescente &
\base{10+}\newline \amel{13+} &
\base{\portee{24}} \newline \amel{\portee{48}} \newline Malédiction &
Reste en jeu &
La cible subit 1D6 touches de Force 4 avec la règle \emph{Attaques Enflammées}. A la fin de chaque Phase de jeu, chaque figurine de l'unité subit une touche de Force 4 avec la règle \emph{Attaques Enflammées} si l'unité a effectué une ou plusieurs des actions suivantes durant la phase: Charge, Charge Irrésistible, Fuite, Marche Forcée, Mouvement Simple, Pivot, Poursuite, Reformation ou Reformation de Combat. Une même unité ne peut subir ces touches qu'une fois par Phase.
\tabularnewline
\hline
6 & Souffler sur les Braises &
\base{11+}\newline \amel{14+} &
\base{\portee{24}} \newline \amel{\portee{48}} \newline Amélioration &
Dure un tour &
La cible gagne +1 en Endurance, une \emph{Sauvegarde Invulnérable (5+)} et la règle \emph{Né du Feu}.
\tabularnewline
\closetable







\newpage

\subsection{Discipline de la Lumière}

\starttable{\colors@light}
A &
Lumière Bienveillante &
&
\portee{48} \newline Amélioration &
Dure un tour &
La cible gagne +1 en Commandement. Aucune figurine ne peut être affectée par ce sort plus d'une fois par Phase de Magie.
\tabularnewline
\hline
0 & Regard Embrasé &
\base{5+}\newline \amel{14+} &
\base{\portee{24}} \newline \amel{\portee{48}} \newline Malédiction \newline Projectile \newline Dégâts &
Immédiat &
La cible subit 1D6 touches de Force \base{4}/\amel{6} avec la règle \emph{Attaques Enflammées}.

Si la cible suit la règle spéciale \emph{Mort-vivant} ou \emph{D'Outre-Monde}, elle subit 2D6 touches à la place.
\tabularnewline
\hline
1 & Bouclier Protecteur &
\base{7+}\newline \amel{10+} &
\base{\portee{24}} \newline \amel{\portee{6}} \newline \amel{Aura} \newline Amélioration &
Dure un tour &
La cible gagne les règles \emph{Camouflé} et \emph{Distrayant}. Les figurines voulant attaquer la cible avec des armes de tir ne nécessitant pas l'utilisation de la CT doivent jeter 1D6. Sur 4+, le ou les tirs sont perdus.
\tabularnewline
\hline
2 & Étincelle de Courage &
7+ &
\portee{24} \newline Amélioration &
Immédiat \newline Dure un tour &
Si la cible n'est pas engagée dans un corps à corps et n'est pas en fuite, elle peut effectuer immédiatement une Reformation. Si la cible est engagée au corps à corps, elle peut effectuer une Reformation de Combat. De plus, la cible apporte un bonus de +2 au Résultat de Combat de son camp.
\tabularnewline
\hline
3 & Célérité &
\base{8+}\newline \amel{12+} &
\base{\portee{24}} \newline \amel{\portee{12}} \newline \amel{Aura} \newline Amélioration &
Dure un tour &
La Capacité de Combat et l'Initiative reçoivent un bonus de +3.
\tabularnewline
\hline
4 & Toile Scintillante &
\base{9+}\newline \amel{12+} &
\base{\portee{24}} \newline \amel{\portee{48}} \newline Malédiction &
Dure un tour &
Au début de chaque phase, jetez un dé. Sur 5+, la cible ne peut pas effectuer les actions suivantes : \newline
\textbf{Phase de Mouvement}: déclarer une charge. \newline
\textbf{Phase de Magie}: lancer des sorts. \newline
\textbf{Phase de Tir}: tirer. \newline
\textbf{Phase de Corps à corps}: poursuivre et effectuer une Charge Irrésistible.
\tabularnewline
\hline
5 & Distorsion Temporelle &
\base{10+}\newline \amel{15+} &
\portee{12} \newline Amélioration \newline \amel{Aura} &
Dure un tour &
La cible gagne +1 Attaque, la règle \emph{Réflexes foudroyants} et double son Mouvement jusqu'à un maximum de 10.
\tabularnewline
\hline
6 & Exorcisme &
\base{10+}\newline \amel{13+} &
\portee{24} \newline Malédiction \newline Projectile \newline Dégâts &
Immédiat &
La cible subit 2D6 touches de Force 4 avec la règle \emph{Attaques Divines}. Pour chaque Sorcier connaissant au moins un sort de la Discipline de la Lumière dans un rayon de \distance{12} autour du lanceur, ajoutez \base{+1 pour blesser}/\amel{+1 en Force}.

Si la cible suit la règle spéciale \emph{Mort-Vivant} ou \emph{D'Outre-Monde}, elle subit 3D6 touches à la place.
\tabularnewline
\closetable






\newpage

\subsection{Discipline de la Mort}

\starttable[white]{\colors@death}
A &
Nuage de Désespoir &
 &
\portee{24} \newline Malédiction &
Dure un tour &
La cible subit un malus de -1 en Commandement. Aucune figurine ne peut être affectée par ce sort plus d'une fois par Phase de Magie.
\tabularnewline
\hline
0 & Le Baiser de la \newline Faucheuse &
\base{8+}\newline \amel{11+} &
\portee{18} \newline Focalisé \newline Malédiction \newline Direct \newline Dégâts &
Immédiat &
La cible subit 1 blessure avec la règle \emph{Perforant (6)}. \amel{Si la figurine visée perd un PV, le lanceur Récupère 1 PV}.
\tabularnewline
\hline
1 & Malédiction Létale &
\base{6+}\newline \amel{9+} &
\portee{24} \newline Malédiction &
Dure un tour &
La cible subit un malus de -1 en Force \amel{et -1 en Endurance}, jusqu'à un minimum de 1.
\tabularnewline
\hline
2 & Gangrène Spirituelle &
\base{7+}\newline \amel{9+} &
\portee{24} \newline Malédiction \newline Projectile \newline Dégâts &
Immédiat &
La cible subit \base{2D6}/\amel{3D6} touches de Force 2 avec la règle \emph{Perforant (6)}.
\tabularnewline
\hline
3 & Sangsue Psychique &
\base{7+}\newline \amel{10+} &
\base{\portee{12}} \newline \amel{\portee{24}} \newline Focalisé \newline Malédiction \newline Direct \newline Dégâts &
Immédiat &
Le lanceur et la cible lancent chacun 1D6 et ajoutent leur Commandement actuel au résultat. Si le total du lanceur est plus élevé, la cible subit un nombre de blessures avec la règle \emph{Perforant (6)} égal à la différence 
entre leurs totaux respectifs.
\tabularnewline
\hline
4 & Moisson d'Âmes &
\base{8+}\newline \amel{12+} &
\portee{24} \newline Amélioration &
Dure un tour &
Les attaques au corps à corps de la cible gagnent la règle \emph{Attaques Divines}.
\newline \amel{Les Sauvegardes d'Armure réussies contre ces attaques doivent être relancées}.
\tabularnewline
\hline
5 & Regard des Abysses &
10+ &
\portee{18} \newline Amélioration &
Dure un tour &
La cible gagne les règles \emph{Coup Fatal} et \emph{Peur}.
\tabularnewline
\hline
6 & Maelstrom d'Âmes &
14+ &
Vortex \newline (\portee{6}, Gabarit \distance{1}) \newline Marqueur &
Immédiat &
Les figurines touchées par le gabarit doivent passer un test d'Initiative. En cas d'échec, elles subissent chacune une blessure avec les règles \emph{Perforant (6)} et \emph{Blessures Multiples (Artillerie)}. La \emph{Regénération} n'est pas permise contre ces blessures.
\tabularnewline
\closetable





\newpage

\subsection{Discipline de la Nature}


\starttable[white]{\colors@nature}
A &
Efflorescence Vitale &
&
\portee{12} \newline Focalisé, Amélioration &
Immédiat &
La cible Récupère un PV précédemment perdu. Aucune figurine ne peut gagner plus d'un PV par Phase de Magie avec ce sort.
\tabularnewline
\hline
0 & Transfusion Tellurique &
\base{4+}\newline \amel{8+} &
\base{Unité du lanceur} \newline \amel{\portee{12}} \newline Amélioration &
Dure un tour &
La cible gagne la règle \emph{Regénération (5+)}/\amelbis{\emph{Regé\-nération (4+)} avec le Trône de Chêne}. 
\tabularnewline
\hline
1 & Maître de la Terre &
\base{6+}\newline \amel{11+} &
\portee{18} \newline Malédiction \newline Direct \newline Dégâts &
Immédiat &
La portée de ce sort peut être mesurée à partir du lanceur ou de tout Terrain Infranchissable ou Colline du champ de bataille. La cible subit \base{1D6}/\amel{2D6} touches de Force 4/\amelbis{Force 5 avec le Trône de Chêne}.
\tabularnewline
\hline
2 & Trône de Chêne &
7+ &
Lanceur &
Reste en jeu &
Si le lanceur provoque un Fiasco lorsqu'il lance un sort autre que le Trône de Chêne, il est compté comme ayant lancé un dé de pouvoir de moins, jusqu'à un minimum de deux, pour les effets de ce \emph{Fiasco}. Si le Trône de Chêne est en jeu alors que certains sorts de cette Discipline sont lancés, la version \amelbis{augmentée} est utilisée.
\tabularnewline
\hline
3 & Esprits des Bois &
\base{9+}\newline \amel{13+} &
\portee{12} \newline \base{Amélioration} \newline \amel{Malédiction} &
Dure un tour &
La portée de ce sort peut être mesurée à partir du lanceur ou de n'importe quelle Forêt sur le champ de bataille. Toutes les figurines de l'unité ciblée sont considérées comme étant dans une Forêt.
\tabularnewline
\hline
4 & De la Chair à la Pierre &
11+ &
\portee{24} \newline Amélioration &
Dure un tour &
La cible gagne +2 en Endurance/\amelbis{+4 en Endurance avec le Trône de Chêne}.
\tabularnewline
\hline
5 & Renaissance &
\base{10+}\newline \amel{15+} &
\base{\portee{24}} \newline \amel{\portee{48}} \newline Amélioration &
Immédiat &
L'unité ciblée Ressuscite 1D3+1 PVs/\amelbis{1D6+1 PVs avec le Trône de Chêne}.
\newline Les unités de taille Moyenne et Grande divisent par deux le nombre de PVs récupérés, en arrondissant au supérieur.
\tabularnewline
\hline
6 & Fureur des Élémentaires &
\base{15+}\newline \amel{18+} &
\base{\portee{12}} \newline \amel{\portee{24}} \newline Malédiction \newline Direct \newline Dégâts &
Immédiat &
Chaque figurine de l'unité ciblée doit passer un test de Force, dans l'ordre choisi par le propriétaire de l'unité. Ignorez le premier test raté. Tout autre figurine ratant son test subit une blessure sans aucune sauvegarde possible avec la règle \emph{Blessures Multiples (10)}.
\tabularnewline
\closetable







\newpage

\subsection{Discipline des Ombres}

\starttable[white]{\colors@shadows}
A &
Coursier d'Éther &
&
\portee{12} \newline Focalisé \newline Amélioration &
Immédiat &
Ce sort ne peut être lancé que sur une unité constituée d'une seule figurine ou sur un personnage. La cible peut effectuer un Mouvement Magique de \distance{10} avec la règle \emph{Vol}. Les \emph{Grandes Cibles} ne peuvent effectuer qu'un mouvement de \distance{2}.
\tabularnewline
\hline
0 & Miasmes Obscurs &
\base{4+}\newline \amel{7+} &
\portee{48} \newline Malédiction &
Dure un tour &
La cible subit un malus de \base{1}/\amel{-1D3} à une des caractéristiques parmi M, CC, CT ou I (jusqu'à un minimum de 1). Le lanceur doit préciser laquelle avant de lancer le sort.
\tabularnewline
\hline
1 & Orbe Ténébreuse &
\base{5+}\newline \amel{8+} &
\portee{24} \newline Marqueur &
Dure un tour &
Placez un gabarit de \distance{3} sur le point de terrain choisi, et à au moins \distance{1} de toute unité. La zone couverte par le gabarit compte pour toute figurine comme étant un Terrain Dangereux (1), et offre un Couvert Lourd.
\newline \amel{Cette zone est également un Décor Occultant.}
\tabularnewline
\hline
2 & Parti en Fumée &
\base{7+}\newline \amel{13+} &
\base{\portee{18}} \newline \amel{\portee{36}} \newline Malédiction &
Reste en jeu &
La cible perd \base{1}/\amel{1D3} en Force, jusqu'à un minimum de 1.
\tabularnewline
\hline
3 & Char de l'Ankou &
\base{11+}\newline \amel{14+} &
\base{\portee{12}} \newline \amel{\portee{24}} \newline Amélioration &
Immédiat &
La cible peut effectuer un Mouvement Magique avec la règle \emph{Vol} de \distance{8}.
\newline À l'issue de ce mouvement, elle peut se reformer, ce qui ne l'empêchera pas de tirer.
\tabularnewline
\hline
4 & Expérience de Mort \newline Imminente &
\base{9+}\newline \amel{15+} &
\base{\portee{18}} \newline \amel{\portee{36}} \newline Malédiction &
Reste en jeu &
La cible perd \base{1}/\amel{1D3} en Endurance, jusqu'à un minimum de 1.
\tabularnewline
\hline
5 & Ombres Dévorantes &
12+ &
\portee{24} \newline Malédiction \newline Direct \newline Dégâts &
Immédiat &
Placez le centre d'un gabarit de \distance{3} sur l'unité ciblée, à portée du sort, puis faites-le dévier de \distance{1D6}. Toutes les figurines sous le gabarit doivent passer un test d'Initiative ou subir une blessure avec les règles \emph{Blessures Multiples (Artillerie)} et \emph{Perforant (6)} ne permettant pas la \emph{Régénération}.
\tabularnewline
\hline
6 & Scalpel Psychique &
\base{15+}\newline \amel{18+} &
\base{\portee{18}} \newline \amel{\portee{36}} \newline Amélioration &
Dure un tour &
Les attaques au corps à corps non spéciales de la cible blessent automatiquement et gagnent la règle \emph{Perforant (1)}.
\tabularnewline
\closetable







\newpage

\subsection{Discipline de la Sauvagerie Bestiale}
\starttable[white]{\colors@wilderness}
A &
La Chasse Sauvage &
&
\portee{12} \newline Amélioration &
Immédiat &
Ce sort peut uniquement affecter les unités de type Bête de Guerre, Bête Monstrueuse, Cavalerie, Cavalerie Monstrueuse, Char, Monstre, Monstre Monté, toute unité du Livre d'Armée Hardes Bestiales ou enfin l'unité du lanceur. \newline
La cible peut effectuer un Mouvement Magique de \distance{1D3+2}.
\tabularnewline
\hline
0 & La Bête qui Sommeille &
\base{9+} \newline \amel{12+} &
\base{\portee{12}} \newline \amel{\portee{24}} \newline Amélioration &
Dure un tour &
La cible gagne +1 en Force et +1 en Endurance.
\tabularnewline
\hline
1 & Essaim d'Insectes &
\base{5+}\newline \amel{8+} &
\base{\portee{24}} \newline \amel{\portee{48}} \newline Malédiction \newline Projectile \newline Dégâts &
Immédiat &
La cible subit 5D6 touches de Force 1.
\tabularnewline
\hline
2 & Rage Intérieure &
\base{4+} \newline \amel{8+} &
\base{Unité du lanceur} \newline \amel{\portee{12}} \newline Universel &
Dure un tour &
La cible gagne la règle \emph{Frénésie}. 
\tabularnewline
\hline
3 & Javelot d'Ambre &
\base{8+} \newline \amel{14+} &
\portee{24} \newline  Malédiction \newline Projectile \newline Dégâts &
Immédiat &
La cible subit une touche de Force \base{6}/\amel{10} avec les règles \emph{Blessures Multiples (\base{1D3}/\amel{Artillerie})} et \emph{Perforant (6)}. Les rangs de l'unité ciblée sont pénétrés de la même façon que par une \emph{Baliste}. 
\tabularnewline
\hline
4 & Tempête Furieuse &
10+ &
\portee{24} 
\newline Malédiction &
Dure un tour &
La cible ne peut pas effectuer d'attaques de tir et ne peut pas utiliser la règle \emph{Vol}.
\tabularnewline
\hline
5 & Calamité des Bois Sauvages &
\base{9+} \newline \amel{12+} &
\base{\portee{36}} \newline \amel{\portee{72}} \newline Malédiction &
Dure un tour &
La cible subit un malus de -1 pour toucher et traite tous les terrains, y compris les Terrains Découverts, comme des Terrains Dangereux (2).
\tabularnewline
\hline
6 & Métamorphose \newline Monstrueuse &
\base{11+} \newline \amel{14+} &
\base{\portee{6}} \newline \amel{\portee{12}} \newline Universel \newline Focalisé \newline Personnage uniquement &
Dure un tour &
La cible subit des changements de caractéristiques et de règles spéciales selon les aspects ci-dessous, à choisir au moment de lancer le sort : \newline
\textbf{Aspect de l'Hydre}. CC6, F5, E5, A6, \emph{Regénération (4+)} \newline
\textbf{Aspect de la Manticore}. CC6, F5, E5, A4, \emph{Coup Fatal}, \emph{Blessures Multiples (1D3)} \newline
\textbf{Aspect du Dragon}. CC6, F6, E6, A3, \emph{Attaque de Souffle (Force 4, Attaques Enflammées)} \newline
\tabularnewline
\closetable

\newpage
\subsection{Discipline des Huit Vents}

La Discipline des Huit Vents est accessible à de rares Sorciers. Elle regroupe les sorts Primaires des huit Disciplines Communes, et chacun de ces sorts déclenche l'Attribut de sa Discipline lorsqu'il est lancé avec succès.

\begin{center}
\begin{tabular}{>{\bf}M{2cm}>{\raggedright}m{1.8cm}M{0.8cm}M{1.8cm}M{1.2cm}m{6cm}}
\rowcolor{black!30} Discipline &
\textbf{Nom} &
\textbf{Lancement} &
\textbf{Type} &
\textbf{Durée} &
\centering\textbf{Effet}
\tabularnewline
\cellcolor[HTML]{\colors@alchemy} & Pierre Philosophale & A & \portee{12} \newline Amélioration & Dure un tour & La cible gagne +1 en Sauvegarde d'Armure. Aucune figurine ne peut obtenir mieux que 3+ de Sauvegarde d'Armure grâce à ce sort. \\
\multirow{-1}{*}[0.8cm]{Alchimie}
\cellcolor[HTML]{\colors@alchemy} & Plomb Fondu & \base{9+}\newline  \amel{17+} & \portee{24} \newline Malédiction \newline Projectile \newline Dégâts & Immédiat & La cible subit \base{1D6}/\amel{2D6} touches avec la règle \emph{Fusion du Métal}.
\tabularnewline
\hline
\cellcolor[HTML]{\colors@heavens} & Second Sceau & A & Spécial & Dure un tour & L'armée du lanceur gagne un marqueur \og Second Sceau \fg{}. Ce marqueur peut être dépensé pour relancer un unique D6, s'il s'agit d'un jet pour toucher, d'un jet pour blesser ou d'une Sauvegarde d'Armure. \\
\multirow{-1}{*}[0.7cm]{Cieux}
\cellcolor[HTML]{\colors@heavens} & Aquilon & \base{7+}\newline \amel{10+} & \base{\portee{12}} \newline \amel{\portee{36}} \newline Malédiction & Dure un tour & La cible subit un malus de -1 pour toucher et de -1 en Commandement. Toute figurine de l'unité cible qui veut employer des armes de tir ne nécessitant pas l'utilisation de la CT doit jeter 1D6. Sur 4+, le ou les tirs sont perdus.
\tabularnewline
\hline
\cellcolor[HTML]{\colors@fire} & Feu Enragé & A & \portee{18} \newline Malédiction \newline Direct \newline Dégâts & Immédiat & La cible subit 1D3 touches de Force 4 avec la règle \emph{Attaques Enflammées}.\\
\multirow{-1}{*}[0.9cm]{\textcolor{white}{Feu}}
\cellcolor[HTML]{\colors@fire} & Boule de Feu & \base{5+}\newline \amel{10+} \newline \amelbis{14+} & \base{\portee{24}} \newline \amel{\portee{36}} \newline \amelbis{\portee{48}} \newline Malédiction \newline Projectile \newline Dégâts & Immédiat & La cible subit \base{1D6}/\amel{2D6}/\amelbis{3D6} touches de Force 4 avec la règle \emph{Attaques Enflammées}.
\tabularnewline
\hline
\cellcolor[HTML]{\colors@light} & Lumière Bienveillante & A & \portee{48} \newline Amélioration & Dure un tour & La cible gagne +1 en Commandement. Aucune figurine ne peut être affectée par ce sort plus d'une fois par Phase de Magie.\\
\multirow{-1}{*}[1.0cm]{Lumière}
\cellcolor[HTML]{\colors@light} & Regard Embrasé & \base{5+}\newline \amel{14+} & \base{\portee{24}} \newline \amel{\portee{48}} \newline Malédiction \newline Projectile \newline Dégâts & Immédiat & La cible subit 1D6 touches de Force \base{4}/\amel{6} avec la règle \emph{Attaques Enflammées}. Si la cible suit la règle spéciale \emph{Mort-vivant} ou \emph{D'Outre-Monde}, elle subit 2D6 touches à la place.
\tabularnewline
\hline
\end{tabular}
\end{center}

\newpage

\begin{center}
\begin{tabular}{>{\bf}M{2cm}>{\raggedright}m{1.8cm}M{0.8cm}M{1.8cm}M{1.2cm}m{6cm}}
\rowcolor{black!30} Discipline &
\textbf{Nom} &
\textbf{Lancement} &
\textbf{Type} &
\textbf{Durée} &
\centering\textbf{Effet}
\tabularnewline
\cellcolor[HTML]{\colors@death} & Nuage de Désespoir & A & \portee{24} \newline Malédiction & Dure un tour & La cible subit un malus de -1 en Commandement. Aucune figurine ne peut être affectée par ce sort plus d'une fois par Phase de Magie.\\
\multirow{-1}{*}[0.8cm]{\textcolor{white}{Mort}}
\cellcolor[HTML]{\colors@death} & Le Baiser de la \newline Faucheuse & \base{8+}\newline \amel{11+} & \portee{18} \newline Focalisé \newline Malédiction \newline Direct \newline Dégâts & Immédiat & La cible subit 1 blessure avec la règle \emph{Perforant (6)}. \amel{Si la figurine visée perd un PV, le lanceur Récupère 1 PV}.
\tabularnewline
\hline
\cellcolor[HTML]{\colors@nature} & Efflorescence Vitale & A & \portee{12} \newline Focalisé, Amélioration & Immédiat & La cible Récupère un PV précédemment perdu. Aucune figurine ne peut gagner plus d'un PV par Phase de Magie avec ce sort.\\
\multirow{-1}{*}[0.8cm]{\textcolor{white}{Nature}}
\cellcolor[HTML]{\colors@nature} & Transfusion Tellurique & \base{4+}\newline \amel{8+} & \base{Unité du lanceur} \newline \amel{\portee{12}} \newline Amélioration & Dure un tour & La cible gagne la règle \emph{Regénération (5+)}. 
\tabularnewline
\hline
\cellcolor[HTML]{\colors@shadows} & Coursier d'Éther & A & \portee{12} \newline Focalisé \newline Amélioration  & Immédiat & Ce sort ne peut être lancé que sur une unité constituée d'une seule figurine ou sur un personnage. La cible peut effectuer un Mouvement Magique de \distance{10} avec la règle \emph{Vol}. Les \emph{Grandes Cibles} ne peuvent effectuer qu'un mouvement de \distance{2}.\\
\multirow{-1}{*}[1.1cm]{\textcolor{white}{Ombres}}
\cellcolor[HTML]{\colors@shadows} & Miasmes Obscurs & \base{4+}\newline \amel{7+} & \portee{48} \newline Malédiction & Dure un tour &
La cible subit un malus de \base{1}/\amel{-1D3} à une des caractéristiques parmi M, CC, CT ou I (jusqu'à un minimum de 1). Le lanceur doit préciser laquelle avant de lancer le sort.
\tabularnewline
\hline
\cellcolor[HTML]{\colors@wilderness} & La Chasse Sauvage & A & \portee{12} \newline Amélioration & Immédiat & Ce sort peut uniquement affecter les unités de type Bête de Guerre, Bête Monstrueuse, Cavalerie, Cavalerie Monstrueuse, Char, Monstre, Monstre Monté, toute unité du Livre d'Armée Hardes Bestiales ou enfin l'unité du lanceur.\newline
La cible peut effectuer un Mouvement Magique de \distance{1D3+2}.\\
\multirow{-2}{*}[1.1cm]{\begingroup \renewcommand{\arraystretch}{1} \begin{tabular}{l}\textcolor{white}{Sauvagerie} \\ \textcolor{white}{Bestiale}\end{tabular} \endgroup}
\cellcolor[HTML]{\colors@wilderness} & La Bête qui Sommeille & \base{9+} \newline \amel{12+} & \base{\portee{12}} \newline\amel{\portee{24}} \newline Amélioration & Dure un tour & La cible gagne +1 en Force et +1 en Endurance.
\tabularnewline
\hline

\end{tabular}
\end{center}


\newpage

\section{Disciplines Spécifiques}

\subsection{Discipline de la Boucherie}

\starttable[white]{\colors@butchery}
A &
Bouillon de Sang &
&
Lanceur &
Immédiat \newline Dure un tour &
La cible Récupère un PV précédemment perdu et gagne +1 en Endurance. De plus, les \emph{Attaques Empoisonnées} perdent cette règle contre la cible.
\tabularnewline
\hline
0 & Brise-Dents &
\base{7+}\newline \amel{11+} &
\base{\portee{18}} \newline \amel{\portee{12}} \newline \amel{Aura} \newline Amélioration &
Dure un tour &
La cible gagne +1 en Endurance. 
\tabularnewline
\hline
1 & Aspire-Moelle &
\base{6+}\newline \amel{9+} &
\base{\portee{12}} \newline \amel{\portee{24}} \newline Amélioration &
Dure un tour &
La cible gagne la règle \emph{Tenace}.
\tabularnewline
\hline
2 & Festin de Tripaille &
\base{6+}\newline \amel{10+} &
\base{\portee{18}} \newline \amel{\portee{12}} \newline \amel{Aura} \newline Amélioration &
Dure un tour &
La cible gagne +1 en Force.
\tabularnewline
\hline
3 & Concasseur d'Os &
\base{7+}\newline \amel{12+} &
\base{\portee{24}} \newline \amel{\portee{36}} \newline Malédiction \newline Projectile \newline Dégâts &
Immédiat &
La cible subit 2D6 touches de Force \base{2}/\amel{3} avec la règle \emph{Perforant (6)}.
\tabularnewline
\hline
4 & Suce-Cervelle &
\base{7+}\newline \amel{10+} &
\base{\portee{36}} \newline \amel{\portee{72}} \newline Malédiction &
Immédiat \newline Dure un tour &
La cible doit immédiatement effectuer un test de Panique.
\newline \amel{De plus, toutes les unités gagnent la règle \emph{Haine} contre la cible.}
\tabularnewline
\hline
5 & Cœur de Troll &
\base{11+}\newline \amel{14+} &
\base{\portee{12}} \newline \amel{\portee{24}} \newline Amélioration &
Dure un tour &
La cible gagne la \emph{Régénération (4+)}.
\tabularnewline
\hline
6 & La Gueule &
\base{12+} \newline \amel{14+} &
\base{\portee{18}} \newline \amel{\portee{24}} \newline Malédiction \newline Direct \newline Dégâts &
Immédiat \newline Dure un tour &
La cible subit immédiatement 1D6 touches de Force 5 avec la règle \emph{Perforant (6)}. Si une ou plusieurs blessures sont infligées, la cible ne peut pas faire de Marche Forcée et jette un dé de moins que la norme pour déterminer ses distances de charge, fuite, poursuite et Charge Irrésistible.
\tabularnewline
\closetable






\newpage

\subsection{Discipline du Changement}

\starttable[white]{\colors@change}
A &
Vent de l'Altération &
&
Lanceur &
Permanent &
L'armée gagne un marqueur Altération. Lorsqu'une unité de l'armée lance un sort de la Discipline du Changement non lié à un objet de sort, un marqueur peut être dépensé pour chaque Dé de Pouvoir obtenant un \result{1}. Pour chaque marqueur ainsi utilisé, le \result{1} peut être relancé.
\tabularnewline
\hline
0 & Feu d'Azur &
\base{5+}\newline \amel{8+} \newline \amelbis{11+} &
\base{\portee{24}} \newline \amel{\portee{48}} \newline \amelbis{\portee{48}} \newline Malédiction \newline Projectile \newline Dégâts &
Immédiat &
La cible subit \base{1D6}/\amel{1D6}/\amelbis{1D6+1} touches de Force 1D6/\amel{1D6}/\newline\amelbis{1D6+1} avec la règle \emph{Flammes de l'Enfer}.
\tabularnewline
\hline
1 & Confusion &
9+ &
\portee{24} \newline Malédiction &
Dure un tour &
La cible ne peut plus utiliser les règles \emph{Tenez les Rangs} et \emph{Présence Charismatique}.
\tabularnewline
\hline
2 & Bûcher Carmin &
\base{6+}\newline \amel{11+} &
\base{Lanceur} \newline \amel{\portee{24}} \newline \amel{Focalisé} \newline \amel{Amélioration} &
Dure un tour &
La cible gagne la règle \emph{Attaque de Souffle (Force 1D3+2, Flammes de l'Enfer)}.
\tabularnewline
\hline
3 & Éclair Fluctuant &
7+ &
\portee{24} \newline Malédiction \newline Projectile \newline Dégâts &
Immédiat &
La cible subit une touche de Force 1D6+4 avec les règles \emph{Blessures Multiples (1D3)}, \emph{Flammes de l'Enfer} et \emph{Perforant (6)}. Les rangs de l'unité ciblée sont pénétrés de la même façon que par une \emph{Baliste}. 
\tabularnewline
\hline
4 & Rabotage Mental &
7+ &
\portee{18} \newline Focalisé \newline Direct \newline Malédiction \newline Dégâts &
Immédiat &
Le lanceur et la cible lancent chacun 1D6 et ajoutent leur niveau de Magie au résultat. Si le total du lanceur est plus élevé, la cible subit une touche de Force 3 avec la règle \emph{Perforant (6)} et perd un niveau de Magie, si elle en a.
\tabularnewline
\hline
5 & Traîtrise &
9+ &
\portee{24} \newline Malédiction &
Dure un tour &
Lorsque la cible effectue une ou des attaques de corps à corps ou de tir, tous ses jets pour toucher de \result{1} (après relances éventuelles) sont mis de côté. Pour chaque dé ainsi mis de côté, la cible subit immédiatement une touche de la même Force et suivant les mêmes règles spéciales que l'attaque initiale. Les attaques de tir infligées ainsi comptent comme des attaques de tir à tous points de vue, et les attaques de corps à corps comptent comme des attaques de corps à corps, de la part du camp opposé, y compris en ce qui concerne le Résultat du Combat. Les unités constituées d'une seule figurine ne peuvent pas être affectées par ce sort.
\tabularnewline
\hline
6 & Portail Éternel &
15+ &
\portee{24} \newline Direct \newline Malédiction \newline Dégâts &
Immédiat &
La cible subit 2D6 touches de Force 2D6 avec la règle \emph{Flammes de l’Enfer}. Si la Force des attaques est de 11 ou 12, considérez la Force comme étant de 10.
\tabularnewline
\closetable








\newpage

\subsection{Discipline de la Forge}

\starttable[white]{\colors@forge}
A &
Fournaise Haineuse &
&
\portee{18} \newline Malédiction &
Dure un tour &
La cible gagne la règle \emph{Inflammable}. 
\tabularnewline
\hline
0 & Bouclier de Sombrefeu &
9+ &
\portee{12} \newline Amélioration &
Reste en jeu &
Les attaques de la cible gagnent les règles \emph{Attaques Magiques} et \emph{Attaques Enflammées}, et toutes les attaques dirigées contre elle souffrent d'un malus de -1 pour blesser.
\tabularnewline
\hline
1 & Rage Incendiaire &
\base{5+}\newline \amel{11+} &
\portee{12} \newline Malédiction \newline Projectile \newline Dégâts &
Immédiat &
La cible subit \base{1D6}/\amel{2D6} touches de Force 6 avec la règle \emph{Attaques Enflammées}.
\tabularnewline
\hline
2 & Sombre Envoûtement &
7+ &
\portee{24} \newline Malédiction &
Permanent &
La cible subit un malus de -1 à son Commandement.
\tabularnewline
\hline
3 & Souffle Malveillant &
7+ &
\portee{12} \newline Amélioration &
Dure un tour &
La cible peut relancer ses jets pour toucher au corps à corps.
\tabularnewline
\hline
4 & Anathème de Noirceur &
9+ &
\portee{18} \newline Focalisé \newline Malédiction \newline Direct \newline Dégâts &
Immédiat &
La cible subit un nombre de touches égal à 2D6 moins son Endurance. Ces touches blessent sur 4+ et suivent la règle \emph{Perforant (6)}.
\tabularnewline
\hline
5 & Tempête de Cendres &
\base{11+} \newline \amel{14+} &
\base{\portee{24}} \newline \amel{\portee{48}} \newline Malédiction &
Dure un tour &
La cible subit un malus pour toucher de -1 au corps à corps et -2 au tir. L'unité ne peut pas effectuer de Marche Forcée durant l'étape des Autres Mouvements. Si l'unité veut charger, elle doit diviser par deux (arrondi au supérieur) son jet de distance de charge. Tous les sorts de la cible voient leur portée réduite à un maximum de \distance{12}.
\tabularnewline
\hline
6 & Flammes de la Forge &
\base{14+} \newline \amel{17+} &
\portee{36} \newline Malédiction \newline Projectile \newline Dégâts &
Immédiat &
Placez un gabarit de \distance{3} avec le centre au dessus de la cible et à portée. Faites dévier le gabarit de \distance{1D6}. Toutes les figurines sous le gabarit subissent une touche de Force \base{5}/\amel{7} avec les règles \emph{Blessures Multiples (Artillerie)} et \emph{Attaques Enflammées}.
\tabularnewline
\closetable





\newpage

\subsection{Discipline des Grands Dieux Verts}

\starttable[white]{\colors@biggreengods}
A &
Chopez-les ! &
&
\portee{24} \newline Amélioration &
Dure un tour &
La cible peut relancer ses jets pour blesser ayant donné \result{1} en combat. 
\tabularnewline
\hline
0 &
C'est Parti ! &
10+ &
\portee{18} \newline Malédiction &
Dure un tour &
Toutes les unités peuvent relancer les jets pour toucher ratés contre cette unité au corps à corps.
\tabularnewline
\hline
1 & Casse-Crâne &
\base{6+} \newline \amel{13+} &
\base{\portee{24}} \newline \amel{\portee{18}} \newline \amel{Aura} \newline Focalisé \newline Direct \newline Malédiction \newline Dégâts &
Immédiat &
Affecte uniquement les Sorciers. La cible subit une touche de Force 5 avec la règle \emph{Perforant (6)}.
\tabularnewline
\hline
2 & Poings Cogneurs &
\base{6+} \newline \amel{11+} &
\base{Lanceur} \newline \amel{\portee{12}} \newline \amel{Focalisé} \newline \amel{Amélioration} \newline \amel{Personnage uniquement} &
Reste en jeu &
La cible gagne +3 en Force, +3 Attaques et la règle \emph{Attaques Magiques}. 
\tabularnewline
\hline
3 & Même Pas Mal! &
\base{8+} \newline \amel{11+} &
\base{\portee{12}}\newline \amel{\portee{24}} \newline Amélioration &
Dure un tour &
La cible gagne une \emph{Sauvegarde Invulnérable (5+)}.
\tabularnewline
\hline
4 & Grande Main Verte &
\base{11+} \newline \amel{14+} &
\portee{24} \newline Amélioration &
Dure un tour &
La cible peut faire un Pivot, en ignorant tout obstacle, puis effectuer un Mouvement Magique avec la règle \emph{Vol} de \base{\distance{3D6}}/\amel{\distance{5D6}}. Jetez les dés pour la distance avant de pivoter. A la fin de ce mouvement, l'unité peut encore faire un Pivot en ignorant les obstacles. La position finale de l'unité ne peut pas être à moins de \distance{1} d'une autre unité ou d'un Terrain Infranchissable. 
\tabularnewline
\hline
5 &
Boum! &
11+ &
\portee{18} \newline Direct \newline Malédiction \newline Dégâts &
Immédiat &
La cible subit 2D6 touches de Force 5.
\tabularnewline
\hline
6 & Écrabouille-Les! &
\base{13+} \newline \amel{16+} &
\portee{36} \newline Direct \newline Malédiction \newline Dégâts &
Immédiat &
Placez le centre d'un gabarit de \distance{3} sur l'unité ciblée et à portée du sort, puis faites le dévier de \distance{1D6}. Toutes les figurines sous le gabarit subissent chacune une touche de Force \base{6}/\amel{8} avec la règle \emph{Blessures Multiples (1D3)}.
\tabularnewline
\closetable






\newpage

\subsection{Discipline des Petits Dieux Verts}

Après le déploiement, tout personnage connaissant un ou plusieurs sorts de cette Discipline doit choisir un sort (qui n'est pas un objet de sort) connu par n'importe quel sorcier sur la table, et utiliser l'Attribut de Discipline de ce sort pour ses propres lancements.

\starttable{\colors@littlegreengods}
A &
Vol Fourbe &
&
Spécial &
Spécial &
Utilisez l'Attribut de Discipline du sort choisi en début de partie. Remplacez toutes les références à la Discipline de l'Attribut par la Discipline des Petits Dieux Verts.
\tabularnewline
\hline
0 & Œil Mauvais &
\base{4+}\newline \amel{9+} &
\portee{24} \newline Malédiction \newline Projectile \newline Dégâts &
Immédiat &
La cible subit \base{2D6}/\amel{3D6} touches de Force 3.
\tabularnewline
\hline
1 &
Taillades Sournoises &
\base{5+}\newline \amel{8+} &
\base{\portee{12}} \newline \amel{\portee{24}} \newline Amélioration &
Dure un tour &
La cible gagne la règle \emph{Perforant (1)} pour ses Attaques de tir et de corps à corps. Si la cible attaque une unité sur son flanc ou son arrière, la cible peut relancer ses jets pour toucher et pour blesser ratés.
\tabularnewline
\hline
2 & Louée Soit la Grande \newline Araignée! &
\base{6+}\newline \amel{9+} &
\base{\portee{12}} \newline \amel{\portee{24}} \newline Amélioration &
Dure un tour &
La cible gagne la règle \emph{Attaques Empoisonnées}. Si elle en disposait déjà, alors ses \emph{Attaques Empoisonnées} blessent automatiquement sur 5+.
\tabularnewline
\hline
3 & Ça Démange ? &
6+ &
\portee{24} \newline Malédiction &
Dure un tour &
La cible subit un malus de -1D3 à ses caractéristiques de Mouvement et d'Initiative, jusqu'à un minimum de 1. 
\tabularnewline
\hline
4 &
Chut! Pas un Bruit... &
\base{7+} \newline \amel{10+} &
\base{Unité du lanceur} \newline \amel{\portee{6}} \newline \amel{Aura} \newline Amélioration &
Dure un tour &
La cible gagne les règles \emph{Camouflé} et \emph{Distrayant}. Toute figurine chargeant avec succès l'unité ciblée doit effectuer un test de Terrain Dangereux (1).
\tabularnewline
\hline
5 &
On s'en Occupe &
9+ &
\portee{24} \newline Malédiction &
Dure un tour &
La cible doit relancer ses jets pour toucher, pour blesser et de Sauvegardes de tous types ayant donné \result{6}.
\tabularnewline
\hline
6 & Malédiction de la Lune Verte &
13+ &
Vortex \newline (\portee{4}, Gabarit \distance{5}) \newline Marqueur &
Immédiat \newline Dure un tour &
Toutes les figurines touchées par le gabarit subissent chacune une touche de Force 3 avec la règle \emph{Perforant (6)}. Toute unité touchée par le gabarit subit un malus de -1 en Capacité de Combat.
\tabularnewline
\closetable






\newpage

\subsection{Discipline de la Luxure}

\starttable{\colors@lust}
A &
Extase de la Souffrance &
&
\portee{12} \newline Amélioration \newline Focalisé &
Dure un tour &
La cible gagne +3 en Initiative et +1 Attaque, ou \emph{Perforant (+1)}.
\newline Le choix doit être fait au moment de lancer le sort. Chaque figurine ne peut être affectée par chacun des deux effets qu'une fois par Phase de Magie.
\tabularnewline
\hline
0 &
Flagellation &
6+ &
\portee{24} \newline Marqueur \newline Direct \newline Gabarit de Ligne &
Immédiat &
Toutes les figurines sous le gabarit subissent une touche de Force 4 avec la règle \emph{Perforant (2)}.
\tabularnewline
\hline
1 &
Grâce Hypnotique &
\base{5+} \newline \amel{11+} &
\base{\portee{24}} \newline \amel{\portee{36}} \newline Malédiction &
Dure un tour &
La cible porte ses attaques avec une Initiative réduite à 0 \amel{et suit la règle \emph{Mouvement Aléatoire (1D6)}}. 
\tabularnewline
\hline
2 & Valse Irrésistible &
\base{7+}\newline \amel{11+} &
\base{\portee{12}} \newline \amel{\portee{24}} \newline Focalisé \newline Malédiction \newline Direct \newline Dégâts &
Immédiat &
La cible doit effectuer un test de Commandement en jetant 1D6 supplémentaire. Si le test est raté, la cible subit une blessure avec la règle \emph{Perforant (6)}, et doit continuer à passer des tests de Commandement de la même façon, avec les mêmes effets, jusqu'à ce qu'elle réussisse ou succombe.
\tabularnewline
\hline
3 & Hystérie &
8+ &
\portee{24} \newline Universel &
Reste en jeu &
La cible gagne la règle \emph{Frénésie}. Si elle l'avait déjà, elle gagne encore une attaque supplémentaire, jusqu'à ce qu'elle perde la \emph{Frénésie}. À la fin de chaque Phase de Magie du lanceur, la cible subit 1D6 touches de Force 3.
\tabularnewline
\hline
4 & Fantasmagorie &
\base{8+}\newline \amel{12+} &
\base{\portee{24}} \newline \amel{\portee{12}} \newline \amel{Aura} \newline Malédiction &
Dure un tour &
À chaque fois que la cible effectue un test de Commandement, elle lance un 1D6 supplémentaire et ignore le dé avec le résultat le plus bas.
\tabularnewline
\hline
5 & Échardes Psychiques &
9+ &
\portee{24} \newline Malédiction \newline Projectile \newline Dégâts &
Immédiat &
La cible subit 1D6 touches de Force 4 avec la règle \emph{Perforant (1)}. Une fois ces touches résolues, la cible doit effectuer un test de Commandement sans aucune relance autorisée. En cas d'échec, elle subit à nouveau 1D6 touches, et doit continuer à passer des tests de Commandement de la même façon, avec les mêmes effets, jusqu'à ce qu'elle réussisse ou succombe.
\tabularnewline
\hline
6 & Chœur Dissonant &
13+ &
\portee{12} \newline Malédiction &
Immédiat \newline Dure un tour &
La cible subit 2D6 touches blessant sur 4+ avec la règle \emph{Perforant (6)}.
\newline Si au moins une blessure est infligée ainsi, la cible porte ses attaques avec une Initiative réduite à 0 et suit la règle \emph{Mouvement Aléatoire (1D6)}.
\tabularnewline
\closetable






\newpage

\subsection{Discipline de la Magie Blanche}

\starttable{\colors@whitemagic}
A &
Bouclier des Anciens &
&
\portee{18} \newline Amélioration &
Permanent &
Placez un marqueur \emph{Bouclier} sur la cible, qui ne peut pas être une \emph{Grande Cible}.
\newline Ignorez la prochaine blessure non sauvegardée (après avoir calculé les éventuelles \emph{Blessures Multiples}), et retirez le marqueur \emph{Bouclier}. Une unité ne peut avoir qu'un seul marqueur \emph{Bouclier} à la fois.
\tabularnewline
\hline
0 & Traits de Lumière &
\base{9+} \newline \amel{14+} &
\portee{24} \newline Malédiction \newline Projectile \newline Dégâts &
Immédiat &
La cible subit \base{2D6}/\amel{3D6} touches de Force 4 \amel{avec la règle \emph{Attaques Divines}}.
\tabularnewline
\hline
1 & Vol du Phœnix &
\base{4+}\newline \amel{8+} &
\portee{18} \newline Amélioration \newline Focalisé &
Immédiat \newline \amel{Dure un tour}  &
La cible Récupère 1 PV.
\newline \amel{Son unité et elle gagnent également +1 en Force}.
\tabularnewline
\hline
2 & Main du Destin &
\base{6+} \newline \amel{9+} &
\portee{18} \newline Amélioration &
Dure un tour &
\base{La cible gagne 1D3 à une des caractéristiques parmi M, CC, CT ou I. Le lanceur doit préciser laquelle avant de lancer le sort}.
\newline \amel{La cible gagne 1D3 aux 4 caractéristiques M, CC, CT et I (un seul jet de dé)}.
\tabularnewline
\hline
3 & Sentiers Secrets &
\base{7+} \newline \amel{15+} &
\portee{24} \newline Amélioration &
Spécial &
La cible gagne la règle spéciale \emph{Éthéré} jusqu'à la fin de la Phase de Magie et peut faire un Mouvement Magique de \base{\distance{8}}/\amel{\distance{16}}.
\tabularnewline
\hline
4 & Bénédiction &
\base{9+} \newline \amel{12+} &
\base{\portee{12}} \newline \amel{\portee{24}} \newline Amélioration &
Dure un tour &
La cible gagne une \emph{Sauvegarde Invulnérable (5+)}. Les figurines affectées et ayant déjà une \emph{Sauvegarde Invulnérable} la voient augmenter de 1, jusqu'à un maximum de 3+.
\tabularnewline
\hline
5 & Fusion d'Artefact &
11+ &
\portee{24} \newline Focalisé \newline Direct \newline Malédiction \newline Dégâts &
Immédiat &
La cible subit une touche avec la règle \emph{Fusion du Métal}. Si la cible porte un ou plusieurs objets magiques, l'un d'entre eux, au hasard, est détruit. Les objets à \emph{Usage Unique} ne sont jamais pris en compte.
\tabularnewline
\hline
6 & Cataclysme &
\base{14+} \newline \amel{18+} &
\portee{24} \newline Direct \newline Malédiction \newline Dégâts &
Reste en jeu &
A la fin de chaque Phase de Magie, chaque figurine de l'unité ciblée subit une touche de Force \base{3}/\amel{4} avec la règle \emph{Attaques Enflammées}.
\tabularnewline
\closetable






\newpage

\subsection{Discipline de la Magie Noire}

\starttable[white]{\colors@blackmagic}
A &
Soif d'Âmes &
&
\portee{18} \newline Malédiction \newline Projectile \newline Dégâts &
Immédiat &
La cible subit une touche de Force 5.
Si cela cause une blessure non sauvegardée, le lanceur peut Récupérer 1 PV. Aucune figurine ne peut gagner plus d'un PV par Phase de Magie avec ce sort.
\tabularnewline
\hline
0 & Puissance du Vide &
\base{8+} \newline \amel{10+} &
\base{\portee{6}} \newline \amel{\portee{18}} \newline Amélioration &
Dure un tour &
La cible gagne +1 en Force \amel{et \emph{Perforant (1)}}.
\tabularnewline
\hline
1 & Souffle Glacial &
\base{4+} \newline \amel{9+} &
\base{\portee{24}} \newline \amel{\portee{36}} \newline Malédiction \newline Direct \newline Dégâts &
Immédiat \newline \amel{ou Dure un tour} &
La cible subit 1D6 touches de Force 4.
\newline \amel{La cible subit un malus de -1D3 en CT, jusqu'à un minimum de 1.}
\tabularnewline
\hline
2 & Tourbillon de Lames &
\base{7+} \newline \amel{9+} &
\portee{24} \newline Malédiction &
Dure un tour &
La cible subit un malus de -1D3 en CC, jusqu'à un minimum de 1.
\newline \amel{La cible ne peut pas utiliser les règles \emph{Distrayant} et \emph{Parade}.}
\tabularnewline
\hline
3 & Râle d'Agonie &
\base{8+} \newline \amel{14+} &
\base{\portee{18}} \newline \amel{\portee{36}} \newline Malédiction &
Dure un tour &
La cible subit un malus de \base{-1}/\amel{-2} en Force et en Initiative, jusqu'à un minimum de 1.
\tabularnewline
\hline
4 & Linceul Funeste &
\base{8+} \newline \amel{13+} &
\base{\portee{18}} \newline \amel{\portee{36}} \newline Malédiction &
Reste en jeu &
La cible ne peut pas utiliser les règles \emph{Tenez les Rangs} et \emph{Présence Charismatique}.
\tabularnewline
\hline
5 & Trait d'Énergie Sombre &
11+ &
\portee{18} \newline Malédiction \newline Projectile \newline Dégâts &
Immédiat &
La cible subit 2D6 touches de Force 5.
\tabularnewline
\hline
6 & Terreur Noire &
12+ &
Vortex \newline (\portee{6}, Gabarit \distance{1}) \newline Marqueur &
Immédiat &
Les figurines touchées par le gabarit doivent effectuer un test de Force. Toute figurine ratant son test subit une blessure avec les règles \emph{Blessures Multiples (Artillerie)} et \emph{Perforant (6)}, et qui n'autorise pas les Sauvegardes de \emph{Régénération}.
\tabularnewline
\closetable





\newpage

\subsection{Discipline de la Maladie}

\starttable[white]{\colors@disease}
A &
Bubons Joyeux &
&
\portee{12} \newline Amélioration  \newline Focalisé &
Dure un tour &
La cible bénéficie d'un bonus de +1 en Endurance. Une figurine ne peut être affectée par ce sort qu'une fois par Phase de Magie.
\tabularnewline
\hline
0 & Miasmes de Pourriture &
\base{4+} \newline \amel{9+} &
\portee{18} \newline Amélioration &
Dure un tour &
Les unités ennemies en contact socle à socle avec la cible ont un malus de -\base{1}/\amel{1D3} en Capacité de Combat et en Initiative, jusqu'à un minimum de 1.
\tabularnewline
\hline
1 & Haleine Putride &
\base{6+} \newline \amel{9+} &
\base{Lanceur} \newline \amel{\portee{24}} \newline \amel{Focalisé} \newline \amel{Amélioration} &
Dure un tour &
La cible gagne une \emph{Attaque de Souffle} avec la règle \emph{Attaques Toxiques}.
\tabularnewline
\hline
2 & Bénédiction Infecte &
\base{6+} \newline \amel{9+} &
\base{\portee{12}} \newline \amel{\portee{24}} \newline Amélioration &
Dure un tour &
La cible gagne la règle \emph{Attaques Empoisonnées}. Si elle en disposait déjà, le jet pour toucher requis pour que les \emph{Attaques Empoisonnées} blessent automatiquement est diminué de 1 (ainsi, la cible subit une blessure sur 5+ au lieu de 6+, ou 4+ au lieu de 5+, etc.).
\tabularnewline
\hline
3 & Excroissance Adipeuse &
\base{9+} \newline \amel{12+} &
\base{\portee{12}} \newline \amel{\portee{24}} Amélioration &
Dure un tour &
La cible gagne la règle \emph{Régénération (5+)}. Si elle possédait déjà cette règle, sa \emph{Régénération} est améliorée de 1, jusqu'à un maximum de \emph{Régénération (3+)}.
\tabularnewline
\hline
4 & Parasitage Purificateur &
10+ &
\portee{18} \newline Malédiction \newline Projectile \newline Dégâts &
Immédiat &
La cible subit 1D6 touches de Force 5. De plus, l'unité ciblée doit faire un test d'Endurance. S'il est raté, la cible subit à nouveau 1D6 touches de Force 5. Continuez ainsi jusqu'à ce que le test réussisse ou que l'unité soit détruite. 
\tabularnewline
\hline
5 & Atours du Lépreux &
\base{11+} \newline \amel{14+} &
\base{\portee{18}} \newline \amel{\portee{36}} \newline Universel &
Dure un tour &
L'Endurance de la cible est, au choix, augmentée ou diminuée de 1D3, jusqu'à un minimum de 1. Le choix doit être fait au moment du choix de la cible.
\tabularnewline
\hline
6 & Voile Fétide &
14+ &
Vortex \newline (\portee{6}, Gabarit \distance{3}) \newline Marqueur &
Immédiat &
Les figurines touchées par le gabarit subissent chacune une touche avec les règles \emph{Attaques Toxiques} et \emph{Blessures Multiples (1D3)}.
\tabularnewline
\closetable






\newpage

\subsection{Discipline de la Nécromancie}

\starttable[white]{\colors@necromancy}
A &
Tromper la Faucheuse &
&
\portee{12} \newline Focalisé \newline Amélioration &
Immédiat &
L'unité ciblée, ou le Personnage ciblé, \emph{Ressuscite} 1 PV. Aucune unité ne peut être ciblée plus de deux fois par Phase de Magie. 
\tabularnewline
\hline
0 & Adjuration des Morts &
\base{5+} \newline \amel{5+} \newline \amelbis{11+} &
\base{\portee{18}} \newline \amel{\portee{6}} \newline \amel{Aura} \newline \amelbis{\portee{12}} \newline \amelbis{Aura} \newline Amélioration &
Immédiat &
La cible \emph{Ressuscite} des PVs en fonction de la valeur de sa règle Invocation. Une unité ne disposant pas de cette règle ne peut pas être ciblée par ce sort.
\tabularnewline
\hline
1 & Parodie de Vie &
\base{6+} \newline \amel{10+} &
\portee{12} \newline \amel{Aura} \newline Amélioration &
Dure un tour &
La cible peut relancer ses jets pour blesser ratés et gagne la règle \emph{Peur}.
\tabularnewline
\hline
2 & Sarabande Macabre &
\base{8+} \newline \amel{12+} &
\portee{12} \newline \amel{Aura} \newline Amélioration &
Immédiat \newline Dure un tour &
La cible peut immédiatement effectuer un Mouvement Magique de \distance{8} \base{et}/\amel{ou} relancer ses jets pour toucher ratés. \amel{Le choix doit être fait pour chaque cible au moment de lancer le sort.}
\tabularnewline
\hline
3 & Regard Sépulcral de Setesh &
\base{9+} \newline \amel{11+} &
\base{\portee{24}} \newline \amel{\portee{48}} \newline Malédiction \newline Projectile \newline Dégâts &
Immédiat &
La cible subit 2D6 touches de Force 4.
\tabularnewline
\hline
4 & Conscription \newline Profanatoire &
\base{7+} \newline \amel{9+} &
\base{\portee{12}} \newline \amel{\portee{24}} \newline Marqueur &
Immédiat &
Placez une unité invoquée de votre choix parmi celles indiquées dans la règle \emph{Sombre Réveil} du lanceur. Choisissez cette unité avant de lancer le sort. L'unité a un nombre de Points de Vie égal à sa règle d'Invocation, et doit être placée entièrement à portée du sort (au moins l'une des figurines invoquées devant être positionnée sur le marqueur).
\newline Cette unité peut recevoir toutes les améliorations qui lui sont d'ordinaire accessibles, à l'exception de l'État-Major.
\tabularnewline
\hline
5 & Sénescence &
10+ &
\portee{18} \newline Malédiction \newline Direct &
Reste en jeu &
À la fin de chaque Phase de Magie, chaque figurine de l'unité ciblée subit une touche qui blesse sur 6+. Ces blessures ont les règles \emph{Perforant (6)} et \emph{Blessures Multiples (2, Bête Monstrueuse, Cavalerie Monstrueuse, Char, Infanterie Monstrueuse,  Machine de Guerre, Monstre, Monstre Monté, Nuée)}.
\newline Ces touches ont un bonus de +1 pour blesser pour chaque tour de joueur qui s'est terminé depuis que le sort a été lancé.
\tabularnewline
\hline
6 & Vent de l'Au-Delà &
12+ &
Amélioration \newline \portee{18} &
Dure un tour &
L'unité ciblée gagne +1 en Force et \emph{Regénération (5+)}. Les figurines disposant déjà d'une \emph{Regénération} la voient améliorée d'un point, jusqu'à un maximum de \emph{Regénération (4+)}.
\tabularnewline
\closetable





\newpage

\subsection{Discipline de la Ruine}

\starttable[white]{\colors@ruin}
A &
Masse Critique &
 &
\portee{24} \newline Amélioration &
Dure un tour &
La cible peut compter jusqu'à quatre points de Bonus de Rang pour le Résultat de Combat et gagne la règle \emph{Combat avec un rang supplémentaire}.
\tabularnewline
\hline
0 & Éclair Irradiant &
\base{5+} \newline \amel{8+} &
\base{\portee{24}} \newline \amel{\portee{48}} \newline Malédiction \newline Projectile \newline Dégâts
&
Immédiat &
La cible subit 1D6 touches de Force 5 avec la règle \emph{Attaques Foudroyantes}. Si le nombre de touches obtenu est 1, c'est le lanceur qui subit la touche à la place de la cible.
\tabularnewline
\hline
1 & Galeries Contaminées &
\base{6+} \newline \amel{8+} &
\base{\portee{18}} \newline \amel{\portee{30}} \newline Marqueur &
Spécial &
Dure jusqu'au début du prochain tour du lanceur. Placez le gabarit de \distance{5} centré sur le marqueur et à minimum \distance{1} de toute unité. Si une unité touche le gabarit lors d'un déplacement sans \emph{Vol}, toutes ses figurines traitent n'importe quel terrain (y compris les Terrains Découverts) comme Terrain Dangereux (2).
\tabularnewline
\hline
2 & Faim Destructrice &
\base{8+} \newline \amel{11+} &
\base{\portee{12}} \newline \amel{\portee{24}} \newline Amélioration &
Permanent &
La cible gagne la \emph{Frénésie} et +1 Attaque. A la fin de chacune de vos phases de Corps à Corps, elle subit 1D6 touches de Force 4 avec la règle \emph{Perforant (6)}. Tous les effets de ce sort prennent fin lorsque la cible perd la \emph{Frénésie}.
\tabularnewline
\hline
3 & Vent Hurlant &
9+ &
Spécial &
Dure un tour &
Impossible d'effectuer de \emph{Vol} sur tout le champ de bataille. Toutes les unités ennemies subissent un malus de -1 en CT.
\tabularnewline
\hline
4 & Faille Sismique &
\base{9+} \newline \amel{12+} &
\portee{18} \newline Marqueur \newline Direct \newline Gabarit de ligne &
Immédiat &
Toutes les figurines touchées subissent une touche de Force 6 avec les règles \emph{Perforant (6)} \amel{et \emph{Blessures Multiples (1D3)}}.
\tabularnewline
\hline
5 & Nourrissons-les... &
\base{9+} \newline \amel{15+} &
\portee{12} \newline Amélioration &
Dure un tour &
L'unité ciblée peut faire des Attaques de Soutien avec tous ses rangs. \amel{Elle peut aussi relancer ses jets pour toucher ratés}.
\tabularnewline
\hline
6 & Pour Qui Sonne le Glas &
\base{12+} \newline \amel{14+} &
\portee{18} \newline Direct \newline Dégâts &
Immédiat &
Toutes les figurines de l'unité subissent une touche avec les règles \emph{Attaques Toxiques} \amel{et \emph{Attaques Enflammées}}. Ce sort peut être jeté sur une unité engagée en corps à corps, auquel cas toutes les figurines engagées dans le combat subissent une touche avec les mêmes règles.
\tabularnewline
\closetable







\newpage

\subsection{Discipline des Sables}

\starttable{\colors@sand}
A &
Les Morts sans Repos &
&
Amélioration &
Immédiat &
Cet Attribut cible toujours la ou les mêmes unités que le sort qui l'a déclenché, ou une seule unité dans un rayon de \distance{6} du lanceur. \newline
Chaque cible Ressuscite un nombre de PVs égal à sa règle \emph{Invocation}. En cas d'Unité Combinée, déclarez qui de l'unité ou du Personnage Récupérera des PVs.\newline
Les personnages et les \emph{Grandes Cibles} ne peuvent pas Récupérer plus de 2 PVs par phase de magie.
\tabularnewline
\hline
0 & Sirocco &
\base{5+} \newline \amel{11+} &
\base{\portee{18}} \newline \amel{\portee{12}} \newline \amel{Aura} \newline Amélioration &
Immédiat &
La cible peut effectuer un \emph{Mouvement Magique} de \distance{X}, X étant égal à la valeur de Mouvement de la cible.
\tabularnewline
\hline
1 & Lames Maudites &
\base{5+} \newline \amel{10+} &
\base{\portee{18}} \newline \amel{\portee{12}} \newline \amel{Aura} \newline Amélioration &
Dure un tour &
La cible gagne la règle spéciale \emph{Coup Fatal}. Si elle en disposait déjà, elle peut relancer ses jets pour blesser ratés au corps à corps.
\tabularnewline
\hline
2 & Dessication &
\base{7+} \newline \amel{9+} &
\portee{24} \newline Malédiction &
Dure un tour &
La cible subit un malus de -1 en Endurance \amel{et en Force}, jusqu'à un minimum de 1.
\tabularnewline
\hline
3 & Frappes Vengeresses &
\base{7+} \newline \amel{12+} &
\base{\portee{18}} \newline \amel{\portee{12}} \newline \amel{Aura} \newline Amélioration &
Dure un tour &
La cible gagne +1 Attaque, ou Tous les Arcs des dépouilles, Grands arcs des dépouilles et les Arcs géants des dépouilles de l'unité ciblée gagnent la règle \emph{Tirs Multiples (2)}.
\tabularnewline
\hline
4 & Jugement Divin &
\base{8+} \newline \amel{10+} &
\portee{36} \newline Malédiction \newline Dégâts &
Immédiat &
\amel{Ciblez une unité, puis jetez 1D6. Sur 3+, choisissez une nouvelle cible située à moins de \distance{6} de la première. Continuez ainsi, de cible en cible à \distance{6} d'écart, jusqu'à ce que vous obteniez un \result{1} ou un \result{2}, ou qu'aucune cible ne soit éligible. Aucune unité ne peut être ciblée deux fois.}
\newline Chaque cible doit faire un test de Commandement avec un D6 additionnel. Si ce test est raté, elle subit une blessure par point d'écart entre le résultat et son Commandement. Ces blessures ont la règle spéciale \emph{Perforant (6)}.
\tabularnewline
\hline
5 & Sables Mouvants &
\base{9+} \newline \amel{12+} &
\base{\portee{24}} \newline \amel{\portee{48}} \newline Malédiction &
Dure un tour &
La cible subit un malus de -1D3 en Mouvement, jusqu'à un minimum de 1. Elle traite tous les terrains comme des Terrains Dangereux (2), y compris les Terrains Découverts.
\tabularnewline
\hline
6 & Âge d'Or &
\base{10+} \newline \amel{15+} \newline \amelbis{17+} &
\base{\portee{18}} \newline \amel{\portee{9}} \newline \amel{Aura} \newline \amelbis{\portee{15}} \newline \amelbis{Aura} \newline Amélioration &
\base{Dure un tour} \newline \amel{Reste en jeu} \newline \amelbis{Reste en jeu} &
La cible gagne un bonus de +1 en Capacité de Combat, Force et Initiative.
\tabularnewline
\closetable




\newpage

\section{Résumé de la Phase de Magie}

\renewcommand{\arraystretch}{1.5}

\begin{minipage}[t]{.47\linewidth}
\fontsize{10}{10}\selectfont

\subsection*{Séquence de la Phase de Magie}

\begin{tabular}{c|m{6.8cm}}
\textbf{1} & Début de la \emph{Phase de Magie}. Lancez les dés pour déterminer les Flux Magiques et pour les Canalisations. \tabularnewline
\textbf{2} & Les deux joueurs peuvent tenter de dissiper les sorts \emph{Reste en Jeu} lancés lors des phases de magie précédentes. \tabularnewline
\textbf{3} & Le joueur actif peut tenter de lancer ses sorts. \tabularnewline
\textbf{4} & Répétez les étapes 2 et 3 jusqu'à ce qu'aucun joueur ne fasse une action. \tabularnewline
\textbf{5} & Fin de la \emph{Phase de Magie}. Les capacités prenant effet à la fin de la \emph{Phase de Magie} sont déclenchées. \tabularnewline
\end{tabular}

\subsection*{Lancement d'un Sort}

\begin{tabular}{c|m{6.8cm}}
\textbf{1} & Le joueur actif indique quel \emph{Sorcier} tente de lancer quel sort. Il doit préciser s'il opte pour une version améliorée du sort, ainsi que la cible du sort et de celle de l'attribut si nécessaire. Il indique aussi le nombre de Dés de Pouvoir utilisés (entre 1 et 5). \tabularnewline
\textbf{2} & Le joueur actif lance le nombre de Dés de Pouvoir annoncé, en les retirant de sa réserve. Additionnez les résultats des dés avec les modificateurs de lancer (tels qu'un \emph{Pouvoir Irrésistible}). \tabularnewline
\textbf{3} & Le sort est lancé avec succès si le total de lancer est supérieur ou égal à la valeur de lancement. Sinon, le lancement de sort échoue et le lanceur subit l'effet \emph{Perte de Concentration}. \tabularnewline
\end{tabular}

\subsection*{Dissipation d'un Sort}

\begin{tabular}{c|m{6.8cm}}
\textbf{1} & Le joueur réactif peut tenter de dissiper le sort. Dans ce cas, il doit indiquer lequel de ses \emph{Sorciers} n'étant pas en fuite (s'il en a) va tenter la dissipation, et annoncer combien de Dés de Dissipation il va utiliser (un minimum d' un dé est nécessaire, et la totalité de réserve peut être utilisée). Il est possible de tenter une dissipation même sans avoir de \emph{Sorcier}. \tabularnewline
\textbf{2} & Le joueur réactif lance le nombre de Dés de Dissipation annoncé, en les retirant de sa réserve. Additionnez les résultats des dés avec les modificateurs de dissipation (tels qu'un \emph{Pouvoir irrésistible}) pour obtenir le total de dissipation. \tabularnewline
\textbf{3} & Si le résultat est supérieur ou égal au total de lancement, le sort est dissipé, et le lancement du sort a échoué. Sinon, le \emph{Sorcier} ayant tenté la dissipation subit l'effet \emph{Perte de Concentration}. \tabularnewline
\end{tabular}

\end{minipage}
\hfill
\begin{minipage}[t]{.5\linewidth}
\fontsize{10}{10}\selectfont

\subsection*{Table des Fiascos}

\begin{tabular}{cm{6.8cm}}
\hline
\textbf{2 à 4} & Centrez le gabarit de \unit{5}{\pouce} sur le lanceur. Toute figurine recouverte par le gabarit, même partiellement, subit une touche. De plus, si NDP = \textbf{5}, retirez le lanceur de la partie. Si NDP = 4, lancez un D6. Sur un résultat de 1 à 3, le lanceur est retiré de la partie. \tabularnewline
\textbf{5 à 6} & Centrez le gabarit de \unit{3}{\pouce} sur le lanceur. Toute figurine recouverte par le gabarit subit une touche. Le lanceur doit subir une touche. \tabularnewline
\textbf{7} & L'unité du lanceur subit NDP touches, réparties comme des tirs. Le lanceur ne peut cependant subir qu'une seule touche au maximum. \tabularnewline
\textbf{8 à 9} & Le lanceur et tout \emph{Sorcier} ami subissent une touche. \tabularnewline
\textbf{10 à 12} & Le niveau de magie du lanceur est diminué de NDP - 2. Il perd un sort pour chaque niveau de magie perdu, en commençant par le sort ayant causé le \emph{Fiasco} et en tirant les autres au hasard. \tabularnewline
\hline
\end{tabular}

\bigskip
Quelque-soit le résultat du jet, retirez NDP dés de votre réserve.
\bigskip

\begin{tabular}{|m{8cm}|}
\hline
\vspace*{-0.4cm}
\subsection*{Modificateurs magiques}
\noindent Sorcier de niveau 1 ou 2 : Apprenti, +1
\newline Sorcier de niveau 3 ou 4 : Maître, +2
\newline La somme des modificateurs ne peut pas dépasser +3, sauf en cas de \emph{Pouvoir Irrésistible}. \tabularnewline
\hline
\end{tabular}

\bigskip

\begin{tabular}{|m{8cm}|}
\hline
\vspace*{-0.4cm}
\subsection*{Objets de Sorts}
\noindent Pour lancer un Objet de Sort avec succès, le jet de lancement doit être supérieur ou égal à son Niveau de Puissance.
\smallskip
\begin{itemize}
\item Aucun modificateur positif ne peut être ajouté au jet de lancement.
\item Un échec de provoque pas de \emph{Perte de Concentration} pour le lanceur.
\item Un objet de sort ne bénéficie pas du bonus de lancement d'un \emph{Pouvoir Irrésistible}.
\item L'attribut de la Discipline est lancé comme d'ordinaire.
\end{itemize}
\medskip En cas de \emph{Pouvoir Irrésistible} :
\smallskip
\begin{itemize}
\item Si 4 dés ou plus ont été utilisés pour lancer le sort, l'objet de sort est perdu et ne peut plus être lancée pendant la partie.
\item Retirez NDP Dés de Pouvoir de votre réserve.
\end{itemize}
\tabularnewline
\hline
\end{tabular}

\end{minipage}


\end{document}

