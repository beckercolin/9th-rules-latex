\documentclass[a4paper,8pt]{extarticle} % extarticle allows to use font size of 8pt.

\usepackage[a4paper, top=1.6cm, bottom=2cm, left=1.6cm, right=1.6cm]{geometry} % Marge reduction.

%% Language specific package
\usepackage[french]{babel}
\frenchbsetup{StandardLists=true} % Necessary to use enumitem with babel/french.

%% Font and typing packages
\usepackage{fontspec}
\setmainfont[
	Ligatures=TeX,
	ItalicFont={Dancing Script},
	BoldItalicFont={Dancing Script}
	]{PT Serif} % default is Latin Modern
\newfontfamily\antiquefont[Ligatures=TeX]{Caslon Antique} % fancy font
\usepackage{microtype}			% Greatly improves general appearance of the text.
\usepackage{SIunits}			% Unit appearance.
\usepackage{ulem}				% To cross words out. Use \sout{}.

%% Array utilities
\usepackage{array}				% Additionnal options for arrays.
\usepackage{colortbl}			% Additionnal options for coloring arrays.
\usepackage{multirow}
\usepackage[table]{xcolor}		% Auto alternate grey-white rows.

%% List utilities
\usepackage[inline]{enumitem}   % Display inline lists.
\usepackage{etoolbox}           % General utility. Good for lists for instance.
\usepackage{xparse}             % List utilities.

%% Page utilities
\usepackage{multicol}			% Allows to divide a part of the page in multiple columns.
\usepackage{fancyhdr}		% For custom headers and foot texts
\pagestyle{fancy}
	
%% Others
\usepackage{xstring}            % String parsing, cutting, etc.
\usepackage[hidelinks, bookmarks=false, pdfdisplaydoctitle=true, pdfstartview=FitH, pdfpagelabels=false]{hyperref} % Links in PDF.

\makeatletter

%%% Language specific stuff


%%% Commands %%%

\newcommand{\addtosortedlist}[1]{%
	\protected@edef\textarg{#1}%
	\protected@edef\textwithoutspaces{\expandafter\removespaces\expandafter{\textarg}}%
	\substitute\textwithoutspaces{É}{e}% Most used special characters of the language, and equivalent for alphabetical ordering
	\substitute\textwithoutspaces{È}{e}%
	\substitute\textwithoutspaces{Ê}{e}%
	\substitute\textwithoutspaces{é}{e}%
	\substitute\textwithoutspaces{è}{e}%
	\substitute\textwithoutspaces{ê}{e}%
	\substitute\textwithoutspaces{À}{a}%
	\substitute\textwithoutspaces{à}{a}%
	\substitute\textwithoutspaces{ù}{u}%
	\expandafter\sortitem\expandafter[\textwithoutspaces]{#1}%
}%


%%% Labels %%%

% Profile

\newcommand{\labels@M}{M}
\newcommand{\labels@WS}{CC}
\newcommand{\labels@BS}{CT}
\newcommand{\labels@S}{F}
\newcommand{\labels@T}{E}
\newcommand{\labels@W}{PV}
\newcommand{\labels@I}{I}
\newcommand{\labels@A}{A}
\newcommand{\labels@Ld}{Cd}
\newcommand{\labels@Invocation}{Invocation} % For Vampire Covenant profiles

\newcommand{\Strength}{Force}

% Technical

\newcommand{\labels@range}{Portée}
\newcommand{\labels@point}{pt}
\newcommand{\labels@points}{pts}
\newcommand{\labels@only}{uniquement}
\newcommand{\labels@magic}{Magie}
\newcommand{\labels@pathsused}{Génère ses sorts dans la Discipline}
\newcommand{\labels@model}{figurine}
\newcommand{\labels@models}{figurines}
\newcommand{\labels@Singlemodel}{Figurine \textbf{seule}}

% Unit entry labels

\newcommand{\labels@basesize}{Socle}
\newcommand{\labels@trooptype}{Type de troupe}
\newcommand{\labels@specialrules}{Règles spéciales}
\newcommand{\labels@alignment}{Allégeance}
\newcommand{\labels@equipment}{Équipement}
\newcommand{\labels@weapons}{Armes}
\newcommand{\labels@armour}{Armure}
\newcommand{\labels@options}{Options}
\newcommand{\labels@commandgroup}{État-Major}
\newcommand{\labels@mounts}{Montures}
\newcommand{\labels@specialequipment}{Équipement spécial}

% Command groups

\newcommand{\labels@champion}{Champion}
\newcommand{\labels@standardbearer}{Porte-étendard}
\newcommand{\labels@musician}{Musicien}
\newcommand{\labels@singlebannerallowance}{Une seule unité de ce type peut prendre une Bannière magique}
\newcommand{\labels@condsinglebannerallowance}{Une seule unité de ce type peut prendre une Bannière magique si}
\newcommand{\labels@bannerallowance}{Peut prendre une Bannière Magique}
\newcommand{\labels@veteranstandardbearer}{Peut devenir Porte-étendard Vétéran}
\newcommand{\labels@championallowance}{Peut prendre une Arme Magique}

% Titles

\newcommand{\labels@lords}{Seigneurs}
\newcommand{\labels@heroes}{Héros}
\newcommand{\labels@coreunits}{Unités de base}
\newcommand{\labels@specialunits}{Unités spéciales}
\newcommand{\labels@rareunits}{Unités rares}
\newcommand{\labels@armywiderules}{Règles communes de l'armée}
\newcommand{\labels@armyspecialrules}{Règles spéciales de l'armée}
\newcommand{\labels@armoury}{Armurerie}
\newcommand{\labels@magicalitems}{Objets magiques}
\newcommand{\labels@magicalweapons}{Armes magiques}
\newcommand{\labels@magicalarmour}{Armures magiques}
\newcommand{\labels@talismans}{Talismans}
\newcommand{\labels@enchanteditems}{Objets enchantés}
\newcommand{\labels@arcaneitems}{Objets cabalistiques}
\newcommand{\labels@magicalbanners}{Bannières magiques}
\newcommand{\labels@quickrefsheet}{Fiche de référence}
\newcommand{\labels@changelog}{Change Log}

\newcommand{\labels@lordsInitial}{S}
\newcommand{\labels@heroesInitial}{H}
\newcommand{\labels@coreunitsInitial}{B}
\newcommand{\labels@specialunitsInitial}{S}
\newcommand{\labels@rareunitsInitial}{R}
\newcommand{\labels@mountsInitial}{M}


% Titlepage

\newcommand{\labels@fantasybattles}{Batailles Fantastiques}
\newcommand{\labels@NinthAge}{Le 9\ieme Âge}
\newcommand{\labels@creators}{Une collaboration des créateurs de l'ETC et du Swedish Comp System}
\newcommand{\labels@introduction}{%
\noindent {\Largerfontsize\textbf{Note des traducteurs}}
\vspace{0.5cm}

Nous souhaitons remercier chaleureusement l'équipe à l'initiative du 9\ieme Âge pour leur motivation et leur travail continu pour faire vivre notre passion. Nous espérons que ce jeu saura développer les qualités pour plaire au plus grand nombre et réunir les joueurs, amateurs comme habitués des tournois, autour de règles amusantes et équilibrées, pour finalement s'imposer comme un standard du jeu de figurines. Une grande ambition qui ne pourra s'accomplir que \textbf{grâce à vous}, la communauté, via des retours constructifs, afin de modeler le jeu selon nos désirs. N'étant \textbf{en aucun cas à but lucratif}, le 9\ieme Âge part avec un avantage considérable. Les règles des éventuelles nouvelles sorties ne seront pas dictées par le besoin de vendre ces nouveautés. Vous pouvez choisir et acheter vos figurines où bon vous semble, il n'y a pas un unique revendeur toléré. Vous n'êtes pas bloqués dans une spirale infernale où pour continuer à jouer à un jeu, dans lequel vous vous êtes tant investis, vous devez payer toujours plus cher pour entretenir votre collection. Enfin, vous pouvez être assurés que tant que 9\ieme Âge sera joué, vous disposerez d'un \textbf{support continu et régulier}, celui-ci étant offert par la communauté.

Nous attirons votre attention sur le fait que ce jeu en est encore à ses débuts et dans un \textbf{stade de développement}. Ce document correspond à une version de brouillon \textbf{\og{} beta \fg{}}, dont le but et de tester le jeu et le modifier jusqu'à atteindre une version satisfaisante. Attendez-vous donc à trouver des déséquilibres, des incohérences, et à obtenir des mises à jour régulières avec éventuellement des changements importants. N'hésitez pas à nous donner vos avis ! Ce livre d'armée n'est utilisable qu'en compagnie du livre de Règles et du livre de Magie.

Concernant la traduction en elle-même, nous avons fait de notre mieux pour vous offrir une version de qualité, dont nous espérons qu'elle surpasse celle de la version originale ! Si vous constatez des coquilles, des erreurs, merci de nous les signaler en nous contactant sur le forum du 9\ieme Âge, dans le \textbf{sous-forum français} (\url{http://www.the-ninth-age.com/index.php?board/117-french/}). Vous y trouverez aussi les dernières mises à jour. \textbf{En cas de conflit d'interprétation avec la version originale, la version originale fait référence}.

\vspace{0.5cm}
Que ce jeu vous apporte d'innombrables heures de plaisir partagé !

\vspace{0.7cm}
\noindent {\Largerfontsize\textbf{Les traducteurs}}
\vspace{0.1cm}

\ifdef{\translationteam}{
	\begin{multicols}{3}
	\begin{itemize}
		\translationteam
	\end{itemize}
	\end{multicols}
}{}
}
\newcommand{\labels@secondpageannouncement}{%
	\labels@fantasybattles{} : \labels@NinthAge{} est un jeu créé et entretenu par la communauté qui met en scène des affrontements de figurines. Toutes les règles sont disponibles gratuitement sur le site suivant. Vos retours et suggestions sont les bienvenus.
	\newline\url{http://www.the-ninth-age.com/}
}
\newcommand{\labels@rulechanges}{%
	Les changements de règles entre versions sont colorés comme ce paragraphe. Une liste en anglais de ces changements par version est ajoutée à la fin de cet ouvrage.
}
\newcommand{\labels@latexcredit}{Document réalisé à l'aide de \LaTeX .}


%%% Technical commands

\newcommand{\only}[1]{(#1 uniquement)}
\newcommand{\free}{gratuit}
\newcommand{\upto}{jusqu'à}
\newcommand{\Upto}{Jusqu'à}
\newcommand{\unlimited}{sans limite de pts}
\newcommand{\permodel}{/fig.}
\newcommand{\listlastchoice}{ ou}
\newcommand{\notif}[1]{(pas #1)}
\newcommand{\wordand}{et}
\newcommand{\wordwith}{avec}
\newcommand{\ifNmodelsorless}[1]{(#1 figurines ou moins)}
\newcommand{\unitwith}{unité avec}
\newcommand{\From}{De} % From ... to ... models
\newcommand{\wordto}{à}
\newcommand{\wordAll}{Tous}
\newcommand{\spacebeforecolon}{ } % French put a space before colons
\newcommand{\minprice}{Coût min. :}
\newcommand{\mincostfor}{Coût min. pour}
\newcommand{\maxunitsize}{Taille max.}
\newcommand{\additionalfigscost}{Les figurines additionnelles coûtent}


%%% Special rules %%%

\newcommand{\ambush}{Embuscade}
\newcommand{\armourpiercing}[1]{Perforant\ifblank{#1}{}{ (#1)}}
\newcommand{\bodyguard}[1]{Garde du Corps\ifblank{#1}{}{ (#1)}}
\newcommand{\breathweapon}[1]{Attaque de Souffle\ifblank{#1}{}{ (#1)}}
\newcommand{\channel}{Canalisation}
\newcommand{\crushattack}{Attaque Écrasante}
\newcommand{\devastatingcharge}{Charge Dévastatrice}
\newcommand{\distracting}{Distrayant}
\newcommand{\engineer}{Ingénieur}
\newcommand{\ethereal}{Éthéré}
\newcommand{\fastcavalry}{Cavalerie Légère}
\newcommand{\fear}{Peur}
\newcommand{\fightinextrarank}{Combat avec un Rang Supplémentaire}
\newcommand{\fireborn}{Né du Feu}
\newcommand{\flamingattacks}{Attaques Enflammées}
\newcommand{\flammable}{Inflammable}
\newcommand{\lighttroops}{Troupes Légères}
\newcommand{\frenzy}{Frénésie}
\newcommand{\fly}[1]{Vol\ifblank{#1}{}{ (#1)}}
\newcommand{\grindingattacks}[1]{Attaques de Broyage\ifblank{#1}{}{ (#1)}}
\newcommand{\hardtarget}{Camouflé}
\newcommand{\hatred}{Haine}
\newcommand{\hellfire}{Flammes de l'Enfer}
\newcommand{\hidden}{Caché}
\newcommand{\holyattacks}{Attaques Divines}
\newcommand{\immunetopsychology}{Immunisé à la Psychologie}
\newcommand{\impacthits}[1]{Touches d'Impact\ifblank{#1}{}{ (#1)}}
\newcommand{\insignificant}{Insignifiant}
\newcommand{\largetarget}{Grande Cible}
\newcommand{\lethalstrike}{Coup Fatal}
\newcommand{\lightningattacks}{Attaques Foudroyantes}
\newcommand{\lightningreflexes}{Réflexes Foudroyants}
\newcommand{\magicresistance}[1]{Résistance à la Magie\ifblank{#1}{}{ (#1)}}
\newcommand{\magicalattacks}{Attaques Magiques}
\newcommand{\metalshifting}{Fusion du Métal}
\newcommand{\moveorfire}{Mouvement ou Tir}
\newcommand{\multipleshots}[1]{Tirs Multiples\ifblank{#1}{}{ (#1)}}
\newcommand{\multiplewounds}[2]{Blessures Multiples\ifblank{#1}{}{ (#1\ifblank{#2}{)}{, #2)}}}
\newcommand{\notaleader}{Pas un Meneur}
\newcommand{\otherworldly}{D'Outre-Monde}
\newcommand{\pathmaster}[1]{Maître de la Discipline\ifblank{#1}{}{ (#1)}}
\newcommand{\poisonedattacks}{Attaques Empoisonnées}
\newcommand{\quicktofire}{Tir Rapide}
\newcommand{\randommovement}[1]{Mouvement Aléatoire\ifblank{#1}{}{ (#1)}}
\newcommand{\randomattacks}[1]{Attaques Aléatoires\ifblank{#1}{}{ (#1)}}
\newcommand{\regeneration}[1]{Régénération\ifblank{#1}{}{ (#1+)}}
\newcommand{\reload}{Rechargez !}
\newcommand{\requirestwohands}{Arme à deux Mains}
\newcommand{\scythes}{Faux}
\newcommand{\scout}{Éclaireur}
\newcommand{\scouts}{Éclaireurs}
\newcommand{\stomp}[1]{Piétinement\ifblank{#1}{}{ (#1)}}
\newcommand{\strider}[1]{Guide\ifblank{#1}{}{ (#1)}}
\newcommand{\stubborn}{Tenace}
\newcommand{\stupidity}{Stupidité}
\newcommand{\skirmisher}{Tirailleur}
\newcommand{\skirmishers}{Tirailleurs}
\newcommand{\sweepingattack}{Attaque au Passage}
\newcommand{\swiftstride}{Rapide}
\newcommand{\thunderouscharge}{Charge Tonitruante}
\newcommand{\terror}{Terreur}
\newcommand{\toxicattacks}{Attaques Toxiques}
\newcommand{\unbreakable}{Indémoralisable}
\newcommand{\undead}{Mort-Vivant}
\newcommand{\unstable}{Instable}
\newcommand{\unwieldy}{Encombrant}
\newcommand{\vanguard}{Avant-Garde}
\newcommand{\volleyfire}{Tir de Volée}
\newcommand{\warplatform}{Plateforme de Guerre}
\newcommand{\wardsave}[1]{Sauvegarde Invulnérable\ifblank{#1}{}{ (#1+)}}
\newcommand{\weaponmaster}{Maître d'Ar\-mes}
\newcommand{\wizardconclave}[1]{Conclave de Sorciers\ifblank{#1}{}{ (#1)}}


%%% Magic %%%

\newnamemacro{\Pathof}{Discipline}

\newcommand{\battle}{Commune}
\newcommand{\alchemy}{de l'Alchimie}
\newcommand{\death}{de la Mort}
\newcommand{\fire}{du Feu}
\newcommand{\heavens}{des Cieux}
\newcommand{\light}{de la Lumière}
\newcommand{\nature}{de la Nature}
\newcommand{\shadows}{des Ombres}
\newcommand{\wilderness}{de la Sauvagerie Bestiale}
\newcommand{\butchery}{de la Boucherie}
\newcommand{\change}{du Changement}
\newcommand{\thebiggreengods}{des Grands Dieux Verts}
\newcommand{\thelittlegreengods}{des Petits Dieux Verts}
\newcommand{\blackmagic}{de la Magie Noire}
\newcommand{\disease}{de la Maladie}
\newcommand{\lust}{de la Luxure}
\newcommand{\necromancy}{de la Nécromancie}
\newcommand{\ruin}{de la Ruine}
\newcommand{\forge}{de la Forge}
\newcommand{\sands}{des Sables}
\newcommand{\whitemagic}{de la Magie Blanche}

\newcommand{\anyofthebattlemagic}{dans n'importe laquelle des Disciplines Communes}

\newcommand{\magiclevel}[1]{\ifnumcomp{#1}{<}{3}{Sorcier Apprenti}{Maître Sorcier} Niveau #1}
\newcommand{\Level}{Niveau}

\newcommand{\wizard}{Sorcier}
\newcommand{\wizards}{Sorciers}

\newcommand{\boundspell}[1]{Objet de Sort, Puissance #1}


%%% Other rules %%%

\newcommand{\armoursave}{Sauvegarde d'Armure}
\newcommand{\firstinrank}{Au Premier Rang}
\newcommand{\hardcover}{Couvert Lourd}
\newcommand{\holdyourground}{Tenez les Rangs}
\newcommand{\inspiringpresence}{Présence Charismatique}
\newcommand{\lightcover}{Couvert Léger}
\newcommand{\monstrousrank}{Rang Monstrueux}
\newcommand{\ordnance}{Artillerie}
\newcommand{\parry}{Parade}
\newcommand{\raisewounds}{Ressusciter des Figurines}
\newcommand{\recoverwounds}{Récupérer des PVs}
\newcommand{\aideddispel}{Dissipation Assistée}
\newcommand{\rnf}{ordinaires}
\newcommand{\general}{Général}


%%% Equipment %%%

\newcommand{\innatedefence}[1]{Protection Innée\ifblank{#1}{}{~(#1+)}}
\newcommand{\mountsprotection}[1]{Protection de Monture\ifblank{#1}{}{~(#1+)}}
\newcommand{\la}{Armure Légère}
\newcommand{\ha}{Armure Lourde}
\newcommand{\platearmour}{Armure de Plates}
\newcommand{\hw}{Arme de Base}
\newcommand{\pw}{Paire d'Armes}
\newcommand{\spear}{Lance}
\newcommand{\halberd}{Hallebarde}
\newcommand{\gw}{Arme Lourde}
\newcommand{\lance}{Lance de Cavalerie}
\newcommand{\lightlance}{Lance Légère}
\newcommand{\shield}{Bouclier}
\newcommand{\barding}{Caparaçon}
\newcommand{\throwingweapons}{Armes de Jet}
\newcommand{\shortbow}{Arc Court}
\newcommand{\flail}{Fléau}

\newcommand{\cannon}{Canon}
\newcommand{\catapult}{Catapulte}
\newcommand{\volleygun}{Batterie de Tir}
\newcommand{\boltthrower}{Baliste}
\newcommand{\artilleryweapon}{Arme d'Artillerie}


%%% Troop types %%%

\newcommand{\characters}{Personnages}
\newcommand{\infantry}{Infanterie}
\newcommand{\monstrousinfantry}{Infanterie Monstrueuse}
\newcommand{\cavalry}{Cavalerie}
\newcommand{\monstrouscavalry}{Cavalerie Monstrueuse}
\newcommand{\swarm}{Nuée}
\newcommand{\swarms}{Nuées}
\newcommand{\warbeast}{Bête de Guerre}
\newcommand{\warbeasts}{Bêtes de Guerre}
\newcommand{\monster}{Monstre}
\newcommand{\monsters}{Monstres}
\newcommand{\monstrousbeast}{Bête Monstrueuse}
\newcommand{\monstrousbeasts}{Bêtes Monstrueuses}
\newcommand{\chariot}{Char}
\newcommand{\chariots}{Chars}
\newcommand{\riddenmonster}{Monstre Monté}
\newcommand{\riddenmonsters}{Monstres Montés}
\newcommand{\warmachine}{Machine de Guerre}
\newcommand{\warmachines}{Machines de Guerre}


%%% Terrain %%%

\newcommand{\water}{Eaux peu profondes}


%%% Profile wording

\newcommand{\oneofakind}{Uni\-que}
\newcommand{\onechoiceonly}{(un seul choix)}
\newcommand{\onfootonly}{(à pied seulement)}
\newcommand{\closecombatonly}{seulement au Corps à Corps}
\newcommand{\Xmodelsorless}[1]{(#1 figurines ou moins)}
\newcommand{\magicalitemsallowance}{Peut prendre des Objets Magiques}
\newcommand{\magicalweaponallowance}{Peut prendre une Arme Magique}
\newcommand{\notmagicalarmour}{(mais pas d'Armure Magique)}
\newcommand{\anyofthefollowing}{\optionschoice{Peut prendre :}}
\newcommand{\weapononechoice}{\optionschoice{Peut prendre une arme \onechoiceonly{} :}}
\newcommand{\weaponschoice}{\optionschoice{Peut prendre des armes :}}
\newcommand{\shootingweapononechoice}{\optionschoice{Peut prendre une arme de tir \onechoiceonly{} :}}
\newcommand{\combatweapononechoice}{\optionschoice{Peut prendre une arme de corps à corps \onechoiceonly{} :}}
\newcommand{\armouronechoice}{\optionschoice{Peut prendre une armure \onechoiceonly{} :}}
\newcommand{\magiclevelchoice}{\optionschoice{Peut devenir au choix :}}
\newcommand{\bsboption}{Peut devenir Porteur de la Grande Bannière}
\newcommand{\mayupgradeto}{Peut être amélioré en}
\newcommand{\mustbecomeoneofthefollowing}{\optionschoice{Doit devenir un choix parmi :}}
\newcommand{\maybecomeoneofthefollowing}{\optionschoice{Peut devenir un choix parmi :}}
\newcommand{\maytakeoneofthefollowing}{\optionschoice{Peut prendre un choix parmi :}}
\newcommand{\maytakeuptotwoofthefollowing}{\optionschoice{Peut prendre jusqu'à deux choix parmi :}}
\newcommand{\maygain}{Peut gagner la règle}
\newcommand{\maytake}{Peut prendre}
\newcommand{\maytakeashield}{Peut prendre un Bouclier}
\newcommand{\maytakela}{Peut prendre une Armure Légère}
\newcommand{\maytakeha}{Peut prendre une Armure Lourde}
\newcommand{\maytakemountsprotectionX}[1]{Peut prendre une \mountsprotection{#1}}
\newcommand{\maytakeagw}{Peut prendre une Arme Lourde}
\newcommand{\maytakeaspear}{Peut prendre une Lance}
\newcommand{\maytakepw}{Peut prendre une Paire d'Armes}
\newcommand{\maytakethrowingweapons}{Peut prendre des Armes de Jet}
\newcommand{\maytakebarding}{Peut prendre un Caparaçon}
\newcommand{\replaceshieldwithhalberd}{Remplacer le Bouclier par une Hallebarde}
\newcommand{\maybecome}{Peut devenir}

\newcommand{\maytakeonechoiceonly}{\optionschoice{\maytake{} \onechoiceonly{}\spacebeforecolon{}:}}

\newcommand{\mountssectionannouncement}{%
La section Montures concerne les montures de Personnages. Les montures pour non-Personnages suivent les règles données dans leur description d'unité.
}


%%% Technical commands %%%

\newcommand{\amel}[1]{\textcolor{blue}{[#1]}}
\newcommand{\base}{\textcolor{red}}
\newcommand{\amelbis}[1]{\textcolor{olive}{[[#1]]}}

\newcommand{\newrule}{\textcolor{green!50!black}}
\newcommand{\removedrule}[1]{\textcolor{green!50!black}{\sout{#1}}}
\newcommand{\starsymbol}{$\star$}
\newcommand{\refsymbol}{$^\star$}

\newcommand{\inch}{\arcsecond}
\newcommand{\foot}{\arcminute}
\newcommand{\range}[1] {\labels@range~\unit{#1}{\inch}}
\newcommand{\distance}[1] {\unit{#1}{\inch}}
\newcommand{\result}[1] {\texttt{'}#1\texttt{'}}
\newcommand{\plusone}{+1}

\newcommand{\verysmallfontsize}{\fontsize{4}{4.8}\selectfont}
\newcommand{\smallfontsize}{\fontsize{6}{7.2}\selectfont}
\newcommand{\normalfontsize}{\fontsize{8}{9.6}\selectfont}
\newcommand{\largefontsize}{\fontsize{10}{12}\selectfont}
\newcommand{\largerfontsize}{\fontsize{12}{14.4}\selectfont}
\newcommand{\Largefontsize}{\fontsize{14}{16.8}\selectfont}
\newcommand{\Largerfontsize}{\fontsize{15}{18}\selectfont}
\newcommand{\hugefontsize}{\fontsize{18}{21.6}\selectfont}
\newcommand{\Hugefontsize}{\fontsize{25}{30}\selectfont}

%%% Table of Contents %%%

\newcommand{\toctarget}[1]{%
\phantomsection\label{#1}%
\hypertarget{#1}%
}

\newcommand{\tocentry}[2]{%
\noindent\hyperlink{#1}{#2}\dotfill\pageref{#1}%
}


%%% Headers %%%

\renewcommand{\headrulewidth}{0pt}
\fancyfoot[L]{\textcolor{black!30}{%
%\hyperlink{lordtitle}{\labels@lordsInitial}\hspace*{0.4cm}
%\hyperlink{herotitle}{\labels@heroesInitial}\hspace*{0.4cm}
%\hyperlink{coretitle}{\labels@coreunitsInitial}\hspace*{0.4cm}
%\hyperlink{specialtitle}{\labels@specialunitsInitial}\hspace*{0.4cm}
%\hyperlink{raretitle}{\labels@rareunitsInitial}\hspace*{0.4cm}
}}
\fancyfoot[R]{\textcolor{black!30}{%
%\hyperlink{lordtitle}{\labels@lordsInitial}\hspace*{0.4cm}
%\hyperlink{herotitle}{\labels@heroesInitial}\hspace*{0.4cm}
%\hyperlink{coretitle}{\labels@coreunitsInitial}\hspace*{0.4cm}
%\hyperlink{specialtitle}{\labels@specialunitsInitial}\hspace*{0.4cm}
%\hyperlink{raretitle}{\labels@rareunitsInitial}\hspace*{0.4cm}
}}


\setlength{\columnsep}{1cm}

%%% Table parameters %%%

\newcolumntype{M}[1]{>{\centering\let\newline\\\arraybackslash\hspace{0pt}}m{#1}}

\renewcommand{\arraystretch}{3.2}

\arrayrulecolor{black!30}
\setlength{\arrayrulewidth}{2pt}

\newcommand{\starttable}[2][black]{%
\vspace{0.3cm}
\begin{center}
\begin{tabular}{@{}>{\bf\LARGE}M{0.7cm}>{\raggedright}m{3cm}M{0.8cm}M{1.8cm}M{1.4cm}m{6.5cm}@{}}
\rowcolor[HTML]{#2} &
\textcolor{#1}{\textbf{Nom}} &
\textcolor{#1}{\textbf{Lancement}} &
\textcolor{#1}{\textbf{Type}} &
\textcolor{#1}{\textbf{Durée}} &
\centering\textcolor{#1}{\textbf{Effet}}
\tabularnewline
}

\newcommand{\closetable}{%
\end{tabular}
\end{center}
}

\def\colors@alchemy{FFD966}
\def\colors@death{434343}
\def\colors@fire{FF0000}
\def\colors@heavens{C9DAF8}
\def\colors@light{FFF2CC}
\def\colors@nature{274E13}
\def\colors@shadows{999999}
\def\colors@wilderness{7F6000}
\def\colors@butchery{85200C}
\def\colors@change{9900FF}
\def\colors@forge{5B0F00}
\def\colors@biggreengods{38761D}
\def\colors@littlegreengods{93C47D}
\def\colors@lust{D5A6BD}
\def\colors@whitemagic{CFE2F3}
\def\colors@blackmagic{20124D}
\def\colors@disease{7F6000}
\def\colors@necromancy{000000}
\def\colors@ruin{69431C}
\def\colors@sand{FFD966}



\newcommand{\booktitle}{Livre de Magie}
\newcommand{\version}{0.99.9}
\newcommand{\frenchversion}{2.0}

\hypersetup{pdftitle={T9A - \booktitle}}

% Document Titles

\newcommand{\whatisFBTNA}{BF : Le 9\ieme Âge, qu'est-ce que c'est ?}
\newcommand{\howtousethisdoc}{Comment utiliser ce livre ?}
\newcommand{\magicphasesummary}{Résumé de la Phase de Magie}

% Table Titles

\newcommand{\spellsName}{Nom}
\newcommand{\spellsCastingValue}{Lancement}
\newcommand{\spellsType}{Type}
\newcommand{\spellsDuration}{Durée}
\newcommand{\spellsEffect}{Effet}

% Paths TOC

\newcommand{\alchemyTOC}{Alchimie}
\newcommand{\deathTOC}{Mort}
\newcommand{\fireTOC}{Feu}
\newcommand{\heavensTOC}{Cieux}
\newcommand{\lightTOC}{Lumière}
\newcommand{\natureTOC}{Nature}
\newcommand{\shadowsTOC}{Ombres}
\newcommand{\wildernessTOC}{Sauvagerie}

\newcommand{\eightfoldpathTOC}{Octuple}

\newcommand{\whitemagicTOC}{Magie Blanche}
\newcommand{\blackmagicTOC}{Magie Noire}
\newcommand{\necromancyTOC}{Nécromancie}
\newcommand{\sandsTOC}{Sables}
\newcommand{\forgeTOC}{Forge}
\newcommand{\biggreengodsTOC}{Grands Dieux Verts}
\newcommand{\littlegreengodsTOC}{Petits Dieux Verts}
\newcommand{\butcheryTOC}{Boucherie}
\newcommand{\ruinTOC}{Ruine}
\newcommand{\diseaseTOC}{Maladie}
\newcommand{\lustTOC}{Luxure}
\newcommand{\changeTOC}{Changement}







\begin{document}

\newgeometry{margin=1in}

\begin{titlepage}
\begin{center}

\ifdef{\booktitle}{}{\newcommand{\booktitle}{Missing title}}
\ifdef{\version}{}{\newcommand{\version}{Missing version}}

{\titlefont\fontsize{40}{48}\selectfont\noindent\labels@fantasybattles

\labels@NinthAge}

\vspace*{0.7cm}

\newcommand{\iconheight}{1.6cm}
\newcommand{\minipagewidth}{6cm}
\newcommand{\spacebetweeniconrows}{0.3cm}

\hfill\begin{minipage}[c]{\minipagewidth}
\begin{center}
\includegraphics[height=\iconheight]{\alchemyicon}

\vspace*{\spacebetweeniconrows}

\includegraphics[height=\iconheight]{\shamanismicon}

\vspace*{\spacebetweeniconrows}

\includegraphics[height=\iconheight]{\cosmologyicon}

\vspace*{\spacebetweeniconrows}

\includegraphics[height=\iconheight]{\divinationicon}

\vspace*{\spacebetweeniconrows}

\includegraphics[height=\iconheight]{\druidismicon}

\end{center}
\end{minipage}\begin{minipage}[c]{\minipagewidth}
\begin{center}
\includegraphics[height=\iconheight]{\evocationicon}

\vspace*{\spacebetweeniconrows}

\includegraphics[height=\iconheight]{\occultismicon}

\vspace*{\spacebetweeniconrows}

\includegraphics[height=\iconheight]{\pyromancyicon}

\vspace*{\spacebetweeniconrows}

\includegraphics[height=\iconheight]{\witchcrafticon}

\vspace*{\spacebetweeniconrows}

\includegraphics[height=\iconheight]{\thaumaturgyicon}
\end{center}
\end{minipage}\hspace*{\fill}


\vspace*{0.7cm}

{\titlefont\fontsize{50}{60}\selectfont \booktitle
\vspace{0.4cm}

\fontsize{14}{16.8}\selectfont Beta v\version{} - \today{}}

\ifdef{\frenchversion}{{\fontsize{14}{16.8}\selectfont \vspace{0.2cm}\noindent\texttt{VF \frenchversion}}}{}
\vfill

\begin{tabular}{@{}m{2cm}@{\hskip 20pt}m{13cm}@{}}
\includegraphics[width=2cm]{../Layout/pics/seal_9th.png} &
{\fontsize{10}{12}\selectfont \textcolor{black!50}{\noindent\labels@frontpagecredits}}

\ifdef{\frontpageaddstuff}{{\fontsize{10}{12}\selectfont \noindent\textcolor{black!50}{\frontpageaddstuff}}}{}

\vspace*{10pt}
\noindent{\fontsize{10}{12}\selectfont \textcolor{black!50}{\labels@license}}
\tabularnewline
\end{tabular}


\end{center}

\newpage

\thispagestyle{empty}

{\fontsize{10}{12}\selectfont

\begin{center}\hypertarget{tableofcontents}{\noindent{\Largerfontsize\textbf{\labels@tableofcontents}}}\end{center}

\vspace*{0.2cm}

\begin{center}
\noindent\hyperlink{howtousethisdoc}{\howtousethisdoc}
\end{center}

\begin{multicols}{2}

\tocentry{alchemy}{\alchemyTOC}

\tocentry{shamanism}{\shamanismTOC}

\tocentry{cosmology}{\cosmologyTOC}

\tocentry{divination}{\divinationTOC}

\tocentry{druidism}{\druidismTOC}

\tocentry{evocation}{\evocationTOC}

\tocentry{occultism}{\occultismTOC}

\tocentry{pyromancy}{\pyromancyTOC}

\tocentry{witchcraft}{\witchcraftTOC}

\tocentry{thaumaturgy}{\thaumaturgyTOC}

\end{multicols}

\begin{center}
\noindent\hyperlink{magicphasesummary}{\magicphasesummary}
\end{center}

\ifdef{\labels@introduction}{\vspace{0.7cm}\labels@introduction}{\vphantom{1pt}}
\vfill

\noindent\newrule{\labels@rulechanges}

\bigskip
\noindent \labels@latexcredit
}


\end{titlepage}

\restoregeometry

\largefontsize

\basictitle{whatisFBTNA}{\whatisFBTNA}

\spaceaftersection{}

Batailles Fantastiques : Le 9\ieme Âge est un jeu de plateau qui met en scène deux armées s'affrontant, représentées par des figurines adéquates. Plus d'informations sur le jeu ici :
\begin{center}
\noindent\url{http://www.the-ninth-age.com/}
\end{center}

\vspace*{1.5cm}
\basictitle{howtousethisdoc}{\howtousethisdoc}

\spaceaftersection{}

Dans ce livre, toutes les Disciplines de Magie sont présentées. Elles comprennent les 8 \textbf{Disciplines Communes} et les \textbf{Disciplines Spécifiques} des différentes armées. Pour plus d'informations sur les règles du jeu et en particulier la Phase de Magie, merci de vous référer au Livre de Règles.

\vspace*{10pt}
Chaque Discipline comprend 8 sorts. Le premier est appelé l’\textbf{Attribut} de la Discipline, qui est indiqué par la mention \og A \fg{} à la place du numéro du sort. C'est un sort particulier qui est automatiquement lancé quand un autre sort de la même Discipline est lancé avec succès. Les sept sorts suivants sont numérotés de 0 à 6. Le sort 0 est le sort \textbf{Primaire} de la Discipline. Un Sorcier peut toujours choisir d'échanger l'un de ses sorts générés aléatoirement par le sort Primaire, et plusieurs Sorciers peuvent connaître ce sort en même temps.

\vspace*{10pt}
Certains sorts disposent de plusieurs valeurs de lancement. La ou les plus grandes sont pour la  ou les versions \textbf{améliorée(s)} du sort. Le joueur doit choisir la version du sort à lancer, sachant que plus la valeur de lancement est grande, plus le sort sera difficile à lancer, mais aussi plus il sera puissant. Parfois, les versions améliorées voient leur portée et le nombre de cibles augmenter, tandis que pour d'autres sorts, les effets peuvent être modifiés. Les différences entre versions améliorée(s) et basique suivent un code couleur. La partie en noir s'applique à toutes les versions du sort. Les textes en \base{rouge} s'appliquent à la version basique, tandis que les textes en \amel{bleu et entre crochets} s'appliquent à la version améliorée. Dans certains cas, il existe une deuxième amélioration au sort, marquée en \amelbis{vert olive entre accolades}. 

\vspace*{2.5cm}
\begin{center}
\includegraphics[width=1cm]{pics/alchemy.png}\hspace{0.2cm}%
\includegraphics[width=1cm]{pics/heavens.png}\hspace{0.2cm}%
\includegraphics[width=1cm]{pics/fire.png}\hspace{0.2cm}%
\includegraphics[width=1cm]{pics/light.png}%
\end{center}

\begin{center}
\includegraphics[width=1cm]{pics/death.png}\hspace{0.2cm}%
\includegraphics[width=1cm]{pics/nature.png}\hspace{0.2cm}%
\includegraphics[width=1cm]{pics/shadows.png}\hspace{0.2cm}%
\includegraphics[width=1cm]{pics/wilderness.png}%
\end{center}







\normalfontsize

\newbattlepath{alchemy}{\alchemyTOC}

\starttable{\colors@alchemy}
A &
\alchemyattribute &
&
\range{12} \newline
\augment{} &
\lastsoneturn{} &
La cible gagne +1 en Sauvegarde d'Armure. Aucune figurine ne peut obtenir mieux que 3+ en Sauvegarde d'Armure grâce à ce sort avant d'appliquer d'éventuels malus.
\tabularnewline
\hline
0 & \alchemysignature &
\base{9+}\newline  \amel{17+} &
\range{24} \newline
\hex{} \newline
\missile{} \newline
\damage{} &
\instant{} &
La cible subit \base{1D6} \amel{2D6} touches avec la règle \metalshifting{}.
\tabularnewline
\hline
1 & \alchemyspellone{} &
\base{7+} \newline
\amel{11+} &
\base{\range{18}} \newline
\amel{\range{36}} \newline
\augment{} &
\lastsoneturn{} &
Les Attaques de Corps à Corps et de Tir de la cible ont +1 pour toucher et gagnent les règles \magicalattacks{} et \armourpiercing{+1}.
\tabularnewline
\hline
2 & \alchemyspelltwo{} &
\base{7+} \newline
\amel{10+} &
\base{\range{24}} \newline
\amel{\range{48}} \newline
\hex{} &
\permanent{} &
La cible subit un malus de -1 sur sa Sauvegarde d'Armure.
\tabularnewline
\hline
3 & \alchemyspellthree{} &
\base{8+} \newline
\amel{11+} &
\base{\range{12}} \newline
\amel{\range{24}} \newline
\augment{} &
\lastsoneturn{} &
La cible gagne les règles \hardtarget{} et \distracting{}.

\vspace*{5pt}De plus, les Attaques de Corps à Corps contre la cible subissent également une pénalité de -1 sur leur règle \armourpiercing{}.
\tabularnewline
\hline
4 & \alchemyspellfour{} &
\base{8+} \newline
\amel{11+} &
\base{\range{18}} \newline
\amel{\range{36}} \newline
\hex{} \newline 
\missile{} \newline 
\damage{} &
\instant{} &
La cible subit une touche avec les règles \multiplewounds{1D3}{} et \metalshifting{}. Les rangs de l'unité ciblée sont pénétrés de la même façon que pour une \boltthrower{}, mais l'attaque subit un malus de -1 pour blesser au lieu de -1 en Force pour chaque rang pénétré. 
\tabularnewline
\hline
5 & \alchemyspellfive{}  &
\base{9+} \newline
\amel{12+} &
\base{\range{24}} \newline
\amel{\range{48}} \newline
\hex{} &
\lastsoneturn{} &
Les figurines de l'unité ciblée ne peuvent pas recevoir de bonus de Force de leurs Armes de Corps à Corps standard.

\vspace*{5pt}
Les Armes de Tir standard portées par l'unité ciblée subissent un malus de -1 en Force. Ce sort n'affecte que les équipements standard et leur Force, aucune règle spéciale n'est altérée par ce sort.
\tabularnewline
\hline
6 & \alchemyspellsix{} &
\base{15+} \newline
\amel{18+} &
\base{\range{12}} \newline
\amel{\range{24}} \newline
\hex{} \newline
\direct{} \newline
\damage{} &
\instant{} / \newline \lastsoneturn{} &
Le propriétaire de l'unité ciblée lance un dé pour chaque figurine de l'unité, dans l'ordre de son choix. Ignorez le premier 5+ obtenu. Pour les figurines suivantes, sur 5+, la figurine subit une blessure avec la règle \multiplewounds{10}{} sans aucune sauvegarde autorisée.

\vspace*{5pt}
Les unités ennemies à moins de \distance{12} de la cible gagnent la règle \stupidity{}.
\tabularnewline
\closetable



\newbattlepath{heavens}{\heavensTOC}

\starttable{\colors@heavens}
A &
\heavensattribute{} &
&
\specialTYPE{} &
\lastsoneturn{} &
L'armée du lanceur gagne un marqueur \og Second Sceau \fg{}. Ce marqueur peut être dépensé pour relancer un unique D6 d'un jet pour toucher, pour blesser ou de Sauvegarde d'Armure.
\tabularnewline
\hline
0 & \heavenssignature{} &
\base{7+}\newline
\amel{10+} &
\base{\range{12}} \newline
\amel{\range{36}} \newline
\hex{} &
\lastsoneturn{} &
La cible subit un malus de -1 pour toucher et de -1 en Commandement. Les Attaques de Tir ne nécessitant pas l'utilisation de la CT doivent obtenir un 4+ sur 1D6 pour pouvoir être utilisées.
\tabularnewline
\hline
1 & \heavensspellone{} &
\base{6+} \newline
\amel{9+} &
\base{\range{18}} \newline
\amel{\range{36}} \newline
\hex{} &
\lastsoneturn{} &
L'unité ciblée ne peut pas se déplacer de plus de \distance{10} durant l'Étape des Autres Mouvements. Toutes les unités ennemies à moins de \distance{6} de la cible subissent un malus de -1 en Capacité de Tir.
\tabularnewline
\hline
2 & \heavensspelltwo{} &
\base{8+}\newline
\amel{11+} &
\base{\range{24}} \newline
\amel{\range{48}} \newline
\hex{} \newline
\missile{} \newline
\damage{} &
\instant{} &
La cible subit 1D6 touches de Force 6 avec la règle \lightningattacks{}.
\tabularnewline
\hline
3 & \heavensspellthree{} &
\base{9+}\newline
\amel{12+} &
\base{\range{12}} \newline
\amel{\range{24}} \newline
\augment{} &
\lastsoneturn{} &
La cible peut relancer au choix les jets ratés pour toucher, pour blesser ou de Sauvegarde d’Armure. Déclarez votre choix avant de jeter le sort.
\tabularnewline
\hline
4 & \heavensspellfour{} &
\base{9+}\newline
\amel{12+} &
\base{\range{12}} \newline
\amel{\range{24}} \newline
\hex{} &
\lastsoneturn{} &
La cible doit relancer au choix les jets réussis pour toucher, pour blesser ou de Sauvegarde d'Armure. Déclarez votre choix avant de jeter le sort.
\tabularnewline
\hline
5 & \heavensspellfive{} &
13+ &
\range{24} \newline
\hex{} \newline
\direct{} \newline
\damage{} &
\instant{} &
Ciblez une unité, puis jetez 1D6. Sur 3+, choisissez une nouvelle cible située à moins de \distance{6} de la première. Continuez ainsi, de cible en cible à \distance{6} d'écart, jusqu'à ce que vous obteniez un \result{1} ou un \result{2}, ou qu'aucune cible ne soit éligible. Aucune unité ne peut être ciblée deux fois.

\vspace*{5pt}
Chaque cible subit 1D6 touches de Force 6 avec la règle \lightningattacks{}.
\tabularnewline
\hline
6 & \heavensspellsix{} &
\base{13+}\newline
\amel{16+} &
\ground{} &
\permanent{} &
Posez un marqueur où vous voulez sur le champ de bataille.

\vspace*{5pt}
\base{À la fin de chaque Phase de Magie suivante, jetez un dé. Si le résultat est de 4+, la comète arrive.}

\vspace*{5pt}
\amel{Le lanceur choisit le Tour de Joueur, autre que celui en cours, et hors premier Tour de Jeu, où la comète arrive et l'écrit secrètement. La comète arrivera à la fin de la Phase de Magie de ce tour de joueur.}

\vspace*{5pt}
À la fin de chaque Phase de Magie pendant laquelle la comète n'est pas arrivée, ajoutez un marqueur au même endroit. Quand la comète s'écrase, toutes les unités situées dans un rayon de \distance{2D6+X} subissent 2D6 touches de Force 4+X, où X est égal au nombre de marqueurs. Enlevez ensuite tous les marqueurs, et le sort prend fin.
\tabularnewline
\closetable



\newbattlepath{fire}{\fireTOC}

\starttable[white]{\colors@fire}
A &
\fireattribute{} &
&
\range{24} \newline
\hex{} \newline
\missile{} \newline
\damage{} &
\instant{} &
La cible subit 1D3 touches de Force 4 avec la règle \flamingattacks{}.
\tabularnewline
\hline
0 & \firesignature{} &
\base{5+}\newline
\amel{10+} \newline
\amelbis{14+} &
\base{\range{24}} \newline
\amel{\range{36}} \newline
\amelbis{\range{48}} \newline
\hex{} \newline
\missile{} \newline
\damage{} &
\instant{} &
La cible subit \base{1D6} \amel{2D6} \amelbis{3D6} touches de Force 4 avec la règle \flamingattacks{}.
\tabularnewline
\hline
1 & \firespellone{} &
\base{7+}\newline
\amel{10+} &
\base{\range{18}} \newline
\amel{\range{36}} \newline
\hex{} &
\remainsinplay{} &
A la fin de chaque Phase de Magie, la cible subit 1D6 touches de Force 4 avec la règle \flamingattacks{}.
\tabularnewline
\hline
2 & \firespelltwo{} &
\base{7+}\newline
\amel{13+} &
\base{\range{24}} \newline
\amel{\range{6}} \newline
\amel{\aura} \newline
\augment{} &
\lastsoneturn{} &
Les Attaques de Corps à Corps et de Tir de la cible ont +1 pour blesser et gagnent les règles \flamingattacks{} et \magicalattacks{}.
\tabularnewline
\hline
3 & \firespellthree{} &
\base{9+}\newline
\amel{12+} &
\base{\range{18}} \newline
\amel{\range{36}} \newline
\ground{} \newline
\direct{} \newline
\linetemplate{} &
\instant{} &
Chaque figurine sous le gabarit subit une touche de Force 4 avec la règle \flamingattacks{}. Les unités ennemies touchées par le gabarit doivent effectuer un test de Panique.
\tabularnewline
\hline
4 & \firespellfour{} &
\base{9+}\newline
\amel{12+} &
\base{\range{24}} \newline
\amel{\range{36}} \newline
\hex{} \newline
\missile{} \newline
\damage{} &
\instant{} &
La cible subit 1 touche de Force 4 avec la règle \flamingattacks{} pour chaque rang ou colonne, au choix du lanceur, dans l'unité ciblée.
\tabularnewline
\hline
5 & \firespellfive{} &
\base{10+}\newline
\amel{13+} &
\base{\range{24}} \newline
\amel{\range{48}} \newline
\hex{} &
\remainsinplay{} &
La cible subit 1D6 touches de Force 4 avec la règle \flamingattacks{}.

\vspace*{5pt}
À la fin de chaque phase, chaque figurine de l'unité subit une touche de Force 4 avec la règle \flamingattacks{} si l'unité a effectué une ou plusieurs des actions suivantes durant la phase :

Charge (réussie ou non), Charge Irrésistible, Fuite, Marche Forcée, Mouvement Simple, Pivot, Poursuite, Reformation ou Reformation de Combat.

Une même unité ne peut subir ces touches qu'une fois par Phase.
\tabularnewline
\hline
6 & \firespellsix{} &
\base{11+}\newline
\amel{14+} &
\base{\range{24}} \newline
\amel{\range{48}} \newline
\augment{} &
\lastsoneturn{} &
La cible gagne +1 en Endurance, une Sauvegarde Invulnérable (5+) et la règle \fireborn{}.
\tabularnewline
\closetable



\newbattlepath{light}{\lightTOC}



\newbattlepath{death}{\deathTOC}



\newbattlepath{nature}{\natureTOC}



\newbattlepath{shadows}{\shadowsTOC}



\newbattlepath{wilderness}{\wildernessTOC}



\eightfoldpathtitle{eightfoldpath}{\Pathof{} \eightpaths}



\newspecificpath{butchery}{\butchery}



\newspecificpath{change}{\change}



\newspecificpath{biggreengods}{\thebiggreengods}



\newspecificpath{littlegreengods}{\thelittlegreengods}



\newspecificpath{forge}{\forge}



\newspecificpath{lust}{\lust}



\newspecificpath{whitemagic}{\whitemagic}



\newspecificpath{blackmagic}{\blackmagic}



\newspecificpath{disease}{\disease}



\newspecificpath{necromancy}{\necromancy}



\newspecificpath{ruin}{\ruin}



\newspecificpath{sands}{\sands}


\newpage
\basictitle{magicphasesummary}{\magicphasesummary}

\end{document}
