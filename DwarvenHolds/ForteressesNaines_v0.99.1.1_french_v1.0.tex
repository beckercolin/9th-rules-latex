\input{../Formatage/format_LA.tex}

\begin{document}

\newcommand{\relentless}{\specialrule{Implacable}\xspace}
\newcommand{\shieldwall}{\specialrule{Mur de Boucliers}\xspace}
\newcommand{\holdstone}{\specialrule{Pierre de Serment}\xspace}
\newcommand{\entrench}{\specialrule{Retranchement}\xspace}
\newcommand{\hewnoutofmountains}{\specialrule{Taillés dans les Montagnes}\xspace}
\newcommand{\comewithme}{\specialrule{Tu Partiras avec Moi!}\xspace}
\newcommand{\ancientgrudge}{\specialrule{Vieille Rancune}\xspace}
\newcommand{\sturdy}{\specialrule{Robuste}\xspace}
\newcommand{\biggamehunters}{\specialrule{Plus Ils Sont Gros...}\xspace}
\newcommand{\bombingrun}{\specialrule{Bombardement}\xspace}
\newcommand{\shieldskill}{\specialrule{Mur Impénétrable}\xspace}

\newcommand{\booktitle}{Forteresses Naines}
\newcommand{\version}{0.99}
\newcommand{\frenchversion}{1.0}
\newcommand{\translationteam}{\item \og AEnoriel \fg \item \og Anglachel \fg \item \og Astadriel \fg \item \og Batcat \fg \item \og Eru \fg\item \og Gandarin \fg \item \og Groumbahk \fg \item \og Iluvatar \fg \item \og Lamronchak \fg \item \og Mammstein \fg}

\input{../Formatage/titlepage_LA.tex}

\armyspecialrules


\armyspecialruleentry{\relentless}

Les unités d'infanterie contenant uniquement des figurines dotées de cette règle peuvent tripler leur mouvement lors d'une Marche Forcée au lieu de le doubler. Cette disposition est étendue aux situations où le mouvement est limité d'ordinaire au double de la valeur de mouvement de la figurine, telles que les roues, les reformations, les mouvements d'un personnage à travers une unité, etc... Le mouvement est alors également triplé.

\armyspecialruleentry{\shieldwall}

Une figurine dotée de cette règle gagne une \wardsave{6} au corps à corps lorsqu'elle utilise un Bouclier, uniquement contre les attaques de corps à corps non spéciales provenant d'unités ennemies engagées sur le front de la figurine bénéficiant de la règle \shieldwall. Cette \wardsave{} passe à 5+ durant le premier corps à corps suivant la réception d'une charge. 

\armyspecialruleentry{\holdstone}

Au début de n'importe quelle Phase de Corps à Corps, une figurine dotée d'une \holdstone peut déclarer qu'elle l'utilise. A partir de ce moment, son porteur et l'unité qui l'accompagne considèrent leurs flancs et leurs arrières comme étant de face en ce qui concerne les règles \specialrule{Attaques de Soutien}, \parry et \shieldwall. Ils peuvent néanmoins toujours être \specialrule{Désorganisés}. Tant que la \holdstone est active, le porteur doit toujours lancer et relever les défis à moins qu'une autre figurine amie ne le fasse avant. L'unité ne peut pas poursuivre dans le combat où la \holdstone a été activée. Les effets de la \holdstone disparaissent immédiatement une fois le combat terminé ou à la fin de la phase où son porteur est tué. Tant que le porteur est en vie, la Pierre peut être activée à nouveau lors des combats suivants.

\armyspecialruleentry{\biggamehunters}

Les attaques de corps à corps d'une figurine dotée de cette règle blessent toujours sur 4+ ou mieux. Les figurines gagnent également la règle \swiftstride lorsqu'elles chargent une unité contenant au moins une figurine d'un des types d'unités suivant: Bête Monstrueuse, Cavalerie Monstrueuse, Infanterie Monstrueuse, Monstre ou Monstre Monté.

\armyspecialruleentry{\entrench}

Une fois le déploiement terminé mais avant de placer les éclaireurs et de faire les mouvements \vanguard, une figurine dotée de cette règle peut retrancher une seule Machine de Guerre qui comptera comme étant à Couvert Lourd. Si la Machine de Guerre bouge pour une raison ou une autre, elle perd définitivement ce couvert.

\armyspecialruleentry{\sturdy}

Une figurine avec cette règle gagne la règle \thunderouscharge. De plus, elle ne souffre pas de la pénalité de -1 au tir pour Tenir sa Position et Tirer. 

\armyspecialruleentry{\hewnoutofmountains}

Tant qu'il reste sur la table une unité amie d'une armée Forteresses Naines, tous les sorts lancés par les figurines ennemies voient leur valeur de lancement ou leur niveau de puissance (pour les objets de sorts) augmenté de 1. Cependant, la valeur des sorts restants en jeu ne change pas pour ce qui est de leur dissipation.



\armyspecialruleentry{\comewithme}

Toute figurine dotée de cette règle peut, juste avant qu'elle soit retirée comme perte, porter immédiatement une ultime attaque de corps à corps en utilisant la même arme et toutes les règles spéciales et bonus utilisés pendant le combat. L'attaque doit être portée au choix sur la figurine adverse ayant causé la perte ou bien son unité, auquel cas cette attaque est répartie comme au tir. Chaque figurine ne peut porter qu'une seule attaque grâce à cette règle, dont la Force ne peut pas dépasser 5. Cette règle ne peut pas être utilisée par une unité de Chasseurs de Trolls avec moins de 5 figurines dans son premier rang, ni contre les pertes dues aux \impacthits{}. Par contre, les personnages isolés et les Vengeurs Nains peuvent toujours utiliser cette règle.

\armyspecialruleentry{\ancientgrudge}

Les armées des Forteresses Naines ont un nombre de \specialrule{Vieilles Rancunes} défini comme suit:
\newline-Une \ancientgrudge par Roi Nain présent dans l'armée
\newline-1D3 \ancientgrudge additionnelle pour chaque Roi Nain monté sur un Trône de Pouvoir
\newline-Une \ancientgrudge additionnelle si les Forteresses Naines combattent des figurines issues des livres d'armées Peaux-Vertes et Marée de Vermines.

Après les déploiements des deux joueurs, vous pouvez désigner une figurine ou une unité ennemie pour chaque \ancientgrudge. Toutes les unités de votre armée gagnent la règle \hatred contre ces figurines ou unités. Si lors d'un Combat, une unité disposant déjà de la \hatred affronte une unité désignée, vous pouvez relancer les jets pour blesser ratés au cours du premier round de chaque combat. Si c’est une unité qui est choisie, toutes les figurines qui la composent subissent la \hatred tant qu’elles restent dans l’unité. Si une figurine individuelle est désignée et qu'elle rejoint une unité, les effets de cette règle ne s’étendent pas à l’unité entière. Aucune unité ne peut être ciblée plus d'une fois par cette règle.

\armyarmoury

\armynewsubsection{Armes de corps à corps}

\begin{customdescription}
\item[Chaînes du Jugement :] Arme de base. Le porteur gagne +1 en Force et des \randomattacks{3D3} frappant à Initiative 10. Une figurine avec cette arme ne peut pas être rejointe par un personnage.
\end{customdescription}

\armynewsubsection{Armes de tir}

\begin{customdescription}
\item[Chalumeau :] Canon à Flammes, Portée \distance{3}, Force 3, \flamingattacks. Cette arme ne peut pas tirer après une marche forcée. Ignorez le malus de -1 sur le tableau d'Incident de Tir.
\item[Fusil de la Forge :] Arme de tir. \portee{18}, Force 5, \flamingattacks, \quicktofire.
\item[Mitraillette :] Arme de tir. \portee{24}, Force 4, \armourpiercing{1},  \multipleshots{4} et \quicktofire.
\item[Roquettes Tueuses de Wyrm :] Arme de tir. Portée \distance{24}, Force 6, \flamingattacks, \multiplewounds {1D3}{} et \reload.
\item[Sulfateuse :] Arme de tir. Portée \distance{18}, Force 5, \flamingattacks, \multipleshots{2D3}, \quicktofire.
\end{customdescription}

\armymagicitems

\armynewsubsection{Runes de Bataille}

Les Maîtres des Runes, les Forgerons Runiques et les Enclumes Ancestrales peuvent choisir, selon les règles stipulées dans leur profil, parmi les Runes de bataille suivantes. Aucune d'entre elles ne peut être choisie plus de deux fois dans une armée standard, plus d'une fois dans une patrouille et plus de quatre fois dans une grande armée. Ces Runes se lancent comme des objets de sort.

\begin{customdescription}
\item[Rune de Jugement:] Amélioration, Dure un tour. La cible, qui doit être une unité amie, peut relancer ses jets pour toucher ratés au corps à corps.
\item[Rune Luisante:] Amélioration, Dure un tour. La cible, qui doit être une unité amie, gagne les règles \hardtarget et \distracting.
\item[Rune du Métal:] Amélioration, Dure un tour. La cible peut relancer ses Sauvegardes d'Armure ratées.
\item[Rune d'Endurance:] Amélioration, Dure un tour. Tous les jets pour blesser dirigés contre l'unité amie ciblée subissent une pénalité de -1. Les effets de plusieurs Runes de Résistance ne peuvent pas se cumuler sur la même unité.
\item[Rune de Résolution:] Amélioration, Dure un tour. La cible, qui doit être une unité amie, peut effectuer un \specialrule{Mouvement Magique} de \distance{6}. 
\item[Rune du Serment:] Amélioration, Dure un tour. La cible, qui doit être une unité amie, gagne les règles \stubborn et \immunetopsychology.
\end{customdescription}

\armynewsubsection{Bannières magiques}

Les unités naines peuvent utiliser les bannières magiques communes ou bien des Bannières Runiques. Un Porteur de la Grande Bannière peut prendre une bannière runique ou une bannière magique commune dans ses points d’objets runiques.

\begin{customitemize}
	\item \optiondef{Bannière Runique de Protection}{45}{
	Toutes les unités dans un rayon de \distance{6} du porteur gagnent une \wardsave{5} contre les attaques de tir.}
	\item \optiondef{Bannière Runique de Confusion }{30}{
	Les unités chargeant l'unité brandissant cette bannière réduisent leur distance de charge de \distance{1D6}. Ceci peut les amener à rater leur charge. Jetez un dé par unité tentant de charger.}
	\item \optiondef{Bannière Runique de Hardiesse }{25}{
	L'unité du porteur gagne la règle \vanguard.}

\end{customitemize}

\armynewsubsection{Runes Naines}

Les armées des Forteresses Naines ne peuvent pas utiliser les armes magiques, armures magiques, objets cabalistiques et talismans communs. A la place, elles utilisent les reliques naines, ou forgent de puissantes runes naines sur leurs propres objets, créant de puissants objets runiques. Les objets runiques sont des objets magiques dans tous leurs aspects (par exemple, les armes runiques sont considérées comme des armes magiques, les armures runiques sont considérées comme des armures magiques, etc...).
Chaque combinaison de runes forme un objet magique unique ne pouvant pas être dupliqué.

\armynewsubsection{Runes d'armes}

N'importe quelle Arme de base ou Arme de base additionnelle peut être forgée avec un maximum de trois runes parmi les runes suivantes:

\begin{customitemize}
\item \optiondef{Rune de Pénétration}{40/30}{
\oneofakind. Les blessures infligées par cette arme suivent la règle \armourpiercing{6}.}
\item \optiondef{Rune de Destruction}{40/20}{
\oneofakind. Les blessures infligées par cette arme suivent la règle \multiplewounds{1D3}{}.}
\item \optiondef{Rune Brisante}{35/25}{
\oneofakind. Les attaques portées par cette arme le sont avec une Force de 10 contre les figurines ayant une Endurance de 5 ou plus.}
\item \optiondef{Rune de Puissance}{20}{
Pour chaque rune forgée, les blessures infligées par cette arme gagnent +1 en Force.}
\item \optiondef{Rune de l'Art de la Forge}{20/15}{
\oneofakind. Le type d'arme change pour devenir une Arme lourde.}
\item \optiondef{Rune de Célérité}{20}{
Pour chaque rune forgée, les attaques sont portées avec une initiative de +2.}
\item \optiondef{Rune de Précision}{20}{
Le porteur gagne la règle \lightningreflexes.}
\item \optiondef{Rune de Fureur}{15}{
Pour chaque rune forgée, le porteur gagne +1 Attaque.}
\item \optiondef{Rune de Feu}{5}{
L'arme gagne la règle \flamingattacks.}
\end{customitemize}

\armynewsubsection{Runes d'armure}

Tout un ensemble d'armure peut être forgé avec un maximum de trois runes parmi les runes suivantes:

\begin{customitemize}
\item \optiondef{Rune de Résistance }{45}{
\oneofakind. Les unités ennemies relancent leurs jets pour blesser réussis contre le porteur.}
\item \optiondef{Rune des Montagnes }{30}{
Le porteur gagne +1 en Endurance jusqu'à un maximum de 6.}
\item \optiondef{Rune de Bronze }{30}{
Le porteur gagne +1 Point de Vie.}
\item \optiondef{Rune d'Acier }{30}{
Le porteur peut relancer ses jets de Sauvegarde d'Armure ratés.}
\item \optiondef{Rune d'Aegis }{15}{
Le porteur gagne une \magicresistance{1}. S'il possède déjà une \magicresistance{}, chaque rune  ainsi gravée l’augmente de +1, jusqu'à un maximum de \magicresistance{4}. Ne se cumule pas avec une \magicresistance{} provenant d'une autre figurine dans l'unité.}
\item \optiondef{Rune d'Impact }{10}{
Le porteur gagne la règle \impacthits{1}. Chaque rune supplémentaire après la première ajoute +1 à ses \impacthits{}.}
\item \optiondef{Rune de Fer }{10}{
Le porteur gagne une Armure Naturelle (6+). Chaque rune après la première augmente l’Armure Naturelle de +1, jusqu’à un maximum de Armure Naturelle (5+).}
\end{customitemize}

\armynewsubsection{Runes de Talisman}

Jusqu'à trois runes parmi les suivantes peuvent être forgées sur un seul talisman runique:

\begin{customitemize}
\item \optiondef{Rune du Dragon }{35}{
Le porteur gagne une \breathweapon{Force 4, \flamingattacks et \magicalattacks}.}
\item \optiondef{Rune de Vengeance }{35}{
Une seule par armée. Une seule utilisation. Utilisable au début de n'importe quel phase phase de Mouvement. Toutes les unités amies dans un rayon de \distance{6} gagnent la règle \devastatingcharge jusqu'à la fin du tour.}
\item \optiondef{Rune de Courage }{30}{
Une seule utilisation. Peut être activée au début de n'importe quelle phase de Combat. Pour la durée de cette phase, le porteur gagne la règle \stubborn.}
\item \optiondef{Rune d'Infamie }{20}{
Le porteur gagne la règle \fear.}
\item \optiondef{Rune Bouclier }{15}{
Le porteur gagne une \wardsave{6+}. Cette Rune ne peut se cumuler qu'avec elle-même et/ou la règle \shieldwall, auquel cas elle augmente la \wardsave{} de +1, pour un maximum de 4+. La \wardsave{} ainsi obtenue peut toujours intéragir normalement avec la \magicresistance{}.}
\item \optiondef{Rune d'Empathie }{10}{
Le porteur gagne les règles \scout et \ambush. Il ne peut pas être placé sur un Trône de Guerre.}
\item \optiondef{Rune de la Forge }{5}{
Le porteur gagne la règle \fireborn.}
\end{customitemize}



\armynewsubsection{Runes cabalistiques}

Les figurines avec les règles \specialrule{Forge Runique} ou \specialrule{Maîtrise de la Forge Runique} peuvent forger trois des objets de sort runiques suivants sur un seul objet cabalistique runique:

\begin{customitemize}
\item \optiondef{Rune de Connaissance }{50}{
\oneofakind. A la fin d'une phase de magie amie, le joueur nain peut garder un dé de pouvoir inutilisé et le rajouter à ses dés lors de la prochaine phase de magie, juste après que les flux magiques aient été déterminés.
\item \optiondef{Rune de Maîtrise }{40}{
 Une seule fois par phase de Magie, le porteur peut relancer un dé de Pouvoir lorsqu'il lance un sort lié à un objet de sort runique.}
\item \optiondef{Rune de Déni }{35}{
Une seule par armée. Une seule utilisation. Au lieu de tenter de dissiper le Sort, vous pouvez utiliser cette rune. Le sort est automatiquement dissipé.}
\item \optiondef{Rune Dévoreuse de Sort }{35}{
Une seule par armée. Une seule utilisation. A la place de tenter une dissipation, vous pouvez utiliser cette rune. Le sort est lancé mais est ensuite perdu et ne pourra ensuite plus être utilisé de la partie. Une seule rune par armée. Ne peut pas être combinée avec la Rune de Déni sur le même objet cabalistique runique.}
\item \optiondef{Rune Brillante }{35}{
 Le porteur ajoute +1 à toutes ses dissipations, ainsi qu'à tous ses lancements de sort liés à un objet de sort runique.}
\item \optiondef{Rune de Harcèlement }{25}{
 Le joueur nain a un bonus de +1 pour sa tentative de \channel tandis que l'adversaire a un malus de -1 pour sa tentative de \channel . } 
\end{customitemize}


\armylist

\lordstitle


\showunit{
	name={Roi Nain},
	cost={125},
	profile={
		Roi Nain: 3 7 4 4 5 3 4 4 10
	},
	type=Infanterie,
	unitsize={1},
	basesize=20x20,
	specialrules={\relentless, \shieldwall, \sturdy},
	equipment={Armure de plates},
	options={
		Peut prendre des objets runiques=\upto : 125,
		Peut prendre une \holdstone=30,
		Peut prendre un Bouclier=10,
		\optionschoice{Peut prendre une des armes de tir suivantes}{
			Pistolet=4,
			Arbalète=8,
		    Arquebuse=8},
		Peut prendre une Arme lourde=10,
			},
	mounts={Porteurs de Bouclier=50, Trône de Guerre=150}
	}


	
\showunit{
	name={Maître des Runes},
	cost={125},
	profile={
		Maître des Runes: 3 6 4 4 5 3 3 2 9
	},
	type=Infanterie,
	unitsize={1},
	basesize=20x20,
	specialrules={ \channel, \relentless,  \specialrule{Maîtrise de la Forge Runique}, \shieldwall, \magicresistance{2}, \sturdy},
		equipment={ Armure de plates},
	options={
		Peut prendre des objets runiques=\upto : 125,			
		Peut prendre un Bouclier=3,
		Peut prendre une Arme lourde=8,
		Peut prendre jusqu'à quatre Runes de bataille différentes=5,
		},
	unitrules=\unitrule{Maîtrise de la Forge Runique}{Une figurine dotée de cette règle peut dissiper de la même façon qu'un \magiclevelmaster{3}. Cette figurine dispose également de la règle \armourpiercing{1},  qu'elle confère à toute unité qu'elle rejoint. Elle peut utiliser ses Runes de bataille de deux façons:\newline
-Niveau de Puissance 4, \portee{6}, Amélioration, Dure un tour. \newline -Niveau de Puissance 5, \portee{12}, Amélioration, Dure un tour.},
			}


\showunit{
	name={Maître-Ingénieur},
	cost={120},
	profile={
		Maître Ingénieur: 3 6 4 4 5 3 3 2 9
	},
	type=Infanterie,
	unitsize={1},
	basesize=20x20,
	specialrules={\relentless, \specialrule{Ingénieur}, \specialrule{Maître en Balistique},  \shieldwall, \entrench, \sturdy},
	unitrules=\unitrule{Maître en Balistique}{Au lieu d'utiliser sa règle \specialrule{Ingénieur}, une figurine dotée de la règle \specialrule{Maître en Balistique} peut, au début de sa Phase de Tir, octroyer un des bonus suivants à une unité d'infanterie amie située à moins de \distance{6}, uniquement pour la phase de Tir du tour en cours: \newline
-L'unité ciblée gagne un bonus de +1 pour toucher au tir.
\newline
-L'unité ciblée peut relancer les jets de dé de '1' pour blesser au tir.
\newline
-L'unité ciblée voit sa portée de tir augmenter de \distance{1D6+1}.\newline Le même bonus ne peut pas se cumuler deux fois sur la même unité.},
	equipment={Armure de plates},
	options={
		Peut prendre des objets runiques=\upto : 100,
		Peut prendre un Bouclier=3,
		\optionschoice{Peut prendre une des armes de tir suivantes}{
		Pistolet=4,
		Paire de pistolets=5, 
		Arbalète=8,
		Arquebuse=8,
		Sulfateuse=20,
		Roquettes Tueuses de Wyrm=20},
		Peut prendre une Arme lourde=8,
		},
	}
	
\showunit{
	name={Chasseur de Démons},
	cost={130},
	profile={
		Chasseur de Démons: 3 7 4 5 5 3 5 5 10
	},
	type=Infanterie,
	unitsize={1},
	basesize=20x20,
	specialrules={\vanguard, \lethalstrike, \hatred(Livre d'Armée : Légion Démoniaque), \relentless, \unbreakable, \weaponmaster,  \notaleader, \biggamehunters, \swiftstride, \sturdy, \wardsave{6}, \comewithme},
		equipment={ Paires d'armes, Arme lourde},
	options={
		Peut prendre des Runes d'armes=\upto : 125,
		\optionschoice{Peut choisir une règle spéciale parmi les suivantes}{ \specialrule{Résistance Magique (1)}=15,\specialrule{Peur}=20,
\specialrule{Tueur de Monstres}=30,
\specialrule{Volonté Funeste}=40,}
		},
	unitrules={
	\unitrule{Tueur de Monstres}{La figurine gagne la règle \multiplewounds{2}{Bête Monstrueuse, Cavalerie Monstrueuse, Infanterie Monstrueuse, Monstre et Monstre Monté.}
	\unitrule{Volonté Funeste}{Une figurine dotée de cette règle ajoute le nombre d'ennemis en contact socle à socle avec elle à son nombre d'Attaques.}
	}
	}	
	}
	
\heroestitle

\showunit{
	name={Thane},
	cost={70},
	profile={
		Thain: 3 6 4 4 5 2 3 3 9
	},
	type=Infanterie,
	unitsize={1},
	basesize=20x20,
	specialrules={\relentless, \shieldwall, \sturdy},
	equipment={Armure de plates},
	options={
		Peut prendre des objets runiques=\upto : 75,
		\optionschoice{Peut devenir (un seul choix)}{
		Chef de Clan=30,		
		Porteur de la Grande Bannière=25}
		Peut prendre une \holdstone=25,
		Peut prendre un Bouclier=8,
		\optionschoice{Peut prendre une des armes de tir suivantes}{
		Pistolet=4,
		Arbalète=8,
		Arquebuse=8},
		Peut prendre une Arme lourde=10,
		},
		unitrules=\unitrule{Thane-Guerrier}{Le Thane et tous les Guerriers Nains de son unité gagnent la règle \fightinextrarank. }
	}

\showunit{
	name={Forgeron Runique},
	cost={70},
	profile={
		Forgeron Runique: 3 5 3 4 4 2 3 2 9
	},
	type=Infanterie,
	unitsize={1},
	basesize=20x20,
	specialrules={ \channel, \specialrule{Forge Runique},  \relentless, \shieldwall,  \magicresistance{1}, \sturdy},
		equipment={ Armure de plates},
	options={
		Peut prendre des objets runiques=\upto : 75,			
		Peut prendre un Bouclier=2,
		Peut prendre une Arme lourde=6,
		Peut prendre jusqu'à deux Runes de bataille différentes=5,
		},
	unitrules=\unitrule{Forge Runique}{Une figurine dotée de cette règle peut dissiper de la même façon qu'un \magiclevelapprentice{1}. Cette figurine dispose également de la règle \armourpiercing{1}, qu'elle confère à toute unité qu'elle rejoint. Elle peut utiliser ses Runes de bataille de deux façons:\newline
-Niveau de Puissance 4, Unité du lanceur, Amélioration, Dure un tour. \newline -Niveau de Puissance 5, \portee{6}, Amélioration, Dure un tour.}
	}

\showunit{
	name={Ingénieur},
	cost={65},
	profile={
		Ingénieur: 3 5 4 4 4 2 3 2 9
	},
	type=Infanterie,
	unitsize={1},
	basesize=20x20,
	specialrules={\relentless, \engineer, \shieldwall,\entrench, \sturdy},
	equipment={Armure de plates},
	options={
		Peut prendre des objets runiques=\upto : 50,
		Peut prendre un Bouclier=2,
		\optionschoice{Peut prendre une des armes de tir suivantes}{
		Pistolet=4,
		Paire de pistolets=5,   
		Arbalète=8,
		Arquebuse=8,
		Sulfateuse=20,
		Roquettes Tueuses de Wyrm=20},
		Peut prendre une Arme lourde=6,
		},
	}

\showunit{
	name={Chasseur de Dragons},
	cost={50},
	profile={
		Chasseur de Dragons: 3 6 3 4 5 2 4 4 10
	},
	type=Infanterie,
	unitsize={1},
	basesize=20x20,
	specialrules={\vanguard, \lethalstrike, \relentless, \unbreakable, \weaponmaster, \notaleader, \biggamehunters, \swiftstride, \sturdy, \wardsave{6}, \comewithme},
		equipment={ Paires d'armes, Arme lourde},
	options={
		Peut prendre des Runes d'armes=\upto : 125,
		\optionschoice{Peut choisir une règle spéciale parmi les suivantes}{
		\specialrule{Résistance Magique (1)}=15,
		\specialrule{Tueur de Monstres}=20,
		\specialrule{Peur}=20,
		\specialrule{Volonté Funeste}=25},
		},
	unitrules={\unitrule{Tueur de Monstres}{La figurine gagne la règle \multiplewounds{2}{Bête Monstrueuse, Cavalerie Monstrueuse, Infanterie Monstrueuse, Monstre et Monstre Monté.}
	\unitrule{Volonté Funeste}{Une figurine dotée de cette règle ajoute le nombre d'ennemis en contact socle à socle avec elle à son nombre d'Attaques.}
	}
	}	
	}
	
\baseunitstitle

\showunit{
	name={Guerriers Nains},
	cost={60},
	profile={
		Guerrier Nain : 3 4 3 3 4 1 2 1 9
		},
	costpermodel=7 ,
	additionalmodels=30,
	type=Infanterie,
	unitsize={10},
	basesize=20x20,
	specialrules={\relentless, \shieldwall, \sturdy},
		equipment={Armure lourde},
	options={
		Une seule unité peut choisir la règle \vanguard (maximum 30 figurines)=\permodel :1,
		Peut prendre un Bouclier=\permodel :1,
		\optionschoice{Peut prendre une arme au choix}{
		Paire d'armes=\permodel :1,
		Lance=\permodel :1,
		Arme lourde=\permodel : 3},
		},
commandgroup={\commandgroup{champion=10,banner=10,standardbeareroption=\veteranstandardbearer *25,musician=10}},
	}
	
\showunit{
	name={Barbes-Grises},
	cost={85},
	profile={
		Barbe-Grise : 3 5 3 4 4 1 2 1 9
		},
	costpermodel=11 ,
	additionalmodels=20,
	type=Infanterie,
	unitsize={10},
	basesize=20x20,
	specialrules={\specialrule{Gromellement}, \immunetopsychology, \relentless, \shieldwall, \sturdy},
		equipment={Armure lourde},
	options={
		Si aucune unité de Guerriers Nains n'a choisi la règle \vanguard{,} une seule unité de Barbes-Grises peut choisir cette règle (maximum 20 figurines)=\permodel :1,
		Peut prendre un Bouclier=\permodel :1,
		Peut prendre des Armes de jet=\permodel : 2,
		Peut prendre une Arme lourde=\permodel : 3,
		},
commandgroup={\commandgroup{champion=10,banner=10,standardbeareroption=\veteranstandardbearer *25,musician=10}		},
unitrules=\unitrule{Gromellement}{Toute unité à \distance{6} des Barbes-Grises peut relancer ses tests de Panique ratés.},
	}		
		
\showunit{
	name={Sentinelles Naines},
	cost={120},
	profile={
		Sentinelle Naine : 3 4 3 3 4 1 2 1 9
		},
	costpermodel=12 ,
	additionalmodels=15,
	type=Infanterie,
	unitsize={10},
	basesize=20x20,
	specialrules={\relentless, \sturdy},
		equipment={Armure lourde, Arbalète},
	options={
		\optionschoice{Peut remplacer son Arbalète}{
		Arquebuse= \free},
		Peut prendre un Bouclier=\permodel :1,
		Peut prendre une Arme lourde=\permodel : 3,
		},
	commandgroup=\commandgroup{
	champion=10,
	banner=10,
	musician=10}
	}

	
\specialunitstitle

\showunit{
	name={Gardes des Souterrains},
	cost={110},
	profile={
		Garde des Souterrains : 3 5 3 4 4 1 2 1 9
		},
	costpermodel=15 ,
	additionalmodels=20,
	type=Infanterie,
	unitsize={10},
	basesize=20x20,
	specialrules={ \bodyguard{}, \relentless, \shieldwall, \shieldskill,  \sturdy},
		equipment={Armure de plates, Bouclier},
commandgroup={\commandgroup{champion=10,banner=10,bannerallowance =50,musician=10}},
unitrules=\unitrule{\shieldskill}{La \wardsave{} accordée par la règle \shieldwall est toujours de 5+.}
	}
	

\showunit{
	name={Marteliers Royaux},
	cost={130},
	profile={
		Martelier Royal : 3 5 3 4 4 1 2 2 9
		},
	costpermodel=16 ,
	additionalmodels=20,
	type=Infanterie,
	unitsize={10},
	basesize=20x20,
	specialrules={\bodyguard{Général, Roi Nain}, \relentless, \sturdy},
		equipment={Armure de plates, Arme lourde},
	options={
		Peut prendre un Bouclier=\permodel :1,
		},
commandgroup={\commandgroup{champion=10,banner=10,bannerallowance =50,musician=10}},
	}	


\showunit{
	name={Veilleurs des Forges},
	cost={150},
	profile={
		Veilleur des Forges : 3 5 3 4 4 1 2 1 9
		},
	costpermodel=15 ,
	additionalmodels=15,
	type=Infanterie,
	unitsize={10},
	basesize=20x20,
	specialrules={\relentless, \fireborn, \sturdy, \wardsave{6}},
		equipment={Armure de plates, Fusil de la Forge},
commandgroup={\commandgroup{champion=10,championoption=Roquettes Tueuses de Wyrm:20,banner=10,bannerallowance =25,musician=10}	},
unitrules=\unitrule{Fusil de la Forge}{Arme de tir. \portee{18}, Force 5, \flamingattacks, \quicktofire .},
	}	



\showunit{
	name={Chasseurs de Trolls},
	cost={55},
	profile={
		Chasseur de Trolls : 3 4 3 4 4 1 2 1 10
		},
	costpermodel=11 ,
	additionalmodels=25,
	type=Infanterie,
	unitsize={5},
	basesize=20x20,
	specialrules={\relentless, \unbreakable, \weaponmaster, \biggamehunters, \sturdy, \wardsave{6}, \comewithme},
		equipment={ Arme de base additionnelle, Arme lourde},
	options={
		Peut choisir la règle \vanguard =\permodel : 2,
		Une seule unité peut choisir la règle \specialrule{Tirailleurs} (\oneofakind  - 15 figurines maximum)=\permodel : 2,
		},
commandgroup={\commandgroup{champion=10,banner=10,bannerallowance =25,musician=10}},
	}	

\showunit{
	name={Autogyre à Vapeur},
	cost={-},
	profile={
		Autogyre à Vapeur :1 - - - 5 3 - - -,
		Pilote : - 4 3 4 4 - 2 2 9,
		},
	type=Cavalerie,
	unitsize={1},
	basesize=40x40,
	specialrules={\fly{8}},
		equipment={Armure lourde, Chalumeau \only{Autogyre de Reconnaissance}, \mountprotection{6}, Sulfateuse \only{Autogyre d'Attaque}},
		options={
	\optionschoice{L'Autogyre à Vapeur doit choisir entre deux configurations}
	{\textbf \newline \textbf{Autogyre d'Attaque}=70,
{\optionschoice{Peut recevoir}
{Jusqu'à deux figurines supplémentaires=\permodel : 65,
La règle \bombingrun=30,
Si l'unité comprend au moins deux figurines{,} elles peuvent devenir \skirmishers=\permodel : 5,}
}
\textbf \newline \textbf{Autogyre de Reconnaissance}=85,
{
}
},
 	},
		unitrules={
\unitrule{\bombingrun}{\specialrule{Attaques au Passage}. L'unité ennemie subit 2D6 touches de Force 3 avec la règle \armourpiercing{1}. Le nombre de touches n'augmente pas proportionnellement avec le nombre de figurines de l'unité }
\unitrule{Sulfateuse}{Arme de tir. Portée \distance{18}, Force 5, \flamingattacks, \multipleshots{2D3}, \quicktofire.
}	
\unitrule{Chalumeau}{Canon à Flammes, Portée \distance{3}, Force 3, \flamingattacks. Cette arme ne peut pas tirer après une marche forcée. Ignorez le malus de -1 sur le tableau d'Incident de Tir.}
	},
	}
	
\showunit{
	name={Mineurs},
	cost={100},
	profile={
		Mineur : 3 4 3 4 4 1 2 1 9
		},
	costpermodel=10 ,
	additionalmodels=10,
	type=Infanterie,
	unitsize={10},
	basesize=20x20,
	specialrules={\ambush, \relentless, \sturdy},
		equipment={Armure lourde},
	options={
		Peut prendre un Bouclier=\permodel :1,
		Peut prendre des Armes de jet=\permodel : 2,
		Peut prendre un Pistolet=\permodel : 2,
		Peut prendre une Arme lourde=\permodel : 2},
commandgroup={\commandgroup{champion=10,banner=10,musician=10}},
	}	
	
\showunit{
	name={Artillerie Naine},
	cost={-},
	profile={
		Artillerie Naine : - - - - 7 3 - - -,
		Servants (3) : 3 4 3 3 4 - 2 1 9,
		},
	type=Machine de Guerre,
	unitsize={1},
	basesize=60,
	specialrules={\stubborn},
		equipment={Armure lourde},
		options={
	\optionschoice{Toute Artillerie Naine peut recevoir }
	{\textbf{Tir Enflammé}=5,
Donne les règles \flamingattacks et \magicalattacks aux tirs de l'Artillerie Naine ,
\textbf{Rune d'Ingéniérie}=15,
Ajoute +4 aux jets de dés sur le tableau des incidents de tir.\newline}	
	\optionschoice{L'Artillerie Naine doit choisir entre trois configurations}
	{\textbf \newline \textbf{Catapulte Naine}=90,
{Maximum deux exemplaires par armée standard, une seule par patrouille et quatre par grande armée.\newline Catapulte, \range{60}, Gabarit \distance{3}, Force 3(9), \multiplewounds{\ordnance}{}.
\newline \optionschoice{Peut recevoir}
{\textbf{Forgée de Runes}=50,
Toutes les tirs de la Catapulte Naine ont +1 en Force et suivent les règles \magicalattacks et \armourpiercing{1}.}
}
\textbf \newline \textbf{Canon Nain}=100,
{Maximum deux exemplaires par armée standard, un seul par patrouille et quatre par grande armée.\newline Un canon Nain peut tirer de deux façons:
\newline- Comme un Canon (1D6), \portee{60}, Force 10, \multiplewounds{\ordnance}{}, \armourpiercing{2}. 
\newline- Comme une Batterie de tir{,} \portee{12}, Force 4, \armourpiercing{3}, \multipleshots{2D6}.\newline \optionschoice{Peut recevoir}
{\textbf{Forgé de Runes}=20,
Toutes les tirs du Canon Nain ont +1 en Force (maximum 10) et suivent la règle \magicalattacks.}
},
\textbf \newline \textbf{Canon Orgue}=125,
{Maximum deux exemplaires par armée standard, un seul par patrouille et quatre par grande armée.\newline Batterie de tir{,} \portee{30}, Force 5, \armourpiercing{1}, \multipleshots{2D6X2}.
\newline \optionschoice{Peut recevoir}
{\textbf{Forgé de Runes}=45,
Accorde +1 pour blesser et la règle \magicalattacks aux tirs du Canon Orgue.}
}
}
 	}
	}
	
	

\rareunitstitle

\showunit{
	name={Vengeur Nain},
	cost={60},
	profile={
		Vengeur Nain : 3 5 3 4 4 2 10 * 10
		},
	type=Infanterie,
	unitsize={1},
	basesize=20x20,
	specialrules={\hardtarget, \distracting, \relentless, \unbreakable, \biggamehunters, \wardsave{6}, \comewithme},
	equipment={Chaînes du Jugement},
	unitrules={\unitrule{Chaînes du Jugement}{Arme de base. Le porteur gagne +1 en Force et des \randomattacks{3D3} frappant à Initiative 10. Une figurine avec cette arme ne peut pas être rejointe par un personnage.}},
	}
	
\showunit{
	name={Rangers Nains},
	cost={65},
	profile={
		Ranger Nain : 3 4 4 3 4 1 2 1 9
		},
	costpermodel=10 ,
	additionalmodels=15,
	type=Infanterie,
	unitsize={5},
	basesize=20x20,
	specialrules={\scout, \strider{forêt}, \relentless, \sturdy},
		equipment={Armure lourde},
	options={
		Peut devenir \skirmisher (maximum 10 figurines)=\permodel :2,
		Peut prendre un Bouclier=\permodel :1,
		Peut prendre des Armes de jet=\permodel :1,
		Peut prendre une Arbalète=\permodel : 2,
		\optionschoice{Peut prendre une arme au choix}{
		Paire d'armes=\permodel :1,
		Arme lourde=\permodel : 2},
		},
commandgroup={\commandgroup{champion=10,banner=10,musician=10}},
	}			

\showunit{
	name={Gardiens des Forts},
	cost={155},
	profile={
		Gardien des Forts : 5 4 3 6 5 3 2 2 10
		},
	costpermodel=60 ,
	additionalmodels=5,
	type=Infanterie Monstrueuse,
	unitsize={3},
	basesize=40x40,
	specialrules={\magicalattacks, \multiplewounds{1D3}{}, \immunetopsychology},
		equipment={Armure de plates, \innatedefence{6}},
	options={
		Toutes les figurines de l'unité peuvent gagner les règles \flamingattacks et \fireborn=\permodel : 5,
		},
commandgroup={\commandgroup{champion=10,banner=10,musician=10},		},
	}	

\showunit{
	name={Briseur de Rancunes},
	cost={165},
	profile={
		Briseur de Rancunes : 1 - - 5 5 5 2 - -,
		Pilotes (2) : - 4 3 4 4 - 2 2 9,
		},
	type=Char,
	unitsize={1},
	basesize=50x100,
	specialrules={\grindingattacks{3D3} \only{Briseur de Rancunes uniquement}, \impacthits{+1}, \fly{8}},
		equipment={\innatedefence{3}, Mitraillette},
		unitrules=\unitrule{Mitraillette}{Arme de tir. \portee{24}, Force 4, \armourpiercing{1},  \multipleshots{4} et \quicktofire.
		},
	}				
	
\showunit{
	name={Artillerie Venge-Rancunes},
	cost={-},
	profile={
		Artillerie Venge-Rancunes : - - - - 7 3 - - -,
		Servants (3) : 3 4 3 3 4 - 2 1 9,
		},
	type=Machine de Guerre,
	unitsize={1},
	basesize=60,
	specialrules={\stubborn},
		equipment={Armure lourde},
		options={
	\optionschoice{Toute Artillerie Venge-Rancunes peut recevoir }
	{\textbf{Rune d'Ingéniérie}=15,
Ajoute +4 aux jets de dés sur le tableau des incidents de tir.\newline}	
	\optionschoice{L'Artillerie Venge-Rancunes doit choisir entre deux configurations}
	{
\textbf \newline \textbf{Baliste Naine}=55,
{Baliste, \portee{48}, Force 6, \multiplewounds{1D3}{}, \armourpiercing{6}.
\newline \optionschoice{Peut recevoir}
{\textbf{Tir Enflammé}=5,
Donne les règles \flamingattacks et \magicalattacks aux tirs de la Baliste Naine.,
\textbf{Forgé de Runes}=10,
Les tirs de la Baliste Naine sont des \magicalattacks et ont +1 pour toucher les unités suivant la règle \fly{}.}
},
\textbf \newline \textbf{Lance-Flammes Nain}=110,
{Canon à flammes. \portee{12}, Force 5, \flamingattacks, \multiplewounds{1D3}{}.
\newline \optionschoice{Peut recevoir}
{\textbf{Forgé de Runes}=5,
Donne la règle \magicalattacks aux tirs du Lance-flammes Nain. 
\newline \textbf{Frappé de Runes}=20,
La Portée du Lance-flammes Nain est augmentée de \distance {3} et ses tirs gagnent la règle \magicalattacks.}
}
}
 	}
	}

	
	\showunit{
	name={Enclume Ancestrale},
	cost={150},
	profile={
		Enclume : - - - - 7 3 - - -,
		Gardes de l'Enclume (4) : 3 5 3 4 4 - 2 1 9
		},
	type=Machine de Guerre,
	unitsize={1},
	basesize=60,
	specialrules={\channel, \specialrule{Enclume Runique}, \specialrule{Indémoralisable}, \magicresistance{2}, \wardsave{5}},
		equipment={Armure de plates},
		options={
		\optionschoice{L'enclume Ancestrale peut choisir jusqu'à trois runes parmi les runes suivantes}
		{N'importe quelle Rune de Bataille=\free,		
		Rune d'Apaisement=\free,
		Rune des Tempêtes=5,
		Rune de la Terre Brisée=5},
		},
unitrules=\unitrule{Enclume Runique}{L'Enclume Ancestrale peut dissiper de la même façon qu'un \magiclevelapprentice{1}. Elle peut choisir jusqu'à trois Objets de sort comme indiqué dans les options. Tout objet de sort lancé à partir de l'Enclume Ancestrale a une \portee{36}. Les Runes de bataille sont lancés avec un niveau de puissance de 4 et les Runes d'Enclume avec un Niveau de Puissance de 5.
\item[Rune d'Apaisement:] Rune d'Enclume. Universel, Immédiat. Le joueur contrôlant le lanceur retire du jeu un sort affectant, au choix, une unité ennemie ou amie. Le sort choisi doit être de type Dure un tour ou Reste en jeu. 
\item[Rune des Tempêtes :] Rune d'Enclume. Malédiction, Dégâts, Dure un tour. La cible subit 1D6 touches de Force 6 avec les règles \lightningattacks et \magicalattacks. La cible ne peut plus utliser la règle \specialrule{Vol} pendant la durée du sort.
\item[Rune de la Terre Brisée:] Rune d'Enclume. Malédiction, Dégâts, Dure un tour. La cible subit 2D6 touches de Force 4 avec la règle \specialrule{Attaques Magiques}. Toute unité subissant au moins une blessure avec cet objet de sort runique subit un malus de -1 sur ses jets pour toucher au corps à corps et traite tous les terrains, y compris les \specialrule{Terrains Dégagés}, comme étant des \specialrule{Terrains Dangereux}.
 },
	}	
	
	


	
		
\mountstitle

\showunit{
	name={Porteurs de Bouclier},
	profile={
		Porteurs de Bouclier : 3 5 3 4 4 4 2 2 10
		},
	cost={-},	
	type=Infanterie,
	unitsize={1},
	basesize=40x20,
	specialrules={\relentless, \sturdy},
	equipment={\mountprotection{5}},
		}


\showunit{
	name={Trône de Guerre},
	profile={
		Trône de Guerre : 3 5 3 4 4 6 2 4 10
		},
	cost={-},	
	type=Infanterie,
	unitsize={1},
	basesize=40x60,
	specialrules={\hatred, \specialrule{Majesté des Hauts-Rois}, \sturdy},
	equipment={\mountprotection{5}},
	unitrules={\unitrule{Majesté des Hauts-Rois}{Un personnage monté sur un Trône de Guerre augmente la Portée de sa règle \specialrule{Présence Charismatique} à \distance{18} s'il est le Général de l'armée. De plus, il peut désigner 1D3+1 figurines/unités dans le cadre de la règle \specialrule{Rancunier}. Toutes les unités amies se trouvant dans un rayon de \distance{6} du Trône de Guerre gagnent la règle \swiftstride. Si un Personnage monté sur Trône de Guerre ayant rejoint une unité est tué, l'unité où il se trouvait au moment de sa mort gagne la règle \frenzy. Si un Trône de Guerre est présent dans l'armée, les Artilleries Naines comptent comme des choix Rares.}
		}	
		}	

	
\end{document}

