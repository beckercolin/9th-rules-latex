
\documentclass[a4paper,8pt]{extarticle} % extarticle allows to use font size of 8pt.

\usepackage[a4paper, top=1.6cm, bottom=2cm, left=1.6cm, right=1.6cm]{geometry} % Marge reduction.

%% Language specific package
\usepackage[french]{babel}
\frenchbsetup{StandardLists=true} % Necessary to use enumitem with babel/french.

%% Font and typing packages
\usepackage{fontspec}
\setmainfont[
	Ligatures=TeX,
	ItalicFont={Dancing Script},
	BoldItalicFont={Dancing Script}
	]{PT Serif} % default is Latin Modern
\newfontfamily\antiquefont[Ligatures=TeX]{Caslon Antique} % fancy font
\usepackage{microtype}			% Greatly improves general appearance of the text.
\usepackage{SIunits}			% Unit appearance.
\usepackage{xspace}				% Define commands that appear not to eat spaces.
\usepackage{ulem}				% To cross words out. Use \sout{}.

%% Array utilities
\usepackage{array}				% Additionnal options for arrays.
\usepackage{colortbl}			% Additionnal options for coloring arrays.
\usepackage[table]{xcolor}		% Auto alternate grey-white rows.
\usepackage[export]{adjustbox}		% Centered pics in tables

%% List utilities
\usepackage[inline]{enumitem}   % Display inline lists.
\usepackage{etoolbox}           % General utility. Good for lists for instance.
\usepackage{xparse}             % List utilities.
\usepackage{datatool}	% Handling alphabetical order.

%% Frames
\usepackage{framed}				% Boxes.
\usepackage[framemethod=TikZ]{mdframed}% For fancy frames.
\usepackage{tikz}				% For fancy frames.
\usepackage{wrapfig}			% Fancy insertion of pics in text.

%% Page utilities
\usepackage{multicol}			% Allows to divide a part of the page in multiple columns.
	
%% Others
\usepackage{keyval}             % Used to create maps of commands/labels/objects.
	\makeatletter                  % Mandatory for the usage of keyval.
\usepackage{xstring}            % String parsing, cutting, etc.
\usepackage{hyperref} % Links in PDF.


%%% Update of the dotfill command to always get dots

\newcommand{\predotfill}{\penalty0\hbox{}\nobreak}%


%%% Command to avoid typing \xspace when creating a new name macro

\newcommand{\newnamemacro}[2]{\newcommand{#1}{#2}} % \xspace removed for compatibility with alphabetical ordering

%%% Language specific stuff


%%% Commands %%%

\newcommand{\addtosortedlist}[1]{%
	\protected@edef\textarg{#1}%
	\protected@edef\textwithoutspaces{\expandafter\removespaces\expandafter{\textarg}}%
	\substitute\textwithoutspaces{É}{e}% Most used special characters of the language, and equivalent for alphabetical ordering
	\substitute\textwithoutspaces{È}{e}%
	\substitute\textwithoutspaces{Ê}{e}%
	\substitute\textwithoutspaces{é}{e}%
	\substitute\textwithoutspaces{è}{e}%
	\substitute\textwithoutspaces{ê}{e}%
	\substitute\textwithoutspaces{À}{a}%
	\substitute\textwithoutspaces{à}{a}%
	\substitute\textwithoutspaces{ù}{u}%
	\expandafter\sortitem\expandafter[\textwithoutspaces]{#1}%
}%


%%% Labels %%%

% Profile

\newcommand{\labels@M}{M}
\newcommand{\labels@WS}{CC}
\newcommand{\labels@BS}{CT}
\newcommand{\labels@S}{F}
\newcommand{\labels@T}{E}
\newcommand{\labels@W}{PV}
\newcommand{\labels@I}{I}
\newcommand{\labels@A}{A}
\newcommand{\labels@Ld}{Cd}
\newcommand{\labels@Invocation}{Invocation} % For Vampire Covenant profiles

\newcommand{\Strength}{Force}

% Technical

\newcommand{\labels@range}{Portée}
\newcommand{\labels@point}{pt}
\newcommand{\labels@points}{pts}
\newcommand{\labels@only}{uniquement}
\newcommand{\labels@magic}{Magie}
\newcommand{\labels@pathsused}{Génère ses sorts dans la Discipline}
\newcommand{\labels@model}{figurine}
\newcommand{\labels@models}{figurines}
\newcommand{\labels@Singlemodel}{Figurine \textbf{seule}}

% Unit entry labels

\newcommand{\labels@basesize}{Socle}
\newcommand{\labels@trooptype}{Type de troupe}
\newcommand{\labels@specialrules}{Règles spéciales}
\newcommand{\labels@alignment}{Allégeance}
\newcommand{\labels@equipment}{Équipement}
\newcommand{\labels@weapons}{Armes}
\newcommand{\labels@armour}{Armure}
\newcommand{\labels@options}{Options}
\newcommand{\labels@commandgroup}{État-Major}
\newcommand{\labels@mounts}{Montures}
\newcommand{\labels@specialequipment}{Équipement spécial}

% Command groups

\newcommand{\labels@champion}{Champion}
\newcommand{\labels@standardbearer}{Porte-étendard}
\newcommand{\labels@musician}{Musicien}
\newcommand{\labels@singlebannerallowance}{Une seule unité de ce type peut prendre une Bannière magique}
\newcommand{\labels@condsinglebannerallowance}{Une seule unité de ce type peut prendre une Bannière magique si}
\newcommand{\labels@bannerallowance}{Peut prendre une Bannière Magique}
\newcommand{\labels@veteranstandardbearer}{Peut devenir Porte-étendard Vétéran}
\newcommand{\labels@championallowance}{Peut prendre une Arme Magique}

% Titles

\newcommand{\labels@lords}{Seigneurs}
\newcommand{\labels@heroes}{Héros}
\newcommand{\labels@coreunits}{Unités de base}
\newcommand{\labels@specialunits}{Unités spéciales}
\newcommand{\labels@rareunits}{Unités rares}
\newcommand{\labels@armywiderules}{Règles communes de l'armée}
\newcommand{\labels@armyspecialrules}{Règles spéciales de l'armée}
\newcommand{\labels@armoury}{Armurerie}
\newcommand{\labels@magicalitems}{Objets magiques}
\newcommand{\labels@magicalweapons}{Armes magiques}
\newcommand{\labels@magicalarmour}{Armures magiques}
\newcommand{\labels@talismans}{Talismans}
\newcommand{\labels@enchanteditems}{Objets enchantés}
\newcommand{\labels@arcaneitems}{Objets cabalistiques}
\newcommand{\labels@magicalbanners}{Bannières magiques}
\newcommand{\labels@quickrefsheet}{Fiche de référence}
\newcommand{\labels@changelog}{Change Log}

\newcommand{\labels@lordsInitial}{S}
\newcommand{\labels@heroesInitial}{H}
\newcommand{\labels@coreunitsInitial}{B}
\newcommand{\labels@specialunitsInitial}{S}
\newcommand{\labels@rareunitsInitial}{R}
\newcommand{\labels@mountsInitial}{M}


% Titlepage

\newcommand{\labels@fantasybattles}{Batailles Fantastiques}
\newcommand{\labels@NinthAge}{Le 9\ieme Âge}
\newcommand{\labels@creators}{Une collaboration des créateurs de l'ETC et du Swedish Comp System}
\newcommand{\labels@introduction}{%
\noindent {\Largerfontsize\textbf{Note des traducteurs}}
\vspace{0.5cm}

Nous souhaitons remercier chaleureusement l'équipe à l'initiative du 9\ieme Âge pour leur motivation et leur travail continu pour faire vivre notre passion. Nous espérons que ce jeu saura développer les qualités pour plaire au plus grand nombre et réunir les joueurs, amateurs comme habitués des tournois, autour de règles amusantes et équilibrées, pour finalement s'imposer comme un standard du jeu de figurines. Une grande ambition qui ne pourra s'accomplir que \textbf{grâce à vous}, la communauté, via des retours constructifs, afin de modeler le jeu selon nos désirs. N'étant \textbf{en aucun cas à but lucratif}, le 9\ieme Âge part avec un avantage considérable. Les règles des éventuelles nouvelles sorties ne seront pas dictées par le besoin de vendre ces nouveautés. Vous pouvez choisir et acheter vos figurines où bon vous semble, il n'y a pas un unique revendeur toléré. Vous n'êtes pas bloqués dans une spirale infernale où pour continuer à jouer à un jeu, dans lequel vous vous êtes tant investis, vous devez payer toujours plus cher pour entretenir votre collection. Enfin, vous pouvez être assurés que tant que 9\ieme Âge sera joué, vous disposerez d'un \textbf{support continu et régulier}, celui-ci étant offert par la communauté.

Nous attirons votre attention sur le fait que ce jeu en est encore à ses débuts et dans un \textbf{stade de développement}. Ce document correspond à une version de brouillon \textbf{\og{} beta \fg{}}, dont le but et de tester le jeu et le modifier jusqu'à atteindre une version satisfaisante. Attendez-vous donc à trouver des déséquilibres, des incohérences, et à obtenir des mises à jour régulières avec éventuellement des changements importants. N'hésitez pas à nous donner vos avis ! Ce livre d'armée n'est utilisable qu'en compagnie du livre de Règles et du livre de Magie.

Concernant la traduction en elle-même, nous avons fait de notre mieux pour vous offrir une version de qualité, dont nous espérons qu'elle surpasse celle de la version originale ! Si vous constatez des coquilles, des erreurs, merci de nous les signaler en nous contactant sur le forum du 9\ieme Âge, dans le \textbf{sous-forum français} (\url{http://www.the-ninth-age.com/index.php?board/117-french/}). Vous y trouverez aussi les dernières mises à jour. \textbf{En cas de conflit d'interprétation avec la version originale, la version originale fait référence}.

\vspace{0.5cm}
Que ce jeu vous apporte d'innombrables heures de plaisir partagé !

\vspace{0.7cm}
\noindent {\Largerfontsize\textbf{Les traducteurs}}
\vspace{0.1cm}

\ifdef{\translationteam}{
	\begin{multicols}{3}
	\begin{itemize}
		\translationteam
	\end{itemize}
	\end{multicols}
}{}
}
\newcommand{\labels@secondpageannouncement}{%
	\labels@fantasybattles{} : \labels@NinthAge{} est un jeu créé et entretenu par la communauté qui met en scène des affrontements de figurines. Toutes les règles sont disponibles gratuitement sur le site suivant. Vos retours et suggestions sont les bienvenus.
	\newline\url{http://www.the-ninth-age.com/}
}
\newcommand{\labels@rulechanges}{%
	Les changements de règles entre versions sont colorés comme ce paragraphe. Une liste en anglais de ces changements par version est ajoutée à la fin de cet ouvrage.
}
\newcommand{\labels@latexcredit}{Document réalisé à l'aide de \LaTeX .}


%%% Technical commands

\newcommand{\only}[1]{(#1 uniquement)}
\newcommand{\free}{gratuit}
\newcommand{\upto}{jusqu'à}
\newcommand{\Upto}{Jusqu'à}
\newcommand{\unlimited}{sans limite de pts}
\newcommand{\permodel}{/fig.}
\newcommand{\listlastchoice}{ ou}
\newcommand{\notif}[1]{(pas #1)}
\newcommand{\wordand}{et}
\newcommand{\wordwith}{avec}
\newcommand{\ifNmodelsorless}[1]{(#1 figurines ou moins)}
\newcommand{\unitwith}{unité avec}
\newcommand{\From}{De} % From ... to ... models
\newcommand{\wordto}{à}
\newcommand{\wordAll}{Tous}
\newcommand{\spacebeforecolon}{ } % French put a space before colons
\newcommand{\minprice}{Coût min. :}
\newcommand{\mincostfor}{Coût min. pour}
\newcommand{\maxunitsize}{Taille max.}
\newcommand{\additionalfigscost}{Les figurines additionnelles coûtent}


%%% Special rules %%%

\newcommand{\ambush}{Embuscade}
\newcommand{\armourpiercing}[1]{Perforant\ifblank{#1}{}{ (#1)}}
\newcommand{\bodyguard}[1]{Garde du Corps\ifblank{#1}{}{ (#1)}}
\newcommand{\breathweapon}[1]{Attaque de Souffle\ifblank{#1}{}{ (#1)}}
\newcommand{\channel}{Canalisation}
\newcommand{\crushattack}{Attaque Écrasante}
\newcommand{\devastatingcharge}{Charge Dévastatrice}
\newcommand{\distracting}{Distrayant}
\newcommand{\engineer}{Ingénieur}
\newcommand{\ethereal}{Éthéré}
\newcommand{\fastcavalry}{Cavalerie Légère}
\newcommand{\fear}{Peur}
\newcommand{\fightinextrarank}{Combat avec un Rang Supplémentaire}
\newcommand{\fireborn}{Né du Feu}
\newcommand{\flamingattacks}{Attaques Enflammées}
\newcommand{\flammable}{Inflammable}
\newcommand{\lighttroops}{Troupes Légères}
\newcommand{\frenzy}{Frénésie}
\newcommand{\fly}[1]{Vol\ifblank{#1}{}{ (#1)}}
\newcommand{\grindingattacks}[1]{Attaques de Broyage\ifblank{#1}{}{ (#1)}}
\newcommand{\hardtarget}{Camouflé}
\newcommand{\hatred}{Haine}
\newcommand{\hellfire}{Flammes de l'Enfer}
\newcommand{\hidden}{Caché}
\newcommand{\holyattacks}{Attaques Divines}
\newcommand{\immunetopsychology}{Immunisé à la Psychologie}
\newcommand{\impacthits}[1]{Touches d'Impact\ifblank{#1}{}{ (#1)}}
\newcommand{\insignificant}{Insignifiant}
\newcommand{\largetarget}{Grande Cible}
\newcommand{\lethalstrike}{Coup Fatal}
\newcommand{\lightningattacks}{Attaques Foudroyantes}
\newcommand{\lightningreflexes}{Réflexes Foudroyants}
\newcommand{\magicresistance}[1]{Résistance à la Magie\ifblank{#1}{}{ (#1)}}
\newcommand{\magicalattacks}{Attaques Magiques}
\newcommand{\metalshifting}{Fusion du Métal}
\newcommand{\moveorfire}{Mouvement ou Tir}
\newcommand{\multipleshots}[1]{Tirs Multiples\ifblank{#1}{}{ (#1)}}
\newcommand{\multiplewounds}[2]{Blessures Multiples\ifblank{#1}{}{ (#1\ifblank{#2}{)}{, #2)}}}
\newcommand{\notaleader}{Pas un Meneur}
\newcommand{\otherworldly}{D'Outre-Monde}
\newcommand{\pathmaster}[1]{Maître de la Discipline\ifblank{#1}{}{ (#1)}}
\newcommand{\poisonedattacks}{Attaques Empoisonnées}
\newcommand{\quicktofire}{Tir Rapide}
\newcommand{\randommovement}[1]{Mouvement Aléatoire\ifblank{#1}{}{ (#1)}}
\newcommand{\randomattacks}[1]{Attaques Aléatoires\ifblank{#1}{}{ (#1)}}
\newcommand{\regeneration}[1]{Régénération\ifblank{#1}{}{ (#1+)}}
\newcommand{\reload}{Rechargez !}
\newcommand{\requirestwohands}{Arme à deux Mains}
\newcommand{\scythes}{Faux}
\newcommand{\scout}{Éclaireur}
\newcommand{\scouts}{Éclaireurs}
\newcommand{\stomp}[1]{Piétinement\ifblank{#1}{}{ (#1)}}
\newcommand{\strider}[1]{Guide\ifblank{#1}{}{ (#1)}}
\newcommand{\stubborn}{Tenace}
\newcommand{\stupidity}{Stupidité}
\newcommand{\skirmisher}{Tirailleur}
\newcommand{\skirmishers}{Tirailleurs}
\newcommand{\sweepingattack}{Attaque au Passage}
\newcommand{\swiftstride}{Rapide}
\newcommand{\thunderouscharge}{Charge Tonitruante}
\newcommand{\terror}{Terreur}
\newcommand{\toxicattacks}{Attaques Toxiques}
\newcommand{\unbreakable}{Indémoralisable}
\newcommand{\undead}{Mort-Vivant}
\newcommand{\unstable}{Instable}
\newcommand{\unwieldy}{Encombrant}
\newcommand{\vanguard}{Avant-Garde}
\newcommand{\volleyfire}{Tir de Volée}
\newcommand{\warplatform}{Plateforme de Guerre}
\newcommand{\wardsave}[1]{Sauvegarde Invulnérable\ifblank{#1}{}{ (#1+)}}
\newcommand{\weaponmaster}{Maître d'Ar\-mes}
\newcommand{\wizardconclave}[1]{Conclave de Sorciers\ifblank{#1}{}{ (#1)}}


%%% Magic %%%

\newnamemacro{\Pathof}{Discipline}

\newcommand{\battle}{Commune}
\newcommand{\alchemy}{de l'Alchimie}
\newcommand{\death}{de la Mort}
\newcommand{\fire}{du Feu}
\newcommand{\heavens}{des Cieux}
\newcommand{\light}{de la Lumière}
\newcommand{\nature}{de la Nature}
\newcommand{\shadows}{des Ombres}
\newcommand{\wilderness}{de la Sauvagerie Bestiale}
\newcommand{\butchery}{de la Boucherie}
\newcommand{\change}{du Changement}
\newcommand{\thebiggreengods}{des Grands Dieux Verts}
\newcommand{\thelittlegreengods}{des Petits Dieux Verts}
\newcommand{\blackmagic}{de la Magie Noire}
\newcommand{\disease}{de la Maladie}
\newcommand{\lust}{de la Luxure}
\newcommand{\necromancy}{de la Nécromancie}
\newcommand{\ruin}{de la Ruine}
\newcommand{\forge}{de la Forge}
\newcommand{\sands}{des Sables}
\newcommand{\whitemagic}{de la Magie Blanche}

\newcommand{\anyofthebattlemagic}{dans n'importe laquelle des Disciplines Communes}

\newcommand{\magiclevel}[1]{\ifnumcomp{#1}{<}{3}{Sorcier Apprenti}{Maître Sorcier} Niveau #1}
\newcommand{\Level}{Niveau}

\newcommand{\wizard}{Sorcier}
\newcommand{\wizards}{Sorciers}

\newcommand{\boundspell}[1]{Objet de Sort, Puissance #1}


%%% Other rules %%%

\newcommand{\armoursave}{Sauvegarde d'Armure}
\newcommand{\firstinrank}{Au Premier Rang}
\newcommand{\hardcover}{Couvert Lourd}
\newcommand{\holdyourground}{Tenez les Rangs}
\newcommand{\inspiringpresence}{Présence Charismatique}
\newcommand{\lightcover}{Couvert Léger}
\newcommand{\monstrousrank}{Rang Monstrueux}
\newcommand{\ordnance}{Artillerie}
\newcommand{\parry}{Parade}
\newcommand{\raisewounds}{Ressusciter des Figurines}
\newcommand{\recoverwounds}{Récupérer des PVs}
\newcommand{\aideddispel}{Dissipation Assistée}
\newcommand{\rnf}{ordinaires}
\newcommand{\general}{Général}


%%% Equipment %%%

\newcommand{\innatedefence}[1]{Protection Innée\ifblank{#1}{}{~(#1+)}}
\newcommand{\mountsprotection}[1]{Protection de Monture\ifblank{#1}{}{~(#1+)}}
\newcommand{\la}{Armure Légère}
\newcommand{\ha}{Armure Lourde}
\newcommand{\platearmour}{Armure de Plates}
\newcommand{\hw}{Arme de Base}
\newcommand{\pw}{Paire d'Armes}
\newcommand{\spear}{Lance}
\newcommand{\halberd}{Hallebarde}
\newcommand{\gw}{Arme Lourde}
\newcommand{\lance}{Lance de Cavalerie}
\newcommand{\lightlance}{Lance Légère}
\newcommand{\shield}{Bouclier}
\newcommand{\barding}{Caparaçon}
\newcommand{\throwingweapons}{Armes de Jet}
\newcommand{\shortbow}{Arc Court}
\newcommand{\flail}{Fléau}

\newcommand{\cannon}{Canon}
\newcommand{\catapult}{Catapulte}
\newcommand{\volleygun}{Batterie de Tir}
\newcommand{\boltthrower}{Baliste}
\newcommand{\artilleryweapon}{Arme d'Artillerie}


%%% Troop types %%%

\newcommand{\characters}{Personnages}
\newcommand{\infantry}{Infanterie}
\newcommand{\monstrousinfantry}{Infanterie Monstrueuse}
\newcommand{\cavalry}{Cavalerie}
\newcommand{\monstrouscavalry}{Cavalerie Monstrueuse}
\newcommand{\swarm}{Nuée}
\newcommand{\swarms}{Nuées}
\newcommand{\warbeast}{Bête de Guerre}
\newcommand{\warbeasts}{Bêtes de Guerre}
\newcommand{\monster}{Monstre}
\newcommand{\monsters}{Monstres}
\newcommand{\monstrousbeast}{Bête Monstrueuse}
\newcommand{\monstrousbeasts}{Bêtes Monstrueuses}
\newcommand{\chariot}{Char}
\newcommand{\chariots}{Chars}
\newcommand{\riddenmonster}{Monstre Monté}
\newcommand{\riddenmonsters}{Monstres Montés}
\newcommand{\warmachine}{Machine de Guerre}
\newcommand{\warmachines}{Machines de Guerre}


%%% Terrain %%%

\newcommand{\water}{Eaux peu profondes}


%%% Profile wording

\newcommand{\oneofakind}{Uni\-que}
\newcommand{\onechoiceonly}{(un seul choix)}
\newcommand{\onfootonly}{(à pied seulement)}
\newcommand{\closecombatonly}{seulement au Corps à Corps}
\newcommand{\Xmodelsorless}[1]{(#1 figurines ou moins)}
\newcommand{\magicalitemsallowance}{Peut prendre des Objets Magiques}
\newcommand{\magicalweaponallowance}{Peut prendre une Arme Magique}
\newcommand{\notmagicalarmour}{(mais pas d'Armure Magique)}
\newcommand{\anyofthefollowing}{\optionschoice{Peut prendre :}}
\newcommand{\weapononechoice}{\optionschoice{Peut prendre une arme \onechoiceonly{} :}}
\newcommand{\weaponschoice}{\optionschoice{Peut prendre des armes :}}
\newcommand{\shootingweapononechoice}{\optionschoice{Peut prendre une arme de tir \onechoiceonly{} :}}
\newcommand{\combatweapononechoice}{\optionschoice{Peut prendre une arme de corps à corps \onechoiceonly{} :}}
\newcommand{\armouronechoice}{\optionschoice{Peut prendre une armure \onechoiceonly{} :}}
\newcommand{\magiclevelchoice}{\optionschoice{Peut devenir au choix :}}
\newcommand{\bsboption}{Peut devenir Porteur de la Grande Bannière}
\newcommand{\mayupgradeto}{Peut être amélioré en}
\newcommand{\mustbecomeoneofthefollowing}{\optionschoice{Doit devenir un choix parmi :}}
\newcommand{\maybecomeoneofthefollowing}{\optionschoice{Peut devenir un choix parmi :}}
\newcommand{\maytakeoneofthefollowing}{\optionschoice{Peut prendre un choix parmi :}}
\newcommand{\maytakeuptotwoofthefollowing}{\optionschoice{Peut prendre jusqu'à deux choix parmi :}}
\newcommand{\maygain}{Peut gagner la règle}
\newcommand{\maytake}{Peut prendre}
\newcommand{\maytakeashield}{Peut prendre un Bouclier}
\newcommand{\maytakela}{Peut prendre une Armure Légère}
\newcommand{\maytakeha}{Peut prendre une Armure Lourde}
\newcommand{\maytakemountsprotectionX}[1]{Peut prendre une \mountsprotection{#1}}
\newcommand{\maytakeagw}{Peut prendre une Arme Lourde}
\newcommand{\maytakeaspear}{Peut prendre une Lance}
\newcommand{\maytakepw}{Peut prendre une Paire d'Armes}
\newcommand{\maytakethrowingweapons}{Peut prendre des Armes de Jet}
\newcommand{\maytakebarding}{Peut prendre un Caparaçon}
\newcommand{\replaceshieldwithhalberd}{Remplacer le Bouclier par une Hallebarde}
\newcommand{\maybecome}{Peut devenir}

\newcommand{\maytakeonechoiceonly}{\optionschoice{\maytake{} \onechoiceonly{}\spacebeforecolon{}:}}

\newcommand{\mountssectionannouncement}{%
La section Montures concerne les montures de Personnages. Les montures pour non-Personnages suivent les règles données dans leur description d'unité.
}

%%% Commands to handle strings, better than xstring to handle commands inside the strings %%%

\newcommand{\substitute}[3]{%
  \protected@edef\sub@temp{#1}%
  \saveexpandmode
  \expandarg\StrSubstitute{\sub@temp}{#2}{#3}[#1]%
  \restoreexpandmode
}

\newcommand{\splitatstar}[3]{%
  \protected@edef\split@temp{#1}%
  \saveexpandmode
  \expandarg\StrCut{\split@temp}{*}#2#3%
  \restoreexpandmode
}

\newcommand{\splitatinf}[3]{%
  \protected@edef\split@temp{#1}%
  \saveexpandmode
  \expandarg\StrCut{\split@temp}{<}#2#3%
  \restoreexpandmode
}

\newcommand{\splitatequal}[3]{%
  \protected@edef\split@temp{#1}%
  \saveexpandmode
  \expandarg\StrCut{\split@temp}{=}#2#3%
  \restoreexpandmode
}

\newcommand{\ifsubstring}[4]{%
  \protected@edef\split@temp{#1}%
  \protected@edef\split@tempbis{#2}%
  \saveexpandmode
  \expandarg\IfSubStr{\split@temp}{\split@tempbis}{#3}{#4}%
  \restoreexpandmode
}

\def\removespaces#1{\zap@space#1 \@empty}

%%% Commands for alphabetical ordering %%%

\newcommand{\sortitem}[2][\relax]{%
	\DTLnewrow{list}% Create a new entry
	\ifx#1\relax%
		\DTLnewdbentry{list}{sortlabel}{#2}% Add entry sortlabel (no optional argument)
	\else%
		\DTLnewdbentry{list}{sortlabel}{#1}% Add entry sortlabel (optional argument)
	\fi%
		\DTLnewdbentry{list}{description}{#2}% Add entry description
}
\newenvironment{sortedlist}{%
	\DTLifdbexists{list}{\DTLcleardb{list}}{\DTLnewdb{list}}% Create new/discard old list
}{%
	\DTLsort{sortlabel}{list}% Sort list
	\begin{itemize*}[label={}, itemjoin={,}]%
		\DTLforeach*{list}{\theDesc=description}{%
		\item\theDesc}% Print each item
	\end{itemize*}%
}

\pdfstringdefDisableCommands{\def\textcolor#1{}}

% See language specific file for \addtosortedlist

%%% Database for automatic Quick Ref Sheet %%%

\DTLnewdb{profiles} % Database containing name, category, multiprofile number, profilename (if multi), caraclist, trooptype, invocation for CV.
\newcommand{\profilecategory}{\labels@lords} % Will be updated in relevant categories

\newcommand{\profiledtbfillname}[1]{\DTLnewdbentry{profiles}{name}{#1}}
\newcommand{\profiledtbfillcategory}[1]{\DTLnewdbentry{profiles}{category}{#1}}
\newcommand{\profiledtbfilltrooptype}[1]{\DTLnewdbentry{profiles}{trooptype}{#1}}
\newcommand{\profiledtbfillinvocation}[1]{\DTLnewdbentry{profiles}{invocation}{#1}}
\newcommand{\profiledtbfillprofile}[1]{\DTLnewdbentry{profiles}{profile}{#1}}
\newcommand{\profiledtbfillmultipleprofile}[1]{\DTLnewdbentry{profiles}{multipleprofile}{#1}}

\newcommand{\void}[1]{}
\newcounter{multiprofilecounter}

\newcommand{\profiledtbfillcarac}[1]{%
	\profiledtbfillprofile{#1}
	\parselist{#1}{\locallists@profileslist}% Split of the different profiles in the case of a multiprofile.
	\setcounter{multiprofilecounter}{0}%
	\forlistloop{\stepcounter{multiprofilecounter}\void}{\locallists@profileslist}%
	\expandafter\profiledtbfillmultipleprofile\expandafter{\number\value{multiprofilecounter}}
}


%%% Technical commands %%%

\newcommand{\newrule}{\textcolor{green!50!black}}
\newcommand{\removedrule}[1]{\textcolor{green!50!black}{\sout{#1}}}
\newcommand{\starsymbol}{$\star$}
\newcommand{\refsymbol}{$^\star$}

\newcommand{\inch}{\arcsecond}
\newcommand{\foot}{\arcminute}
\newcommand{\range}[1] {\labels@range~\unit{#1}{\inch}}
\newcommand{\distance}[1] {\unit{#1}{\inch}}
\newcommand{\result}[1] {\texttt{'}#1\texttt{'}}


%%% Fonts and sizes %%%

\newcommand{\bigtitle}[1]{\vspace*{-1.5cm}\section*{}\noindent\begin{center}\Hugefontsize\textbf{\antiquefont\expandafter\uppercase\expandafter{#1}}\end{center}}

\newcommand{\subtitle}[1]{\subsection*{}\noindent{\hugefontsize\antiquefont #1}}

\newcommand{\subsubtitle}[1]{\subsubsection*{}\noindent{\Largerfontsize\antiquefont #1}}

\newcommand{\verysmallfontsize}{\fontsize{4}{4.8}\selectfont}
\newcommand{\smallfontsize}{\fontsize{6}{7.2}\selectfont}
\newcommand{\normalfontsize}{\fontsize{8}{9.6}\selectfont}
\newcommand{\largefontsize}{\fontsize{10}{12}\selectfont}
\newcommand{\largerfontsize}{\fontsize{12}{14.4}\selectfont}
\newcommand{\Largefontsize}{\fontsize{14}{16.8}\selectfont}
\newcommand{\Largerfontsize}{\fontsize{15}{18}\selectfont}
\newcommand{\hugefontsize}{\fontsize{18}{21.6}\selectfont}
\newcommand{\Hugefontsize}{\fontsize{25}{30}\selectfont}

\newcommand{\unitentryformat}[1]{\textit{\largefontsize{#1}}}
\newcommand{\textIT}[1]{\textit{\largefontsize{#1}}}


%%% Titles %%%

\newcommand{\lordstitle}{\def\logolocalpath{../Layout/pics/logo_lord.png}\bigtitle{\labels@lords}}
\newcommand{\heroestitle}{%
\def\logolocalpath{../Layout/pics/logo_hero.png}%
\clearpage\bigtitle{\labels@heroes}%
\renewcommand{\profilecategory}{\labels@heroes}%
}
\newcommand{\coreunitstitle}{%
\def\logolocalpath{../Layout/pics/logo_core.png}%
\clearpage\bigtitle{\labels@coreunits}%
\renewcommand{\profilecategory}{\labels@coreunits}%
}
\newcommand{\specialunitstitle}{%
\def\logolocalpath{../Layout/pics/logo_special.png}%
\clearpage\bigtitle{\labels@specialunits}%
\renewcommand{\profilecategory}{\labels@specialunits}%
}
\newcommand{\rareunitstitle}{%
\def\logolocalpath{../Layout/pics/logo_rare.png}%
\clearpage\bigtitle{\labels@rareunits}%
\renewcommand{\profilecategory}{\labels@rareunits}%
}
\newcommand{\mountstitle}{%
\def\logolocalpath{../Layout/pics/logo_mount.png}%
\clearpage\bigtitle{\labels@charactermounts}%
\renewcommand{\profilecategory}{\labels@mounts}%
}

\newcommand{\startarmywiderules}{\newpage\bigtitle{\labels@armywiderules}\largefontsize}
\newcommand{\closearmywiderules}{\normalfontsize}
\newcommand{\armywideruleentry}[1]{\subtitle{#1}\vspace{5pt}}

\newcommand{\startarmyspecialrules}{\bigtitle{\labels@armyspecialrules}\largefontsize}
\newcommand{\closearmyspecialrules}{\normalfontsize}
\newcommand{\armyspecialruleentry}[1]{\subtitle{#1}\vspace{5pt}}

\newcommand{\startarmyarmoury}{\bigtitle{\labels@armoury}\largefontsize\subtitle{}}
\newcommand{\closearmyarmoury}{\normalfontsize}

\newcommand{\startarmymagicalitems}{\newpage\largefontsize\bigtitle{\labels@magicalitems}\begin{multicols}{2}\raggedcolumns}
\newcommand{\closearmymagicalitems}{\end{multicols}\normalfontsize}

\newcommand{\armymagicalweapons}{\subtitle{\labels@magicalweapons}}
\newcommand{\armymagicalarmour}{\subtitle{\labels@magicalarmour}}
\newcommand{\armytalismans}{\subtitle{\labels@talismans}}
\newcommand{\armyenchanteditems}{\subtitle{\labels@enchanteditems}}
\newcommand{\armyarcaneitems}{\subtitle{\labels@arcaneitems}}
\newcommand{\armymagicalbanners}{\subtitle{\labels@magicalbanners}}

\newcommand{\startarmynewsection}[1]{\newpage\bigtitle{#1}\largefontsize}
\newcommand{\startarmynewsectionSP}[1]{\vspace{1.5cm}\bigtitle{#1}\largefontsize}
\newcommand{\closearmynewsection}{\normalfontsize}

\newcommand{\armynewsubsection}[1]{\subtitle{#1}\vspace{5pt}}
\newcommand{\armynewsubsubsection}[1]{\subsubtitle{#1}\vspace{3pt}}

\newcommand{\armylist}{\clearpage}

\newcommand{\quickrefsheettitle}{\clearpage\newgeometry{top=1.6cm, bottom=2cm, left=1cm, right=1cm}\bigtitle{\labels@quickrefsheet}\vspace*{0.4cm}}
\newcommand{\changelogtitle}{\clearpage\bigtitle{\labels@changelog}\spaceaftersection{}}

\newcommand{\spaceaftersection}{\vspace{0.8cm}}

\newcommand{\separator}{\noindent\begin{center}\textcolor{black!30}{\rule{0.7\columnwidth}{2pt}}\end{center}}


%%% Custom lists and description for first sections of the army books

\newcommand{\startpricelist}{\begin{samepage}\begin{description}[leftmargin=0.3cm, labelindent=0cm, labelsep=0.1cm]}
\def\endpricelist{\end{description}\end{samepage}}
\newcommand{\pricelistitem}[2]{\item \option{\textbf{#1}}{#2}\newline}

\newcommand{\startpricelistNSP}{\begin{description}[leftmargin=0.3cm, labelindent=0cm, labelsep=0.1cm]}
\def\endpricelistNSP{\end{description}}

\newcommand{\startitemlist}{\begin{multicols}{2}\raggedcolumns\begin{description}[leftmargin=0.3cm, labelindent=0cm, labelsep=0.1cm]}
\def\enditemlist{\end{description}\end{multicols}}
\newcommand{\listitem}[1]{\item[#1\spacebeforecolon{}:]}

\newcommand{\startitemlistonecol}{\begin{description}[leftmargin=0.3cm, labelindent=0cm, labelsep=0.1cm]}
\def\enditemlistonecol{\end{description}}
\newcommand{\listitemonecol}[1]{\item \textbf{#1\spacebeforecolon{}:}\newline}

\newenvironment{customitemize}{\begin{description}[leftmargin=0.3cm, labelindent=0cm, labelsep=0cm]}{\end{description}}
\newenvironment{customsubitemize}{\begin{itemize}[label={-}, labelsep=0.1cm, topsep=0cm, parsep=0cm, itemsep=0cm, leftmargin=0.4cm, labelindent=0cm]}{\end{itemize}}

%%% Table parameters %%%

\newcolumntype{M}[1]{>{\centering\let\newline\\\arraybackslash\hspace{0pt}}m{#1}}


%%%  Lists handling %%%

\newcommand{\addlocallist}{\listadd\locallists@dummy}%
\NewDocumentCommand{\parsespacelist}{>{\SplitList{ }} m }{%
	\ProcessList{#1}{\addlocallist}%
}%
\NewDocumentCommand{\parsecommalist}{>{\SplitList{,}} m }{%
	\ProcessList{#1}{\addlocallist}%
}%
\newcommand{\parselist}[3][,]{%
	\renewcommand\addlocallist{\listadd#3}%
  	\undef#3%
  	\ifstrequal{#1}{ }{\parsespacelist{#2}}{\parsecommalist{#2}}%
}


%%% Profiles handling %%%

% Element of a table that contains the characteristics of a model (or part of a model)
\newcommand\caraclist[1]{
	\parselist[ ]{#1}{\locallists@caraclist}%
	\forlistloop{&}{\locallists@caraclist}%
}

\newcommand\caraclistbold[1]{
	\parselist[ ]{#1}{\locallists@caraclist}%
	\forlistloop{&\bfseries}{\locallists@caraclist}%
}

% Line of a profile table, including bottom line. It is meant to contain the name of the model (or part), its characteristics (preferably, the second argument should contain the \carac macro), troop type and base size.
\newcommand{\profilefirstline}[4]{#1 & #2 &   & #3 & #4 }

% Start of a profile table. Includes the table commands, and the column labels. \profilecellsize is the size of the characteristics cells in the profile.
\newcommand{\profilecellsize}{0.56cm}
\newcommand{\profilestart}{%
	\noindent %
	\begin{tabular}{@{}p{3cm}@{}M{\profilecellsize}@{}M{\profilecellsize}@{}M{\profilecellsize}@{}M{\profilecellsize}@{}M{\profilecellsize}@{}M{\profilecellsize}@{}M{\profilecellsize}@{}M{\profilecellsize}@{}M{\profilecellsize}@{}p{2.7cm}@{}p{3.3cm}@{}p{2cm}@{}}%
	 &% \textbf{\labels@profile}
	\labels@M & \labels@WS & \labels@BS & \labels@S & \labels@T & \labels@W & \labels@I & \labels@A & \labels@Ld &%
	&%
	{\unitentryformat{\labels@trooptype}} &%
	{\unitentryformat{\labels@basesize}}%
}

% End of a profile table.
\newcommand{\profileend}{\end{tabular}}

% Algorithm to automatically use and fill previous command, with coherence check.
\providebool{profilefirst}
\newcommand{\profileitem}[1]{%
	\tabularnewline%
	\splitatinf{#1}\local@unitname\local@unitprofile%
	\local@unitname \expandafter\caraclistbold\expandafter{\local@unitprofile}%
	&%
	& \ifbool{profilefirst}{\unit@type}{}%
	& \ifbool{profilefirst}{%
		\ifsubstring{\unit@basesize}{x}{% Rectangular base
			\unit{\unit@basesize}{\milli\meter}%
		}{% Circular base
			\unit{\unit@basesize}{\milli\meter} \labels@roundbase%
		}%
	}{}%
	\global\boolfalse{profilefirst}%
}
\newcommand{\profile}[1]{%
	\parselist{#1}{\locallists@profileslist}%
	\profilestart%
	\global\booltrue{profilefirst}%
	\forlistloop{\profileitem}{\locallists@profileslist}%
	\profileend%
}


%%% Profiles handling in case of invocation %%%

\newcommand{\invocprofilestart}{%
	\noindent %
	\begin{tabular}{@{}p{3cm}@{}M{\profilecellsize}@{}M{\profilecellsize}@{}M{\profilecellsize}@{}M{\profilecellsize}@{}M{\profilecellsize}@{}M{\profilecellsize}@{}M{\profilecellsize}@{}M{\profilecellsize}@{}M{\profilecellsize}@{}M{2.2cm}@{}p{0.5cm}@{}p{3.3cm}@{}p{2cm}@{}}%
	 &% \textbf{\labels@profile}
	\labels@M & \labels@WS & \labels@BS & \labels@S & \labels@T & \labels@W & \labels@I & \labels@A & \labels@Ld & \unitentryformat{\labels@Invocation} &%
	&%
	{\unitentryformat{\labels@trooptype}} &%
	{\unitentryformat{\labels@basesize}}%
}

\newcommand{\invocprofileitem}[1]{%
	\tabularnewline%
	\splitatinf{#1}\local@unitname\local@unitprofile%
	\local@unitname \expandafter\caraclistbold\expandafter{\local@unitprofile}%
	& \ifbool{profilefirst}{\unit@invocation}{} &%
	& \ifbool{profilefirst}{\unit@type}{}%
	& \ifbool{profilefirst}{\unit{\unit@basesize}{\milli\meter}}{}%
	\global\boolfalse{profilefirst}%
}

\newcommand{\invocprofile}[1]{%
	\parselist{#1}{\locallists@profileslist}%
	\invocprofilestart%
	\global\booltrue{profilefirst}%
	\forlistloop{\invocprofileitem}{\locallists@profileslist}%
	\profileend%
}


%%%%%%%%%%%%%%%%%%
%%% Unit rules %%%
%%%%%%%%%%%%%%%%%%

%%% Entry title command %%%

\newcommand{\unitentry}[2]{\ifdefempty{#1}{}{\noindent #2}}


%%% Special rules %%%

% Special rules listing for a unit, with alphabetical order.
\newcommand{\ruleslist}[1]{%
	\parselist[,]{#1}{\locallists@ruleslist}%
	\begin{sortedlist}%
		\forlistloop{\addtosortedlist}{\locallists@ruleslist}%
	\end{sortedlist}%
}

% Special rules entry.
\newcommand{\specialrules}[1]{\unitentry{#1}{\unitentryformat{\labels@specialrules\spacebeforecolon{}:}\newline\hspace*{-\fontdimen2\font}\expandafter\ruleslist\expandafter{#1}.}}
\newcommand{\commonspecialrules}[2]{\unitentry{#2}{\unitentryformat{#1\spacebeforecolon{}:}\newline\hspace*{-\fontdimen2\font}\expandafter\ruleslist\expandafter{#2}.}}


%%% Magical abilities %%%

% Paths listing for a unit.
\newcommand{\pathslist}[1]{%
	\parselist[,]{#1}{\locallists@pathslist}%
	\begin{itemize*}[label={}, itemjoin={,}, itemjoin*={\listlastchoice}]%
		\forlistloop{\item}{\locallists@pathslist}%
	\end{itemize*}%
}

% Magic entry.
\newcommand{\magic}[2]{\unitentry{#2}{\unitentryformat{\labels@magic\spacebeforecolon{}: }\newline\ifdefempty{#1}{}{\textbf{\magiclevel{#1}}. }\labels@pathsused\expandafter\pathslist\expandafter{#2}.}}

% Wizard Conclave.
\newcommand{\magicwizardconclave}[1]{\unitentry{#1}{\unitentryformat{\labels@magic\spacebeforecolon{}: }\newline\textbf{\wizardconclave{}}\spacebeforecolon{}: #1.}}


%%% Equipment %%%

% Equipment listing.
\newcommand{\equipmentlist}[1]{%
	\parselist[,]{#1}{\locallists@equipmentlist}%
	\begin{sortedlist}%
		\forlistloop{\addtosortedlist}{\locallists@equipmentlist}%
	\end{sortedlist}%
}

% Equipment entry.
\newcommand{\weapons}[1]{\unitentry{#1}{\unitentryformat{\labels@weapons\spacebeforecolon{}:}\newline\hspace*{-\fontdimen2\font}\expandafter\equipmentlist\expandafter{#1}.}}

\newcommand{\armour}[1]{\unitentry{#1}{\unitentryformat{\labels@armour\spacebeforecolon{}:}\newline\hspace*{-\fontdimen2\font}\expandafter\equipmentlist\expandafter{#1}.}}


%%% Alignment %%%

\newcommand{\alignment}[1]{\unitentry{#1}{\unitentryformat{\labels@alignment\spacebeforecolon{}:}\newline\textbf{#1}.}}

%%% Green Hide Race %%%

\newcommand{\greenhideraceentry}[1]{\unitentry{#1}{\unitentryformat{\labels@greenhiderace\spacebeforecolon{}:}\newline\textbf{#1}.}}


%%% Options %%%

% Frame commands.
\newcommand{\optionsframestart}{\begin{innerframe}[\labels@options]}
\newcommand{\optionsframeend}{\end{innerframe}}

% Options listing.
\newcommand{\optionslist}[1]{%
	\parselist[,]{#1}{\locallists@optionslist}%
	\begin{description}[leftmargin=0.3cm, labelindent=0cm, labelsep=0cm, itemsep=0cm, parsep=0cm]%
		\forlistloop{\item\setoption}{\locallists@optionslist}%
	\end{description}%
}

% Options entry.
\newcommand{\options}[1]{\ifdefempty{#1}{}{\optionsframestart\vspace*{-0.4cm}\unitentry{#1}{\expandafter\optionslist\expandafter{#1}}\optionsframeend}}

% Option specific commands.
\newcommand{\setoption}[1]{%
	\noexpandarg\StrCut{#1}{=}\optiontext\optionvalue%
	\expandafter\ifstrequal\expandafter{\optionvalue}{}{%
		\optiontext%
	}{%
	\ifsubstring{\optionvalue}{\free}{%
		\option[\free]{\optiontext}{\optionvalue}%
	}{%
	\ifsubstring{\optionvalue}{\unlimited}{%
		\option[\unlimited]{\optiontext}{\optionvalue}%
	}{%
	\ifsubstring{\optionvalue}{\upto}{%
		\splitatinf{\optionvalue}\myoption\myvalue%
		\option[\upto]{\optiontext}{\myvalue}%
	}{%
	\ifsubstring{\optionvalue}{\permodel}{%
		\splitatinf{\optionvalue}\myoption\myvalue%
		\option[\permodel]{\optiontext}{\myvalue}%
	}{%
	\ifsubstring{\optionvalue}{\pershadygit}{% For Orcs N Goblins
		\splitatinf{\optionvalue}\myoption\myvalue%
		\option[\pershadygit]{\optiontext}{\myvalue}%
	}{%
	\ifsubstring{\optionvalue}{\permadgit}{% For Orcs N Goblins
		\splitatinf{\optionvalue}\myoption\myvalue%
		\option[\permadgit]{\optiontext}{\myvalue}%
	}{%	
	\ifsubstring{\optionvalue}{\perrune}{% For Dwarven Holds
		\splitatinf{\optionvalue}\myoption\myvalue%
		\option[\perrune]{\optiontext}{\myvalue}%
	}{%	
		\option{\optiontext}{\optionvalue}%
	}}}}}}}}%
}

\newcommand{\option}[3][]{#2\predotfill\dotfill\nobreak%
	% Add \upto token if necessary.
	\ifstrequal{#1}{\upto}{\upto~}{}%
	% The option can be free, have an unlimited cost, or have a points cost.
	\ifstrequal{#1}{\free}{\free}{\ifstrequal{#1}{\unlimited}{\unlimited}{\pts{#3}}}%
	% Add \permodel if necessary.
	\ifstrequal{#1}{\permodel}{\nobreak\permodel}{}%
	% Add \persomething if necessary.
	\ifstrequal{#1}{\pershadygit}{\nobreak\pershadygit}{}% For Orcs N Goblins
	\ifstrequal{#1}{\permadgit}{\nobreak\permadgit}{}% For Orcs N Goblins
	\ifstrequal{#1}{\perrune}{\nobreak\perrune}{}% For Dwarven Holds
}

\newcommand\optionschoice[2]{%
	\parselist[,]{#2}{\locallists@optionschoice}%
	#1%
	\begin{itemize}[label={}, parsep=0cm, labelindent=0cm, labelwidth=0cm, noitemsep, topsep=0em, leftmargin=0.3cm]%
	\forlistloop{\item\setoption}{\locallists@optionschoice}%
	\end{itemize}%
}

\newcommand\optionschoiceTWOCOL[2]{%
	\parselist[,]{#2}{\locallists@optionschoice}%
	#1%
	\begin{itemize}[label={}, parsep=0cm, labelindent=0cm, labelwidth=0cm, noitemsep, topsep=0em, leftmargin=0.3cm]%
	\setlength{\columnseprule}{0.5pt}
	\renewcommand{\columnseprulecolor}{\color{black!30}}
	\vspace*{-5pt}\begin{multicols}{2}\raggedcolumns
	\forlistloop{\item\setoption}{\locallists@optionschoice}%
	\end{multicols}\setlength{\columnseprule}{0pt}
	\end{itemize}%
}

% Option description in army desc.
\newcommand{\optiondef}[3]{\option{\textbf{#1}}{#2}\ifblank{#3}{}{\\{#3}}}


%%% Mount options %%%

% Frame commands.
\newcommand{\mountsframestart}{\begin{innerframe}[\labels@mounts]}
\newcommand{\mountsframeend}{\end{innerframe}}

% Mount listing.
\newcommand{\mountslist}[1]{%
	\parselist[,]{#1}{\locallists@mountslist}%
	\begin{description}[leftmargin=0.3cm, labelindent=0cm, labelsep=0cm, itemsep=0cm, parsep=0cm]%
		\forlistloop{\item\setoption}{\locallists@mountslist}%
	\end{description}%
}

% Mount entry.
\newcommand{\mounts}[1]{\ifdefempty{#1}{}{\mountsframestart\vspace*{-0.4cm}\unitentry{#1}{\expandafter\mountslist\expandafter{#1}}\mountsframeend}}


%%% Command group %%%

% Command group specific commands.
\define@key{commandgroup}{restriction}            {\def\commandgroup@restriction{#1}}
\define@key{commandgroup}{champion}               {\def\commandgroup@champion{#1}}
\define@key{commandgroup}{championallowance}      {\def\commandgroup@championallowance{#1}}
\define@key{commandgroup}{championoption}         {\def\commandgroup@championoption{#1}}
\define@key{commandgroup}{championprerestriction} {\def\commandgroup@championprerestriction{#1}}
\define@key{commandgroup}{championrestriction}    {\def\commandgroup@championrestriction{#1}}
\define@key{commandgroup}{banner}                 {\def\commandgroup@banner{#1}}
\define@key{commandgroup}{bannerallowance}        {\def\commandgroup@bannerallowance{#1}}
\define@key{commandgroup}{veteranstandardbearer}  {\def\commandgroup@veteranstandardbearer{#1}}
\define@key{commandgroup}{singlebannerallowance}  {\def\commandgroup@singlebannerallowance{#1}}
\define@key{commandgroup}{condsinglebannerallowance}  {\def\commandgroup@condsinglebannerallowance{#1}}
\define@key{commandgroup}{banneroption}           {\def\commandgroup@banneroption{#1}}
\define@key{commandgroup}{bannerrestriction}      {\def\commandgroup@bannerrestriction{#1}}
\define@key{commandgroup}{musician}               {\def\commandgroup@musician{#1}}
\define@key{commandgroup}{musicianrestriction}    {\def\commandgroup@musicianrestriction{#1}}
\newcommand{\defcommandgroup}{%
	\setkeys{commandgroup}{restriction=,
	                       champion=, championallowance=, championoption=, championprerestriction=, 
	                       championrestriction=, banner=, bannerallowance=, veteranstandardbearer=, 
	                       singlebannerallowance=, condsinglebannerallowance=, banneroption=, 
	                       bannerrestriction=, musician=, musicianrestriction=}%
	\setkeys{commandgroup}%
}

% Frame commands.
\newcommand{\commandgroupframestart}{\begin{innerframe}[\labels@commandgroup]}
\newcommand{\commandgroupframeend}{\end{innerframe}}

% Command group entry.
\newcommand{\commandgroup}[1]{%
	\defcommandgroup{#1}%
	\ifstrempty{#1}{}{\commandgroupframestart\vspace*{-0.2cm}%
		\begin{description}[leftmargin=0.3cm, labelindent=0cm, labelsep=0cm, itemsep=0cm, parsep=0cm]%
			% Command group title, including restrictions applying to all the command group
			\item \textbf{\expandafter\ifblank\expandafter{\commandgroup@restriction}{}{ \only{\commandgroup@restriction}\spacebeforecolon{}: }} 
			% Champion handling.
			\ifdefempty{\commandgroup@champion}{}{% We have a champion!
			\ifdefempty{\commandgroup@championprerestriction}{% There is no prerestriction to have a champion
				\item \hspace*{-0.04cm}\option{\labels@champion%
					% Possible restrictions to taking a champion
				    \expandafter\ifblank\expandafter{\commandgroup@championrestriction}{}{ \only{\commandgroup@championrestriction}}%
				    % Cost of a champion
				    }{\commandgroup@champion}%
				    % Magical allowance of the champion. Should probably not be used, champion option can do it as well and is more flexible.
					\ifdefempty{\commandgroup@championallowance}{}{\par\option[\upto]{\hspace*{0.3cm}- \labels@championallowance}{\commandgroup@championallowance}}%
					% Any option available to the champion, in the form option:cost
					\ifdefempty{\commandgroup@championoption}{}{%
						\splitatinf{\commandgroup@championoption}\local@option\local@cost%
						\par\option{\hspace*{0.3cm}- \local@option}{\local@cost}}%
			}{% There is a pre-restriction to have a champion
				\item \hspace*{-0.04cm}\commandgroup@championprerestriction	\newline%
				\option{\labels@champion}{\commandgroup@champion}%
				% Magical allowance of the champion. Should probably not be used, champion option can do it as well and is more flexible.
				\ifdefempty{\commandgroup@championallowance}{}{\par\option[\upto]{\hspace*{0.3cm}- \labels@championallowance}{\commandgroup@championallowance}}%
				% Any option available to the champion, in the form option:cost
				\ifdefempty{\commandgroup@championoption}{}{%
					\splitatinf{\commandgroup@championoption}\local@option\local@cost%
					\par\option{\hspace*{0.3cm}- \local@option}{\local@cost}}%
			} %End of the prerestriction of not condition
			}% End of champion handling
			\ifdefempty{\commandgroup@musician}{}{% We have a musician!
				\item \hspace*{-0.04cm}\option{\labels@musician%
					% Possible restrictions to taking a musician
				    \expandafter\ifblank\expandafter{\commandgroup@musicianrestriction}{}{ \only{\commandgroup@musicianrestriction}}%
				    % Cost of a musician
				    }{\commandgroup@musician}%
			}%
			\ifdefempty{\commandgroup@banner}{}{% We have a banner!
				\item \hspace*{-0.04cm}\option{\labels@standardbearer%
					% Possible restrictions to taking a banner
				    \expandafter\ifblank\expandafter{\commandgroup@bannerrestriction}{}{ \only{\commandgroup@bannerrestriction}}%
				    % Cost of a banner
				    }{\commandgroup@banner}%
				    % Magical banner, if all units of this type can take one.
					\ifdefempty{\commandgroup@bannerallowance}{}{\par\option[\upto]{\hspace*{0.3cm}- \labels@bannerallowance}{\commandgroup@bannerallowance}}%
					% Magical banner, if Veteran.
					\ifdefempty{\commandgroup@veteranstandardbearer}{}{\par\hspace*{0.3cm}- \labels@veteranstandardbearer%
					\expandafter\ifstrequal\expandafter{\commandgroup@veteranstandardbearer}{*}{*}{}%
					}%
					% Magical banner, if only one unit of this type can take one.
					\ifdefempty{\commandgroup@singlebannerallowance}{}{\par\option[\upto]{\hspace*{0.3cm}- \labels@singlebannerallowance}{\commandgroup@singlebannerallowance}}%
					% Magical banner, if only one unit of this type can take one, but with condtions.
					\ifdefempty{\commandgroup@condsinglebannerallowance}{}{%
						\splitatinf{\commandgroup@condsinglebannerallowance}\local@option\local@cost%
						\par\option[\upto]{\hspace*{0.3cm}- \labels@condsinglebannerallowance \local@option}{\local@cost}}%
					% Additional option for the banner, such as Hill Goblin Lookouts for Ogres
					\ifdefempty{\commandgroup@banneroption}{}{%
						\splitatinf{\commandgroup@banneroption}{\local@option}{\local@cost}%
						\par\option{\hspace*{0.3cm}- \local@option}{\local@cost}%
					}%
			}%
		\end{description}%
	\commandgroupframeend%
	 }%
}


%%% Unit rules %%%

% Frame commands.
\newcommand{\unitrulesframestart}{\begin{innerframe}[\labels@specialrules]}
\newcommand{\unitrulesframeend}{\end{innerframe}}

% Unit rules specific commands.
\newcommand{\unitrule}[2]{\item[#1\spacebeforecolon{}:]#2}

% Unit rule entry.
\newcommand{\unitrules}[1]{\ifdefempty{#1}{}{\unitrulesframestart\vspace*{-0.05cm}\begin{description}[leftmargin=0.3cm, labelindent=0cm, labelsep=0.1cm, itemsep=0.2cm, parsep=0cm]#1\end{description}\unitrulesframeend}}


%%% Special equipment %%%

% Frame commands.
\newcommand{\unitequipmentframestart}{\begin{innerframe}[\labels@specialequipment]}
\newcommand{\unitequipmentframeend}{\end{innerframe}}

% Special equipment specific commands.
\newcommand{\equipmentdef}[2]{\item[#1\spacebeforecolon{}:]#2}

% Special equipment entry.
\newcommand{\unitequipment}[1]{\ifdefempty{#1}{}{\unitequipmentframestart\vspace*{-0.05cm}\begin{description}[leftmargin=0.3cm, labelindent=0cm, labelsep=0.1cm, itemsep=0.2cm, parsep=0cm]#1\end{description}\unitequipmentframeend}}






%%%%%%%%%%%%%%%%%%%%%%%%%%%%%%%%
%%% Profile input and layout %%%
%%%%%%%%%%%%%%%%%%%%%%%%%%%%%%%%

%%% Input parameters %%%

\define@key{unit}{notinQRS}{\def\unit@notinQRS{#1}}
\define@key{unit}{name}{\def\unit@name{#1}}
\define@key{unit}{QRSname}{\def\unit@QRSname{#1}}
\define@key{unit}{profile}{\def\unit@profile{#1}}
\define@key{unit}{cost}{\def\unit@cost{#1}}
\define@key{unit}{invocation}{\def\unit@invocation{#1}}
\define@key{unit}{costpermodel}{\def\unit@costpermodel{#1}}
\define@key{unit}{maxmodels}{\def\unit@maxmodels{#1}}
\define@key{unit}{type}{\def\unit@type{#1}}
\define@key{unit}{unitsize}{\def\unit@unitsize{#1}}
\define@key{unit}{basesize}{\def\unit@basesize{#1}}
\define@key{unit}{commonspecialrules}{\def\unit@commonspecialrules{#1}}
\define@key{unit}{commontype}{\def\unit@commontype{#1}}
\define@key{unit}{commonspecialrulesB}{\def\unit@commonspecialrulesB{#1}}
\define@key{unit}{commontypeB}{\def\unit@commontypeB{#1}}
\define@key{unit}{specialrules}{\def\unit@specialrules{#1}}
\define@key{unit}{magiclevel}{\def\unit@magiclevel{#1}}
\define@key{unit}{magicpaths}{\def\unit@magicpaths{#1}}
\define@key{unit}{equipment}{\def\unit@equipment{#1}}
\define@key{unit}{alignment}{\def\unit@alignment{#1}}
\define@key{unit}{greenhiderace}{\def\unit@greenhiderace{#1}}
\define@key{unit}{weapons}{\def\unit@weapons{#1}}
\define@key{unit}{armour}{\def\unit@armour{#1}}
\define@key{unit}{wizardconclave}{\def\unit@wizardconclave{#1}}
\define@key{unit}{unitequipment}{\def\unit@unitequipment{#1}}
\define@key{unit}{options}{\def\unit@options{#1}}
\define@key{unit}{mounts}{\def\unit@mounts{#1}}
\define@key{unit}{commandgroup}{\def\unit@commandgroup{#1}}
\define@key{unit}{unitrules}{\def\unit@unitrules{#1}}
\define@key{unit}{additional}{\def\unit@additional{#1}}


%%% Frames definition %%%

% Unit's big frame.
\tikzset{unitprice/.style={draw=white, fill=white, rectangle, rounded corners, right, minimum height=0.7cm}}
\tikzset{unittitle/.style={draw=white, fill=white, rectangle, rounded corners, right, minimum height=0.7cm, font=\bfseries}}
\tikzset{unitlogo/.style={draw=white, fill=white, rectangle, right, minimum height=0.7cm}}

\newenvironment{unitframe}[2][]{%
	\mdfsetup{%
		nobreak=true,%
		linewidth=1pt,%
		linecolor=black!30,%
		roundcorner=5pt,%
		backgroundcolor=white,%
		innertopmargin=1.2\baselineskip,
		innerbottommargin=1.2\baselineskip,
		singleextra={
			\expandafter\ifblank\expandafter{\unit@cost}{}{%
				\node[unitprice,anchor=east,xshift=-0.5cm] at (P)%
					{%
						{{\smallfontsize\minprice} \Largefontsize\pts{\textbf{\unit@cost}}}%
					};
				}%
				\node[unittitle,xshift=0.5cm] at (P-|O)%
					{\Largefontsize\antiquefont\uppercase\expandafter\expandafter\expandafter{\unit@name}};
				\node[unitlogo, xshift=8.1cm, yshift=0.1cm] at (P-|O)%
					{\includegraphics[width=1.2cm]{\logolocalpath}};
		}
	}%
	\begin{mdframed}[]\relax%
}%
{%
\end{mdframed}%
}

% Inner small frames for options, special rules definition, ...
\tikzset{innertitle/.style={fill=white, rectangle, rounded corners, right, minimum height=8pt, xshift=0.5cm}}

\newenvironment{innerframe}[1][]{%
	\mdfsetup{%
		innerleftmargin=5pt,%
		innerrightmargin=5pt,%
		linecolor=black!30,%
		linewidth=0.5pt,%
		roundcorner=5pt,%
		backgroundcolor=white,%
		innertopmargin=1.1\baselineskip,
		singleextra={
		\node[innertitle] at (P-|O)%
			{\unitentryformat{#1}};
		}
	}%
	\vspace*{-0.2cm}\begin{mdframed}[]\relax%
}%
{%
\end{mdframed}%
}

%%% Command to add a new unit definition %%%

\newcommand{\defunit}{
	\setkeys{unit}{%
		notinQRS=, name=, QRSname=, profile=, cost=, invocation=, costpermodel=, maxmodels=, type=, unitsize=, basesize=, commonspecialrules=, commontype=, commonspecialrulesB=, commontypeB=, specialrules=, magiclevel=, magicpaths=, alignment=, greenhiderace=, equipment=, weapons=, armour=, wizardconclave=, unitequipment=, options=, mounts=, commandgroup=, unitrules=, additional=%
	}%
	\setkeys{unit}%
}

\newcommand{\showunit}[1]{
	\defunit{#1}
	\begin{unitframe}[\unit@name]{\unit@cost}
	\mdfsetup{style=defaultoptions}
	\expandafter\ifblank\expandafter{\unit@unitsize}{}{%
	\expandafter\ifstrequal\expandafter{\unit@unitsize}{1}{% single model
		% Can you add model to this single model ?
		\expandafter\ifblank\expandafter{\unit@maxmodels}{% no		
			{\hspace*{0.25cm}\labels@Singlemodel}%
		}{% yes
			{\hspace*{0.25cm}\mincostfor{} \textbf{1} \labels@model{}. \maxunitsize{}\spacebeforecolon{}: \textbf{\unit@maxmodels} \labels@models{}.\hfill \additionalfigscost{} {\largefontsize\pts{\textbf{\unit@costpermodel{}}}\permodel}\hspace*{0.1cm}}%
		}%
	}{% not single model
		% Test if we wanna print a sentence instead of unit number
		\ifsubstring{\unit@unitsize}{SPECIAL-}{%
			\hspace*{0.25cm}\StrDel{\unit@unitsize}{SPECIAL-}%
		}{%	
			{\hspace*{0.25cm}\mincostfor{} \textbf{\unit@unitsize} \labels@models{}. \maxunitsize{}\spacebeforecolon{}: \textbf{\unit@maxmodels} \labels@models{}.\hfill \additionalfigscost{} {\largefontsize\pts{\textbf{\unit@costpermodel{}}}\permodel}\hspace*{0.1cm}}%
		}%
	}%
	}%
	\vspace*{-0.1cm}
	\noindent\begin{center}\textcolor{black!30}{\rule{\columnwidth}{1pt}}\end{center}
		\expandafter\ifblank\expandafter{\unit@invocation}{%
			\expandafter\profile\expandafter{\unit@profile}
		}{%
			\expandafter\invocprofile\expandafter{\unit@profile}
		}
	\noindent\begin{center}\textcolor{black!30}{\rule{\columnwidth}{1pt}}\end{center}
	\vspace*{-0.2cm}
	\setlength\multicolsep{0pt}
	\begin{multicols}{2}
		\raggedcolumns
		\vspace*{-0.3cm}{\setlength{\parskip}{0.3cm}
		\expandafter\ifblank\expandafter{\unit@alignment}{}{\noindent\parbox{\columnwidth}{\alignment{\unit@alignment}}}
		
		\expandafter\ifblank\expandafter{\unit@greenhiderace}{}{\noindent\parbox{\columnwidth}{\greenhideraceentry{\unit@greenhiderace}}}
		
		\expandafter\ifblank\expandafter{\unit@equipment}{}{\noindent\parbox{\columnwidth}{\equipment{\unit@equipment}}}
				
		\expandafter\ifblank\expandafter{\unit@weapons}{}{\noindent\parbox{\columnwidth}{\weapons{\unit@weapons}}}
		
		\expandafter\ifblank\expandafter{\unit@armour}{}{\noindent\parbox{\columnwidth}{\armour{\unit@armour}}}
		
		\expandafter\ifblank\expandafter{\unit@commonspecialrules}{}{\noindent\parbox{\columnwidth}{\commonspecialrules{\unit@commontype}{\unit@commonspecialrules}}}
		
		\expandafter\ifblank\expandafter{\unit@commonspecialrulesB}{}{\noindent\parbox{\columnwidth}{\commonspecialrules{\unit@commontypeB}{\unit@commonspecialrulesB}}}
		
		\expandafter\ifblank\expandafter{\unit@specialrules}{}{\noindent\parbox{\columnwidth}{\specialrules{\unit@specialrules}}}
		
		\expandafter\ifblank\expandafter{\unit@magicpaths}{}{\noindent\parbox{\columnwidth}{\magic{\unit@magiclevel}{\unit@magicpaths}}}
		
		\expandafter\ifblank\expandafter{\unit@wizardconclave}{}{\noindent\parbox{\columnwidth}{\magicwizardconclave{\unit@wizardconclave}}}
		}
		\vspace{0.1cm}
		\mounts{\unit@mounts}
		\options{\unit@options}
		\expandafter\ifblank\expandafter{\unit@commandgroup}{}{\expandafter\commandgroup\expandafter{\unit@commandgroup}}
		\unitrules{\unit@unitrules}
		\unitequipment{\unit@unitequipment}
	\end{multicols}
	\vspace*{0.1cm}\unit@additional
	\end{unitframe}
	% Database filling for auto QRS
	\expandafter\ifblank\expandafter{\unit@notinQRS}{%
	\DTLnewrow{profiles}%
	\expandafter\ifblank\expandafter{\unit@QRSname}{%
		\expandafter\profiledtbfillname\expandafter{\unit@name}%
	}{%
		\expandafter\profiledtbfillname\expandafter{\unit@QRSname}%
	}
	\expandafter\profiledtbfillcategory\expandafter{\profilecategory}%
	\expandafter\profiledtbfilltrooptype\expandafter{\unit@type}%
	\expandafter\ifblank\expandafter{\unit@invocation}{}{\expandafter\profiledtbfillinvocation\expandafter{\unit@invocation}}%
	\expandafter\profiledtbfillcarac\expandafter{\unit@profile}
	}{}%
}


%%% Changelog commands %%%

\newcommand{\newlog}[2]{%
\vspace*{0.2cm}\noindent{\antiquefont\Large\textbf{V#1}}
\parselist[,]{#2}{\locallists@changelist}%
\begin{itemize}[itemsep=0pt]%
\forlistloop{\item[-]}{\locallists@changelist}%
\end{itemize}%
}

\newcommand{\startchangelog}{\begin{multicols}{2}\vspace*{-0.2cm}}
\def\endchangelog{\end{multicols}}


\newcommand{\booktitle}{Guerriers des Dieux Sombres}
\newcommand{\version}{0.99.9}
\newcommand{\frenchversion}{1.0}
\newcommand{\booklogo}{pics/logo_big_WDG.png}

%\newcommand{\translationteam}{\item \og AEnoriel \fg \item \og Anglachel \fg \item \og Astadriel \fg \item \og Batcat \fg \item \og Eru \fg\item \og Gandarin \fg \item \og Groumbahk \fg \item \og Iluvatar \fg \item \og Lamronchak \fg \item \og Mammstein \fg \item \og Bigfish \fg }

\newcommand{\gazeofthegods}{Regard des Dieux}
\newcommand{\markofthedarkgods}[1]{Marque des Dieux Sombres\ifblank{#1}{}{ (#1)}}
\newcommand{\chosenbythegods}[1]{Choisi par les Dieux\ifblank{#1}{}{ (#1)}}
\newcommand{\lightningrage}{Rage Éclair}
\newcommand{\inspiregreatness}[1]{Inspire la Grandeur\ifblank{#1}{}{ (#1)}}
\newcommand{\giftsofthegods}{Dons des Dieux}
\newcommand{\giftofthegods}{Don des Dieux}
\newcommand{\survivalofthefittest}{Survie du Plus Apte}
\newcommand{\marksofthedaemonlegion}[1]{Marque des Légions Démoniaques\ifblank{#1}{}{ (#1)}}
\newcommand{\wrathpriest}{Prêtre du Courroux}
\newcommand{\wordsofscorn}{Murmures de Dédain}
\newcommand{\wordsofwrath}{Cris de Courroux}
\newcommand{\wordsofhate}{Souffle de Haine}
\newcommand{\osklanderjarl}{Jarl d'Osklander}
\newcommand{\makharkhan}{Jarl de Makhar}
\newcommand{\inspirebarbarians}{Inspire les Barbares}
\newcommand{\wastelandraiders}{Raid du Désespoir}
\newcommand{\striderswhip}{Fouet Cauchemardesque}
\newcommand{\trollbelch}{Bile de Troll}
\newcommand{\warpedregeneration}{Régénération du Warp}
\newcommand{\bloodbeastlink}{Lien de la Bête Sanguinaire}
\newcommand{\wavesofchange}{Vagues du Changement}
\newcommand{\mutantambush}{Embuscade Mutante}
\newcommand{\lumbering}{Char Moisi}
\newcommand{\hellmortar}{Canon Infernal}
\newcommand{\unchained}{Déchainé}
\newcommand{\hellscreammortar}{Mortier de l'Enfer}
\newcommand{\attentionfrombeyond}{Les Dieux Observent}
\newcommand{\giantattacks}{Attaques de Géant}
\newcommand{\MarkofTrueChaos}[1]{Marque du Chaos intégral\ifblank{#1}{}{ (#1)}}
\newcommand{\MarkofChange}[1]{Marque du Changement\ifblank{#1}{}{ (#1)}}
\newcommand{\MarkofLust}[1]{Marque de la Luxure\ifblank{#1}{}{ (#1)}}
\newcommand{\MarkofPestilence}[1]{Marque de la Pestilence\ifblank{#1}{}{ (#1)}}
\newcommand{\MarkofWrath}[1]{Marque du Courroux\ifblank{#1}{}{ (#1)}}

\newcommand{\TrueChaos}[1]{Chaos intégral\ifblank{#1}{}{ (#1)}}
\newcommand{\Change}[1]{Changement\ifblank{#1}{}{ (#1)}}
\newcommand{\Lust}[1]{Luxure\ifblank{#1}{}{ (#1)}}
\newcommand{\Pestilence}[1]{Pestilence\ifblank{#1}{}{ (#1)}}
\newcommand{\Wrath}[1]{Courroux\ifblank{#1}{}{ (#1)}}

\newcommand{\ToxicAttacks}{Attaques toxiques}
\newcommand{\daemonicinstability}{Instabilité démoniaque}

% Army common type special rules

\newcommand{\alliancecommonrules}{Alliance}
\newcommand{\allianceoptionscommonrules}{Options d'Alliance (pts)}

% QRS Invocation table

% Profile wording

% Profile rules

\newcommand{\markchart}{%
\textbf{Démon du Dieu Sombre}: Le Prince Démoniaque gagne les règles spéciales et l'accès aux différentes Disciplines de Magie en fonction du dieu auquel il appartient.
\vspace*{0.1cm}
\renewcommand{\arraystretch}{1.5}
\begin{center}\begin{tabular}{M{2cm}m{10cm}M{3cm}}
\hline
		& \centering\textbf{Bonus} & \centering\textbf{Discipline de Magie}\tabularnewline

\textbf{\TrueChaos{}}		   & Un démon de la marque du \TrueChaos{} gagne +1 en Commandement.	& \begin{flushleft}
Si Sorcier, il a accès aux Disciplines \alchemy, \death, \fire, \heavens ou \shadows.
\end{flushleft} \tabularnewline

\textbf{\Change{}}		   & Une figurine avec cette marque peut choisir à chaque phase une des règles suivantes pour n'importe quelle attaque : \divineattacks, \flamingattacks\ ou \hellfire. {Choisissez au début de chaque phase de combat et avant de tirer  avec cette figurine. Les Attaques Spéciales ne sont pas affectées. Un Sorcier peut choisir de relancer tous les dés de sa génération de sorts.} Si Sorcier, il a accès aux Disciplines \change\ ou \alchemy.	& \begin{flushleft}
Si sorcier, il a accès aux Disciplines \change\ ou \alchemy
\end{flushleft} \tabularnewline

\textbf{\Lust{}}		   & {Le porteur gagne la règle \armourpiercing{+1}.}	& \begin{flushleft}
Si Sorcier, il a accès aux Disciplines \lust\ ou \shadows.
\end{flushleft} \tabularnewline

\textbf{\Pestilence{}}		   & {Le porteur gagne les règles \poisonedattacks et \regeneration{5}. {Les \toxicattacks ont -1 pour blesser contre le porteur.}}	& \begin{flushleft}
Si Sorcier, il a accès aux Disciplines \disease\ ou \death.
\end{flushleft} \tabularnewline

\textbf{\Wrath{}}		   &  Une figurine avec cette marque gagne +1 en Force au premier round de chaque combat.	& \begin{flushleft}
Les Sorciers ne peuvent jamais prendre, ni gagner cette marque.
\end{flushleft} \tabularnewline

\hline
\end{tabular}\end{center}
}

\newcommand{\specialshootingweapon}{%
\vspace*{0.1cm}
\renewcommand{\arraystretch}{1.5}
\rowcolors{1}{}{lightgray}
\begin{center}\begin{tabular}{M{3cm}m{5cm}M{1cm}M{0.5cm}M{2cm}M{2cm}M{2cm}}
\hline
		& \centering\textbf{\underline{SPECIAL SHOOTING WEAPON}} & \centering\textbf{Portée} & \centering\textbf{S} & Tir Multiple & Blessure Multiple & \centering\textbf{\armourpiercing{}}\tabularnewline

Hellscream Cannon & Hellscream Cannon (1) Catapult (\distance{3}) & 12-60 & 4[9] & - & [Ordnance] & 1\tabularnewline

& Hellscream Cannon (2) Catapult (\distance{3}) & 12-24 & 3 & - & - & 1\tabularnewline

\hline
\end{tabular}\end{center}
}

\begin{document}

\newgeometry{margin=1in}

% Table options
\arrayrulecolor{black!30}
\setlength{\arrayrulewidth}{0.5pt}
\renewcommand{\arraystretch}{1.2}

\begin{titlepage}
\begin{center}

\ifdef{\booktitle}{}{\newcommand{\booktitle}{Missing title}}
\ifdef{\version}{}{\newcommand{\version}{Missing version}}

{\antiquefont\fontsize{40}{48}\selectfont\noindent\labels@fantasybattles

\labels@NinthAge}

\vspace*{0.5cm}
\ifdef{\booklogo}{%
\includegraphics[height=10cm]{\booklogo}%
}{%
\includegraphics[height=10cm]{../Layout/pics/logo_9th.png}%
}

\vspace*{-1cm}
{\antiquefont\fontsize{50}{60}\selectfont \booktitle
\vspace{0.4cm}

\fontsize{14}{16.8}\selectfont \labels@armyrules{}

Beta v\version{} - \today{}}

\ifdef{\frenchversion}{{\fontsize{14}{16.8}\selectfont \vspace{0.2cm}\noindent\texttt{VF \frenchversion}}}{}
\vfill

\begin{tabular}{@{}m{2cm}@{\hskip 20pt}m{13cm}@{}}
\includegraphics[width=2cm]{../Layout/pics/seal_9th.png} &
{\fontsize{10}{12}\selectfont \textcolor{black!50}{\noindent\labels@frontpagecredits}}

\ifdef{\frontpageaddstuff}{{\fontsize{10}{12}\selectfont \noindent\textcolor{black!50}{\frontpageaddstuff}}}{}

\vspace*{10pt}
\noindent{\fontsize{10}{12}\selectfont \textcolor{black!50}{\labels@license}}
\tabularnewline
\end{tabular}


\end{center}

\newpage

\thispagestyle{empty}

{\fontsize{10}{12}\selectfont

\begin{center}\noindent{\Largerfontsize\textbf{\labels@tableofcontents}}\end{center}

\vspace*{0.2cm}\begin{multicols}{2}

\tocfirstcolumn

\vspace*{\fill}\columnbreak

\tocentry{lordtitle}{\labels@lords}

\tocentry{herotitle}{\labels@heroes}

\ifdef{\tocmounts}{\tocentry{mountstitle}{\tocmounts}}{}

\tocentry{coretitle}{\labels@coreunits}

\tocentry{specialtitle}{\labels@specialunits}

\tocentry{raretitle}{\labels@rareunits}

\vspace*{\fill}\end{multicols}

\ifdef{\labels@introduction}{\vspace{0.7cm}\labels@introduction}{\vphantom{1pt}}
\vfill

\noindent\newrule{\labels@rulechanges}

\bigskip
\noindent \labels@latexcredit
}


\end{titlepage}

\restoregeometry

% Règles spéciales de l'Armée

\startarmyspecialrules

\armyspecialruleentry{\markofthedarkgods{}}\\
Chaque \markofthedarkgods{} entraîne des effets différents décrits ci-après. Chaque \markofthedarkgods{} est spécifiée entre parenthèses. Les figurines avec une \markofthedarkgods{} autre que \TrueChaos{} ne peuvent rejoindre que des unités avec la même \markofthedarkgods{} que la leur, ou des unités avec la \markofthedarkgods{Chaos Intégral}. La liste d'armée doit préciser quelle \markofthedarkgods{} a été choisie pour ces figurines. Les Sorciers ne peuvent pas rejoindre d'unité avec une \markofthedarkgods{Courroux} ou des figurines avec la règle \chosenbythegods{Courroux} et ne peuvent pas prendre de \markofthedarkgods{Courroux}. Une figurine avec la règle \markofthedarkgods{Courroux} ne peut rejoindre d'unité contenant au moins un Sorcier.\\

\begin{wrapfigure}[5]{L}{0cm}
\centering
\includegraphics[width=2cm]{pics/WDG1.png}
\end{wrapfigure}
\textbf{\MarkofTrueChaos{}}\\
Les unités dont la majorité des figurines possède la \markofthedarkgods{Chaos Intégral} peuvent relancer leurs tests de panique ratés. Les Sorciers ayant cette marque ont accès aux Disciplines suivantes : \alchemy, \heavens \only{Seigneur}, \fire, \death{} ou \shadows.\\

\vspace{0.5cm}

\begin{wrapfigure}[5]{L}{0cm}
\centering
\includegraphics[width=2cm]{pics/WDG2.png}
\end{wrapfigure}
\textbf{\MarkofChange{}}\\
Lorsqu'ils tirent, ou au début de chaque combat, les éléments de la figurine possédant la \markofthedarkgods{Changement} peuvent choisir une des règles spéciales suivantes qui s'applique à leurs attaques de tir de cette phase, ou aux attaques de corps à corps de ce combat : \flamingattacks, \divineattacks,{\magicalattacks} ou \hellfire. Chaque figurine de l'unité doit choisir le même effet. Les attaques spéciales ne sont pas affectées. {Les Personnages } {ajoutent +1 à leurs lancements de sorts.} Les Sorciers ayant cette marque ont accès aux Disciplines suivantes : \alchemy ou \change{}. \\ \\

\vspace{0.5cm}

\begin{wrapfigure}[5]{L}{0cm}
\centering
\includegraphics[width=2cm]{pics/WDG3.png}
\end{wrapfigure}

\textbf{\MarkofWrath{}}\\
Les éléments de figurines possédant la \markofthedarkgods{Courroux} gagnent +1 pour toucher au corps à corps {lorsqu'ils attaquent de front}, et ne peuvent pas Fuir en réaction à une charge.\\ \\

\vspace{0.5cm}

\begin{wrapfigure}[5]{L}{0cm}
\centering
\includegraphics[width=2cm]{pics/WDG4.png}
\end{wrapfigure}
\textbf{\MarkofLust{}}\\
Les unités avec une majorité de figurines ayant la \markofthedarkgods{Luxure} peuvent relancer leurs jets de charge, poursuite et de \randommovement{}. Elles sont immunisées à la panique. Les Sorciers ayant cette marque ont accès aux Disciplines suivantes : \lust{} ou \shadows.\\ \\

\vspace{0.5cm}

\begin{wrapfigure}[5]{L}{0cm}
\centering
\includegraphics[width=2cm]{pics/WDG4.png}
\end{wrapfigure}
\textbf{{\LARGE \MarkofPestilence{}}}\\
Les Attaques de corps à corps effectuées de front contre une figurine possédant la règle {\markofthedarkgods{Pestilence}} {reçoivent un malus de -1 pour toucher,} {mais touchent toujours sur 5+ ou mieux.} {Les éléments de figurine avec cette Marque ont} {-1 en Initiative (minimum 1).} Les Sorciers ayant cette marque ont accès aux Disciplines suivantes : \disease{} ou \death.

% Chosen of Gods

\armyspecialruleentry{\chosenbythegods{}}\\
Les figurines avec la règle \chosenbythegods{} gagnent les effets suivants {en plus des effets de leur \markofthedarkgods{}} :\\


\begin{tabular}{ll}

\begin{tabular}{ll}
\includegraphics[width=1cm]{pics/WDG1.png} & \begin{tabular}{p{6cm}} {\LARGE \textbf{\MarkofChange{}}}\\\wizardconclave{Niveau 1 : Bûcher Carmin, Feu d'Azur}.
Les {3 premiers} sorts de dégâts de la Discipline \change
lancés (réussis ou non) par des Sorciers dans l'unité gagnent +1 en Force. 
\end{tabular}\\
\end{tabular} 

&

\begin{tabular}{ll}
\includegraphics[width=1cm]{pics/WDG1.png} & \begin{tabular}{p{6cm}} {\LARGE \textbf{\MarkofLust{}}}\\
{+2} en Mouvement et \skirmisher,
{mais ne peuvent ajouter aucune figurine supplémentaire}
{à leur taille initiale.}
\end{tabular}\\
\end{tabular}\\

\begin{tabular}{ll}
\includegraphics[width=1cm]{pics/WDG1.png} & \begin{tabular}{l} {\LARGE \textbf{\MarkofPestilence{}}}\\Gagne \textsc{\fear} \end{tabular}\\
\end{tabular}

& 

\begin{tabular}{ll}
\includegraphics[width=1cm]{pics/WDG1.png} & \begin{tabular}{l} {\LARGE \textbf{\MarkofWrath{}}}\\ Gagne \textsc{\frenzy} \end{tabular}\\
\end{tabular}

\end{tabular}


\armyspecialruleentry{\gazeofthegods}
Une figurine bénéficiant de cette règle spéciale ne peut pas refuser un défi, et doit le lancer si aucune autre figurine ne le fait. Immédiatement après avoir tué un Personnage qui n’est pas un Champion au cours d’un défi, ou tué un monstre, {l'élément de figurine avec la règle \gazeofthegods peut relancer tous ses jets pour toucher et pour blesser jusqu'à la prochaine phase de magie du joueur actuel. Si deux figurines avec cette règle tuent un Monstre à la même étape d'Initiative, choisissez laquelle bénéficiera des effets bonus.}

\armyspecialruleentry{\lightningrage}
La figurine a une \wardsave{2} contre les attaques avec la règle \lightningattacks. Si une figurine est touchée par une \textit{Attaque Foudroyante}, elle gagne la règle spéciale \frenzy.

\armyspecialruleentry{\inspiregreatness{}}
Tant qu'au moins une figurine avec cette règle se trouve dans une unité du même type de troupe qu'elle-même, {toutes les figurines d'Infanterie de} l'unité peuvent effectuer une attaque de soutien supplémentaire au deuxième rang (mais pas au troisième).

\armyspecialruleentry{{\survivalofthefittest}}
{Votre armée ne peut inclure qu'au maximum 2 figurines avec à la fois les règles \largetarget et \fly{}. Cette limite passe à 4 pour une Grande Armée, et à 1 pour une Patrouille.}

\vspace{1cm}

\closearmyspecialrules

% Armurerie

\startarmyarmoury

{\Large \textbf{Lame démoniaque}}: Arme de corps à corps. Lorsqu'il utilise cette arme, le porteur gagne +1 en Force, \magicalattacks.

\closearmyarmoury

\newpage

% Gifts of the Dark Gods

\startarmynewsection{\giftsofthegods}

\spaceaftersection{}

\begin{tabular}{p{8cm}}
\begin{tabular}{c}
Les \giftsofthegods ne peuvent pas être dupliqués dans une armée, sauf mention contraire.
\end{tabular}
\end{tabular}

\begin{multicols}{2}\raggedcolumns

\startpricelist

\pricelistitem{Briseur Bestial}{70} Le porteur gagne la règle \terror. Si le porteur est votre Général, la limite maximale des figurines soumises à la règle \survivalofthefittest est augmentée de 1.

\pricelistitem{Ailes Démoniaques}{50} \only{Figurine à pied}. Le porteur gagne \fly{8}.

\pricelistitem{Voie du Déchu}{45} \only{Personnage d'Infanterie à pied}. Une figurine avec ce Don gagne +1 en Mouvement, +1 Point de Vie, -1 en Initiative, un socle de 40x40 mm et devient du type Infanterie Monstrueuse.

\pricelistitem{Troisième Œil du Changement}{Prince Démon 45-Les autres 30} Le porteur de ce \giftsofthegods{} augmente sa \wardsave{} d'un point, jusqu'à un maximum de 4+.

\pricelistitem{Sang de Sauvageon}{40} Le porteur gagne \inspiringpresence, qu'il soit le Général ou non. Mais seules les unités avec le type de troupe Bête de Guerre, Bête Monstrueuse, Infanterie Monstrueuse ou Monstre peuvent recevoir cette \inspiringpresence.\\
\\
Si le porteur est votre Général, sa \inspiringpresence n'est plus limitée aux unités ci-dessus. De plus, une unique unité de Trolls des Désolations (6 figurines maximum) peut être prise en Unité de Base.\\
\\
Si le porteur est Sorcier, il doit utiliser la Discipline \wilderness, même s'il n'y a pas accès.

\endpricelist

\columnbreak

\startpricelist

\pricelistitem{Miasmes Nécrosés}{40} \markofthedarkgods{Pestilence} seulement. La figurine gagne une \breathweapon{\ToxicAttacks{}}. De plus, à chaque Phase de combat, chaque figurine ennemie en contact avec le porteur subit une touche Force 1 avec la règle \armourpiercing{6} à Initiative 10.

\pricelistitem{Peau Tannée de la Désolation}{30} Le porteur gagne \innatedefence{6} {ou \innatedefence{5} si le porteur est du type Infanterie}. Ce \giftsofthegods{} ne peut être pris par une figurine montée sur Manticore.

\pricelistitem{Grâce Infernale}{25} \only{\markofthedarkgods{Luxure}}. Figurine à pied. Si le porteur est du type de troupe Infanterie, il gagne +2 en Mouvement, sinon il gagne la règle \swiftstride.

\pricelistitem{Suceur d'Âmes}{20} \only{\markofthedarkgods{Courroux}}. Si le porteur inflige au moins une blessure au corps à corps (sans compter les Attaques Spéciales), lancez 1D6 à la fin de la phase. Sur un résultat de 4+, le porteur regagne 1 Point de Vie. (Notez que cette règle peut être utilisée pour redonner des points de vie à une figurine avec profil combiné, mais pas à une figurine du type Monstre.)

\endpricelist

\end{multicols}

\closearmynewsection

% Magical Items

%\startarmymagicalitems

\closearmynewsection

\startarmymagicalitems

\begin{tabular}{c}
{Certains objets sont réservés à certaines figurines ayant 
la \markofthedarkgods{} ou le \giftofthegods{}.}
\end{tabular}

\armymagicalweapons

\startpricelist

\pricelistitem{Lame Brûlante du Chaos}{65}

{Type : \hw{}. Arme à une main. Les attaques faites par cette arme ont \flamingattacks, \multiplewounds{1D3}{} et \armourpiercing{6}. Après que toutes les attaques ont été résolues, pour chaque figurine tuée par cette épée, son unité subit une touche de Force 4 avec la règle \flamingattacks.}

\pricelistitem{Lance de Gagnir}{25}

{Type : Lance. Les Attaques effectuées avec cette arme gagnent +1 en Force et la règle \lethalstrike.}

\endpricelist

\armymagicalarmour

\startpricelist

\pricelistitem{Bouclier Runique du Dueliste}{10}

{Type: Bouclier. Au début de chaque phase de Combat, désignez une figurine ennemie en contact socle à socle avec le porteur. Pour la durée de cette phase, un élément de cette figurine (de votre choix) a -1 Attaque (minimum de 1).}

\endpricelist

\armyarcaneitems

\startpricelist

\pricelistitem{Idole Démoniaque}{35}
{À chaque fois qu'une figurine ennemie effectue un jet sur le tableau des Fiascos, vous pouvez augmenter ou réduire de 1 ce résultat.}

\endpricelist

\armymagicalbanners

\startpricelist

\pricelistitem{Étendard aux Neuf Queues}{50} - Infanterie seulement

{Les unités alliées uniquement d'infanterie n'ayant pas la règle \skirmisher à \distance{12} de cette bannière gagnent +1 en Mouvement.}

\pricelistitem{Bannière de Transmutation}{30}{\markofthedarkgods{Changement} seulement.}

{Toutes les attaques de tir faites contre cette unité ont -1 sur leurs jets pour blesser.}

\pricelistitem{Bannière de Furie}{25}{\markofthedarkgods{Courroux} seulement.} 

{Le porteur gagne la règle \frenzy et ne peut la perdre que si la bannière est capturée ou détruite. Toutes les figurines de l'unité du porteur gagnent la règle \frenzy, tant que le porteur a aussi la \frenzy et reste dans l'unité.}

\pricelistitem{Bannière de la Tentation}{25 - \markofthedarkgods{Luxure} seulement.}

{Une seule utilisation. La bannière peut être activée lorsque le porteur déclare une charge.} {La cible} chargée par le porteur de cette bannière ne peut que Maintenir sa Position, à moins qu'elle ne soit déjà en train de fuir. {L'unité peut déclarer une réaction de Fuite si elle est chargée ultérieurement par une autre unité.}

\pricelistitem{Bannière d'Immondices}{25}

{\markofthedarkgods{Pestilence} seulement. Toutes les figurines de l'unité du porteur ont \poisonedattacks.}

\endpricelist

\closearmymagicalitems


%%% START OF THE ARMYLIST - Translators shouldn't have to edit it %%%


%%% v0.99.9

\armylist

\lordstitle

\showunit{
	name={Prince Démoniaque},
	cost={245},
	profile={< 8 9 5 6 5 4 8 5 9},
	type=\monster{},
	basesize=50x50,
	unitsize={1},
	commontype=\alliancecommonrules,
	commonspecialrules={\markofthedarkgods{Chaos Intégral}},
	specialrules={\otherworldly, \daemonicinstability, \stubborn},
	options={
    	\magicalitemsallowance =\upto{}<25,
    	Peut choisir jusqu'à 2 \giftsofthegods =\unlimited,
 		{Armure de plates}=60,
		\optionschoice{Peut remplacer sa marque Chaos Intégral par une autre marque:}{
			{Changement}={30},
			{Courroux}={30},
			{Luxure}={20},
			{Pestilence}=40},	 		
 		\magiclevelchoice{
			\magiclevel{1}=40,
			\magiclevel{2}=65,
			\magiclevel{3}=130,
			\magiclevel{4}=160,
			},
		},
	additional={%
		\markchart
	},		
}

\showunit{
	name={Conquérant du Néant},
	cost={160},
	profile={< 4 8 3 5 5 3 7 5 9},
	type=\infantry{},
	basesize=25x25,
	unitsize=1,
	commontype=\alliancecommonrules,
	commonspecialrules={\markofthedarkgods{Chaos Intégral}},	
	specialrules={\inspiregreatness{}, \markofthedarkgods{Chaos Intégral}, \gazeofthegods},
	armour={Armure de plates},
	options={
	    Peut choisir un seul \giftofthegods{} = \unlimited,
		\magicalitemsallowance =\upto{}<100,
		\optionschoice{Peut remplacer sa marque Chaos Intégral par une autre marque:}{
			{Changement}={30},
			{Courroux}={30},
			{Luxure}={20},
			{Pestilence}=40},		
		\shield{}=5,
		\weapononechoice{
			\pw{}=10,
			Fléau=10,
			\halberd{}=15,
			\gw{}=15,
			\lance{}=15,
		},
	},
	mounts={
		Palanquin de Pestilence={15},		
		Broyeur=40,		
		Monture-Démon={40},
		Monture du Néant={40},
		Monture de Luxure={40},
		Char de la Désolation={40},
		Disque du Changement=60,
		Manticore=120,
		Dragon du Warp=270,
	},
}

\showunit{
	name={Seigneur Sorcier},
	cost={200},
	profile={< 4 5 3 4 4 3 5 3 8},
	type=\infantry{},
	basesize=25x25,
	unitsize=1,
	commontype=\alliancecommonrules,
	commonspecialrules={\markofthedarkgods{Chaos Intégral}},
	armour={Armure de plates},
	specialrules={\gazeofthegods},
	magiclevel=3,
	magicpaths={qui dépend de la \markofthedarkgods{}},
	options={
	    Peut choisir un seul \giftofthegods{} = \unlimited,	
		\magiclevel{4}=30,
		\magicalitemsallowance =\upto{}<100,
		\optionschoice{Peut remplacer sa marque Chaos Intégral par une autre marque:}{
			{Changement}={40},
			{Luxure}={10},
			{Pestilence}=15},		
	},
	mounts={
		Palanquin de Pestilence={15},
		Monture de Luxure={30},
		Monture du Néant={35},
		Char de la Désolation={40},
		Disque du Changement={50},		
		Monture-Démon={60},
		Manticore={120},
		Dragon du Warp={320},
	},
}





\heroestitle

\showunit{
	name={Héraut de l'Abîme},
	cost=100,
	profile={< 4 7 3 5 4 2 6 4 8},
	type=\infantry{},
	basesize=25x25,
	unitsize=1,
	commontype=\alliancecommonrules,
	commonspecialrules={\markofthedarkgods{Chaos Intégral}},
	armour={Armure de plates},
	specialrules={\inspiregreatness{}, \gazeofthegods},	
	options={
	    Peut choisir un seul \giftofthegods{} = \unlimited,		
		\optionschoice{Peut remplacer sa marque Chaos Intégral par une autre marque:}{
			{Changement}={20},
			{Courroux}={20},
			{Luxure}={10},
			{Pestilence}=30},		
		\bsb{}=25,
		\magicalitemsallowance{}=\upto{}<50,
		\shield{}=5,
		\weapononechoice{
			\pw{}=5,
			\flail{}=5,
			\halberd{}=10,
			\gw{}=10,
			\lance{}=15,
		},
	},	
	mounts={
		Monture du Néant=30,
		Palanquin de Pestilence={30},
		Broyeur=35,
		Monture de Luxure=35,
		Disque du Changement=50,
		Char de la Désolation={50},
		Monture-Démon={60},
		Manticore ={155},
	},
}

\showunit{
	name={Sorcier},
	cost=-,
	profile={< 4 5 3 4 4 2 4 2 8},
	type=\infantry{},
	basesize=25x25,
	unitsize=1,
	armour={Armure de plates},
	specialrules={\gazeofthegods},			
	additional={%
		\begin{center}\mustbecomeoneofthefollowingNOC{}\end{center}
		\setlength{\columnseprule}{0.5pt}
		\renewcommand{\columnseprulecolor}{\color{black!30}}
		\begin{multicols}{2}\raggedcolumns
\begin{center}\largerfontsize\antiquefont Sorcier (90 pts)\end{center}%	
		\textit{\large \alliancecommonrules{}}\newline
		\markofthedarkgods{Chaos Intégral}\newline
		\newline
		{\normalsize \textit{\large Magie:}}\newline
		\textbf{\magiclevel{1}}
		Utilise les sorts des Disciplines qui dépendent de sa \markofthedarkgods{}
		\def\tempoptions{%
		    Peut choisir un seul \giftofthegods{} = \unlimited,
			\magicalweaponallowance{}=\upto{}<50,
			\magiclevel{2}=25,
			\optionschoice{Peut remplacer sa marque Chaos Intégral par une autre marque:}{
			{Changement}={25},
			{Luxure}={10},
			{Pestilence}=15},
		}%

		\def\tempmounts{Monture de Luxure=20,
		Monture du Néant=25,
		Palanquin de Pestilence={30},
		Disque du Changement=50,
		Char de la Désolation={50},
		Monture-Démon={60},}%
		
		\def\tempmagie{magiclevel{3}}%	
		
		\vspace*{0.1cm}
		\options{\tempoptions}
		\vspace*{0.1cm}
		\mounts{\tempmounts}
		\columnbreak
		\begin{center}\largerfontsize\antiquefont 				        \wrathpriest{} (\pts{100})\end{center}
		\textit{{\large \alliancecommonrules}}\newline		
		\MarkofWrath{}\newline

		\def\tempspecialrules{\inspiregreatness{}, \magicresistance{2}}%
		\specialrules{\tempspecialrules}%
		\vskip 0.1cm
		\textbf{\wordsofscorn{}}: {Peut dissiper comme s'il était un \magiclevel{3}.}
		\def\tempoptions{%
		    Peut choisir un seul \giftofthegods{} = \unlimited,
			\magicalweaponallowance{}=\upto{}<50,
			\weapononechoice{
				\flail{}=5,
				\pw{}=10,			
			},
		}%

		\options{\tempoptions}		
		\def\tempmounts{Monture du Néant=25,
		Broyeur={30},
		Char de la Désolation={50},
		Monture-Démon={60},}%		
		\mounts{\tempmounts}%
		\end{multicols}
		\setlength{\columnseprule}{0pt}
	},
}

\showunit{
	name={Jarl Barbare},
	cost=45,
	profile={< 4 5 4 4 4 2 5 3 8},
	type=\infantry{},
	basesize=25x25,
	unitsize=1,
	commontype=\alliancecommonrules,
	commonspecialrules={\markofthedarkgods{Chaos Intégral}},
	armour={Armure légère},
	specialrules={\textbf{\inspirebarbarians{}}: lorsque cette figurine rejoint une unité de Barbares ou de Cavaliers Barbares, celle-ci gagne la règle \fightinextrarank{}. 
	Une unité ne peut bénéficier que d'une de ces règles : \inspirebarbarians{} ou \inspiregreatness{}. Choisissez laquelle au début de chaque phase de combat},	
	options={
	    Peut choisir un seul \giftofthegods{} = \unlimited,		
		\optionschoice{Peut remplacer sa marque Chaos Intégral par une autre marque:}{
			{Changement}={5},
			{Luxure}={5},			
			{Courroux}={10},
			{Pestilence}=15},		
		\bsb{}\refsymbol{}=25,
		{\refsymbol{} uniquement si un autre Chef Barbare est Général},
		\magicalitemsallowance{}=\upto{}<50,
		\shield{}=4,
		\ha{}=5,
		\weapononechoice{
			\pw{}=3,
			\flail{}=3,
			\spear{}=3,
			\lance{}=3,
			\gw{}=4,
		},
	},	
	mounts={
		Cheval de guerre=20,
	},
	additional={\refsymbol{} Uniquement si un autre Chef Barbare est Général},
	additional={%
		\begin{center}Peut avoir une des améliorations suivantes\end{center}
		\setlength{\columnseprule}{0.5pt}
		\renewcommand{\columnseprulecolor}{\color{black!30}}
		\begin{multicols}{2}\raggedcolumns
		\begin{center}\largerfontsize\antiquefont \osklanderjarl{} (40 pts) - \oneofakind, \only{Infanterie}\end{center}%	
		\def\tempspecialrules{Cette figurine gagne \vanguard{} et \ambush{}. {Avant le début de la partie, vous pouvez} choisir une unité de Barbares qui possède la même Marque que le \osklanderjarl, celle-ci gagne \vanguard. {Si elle est composée de 25 figurines maximum, elle gagne aussi} \ambush{}. Ce Jarl Barbare doit être déployé dans cette unité}%
		\specialrules{\tempspecialrules}%		
		\columnbreak
		\begin{center}\largerfontsize\antiquefont 				        \makharkhan{} (\pts{30}) - \oneofakind \end{center}%
		\def\tempspecialrules{Ce Jarl Barbare est monté sur un Cheval de Guerre (gratuitement). Cette figurine gagne {\thunderouscharge}. Si cette figurine rejoint une unité de Cavaliers Barbares, cette unité gagne également \thunderouscharge}%
		\specialrules{\tempspecialrules}%
		\end{multicols}
		\setlength{\columnseprule}{0pt}
	},	
}





\mountstitle

La section Montures concerne les montures de personnages. Les montures pour non-personnages suivent les règles données dans leur description d'unité.

\showunit{
	name={Cheval de Guerre},
	profile={< 8 3 - 3 3 1 3 1 5,},
	type=\warbeast{},
	basesize=25x50,
	specialrules={\fastcavalry},
	armour={\mountsprotection{6}},
	options={
		Peut échanger \fastcavalry{} avec \mountsprotection{5}=10,
	},
}

\showunit{
	name={Monture du Néant},
	profile={< 8 3 - 4 3 1 3 1 5},
	type=\warbeast{},
	basesize=25x50,
	armour={\mountsprotection{6}, Caparaçon},
}

\showunit{
	name={Monture-Démon},
	profile={< 8 4 - 5 5 3 2 2 8},
	type=Bête Monstrueuse,
	basesize=50x50,
	specialrules={\magicalattacks, \fear},
	armour={\mountsprotection{6}},
	options={{Caparaçon}=10},
}

\showunit{
	name={Disque du Changement},
	profile={< 1 3 - 4 4 1 4 3 7},
	type=Bête de Guerre,
	basesize=50x50,
	specialrules={\magicalattacks, \fly{{8}}},
	armour={\mountsprotection{6}},
	unitrules={Le Personnage doit posséder la \MarkofChange{}},
}

\showunit{
	name={Palanquin de Pestilence},
	profile={< 4 3 3 3 3 3 3 6 7},
	type=Infanterie,
	basesize=50x50,
	specialrules={\poisonedattacks, \magicalattacks},
	armour={\mountsprotection{6}},
	unitrules={Le Personnage doit posséder la \MarkofPestilence{}},
}

\showunit{
	name={Broyeur},
	profile={< 7 5 - 5 4 3 2 3 7},
	type=Bête Monstrueuse,
	basesize=50x75,
	specialrules={\magicalattacks, \fear},
	armour={\mountsprotection{6}},
	unitrules={Le Personnage doit posséder la \MarkofWrath{}},
}

\showunit{
	name={Monture de Luxure},
	profile={< 10 3 - 3 3 1 5 1 7},
	type=Bête de Guerre,
	basesize=25x50,
	specialrules={\poisonedattacks, \magicalattacks, \vanguard},
	armour={\mountsprotection{6}},
	unitrules={Le Personnage doit posséder la \MarkofLust{}},
}

\showunit{
	name={Char de la Ruine},
	profile={Char < - - - 5 5 4 - - -,
			 Équipage (1) < - 5 3 4 - - 4 2 8,
			 Monture du Néant (2) < 8 3 - 4 - - 3 1 -
			 },
	type=Char,
	basesize=50x100,
	specialrules={\markofthedarkgods{la même marque que celle du Personnage monté, uniquement l'équipage}, {\impacthits{+1}}},
	unitrules={\unitrule{{\wastelandraiders}}{{N'importe quel Char de la Ruine avec paire de Monture du Néant pris dans l'armée avec une figurine avec cette règle peut être pris en unité de base au lieu d'unité spéciale. \newline Note : Le Personnage doit alors payer 20 pts supplémentaire par unité de ce type prise en choix de base.}}},
	weapons={Hallebarde (Équipage seulement)},
	armour={\mountsprotection{6}},
	options={
		Caparaçon=20,
		{\wastelandraiders{} \only{Général}}=\permodel{}<20,
	},
}

\showunit{
	name={Manticore},
	profile={< 6 5 - 5 5 4 5 3 5},
	type=Bête Monstrueuse,
	basesize=50x100,
	specialrules={\multiplewounds{1D3}{}, \lethalstrike, \frenzy, \largetarget, \fear, {\survivalofthefittest}, \fly{8}},
}

\showunit{
	name={Dragon du Néant},
	profile={< 6 5 1 6 6 6 3 6 9},
	type=Monstre,
	basesize=50x100,
	specialrules={\breathweapon{Force 4, \flamingattacks}, \breathweapon{Force 3, \armourpiercing{3}}, {\survivalofthefittest}, \fly{7}},
	armour={\innatedefence{3}},	
}




\coreunitstitle

\showunit{
	name={Guerriers du Néant},
	cost={110},
	profile={< 4 5 3 4 4 1 4 2 8},
	type=Infanterie,
	unitsize=10,
	costpermodel=13,
	maxmodels=20,
	basesize=25x25,
	specialrules={\markofthedarkgods{Chaos Intégral}},
	armour={Armure de plates, {Bouclier}},
	options={
		\optionschoice{Peut remplacer sa marque Chaos Intégral par une autre marque ({max 25 figurines})}{
			{Changement}=\permodel{}<1,
			{Courroux}=\permodel{}<2,
			{Luxure}=\permodel{}<1,
			{Pestilence}=\permodel{}<3},
		\optionschoice{Peut avoir (un seul choix)}{
		{Paire d'armes}=\permodel{}<1,
		{Arme lourde} =\permodel{}<2,
		Hallebarde =\permodel{}< 3},
	},
    commandgroup={champion=10,  musician=10, banner=10, veteranstandardbearer=yessir},
}

\showunit{
	name={Déchus},
	cost={85},
	profile={< 6 4 - 4 4 1 4 * 8},
	type=Infanterie,
	unitsize=5,
	costpermodel=15,
	maxmodels=7,
	basesize=25x25,
	specialrules={\randomattacks{1D3}, \frenzy, \immunetopsychology, \markofthedarkgods{Chaos Intégral}, \skirmishers},
	armour={Armure de plates},
	options={
		\optionschoice{Peut remplacer sa marque Chaos Intégral par une autre marque}			{
			{Changement}=\permodel{}<1,
			{Courroux}=\permodel{}<2,
			{Luxure}=\permodel{}<1,
			{Pestilence}=\permodel{}<3},
	},
	commandgroup={champion=10},
}

\showunit{
	name={Molosses de Guerre},
	cost={45},
	profile={< 7 4 - 3 3 1 3 1 5},
	type=Bête de Guerre,
	unitsize=5,
	maxmodels=30,
	costpermodel=4,
	basesize=25x50,
	specialrules={\poisonedattacks, \vanguard, \insignificant},
	options={
		\innatedefence{{5}}=\permodel{}<2,
	},
}

\showunit{
	name={Barbares},
	cost={70},
	profile={< 4 4 3 3 3 1 3 1 7},
	type=Infanterie,
	unitsize={20},
	maxmodels={30},
	costpermodel=5,
	basesize=25x25,
	specialrules={\markofthedarkgods{Chaos Intégral}},
	armour={Armure légère},
	options={
		\optionschoice{Peut remplacer sa marque Chaos Intégral par une autre marque}{
			{Changement}=\permodel{}<1,
			{Courroux}=\permodel{}<1,
			{Luxure}=\permodel{}<1,
			{Pestilence}=\permodel{}<2},
		{Armes de jet} =\permodel{}<1,
		\optionschoice{Peut prendre (un seul choix)}{
		Paire d'armes=\free,
		Bouclier=\permodel{}<1,
		Lance {et bouclier}=\permodel{}<1,
		{Fléau}=\permodel{}<2,
		Arme lourde=\permodel{}<3,}
	},
	commandgroup={champion=10, banner=10, veteranstandardbearer=yessir, musician=10},
}

\showunit{
	name={Cavaliers Barbares},
	cost={75},
	profile={Barbare < 4 4 3 3 3 1 3 1 7,
			 Cheval de guerre < 8 3 - 3 3 1 3 1 5},
	type=Cavalerie,
	unitsize=5,
	maxmodels=10,
	costpermodel=11,
	basesize=25x50,
	specialrules={\fastcavalry, \markofthedarkgods{Chaos Intégral, Barbare uniquement}},
	armour={Armure légère, {\mountsprotection{6}}},
	options={
		\optionschoice{Peut remplacer sa marque Chaos Intégral par une autre marque}{
			{Changement}=\permodel{}<1,
			{Courroux}=\permodel{}<2,
			{Luxure}=\permodel{}<2,
			{Pestilence}=\permodel{}<2},
		{Peut remplacer \fastcavalry par \mountsprotection{5}}=\permodel{}<1,
		Bouclier=\permodel{}<1,
		Armes de jet =\permodel{}<2,
		\optionschoice{Peut prendre (un seul choix)}{
		Lance légère =\permodel{}<1,
		Fléau =\permodel{}<1,},
	},
	commandgroup={champion=10, banner=10, veteranstandardbearer=yessir, musician=10},
}






\specialunitstitle

\showunit{
	name={Char de la Ruine},
	cost={95},
	profile={Char < - - - 5 5 - - - -,
			 Équipage (2) < - 5 3 4 - - 4 2 8,
			 [Monture du Néant (2)] < 8 3 - 4 - 4 3 1 5,
			 [Bête Broyeuse (1)] < 6 4 - 5 - 6 2 3 6
			 },
	type=Char,
	unitsize=1,
	basesize=50x100,
	specialrules={\markofthedarkgods{Chaos Intégral}(Équipage seulement), \impacthits{+1}},
	unitrules={
		\unitrule{Bête Broyeuse}{{Une figurine tirée par une Bête Broyeuse gagne les règles \grindingattacks{1D3} \only{Bête Broyeuse}, \fear{} et \mountsprotection{6}. \\
		Elle peut avoir \mountsprotection{5} pour un coût de 15 pts..}}},
	armour={Armure de plates},
	weapons={Hallebarde (Équipage seulement)},
	options={
		\optionschoice{Peut remplacer sa marque Chaos Intégral par une autre marque}{
			{Changement}=10,
			{Courroux}=10,
			{Luxure}=20,
			{Pestilence}=20},
		\optionschoice{{Doit choisir les montures du Char (un seul choix)}}{
			{Monture du Néant}=\free,
			{Bête Broyeuse}=45},
	}
}

\showunit{
	name={Chevaliers de la Destruction},
	cost={170},	
	profile={Chevalier < 4 5 3 4 4 1 5 2 8,
			 Monture du Néant < 8 3 - 4 3 1 3 1 5},
	type=Cavalerie,
	unitsize=5,
	maxmodels=5,
	costpermodel=32,
	basesize=25x50,
	specialrules={\markofthedarkgods{Chaos Intégral}(Chevalier uniquement), \fear},
	armour={Armure de plates, Bouclier, Caparaçon, \mountsprotection{6}},
	weapons={Lance de cavalerie (Chevalier uniquement)},
	options={
		\optionschoice{Peut remplacer sa marque Chaos Intégral par une autre marque}{
			{Changement}=\permodel{}<2,
			{Courroux}=\permodel{}<3,
			{Luxure}=\permodel{}<3,
			{Pestilence}=\permodel{}<4},
		Peut remplacer sa Lance de cavalerie par une Lame démoniaque =\permodel{}<4,
	},
	commandgroup={champion=10, banner=10, bannerallowance=50, musician=10},
}

\showunit{
	name={Bénis par les Dieux},
	cost={120},
	profile={< 4 6 3 4 4 1 5 2 8},
	type=Infanterie,
	unitsize=10,
	maxmodels={15},
	costpermodel=12,
	basesize=25x25,
	specialrules={\immunetopsychology, \chosenbythegods{Chaos Intégral}},
	armour={Armure de plates, {Bouclier}},
	options={
		\optionschoice{Peut remplacer sa marque Chaos Intégral par une autre marque}{
			{Changement}=\permodel{}<2,
			{Courroux}=\permodel{}<3,
			{Luxure}=40,
			{Pestilence}=\permodel{}<4},
		\optionschoice{Peut prendre (un seul choix)}{
		{Paire d'armes}=\permodel{}<1,
		{Arme lourde}=\permodel{}<2,
		Hallebarde=\permodel{}<3,
		},
	},
	commandgroup={champion=10, championallowance=25, banner=10, bannerallowance=50, musician=10},
}



\showunit{
	name={Favoris des Dieux},
	cost={100},
	profile={< 5 5 3 4 4 3 4 3 8},
	type=Infanterie Monstrueuse,
	unitsize=3,
	maxmodels=6,
	costpermodel=30,
	basesize=40x40,
	specialrules={\chosenbythegods{Chaos Intégral}},
	armour={Armure de plates},
	options={
		\optionschoice{Peut remplacer sa marque Chaos Intégral par une autre marque}{
			{Changement}=\permodel{}<4,
			{Courroux}=\permodel{}<6,
			{Luxure}=50,
			{Pestilence}=\permodel{}<8},
		Bouclier=\permodel{}<3,
		\optionschoice{Peut prendre (un seul choix)}{
		{Paire d'armes}=\permodel{}<3,
		{Fléau}=\permodel{}<5,
		{Hallebarde}=\permodel{}<7,
		{Arme lourde}=\permodel{}<7,
		},
	},
	commandgroup={champion=10, championallowance=25, banner=10, bannerallowance=25, musician=10},
}


\showunit{
	name={Arpenteurs de Luxure},
	cost={{75}},
	profile={Arpenteur < 4 4 4 3 3 1 5 1 7,
			 Monture de Luxure < 10 3 - 3 3 1 5 1 7},
	type=Cavalerie,
	unitsize=5,
	maxmodels=10,
	costpermodel=10,
	basesize=25x50,
	specialrules={\poisonedattacks{} (Monture de Luxure uniquement), \fastcavalry{}, \MarkofLust{} \only{Arpenteur}, \lightningreflexes{} (Arpenteur uniquement)},
	armour={Bouclier, Lance légère, \striderswhip, \mountsprotection{6}},
	unitrules={\unitrule{\striderswhip}{Arme de tir avec le profil suivant : portée 6", Force de l'utilisateur, \quicktofire{}.\newline {Les unités qui subissent au moins une touche de cette arme {voient leur Initiative réduite à 1 au corps à corps}, et gagnent la règle \stupidity. Ces effets durent jusqu'au début de votre prochaine phase de Tir.}}},
	commandgroup={champion=10, banner=10, musician=10},
}

\showunit{
	name={Trolls des Désolations},
	cost={100},
	profile={< 6 3 1 5 4 3 1 3 4},
	type=Infanterie Monstrueuse,
	unitsize=3,
	maxmodels={7},
	costpermodel=42,
	basesize=40x40,
	specialrules={\trollbelch, {\fear}, \regeneration{4}, \stupidity},
	unitrules={
		\unitrule{\trollbelch}{À la place de ses attaques normales, n'importe quel Troll  peut choisir d'effectuer une seule attaque qui touche automatiquement, à Force 5 et \armourpiercing{6}.}
	},
	options={
		{Peut recevoir la \MarkofPestilence{}}=\permodel{}<6,
		Paire d'armes=\permodel{}<3,
	},
}

\showunit{
	name={Centaures-Dragon},
	cost={205},
	profile={< 7 4 2 5 5 4 2 3 8},
	type=Bête Monstrueuse,
	unitsize=3,
	maxmodels=2,
	costpermodel=68,
	basesize=50x75,
	specialrules={\markofthedarkgods{Chaos Intégral}, \stomp{2}, \lightningrage},
	armour={Armure légère, \innatedefence{5}},
	options={
		\optionschoice{Peut prendre (un seul choix)}{
		Paire d'armes=\permodel{}< 3,
		Hallebarde =\permodel{}< 6,
		{Arme lourde} =\permodel{}< 10,
		},
	},
	commandgroup={champion=10, banner=10, musician=10},
}

\showunit{
	name={Progéniture Mutante},
	cost={65},
	profile={< * 3 - 4 5 3 2 * 10},
	type=Bête Monstrueuse,
	unitsize=1,
	basesize=40x40,
	specialrules={\randomattacks{1D6+1}, \mutantambush, \unbreakable, \markofthedarkgods{Chaos Intégral}, \randommovement{3D6}, \fear},
	unitrules={\unitrule{\mutantambush}{Cette règle fonctionne comme une \ambush normale avec une règle supplémentaire. Lorsque la Progéniture Mutante entre en jeu, elle peut faire un \randommovement{2D6} comme si elle était dans la sous-phase des Mouvements Obligatoires, mais en traitant toutes les unités (amies comme ennemies) comme des Terrains Infranchissables.}},
	options={
		\optionschoice{Peut remplacer sa marque Chaos Intégral par une autre marque}{
			{Changement}=5,
			{Courroux}=5,
			{Luxure}=15,
			{Pestilence}=10},
	},
}

\showunit{
	name={Bête Sanguinaire},
	cost=175,
	profile={< 7 3 0 6 5 5 3 5 4},
	type=Monstre,
	unitsize=1,
	basesize=50x100,
	specialrules={\frenzy, \hatred, \bloodbeastlink},
	unitrules={\unitrule{\bloodbeastlink}{Lorsque vous déployez la Bête Sanguinaire, choisissez un personnage de votre armée pour qu'il devienne son maître. Tant que le maître est en vie, la Bête Sanguinaire peut utiliser sa Capacité de Combat et son Commandement. Un personnage ne peut être le maître que d'une Bête Sanguinaire. Une Bête Sanguinaire ne peut recevoir la \inspiringpresence{} du Général.}},
	armour={\innatedefence{4}},	
}






\rareunitstitle

\showunit{
	name={Broyeurs du Courroux},
	cost={170},
	profile={Cavalier < 4 5 3 4 4 1 5 2 8,
			 Broyeur < 7 5 - 5 4 3 2 3 7},
	type=Cavalerie Monstrueuse,
	unitsize=2,
	maxmodels=3,
	costpermodel=55,
	basesize=50x75,
	specialrules={\magicalattacks{} (Broyeur uniquement), \frenzy{} (Cavalier uniquement), \MarkofWrath{}, \fear},
	armour={Armure de plates, Bouclier, \mountsprotection{6}},
	options={
		\optionschoice{Peut prendre une arme (un seul choix)}{
		{Lance de cavalerie}=\permodel{}<2,
		{Lame démoniaque}=\permodel{}<6,}
	},
	commandgroup={champion=10, banner=10, bannerallowance=50, musician=10},
}

\showunit{
	name={Canon Hurleur},
	cost={190},
	profile={< 4 4 3 5 6 5 1 4 7},
	type=Monstre,
	unitsize=1,
	basesize=100x150,
	specialrules={\hellmortar, \otherworldly, \frenzy, {\daemonicinstability}, \hellscreammortar, {\stubborn}},
	unitrules={
		\unitrule{\hellmortar}{Catapulte (\distance{3}) avec le profil suivant : portée \distance{12-60}, Force 4[9], [\multiplewounds{\ordnance}{}], \armourpiercing{1}, {\moveorfire}. Une unité subissant au moins une perte du \hellmortar doit immédiatement effectuer un test de panique, comme si elle avait subi 25\% de perte.}
		\unitrule{\hellscreammortar}{{Catapulte (\distance{3}) avec le profil suivant : portée \distance{6-24}, Force 3, \armourpiercing{1}. Une unité subissant au moins une perte du \hellmortar doit immédiatement effectuer un test de panique, comme si elle avait subi 25\% de perte.}}},
	armour={\innatedefence{5}},	
}

\showunit{
	name={Autel de Bataille},
	cost={130},
	profile={ Prêtre (1) < - 5 3 4 - - 4 2 8,
			 Porteur (1) < 5 3 - 4 5 5 2 * 7},
	type=Infanterie Monstrueuse,
	unitsize=1,
	basesize=50x100,
	specialrules={\randomattacks{3D3} (Porteur uniquement), {\lumbering}, \largetarget, \attentionfrombeyond, \markofthedarkgods{Chaos Intégral}, \fear, \wardsave{4}},
	unitrules={
	\unitrule{\attentionfrombeyond}{Cette figurine donne {\hardtarget} aux unités amies dans un rayon de \distance{{6}}. Les figurines avec la règle \largetarget{} ne bénéficient pas de cette règle. Cette figurine gagne aussi un effet spécial en fonction de sa \markofthedarkgods{}. Si elle gagne un objet de sort du type Augmentation, alors elle ne peut que cibler les unités amies dont la majorité des figurines partage la même \markofthedarkgods{}, ou les unités avec \markofthedarkgods{Chaos Intégral}. Il en va de même pour les sorts du type Aura.\\
	\newline
	\begin{tabular}{M{1.5cm}|m{5cm}}
	\hline
	Chaos Intégral & objet de sort (Puissance 5) :\\
	& contient le sort Ouragan de la Discipline \heavens.\\
	\hline
	Changement & objet de sort (Puissance 5) :\\
	& contient le sort Éclair Fluctuant de la Discipline \change.\\
	\hline
	Courroux & Cette figurine donne la règle spéciale \magicresistance{2} à toutes les unités amies {dont la majorité des figurines} {ont \MarkofWrath{}} {ou \markofthedarkgods{Chaos Intégral}} dans les \distance{6}.\\
	\hline
	Luxure & objet de sort (Puissance 5) :\\
	& contient le sort Hystérie de la Discipline \lust.\\
	\hline
	Pestilence & objet de sort (Puissance 5) :\\
	& contient le sort {Miasmes de Pourriture} de la Discipline {\disease}.\\
	\hline
	\end{tabular}
	}
},
	armour={Armure lourde, \mountsprotection{6}},
	options={
		Peut remplacer sa marque Chaos Intégral par une autre marque = 10,
	},
}

\showunit{
	name={Chimère},
	cost={200},
	profile={< 6 4 - 6 5 4 3 7 5},
	type=Monstre,
	unitsize=1,
	basesize=50x100,
	specialrules={\regeneration{5}, \survivalofthefittest, \fly{8}},
	armour={\innatedefence{4}},	
	options={
		\breathweapon{Force 4, \flamingattacks}=30,
	},
}

\showunit{
	name={Monstruosité à Vortex},
	cost=175,
	profile={< 6 4 - 5 5 5 3 * 8},
	type=Monstre,
	unitsize=1,
	basesize=50x100,
	specialrules={\randomattacks{1D6+2}, \hardtarget, \channel, {\markofthedarkgods{Changement}}, \wardsave{5}, \wavesofchange},
	unitrules={\unitrule{\wavesofchange}{Les unités ennemies à \distance{6} ou moins d'une figurine avec cette règle spéciale sont affectées par la \fear{} comme si elles étaient en contact avec celle-ci. Chaque unité amie à \distance{6} ou moins d'une figurine avec cette règle spéciale est immunisée à la \fear{}.\newline
	 Les Sorciers alliés peuvent lancer des sorts du type Dommage à travers cette figurine si elle se trouve dans un rayon de \distance{24} du lanceur. Lorsque le Sorcier utilise cette faculté, mesurez la distance à partir de la Monstruosité à Vortex, et utilisez son arc de vue et sa ligne de vue. Un Sorcier peut lancer des sorts du type Projectile même s'il est engagé en combat tant que la Monstruosité à Vortex ne l'est pas. Si le sort est un fiasco, le Sorcier subit les effets habituels tandis que la Monstruosité à Vortex subit une touche de Force NDP+2.}},
	armour={\innatedefence{5}},	
}


\showunit{
	name={Centaure-Dragon Ancien},
	cost={230},
	profile={< 7 6 3 6 6 6 4 5 9},
	type=Monstre,
	unitsize=1,
	basesize=50x75,
	specialrules={\immunetopsychology, \lightningrage, \swiftstride},
	armour={\innatedefence{4}},	
	options={
		{Armure légère}=20,
		\optionschoice{Peut prendre une seule arme}{
		{Paire d'armes}= 20,
		{Arme lourde} =20,
		{Hallebarde} =20,
	}},
}



\showunit{
	name={Géant des Désolations},
	cost=140,
	profile={Géant des Désolations < 6 3 - 6 5 6 3 * 10},
	type=Monstre,
	unitsize=1,
	basesize=50x75,
	specialrules={\giantattacks, {\chosenbythegods{Chaos Intégral}}, \immunetopsychology, \stubborn},
	unitrules={\unitrule{{\chosenbythegods{Changement}}}{{Un \chosenbythegods{Changement} peut devenir Champion pour 90 pts.}}},
	options={
		\optionschoice{Peut remplacer sa marque Chaos Intégral par une autre marque}{
			{Changement}=30,
			{Courroux}=\free,
			{Luxure}=60,
			{Pestilence}=10},
	},
	additional={
		\begin{description}
			\item[*Attaques de Géant :] Quand un Géant attaque au corps à corps, au lieu d'attaquer normalement, choisissez une unité en contact socle à socle avec lui qui va subir son attaque, lancez 1D6 et consultez ce que donne le résultat de son attaque dans l'une des deux tables ci-dessous en fonction du type de troupe de l'unité. Il est important de noter que l'attaque du Géant compte comme une attaque de corps à corps et suit ainsi normalement les règles des attaques de corps à corps. Le Géant peut également faire son \stomp{} normalement.
			\setlength\multicolsep{12.0pt plus 4.0pt minus 3pt}
			\begin{multicols}{2}
				Si l'unité est de type Infanterie, Bête de Guerre, Nuée, Machine de Guerre, Cavalerie :	
				\renewcommand{\arraystretch}{1.5}	
				\begin{center}\begin{tabular}{cl}
   					\hline
					1 & Hurle \tabularnewline
					2 & Saute \tabularnewline
					3 & Ramasse \tabularnewline
					4-6 & Frappe \tabularnewline
					\hline
				\end{tabular}\end{center}
				Si l'unité est de type Bête Monstrueuse, Infanterie Monstrueuse, Cavalerie Monstrueuse, Monstre, Monstre Monté, Char :
				\begin{center}\begin{tabular}{cl}
					\hline	
					1 & Hurle \tabularnewline
					2-3 & Tape comme un sourd \tabularnewline
					4-6 & Fracasse \tabularnewline
					\hline
				\end{tabular}\end{center}
				\renewcommand{\arraystretch}{1.2} % return to default
			\end{multicols}
			\item[Hurle :] Ni le Géant, ni l'unité sélectionnée par le Géant ne peuvent faire d'attaques au cours de cette Phase de Corps à Corps. Les attaques déjà réalisées, incluant celles réalisées simultanément avec cette attaque, ne sont pas concernées. Le camp du Géant gagne automatiquement le combat de 2. Si deux Géants opposés, ou plus, \og Hurlent \fg , le résultat du combat est un match nul.
			\item[Saute :]  L'unité sélectionnée subit 1D6 touches avec la Force du Géant. Le Géant doit faire un test de Terrain Dangereux (1).
			\item[Ramasse :] Choisissez une figurine dans l'unité préalablement sélectionnée et en contact socle à socle avec le Géant. Cette figurine doit faire un test de Force et un test de Capacité de Combat. Pour chaque test raté, la figurine subit une touche avec la Force du Géant et suivant la règle spéciale \multiplewounds{1D3}{}.
			\item[Frappe :] Le Géant fait 2D6 attaques sur l'unité qu'il a préalablement choisie.
			\item [Tape comme un sourd :] Choisissez une figurine en contact socle à socle avec le Géant dans l'unité préalablement sélectionnée. Cette figurine doit faire un test d'Initiative. Si elle échoue, la figurine subit 2D6 blessures avec la règle spéciale \armourpiercing{6}.
			\item[Fracasse :] Choisissez une figurine dans l'unité préalablement sélectionnée et en contact socle à socle avec le Géant. Cette figurine subit une blessure avec la règle spéciale \armourpiercing{6}. Si la figurine n'a pas encore attaqué, elle ne peut pas le faire au cours de cette manche. Si la figurine a déjà réalisé ses attaques, elle ne pourra pas attaquer au cours du tour à venir de l'autre joueur.
		\end{description}
	},
}


%%% Quick Reference Sheet - AB_qrs.tex is automatic and shouldn't be edited %%%

\quickrefsheettitle

\input{../Layout/AB_qrs.tex}

\specialshootingweapon

\end{document}
