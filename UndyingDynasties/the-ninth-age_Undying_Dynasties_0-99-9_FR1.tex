

\documentclass[a4paper,8pt]{extarticle} % extarticle allows to use font size of 8pt.

\usepackage[a4paper, top=1.6cm, bottom=2cm, left=1.6cm, right=1.6cm]{geometry} % Marge reduction.

%% Language specific package
\usepackage[french]{babel}
\frenchbsetup{StandardLists=true} % Necessary to use enumitem with babel/french.

%% Font and typing packages
\usepackage{fontspec}
\setmainfont[
	Ligatures=TeX,
	ItalicFont={Dancing Script},
	BoldItalicFont={Dancing Script}
	]{PT Serif} % default is Latin Modern
\newfontfamily\antiquefont[Ligatures=TeX]{Caslon Antique} % fancy font
\usepackage{microtype}			% Greatly improves general appearance of the text.
\usepackage{SIunits}			% Unit appearance.
\usepackage{xspace}				% Define commands that appear not to eat spaces.
\usepackage{ulem}				% To cross words out. Use \sout{}.

%% Array utilities
\usepackage{array}				% Additionnal options for arrays.
\usepackage{colortbl}			% Additionnal options for coloring arrays.
\usepackage[table]{xcolor}		% Auto alternate grey-white rows.
\usepackage[export]{adjustbox}		% Centered pics in tables

%% List utilities
\usepackage[inline]{enumitem}   % Display inline lists.
\usepackage{etoolbox}           % General utility. Good for lists for instance.
\usepackage{xparse}             % List utilities.
\usepackage{datatool}	% Handling alphabetical order.

%% Frames
\usepackage{framed}				% Boxes.
\usepackage[framemethod=TikZ]{mdframed}% For fancy frames.
\usepackage{tikz}				% For fancy frames.
\usepackage{wrapfig}			% Fancy insertion of pics in text.

%% Page utilities
\usepackage{multicol}			% Allows to divide a part of the page in multiple columns.
	
%% Others
\usepackage{keyval}             % Used to create maps of commands/labels/objects.
	\makeatletter                  % Mandatory for the usage of keyval.
\usepackage{xstring}            % String parsing, cutting, etc.
\usepackage{hyperref} % Links in PDF.


%%% Update of the dotfill command to always get dots

\newcommand{\predotfill}{\penalty0\hbox{}\nobreak}%


%%% Command to avoid typing \xspace when creating a new name macro

\newcommand{\newnamemacro}[2]{\newcommand{#1}{#2}} % \xspace removed for compatibility with alphabetical ordering

%%% Language specific stuff


%%% Commands %%%

\newcommand{\addtosortedlist}[1]{%
	\protected@edef\textarg{#1}%
	\protected@edef\textwithoutspaces{\expandafter\removespaces\expandafter{\textarg}}%
	\substitute\textwithoutspaces{É}{e}% Most used special characters of the language, and equivalent for alphabetical ordering
	\substitute\textwithoutspaces{È}{e}%
	\substitute\textwithoutspaces{Ê}{e}%
	\substitute\textwithoutspaces{é}{e}%
	\substitute\textwithoutspaces{è}{e}%
	\substitute\textwithoutspaces{ê}{e}%
	\substitute\textwithoutspaces{À}{a}%
	\substitute\textwithoutspaces{à}{a}%
	\substitute\textwithoutspaces{ù}{u}%
	\expandafter\sortitem\expandafter[\textwithoutspaces]{#1}%
}%


%%% Labels %%%

% Profile

\newcommand{\labels@M}{M}
\newcommand{\labels@WS}{CC}
\newcommand{\labels@BS}{CT}
\newcommand{\labels@S}{F}
\newcommand{\labels@T}{E}
\newcommand{\labels@W}{PV}
\newcommand{\labels@I}{I}
\newcommand{\labels@A}{A}
\newcommand{\labels@Ld}{Cd}
\newcommand{\labels@Invocation}{Invocation} % For Vampire Covenant profiles

\newcommand{\Strength}{Force}

% Technical

\newcommand{\labels@range}{Portée}
\newcommand{\labels@point}{pt}
\newcommand{\labels@points}{pts}
\newcommand{\labels@only}{uniquement}
\newcommand{\labels@magic}{Magie}
\newcommand{\labels@pathsused}{Génère ses sorts dans la Discipline}
\newcommand{\labels@model}{figurine}
\newcommand{\labels@models}{figurines}
\newcommand{\labels@Singlemodel}{Figurine \textbf{seule}}

% Unit entry labels

\newcommand{\labels@basesize}{Socle}
\newcommand{\labels@trooptype}{Type de troupe}
\newcommand{\labels@specialrules}{Règles spéciales}
\newcommand{\labels@alignment}{Allégeance}
\newcommand{\labels@equipment}{Équipement}
\newcommand{\labels@weapons}{Armes}
\newcommand{\labels@armour}{Armure}
\newcommand{\labels@options}{Options}
\newcommand{\labels@commandgroup}{État-Major}
\newcommand{\labels@mounts}{Montures}
\newcommand{\labels@specialequipment}{Équipement spécial}

% Command groups

\newcommand{\labels@champion}{Champion}
\newcommand{\labels@standardbearer}{Porte-étendard}
\newcommand{\labels@musician}{Musicien}
\newcommand{\labels@singlebannerallowance}{Une seule unité de ce type peut prendre une Bannière magique}
\newcommand{\labels@condsinglebannerallowance}{Une seule unité de ce type peut prendre une Bannière magique si}
\newcommand{\labels@bannerallowance}{Peut prendre une Bannière Magique}
\newcommand{\labels@veteranstandardbearer}{Peut devenir Porte-étendard Vétéran}
\newcommand{\labels@championallowance}{Peut prendre une Arme Magique}

% Titles

\newcommand{\labels@lords}{Seigneurs}
\newcommand{\labels@heroes}{Héros}
\newcommand{\labels@coreunits}{Unités de base}
\newcommand{\labels@specialunits}{Unités spéciales}
\newcommand{\labels@rareunits}{Unités rares}
\newcommand{\labels@armywiderules}{Règles communes de l'armée}
\newcommand{\labels@armyspecialrules}{Règles spéciales de l'armée}
\newcommand{\labels@armoury}{Armurerie}
\newcommand{\labels@magicalitems}{Objets magiques}
\newcommand{\labels@magicalweapons}{Armes magiques}
\newcommand{\labels@magicalarmour}{Armures magiques}
\newcommand{\labels@talismans}{Talismans}
\newcommand{\labels@enchanteditems}{Objets enchantés}
\newcommand{\labels@arcaneitems}{Objets cabalistiques}
\newcommand{\labels@magicalbanners}{Bannières magiques}
\newcommand{\labels@quickrefsheet}{Fiche de référence}
\newcommand{\labels@changelog}{Change Log}

\newcommand{\labels@lordsInitial}{S}
\newcommand{\labels@heroesInitial}{H}
\newcommand{\labels@coreunitsInitial}{B}
\newcommand{\labels@specialunitsInitial}{S}
\newcommand{\labels@rareunitsInitial}{R}
\newcommand{\labels@mountsInitial}{M}


% Titlepage

\newcommand{\labels@fantasybattles}{Batailles Fantastiques}
\newcommand{\labels@NinthAge}{Le 9\ieme Âge}
\newcommand{\labels@creators}{Une collaboration des créateurs de l'ETC et du Swedish Comp System}
\newcommand{\labels@introduction}{%
\noindent {\Largerfontsize\textbf{Note des traducteurs}}
\vspace{0.5cm}

Nous souhaitons remercier chaleureusement l'équipe à l'initiative du 9\ieme Âge pour leur motivation et leur travail continu pour faire vivre notre passion. Nous espérons que ce jeu saura développer les qualités pour plaire au plus grand nombre et réunir les joueurs, amateurs comme habitués des tournois, autour de règles amusantes et équilibrées, pour finalement s'imposer comme un standard du jeu de figurines. Une grande ambition qui ne pourra s'accomplir que \textbf{grâce à vous}, la communauté, via des retours constructifs, afin de modeler le jeu selon nos désirs. N'étant \textbf{en aucun cas à but lucratif}, le 9\ieme Âge part avec un avantage considérable. Les règles des éventuelles nouvelles sorties ne seront pas dictées par le besoin de vendre ces nouveautés. Vous pouvez choisir et acheter vos figurines où bon vous semble, il n'y a pas un unique revendeur toléré. Vous n'êtes pas bloqués dans une spirale infernale où pour continuer à jouer à un jeu, dans lequel vous vous êtes tant investis, vous devez payer toujours plus cher pour entretenir votre collection. Enfin, vous pouvez être assurés que tant que 9\ieme Âge sera joué, vous disposerez d'un \textbf{support continu et régulier}, celui-ci étant offert par la communauté.

Nous attirons votre attention sur le fait que ce jeu en est encore à ses débuts et dans un \textbf{stade de développement}. Ce document correspond à une version de brouillon \textbf{\og{} beta \fg{}}, dont le but et de tester le jeu et le modifier jusqu'à atteindre une version satisfaisante. Attendez-vous donc à trouver des déséquilibres, des incohérences, et à obtenir des mises à jour régulières avec éventuellement des changements importants. N'hésitez pas à nous donner vos avis ! Ce livre d'armée n'est utilisable qu'en compagnie du livre de Règles et du livre de Magie.

Concernant la traduction en elle-même, nous avons fait de notre mieux pour vous offrir une version de qualité, dont nous espérons qu'elle surpasse celle de la version originale ! Si vous constatez des coquilles, des erreurs, merci de nous les signaler en nous contactant sur le forum du 9\ieme Âge, dans le \textbf{sous-forum français} (\url{http://www.the-ninth-age.com/index.php?board/117-french/}). Vous y trouverez aussi les dernières mises à jour. \textbf{En cas de conflit d'interprétation avec la version originale, la version originale fait référence}.

\vspace{0.5cm}
Que ce jeu vous apporte d'innombrables heures de plaisir partagé !

\vspace{0.7cm}
\noindent {\Largerfontsize\textbf{Les traducteurs}}
\vspace{0.1cm}

\ifdef{\translationteam}{
	\begin{multicols}{3}
	\begin{itemize}
		\translationteam
	\end{itemize}
	\end{multicols}
}{}
}
\newcommand{\labels@secondpageannouncement}{%
	\labels@fantasybattles{} : \labels@NinthAge{} est un jeu créé et entretenu par la communauté qui met en scène des affrontements de figurines. Toutes les règles sont disponibles gratuitement sur le site suivant. Vos retours et suggestions sont les bienvenus.
	\newline\url{http://www.the-ninth-age.com/}
}
\newcommand{\labels@rulechanges}{%
	Les changements de règles entre versions sont colorés comme ce paragraphe. Une liste en anglais de ces changements par version est ajoutée à la fin de cet ouvrage.
}
\newcommand{\labels@latexcredit}{Document réalisé à l'aide de \LaTeX .}


%%% Technical commands

\newcommand{\only}[1]{(#1 uniquement)}
\newcommand{\free}{gratuit}
\newcommand{\upto}{jusqu'à}
\newcommand{\Upto}{Jusqu'à}
\newcommand{\unlimited}{sans limite de pts}
\newcommand{\permodel}{/fig.}
\newcommand{\listlastchoice}{ ou}
\newcommand{\notif}[1]{(pas #1)}
\newcommand{\wordand}{et}
\newcommand{\wordwith}{avec}
\newcommand{\ifNmodelsorless}[1]{(#1 figurines ou moins)}
\newcommand{\unitwith}{unité avec}
\newcommand{\From}{De} % From ... to ... models
\newcommand{\wordto}{à}
\newcommand{\wordAll}{Tous}
\newcommand{\spacebeforecolon}{ } % French put a space before colons
\newcommand{\minprice}{Coût min. :}
\newcommand{\mincostfor}{Coût min. pour}
\newcommand{\maxunitsize}{Taille max.}
\newcommand{\additionalfigscost}{Les figurines additionnelles coûtent}


%%% Special rules %%%

\newcommand{\ambush}{Embuscade}
\newcommand{\armourpiercing}[1]{Perforant\ifblank{#1}{}{ (#1)}}
\newcommand{\bodyguard}[1]{Garde du Corps\ifblank{#1}{}{ (#1)}}
\newcommand{\breathweapon}[1]{Attaque de Souffle\ifblank{#1}{}{ (#1)}}
\newcommand{\channel}{Canalisation}
\newcommand{\crushattack}{Attaque Écrasante}
\newcommand{\devastatingcharge}{Charge Dévastatrice}
\newcommand{\distracting}{Distrayant}
\newcommand{\engineer}{Ingénieur}
\newcommand{\ethereal}{Éthéré}
\newcommand{\fastcavalry}{Cavalerie Légère}
\newcommand{\fear}{Peur}
\newcommand{\fightinextrarank}{Combat avec un Rang Supplémentaire}
\newcommand{\fireborn}{Né du Feu}
\newcommand{\flamingattacks}{Attaques Enflammées}
\newcommand{\flammable}{Inflammable}
\newcommand{\lighttroops}{Troupes Légères}
\newcommand{\frenzy}{Frénésie}
\newcommand{\fly}[1]{Vol\ifblank{#1}{}{ (#1)}}
\newcommand{\grindingattacks}[1]{Attaques de Broyage\ifblank{#1}{}{ (#1)}}
\newcommand{\hardtarget}{Camouflé}
\newcommand{\hatred}{Haine}
\newcommand{\hellfire}{Flammes de l'Enfer}
\newcommand{\hidden}{Caché}
\newcommand{\holyattacks}{Attaques Divines}
\newcommand{\immunetopsychology}{Immunisé à la Psychologie}
\newcommand{\impacthits}[1]{Touches d'Impact\ifblank{#1}{}{ (#1)}}
\newcommand{\insignificant}{Insignifiant}
\newcommand{\largetarget}{Grande Cible}
\newcommand{\lethalstrike}{Coup Fatal}
\newcommand{\lightningattacks}{Attaques Foudroyantes}
\newcommand{\lightningreflexes}{Réflexes Foudroyants}
\newcommand{\magicresistance}[1]{Résistance à la Magie\ifblank{#1}{}{ (#1)}}
\newcommand{\magicalattacks}{Attaques Magiques}
\newcommand{\metalshifting}{Fusion du Métal}
\newcommand{\moveorfire}{Mouvement ou Tir}
\newcommand{\multipleshots}[1]{Tirs Multiples\ifblank{#1}{}{ (#1)}}
\newcommand{\multiplewounds}[2]{Blessures Multiples\ifblank{#1}{}{ (#1\ifblank{#2}{)}{, #2)}}}
\newcommand{\notaleader}{Pas un Meneur}
\newcommand{\otherworldly}{D'Outre-Monde}
\newcommand{\pathmaster}[1]{Maître de la Discipline\ifblank{#1}{}{ (#1)}}
\newcommand{\poisonedattacks}{Attaques Empoisonnées}
\newcommand{\quicktofire}{Tir Rapide}
\newcommand{\randommovement}[1]{Mouvement Aléatoire\ifblank{#1}{}{ (#1)}}
\newcommand{\randomattacks}[1]{Attaques Aléatoires\ifblank{#1}{}{ (#1)}}
\newcommand{\regeneration}[1]{Régénération\ifblank{#1}{}{ (#1+)}}
\newcommand{\reload}{Rechargez !}
\newcommand{\requirestwohands}{Arme à deux Mains}
\newcommand{\scythes}{Faux}
\newcommand{\scout}{Éclaireur}
\newcommand{\scouts}{Éclaireurs}
\newcommand{\stomp}[1]{Piétinement\ifblank{#1}{}{ (#1)}}
\newcommand{\strider}[1]{Guide\ifblank{#1}{}{ (#1)}}
\newcommand{\stubborn}{Tenace}
\newcommand{\stupidity}{Stupidité}
\newcommand{\skirmisher}{Tirailleur}
\newcommand{\skirmishers}{Tirailleurs}
\newcommand{\sweepingattack}{Attaque au Passage}
\newcommand{\swiftstride}{Rapide}
\newcommand{\thunderouscharge}{Charge Tonitruante}
\newcommand{\terror}{Terreur}
\newcommand{\toxicattacks}{Attaques Toxiques}
\newcommand{\unbreakable}{Indémoralisable}
\newcommand{\undead}{Mort-Vivant}
\newcommand{\unstable}{Instable}
\newcommand{\unwieldy}{Encombrant}
\newcommand{\vanguard}{Avant-Garde}
\newcommand{\volleyfire}{Tir de Volée}
\newcommand{\warplatform}{Plateforme de Guerre}
\newcommand{\wardsave}[1]{Sauvegarde Invulnérable\ifblank{#1}{}{ (#1+)}}
\newcommand{\weaponmaster}{Maître d'Ar\-mes}
\newcommand{\wizardconclave}[1]{Conclave de Sorciers\ifblank{#1}{}{ (#1)}}


%%% Magic %%%

\newnamemacro{\Pathof}{Discipline}

\newcommand{\battle}{Commune}
\newcommand{\alchemy}{de l'Alchimie}
\newcommand{\death}{de la Mort}
\newcommand{\fire}{du Feu}
\newcommand{\heavens}{des Cieux}
\newcommand{\light}{de la Lumière}
\newcommand{\nature}{de la Nature}
\newcommand{\shadows}{des Ombres}
\newcommand{\wilderness}{de la Sauvagerie Bestiale}
\newcommand{\butchery}{de la Boucherie}
\newcommand{\change}{du Changement}
\newcommand{\thebiggreengods}{des Grands Dieux Verts}
\newcommand{\thelittlegreengods}{des Petits Dieux Verts}
\newcommand{\blackmagic}{de la Magie Noire}
\newcommand{\disease}{de la Maladie}
\newcommand{\lust}{de la Luxure}
\newcommand{\necromancy}{de la Nécromancie}
\newcommand{\ruin}{de la Ruine}
\newcommand{\forge}{de la Forge}
\newcommand{\sands}{des Sables}
\newcommand{\whitemagic}{de la Magie Blanche}

\newcommand{\anyofthebattlemagic}{dans n'importe laquelle des Disciplines Communes}

\newcommand{\magiclevel}[1]{\ifnumcomp{#1}{<}{3}{Sorcier Apprenti}{Maître Sorcier} Niveau #1}
\newcommand{\Level}{Niveau}

\newcommand{\wizard}{Sorcier}
\newcommand{\wizards}{Sorciers}

\newcommand{\boundspell}[1]{Objet de Sort, Puissance #1}


%%% Other rules %%%

\newcommand{\armoursave}{Sauvegarde d'Armure}
\newcommand{\firstinrank}{Au Premier Rang}
\newcommand{\hardcover}{Couvert Lourd}
\newcommand{\holdyourground}{Tenez les Rangs}
\newcommand{\inspiringpresence}{Présence Charismatique}
\newcommand{\lightcover}{Couvert Léger}
\newcommand{\monstrousrank}{Rang Monstrueux}
\newcommand{\ordnance}{Artillerie}
\newcommand{\parry}{Parade}
\newcommand{\raisewounds}{Ressusciter des Figurines}
\newcommand{\recoverwounds}{Récupérer des PVs}
\newcommand{\aideddispel}{Dissipation Assistée}
\newcommand{\rnf}{ordinaires}
\newcommand{\general}{Général}


%%% Equipment %%%

\newcommand{\innatedefence}[1]{Protection Innée\ifblank{#1}{}{~(#1+)}}
\newcommand{\mountsprotection}[1]{Protection de Monture\ifblank{#1}{}{~(#1+)}}
\newcommand{\la}{Armure Légère}
\newcommand{\ha}{Armure Lourde}
\newcommand{\platearmour}{Armure de Plates}
\newcommand{\hw}{Arme de Base}
\newcommand{\pw}{Paire d'Armes}
\newcommand{\spear}{Lance}
\newcommand{\halberd}{Hallebarde}
\newcommand{\gw}{Arme Lourde}
\newcommand{\lance}{Lance de Cavalerie}
\newcommand{\lightlance}{Lance Légère}
\newcommand{\shield}{Bouclier}
\newcommand{\barding}{Caparaçon}
\newcommand{\throwingweapons}{Armes de Jet}
\newcommand{\shortbow}{Arc Court}
\newcommand{\flail}{Fléau}

\newcommand{\cannon}{Canon}
\newcommand{\catapult}{Catapulte}
\newcommand{\volleygun}{Batterie de Tir}
\newcommand{\boltthrower}{Baliste}
\newcommand{\artilleryweapon}{Arme d'Artillerie}


%%% Troop types %%%

\newcommand{\characters}{Personnages}
\newcommand{\infantry}{Infanterie}
\newcommand{\monstrousinfantry}{Infanterie Monstrueuse}
\newcommand{\cavalry}{Cavalerie}
\newcommand{\monstrouscavalry}{Cavalerie Monstrueuse}
\newcommand{\swarm}{Nuée}
\newcommand{\swarms}{Nuées}
\newcommand{\warbeast}{Bête de Guerre}
\newcommand{\warbeasts}{Bêtes de Guerre}
\newcommand{\monster}{Monstre}
\newcommand{\monsters}{Monstres}
\newcommand{\monstrousbeast}{Bête Monstrueuse}
\newcommand{\monstrousbeasts}{Bêtes Monstrueuses}
\newcommand{\chariot}{Char}
\newcommand{\chariots}{Chars}
\newcommand{\riddenmonster}{Monstre Monté}
\newcommand{\riddenmonsters}{Monstres Montés}
\newcommand{\warmachine}{Machine de Guerre}
\newcommand{\warmachines}{Machines de Guerre}


%%% Terrain %%%

\newcommand{\water}{Eaux peu profondes}


%%% Profile wording

\newcommand{\oneofakind}{Uni\-que}
\newcommand{\onechoiceonly}{(un seul choix)}
\newcommand{\onfootonly}{(à pied seulement)}
\newcommand{\closecombatonly}{seulement au Corps à Corps}
\newcommand{\Xmodelsorless}[1]{(#1 figurines ou moins)}
\newcommand{\magicalitemsallowance}{Peut prendre des Objets Magiques}
\newcommand{\magicalweaponallowance}{Peut prendre une Arme Magique}
\newcommand{\notmagicalarmour}{(mais pas d'Armure Magique)}
\newcommand{\anyofthefollowing}{\optionschoice{Peut prendre :}}
\newcommand{\weapononechoice}{\optionschoice{Peut prendre une arme \onechoiceonly{} :}}
\newcommand{\weaponschoice}{\optionschoice{Peut prendre des armes :}}
\newcommand{\shootingweapononechoice}{\optionschoice{Peut prendre une arme de tir \onechoiceonly{} :}}
\newcommand{\combatweapononechoice}{\optionschoice{Peut prendre une arme de corps à corps \onechoiceonly{} :}}
\newcommand{\armouronechoice}{\optionschoice{Peut prendre une armure \onechoiceonly{} :}}
\newcommand{\magiclevelchoice}{\optionschoice{Peut devenir au choix :}}
\newcommand{\bsboption}{Peut devenir Porteur de la Grande Bannière}
\newcommand{\mayupgradeto}{Peut être amélioré en}
\newcommand{\mustbecomeoneofthefollowing}{\optionschoice{Doit devenir un choix parmi :}}
\newcommand{\maybecomeoneofthefollowing}{\optionschoice{Peut devenir un choix parmi :}}
\newcommand{\maytakeoneofthefollowing}{\optionschoice{Peut prendre un choix parmi :}}
\newcommand{\maytakeuptotwoofthefollowing}{\optionschoice{Peut prendre jusqu'à deux choix parmi :}}
\newcommand{\maygain}{Peut gagner la règle}
\newcommand{\maytake}{Peut prendre}
\newcommand{\maytakeashield}{Peut prendre un Bouclier}
\newcommand{\maytakela}{Peut prendre une Armure Légère}
\newcommand{\maytakeha}{Peut prendre une Armure Lourde}
\newcommand{\maytakemountsprotectionX}[1]{Peut prendre une \mountsprotection{#1}}
\newcommand{\maytakeagw}{Peut prendre une Arme Lourde}
\newcommand{\maytakeaspear}{Peut prendre une Lance}
\newcommand{\maytakepw}{Peut prendre une Paire d'Armes}
\newcommand{\maytakethrowingweapons}{Peut prendre des Armes de Jet}
\newcommand{\maytakebarding}{Peut prendre un Caparaçon}
\newcommand{\replaceshieldwithhalberd}{Remplacer le Bouclier par une Hallebarde}
\newcommand{\maybecome}{Peut devenir}

\newcommand{\maytakeonechoiceonly}{\optionschoice{\maytake{} \onechoiceonly{}\spacebeforecolon{}:}}

\newcommand{\mountssectionannouncement}{%
La section Montures concerne les montures de Personnages. Les montures pour non-Personnages suivent les règles données dans leur description d'unité.
}

%%% Commands to handle strings, better than xstring to handle commands inside the strings %%%

\newcommand{\substitute}[3]{%
  \protected@edef\sub@temp{#1}%
  \saveexpandmode
  \expandarg\StrSubstitute{\sub@temp}{#2}{#3}[#1]%
  \restoreexpandmode
}

\newcommand{\splitatstar}[3]{%
  \protected@edef\split@temp{#1}%
  \saveexpandmode
  \expandarg\StrCut{\split@temp}{*}#2#3%
  \restoreexpandmode
}

\newcommand{\splitatinf}[3]{%
  \protected@edef\split@temp{#1}%
  \saveexpandmode
  \expandarg\StrCut{\split@temp}{<}#2#3%
  \restoreexpandmode
}

\newcommand{\splitatequal}[3]{%
  \protected@edef\split@temp{#1}%
  \saveexpandmode
  \expandarg\StrCut{\split@temp}{=}#2#3%
  \restoreexpandmode
}

\newcommand{\ifsubstring}[4]{%
  \protected@edef\split@temp{#1}%
  \protected@edef\split@tempbis{#2}%
  \saveexpandmode
  \expandarg\IfSubStr{\split@temp}{\split@tempbis}{#3}{#4}%
  \restoreexpandmode
}

\def\removespaces#1{\zap@space#1 \@empty}

%%% Commands for alphabetical ordering %%%

\newcommand{\sortitem}[2][\relax]{%
	\DTLnewrow{list}% Create a new entry
	\ifx#1\relax%
		\DTLnewdbentry{list}{sortlabel}{#2}% Add entry sortlabel (no optional argument)
	\else%
		\DTLnewdbentry{list}{sortlabel}{#1}% Add entry sortlabel (optional argument)
	\fi%
		\DTLnewdbentry{list}{description}{#2}% Add entry description
}
\newenvironment{sortedlist}{%
	\DTLifdbexists{list}{\DTLcleardb{list}}{\DTLnewdb{list}}% Create new/discard old list
}{%
	\DTLsort{sortlabel}{list}% Sort list
	\begin{itemize*}[label={}, itemjoin={,}]%
		\DTLforeach*{list}{\theDesc=description}{%
		\item\theDesc}% Print each item
	\end{itemize*}%
}

\pdfstringdefDisableCommands{\def\textcolor#1{}}

% See language specific file for \addtosortedlist

%%% Database for automatic Quick Ref Sheet %%%

\DTLnewdb{profiles} % Database containing name, category, multiprofile number, profilename (if multi), caraclist, trooptype, invocation for CV.
\newcommand{\profilecategory}{\labels@lords} % Will be updated in relevant categories

\newcommand{\profiledtbfillname}[1]{\DTLnewdbentry{profiles}{name}{#1}}
\newcommand{\profiledtbfillcategory}[1]{\DTLnewdbentry{profiles}{category}{#1}}
\newcommand{\profiledtbfilltrooptype}[1]{\DTLnewdbentry{profiles}{trooptype}{#1}}
\newcommand{\profiledtbfillinvocation}[1]{\DTLnewdbentry{profiles}{invocation}{#1}}
\newcommand{\profiledtbfillprofile}[1]{\DTLnewdbentry{profiles}{profile}{#1}}
\newcommand{\profiledtbfillmultipleprofile}[1]{\DTLnewdbentry{profiles}{multipleprofile}{#1}}

\newcommand{\void}[1]{}
\newcounter{multiprofilecounter}

\newcommand{\profiledtbfillcarac}[1]{%
	\profiledtbfillprofile{#1}
	\parselist{#1}{\locallists@profileslist}% Split of the different profiles in the case of a multiprofile.
	\setcounter{multiprofilecounter}{0}%
	\forlistloop{\stepcounter{multiprofilecounter}\void}{\locallists@profileslist}%
	\expandafter\profiledtbfillmultipleprofile\expandafter{\number\value{multiprofilecounter}}
}


%%% Technical commands %%%

\newcommand{\newrule}{\textcolor{green!50!black}}
\newcommand{\removedrule}[1]{\textcolor{green!50!black}{\sout{#1}}}
\newcommand{\starsymbol}{$\star$}
\newcommand{\refsymbol}{$^\star$}

\newcommand{\inch}{\arcsecond}
\newcommand{\foot}{\arcminute}
\newcommand{\range}[1] {\labels@range~\unit{#1}{\inch}}
\newcommand{\distance}[1] {\unit{#1}{\inch}}
\newcommand{\result}[1] {\texttt{'}#1\texttt{'}}


%%% Fonts and sizes %%%

\newcommand{\bigtitle}[1]{\vspace*{-1.5cm}\section*{}\noindent\begin{center}\Hugefontsize\textbf{\antiquefont\expandafter\uppercase\expandafter{#1}}\end{center}}

\newcommand{\subtitle}[1]{\subsection*{}\noindent{\hugefontsize\antiquefont #1}}

\newcommand{\subsubtitle}[1]{\subsubsection*{}\noindent{\Largerfontsize\antiquefont #1}}

\newcommand{\verysmallfontsize}{\fontsize{4}{4.8}\selectfont}
\newcommand{\smallfontsize}{\fontsize{6}{7.2}\selectfont}
\newcommand{\normalfontsize}{\fontsize{8}{9.6}\selectfont}
\newcommand{\largefontsize}{\fontsize{10}{12}\selectfont}
\newcommand{\largerfontsize}{\fontsize{12}{14.4}\selectfont}
\newcommand{\Largefontsize}{\fontsize{14}{16.8}\selectfont}
\newcommand{\Largerfontsize}{\fontsize{15}{18}\selectfont}
\newcommand{\hugefontsize}{\fontsize{18}{21.6}\selectfont}
\newcommand{\Hugefontsize}{\fontsize{25}{30}\selectfont}

\newcommand{\unitentryformat}[1]{\textit{\largefontsize{#1}}}
\newcommand{\textIT}[1]{\textit{\largefontsize{#1}}}


%%% Titles %%%

\newcommand{\lordstitle}{\def\logolocalpath{../Layout/pics/logo_lord.png}\bigtitle{\labels@lords}}
\newcommand{\heroestitle}{%
\def\logolocalpath{../Layout/pics/logo_hero.png}%
\clearpage\bigtitle{\labels@heroes}%
\renewcommand{\profilecategory}{\labels@heroes}%
}
\newcommand{\coreunitstitle}{%
\def\logolocalpath{../Layout/pics/logo_core.png}%
\clearpage\bigtitle{\labels@coreunits}%
\renewcommand{\profilecategory}{\labels@coreunits}%
}
\newcommand{\specialunitstitle}{%
\def\logolocalpath{../Layout/pics/logo_special.png}%
\clearpage\bigtitle{\labels@specialunits}%
\renewcommand{\profilecategory}{\labels@specialunits}%
}
\newcommand{\rareunitstitle}{%
\def\logolocalpath{../Layout/pics/logo_rare.png}%
\clearpage\bigtitle{\labels@rareunits}%
\renewcommand{\profilecategory}{\labels@rareunits}%
}
\newcommand{\mountstitle}{%
\def\logolocalpath{../Layout/pics/logo_mount.png}%
\clearpage\bigtitle{\labels@charactermounts}%
\renewcommand{\profilecategory}{\labels@mounts}%
}

\newcommand{\startarmywiderules}{\newpage\bigtitle{\labels@armywiderules}\largefontsize}
\newcommand{\closearmywiderules}{\normalfontsize}
\newcommand{\armywideruleentry}[1]{\subtitle{#1}\vspace{5pt}}

\newcommand{\startarmyspecialrules}{\bigtitle{\labels@armyspecialrules}\largefontsize}
\newcommand{\closearmyspecialrules}{\normalfontsize}
\newcommand{\armyspecialruleentry}[1]{\subtitle{#1}\vspace{5pt}}

\newcommand{\startarmyarmoury}{\bigtitle{\labels@armoury}\largefontsize\subtitle{}}
\newcommand{\closearmyarmoury}{\normalfontsize}

\newcommand{\startarmymagicalitems}{\newpage\largefontsize\bigtitle{\labels@magicalitems}\begin{multicols}{2}\raggedcolumns}
\newcommand{\closearmymagicalitems}{\end{multicols}\normalfontsize}

\newcommand{\armymagicalweapons}{\subtitle{\labels@magicalweapons}}
\newcommand{\armymagicalarmour}{\subtitle{\labels@magicalarmour}}
\newcommand{\armytalismans}{\subtitle{\labels@talismans}}
\newcommand{\armyenchanteditems}{\subtitle{\labels@enchanteditems}}
\newcommand{\armyarcaneitems}{\subtitle{\labels@arcaneitems}}
\newcommand{\armymagicalbanners}{\subtitle{\labels@magicalbanners}}

\newcommand{\startarmynewsection}[1]{\newpage\bigtitle{#1}\largefontsize}
\newcommand{\startarmynewsectionSP}[1]{\vspace{1.5cm}\bigtitle{#1}\largefontsize}
\newcommand{\closearmynewsection}{\normalfontsize}

\newcommand{\armynewsubsection}[1]{\subtitle{#1}\vspace{5pt}}
\newcommand{\armynewsubsubsection}[1]{\subsubtitle{#1}\vspace{3pt}}

\newcommand{\armylist}{\clearpage}

\newcommand{\quickrefsheettitle}{\clearpage\newgeometry{top=1.6cm, bottom=2cm, left=1cm, right=1cm}\bigtitle{\labels@quickrefsheet}\vspace*{0.4cm}}
\newcommand{\changelogtitle}{\clearpage\bigtitle{\labels@changelog}\spaceaftersection{}}

\newcommand{\spaceaftersection}{\vspace{0.8cm}}

\newcommand{\separator}{\noindent\begin{center}\textcolor{black!30}{\rule{0.7\columnwidth}{2pt}}\end{center}}


%%% Custom lists and description for first sections of the army books

\newcommand{\startpricelist}{\begin{samepage}\begin{description}[leftmargin=0.3cm, labelindent=0cm, labelsep=0.1cm]}
\def\endpricelist{\end{description}\end{samepage}}
\newcommand{\pricelistitem}[2]{\item \option{\textbf{#1}}{#2}\newline}

\newcommand{\startpricelistNSP}{\begin{description}[leftmargin=0.3cm, labelindent=0cm, labelsep=0.1cm]}
\def\endpricelistNSP{\end{description}}

\newcommand{\startitemlist}{\begin{multicols}{2}\raggedcolumns\begin{description}[leftmargin=0.3cm, labelindent=0cm, labelsep=0.1cm]}
\def\enditemlist{\end{description}\end{multicols}}
\newcommand{\listitem}[1]{\item[#1\spacebeforecolon{}:]}

\newcommand{\startitemlistonecol}{\begin{description}[leftmargin=0.3cm, labelindent=0cm, labelsep=0.1cm]}
\def\enditemlistonecol{\end{description}}
\newcommand{\listitemonecol}[1]{\item \textbf{#1\spacebeforecolon{}:}\newline}

\newenvironment{customitemize}{\begin{description}[leftmargin=0.3cm, labelindent=0cm, labelsep=0cm]}{\end{description}}
\newenvironment{customsubitemize}{\begin{itemize}[label={-}, labelsep=0.1cm, topsep=0cm, parsep=0cm, itemsep=0cm, leftmargin=0.4cm, labelindent=0cm]}{\end{itemize}}

%%% Table parameters %%%

\newcolumntype{M}[1]{>{\centering\let\newline\\\arraybackslash\hspace{0pt}}m{#1}}


%%%  Lists handling %%%

\newcommand{\addlocallist}{\listadd\locallists@dummy}%
\NewDocumentCommand{\parsespacelist}{>{\SplitList{ }} m }{%
	\ProcessList{#1}{\addlocallist}%
}%
\NewDocumentCommand{\parsecommalist}{>{\SplitList{,}} m }{%
	\ProcessList{#1}{\addlocallist}%
}%
\newcommand{\parselist}[3][,]{%
	\renewcommand\addlocallist{\listadd#3}%
  	\undef#3%
  	\ifstrequal{#1}{ }{\parsespacelist{#2}}{\parsecommalist{#2}}%
}


%%% Profiles handling %%%

% Element of a table that contains the characteristics of a model (or part of a model)
\newcommand\caraclist[1]{
	\parselist[ ]{#1}{\locallists@caraclist}%
	\forlistloop{&}{\locallists@caraclist}%
}

\newcommand\caraclistbold[1]{
	\parselist[ ]{#1}{\locallists@caraclist}%
	\forlistloop{&\bfseries}{\locallists@caraclist}%
}

% Line of a profile table, including bottom line. It is meant to contain the name of the model (or part), its characteristics (preferably, the second argument should contain the \carac macro), troop type and base size.
\newcommand{\profilefirstline}[4]{#1 & #2 &   & #3 & #4 }

% Start of a profile table. Includes the table commands, and the column labels. \profilecellsize is the size of the characteristics cells in the profile.
\newcommand{\profilecellsize}{0.56cm}
\newcommand{\profilestart}{%
	\noindent %
	\begin{tabular}{@{}p{3cm}@{}M{\profilecellsize}@{}M{\profilecellsize}@{}M{\profilecellsize}@{}M{\profilecellsize}@{}M{\profilecellsize}@{}M{\profilecellsize}@{}M{\profilecellsize}@{}M{\profilecellsize}@{}M{\profilecellsize}@{}p{2.7cm}@{}p{3.3cm}@{}p{2cm}@{}}%
	 &% \textbf{\labels@profile}
	\labels@M & \labels@WS & \labels@BS & \labels@S & \labels@T & \labels@W & \labels@I & \labels@A & \labels@Ld &%
	&%
	{\unitentryformat{\labels@trooptype}} &%
	{\unitentryformat{\labels@basesize}}%
}

% End of a profile table.
\newcommand{\profileend}{\end{tabular}}

% Algorithm to automatically use and fill previous command, with coherence check.
\providebool{profilefirst}
\newcommand{\profileitem}[1]{%
	\tabularnewline%
	\splitatinf{#1}\local@unitname\local@unitprofile%
	\local@unitname \expandafter\caraclistbold\expandafter{\local@unitprofile}%
	&%
	& \ifbool{profilefirst}{\unit@type}{}%
	& \ifbool{profilefirst}{%
		\ifsubstring{\unit@basesize}{x}{% Rectangular base
			\unit{\unit@basesize}{\milli\meter}%
		}{% Circular base
			\unit{\unit@basesize}{\milli\meter} \labels@roundbase%
		}%
	}{}%
	\global\boolfalse{profilefirst}%
}
\newcommand{\profile}[1]{%
	\parselist{#1}{\locallists@profileslist}%
	\profilestart%
	\global\booltrue{profilefirst}%
	\forlistloop{\profileitem}{\locallists@profileslist}%
	\profileend%
}


%%% Profiles handling in case of invocation %%%

\newcommand{\invocprofilestart}{%
	\noindent %
	\begin{tabular}{@{}p{3cm}@{}M{\profilecellsize}@{}M{\profilecellsize}@{}M{\profilecellsize}@{}M{\profilecellsize}@{}M{\profilecellsize}@{}M{\profilecellsize}@{}M{\profilecellsize}@{}M{\profilecellsize}@{}M{\profilecellsize}@{}M{2.2cm}@{}p{0.5cm}@{}p{3.3cm}@{}p{2cm}@{}}%
	 &% \textbf{\labels@profile}
	\labels@M & \labels@WS & \labels@BS & \labels@S & \labels@T & \labels@W & \labels@I & \labels@A & \labels@Ld & \unitentryformat{\labels@Invocation} &%
	&%
	{\unitentryformat{\labels@trooptype}} &%
	{\unitentryformat{\labels@basesize}}%
}

\newcommand{\invocprofileitem}[1]{%
	\tabularnewline%
	\splitatinf{#1}\local@unitname\local@unitprofile%
	\local@unitname \expandafter\caraclistbold\expandafter{\local@unitprofile}%
	& \ifbool{profilefirst}{\unit@invocation}{} &%
	& \ifbool{profilefirst}{\unit@type}{}%
	& \ifbool{profilefirst}{\unit{\unit@basesize}{\milli\meter}}{}%
	\global\boolfalse{profilefirst}%
}

\newcommand{\invocprofile}[1]{%
	\parselist{#1}{\locallists@profileslist}%
	\invocprofilestart%
	\global\booltrue{profilefirst}%
	\forlistloop{\invocprofileitem}{\locallists@profileslist}%
	\profileend%
}


%%%%%%%%%%%%%%%%%%
%%% Unit rules %%%
%%%%%%%%%%%%%%%%%%

%%% Entry title command %%%

\newcommand{\unitentry}[2]{\ifdefempty{#1}{}{\noindent #2}}


%%% Special rules %%%

% Special rules listing for a unit, with alphabetical order.
\newcommand{\ruleslist}[1]{%
	\parselist[,]{#1}{\locallists@ruleslist}%
	\begin{sortedlist}%
		\forlistloop{\addtosortedlist}{\locallists@ruleslist}%
	\end{sortedlist}%
}

% Special rules entry.
\newcommand{\specialrules}[1]{\unitentry{#1}{\unitentryformat{\labels@specialrules\spacebeforecolon{}:}\newline\hspace*{-\fontdimen2\font}\expandafter\ruleslist\expandafter{#1}.}}
\newcommand{\commonspecialrules}[2]{\unitentry{#2}{\unitentryformat{#1\spacebeforecolon{}:}\newline\hspace*{-\fontdimen2\font}\expandafter\ruleslist\expandafter{#2}.}}


%%% Magical abilities %%%

% Paths listing for a unit.
\newcommand{\pathslist}[1]{%
	\parselist[,]{#1}{\locallists@pathslist}%
	\begin{itemize*}[label={}, itemjoin={,}, itemjoin*={\listlastchoice}]%
		\forlistloop{\item}{\locallists@pathslist}%
	\end{itemize*}%
}

% Magic entry.
\newcommand{\magic}[2]{\unitentry{#2}{\unitentryformat{\labels@magic\spacebeforecolon{}: }\newline\ifdefempty{#1}{}{\textbf{\magiclevel{#1}}. }\labels@pathsused\expandafter\pathslist\expandafter{#2}.}}

% Wizard Conclave.
\newcommand{\magicwizardconclave}[1]{\unitentry{#1}{\unitentryformat{\labels@magic\spacebeforecolon{}: }\newline\textbf{\wizardconclave{}}\spacebeforecolon{}: #1.}}


%%% Equipment %%%

% Equipment listing.
\newcommand{\equipmentlist}[1]{%
	\parselist[,]{#1}{\locallists@equipmentlist}%
	\begin{sortedlist}%
		\forlistloop{\addtosortedlist}{\locallists@equipmentlist}%
	\end{sortedlist}%
}

% Equipment entry.
\newcommand{\weapons}[1]{\unitentry{#1}{\unitentryformat{\labels@weapons\spacebeforecolon{}:}\newline\hspace*{-\fontdimen2\font}\expandafter\equipmentlist\expandafter{#1}.}}

\newcommand{\armour}[1]{\unitentry{#1}{\unitentryformat{\labels@armour\spacebeforecolon{}:}\newline\hspace*{-\fontdimen2\font}\expandafter\equipmentlist\expandafter{#1}.}}


%%% Alignment %%%

\newcommand{\alignment}[1]{\unitentry{#1}{\unitentryformat{\labels@alignment\spacebeforecolon{}:}\newline\textbf{#1}.}}

%%% Green Hide Race %%%

\newcommand{\greenhideraceentry}[1]{\unitentry{#1}{\unitentryformat{\labels@greenhiderace\spacebeforecolon{}:}\newline\textbf{#1}.}}


%%% Options %%%

% Frame commands.
\newcommand{\optionsframestart}{\begin{innerframe}[\labels@options]}
\newcommand{\optionsframeend}{\end{innerframe}}

% Options listing.
\newcommand{\optionslist}[1]{%
	\parselist[,]{#1}{\locallists@optionslist}%
	\begin{description}[leftmargin=0.3cm, labelindent=0cm, labelsep=0cm, itemsep=0cm, parsep=0cm]%
		\forlistloop{\item\setoption}{\locallists@optionslist}%
	\end{description}%
}

% Options entry.
\newcommand{\options}[1]{\ifdefempty{#1}{}{\optionsframestart\vspace*{-0.4cm}\unitentry{#1}{\expandafter\optionslist\expandafter{#1}}\optionsframeend}}

% Option specific commands.
\newcommand{\setoption}[1]{%
	\noexpandarg\StrCut{#1}{=}\optiontext\optionvalue%
	\expandafter\ifstrequal\expandafter{\optionvalue}{}{%
		\optiontext%
	}{%
	\ifsubstring{\optionvalue}{\free}{%
		\option[\free]{\optiontext}{\optionvalue}%
	}{%
	\ifsubstring{\optionvalue}{\unlimited}{%
		\option[\unlimited]{\optiontext}{\optionvalue}%
	}{%
	\ifsubstring{\optionvalue}{\upto}{%
		\splitatinf{\optionvalue}\myoption\myvalue%
		\option[\upto]{\optiontext}{\myvalue}%
	}{%
	\ifsubstring{\optionvalue}{\permodel}{%
		\splitatinf{\optionvalue}\myoption\myvalue%
		\option[\permodel]{\optiontext}{\myvalue}%
	}{%
	\ifsubstring{\optionvalue}{\pershadygit}{% For Orcs N Goblins
		\splitatinf{\optionvalue}\myoption\myvalue%
		\option[\pershadygit]{\optiontext}{\myvalue}%
	}{%
	\ifsubstring{\optionvalue}{\permadgit}{% For Orcs N Goblins
		\splitatinf{\optionvalue}\myoption\myvalue%
		\option[\permadgit]{\optiontext}{\myvalue}%
	}{%	
	\ifsubstring{\optionvalue}{\perrune}{% For Dwarven Holds
		\splitatinf{\optionvalue}\myoption\myvalue%
		\option[\perrune]{\optiontext}{\myvalue}%
	}{%	
		\option{\optiontext}{\optionvalue}%
	}}}}}}}}%
}

\newcommand{\option}[3][]{#2\predotfill\dotfill\nobreak%
	% Add \upto token if necessary.
	\ifstrequal{#1}{\upto}{\upto~}{}%
	% The option can be free, have an unlimited cost, or have a points cost.
	\ifstrequal{#1}{\free}{\free}{\ifstrequal{#1}{\unlimited}{\unlimited}{\pts{#3}}}%
	% Add \permodel if necessary.
	\ifstrequal{#1}{\permodel}{\nobreak\permodel}{}%
	% Add \persomething if necessary.
	\ifstrequal{#1}{\pershadygit}{\nobreak\pershadygit}{}% For Orcs N Goblins
	\ifstrequal{#1}{\permadgit}{\nobreak\permadgit}{}% For Orcs N Goblins
	\ifstrequal{#1}{\perrune}{\nobreak\perrune}{}% For Dwarven Holds
}

\newcommand\optionschoice[2]{%
	\parselist[,]{#2}{\locallists@optionschoice}%
	#1%
	\begin{itemize}[label={}, parsep=0cm, labelindent=0cm, labelwidth=0cm, noitemsep, topsep=0em, leftmargin=0.3cm]%
	\forlistloop{\item\setoption}{\locallists@optionschoice}%
	\end{itemize}%
}

\newcommand\optionschoiceTWOCOL[2]{%
	\parselist[,]{#2}{\locallists@optionschoice}%
	#1%
	\begin{itemize}[label={}, parsep=0cm, labelindent=0cm, labelwidth=0cm, noitemsep, topsep=0em, leftmargin=0.3cm]%
	\setlength{\columnseprule}{0.5pt}
	\renewcommand{\columnseprulecolor}{\color{black!30}}
	\vspace*{-5pt}\begin{multicols}{2}\raggedcolumns
	\forlistloop{\item\setoption}{\locallists@optionschoice}%
	\end{multicols}\setlength{\columnseprule}{0pt}
	\end{itemize}%
}

% Option description in army desc.
\newcommand{\optiondef}[3]{\option{\textbf{#1}}{#2}\ifblank{#3}{}{\\{#3}}}


%%% Mount options %%%

% Frame commands.
\newcommand{\mountsframestart}{\begin{innerframe}[\labels@mounts]}
\newcommand{\mountsframeend}{\end{innerframe}}

% Mount listing.
\newcommand{\mountslist}[1]{%
	\parselist[,]{#1}{\locallists@mountslist}%
	\begin{description}[leftmargin=0.3cm, labelindent=0cm, labelsep=0cm, itemsep=0cm, parsep=0cm]%
		\forlistloop{\item\setoption}{\locallists@mountslist}%
	\end{description}%
}

% Mount entry.
\newcommand{\mounts}[1]{\ifdefempty{#1}{}{\mountsframestart\vspace*{-0.4cm}\unitentry{#1}{\expandafter\mountslist\expandafter{#1}}\mountsframeend}}


%%% Command group %%%

% Command group specific commands.
\define@key{commandgroup}{restriction}            {\def\commandgroup@restriction{#1}}
\define@key{commandgroup}{champion}               {\def\commandgroup@champion{#1}}
\define@key{commandgroup}{championallowance}      {\def\commandgroup@championallowance{#1}}
\define@key{commandgroup}{championoption}         {\def\commandgroup@championoption{#1}}
\define@key{commandgroup}{championprerestriction} {\def\commandgroup@championprerestriction{#1}}
\define@key{commandgroup}{championrestriction}    {\def\commandgroup@championrestriction{#1}}
\define@key{commandgroup}{banner}                 {\def\commandgroup@banner{#1}}
\define@key{commandgroup}{bannerallowance}        {\def\commandgroup@bannerallowance{#1}}
\define@key{commandgroup}{veteranstandardbearer}  {\def\commandgroup@veteranstandardbearer{#1}}
\define@key{commandgroup}{singlebannerallowance}  {\def\commandgroup@singlebannerallowance{#1}}
\define@key{commandgroup}{condsinglebannerallowance}  {\def\commandgroup@condsinglebannerallowance{#1}}
\define@key{commandgroup}{banneroption}           {\def\commandgroup@banneroption{#1}}
\define@key{commandgroup}{bannerrestriction}      {\def\commandgroup@bannerrestriction{#1}}
\define@key{commandgroup}{musician}               {\def\commandgroup@musician{#1}}
\define@key{commandgroup}{musicianrestriction}    {\def\commandgroup@musicianrestriction{#1}}
\newcommand{\defcommandgroup}{%
	\setkeys{commandgroup}{restriction=,
	                       champion=, championallowance=, championoption=, championprerestriction=, 
	                       championrestriction=, banner=, bannerallowance=, veteranstandardbearer=, 
	                       singlebannerallowance=, condsinglebannerallowance=, banneroption=, 
	                       bannerrestriction=, musician=, musicianrestriction=}%
	\setkeys{commandgroup}%
}

% Frame commands.
\newcommand{\commandgroupframestart}{\begin{innerframe}[\labels@commandgroup]}
\newcommand{\commandgroupframeend}{\end{innerframe}}

% Command group entry.
\newcommand{\commandgroup}[1]{%
	\defcommandgroup{#1}%
	\ifstrempty{#1}{}{\commandgroupframestart\vspace*{-0.2cm}%
		\begin{description}[leftmargin=0.3cm, labelindent=0cm, labelsep=0cm, itemsep=0cm, parsep=0cm]%
			% Command group title, including restrictions applying to all the command group
			\item \textbf{\expandafter\ifblank\expandafter{\commandgroup@restriction}{}{ \only{\commandgroup@restriction}\spacebeforecolon{}: }} 
			% Champion handling.
			\ifdefempty{\commandgroup@champion}{}{% We have a champion!
			\ifdefempty{\commandgroup@championprerestriction}{% There is no prerestriction to have a champion
				\item \hspace*{-0.04cm}\option{\labels@champion%
					% Possible restrictions to taking a champion
				    \expandafter\ifblank\expandafter{\commandgroup@championrestriction}{}{ \only{\commandgroup@championrestriction}}%
				    % Cost of a champion
				    }{\commandgroup@champion}%
				    % Magical allowance of the champion. Should probably not be used, champion option can do it as well and is more flexible.
					\ifdefempty{\commandgroup@championallowance}{}{\par\option[\upto]{\hspace*{0.3cm}- \labels@championallowance}{\commandgroup@championallowance}}%
					% Any option available to the champion, in the form option:cost
					\ifdefempty{\commandgroup@championoption}{}{%
						\splitatinf{\commandgroup@championoption}\local@option\local@cost%
						\par\option{\hspace*{0.3cm}- \local@option}{\local@cost}}%
			}{% There is a pre-restriction to have a champion
				\item \hspace*{-0.04cm}\commandgroup@championprerestriction	\newline%
				\option{\labels@champion}{\commandgroup@champion}%
				% Magical allowance of the champion. Should probably not be used, champion option can do it as well and is more flexible.
				\ifdefempty{\commandgroup@championallowance}{}{\par\option[\upto]{\hspace*{0.3cm}- \labels@championallowance}{\commandgroup@championallowance}}%
				% Any option available to the champion, in the form option:cost
				\ifdefempty{\commandgroup@championoption}{}{%
					\splitatinf{\commandgroup@championoption}\local@option\local@cost%
					\par\option{\hspace*{0.3cm}- \local@option}{\local@cost}}%
			} %End of the prerestriction of not condition
			}% End of champion handling
			\ifdefempty{\commandgroup@musician}{}{% We have a musician!
				\item \hspace*{-0.04cm}\option{\labels@musician%
					% Possible restrictions to taking a musician
				    \expandafter\ifblank\expandafter{\commandgroup@musicianrestriction}{}{ \only{\commandgroup@musicianrestriction}}%
				    % Cost of a musician
				    }{\commandgroup@musician}%
			}%
			\ifdefempty{\commandgroup@banner}{}{% We have a banner!
				\item \hspace*{-0.04cm}\option{\labels@standardbearer%
					% Possible restrictions to taking a banner
				    \expandafter\ifblank\expandafter{\commandgroup@bannerrestriction}{}{ \only{\commandgroup@bannerrestriction}}%
				    % Cost of a banner
				    }{\commandgroup@banner}%
				    % Magical banner, if all units of this type can take one.
					\ifdefempty{\commandgroup@bannerallowance}{}{\par\option[\upto]{\hspace*{0.3cm}- \labels@bannerallowance}{\commandgroup@bannerallowance}}%
					% Magical banner, if Veteran.
					\ifdefempty{\commandgroup@veteranstandardbearer}{}{\par\hspace*{0.3cm}- \labels@veteranstandardbearer%
					\expandafter\ifstrequal\expandafter{\commandgroup@veteranstandardbearer}{*}{*}{}%
					}%
					% Magical banner, if only one unit of this type can take one.
					\ifdefempty{\commandgroup@singlebannerallowance}{}{\par\option[\upto]{\hspace*{0.3cm}- \labels@singlebannerallowance}{\commandgroup@singlebannerallowance}}%
					% Magical banner, if only one unit of this type can take one, but with condtions.
					\ifdefempty{\commandgroup@condsinglebannerallowance}{}{%
						\splitatinf{\commandgroup@condsinglebannerallowance}\local@option\local@cost%
						\par\option[\upto]{\hspace*{0.3cm}- \labels@condsinglebannerallowance \local@option}{\local@cost}}%
					% Additional option for the banner, such as Hill Goblin Lookouts for Ogres
					\ifdefempty{\commandgroup@banneroption}{}{%
						\splitatinf{\commandgroup@banneroption}{\local@option}{\local@cost}%
						\par\option{\hspace*{0.3cm}- \local@option}{\local@cost}%
					}%
			}%
		\end{description}%
	\commandgroupframeend%
	 }%
}


%%% Unit rules %%%

% Frame commands.
\newcommand{\unitrulesframestart}{\begin{innerframe}[\labels@specialrules]}
\newcommand{\unitrulesframeend}{\end{innerframe}}

% Unit rules specific commands.
\newcommand{\unitrule}[2]{\item[#1\spacebeforecolon{}:]#2}

% Unit rule entry.
\newcommand{\unitrules}[1]{\ifdefempty{#1}{}{\unitrulesframestart\vspace*{-0.05cm}\begin{description}[leftmargin=0.3cm, labelindent=0cm, labelsep=0.1cm, itemsep=0.2cm, parsep=0cm]#1\end{description}\unitrulesframeend}}


%%% Special equipment %%%

% Frame commands.
\newcommand{\unitequipmentframestart}{\begin{innerframe}[\labels@specialequipment]}
\newcommand{\unitequipmentframeend}{\end{innerframe}}

% Special equipment specific commands.
\newcommand{\equipmentdef}[2]{\item[#1\spacebeforecolon{}:]#2}

% Special equipment entry.
\newcommand{\unitequipment}[1]{\ifdefempty{#1}{}{\unitequipmentframestart\vspace*{-0.05cm}\begin{description}[leftmargin=0.3cm, labelindent=0cm, labelsep=0.1cm, itemsep=0.2cm, parsep=0cm]#1\end{description}\unitequipmentframeend}}






%%%%%%%%%%%%%%%%%%%%%%%%%%%%%%%%
%%% Profile input and layout %%%
%%%%%%%%%%%%%%%%%%%%%%%%%%%%%%%%

%%% Input parameters %%%

\define@key{unit}{notinQRS}{\def\unit@notinQRS{#1}}
\define@key{unit}{name}{\def\unit@name{#1}}
\define@key{unit}{QRSname}{\def\unit@QRSname{#1}}
\define@key{unit}{profile}{\def\unit@profile{#1}}
\define@key{unit}{cost}{\def\unit@cost{#1}}
\define@key{unit}{invocation}{\def\unit@invocation{#1}}
\define@key{unit}{costpermodel}{\def\unit@costpermodel{#1}}
\define@key{unit}{maxmodels}{\def\unit@maxmodels{#1}}
\define@key{unit}{type}{\def\unit@type{#1}}
\define@key{unit}{unitsize}{\def\unit@unitsize{#1}}
\define@key{unit}{basesize}{\def\unit@basesize{#1}}
\define@key{unit}{commonspecialrules}{\def\unit@commonspecialrules{#1}}
\define@key{unit}{commontype}{\def\unit@commontype{#1}}
\define@key{unit}{commonspecialrulesB}{\def\unit@commonspecialrulesB{#1}}
\define@key{unit}{commontypeB}{\def\unit@commontypeB{#1}}
\define@key{unit}{specialrules}{\def\unit@specialrules{#1}}
\define@key{unit}{magiclevel}{\def\unit@magiclevel{#1}}
\define@key{unit}{magicpaths}{\def\unit@magicpaths{#1}}
\define@key{unit}{equipment}{\def\unit@equipment{#1}}
\define@key{unit}{alignment}{\def\unit@alignment{#1}}
\define@key{unit}{greenhiderace}{\def\unit@greenhiderace{#1}}
\define@key{unit}{weapons}{\def\unit@weapons{#1}}
\define@key{unit}{armour}{\def\unit@armour{#1}}
\define@key{unit}{wizardconclave}{\def\unit@wizardconclave{#1}}
\define@key{unit}{unitequipment}{\def\unit@unitequipment{#1}}
\define@key{unit}{options}{\def\unit@options{#1}}
\define@key{unit}{mounts}{\def\unit@mounts{#1}}
\define@key{unit}{commandgroup}{\def\unit@commandgroup{#1}}
\define@key{unit}{unitrules}{\def\unit@unitrules{#1}}
\define@key{unit}{additional}{\def\unit@additional{#1}}


%%% Frames definition %%%

% Unit's big frame.
\tikzset{unitprice/.style={draw=white, fill=white, rectangle, rounded corners, right, minimum height=0.7cm}}
\tikzset{unittitle/.style={draw=white, fill=white, rectangle, rounded corners, right, minimum height=0.7cm, font=\bfseries}}
\tikzset{unitlogo/.style={draw=white, fill=white, rectangle, right, minimum height=0.7cm}}

\newenvironment{unitframe}[2][]{%
	\mdfsetup{%
		nobreak=true,%
		linewidth=1pt,%
		linecolor=black!30,%
		roundcorner=5pt,%
		backgroundcolor=white,%
		innertopmargin=1.2\baselineskip,
		innerbottommargin=1.2\baselineskip,
		singleextra={
			\expandafter\ifblank\expandafter{\unit@cost}{}{%
				\node[unitprice,anchor=east,xshift=-0.5cm] at (P)%
					{%
						{{\smallfontsize\minprice} \Largefontsize\pts{\textbf{\unit@cost}}}%
					};
				}%
				\node[unittitle,xshift=0.5cm] at (P-|O)%
					{\Largefontsize\antiquefont\uppercase\expandafter\expandafter\expandafter{\unit@name}};
				\node[unitlogo, xshift=8.1cm, yshift=0.1cm] at (P-|O)%
					{\includegraphics[width=1.2cm]{\logolocalpath}};
		}
	}%
	\begin{mdframed}[]\relax%
}%
{%
\end{mdframed}%
}

% Inner small frames for options, special rules definition, ...
\tikzset{innertitle/.style={fill=white, rectangle, rounded corners, right, minimum height=8pt, xshift=0.5cm}}

\newenvironment{innerframe}[1][]{%
	\mdfsetup{%
		innerleftmargin=5pt,%
		innerrightmargin=5pt,%
		linecolor=black!30,%
		linewidth=0.5pt,%
		roundcorner=5pt,%
		backgroundcolor=white,%
		innertopmargin=1.1\baselineskip,
		singleextra={
		\node[innertitle] at (P-|O)%
			{\unitentryformat{#1}};
		}
	}%
	\vspace*{-0.2cm}\begin{mdframed}[]\relax%
}%
{%
\end{mdframed}%
}

%%% Command to add a new unit definition %%%

\newcommand{\defunit}{
	\setkeys{unit}{%
		notinQRS=, name=, QRSname=, profile=, cost=, invocation=, costpermodel=, maxmodels=, type=, unitsize=, basesize=, commonspecialrules=, commontype=, commonspecialrulesB=, commontypeB=, specialrules=, magiclevel=, magicpaths=, alignment=, greenhiderace=, equipment=, weapons=, armour=, wizardconclave=, unitequipment=, options=, mounts=, commandgroup=, unitrules=, additional=%
	}%
	\setkeys{unit}%
}

\newcommand{\showunit}[1]{
	\defunit{#1}
	\begin{unitframe}[\unit@name]{\unit@cost}
	\mdfsetup{style=defaultoptions}
	\expandafter\ifblank\expandafter{\unit@unitsize}{}{%
	\expandafter\ifstrequal\expandafter{\unit@unitsize}{1}{% single model
		% Can you add model to this single model ?
		\expandafter\ifblank\expandafter{\unit@maxmodels}{% no		
			{\hspace*{0.25cm}\labels@Singlemodel}%
		}{% yes
			{\hspace*{0.25cm}\mincostfor{} \textbf{1} \labels@model{}. \maxunitsize{}\spacebeforecolon{}: \textbf{\unit@maxmodels} \labels@models{}.\hfill \additionalfigscost{} {\largefontsize\pts{\textbf{\unit@costpermodel{}}}\permodel}\hspace*{0.1cm}}%
		}%
	}{% not single model
		% Test if we wanna print a sentence instead of unit number
		\ifsubstring{\unit@unitsize}{SPECIAL-}{%
			\hspace*{0.25cm}\StrDel{\unit@unitsize}{SPECIAL-}%
		}{%	
			{\hspace*{0.25cm}\mincostfor{} \textbf{\unit@unitsize} \labels@models{}. \maxunitsize{}\spacebeforecolon{}: \textbf{\unit@maxmodels} \labels@models{}.\hfill \additionalfigscost{} {\largefontsize\pts{\textbf{\unit@costpermodel{}}}\permodel}\hspace*{0.1cm}}%
		}%
	}%
	}%
	\vspace*{-0.1cm}
	\noindent\begin{center}\textcolor{black!30}{\rule{\columnwidth}{1pt}}\end{center}
		\expandafter\ifblank\expandafter{\unit@invocation}{%
			\expandafter\profile\expandafter{\unit@profile}
		}{%
			\expandafter\invocprofile\expandafter{\unit@profile}
		}
	\noindent\begin{center}\textcolor{black!30}{\rule{\columnwidth}{1pt}}\end{center}
	\vspace*{-0.2cm}
	\setlength\multicolsep{0pt}
	\begin{multicols}{2}
		\raggedcolumns
		\vspace*{-0.3cm}{\setlength{\parskip}{0.3cm}
		\expandafter\ifblank\expandafter{\unit@alignment}{}{\noindent\parbox{\columnwidth}{\alignment{\unit@alignment}}}
		
		\expandafter\ifblank\expandafter{\unit@greenhiderace}{}{\noindent\parbox{\columnwidth}{\greenhideraceentry{\unit@greenhiderace}}}
		
		\expandafter\ifblank\expandafter{\unit@equipment}{}{\noindent\parbox{\columnwidth}{\equipment{\unit@equipment}}}
				
		\expandafter\ifblank\expandafter{\unit@weapons}{}{\noindent\parbox{\columnwidth}{\weapons{\unit@weapons}}}
		
		\expandafter\ifblank\expandafter{\unit@armour}{}{\noindent\parbox{\columnwidth}{\armour{\unit@armour}}}
		
		\expandafter\ifblank\expandafter{\unit@commonspecialrules}{}{\noindent\parbox{\columnwidth}{\commonspecialrules{\unit@commontype}{\unit@commonspecialrules}}}
		
		\expandafter\ifblank\expandafter{\unit@commonspecialrulesB}{}{\noindent\parbox{\columnwidth}{\commonspecialrules{\unit@commontypeB}{\unit@commonspecialrulesB}}}
		
		\expandafter\ifblank\expandafter{\unit@specialrules}{}{\noindent\parbox{\columnwidth}{\specialrules{\unit@specialrules}}}
		
		\expandafter\ifblank\expandafter{\unit@magicpaths}{}{\noindent\parbox{\columnwidth}{\magic{\unit@magiclevel}{\unit@magicpaths}}}
		
		\expandafter\ifblank\expandafter{\unit@wizardconclave}{}{\noindent\parbox{\columnwidth}{\magicwizardconclave{\unit@wizardconclave}}}
		}
		\vspace{0.1cm}
		\mounts{\unit@mounts}
		\options{\unit@options}
		\expandafter\ifblank\expandafter{\unit@commandgroup}{}{\expandafter\commandgroup\expandafter{\unit@commandgroup}}
		\unitrules{\unit@unitrules}
		\unitequipment{\unit@unitequipment}
	\end{multicols}
	\vspace*{0.1cm}\unit@additional
	\end{unitframe}
	% Database filling for auto QRS
	\expandafter\ifblank\expandafter{\unit@notinQRS}{%
	\DTLnewrow{profiles}%
	\expandafter\ifblank\expandafter{\unit@QRSname}{%
		\expandafter\profiledtbfillname\expandafter{\unit@name}%
	}{%
		\expandafter\profiledtbfillname\expandafter{\unit@QRSname}%
	}
	\expandafter\profiledtbfillcategory\expandafter{\profilecategory}%
	\expandafter\profiledtbfilltrooptype\expandafter{\unit@type}%
	\expandafter\ifblank\expandafter{\unit@invocation}{}{\expandafter\profiledtbfillinvocation\expandafter{\unit@invocation}}%
	\expandafter\profiledtbfillcarac\expandafter{\unit@profile}
	}{}%
}


%%% Changelog commands %%%

\newcommand{\newlog}[2]{%
\vspace*{0.2cm}\noindent{\antiquefont\Large\textbf{V#1}}
\parselist[,]{#2}{\locallists@changelist}%
\begin{itemize}[itemsep=0pt]%
\forlistloop{\item[-]}{\locallists@changelist}%
\end{itemize}%
}

\newcommand{\startchangelog}{\begin{multicols}{2}\vspace*{-0.2cm}}
\def\endchangelog{\end{multicols}}


\newcommand{\booktitle}{Dynasties Immortelles}
\newcommand{\version}{0.99.9}
\newcommand{\frenchversion}{2.1}
\newcommand{\booklogo}{pics/logo_big_UD.png}
\newcommand{\tocfirstcolumn}{%
\tocentry{widerulestitle}{\labels@armywiderules}

\tocentry{specialrulestitle}{\labels@armyspecialrules}

\tocentry{armourytitle}{\labels@armoury}

\tocentry{monarchstitle}{Monarques des morts-vivants}

\tocentry{magicalitemstitle}{\labels@magicalitems}

\tocentry{lordtitle}{\labels@armylist}

\tocentry{QRStitle}{\labels@quickrefsheet}
}
\hypersetup{pdftitle={T9A - \booktitle}}

% Spells

\newcommand{\sandattribute}{Les Morts sans Repos}
\newcommand{\heavensspellone}{Ouragan}
\newcommand{\heavensspellfour}{Fléau du Ponant}
\newcommand{\sandsspellfour}{Jugement Divin}
\newcommand{\sandsspellfive}{Sables Mouvants}
\newcommand{\deathsignature}{Le Baiser de la Faucheuse}
\newcommand{\lightsignature}{Regard Embrasé}


% Army wide rules

\newcommand{\risen}{Résurrection}
\renewcommand{\labels@Invocation}{\risen}
\newcommand{\rulersofthedead}{Maître des Morts}

% Army special rules

\newcommand{\dusttodust}{De la Poussière à la Poussière}
\newcommand{\undeadconstruct}{Construction Mort-Vivante}
\newcommand{\undyingwill}{Volonté Immuable}
\newcommand{\necromanticaura}{Aura Nécromantique}
\newcommand{\mummyscurse}{Malédiction de la Momie}
\newcommand{\undergroundambush}{Embuscade Souterraine}

% Monarques

\newcommand{\monarchsofundead}{Monarques des Morts-Vivants}
\newcommand{\commanderoftheterracottaarmy}{Roi de l'Armée de Terre Cuite}
\newcommand{\lordofthebarrowlegion}{Seigneur de la Légion des Tertres}

% Armoury

\newcommand{\aspenbow}{Arc Aspic}
\newcommand{\greataspenbow}{Grand Arc Aspic}
\newcommand{\giantaspenbow}{Arc Aspic Géant}

% Magical Items

\newcommand{\vanquishereternal}{L’Éternel Vainqueur}
\newcommand{\scourgeofkings}{Fléau des Rois}

\newcommand{\crownofthepharaohs}{Couronne des Pharaons}
\newcommand{\armourofeternities}{Armure des Éternités}

\newcommand{\broochofthesun}{Broche Solaire}

\newcommand{\deathmaskofteput}{Masque de Mort de Teput}
\newcommand{\sandstormcloak}{Cape des Tempêtes de Sable}
\newcommand{\chariotofnephetra}{Char Solaire de Nephet-Râ}

\newcommand{\bookofthedead}{Livre des Morts}

\newcommand{\banneroftheentombed}{Bannière des Ensablés}

% Army common type special rules

\newcommand{\undeadcommonrules}{Règles Spéciales des Morts-Vivants}


% Other Rules

% Hiérarque 

\newcommand{\soulconduit}{Chenal Spirituel}

% Héraut des Tombes

\newcommand{\guardianswrath}{Colère des Gardiens}
\newcommand{\royalguard}{Escorte Royale}

% Nécrarchitecte

\newcommand{\mastermason}{Maître Maçon}
\newcommand{\masterofstone}{Maître de la Pierre}
\newcommand{\masonsmenagerie}{Ménagerie du Maçon}

% Guetteurs des Dunes

\newcommand{\petrifyinggaze}{Vision Pétrifiante}

% Faucheurs Ailés

\newcommand{\autonomous}{Autonomes}

% Colosse

\newcommand{\scalesofdestiny}{Balance du Destin}

% Sarcophage de Phatep

\newcommand{\divinelight}{Lumière Divine}
\newcommand{\phatepscurse}{Malédiction de Phatep}

% Arche d'Alliance

\newcommand{\divineprotection}{Protection Divine}
\newcommand{\sacredark}{Arche Sacrée}

% Catapulte d'Ossements

\newcommand{\cursedammunition}{Projectiles Maudits}


% Characters

\newcommand{\pharaoh}{Pharaon}
\newcommand{\pharaohs}{Pharaons}
\newcommand{\deathculthierarch}{Hiérarque du Culte des Morts}
\newcommand{\deathculthierarchshort}{Hiérarque du Culte des M.}
\newcommand{\deathculthierarchs}{Hiérarques du Culte des Morts}
\newcommand{\nomarch}{Nomarque}
\newcommand{\nomarchs}{Nomarques}
\newcommand{\deathcultaccolyte}{Acolyte du Culte des Morts}
\newcommand{\deathcultaccolyteshort}{Acolyte du Culte des M.}
\newcommand{\deathcultaccolytes}{Acolytes du Culte des Morts}
\newcommand{\tombharbinger}{Héraut des Tombes}
\newcommand{\tombharbingers}{Héraut des Tombes}
\newcommand{\tombarchitect}{Architecte des Tombes}
\newcommand{\tombarchitects}{Architectes des Tombes}

% Core

\newcommand{\skeleton}{Squelette}
\newcommand{\skeletons}{Squelettes}
\newcommand{\skeletonarcher}{Archer Squelette}
\newcommand{\skeletonarchers}{Archers Squelettes}
\newcommand{\skeletoncavalry}{Cavaliers Squelettes}
\newcommand{\skeletoncavalrySING}{Cavalier Squelette}
\newcommand{\skeletonchariot}{Char Squelette}
\newcommand{\skeletonchariots}{Chars Squelettes}


% Special

\newcommand{\necropolisguard}{Garde des Nécropoles}
\newcommand{\necropolisguards}{Gardes des Nécropoles}
\newcommand{\scarabswarm}{Nuée de Scarabées}
\newcommand{\scarabswarms}{Nuées de Scarabées}
\newcommand{\shabti}{Shabti}
\newcommand{\shabtis}{Shabtis}
\newcommand{\tombcataphract}{Cataphracte Tombale}
\newcommand{\tombcataphracts}{Cataphractes Tombales}
\newcommand{\greatvulture}{Grand Vautour}
\newcommand{\greatvultures}{Grands Vautours}
\newcommand{\sandscorpion}{Scorpion des Sables}
\newcommand{\sandscorpions}{Scorpions des Sables}
\newcommand{\sandstalker}{Guetteur des Dunes}
\newcommand{\sandstalkers}{Guetteurs des Dunes}
\newcommand{\battlesphinx}{Sphinx de Guerre}
\newcommand{\battlesphinxs}{Sphinx de Guerre}


% Rare

\newcommand{\wingedreaper}{Faucheur Ailé}
\newcommand{\wingedreapers}{Faucheurs Ailés}
\newcommand{\dreadsphinx}{Sphinx de l'Effroi}
\newcommand{\dreadsphinxs}{Sphinx de l'Effroi}
\newcommand{\colossus}{Colosse}
\newcommand{\colossuss}{Colosses}
\newcommand{\casketofphatep}{Sarcophage de Phatep}
\newcommand{\casketofphateps}{Sarcophages de Phatep}
\newcommand{\charnelcatapult}{Catapulte d'Ossements}
\newcommand{\charnelcatapults}{Catapultes d'Ossements}


% Mounts

\newcommand{\skeletalhorse}{Cheval Squelette}
\newcommand{\skeletalhorses}{Chevaux Squelettes}
\newcommand{\amuut}{Amuut}
\newcommand{\amuuts}{Amuuts}
\newcommand{\royalsphinx}{Sphinx Royal}
\newcommand{\royalsphinxs}{Sphinx Royaux}
\newcommand{\arkofages}{Arche des Âges}
\newcommand{\arksofages}{Arches des Âges}


% Profile names

\newcommand{\rider}{Cavalier}
\newcommand{\charioteer}{Aurige}
\newcommand{\charioteers}{Auriges}
\newcommand{\cataphract}{Cataphracte}
\newcommand{\cataphracts}{Cataphractes}
\newcommand{\ark}{Arche}
\newcommand{\guard}{Garde}
\newcommand{\boundspirits}{Esprits Asservis}
\newcommand{\sphinx}{Sphinx}
\newcommand{\casket}{Sarcophage}
\newcommand{\caskets}{Sarcophages}


% Profile wording

\newcommand{\ifgeneral}{Si Général}
\newcommand{\twoadditionalskeletalhorses}{Deux \skeletalhorses{} supplémentaires}
\newcommand{\increasedbasesize}{Taille de socle augmentée à \unit{100x100}{\milli\meter}}
\newcommand{\exchangescoutandlighttroopsforla}{Remplacer les règles \scout{} et \lighttroops{} par une \la{}}
\newcommand{\exchangeshieldandvanguardforaspenbow}{Remplacer le \shield{} et la règle \vanguard{} par un \aspenbow{}}
\newcommand{\replacecharnelcatapultwithcursedammunition}{Remplacer la \charnelcatapult{} par les\newline \cursedammunition{}}


% Profile rules

% Hiérarque du Culte des Morts

\newcommand{\soulconduitrule}{%
Si une figurine alliée avec cette règle se trouve sur le champ de bataille au début de votre Phase de Magie, les Flux de Magie sont modifiés pour vous en 1D3+7 à la place de 2D6. Le nombre de Dés de Dissipation de l'adversaire est toujours de 6, sans compter les éventuels dés additionnels de la Canalisation ou d'autres apports.
}

%Héraut des Tombes

\newcommand{\guardianswrathrule}{%
La figurine et son unité gagnent la règle \hatred{}. Les montures ne sont pas affectées. 
}

\newcommand{\royalguardrule}{%
Quand un \pharaoh{} ou un \nomarch{} dans la même unité que la figurine est attaqué au Corps à Corps, une touche peut être transférée à la figurine à la place, avant le jet pour blesser. La figurine ne peut intercepter qu'une seule touche par Manche de Corps à Corps, et ne peut pas le faire pour des touches subies lors d'un Défi.
}

%Nécrarchitecte

\newcommand{\mastermasonrule}{%
La portée de la règle \masterofstone{} est augmentée de \distance{6}.
}

\newcommand{\masterofstonerule}{%
Après le déploiement des Éclaireurs, puis au début de chacun de vos Tours de Joueur, la figurine peut donner la règle \regeneration{5} à une unité alliée à moins de \distance{12} si elle est entièrement composée de figurines avec la règle \undeadconstruct{}. L'effet dure jusqu'à la fin du prochain Tour de Joueur, et prend fin si l'\tombarchitect{} est retiré du jeu.
}

\newcommand{\masonsmenagerierule}{%
Si cette figurine est le Général, une unité de \shabtis{} de 6 figurines au plus peut être prise comme choix d'Unité de Base plutôt qu'Unité Spéciale.
}

%Ark of Ages

\newcommand{\sacredarkrule}{%
Le Sorcier monté sur l'\arkofages{} ajoute \distance{3} à la portée de ses sorts qui ne sont pas de type Vortex. Il gagne aussi trois nouveaux sorts en plus de ses sorts normaux :
\begin{itemize}[label={-}]
\item \heavensspellone{} (Discipline \heavens{}),
\item \heavensspellfour{} (Discipline \heavens{}),
\item \sandsspellfive{} (Discipline \sands{}).
\end{itemize}
En cas de sort en double, suivez les règles habituelles.
}

\newcommand{\divineprotectionrule}{%
Lorsqu'elle est montée par le Hiérophante, la figurine gagne une \wardsave{4}.
}

%Guetteurs des Dunes

\newcommand{\petrifyinggazerule}{%
La figurine peut effectuer une Attaque Spéciale de Tir :\newline
\range{12}, \Strength{} 2, \armourpiercing{6}, \multipleshots{1D6+1}, \quicktofire{}. Lors du jet pour blesser, prenez en compte l'Initiative de la cible plutôt que son Endurance.
}

%Faucheurs Ailés

\newcommand{\autonomousrule}{%
L'unité peut effectuer des Marches Forcées même si elle est hors de portée de la \inspiringpresence{} du Général.
}

%Colosse

\newcommand{\scalesofdestinyrule}{%
Arme de Corps à Corps. Type : \hw{}. Le porteur a -1 Attaque au Corps à Corps. Le porteur gagne deux \boundspells{4} : \deathsignature{} (Discipline \death{}) et \lightsignature{} (Discipline \light{}).
}

\newcommand{\giantaspenbowrule}{%
\textbf{\artilleryweapon} de type \textbf{\boltthrower}.\newline
\range{48}, Force 6, \multiplewounds{1D3}{}, \armourpiercing{6}. Cette arme ignore tous les modificateurs pour toucher au tir.
}

%Sarcophage de Phatep

\newcommand{\divinelightrule}{%
La figurine ajoute +1 à votre jet de Canalisation pendant les Phases de Magie alliées. Les Sorciers ennemis à moins de \distance{36} d'au moins un \casketofphatep{} subissent un malus de -1 sur leurs jets de lancement de sort. Lorsque le \casketofphatep{} est retiré du jeu, toutes les unités à moins de \distance{12} subissent 3D3+3 touches de Force 1 avec la règle \armourpiercing{6}.
}

\newcommand{\phatepscurserule}{%
Le \casketofphatep{} possède un \boundspell{4} : \sandsspellfour{} (Discipline \sands{}). Il ne peut l'utiliser que s'il ne s'est pas déplacé durant ce Tour de Joueur.
}

%Catapulte d'Ossements

\newcommand{\charnelcatapultrule}{%
\textbf{\artilleryweapon} de type \textbf{\catapult{} (\distance{3})}.\newline
\range{12-60}, \Strength{} 3 [9], [\multiplewounds{\ordnance}{}].
}

\newcommand{\cursedammunitionrule}{%
\textbf{\artilleryweapon} de type \textbf{\catapult{} (\distance{5})}.\newline
\range{12-48}, \Strength{} 3, \flamingattacks{}, \magicalattacks{}.

Une unité qui subit au moins une perte causée par cette arme doit subir un test de Panique comme si elle avait subi au moins 25\% de pertes. Ce test est fait avec un malus de -1 en Commandement.
}

% QRS

\newcommand{\QRSnote}{%
\noindent\refsymbol{} Perd les \charioteers{} quand il sert de monture.
}

\newcommand{\allcharacters}{Personnages}
\newcommand{\allundeadconstructs}{Constructions Mort-Vivantes}
\newcommand{\allwarmachines}{Machines de Guerre}












\begin{document}

\newgeometry{margin=1in}

% Table options
\arrayrulecolor{black!30}
\setlength{\arrayrulewidth}{0.5pt}
\renewcommand{\arraystretch}{1.2}

\begin{titlepage}
\begin{center}

\ifdef{\booktitle}{}{\newcommand{\booktitle}{Missing title}}
\ifdef{\version}{}{\newcommand{\version}{Missing version}}

{\antiquefont\fontsize{40}{48}\selectfont\noindent\labels@fantasybattles

\labels@NinthAge}

\vspace*{0.5cm}
\ifdef{\booklogo}{%
\includegraphics[height=10cm]{\booklogo}%
}{%
\includegraphics[height=10cm]{../Layout/pics/logo_9th.png}%
}

\vspace*{-1cm}
{\antiquefont\fontsize{50}{60}\selectfont \booktitle
\vspace{0.4cm}

\fontsize{14}{16.8}\selectfont \labels@armyrules{}

Beta v\version{} - \today{}}

\ifdef{\frenchversion}{{\fontsize{14}{16.8}\selectfont \vspace{0.2cm}\noindent\texttt{VF \frenchversion}}}{}
\vfill

\begin{tabular}{@{}m{2cm}@{\hskip 20pt}m{13cm}@{}}
\includegraphics[width=2cm]{../Layout/pics/seal_9th.png} &
{\fontsize{10}{12}\selectfont \textcolor{black!50}{\noindent\labels@frontpagecredits}}

\ifdef{\frontpageaddstuff}{{\fontsize{10}{12}\selectfont \noindent\textcolor{black!50}{\frontpageaddstuff}}}{}

\vspace*{10pt}
\noindent{\fontsize{10}{12}\selectfont \textcolor{black!50}{\labels@license}}
\tabularnewline
\end{tabular}


\end{center}

\newpage

\thispagestyle{empty}

{\fontsize{10}{12}\selectfont

\begin{center}\noindent{\Largerfontsize\textbf{\labels@tableofcontents}}\end{center}

\vspace*{0.2cm}\begin{multicols}{2}

\tocfirstcolumn

\vspace*{\fill}\columnbreak

\tocentry{lordtitle}{\labels@lords}

\tocentry{herotitle}{\labels@heroes}

\ifdef{\tocmounts}{\tocentry{mountstitle}{\tocmounts}}{}

\tocentry{coretitle}{\labels@coreunits}

\tocentry{specialtitle}{\labels@specialunits}

\tocentry{raretitle}{\labels@rareunits}

\vspace*{\fill}\end{multicols}

\ifdef{\labels@introduction}{\vspace{0.7cm}\labels@introduction}{\vphantom{1pt}}
\vfill

\noindent\newrule{\labels@rulechanges}

\bigskip
\noindent \labels@latexcredit
}


\end{titlepage}

\restoregeometry

\startarmywiderules

\armyspecialruleentry{\risen}

Certains profils d'unité contiennent une catégorie appelée \risen{} qui donne le nombre de Points de Vie Ressuscités grâce à l'attribut \sandattribute{} de la Discipline \sands{}.

\armyspecialruleentry{\rulersofthedead}

Votre armée doit contenir au moins un Sorcier qui utilise la Discipline \sands{} que vous désignez comme étant votre Hiérophante. Il doit être indiqué sur votre liste d'armée. Le Hiérophante et toutes les figurines de son unité gagnent la règle \regeneration{6}.

\closearmywiderules








\vspace*{1.5cm}
\startarmyspecialrules

\armyspecialruleentry{\dusttodust}

À la fin de n'importe quelle phase durant laquelle le Hiérophante est retiré du jeu en tant que perte, toute unité de l'armée qui possède au moins une figurine avec la règle \dusttodust{} doit faire un test de Commandement. Si le test échoue, l'unité subit un nombre de blessures équivalent à la différence entre le résultat obtenu et la valeur de Commandement du test, sans aucune sauvegarde autorisée. Ces blessures sont réparties comme pour la règle \unstable{}, mais ne peuvent pas être assignées à une figurine n'ayant pas la règle \dusttodust{}. Le montant de blessures est réduit de un si l'unité est à portée de la règle \holdyourground{}.

À la fin du Tour de Joueur suivant la mort du Hiérophante, un nouveau Hiérophante peut être choisi. Pour ce faire, vous devez nommer un autre Personnage éligible, c'est à dire un Sorcier utilisant la Discipline \sands{}. Ce Personnage est le nouveau Hiérophante.

Au début de chacun de vos Tours de Joueur sans nouveau Hiérophante, toute unité qui possède au moins une figurine avec la règle \dusttodust{} doit faire un nouveau test de Commandement, et subir des blessures comme décrit ci-dessus.

\armyspecialruleentry{\undeadconstruct}

La figurine gagne les règles \dusttodust{} et \undead{}. Par ailleurs, une unité subit une blessure de moins lors de l'application de ces règles si au moins la moitié des figurines de l'unité possède la règle \undeadconstruct{}.

\armyspecialruleentry{\undyingwill}

Au début de n'importe quelle Phase de Corps à Corps, le Personnage peut conférer la valeur non modifiée de sa Capacité de Combat à toutes les figurines avec la règle \undead{} de son unité. S'il est monté sur une \largetarget{}, il peut à la place décider de conférer ce bonus à une unité alliée avec la règle \undead{} à moins de \distance{6}, à moins qu'il ne soit lui-même engagé au corps à corps. Dans ce dernier cas, il ne peut conférer ce bonus qu'à une unité \undead{} engagée au corps à corps avec la même unité ennemie. Dans tous les cas, l'effet dure jusqu'à la fin de cette phase.

\armyspecialruleentry{\necromanticaura}

Toutes les unités alliées dans un rayon de \distance{6} d'une ou plusieurs figurines avec cette règle réduisent le nombre de blessures qu'elle subissent par les règles \dusttodust{} et \unstable{} de 1. Les figurines avec la règle \necromanticaura{} ne peuvent pas en bénéficier elles-mêmes.

\armyspecialruleentry{\mummyscurse}

Quand la figurine est retirée de la partie, la figurine qui lui a causé la blessure fatale subit une touche de Force 6 avec la règle \armourpiercing{6}. Si plusieurs figurines sont impliquées dans cette dernière blessure, déterminez aléatoirement qui est touché.

\armyspecialruleentry{\undergroundambush}

L'unité suit la règle \ambush{}. Cependant, au lieu d'entrer sur le champ de bataille à partir d'un bord de table, elle apparait à un endroit appelé le Point Souterrain. Au moment de le déterminer, le propriétaire commence par désigner un point à plus de \distance{3} de toute unité ennemie et à plus de \distance{0,5} de tout Terrain Infranchissable. Faites dévier ce point de \distance{2D6} pour obtenir le point souterrain. L'unité est alors placée avec le front du premier rang ou du rang arrière en contact avec le point souterrain. Si le point se trouve sous une unité ennemie, placez plutôt l'unité embusquée en contact socle à socle avec le front avant de cette unité, en maximisant les figurines en contact comme en cas de charge. L'unité qui vient d'arriver est considérée comme ayant chargé et aucune réaction de charge ne peut être déclarée. S'il n'est pas possible de placer l'unité embusquée, considérez que le jet d'\ambush{} est raté et relancez le dé au prochain tour.

\closearmyspecialrules









\vspace*{1.5cm}
\startarmyarmoury

\startitemlistonecol

\listitemonecol{\aspenbow} Arme de Tir. \range{24}, Force 3, \volleyfire{}. Cette arme ignore tous les modificateurs pour toucher au tir.

\listitemonecol{\greataspenbow} Arme de Tir. \range{36}, Force 5, \volleyfire{}. Cette arme ignore tous les modificateurs pour toucher au tir. Compte comme une \hw{} avec -1 en Force au Corps à Corps.

\enditemlistonecol

\closearmyarmoury






\newpage
\toctarget{monarchstitle}{\startarmynewsection{\monarchsofundead}}

\spaceaftersection{}

\begin{center}
\noindent Ces options représentent des formes alternatives de l'armée que l'on peut rencontrer dans le vaste monde. Un \pharaoh{} peut décider de commander une des deux forces suivantes plutôt qu'une armée classique des Dynasties Immortelles.
\end{center}

\begin{multicols}{2}\raggedcolumns

\begin{center}\armynewsubsection{\commanderoftheterracottaarmy}\end{center}

\begin{itemize}[label={-}, leftmargin=*]
\item Tous les \skeletons{}, \skeletonarchers{}, \skeletoncavalry{} et \necropolisguards{} \textbf{doivent} être améliorés au coût de \pts{3}\permodel{} pour gagner +1 en Endurance, -1 en Initiative et la règle \undeadconstruct{}.

\item Une unité de \necropolisguards{} ne peut pas contenir plus de \textbf{30} figurines.

\item Toutes les autres unités n'ayant pas déjà la règle \undeadconstruct{}, y-compris les Personnages, \textbf{doivent} être améliorés au coût de \pts{15}\permodel{} pour gagner +1 en Endurance, -1 en Initiative et la règle \undeadconstruct{}. Elles perdent la règle \flammable{} si elles l'avaient.

\item Une unité de \skeletonchariots{} ne peut pas contenir plus de \textbf{6} figurines.

\item La caractéristique \risen{} de toutes les figurines est fixée à 1.

\item Les figurines non volantes qui possèdent la ou les règles \undergroundambush{} ou \lighttroops{} les perdent, et ne peuvent en aucun cas gagner ces règles.

\item Les \greatvultures{}, \scarabswarms{} et \wingedreapers{} ne peuvent pas être pris dans cette armée.
\end{itemize}

\vspace*{\fill}\columnbreak

\begin{center}\armynewsubsection{\lordofthebarrowlegion}\end{center}

\begin{itemize}[label={-}, leftmargin=*]
\item Les \skeletons{} et \skeletonarchers{} \textbf{doivent} pren\-dre une \ha{} pour \pts{2}\permodel{}

\item Les \skeletons{} peuvent échanger la \spear{} et le \shield{} contre une \halberd{} pour \pts{1}\permodel{}

\item Les \skeletoncavalry{} peuvent disposer d'une \lance{} pour \pts{3}\permodel{} et d'un \barding{} pour \pts{3}\permodel{}

\item Les \skeletonchariots{} \textbf{doivent} prendre une \ha{} pour \pts{5}\permodel{} et peuvent prendre une \halberd{} pour \pts{5}\permodel{} Une unité de \skeletonchariots{} ne peut pas contenir plus de \textbf{7} figurines.

\item Les \necropolisguards{} \textbf{doivent} prendre une \ha{} pour \pts{2}\permodel{} Une unité de \necropolisguards{} ne peut pas contenir plus de \textbf{35} figurines.

\item Les \scarabswarms{} \textbf{doivent} gagner la règle \ethereal{} pour \pts{15}\permodel{} Une unité de \scarabswarms{} ne peut pas contenir plus de \textbf{4} figurines.

\item La caractéristique \risen{} des figurines avec la règle \ethereal{} est fixée à 1.

\item L'armée ne peut pas contenir de figurines de \monstrouscavalry{} ou avec la règle \largetarget{}.

\item Les figurines qui possèdent la ou les règles \undergroundambush{} ou \scout{} les perdent, et ne peuvent en aucun cas gagner ces règles.

\item Les figurines non volantes qui possèdent une \ha{} et la règle \lighttroops{} perdent cette règle, et ne peuvent en aucun cas la regagner.
\end{itemize}

\vspace*{\fill}\end{multicols}

\closearmynewsection








\startarmymagicalitems

\armymagicalweapons

\startpricelist

\pricelistitem{\vanquishereternal}{55}Figurine à pied uniquement.

 Type : \halberd{}. Les attaques portées avec cette arme possèdent la règle \lethalstrike{}. Le porteur peut utiliser l'arme de deux manières : Frappe Ciblée ou Frappe Faucheuse. Choisissez au début de chaque Manche de Corps à Corps.
 \begin{itemize}[label={-}]
\item \textbf{Frappe Ciblée} : Les attaques portées avec cette arme gagnent la règle \multiplewounds{1D3}{}.
\item \textbf{Frappe Faucheuse} : Toutes les attaques du porteur sont échangées contre une touche automatique à chaque figurine en contact socle à socle avec le porteur. Toutes les figurines qui peuvent effectuer une attaque de soutien contre le porteur subissent aussi une attaque qui touche sur 4+. Ces attaques suivent les règles normales de l'arme (\halberd{}, \lethalstrike{}).
\end{itemize}

\pricelistitem{\scourgeofkings}{50/30} Type : \hw{}. Les attaques effectuées avec cette arme possèdent la règle \armourpiercing{1}. Chaque touche réussie avec cette arme occasionne une touche supplémentaire automatique pour un total de deux touches.

\endpricelist

\armymagicalarmour

\startpricelist

\pricelistitem{\crownofthepharaohs}{45} Type : Aucun (Sauvegarde d'Armure 6+). Le porteur peut utiliser la règle \undyingwill{} sur une seule unité alliée à moins de \distance{6}. Si le porteur est engagé au corps à corps, il ne peut transférer sa Capacité de Combat non modifiée qu'à une unité avec la règle \undead{} alliée au corps à corps avec la même unité ennemie.

De plus, vous pouvez choisir d'utiliser la règle \undyingwill{} durant la Phase de Tir en transférant sa Capacité de Tir à la place de sa Capacité de Combat. La règle \undyingwill{} ne peut être utilisée que lors d'une seule phase par Tour de Joueur. Si le porteur est monté sur une \largetarget{}, ajoutez \distance{6} à la portée de l'objet.

\pricelistitem{\armourofeternities}{35}Figurine à pied uniquement.

Type : \platearmour{}. Le porteur gagne +1 Point de Vie.

\endpricelist

\armytalismans

\startpricelist

\pricelistitem{\broochofthesun}{15} Au début de chaque Manche de Corps à Corps, choisissez un élément de figurine en contact socle à socle avec le porteur. Il subit un malus de -1 sur sa caractéristique Attaque, jusqu'à un minimum de 1.

\endpricelist

\armyenchanteditems

\startpricelist

\pricelistitem{\deathmaskofteput}{35} Au début de chaque Manche de Corps à Corps, choisissez la règle \inspiringpresence{} ou \holdyourground{}. Les unités ennemies en contact socle à socle avec le porteur ne peuvent plus bénéficier de la règle choisie pour la durée de cette phase.

\pricelistitem{\sandstormcloak}{30} L'unité du porteur gagne la règle \hardtarget{}.

\pricelistitem{\chariotofnephetra}{25} Figurine sur \chariot{} uniquement.

Les \impacthits{} causées par le \chariot{} du porteur et les Attaques de ses montures gagnent les règles \flamingattacks{} et \magicalattacks{} et sont résolues avec +1 en Force.

\endpricelist

\armyarcaneitems

\startpricelist

\pricelistitem{\bookofthedead}{50/35} La valeur de lancement des sorts de la Discipline \sands{} est réduite de 1 pour le porteur. De plus, l'Attribut de la Discipline \sands{} lancé par le porteur Ressuscite 1 PV supplémentaire sur la ou les unités ciblées qui ne sont ni des Personnages ni des Grandes Cibles.

\endpricelist

\armymagicalbanners

\startpricelist

\pricelistitem{\banneroftheentombed}{65} Les figurines avec la règle \undergroundambush{} de l'armée peuvent ajouter +1 à leur jet d'\ambush{}. Signalez-le avant de lancer le dé. Si vous utilisez ce bonus, le point souterrain doit se trouver à moins de \distance{24} du porteur et ne dévie que de \distance{1D6}.

\endpricelist

\closearmymagicalitems








%%% START OF THE ARMYLIST - Translators shouldn't have to edit it %%%

\armylist

\lordstitle

\showunit{
	name={\pharaoh},
	cost=160,
	profile={< 4 6 3 5 5 4 3 4 10},
	type=\infantry{},
	unitsize=1,
	invocation=1,
	basesize=20x20,
	commontype=\undeadcommonrules{},
	commonspecialrules={\undead{},\dusttodust{}},
	specialrules={\mummyscurse{},\flammable{},\fear{},\undyingwill{}},
	armour={\la},
	options={
		\magicalitemsallowance{}=\upto{}<100,
		\shield{}=3,
		\ha{}=12,
		\greataspenbow{}=10,
		\weapononechoice{
			\flail{}=5,
			\pw{}=5,
			\halberd{}=10,
			\gw{}=15,
			\lance{}=15,
		},
		\onechoiceonly{} \only{\general}{
			\commanderoftheterracottaarmy{}=\free{},
			\lordofthebarrowlegion{}=\free{},
		},
	}
	mounts={\skeletalhorse{}=20,\skeletonchariot{}=35,\royalsphinx{}=185},
}
	
\showunit{
	name={\deathculthierarch},
	cost=170,
	profile={< 4 3 3 3 4 3 2 1 8},
	type=\infantry{},
	unitsize=1,
	invocation=1,
	basesize=20x20,
	commontype=\undeadcommonrules{},
	commonspecialrules={\undead{},\dusttodust{}},
	magiclevel=3,
	magicpaths={\light{},\death{},\sands},
	options={
		\magiclevel{4}=30,
		\soulconduit{}=50,
		\magicalitemsallowance{}=\upto{}<100,
		},
	mounts={\skeletalhorse{}=20,\arkofages{}=170},
	unitrules={
		\unitrule{\soulconduit{}}{\soulconduitrule{}}
		},	
}
		






\heroestitle	

%\showunit{
%	name={\nomarch{}},
%	cost={100},
%	profile={< 4 5 3 4 5 3 3 3 9},
%	type=\infantry{},
%	unitsize={1},
%	invocation={1},
%	basesize=20x20,
%	commontype=\undeadcommonrules{},
%	commonspecialrules={\undead{}, \dusttodust{}},
%	specialrules={\mummyscurse{}, \undyingwill{}, \flammable{}, \fear{}},
%	armour={\la{}},
%	options={
%		\magicalitemsallowance{}=\upto{}<50,
%		\shield{}=3
%		\ha{}=12,
%		\aspenbow{}=3,
%		\weapononechoice
%			{
%			\pw{}=3,
%			\flail{}=3,
%			\halberd{}=4,
%			\lance{}=6,
%			\gw{}=6,
%			}
%		}
%	mounts={\skeletalhorse{}=20,\skeletonchariot{}=35,\royalsphinx{}=200},
%}
%
%	
%\showunit{
%	name={\deathcultaccolyte{}},
%	cost={65},
%	profile={< 4 3 3 3 3 2 2 1 7},
%	type=\infantry{},
%	unitsize={1},
%	invocation={1},
%	basesize=20x20,
%	commontype=\undeadcommonrules{},
%	commonspecialrules={\undead{},\dusttodust{}},
%	magiclevel=1,
%	magicpaths={\light, \death, \sands},
%	options={
%		\magicalitemsallowance{}=\upto{}<50,
%		\magiclevel{2}=25,
%		},
%	mounts={\skeletalhorse{}=15,\arkofages{}=180},
%}
%	
%\showunit{
%	name={\tombharbinger{}},
%	cost={70},
%	profile={< 4 4 3 4 5 2 3 3 8},
%	type=\infantry{},
%	unitsize={1},
%	invocation={1},
%	basesize=20x20,
%	commontype=\undeadcommonrules{},
%	commonspecialrules={\undead{}, \dusttodust{}},
%	specialrules={\poisonedattacks{}, \lethalstrike{}, \flammable{}, \guardianswrath{}, \royalguard{}},
%	armour={\la{}},
%	options={
%		\bsb{}=25,
%		\magicalitemsallowance{}=\upto{}<50,
%		\shield{}=3
%		\ha{}=12,
%		\aspenbow{}=3,
%		\weapononechoice
%			{
%			\pw{}=3,
%			\flail{}=3,
%			\halberd{}=4,
%			\lance{}=6,
%			\gw{}=6,
%			},
%		}
%	mounts={\skeletalhorse{}=20,\skeletonchariot{}=50,\amuut{}=50},
%	unitrules={
%		\unitrule{\guardianswrath{}}{\guardianswrathrule{}}
%		\unitrule{\royalguard{}}{\royalguardrule{}}
%		},	
%}
%
%\showunit{
%	name={\tombarchitect{}},
%	cost={50},
%	profile={< 4 4 3 4 4 2 3 2 7},
%	type=\infantry{},
%	unitsize={1},
%	invocation={1},
%	basesize=20x20,
%	commontype=\undeadcommonrules{},
%	commonspecialrules={\undead{}, \dusttodust{}},
%	specialrules={\flammable{}, \masterofstone{}, \masonsmenagerie{}},
%	armour={\la{}},
%	options={
%		\magicalitemsallowance{}=\upto{}<50,
%		\weapononechoice
%			{
%			\pw{}=3,
%			\lance{}=6,
%			},
%		\mastermason{}=25,		
%		}
%	mounts={\skeletalhorse{}=15,\skeletonchariot{}=50,\amuut{}=50},
%	unitrules={
%		\unitrule{\masonsmenagerie{}}{\masonsmenagerierule{}}
%		\unitrule{\masterofstone{}}{\masterofstonerule{}}
%		\unitrule{\mastermason{}}{\mastermasonrule{}}
%		},	
%}








\mountstitle

%\showunit{
%	name={\skeletalhorse{}},
%	profile={< 8 2 - 3 3 1 2 1 3},
%	type=\warbeast{},
%	basesize=25x50,
%	armour={\mountsprotection{6}},
%	options={
%			\barding{}=10
%		}, 		
%	}
%	
%\showunit{
%	name={\amuut{}},
%	profile={< 7 3 - 5 4 3 3 3 8},
%	type=\monstrousbeast{},
%	basesize=50x100,
%	commontype=\undeadcommonrules{},
%	commonspecialrules={\undeadconstruct{}},
%	specialrules={\poisonedattacks{}, \fear{}},
%	armour={\mountsprotection{6}},
%		}
%	
%\showunit{
%	name={\royalsphinx{}},
%	profile={< 6 4 - 5 6 5 1 4 8},
%	type=\monstrousbeast{},
%	basesize=50x100,
%	commontype=\undeadcommonrules{},
%	commonspecialrules={\undeadconstruct{}},
%	specialrules={\poisonedattacks{}, \largetarget{}, \stomp{1D6}, \terror{}},
%	options={
%			\necromanticaura{}=15,
%			\lethalstrike{}=25,
%			}, 		
%	}
%	
%\showunit{
%	name={\skeletonchariot{}},
%	profile={
%	\skeletonchariot{}< - - - 4 4 3 - - -,
%	\skeletalhorse{}(2)< 8 2 - 3 - - 2 1 -,
%	},
%	type=\chariot{},
%	basesize=50x100,
%	armour={\mountsprotection{6}},
%	options={
%			\twoadditionalhorses{}=\free,
%			}, 		
%	}
%	
%\showunit{
%	name={\arkofages{}},
%	profile={
%	\arkofages{}< - - - 4 5 5 - - -,
%	\guard{}(2)< - 3 3 4 - - 3 1 8,
%	\boundspirits{}(1)< 4 2 - 2 - - 2 6 -,
%	},
%	type=\chariot{},
%	basesize=60x100,
%	commontype=\undeadcommonrules{},
%	commonspecialrules={\undeadconstruct{}},
%	specialrules={\poisonedattacks{}\only{\guard{}}, \magicalattacks{}, \lethalstrike{}\only{\guard{}}, \warplatform{}, \wardsave{5}, \divineprotection{}, \sacredark{}},
%	armour={\mountsprotection{5}},
%	weapons={\aspenbow{}\only{\guard{}}}
%	options={
%			\necromanticaura{}=15,
%			}, 	
%	unitrules={
%		\unitrule{\sacredark{}}{\sacredarkrule{}}
%		\unitrule{\divineprotection{}}{\divineprotectionrule{}}
%		},				
%	}	








\coreunitstitle		
	
%\showunit{
%	name={\skeletons{}},
%	QRSname={\skeleton{}},
%	cost={80},
%	profile={< 4 2 2 3 3 1 2 1 4},
%	type=\infantry{},
%	invocation={1D3+3},
%	unitsize=20,
%	maxmodels=60,
%	costpermodel=5,
%	basesize=20x20,
%	commontype=\undeadcommonrules{},
%	commonspecialrules={\undead{},\dusttodust{}},
%	armour={\la{},\shield{}},
%	options={
%		\spear{}=\free,
%	},
%	commandgroup={champion=10, musician=10, banner=10, veteranstandardbearer=yessir},
%}
%
%\showunit{
%	name={\skeletonarchers{}},
%	QRSname={\skeletonarcher{}},
%	cost={60},
%	profile={< 4 2 2 3 3 1 2 1 4},
%	type=\infantry{},
%	invocation={1D3+3},
%	unitsize=10,
%	maxmodels=30,
%	costpermodel=6,
%	basesize=20x20,
%	commontype=\undeadcommonrules{},
%	commonspecialrules={\undead{},\dusttodust{}},
%	weapons={\aspenbow{}},
%	armour={\la{}},
%	commandgroup={champion=10, musician=10, banner=10, veteranstandardbearer=yessir},
%}
%
%\showunit{
%	name={\skeletoncavalry{}},
%	cost={65},
%	profile={
%		\rider{}< 4 3 2 3 3 1 2 1 6,
%		\skeletalhorse{}< 8 2 - 3 3 1 2 1 3,
%	},
%	type=\cavalry{},
%	invocation={1D3+2},
%	unitsize=5,
%	maxmodels=20,
%	costpermodel=11,
%	basesize=25x50,
%	commontype=\undeadcommonrules{},
%	commonspecialrules={\undead{},\dusttodust{}},
%	specialrules={\vanguard{},\scout{},\lighttroops{}},
%	armour={\shield{},\mountsprotection{6}},
%	options={
%		\exchangescoutandlighttroopsforla{}=\permodel{}<1,
%		\onechoiceonly{
%			\exchangeshieldandvanguardforaspenbow{}=\free{},
%			\lightlance{}=\permodel{}<1,
%		},
%	},
%	commandgroup={champion=10, musician=10, banner=10, veteranstandardbearer=yessir},
%}
%
%\showunit{
%	name={\skeletonchariots{}},
%	cost={135},
%	profile={
%		\chariot{}< - - - 4 4 3 - - -,
%		\charioteer{} (2)< - 3 2 3 - - 2 2 7,
%		\skeletalhorse{} (2)< 8 2 - 3 - - 2 1 -,
%	},
%	type=\chariot{},
%	invocation={1D3+1},
%	unitsize=5,
%	maxmodels=10,
%	costpermodel=35,
%	basesize=50x100,
%	commontype=\undeadcommonrules{},
%	commonspecialrules={\undead{},\dusttodust{}},
%	weapons={\aspenbow{} \only{\charioteer{}},\lightlance{} \only{\charioteer{}}},
%	armour={\la{},\mountsprotection{6}},
%	options={
%		\lighttroops{}=\free{},
%	},
%	commandgroup={champion=10, musician=10, banner=10, veteranstandardbearer=yessir},
%}









\specialunitstitle

%\showunit{
%	name={\necropolisguard{}},
%	cost={70},
%	profile={< 4 3 3 4 4 1 3 1 8},
%	type=\infantry{},
%	invocation={1D3+1},
%	unitsize=10,
%	maxmodels=40,
%	costpermodel=11,
%	basesize=20x20,
%	commontype=\undeadcommonrules{},
%	commonspecialrules={\undead{},\dusttodust{}},
%	specialrules={\lethalstrike{},\bodyguard{},\magicalattacks{},\poisonedattacks{}},
%	armour={\la{}},
%	options={
%		\shield{}=\permodel{}<1,
%		\onechoiceonly{
%			\pw{}=\permodel{}<1,
%			\halberd{}=\permodel{}<2,
%		},
%	},
%	commandgroup={champion=10, musician=10, banner=10, bannerallowance=50},
%}
%
%\showunit{
%	name={\scarabswarms{}},
%	QRSname={\scarabswarm{}},
%	cost={70},
%	profile={< 5 3 - 2 2 5 1 5 10},
%	type=\swarm{},
%	invocation={1D3+3},
%	unitsize=2,
%	maxmodels=7,
%	costpermodel=25,
%	basesize=40x40,
%	commontype=\undeadcommonrules{},
%	commonspecialrules={\undead{},\dusttodust{}},
%	specialrules={\armourpiercing{1},\poisonedattacks{}\hardtarget{},\distracting{},\undergroundambush{}},
%}
%
%\showunit{
%	name={\shabtis{}},
%	QRSname={\shabti{}},
%	cost={100},
%	profile={< 6 4 2 5 4 3 3 3 8},
%	type=\monstrousinfantry{},
%	invocation={1},
%	unitsize=3,
%	maxmodels=10,
%	costpermodel=37,
%	basesize=40x40,
%	commontype=\undeadcommonrules{},
%	commonspecialrules={\undeadconstruct{}},
%	specialrules={\fear{}},
%	armour={\la{},\innatedefence{5}},
%	options={
%		\weapononechoice{
%			\greataspenbow{}=\free{},
%			\pw{}=\permodel{}<5,
%			\halberd{}=\permodel{}10,
%		}
%	},
%	commandgroup={champion=10, musician=10, banner=10, bannerallowance=25},
%}
%
%\showunit{
%	name={\tombcataphracts{}},
%	QRSname={\tombcataphract{}},
%	cost={165},
%	profile={
%		\rider{}< 4 4 3 4 4 1 3 2 8,
%		\amuut{}< 7 3 - 5 4 3 3 3 8,
%	},
%	type=\monstrouscavalry{},
%	invocation={1},
%	unitsize=3,
%	maxmodels=6,
%	costpermodel=55,
%	basesize=50x100,
%	commontype=\undeadcommonrules{},
%	commonspecialrules={\undeadconstruct{}},
%	specialrules={\fear{},\lethalstrike{} \only{\rider{}},\poisonedattacks{} \only{\amuut{}}},
%	weapons={\lightlance{} \only{\rider{}}},
%	armour={\la{},\innatedefence{5}, \mountsprotection{6}},
%	options={
%		\undergroundambush{}=20,
%		}
%	},
%	commandgroup={champion=10, musician=10, banner=10, bannerallowance=50},
%}
%
%\showunit{
%	name={\greatvultures{}},
%	QRSname={\greatvulture{}},
%	cost={80},
%	profile={< 2 3 - 4 4 2 3 3 4},
%	type=\warbeast{},
%	invocation={1D3+1},
%	unitsize=3,
%	maxmodels=9,
%	costpermodel=20,
%	basesize=40x40,
%	commontype=\undeadcommonrules{},
%	commonspecialrules={\undead{},\dusttodust{}},
%	specialrules={\fly{9},\skirmisher{}},
%}
%
%\showunit{
%	name={\sandscorpion{}},
%	cost={85},
%	profile={< 7 4 - 5 5 4 3 4 8},
%	type=\monstrousbeast{},
%	invocation={1},
%	unitsize=1,
%	basesize=50x50,
%	commontype=\undeadcommonrules{},
%	commonspecialrules={\undeadconstruct{}},
%	specialrules={\fear{},\lethalstrike{},\poisonedattacks{},\magicresistance{2},\undergroundambush{}},
%	armour={\innatedefence{5}},
%}
%
%\showunit{
%	name={\sandstalkers{}},
%	QRSname={\sandstalker{}},
%	cost={165},
%	profile={< 7 3 5 4 4 3 3 2 8},
%	type=\monstrousbeast{},
%	invocation={1},
%	unitsize=3,
%	maxmodels=7,
%	costpermodel=50,
%	basesize=50x100,
%	commontype=\undeadcommonrules{},
%	commonspecialrules={\undeadconstruct{}},
%	specialrules={\fear{},\lighttroops{}},
%	weapons={\halberd{}},
%	armour={\innatedefence{5}},
%	options={
%		\undergroundambush{}=20,
%		}
%	},
%	commandgroup={champion=10},
%	unitequipment={\equipmentdef{\petrifyinggaze{}}{\petrifyinggazerule{}}},
%}
%
%\showunit{
%	name={\battlesphinx{}},
%	cost={220},
%	profile={
%		\battlesphinx{}< 6 4 - 5 8 5 1 4 8,
%		\rider{} (4)< - 4 3 4 - - 3 2 8,
%	},
%	type=\riddenmonster{},
%	invocation={1},
%	unitsize=1,
%	basesize=50x100,
%	commontype=\undeadcommonrules{},
%	commonspecialrules={\undeadconstruct{}},
%	specialrules={\lethalstrike{} \only{\rider{}},\poisonedattacks{} \only{\battlesphinx{}}},
%	weapons={\lightlance{} \only{\rider{}}},
%	armour={\innatedefence{5}},
%	options={
%		\innatedefence{4}=25,
%		\breathweaponbattlesphinx{}=25,
%	},
%}








\rareunitstitle

%\showunit{
%	name={\wingedreapers{}},
%	QRSname={\wingedreaper{}},
%	cost={155},
%	profile={< 6 5 3 5 5 4 4 4 10},
%	type=\monstrousinfantry{},
%	invocation={1},
%	unitsize=2,
%	maxmodels=5,
%	costpermodel=72,
%	basesize=50x75,
%	commontype=\undeadcommonrules{},
%	commonspecialrules={\undeadconstruct{}},
%	specialrules={\fear{},\fly{6},\lethalstrike{}},
%	armour={\innatedefence{5}},
%	options={
%		\onechoiceonly{
%			\autonomous{}=\permodel{}<10,
%			\necromanticaura{}=20,
%		},
%		\la{}=\permodel{}<10,
%		\weapononechoice{
%			\pw{}=\permodel{}<5,
%			\halberd{}=\permodel{}12,
%		}
%	},
%	unitrules={\unitrule{\autonomous{}}{\autonomous{}}},
%}
%
%\showunit{
%	name={\dreadsphinx{}},
%	cost={245},
%	profile={< 6 5 - 6 8 5 1 4 8},
%	type=\monster{},
%	invocation={1},
%	unitsize=1,
%	basesize=50x100,
%	commontype=\undeadcommonrules{},
%	commonspecialrules={\undeadconstruct{}},
%	specialrules={\fly{6},\lethalstrike{},\poisonedattacks{},\multiplewounds{2}{\monster{},\riddenmonster{}}},
%	weapons={\pw{}},
%	armour={\innatedefence{5}},
%	options={
%		\innatedefence{4}=25,
%	},
%}
%
%\showunit{
%	name={\colossus{}},
%	cost={195},
%	profile={< 6 4 2 6 6 5 2 5 8},
%	type=\monster{},
%	invocation={1},
%	unitsize=1,
%	basesize=50x50,
%	commontype=\undeadcommonrules{},
%	commonspecialrules={\undeadconstruct{}},
%	specialrules={\grindingattacks{1D3+1}},
%	armour={\la{},\innatedefence{5}},
%	options={
%		\weapononechoice{
%			\scalesofdestiny{}=5,
%			\pw{}=10,
%			\gw{}=20,
%			\giantaspenbow{}=10,
%		},
%	},
%	unitrules={
%		\unitrule{\scalesofdestiny{}}{\scalesofdestinyrule{}}
%		\unitrule{\giantaspenbow{}}{\giantaspenbowrule{}}
%	},
%}
%
%\showunit{
%	name={\casketofphatep{}},
%	cost={115},
%	profile={
%		\casket{}< - - - - 7 3 - - -,
%		\necropolisguard{} (3)< 4 3 3 4 4 - 3 1 8,
%	},
%	type=\warmachine{},
%	invocation={1},
%	unitsize=1,
%	basesize=75,
%	commontype=\undeadcommonrules{},
%	commonspecialrules={\undead{},\dusttodust{}},
%	specialrules={\lethalstrike{} \only{\necropolisguard{}},\poisonedattacks{} \only{\necropolisguard{}},\magicalattacks{},\wardsave{5},\divinelight{},\phatepscurse{}},
%	weapons={\halberd{} \only{\necropolisguard{}}},
%	armour={\la{}},
%	unitrules={
%		\unitrule{\divinelight{}}{\divinelightrule{}}
%		\unitrule{\phatepscurse{}}{\phatepscurserule{}}
%	},
%}
%
%\showunit{
%	name={\charnelcatapult{}},
%	cost={90},
%	profile={
%		\charnelcatapult{}< - - - - 7 3 - - -,
%		\skeleton{} (3)< 4 2 2 3 3 - 2 1 4,
%	},
%	type=\warmachine{},
%	invocation={1},
%	unitsize=1,
%	basesize=75,
%	commontype=\undeadcommonrules{},
%	commonspecialrules={\undead{},\dusttodust{}},
%	unitrules={\unitrule{\cursedammunition{}}{\cursedammunitionrule{}}},
%	unitequipment={\equipmentdef{\charnelcatapult{}}{\charnelcatapultrule{}}},
%	options={
%		\replacecharnelcatapultwithcursedammunition{}=35
%	},
%}



%%% Quick Reference Sheet - AB_qrs.tex is automatic and shouldn't be edited %%%

\quickrefsheettitle

\input{../Layout/AB_qrs.tex}
\bigskip

\begin{center}
\noindent{\antiquefont\Largefontsize\textbf{Armes de Tir des Dynasties Immortelles}}
\medskip

\rowcolors{1}{white}{black!10}
\noindent\begin{tabular}{lcccccc}
\textbf{Nom} & \textbf{Artillerie} & \textbf{Portée} & \textbf{\labels@S{}} & \textbf{\multipleshots{}} & \textbf{\multiplewounds{}} & \textbf{\armourpiercing{}} \tabularnewline
\aspenbow{} & - & \distance{24} & 3 & - & - & - \tabularnewline
\greataspenbow{} & - & \distance{36} & 5 & - & - & - \tabularnewline
\giantaspenbow{} & \boltthrower{} & \distance{48} & 6 & - & 1D3 & 6 \tabularnewline
\petrifyinggaze{} & - & \distance{12} & 2 & 1D6+1 & - & 6 \tabularnewline
\charnelcatapult{} & \catapult{} (\distance{3}) & \distance{12-60} & 3 [9] & - & [\ordnance{}] & - \tabularnewline
\cursedammunition{} & \catapult{} (\distance{5}) & \distance{12-48} & 3 & - & - & - \tabularnewline
\end{tabular}
\end{center}

\medskip
\begin{center}\noindent{\antiquefont\Largefontsize\textbf{\labels@Invocation}}\end{center}

\begin{multicols}{3}\raggedcolumns
\rowcolors{1}{white}{black!10}
\noindent\begin{tabular}{p{4cm}>{\centering\let\newline\\\arraybackslash\hspace{0pt}}p{1cm}@{}}%
\textbf{\allcharacters} & 1 \tabularnewline
\textbf{\allundeadconstructs} & 1 \tabularnewline
\textbf{\allwarmachines} & 1 \tabularnewline
\end{tabular}
\vspace*{\fill}\columnbreak

\rowcolors{1}{white}{black!10}
\noindent\begin{tabular}{p{4cm}>{\centering\let\newline\\\arraybackslash\hspace{0pt}}p{1cm}@{}}%
\skeletons{} & 1D3+3 \tabularnewline
\skeletonarchers{} & 1D3+3 \tabularnewline
\scarabswarms{} & 1D3+3 \tabularnewline
\skeletoncavalry{} & 1D3+2 \tabularnewline
\end{tabular}
\vspace*{\fill}\columnbreak

\rowcolors{1}{white}{black!10}
\noindent\begin{tabular}{p{4cm}>{\centering\let\newline\\\arraybackslash\hspace{0pt}}p{1cm}@{}}%
\necropolisguard{} & 1D3+1 \tabularnewline
\skeletonchariots{} & 1D3+1 \tabularnewline
\greatvultures{} & 1D3+1 \tabularnewline
\end{tabular}
\vspace*{\fill}\end{multicols}

\restoregeometry

\end{document}

