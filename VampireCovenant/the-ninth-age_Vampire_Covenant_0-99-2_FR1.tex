
\documentclass[a4paper,8pt]{extarticle} % extarticle allows to use font size of 8pt.

\usepackage[a4paper, top=1.6cm, bottom=2cm, left=1.6cm, right=1.6cm]{geometry} % Marge reduction.

%% Language specific package
\usepackage[french]{babel}
\frenchbsetup{StandardLists=true} % Necessary to use enumitem with babel/french.

%% Font and typing packages
\usepackage{fontspec}
\setmainfont[
	Ligatures=TeX,
	ItalicFont={Dancing Script},
	BoldItalicFont={Dancing Script}
	]{PT Serif} % default is Latin Modern
\newfontfamily\antiquefont[Ligatures=TeX]{Caslon Antique} % fancy font
\usepackage{microtype}			% Greatly improves general appearance of the text.
\usepackage{SIunits}			% Unit appearance.
\usepackage{xspace}				% Define commands that appear not to eat spaces.
\usepackage{ulem}				% To cross words out. Use \sout{}.

%% Array utilities
\usepackage{array}				% Additionnal options for arrays.
\usepackage{colortbl}			% Additionnal options for coloring arrays.
\usepackage[table]{xcolor}		% Auto alternate grey-white rows.
\usepackage[export]{adjustbox}		% Centered pics in tables

%% List utilities
\usepackage[inline]{enumitem}   % Display inline lists.
\usepackage{etoolbox}           % General utility. Good for lists for instance.
\usepackage{xparse}             % List utilities.
\usepackage{datatool}	% Handling alphabetical order.

%% Frames
\usepackage{framed}				% Boxes.
\usepackage[framemethod=TikZ]{mdframed}% For fancy frames.
\usepackage{tikz}				% For fancy frames.
\usepackage{wrapfig}			% Fancy insertion of pics in text.

%% Page utilities
\usepackage{multicol}			% Allows to divide a part of the page in multiple columns.
	
%% Others
\usepackage{keyval}             % Used to create maps of commands/labels/objects.
	\makeatletter                  % Mandatory for the usage of keyval.
\usepackage{xstring}            % String parsing, cutting, etc.
\usepackage{hyperref} % Links in PDF.


%%% Update of the dotfill command to always get dots

\newcommand{\predotfill}{\penalty0\hbox{}\nobreak}%


%%% Command to avoid typing \xspace when creating a new name macro

\newcommand{\newnamemacro}[2]{\newcommand{#1}{#2}} % \xspace removed for compatibility with alphabetical ordering

%%% Language specific stuff


%%% Commands %%%

\newcommand{\addtosortedlist}[1]{%
	\protected@edef\textarg{#1}%
	\protected@edef\textwithoutspaces{\expandafter\removespaces\expandafter{\textarg}}%
	\substitute\textwithoutspaces{É}{e}% Most used special characters of the language, and equivalent for alphabetical ordering
	\substitute\textwithoutspaces{È}{e}%
	\substitute\textwithoutspaces{Ê}{e}%
	\substitute\textwithoutspaces{é}{e}%
	\substitute\textwithoutspaces{è}{e}%
	\substitute\textwithoutspaces{ê}{e}%
	\substitute\textwithoutspaces{À}{a}%
	\substitute\textwithoutspaces{à}{a}%
	\substitute\textwithoutspaces{ù}{u}%
	\expandafter\sortitem\expandafter[\textwithoutspaces]{#1}%
}%


%%% Labels %%%

% Profile

\newcommand{\labels@M}{M}
\newcommand{\labels@WS}{CC}
\newcommand{\labels@BS}{CT}
\newcommand{\labels@S}{F}
\newcommand{\labels@T}{E}
\newcommand{\labels@W}{PV}
\newcommand{\labels@I}{I}
\newcommand{\labels@A}{A}
\newcommand{\labels@Ld}{Cd}
\newcommand{\labels@Invocation}{Invocation} % For Vampire Covenant profiles

\newcommand{\Strength}{Force}

% Technical

\newcommand{\labels@range}{Portée}
\newcommand{\labels@point}{pt}
\newcommand{\labels@points}{pts}
\newcommand{\labels@only}{uniquement}
\newcommand{\labels@magic}{Magie}
\newcommand{\labels@pathsused}{Génère ses sorts dans la Discipline}
\newcommand{\labels@model}{figurine}
\newcommand{\labels@models}{figurines}
\newcommand{\labels@Singlemodel}{Figurine \textbf{seule}}

% Unit entry labels

\newcommand{\labels@basesize}{Socle}
\newcommand{\labels@trooptype}{Type de troupe}
\newcommand{\labels@specialrules}{Règles spéciales}
\newcommand{\labels@alignment}{Allégeance}
\newcommand{\labels@equipment}{Équipement}
\newcommand{\labels@weapons}{Armes}
\newcommand{\labels@armour}{Armure}
\newcommand{\labels@options}{Options}
\newcommand{\labels@commandgroup}{État-Major}
\newcommand{\labels@mounts}{Montures}
\newcommand{\labels@specialequipment}{Équipement spécial}

% Command groups

\newcommand{\labels@champion}{Champion}
\newcommand{\labels@standardbearer}{Porte-étendard}
\newcommand{\labels@musician}{Musicien}
\newcommand{\labels@singlebannerallowance}{Une seule unité de ce type peut prendre une Bannière magique}
\newcommand{\labels@condsinglebannerallowance}{Une seule unité de ce type peut prendre une Bannière magique si}
\newcommand{\labels@bannerallowance}{Peut prendre une Bannière Magique}
\newcommand{\labels@veteranstandardbearer}{Peut devenir Porte-étendard Vétéran}
\newcommand{\labels@championallowance}{Peut prendre une Arme Magique}

% Titles

\newcommand{\labels@lords}{Seigneurs}
\newcommand{\labels@heroes}{Héros}
\newcommand{\labels@coreunits}{Unités de base}
\newcommand{\labels@specialunits}{Unités spéciales}
\newcommand{\labels@rareunits}{Unités rares}
\newcommand{\labels@armywiderules}{Règles communes de l'armée}
\newcommand{\labels@armyspecialrules}{Règles spéciales de l'armée}
\newcommand{\labels@armoury}{Armurerie}
\newcommand{\labels@magicalitems}{Objets magiques}
\newcommand{\labels@magicalweapons}{Armes magiques}
\newcommand{\labels@magicalarmour}{Armures magiques}
\newcommand{\labels@talismans}{Talismans}
\newcommand{\labels@enchanteditems}{Objets enchantés}
\newcommand{\labels@arcaneitems}{Objets cabalistiques}
\newcommand{\labels@magicalbanners}{Bannières magiques}
\newcommand{\labels@quickrefsheet}{Fiche de référence}
\newcommand{\labels@changelog}{Change Log}

\newcommand{\labels@lordsInitial}{S}
\newcommand{\labels@heroesInitial}{H}
\newcommand{\labels@coreunitsInitial}{B}
\newcommand{\labels@specialunitsInitial}{S}
\newcommand{\labels@rareunitsInitial}{R}
\newcommand{\labels@mountsInitial}{M}


% Titlepage

\newcommand{\labels@fantasybattles}{Batailles Fantastiques}
\newcommand{\labels@NinthAge}{Le 9\ieme Âge}
\newcommand{\labels@creators}{Une collaboration des créateurs de l'ETC et du Swedish Comp System}
\newcommand{\labels@introduction}{%
\noindent {\Largerfontsize\textbf{Note des traducteurs}}
\vspace{0.5cm}

Nous souhaitons remercier chaleureusement l'équipe à l'initiative du 9\ieme Âge pour leur motivation et leur travail continu pour faire vivre notre passion. Nous espérons que ce jeu saura développer les qualités pour plaire au plus grand nombre et réunir les joueurs, amateurs comme habitués des tournois, autour de règles amusantes et équilibrées, pour finalement s'imposer comme un standard du jeu de figurines. Une grande ambition qui ne pourra s'accomplir que \textbf{grâce à vous}, la communauté, via des retours constructifs, afin de modeler le jeu selon nos désirs. N'étant \textbf{en aucun cas à but lucratif}, le 9\ieme Âge part avec un avantage considérable. Les règles des éventuelles nouvelles sorties ne seront pas dictées par le besoin de vendre ces nouveautés. Vous pouvez choisir et acheter vos figurines où bon vous semble, il n'y a pas un unique revendeur toléré. Vous n'êtes pas bloqués dans une spirale infernale où pour continuer à jouer à un jeu, dans lequel vous vous êtes tant investis, vous devez payer toujours plus cher pour entretenir votre collection. Enfin, vous pouvez être assurés que tant que 9\ieme Âge sera joué, vous disposerez d'un \textbf{support continu et régulier}, celui-ci étant offert par la communauté.

Nous attirons votre attention sur le fait que ce jeu en est encore à ses débuts et dans un \textbf{stade de développement}. Ce document correspond à une version de brouillon \textbf{\og{} beta \fg{}}, dont le but et de tester le jeu et le modifier jusqu'à atteindre une version satisfaisante. Attendez-vous donc à trouver des déséquilibres, des incohérences, et à obtenir des mises à jour régulières avec éventuellement des changements importants. N'hésitez pas à nous donner vos avis ! Ce livre d'armée n'est utilisable qu'en compagnie du livre de Règles et du livre de Magie.

Concernant la traduction en elle-même, nous avons fait de notre mieux pour vous offrir une version de qualité, dont nous espérons qu'elle surpasse celle de la version originale ! Si vous constatez des coquilles, des erreurs, merci de nous les signaler en nous contactant sur le forum du 9\ieme Âge, dans le \textbf{sous-forum français} (\url{http://www.the-ninth-age.com/index.php?board/117-french/}). Vous y trouverez aussi les dernières mises à jour. \textbf{En cas de conflit d'interprétation avec la version originale, la version originale fait référence}.

\vspace{0.5cm}
Que ce jeu vous apporte d'innombrables heures de plaisir partagé !

\vspace{0.7cm}
\noindent {\Largerfontsize\textbf{Les traducteurs}}
\vspace{0.1cm}

\ifdef{\translationteam}{
	\begin{multicols}{3}
	\begin{itemize}
		\translationteam
	\end{itemize}
	\end{multicols}
}{}
}
\newcommand{\labels@secondpageannouncement}{%
	\labels@fantasybattles{} : \labels@NinthAge{} est un jeu créé et entretenu par la communauté qui met en scène des affrontements de figurines. Toutes les règles sont disponibles gratuitement sur le site suivant. Vos retours et suggestions sont les bienvenus.
	\newline\url{http://www.the-ninth-age.com/}
}
\newcommand{\labels@rulechanges}{%
	Les changements de règles entre versions sont colorés comme ce paragraphe. Une liste en anglais de ces changements par version est ajoutée à la fin de cet ouvrage.
}
\newcommand{\labels@latexcredit}{Document réalisé à l'aide de \LaTeX .}


%%% Technical commands

\newcommand{\only}[1]{(#1 uniquement)}
\newcommand{\free}{gratuit}
\newcommand{\upto}{jusqu'à}
\newcommand{\Upto}{Jusqu'à}
\newcommand{\unlimited}{sans limite de pts}
\newcommand{\permodel}{/fig.}
\newcommand{\listlastchoice}{ ou}
\newcommand{\notif}[1]{(pas #1)}
\newcommand{\wordand}{et}
\newcommand{\wordwith}{avec}
\newcommand{\ifNmodelsorless}[1]{(#1 figurines ou moins)}
\newcommand{\unitwith}{unité avec}
\newcommand{\From}{De} % From ... to ... models
\newcommand{\wordto}{à}
\newcommand{\wordAll}{Tous}
\newcommand{\spacebeforecolon}{ } % French put a space before colons
\newcommand{\minprice}{Coût min. :}
\newcommand{\mincostfor}{Coût min. pour}
\newcommand{\maxunitsize}{Taille max.}
\newcommand{\additionalfigscost}{Les figurines additionnelles coûtent}


%%% Special rules %%%

\newcommand{\ambush}{Embuscade}
\newcommand{\armourpiercing}[1]{Perforant\ifblank{#1}{}{ (#1)}}
\newcommand{\bodyguard}[1]{Garde du Corps\ifblank{#1}{}{ (#1)}}
\newcommand{\breathweapon}[1]{Attaque de Souffle\ifblank{#1}{}{ (#1)}}
\newcommand{\channel}{Canalisation}
\newcommand{\crushattack}{Attaque Écrasante}
\newcommand{\devastatingcharge}{Charge Dévastatrice}
\newcommand{\distracting}{Distrayant}
\newcommand{\engineer}{Ingénieur}
\newcommand{\ethereal}{Éthéré}
\newcommand{\fastcavalry}{Cavalerie Légère}
\newcommand{\fear}{Peur}
\newcommand{\fightinextrarank}{Combat avec un Rang Supplémentaire}
\newcommand{\fireborn}{Né du Feu}
\newcommand{\flamingattacks}{Attaques Enflammées}
\newcommand{\flammable}{Inflammable}
\newcommand{\lighttroops}{Troupes Légères}
\newcommand{\frenzy}{Frénésie}
\newcommand{\fly}[1]{Vol\ifblank{#1}{}{ (#1)}}
\newcommand{\grindingattacks}[1]{Attaques de Broyage\ifblank{#1}{}{ (#1)}}
\newcommand{\hardtarget}{Camouflé}
\newcommand{\hatred}{Haine}
\newcommand{\hellfire}{Flammes de l'Enfer}
\newcommand{\hidden}{Caché}
\newcommand{\holyattacks}{Attaques Divines}
\newcommand{\immunetopsychology}{Immunisé à la Psychologie}
\newcommand{\impacthits}[1]{Touches d'Impact\ifblank{#1}{}{ (#1)}}
\newcommand{\insignificant}{Insignifiant}
\newcommand{\largetarget}{Grande Cible}
\newcommand{\lethalstrike}{Coup Fatal}
\newcommand{\lightningattacks}{Attaques Foudroyantes}
\newcommand{\lightningreflexes}{Réflexes Foudroyants}
\newcommand{\magicresistance}[1]{Résistance à la Magie\ifblank{#1}{}{ (#1)}}
\newcommand{\magicalattacks}{Attaques Magiques}
\newcommand{\metalshifting}{Fusion du Métal}
\newcommand{\moveorfire}{Mouvement ou Tir}
\newcommand{\multipleshots}[1]{Tirs Multiples\ifblank{#1}{}{ (#1)}}
\newcommand{\multiplewounds}[2]{Blessures Multiples\ifblank{#1}{}{ (#1\ifblank{#2}{)}{, #2)}}}
\newcommand{\notaleader}{Pas un Meneur}
\newcommand{\otherworldly}{D'Outre-Monde}
\newcommand{\pathmaster}[1]{Maître de la Discipline\ifblank{#1}{}{ (#1)}}
\newcommand{\poisonedattacks}{Attaques Empoisonnées}
\newcommand{\quicktofire}{Tir Rapide}
\newcommand{\randommovement}[1]{Mouvement Aléatoire\ifblank{#1}{}{ (#1)}}
\newcommand{\randomattacks}[1]{Attaques Aléatoires\ifblank{#1}{}{ (#1)}}
\newcommand{\regeneration}[1]{Régénération\ifblank{#1}{}{ (#1+)}}
\newcommand{\reload}{Rechargez !}
\newcommand{\requirestwohands}{Arme à deux Mains}
\newcommand{\scythes}{Faux}
\newcommand{\scout}{Éclaireur}
\newcommand{\scouts}{Éclaireurs}
\newcommand{\stomp}[1]{Piétinement\ifblank{#1}{}{ (#1)}}
\newcommand{\strider}[1]{Guide\ifblank{#1}{}{ (#1)}}
\newcommand{\stubborn}{Tenace}
\newcommand{\stupidity}{Stupidité}
\newcommand{\skirmisher}{Tirailleur}
\newcommand{\skirmishers}{Tirailleurs}
\newcommand{\sweepingattack}{Attaque au Passage}
\newcommand{\swiftstride}{Rapide}
\newcommand{\thunderouscharge}{Charge Tonitruante}
\newcommand{\terror}{Terreur}
\newcommand{\toxicattacks}{Attaques Toxiques}
\newcommand{\unbreakable}{Indémoralisable}
\newcommand{\undead}{Mort-Vivant}
\newcommand{\unstable}{Instable}
\newcommand{\unwieldy}{Encombrant}
\newcommand{\vanguard}{Avant-Garde}
\newcommand{\volleyfire}{Tir de Volée}
\newcommand{\warplatform}{Plateforme de Guerre}
\newcommand{\wardsave}[1]{Sauvegarde Invulnérable\ifblank{#1}{}{ (#1+)}}
\newcommand{\weaponmaster}{Maître d'Ar\-mes}
\newcommand{\wizardconclave}[1]{Conclave de Sorciers\ifblank{#1}{}{ (#1)}}


%%% Magic %%%

\newnamemacro{\Pathof}{Discipline}

\newcommand{\battle}{Commune}
\newcommand{\alchemy}{de l'Alchimie}
\newcommand{\death}{de la Mort}
\newcommand{\fire}{du Feu}
\newcommand{\heavens}{des Cieux}
\newcommand{\light}{de la Lumière}
\newcommand{\nature}{de la Nature}
\newcommand{\shadows}{des Ombres}
\newcommand{\wilderness}{de la Sauvagerie Bestiale}
\newcommand{\butchery}{de la Boucherie}
\newcommand{\change}{du Changement}
\newcommand{\thebiggreengods}{des Grands Dieux Verts}
\newcommand{\thelittlegreengods}{des Petits Dieux Verts}
\newcommand{\blackmagic}{de la Magie Noire}
\newcommand{\disease}{de la Maladie}
\newcommand{\lust}{de la Luxure}
\newcommand{\necromancy}{de la Nécromancie}
\newcommand{\ruin}{de la Ruine}
\newcommand{\forge}{de la Forge}
\newcommand{\sands}{des Sables}
\newcommand{\whitemagic}{de la Magie Blanche}

\newcommand{\anyofthebattlemagic}{dans n'importe laquelle des Disciplines Communes}

\newcommand{\magiclevel}[1]{\ifnumcomp{#1}{<}{3}{Sorcier Apprenti}{Maître Sorcier} Niveau #1}
\newcommand{\Level}{Niveau}

\newcommand{\wizard}{Sorcier}
\newcommand{\wizards}{Sorciers}

\newcommand{\boundspell}[1]{Objet de Sort, Puissance #1}


%%% Other rules %%%

\newcommand{\armoursave}{Sauvegarde d'Armure}
\newcommand{\firstinrank}{Au Premier Rang}
\newcommand{\hardcover}{Couvert Lourd}
\newcommand{\holdyourground}{Tenez les Rangs}
\newcommand{\inspiringpresence}{Présence Charismatique}
\newcommand{\lightcover}{Couvert Léger}
\newcommand{\monstrousrank}{Rang Monstrueux}
\newcommand{\ordnance}{Artillerie}
\newcommand{\parry}{Parade}
\newcommand{\raisewounds}{Ressusciter des Figurines}
\newcommand{\recoverwounds}{Récupérer des PVs}
\newcommand{\aideddispel}{Dissipation Assistée}
\newcommand{\rnf}{ordinaires}
\newcommand{\general}{Général}


%%% Equipment %%%

\newcommand{\innatedefence}[1]{Protection Innée\ifblank{#1}{}{~(#1+)}}
\newcommand{\mountsprotection}[1]{Protection de Monture\ifblank{#1}{}{~(#1+)}}
\newcommand{\la}{Armure Légère}
\newcommand{\ha}{Armure Lourde}
\newcommand{\platearmour}{Armure de Plates}
\newcommand{\hw}{Arme de Base}
\newcommand{\pw}{Paire d'Armes}
\newcommand{\spear}{Lance}
\newcommand{\halberd}{Hallebarde}
\newcommand{\gw}{Arme Lourde}
\newcommand{\lance}{Lance de Cavalerie}
\newcommand{\lightlance}{Lance Légère}
\newcommand{\shield}{Bouclier}
\newcommand{\barding}{Caparaçon}
\newcommand{\throwingweapons}{Armes de Jet}
\newcommand{\shortbow}{Arc Court}
\newcommand{\flail}{Fléau}

\newcommand{\cannon}{Canon}
\newcommand{\catapult}{Catapulte}
\newcommand{\volleygun}{Batterie de Tir}
\newcommand{\boltthrower}{Baliste}
\newcommand{\artilleryweapon}{Arme d'Artillerie}


%%% Troop types %%%

\newcommand{\characters}{Personnages}
\newcommand{\infantry}{Infanterie}
\newcommand{\monstrousinfantry}{Infanterie Monstrueuse}
\newcommand{\cavalry}{Cavalerie}
\newcommand{\monstrouscavalry}{Cavalerie Monstrueuse}
\newcommand{\swarm}{Nuée}
\newcommand{\swarms}{Nuées}
\newcommand{\warbeast}{Bête de Guerre}
\newcommand{\warbeasts}{Bêtes de Guerre}
\newcommand{\monster}{Monstre}
\newcommand{\monsters}{Monstres}
\newcommand{\monstrousbeast}{Bête Monstrueuse}
\newcommand{\monstrousbeasts}{Bêtes Monstrueuses}
\newcommand{\chariot}{Char}
\newcommand{\chariots}{Chars}
\newcommand{\riddenmonster}{Monstre Monté}
\newcommand{\riddenmonsters}{Monstres Montés}
\newcommand{\warmachine}{Machine de Guerre}
\newcommand{\warmachines}{Machines de Guerre}


%%% Terrain %%%

\newcommand{\water}{Eaux peu profondes}


%%% Profile wording

\newcommand{\oneofakind}{Uni\-que}
\newcommand{\onechoiceonly}{(un seul choix)}
\newcommand{\onfootonly}{(à pied seulement)}
\newcommand{\closecombatonly}{seulement au Corps à Corps}
\newcommand{\Xmodelsorless}[1]{(#1 figurines ou moins)}
\newcommand{\magicalitemsallowance}{Peut prendre des Objets Magiques}
\newcommand{\magicalweaponallowance}{Peut prendre une Arme Magique}
\newcommand{\notmagicalarmour}{(mais pas d'Armure Magique)}
\newcommand{\anyofthefollowing}{\optionschoice{Peut prendre :}}
\newcommand{\weapononechoice}{\optionschoice{Peut prendre une arme \onechoiceonly{} :}}
\newcommand{\weaponschoice}{\optionschoice{Peut prendre des armes :}}
\newcommand{\shootingweapononechoice}{\optionschoice{Peut prendre une arme de tir \onechoiceonly{} :}}
\newcommand{\combatweapononechoice}{\optionschoice{Peut prendre une arme de corps à corps \onechoiceonly{} :}}
\newcommand{\armouronechoice}{\optionschoice{Peut prendre une armure \onechoiceonly{} :}}
\newcommand{\magiclevelchoice}{\optionschoice{Peut devenir au choix :}}
\newcommand{\bsboption}{Peut devenir Porteur de la Grande Bannière}
\newcommand{\mayupgradeto}{Peut être amélioré en}
\newcommand{\mustbecomeoneofthefollowing}{\optionschoice{Doit devenir un choix parmi :}}
\newcommand{\maybecomeoneofthefollowing}{\optionschoice{Peut devenir un choix parmi :}}
\newcommand{\maytakeoneofthefollowing}{\optionschoice{Peut prendre un choix parmi :}}
\newcommand{\maytakeuptotwoofthefollowing}{\optionschoice{Peut prendre jusqu'à deux choix parmi :}}
\newcommand{\maygain}{Peut gagner la règle}
\newcommand{\maytake}{Peut prendre}
\newcommand{\maytakeashield}{Peut prendre un Bouclier}
\newcommand{\maytakela}{Peut prendre une Armure Légère}
\newcommand{\maytakeha}{Peut prendre une Armure Lourde}
\newcommand{\maytakemountsprotectionX}[1]{Peut prendre une \mountsprotection{#1}}
\newcommand{\maytakeagw}{Peut prendre une Arme Lourde}
\newcommand{\maytakeaspear}{Peut prendre une Lance}
\newcommand{\maytakepw}{Peut prendre une Paire d'Armes}
\newcommand{\maytakethrowingweapons}{Peut prendre des Armes de Jet}
\newcommand{\maytakebarding}{Peut prendre un Caparaçon}
\newcommand{\replaceshieldwithhalberd}{Remplacer le Bouclier par une Hallebarde}
\newcommand{\maybecome}{Peut devenir}

\newcommand{\maytakeonechoiceonly}{\optionschoice{\maytake{} \onechoiceonly{}\spacebeforecolon{}:}}

\newcommand{\mountssectionannouncement}{%
La section Montures concerne les montures de Personnages. Les montures pour non-Personnages suivent les règles données dans leur description d'unité.
}

%%% Commands to handle strings, better than xstring to handle commands inside the strings %%%

\newcommand{\substitute}[3]{%
  \protected@edef\sub@temp{#1}%
  \saveexpandmode
  \expandarg\StrSubstitute{\sub@temp}{#2}{#3}[#1]%
  \restoreexpandmode
}

\newcommand{\splitatstar}[3]{%
  \protected@edef\split@temp{#1}%
  \saveexpandmode
  \expandarg\StrCut{\split@temp}{*}#2#3%
  \restoreexpandmode
}

\newcommand{\splitatinf}[3]{%
  \protected@edef\split@temp{#1}%
  \saveexpandmode
  \expandarg\StrCut{\split@temp}{<}#2#3%
  \restoreexpandmode
}

\newcommand{\splitatequal}[3]{%
  \protected@edef\split@temp{#1}%
  \saveexpandmode
  \expandarg\StrCut{\split@temp}{=}#2#3%
  \restoreexpandmode
}

\newcommand{\ifsubstring}[4]{%
  \protected@edef\split@temp{#1}%
  \protected@edef\split@tempbis{#2}%
  \saveexpandmode
  \expandarg\IfSubStr{\split@temp}{\split@tempbis}{#3}{#4}%
  \restoreexpandmode
}

\def\removespaces#1{\zap@space#1 \@empty}

%%% Commands for alphabetical ordering %%%

\newcommand{\sortitem}[2][\relax]{%
	\DTLnewrow{list}% Create a new entry
	\ifx#1\relax%
		\DTLnewdbentry{list}{sortlabel}{#2}% Add entry sortlabel (no optional argument)
	\else%
		\DTLnewdbentry{list}{sortlabel}{#1}% Add entry sortlabel (optional argument)
	\fi%
		\DTLnewdbentry{list}{description}{#2}% Add entry description
}
\newenvironment{sortedlist}{%
	\DTLifdbexists{list}{\DTLcleardb{list}}{\DTLnewdb{list}}% Create new/discard old list
}{%
	\DTLsort{sortlabel}{list}% Sort list
	\begin{itemize*}[label={}, itemjoin={,}]%
		\DTLforeach*{list}{\theDesc=description}{%
		\item\theDesc}% Print each item
	\end{itemize*}%
}

\pdfstringdefDisableCommands{\def\textcolor#1{}}

% See language specific file for \addtosortedlist

%%% Database for automatic Quick Ref Sheet %%%

\DTLnewdb{profiles} % Database containing name, category, multiprofile number, profilename (if multi), caraclist, trooptype, invocation for CV.
\newcommand{\profilecategory}{\labels@lords} % Will be updated in relevant categories

\newcommand{\profiledtbfillname}[1]{\DTLnewdbentry{profiles}{name}{#1}}
\newcommand{\profiledtbfillcategory}[1]{\DTLnewdbentry{profiles}{category}{#1}}
\newcommand{\profiledtbfilltrooptype}[1]{\DTLnewdbentry{profiles}{trooptype}{#1}}
\newcommand{\profiledtbfillinvocation}[1]{\DTLnewdbentry{profiles}{invocation}{#1}}
\newcommand{\profiledtbfillprofile}[1]{\DTLnewdbentry{profiles}{profile}{#1}}
\newcommand{\profiledtbfillmultipleprofile}[1]{\DTLnewdbentry{profiles}{multipleprofile}{#1}}

\newcommand{\void}[1]{}
\newcounter{multiprofilecounter}

\newcommand{\profiledtbfillcarac}[1]{%
	\profiledtbfillprofile{#1}
	\parselist{#1}{\locallists@profileslist}% Split of the different profiles in the case of a multiprofile.
	\setcounter{multiprofilecounter}{0}%
	\forlistloop{\stepcounter{multiprofilecounter}\void}{\locallists@profileslist}%
	\expandafter\profiledtbfillmultipleprofile\expandafter{\number\value{multiprofilecounter}}
}


%%% Technical commands %%%

\newcommand{\newrule}{\textcolor{green!50!black}}
\newcommand{\removedrule}[1]{\textcolor{green!50!black}{\sout{#1}}}
\newcommand{\starsymbol}{$\star$}
\newcommand{\refsymbol}{$^\star$}

\newcommand{\inch}{\arcsecond}
\newcommand{\foot}{\arcminute}
\newcommand{\range}[1] {\labels@range~\unit{#1}{\inch}}
\newcommand{\distance}[1] {\unit{#1}{\inch}}
\newcommand{\result}[1] {\texttt{'}#1\texttt{'}}


%%% Fonts and sizes %%%

\newcommand{\bigtitle}[1]{\vspace*{-1.5cm}\section*{}\noindent\begin{center}\Hugefontsize\textbf{\antiquefont\expandafter\uppercase\expandafter{#1}}\end{center}}

\newcommand{\subtitle}[1]{\subsection*{}\noindent{\hugefontsize\antiquefont #1}}

\newcommand{\subsubtitle}[1]{\subsubsection*{}\noindent{\Largerfontsize\antiquefont #1}}

\newcommand{\verysmallfontsize}{\fontsize{4}{4.8}\selectfont}
\newcommand{\smallfontsize}{\fontsize{6}{7.2}\selectfont}
\newcommand{\normalfontsize}{\fontsize{8}{9.6}\selectfont}
\newcommand{\largefontsize}{\fontsize{10}{12}\selectfont}
\newcommand{\largerfontsize}{\fontsize{12}{14.4}\selectfont}
\newcommand{\Largefontsize}{\fontsize{14}{16.8}\selectfont}
\newcommand{\Largerfontsize}{\fontsize{15}{18}\selectfont}
\newcommand{\hugefontsize}{\fontsize{18}{21.6}\selectfont}
\newcommand{\Hugefontsize}{\fontsize{25}{30}\selectfont}

\newcommand{\unitentryformat}[1]{\textit{\largefontsize{#1}}}
\newcommand{\textIT}[1]{\textit{\largefontsize{#1}}}


%%% Titles %%%

\newcommand{\lordstitle}{\def\logolocalpath{../Layout/pics/logo_lord.png}\bigtitle{\labels@lords}}
\newcommand{\heroestitle}{%
\def\logolocalpath{../Layout/pics/logo_hero.png}%
\clearpage\bigtitle{\labels@heroes}%
\renewcommand{\profilecategory}{\labels@heroes}%
}
\newcommand{\coreunitstitle}{%
\def\logolocalpath{../Layout/pics/logo_core.png}%
\clearpage\bigtitle{\labels@coreunits}%
\renewcommand{\profilecategory}{\labels@coreunits}%
}
\newcommand{\specialunitstitle}{%
\def\logolocalpath{../Layout/pics/logo_special.png}%
\clearpage\bigtitle{\labels@specialunits}%
\renewcommand{\profilecategory}{\labels@specialunits}%
}
\newcommand{\rareunitstitle}{%
\def\logolocalpath{../Layout/pics/logo_rare.png}%
\clearpage\bigtitle{\labels@rareunits}%
\renewcommand{\profilecategory}{\labels@rareunits}%
}
\newcommand{\mountstitle}{%
\def\logolocalpath{../Layout/pics/logo_mount.png}%
\clearpage\bigtitle{\labels@charactermounts}%
\renewcommand{\profilecategory}{\labels@mounts}%
}

\newcommand{\startarmywiderules}{\newpage\bigtitle{\labels@armywiderules}\largefontsize}
\newcommand{\closearmywiderules}{\normalfontsize}
\newcommand{\armywideruleentry}[1]{\subtitle{#1}\vspace{5pt}}

\newcommand{\startarmyspecialrules}{\bigtitle{\labels@armyspecialrules}\largefontsize}
\newcommand{\closearmyspecialrules}{\normalfontsize}
\newcommand{\armyspecialruleentry}[1]{\subtitle{#1}\vspace{5pt}}

\newcommand{\startarmyarmoury}{\bigtitle{\labels@armoury}\largefontsize\subtitle{}}
\newcommand{\closearmyarmoury}{\normalfontsize}

\newcommand{\startarmymagicalitems}{\newpage\largefontsize\bigtitle{\labels@magicalitems}\begin{multicols}{2}\raggedcolumns}
\newcommand{\closearmymagicalitems}{\end{multicols}\normalfontsize}

\newcommand{\armymagicalweapons}{\subtitle{\labels@magicalweapons}}
\newcommand{\armymagicalarmour}{\subtitle{\labels@magicalarmour}}
\newcommand{\armytalismans}{\subtitle{\labels@talismans}}
\newcommand{\armyenchanteditems}{\subtitle{\labels@enchanteditems}}
\newcommand{\armyarcaneitems}{\subtitle{\labels@arcaneitems}}
\newcommand{\armymagicalbanners}{\subtitle{\labels@magicalbanners}}

\newcommand{\startarmynewsection}[1]{\newpage\bigtitle{#1}\largefontsize}
\newcommand{\startarmynewsectionSP}[1]{\vspace{1.5cm}\bigtitle{#1}\largefontsize}
\newcommand{\closearmynewsection}{\normalfontsize}

\newcommand{\armynewsubsection}[1]{\subtitle{#1}\vspace{5pt}}
\newcommand{\armynewsubsubsection}[1]{\subsubtitle{#1}\vspace{3pt}}

\newcommand{\armylist}{\clearpage}

\newcommand{\quickrefsheettitle}{\clearpage\newgeometry{top=1.6cm, bottom=2cm, left=1cm, right=1cm}\bigtitle{\labels@quickrefsheet}\vspace*{0.4cm}}
\newcommand{\changelogtitle}{\clearpage\bigtitle{\labels@changelog}\spaceaftersection{}}

\newcommand{\spaceaftersection}{\vspace{0.8cm}}

\newcommand{\separator}{\noindent\begin{center}\textcolor{black!30}{\rule{0.7\columnwidth}{2pt}}\end{center}}


%%% Custom lists and description for first sections of the army books

\newcommand{\startpricelist}{\begin{samepage}\begin{description}[leftmargin=0.3cm, labelindent=0cm, labelsep=0.1cm]}
\def\endpricelist{\end{description}\end{samepage}}
\newcommand{\pricelistitem}[2]{\item \option{\textbf{#1}}{#2}\newline}

\newcommand{\startpricelistNSP}{\begin{description}[leftmargin=0.3cm, labelindent=0cm, labelsep=0.1cm]}
\def\endpricelistNSP{\end{description}}

\newcommand{\startitemlist}{\begin{multicols}{2}\raggedcolumns\begin{description}[leftmargin=0.3cm, labelindent=0cm, labelsep=0.1cm]}
\def\enditemlist{\end{description}\end{multicols}}
\newcommand{\listitem}[1]{\item[#1\spacebeforecolon{}:]}

\newcommand{\startitemlistonecol}{\begin{description}[leftmargin=0.3cm, labelindent=0cm, labelsep=0.1cm]}
\def\enditemlistonecol{\end{description}}
\newcommand{\listitemonecol}[1]{\item \textbf{#1\spacebeforecolon{}:}\newline}

\newenvironment{customitemize}{\begin{description}[leftmargin=0.3cm, labelindent=0cm, labelsep=0cm]}{\end{description}}
\newenvironment{customsubitemize}{\begin{itemize}[label={-}, labelsep=0.1cm, topsep=0cm, parsep=0cm, itemsep=0cm, leftmargin=0.4cm, labelindent=0cm]}{\end{itemize}}

%%% Table parameters %%%

\newcolumntype{M}[1]{>{\centering\let\newline\\\arraybackslash\hspace{0pt}}m{#1}}


%%%  Lists handling %%%

\newcommand{\addlocallist}{\listadd\locallists@dummy}%
\NewDocumentCommand{\parsespacelist}{>{\SplitList{ }} m }{%
	\ProcessList{#1}{\addlocallist}%
}%
\NewDocumentCommand{\parsecommalist}{>{\SplitList{,}} m }{%
	\ProcessList{#1}{\addlocallist}%
}%
\newcommand{\parselist}[3][,]{%
	\renewcommand\addlocallist{\listadd#3}%
  	\undef#3%
  	\ifstrequal{#1}{ }{\parsespacelist{#2}}{\parsecommalist{#2}}%
}


%%% Profiles handling %%%

% Element of a table that contains the characteristics of a model (or part of a model)
\newcommand\caraclist[1]{
	\parselist[ ]{#1}{\locallists@caraclist}%
	\forlistloop{&}{\locallists@caraclist}%
}

\newcommand\caraclistbold[1]{
	\parselist[ ]{#1}{\locallists@caraclist}%
	\forlistloop{&\bfseries}{\locallists@caraclist}%
}

% Line of a profile table, including bottom line. It is meant to contain the name of the model (or part), its characteristics (preferably, the second argument should contain the \carac macro), troop type and base size.
\newcommand{\profilefirstline}[4]{#1 & #2 &   & #3 & #4 }

% Start of a profile table. Includes the table commands, and the column labels. \profilecellsize is the size of the characteristics cells in the profile.
\newcommand{\profilecellsize}{0.56cm}
\newcommand{\profilestart}{%
	\noindent %
	\begin{tabular}{@{}p{3cm}@{}M{\profilecellsize}@{}M{\profilecellsize}@{}M{\profilecellsize}@{}M{\profilecellsize}@{}M{\profilecellsize}@{}M{\profilecellsize}@{}M{\profilecellsize}@{}M{\profilecellsize}@{}M{\profilecellsize}@{}p{2.7cm}@{}p{3.3cm}@{}p{2cm}@{}}%
	 &% \textbf{\labels@profile}
	\labels@M & \labels@WS & \labels@BS & \labels@S & \labels@T & \labels@W & \labels@I & \labels@A & \labels@Ld &%
	&%
	{\unitentryformat{\labels@trooptype}} &%
	{\unitentryformat{\labels@basesize}}%
}

% End of a profile table.
\newcommand{\profileend}{\end{tabular}}

% Algorithm to automatically use and fill previous command, with coherence check.
\providebool{profilefirst}
\newcommand{\profileitem}[1]{%
	\tabularnewline%
	\splitatinf{#1}\local@unitname\local@unitprofile%
	\local@unitname \expandafter\caraclistbold\expandafter{\local@unitprofile}%
	&%
	& \ifbool{profilefirst}{\unit@type}{}%
	& \ifbool{profilefirst}{%
		\ifsubstring{\unit@basesize}{x}{% Rectangular base
			\unit{\unit@basesize}{\milli\meter}%
		}{% Circular base
			\unit{\unit@basesize}{\milli\meter} \labels@roundbase%
		}%
	}{}%
	\global\boolfalse{profilefirst}%
}
\newcommand{\profile}[1]{%
	\parselist{#1}{\locallists@profileslist}%
	\profilestart%
	\global\booltrue{profilefirst}%
	\forlistloop{\profileitem}{\locallists@profileslist}%
	\profileend%
}


%%% Profiles handling in case of invocation %%%

\newcommand{\invocprofilestart}{%
	\noindent %
	\begin{tabular}{@{}p{3cm}@{}M{\profilecellsize}@{}M{\profilecellsize}@{}M{\profilecellsize}@{}M{\profilecellsize}@{}M{\profilecellsize}@{}M{\profilecellsize}@{}M{\profilecellsize}@{}M{\profilecellsize}@{}M{\profilecellsize}@{}M{2.2cm}@{}p{0.5cm}@{}p{3.3cm}@{}p{2cm}@{}}%
	 &% \textbf{\labels@profile}
	\labels@M & \labels@WS & \labels@BS & \labels@S & \labels@T & \labels@W & \labels@I & \labels@A & \labels@Ld & \unitentryformat{\labels@Invocation} &%
	&%
	{\unitentryformat{\labels@trooptype}} &%
	{\unitentryformat{\labels@basesize}}%
}

\newcommand{\invocprofileitem}[1]{%
	\tabularnewline%
	\splitatinf{#1}\local@unitname\local@unitprofile%
	\local@unitname \expandafter\caraclistbold\expandafter{\local@unitprofile}%
	& \ifbool{profilefirst}{\unit@invocation}{} &%
	& \ifbool{profilefirst}{\unit@type}{}%
	& \ifbool{profilefirst}{\unit{\unit@basesize}{\milli\meter}}{}%
	\global\boolfalse{profilefirst}%
}

\newcommand{\invocprofile}[1]{%
	\parselist{#1}{\locallists@profileslist}%
	\invocprofilestart%
	\global\booltrue{profilefirst}%
	\forlistloop{\invocprofileitem}{\locallists@profileslist}%
	\profileend%
}


%%%%%%%%%%%%%%%%%%
%%% Unit rules %%%
%%%%%%%%%%%%%%%%%%

%%% Entry title command %%%

\newcommand{\unitentry}[2]{\ifdefempty{#1}{}{\noindent #2}}


%%% Special rules %%%

% Special rules listing for a unit, with alphabetical order.
\newcommand{\ruleslist}[1]{%
	\parselist[,]{#1}{\locallists@ruleslist}%
	\begin{sortedlist}%
		\forlistloop{\addtosortedlist}{\locallists@ruleslist}%
	\end{sortedlist}%
}

% Special rules entry.
\newcommand{\specialrules}[1]{\unitentry{#1}{\unitentryformat{\labels@specialrules\spacebeforecolon{}:}\newline\hspace*{-\fontdimen2\font}\expandafter\ruleslist\expandafter{#1}.}}
\newcommand{\commonspecialrules}[2]{\unitentry{#2}{\unitentryformat{#1\spacebeforecolon{}:}\newline\hspace*{-\fontdimen2\font}\expandafter\ruleslist\expandafter{#2}.}}


%%% Magical abilities %%%

% Paths listing for a unit.
\newcommand{\pathslist}[1]{%
	\parselist[,]{#1}{\locallists@pathslist}%
	\begin{itemize*}[label={}, itemjoin={,}, itemjoin*={\listlastchoice}]%
		\forlistloop{\item}{\locallists@pathslist}%
	\end{itemize*}%
}

% Magic entry.
\newcommand{\magic}[2]{\unitentry{#2}{\unitentryformat{\labels@magic\spacebeforecolon{}: }\newline\ifdefempty{#1}{}{\textbf{\magiclevel{#1}}. }\labels@pathsused\expandafter\pathslist\expandafter{#2}.}}

% Wizard Conclave.
\newcommand{\magicwizardconclave}[1]{\unitentry{#1}{\unitentryformat{\labels@magic\spacebeforecolon{}: }\newline\textbf{\wizardconclave{}}\spacebeforecolon{}: #1.}}


%%% Equipment %%%

% Equipment listing.
\newcommand{\equipmentlist}[1]{%
	\parselist[,]{#1}{\locallists@equipmentlist}%
	\begin{sortedlist}%
		\forlistloop{\addtosortedlist}{\locallists@equipmentlist}%
	\end{sortedlist}%
}

% Equipment entry.
\newcommand{\weapons}[1]{\unitentry{#1}{\unitentryformat{\labels@weapons\spacebeforecolon{}:}\newline\hspace*{-\fontdimen2\font}\expandafter\equipmentlist\expandafter{#1}.}}

\newcommand{\armour}[1]{\unitentry{#1}{\unitentryformat{\labels@armour\spacebeforecolon{}:}\newline\hspace*{-\fontdimen2\font}\expandafter\equipmentlist\expandafter{#1}.}}


%%% Alignment %%%

\newcommand{\alignment}[1]{\unitentry{#1}{\unitentryformat{\labels@alignment\spacebeforecolon{}:}\newline\textbf{#1}.}}

%%% Green Hide Race %%%

\newcommand{\greenhideraceentry}[1]{\unitentry{#1}{\unitentryformat{\labels@greenhiderace\spacebeforecolon{}:}\newline\textbf{#1}.}}


%%% Options %%%

% Frame commands.
\newcommand{\optionsframestart}{\begin{innerframe}[\labels@options]}
\newcommand{\optionsframeend}{\end{innerframe}}

% Options listing.
\newcommand{\optionslist}[1]{%
	\parselist[,]{#1}{\locallists@optionslist}%
	\begin{description}[leftmargin=0.3cm, labelindent=0cm, labelsep=0cm, itemsep=0cm, parsep=0cm]%
		\forlistloop{\item\setoption}{\locallists@optionslist}%
	\end{description}%
}

% Options entry.
\newcommand{\options}[1]{\ifdefempty{#1}{}{\optionsframestart\vspace*{-0.4cm}\unitentry{#1}{\expandafter\optionslist\expandafter{#1}}\optionsframeend}}

% Option specific commands.
\newcommand{\setoption}[1]{%
	\noexpandarg\StrCut{#1}{=}\optiontext\optionvalue%
	\expandafter\ifstrequal\expandafter{\optionvalue}{}{%
		\optiontext%
	}{%
	\ifsubstring{\optionvalue}{\free}{%
		\option[\free]{\optiontext}{\optionvalue}%
	}{%
	\ifsubstring{\optionvalue}{\unlimited}{%
		\option[\unlimited]{\optiontext}{\optionvalue}%
	}{%
	\ifsubstring{\optionvalue}{\upto}{%
		\splitatinf{\optionvalue}\myoption\myvalue%
		\option[\upto]{\optiontext}{\myvalue}%
	}{%
	\ifsubstring{\optionvalue}{\permodel}{%
		\splitatinf{\optionvalue}\myoption\myvalue%
		\option[\permodel]{\optiontext}{\myvalue}%
	}{%
	\ifsubstring{\optionvalue}{\pershadygit}{% For Orcs N Goblins
		\splitatinf{\optionvalue}\myoption\myvalue%
		\option[\pershadygit]{\optiontext}{\myvalue}%
	}{%
	\ifsubstring{\optionvalue}{\permadgit}{% For Orcs N Goblins
		\splitatinf{\optionvalue}\myoption\myvalue%
		\option[\permadgit]{\optiontext}{\myvalue}%
	}{%	
	\ifsubstring{\optionvalue}{\perrune}{% For Dwarven Holds
		\splitatinf{\optionvalue}\myoption\myvalue%
		\option[\perrune]{\optiontext}{\myvalue}%
	}{%	
		\option{\optiontext}{\optionvalue}%
	}}}}}}}}%
}

\newcommand{\option}[3][]{#2\predotfill\dotfill\nobreak%
	% Add \upto token if necessary.
	\ifstrequal{#1}{\upto}{\upto~}{}%
	% The option can be free, have an unlimited cost, or have a points cost.
	\ifstrequal{#1}{\free}{\free}{\ifstrequal{#1}{\unlimited}{\unlimited}{\pts{#3}}}%
	% Add \permodel if necessary.
	\ifstrequal{#1}{\permodel}{\nobreak\permodel}{}%
	% Add \persomething if necessary.
	\ifstrequal{#1}{\pershadygit}{\nobreak\pershadygit}{}% For Orcs N Goblins
	\ifstrequal{#1}{\permadgit}{\nobreak\permadgit}{}% For Orcs N Goblins
	\ifstrequal{#1}{\perrune}{\nobreak\perrune}{}% For Dwarven Holds
}

\newcommand\optionschoice[2]{%
	\parselist[,]{#2}{\locallists@optionschoice}%
	#1%
	\begin{itemize}[label={}, parsep=0cm, labelindent=0cm, labelwidth=0cm, noitemsep, topsep=0em, leftmargin=0.3cm]%
	\forlistloop{\item\setoption}{\locallists@optionschoice}%
	\end{itemize}%
}

\newcommand\optionschoiceTWOCOL[2]{%
	\parselist[,]{#2}{\locallists@optionschoice}%
	#1%
	\begin{itemize}[label={}, parsep=0cm, labelindent=0cm, labelwidth=0cm, noitemsep, topsep=0em, leftmargin=0.3cm]%
	\setlength{\columnseprule}{0.5pt}
	\renewcommand{\columnseprulecolor}{\color{black!30}}
	\vspace*{-5pt}\begin{multicols}{2}\raggedcolumns
	\forlistloop{\item\setoption}{\locallists@optionschoice}%
	\end{multicols}\setlength{\columnseprule}{0pt}
	\end{itemize}%
}

% Option description in army desc.
\newcommand{\optiondef}[3]{\option{\textbf{#1}}{#2}\ifblank{#3}{}{\\{#3}}}


%%% Mount options %%%

% Frame commands.
\newcommand{\mountsframestart}{\begin{innerframe}[\labels@mounts]}
\newcommand{\mountsframeend}{\end{innerframe}}

% Mount listing.
\newcommand{\mountslist}[1]{%
	\parselist[,]{#1}{\locallists@mountslist}%
	\begin{description}[leftmargin=0.3cm, labelindent=0cm, labelsep=0cm, itemsep=0cm, parsep=0cm]%
		\forlistloop{\item\setoption}{\locallists@mountslist}%
	\end{description}%
}

% Mount entry.
\newcommand{\mounts}[1]{\ifdefempty{#1}{}{\mountsframestart\vspace*{-0.4cm}\unitentry{#1}{\expandafter\mountslist\expandafter{#1}}\mountsframeend}}


%%% Command group %%%

% Command group specific commands.
\define@key{commandgroup}{restriction}            {\def\commandgroup@restriction{#1}}
\define@key{commandgroup}{champion}               {\def\commandgroup@champion{#1}}
\define@key{commandgroup}{championallowance}      {\def\commandgroup@championallowance{#1}}
\define@key{commandgroup}{championoption}         {\def\commandgroup@championoption{#1}}
\define@key{commandgroup}{championprerestriction} {\def\commandgroup@championprerestriction{#1}}
\define@key{commandgroup}{championrestriction}    {\def\commandgroup@championrestriction{#1}}
\define@key{commandgroup}{banner}                 {\def\commandgroup@banner{#1}}
\define@key{commandgroup}{bannerallowance}        {\def\commandgroup@bannerallowance{#1}}
\define@key{commandgroup}{veteranstandardbearer}  {\def\commandgroup@veteranstandardbearer{#1}}
\define@key{commandgroup}{singlebannerallowance}  {\def\commandgroup@singlebannerallowance{#1}}
\define@key{commandgroup}{condsinglebannerallowance}  {\def\commandgroup@condsinglebannerallowance{#1}}
\define@key{commandgroup}{banneroption}           {\def\commandgroup@banneroption{#1}}
\define@key{commandgroup}{bannerrestriction}      {\def\commandgroup@bannerrestriction{#1}}
\define@key{commandgroup}{musician}               {\def\commandgroup@musician{#1}}
\define@key{commandgroup}{musicianrestriction}    {\def\commandgroup@musicianrestriction{#1}}
\newcommand{\defcommandgroup}{%
	\setkeys{commandgroup}{restriction=,
	                       champion=, championallowance=, championoption=, championprerestriction=, 
	                       championrestriction=, banner=, bannerallowance=, veteranstandardbearer=, 
	                       singlebannerallowance=, condsinglebannerallowance=, banneroption=, 
	                       bannerrestriction=, musician=, musicianrestriction=}%
	\setkeys{commandgroup}%
}

% Frame commands.
\newcommand{\commandgroupframestart}{\begin{innerframe}[\labels@commandgroup]}
\newcommand{\commandgroupframeend}{\end{innerframe}}

% Command group entry.
\newcommand{\commandgroup}[1]{%
	\defcommandgroup{#1}%
	\ifstrempty{#1}{}{\commandgroupframestart\vspace*{-0.2cm}%
		\begin{description}[leftmargin=0.3cm, labelindent=0cm, labelsep=0cm, itemsep=0cm, parsep=0cm]%
			% Command group title, including restrictions applying to all the command group
			\item \textbf{\expandafter\ifblank\expandafter{\commandgroup@restriction}{}{ \only{\commandgroup@restriction}\spacebeforecolon{}: }} 
			% Champion handling.
			\ifdefempty{\commandgroup@champion}{}{% We have a champion!
			\ifdefempty{\commandgroup@championprerestriction}{% There is no prerestriction to have a champion
				\item \hspace*{-0.04cm}\option{\labels@champion%
					% Possible restrictions to taking a champion
				    \expandafter\ifblank\expandafter{\commandgroup@championrestriction}{}{ \only{\commandgroup@championrestriction}}%
				    % Cost of a champion
				    }{\commandgroup@champion}%
				    % Magical allowance of the champion. Should probably not be used, champion option can do it as well and is more flexible.
					\ifdefempty{\commandgroup@championallowance}{}{\par\option[\upto]{\hspace*{0.3cm}- \labels@championallowance}{\commandgroup@championallowance}}%
					% Any option available to the champion, in the form option:cost
					\ifdefempty{\commandgroup@championoption}{}{%
						\splitatinf{\commandgroup@championoption}\local@option\local@cost%
						\par\option{\hspace*{0.3cm}- \local@option}{\local@cost}}%
			}{% There is a pre-restriction to have a champion
				\item \hspace*{-0.04cm}\commandgroup@championprerestriction	\newline%
				\option{\labels@champion}{\commandgroup@champion}%
				% Magical allowance of the champion. Should probably not be used, champion option can do it as well and is more flexible.
				\ifdefempty{\commandgroup@championallowance}{}{\par\option[\upto]{\hspace*{0.3cm}- \labels@championallowance}{\commandgroup@championallowance}}%
				% Any option available to the champion, in the form option:cost
				\ifdefempty{\commandgroup@championoption}{}{%
					\splitatinf{\commandgroup@championoption}\local@option\local@cost%
					\par\option{\hspace*{0.3cm}- \local@option}{\local@cost}}%
			} %End of the prerestriction of not condition
			}% End of champion handling
			\ifdefempty{\commandgroup@musician}{}{% We have a musician!
				\item \hspace*{-0.04cm}\option{\labels@musician%
					% Possible restrictions to taking a musician
				    \expandafter\ifblank\expandafter{\commandgroup@musicianrestriction}{}{ \only{\commandgroup@musicianrestriction}}%
				    % Cost of a musician
				    }{\commandgroup@musician}%
			}%
			\ifdefempty{\commandgroup@banner}{}{% We have a banner!
				\item \hspace*{-0.04cm}\option{\labels@standardbearer%
					% Possible restrictions to taking a banner
				    \expandafter\ifblank\expandafter{\commandgroup@bannerrestriction}{}{ \only{\commandgroup@bannerrestriction}}%
				    % Cost of a banner
				    }{\commandgroup@banner}%
				    % Magical banner, if all units of this type can take one.
					\ifdefempty{\commandgroup@bannerallowance}{}{\par\option[\upto]{\hspace*{0.3cm}- \labels@bannerallowance}{\commandgroup@bannerallowance}}%
					% Magical banner, if Veteran.
					\ifdefempty{\commandgroup@veteranstandardbearer}{}{\par\hspace*{0.3cm}- \labels@veteranstandardbearer%
					\expandafter\ifstrequal\expandafter{\commandgroup@veteranstandardbearer}{*}{*}{}%
					}%
					% Magical banner, if only one unit of this type can take one.
					\ifdefempty{\commandgroup@singlebannerallowance}{}{\par\option[\upto]{\hspace*{0.3cm}- \labels@singlebannerallowance}{\commandgroup@singlebannerallowance}}%
					% Magical banner, if only one unit of this type can take one, but with condtions.
					\ifdefempty{\commandgroup@condsinglebannerallowance}{}{%
						\splitatinf{\commandgroup@condsinglebannerallowance}\local@option\local@cost%
						\par\option[\upto]{\hspace*{0.3cm}- \labels@condsinglebannerallowance \local@option}{\local@cost}}%
					% Additional option for the banner, such as Hill Goblin Lookouts for Ogres
					\ifdefempty{\commandgroup@banneroption}{}{%
						\splitatinf{\commandgroup@banneroption}{\local@option}{\local@cost}%
						\par\option{\hspace*{0.3cm}- \local@option}{\local@cost}%
					}%
			}%
		\end{description}%
	\commandgroupframeend%
	 }%
}


%%% Unit rules %%%

% Frame commands.
\newcommand{\unitrulesframestart}{\begin{innerframe}[\labels@specialrules]}
\newcommand{\unitrulesframeend}{\end{innerframe}}

% Unit rules specific commands.
\newcommand{\unitrule}[2]{\item[#1\spacebeforecolon{}:]#2}

% Unit rule entry.
\newcommand{\unitrules}[1]{\ifdefempty{#1}{}{\unitrulesframestart\vspace*{-0.05cm}\begin{description}[leftmargin=0.3cm, labelindent=0cm, labelsep=0.1cm, itemsep=0.2cm, parsep=0cm]#1\end{description}\unitrulesframeend}}


%%% Special equipment %%%

% Frame commands.
\newcommand{\unitequipmentframestart}{\begin{innerframe}[\labels@specialequipment]}
\newcommand{\unitequipmentframeend}{\end{innerframe}}

% Special equipment specific commands.
\newcommand{\equipmentdef}[2]{\item[#1\spacebeforecolon{}:]#2}

% Special equipment entry.
\newcommand{\unitequipment}[1]{\ifdefempty{#1}{}{\unitequipmentframestart\vspace*{-0.05cm}\begin{description}[leftmargin=0.3cm, labelindent=0cm, labelsep=0.1cm, itemsep=0.2cm, parsep=0cm]#1\end{description}\unitequipmentframeend}}






%%%%%%%%%%%%%%%%%%%%%%%%%%%%%%%%
%%% Profile input and layout %%%
%%%%%%%%%%%%%%%%%%%%%%%%%%%%%%%%

%%% Input parameters %%%

\define@key{unit}{notinQRS}{\def\unit@notinQRS{#1}}
\define@key{unit}{name}{\def\unit@name{#1}}
\define@key{unit}{QRSname}{\def\unit@QRSname{#1}}
\define@key{unit}{profile}{\def\unit@profile{#1}}
\define@key{unit}{cost}{\def\unit@cost{#1}}
\define@key{unit}{invocation}{\def\unit@invocation{#1}}
\define@key{unit}{costpermodel}{\def\unit@costpermodel{#1}}
\define@key{unit}{maxmodels}{\def\unit@maxmodels{#1}}
\define@key{unit}{type}{\def\unit@type{#1}}
\define@key{unit}{unitsize}{\def\unit@unitsize{#1}}
\define@key{unit}{basesize}{\def\unit@basesize{#1}}
\define@key{unit}{commonspecialrules}{\def\unit@commonspecialrules{#1}}
\define@key{unit}{commontype}{\def\unit@commontype{#1}}
\define@key{unit}{commonspecialrulesB}{\def\unit@commonspecialrulesB{#1}}
\define@key{unit}{commontypeB}{\def\unit@commontypeB{#1}}
\define@key{unit}{specialrules}{\def\unit@specialrules{#1}}
\define@key{unit}{magiclevel}{\def\unit@magiclevel{#1}}
\define@key{unit}{magicpaths}{\def\unit@magicpaths{#1}}
\define@key{unit}{equipment}{\def\unit@equipment{#1}}
\define@key{unit}{alignment}{\def\unit@alignment{#1}}
\define@key{unit}{greenhiderace}{\def\unit@greenhiderace{#1}}
\define@key{unit}{weapons}{\def\unit@weapons{#1}}
\define@key{unit}{armour}{\def\unit@armour{#1}}
\define@key{unit}{wizardconclave}{\def\unit@wizardconclave{#1}}
\define@key{unit}{unitequipment}{\def\unit@unitequipment{#1}}
\define@key{unit}{options}{\def\unit@options{#1}}
\define@key{unit}{mounts}{\def\unit@mounts{#1}}
\define@key{unit}{commandgroup}{\def\unit@commandgroup{#1}}
\define@key{unit}{unitrules}{\def\unit@unitrules{#1}}
\define@key{unit}{additional}{\def\unit@additional{#1}}


%%% Frames definition %%%

% Unit's big frame.
\tikzset{unitprice/.style={draw=white, fill=white, rectangle, rounded corners, right, minimum height=0.7cm}}
\tikzset{unittitle/.style={draw=white, fill=white, rectangle, rounded corners, right, minimum height=0.7cm, font=\bfseries}}
\tikzset{unitlogo/.style={draw=white, fill=white, rectangle, right, minimum height=0.7cm}}

\newenvironment{unitframe}[2][]{%
	\mdfsetup{%
		nobreak=true,%
		linewidth=1pt,%
		linecolor=black!30,%
		roundcorner=5pt,%
		backgroundcolor=white,%
		innertopmargin=1.2\baselineskip,
		innerbottommargin=1.2\baselineskip,
		singleextra={
			\expandafter\ifblank\expandafter{\unit@cost}{}{%
				\node[unitprice,anchor=east,xshift=-0.5cm] at (P)%
					{%
						{{\smallfontsize\minprice} \Largefontsize\pts{\textbf{\unit@cost}}}%
					};
				}%
				\node[unittitle,xshift=0.5cm] at (P-|O)%
					{\Largefontsize\antiquefont\uppercase\expandafter\expandafter\expandafter{\unit@name}};
				\node[unitlogo, xshift=8.1cm, yshift=0.1cm] at (P-|O)%
					{\includegraphics[width=1.2cm]{\logolocalpath}};
		}
	}%
	\begin{mdframed}[]\relax%
}%
{%
\end{mdframed}%
}

% Inner small frames for options, special rules definition, ...
\tikzset{innertitle/.style={fill=white, rectangle, rounded corners, right, minimum height=8pt, xshift=0.5cm}}

\newenvironment{innerframe}[1][]{%
	\mdfsetup{%
		innerleftmargin=5pt,%
		innerrightmargin=5pt,%
		linecolor=black!30,%
		linewidth=0.5pt,%
		roundcorner=5pt,%
		backgroundcolor=white,%
		innertopmargin=1.1\baselineskip,
		singleextra={
		\node[innertitle] at (P-|O)%
			{\unitentryformat{#1}};
		}
	}%
	\vspace*{-0.2cm}\begin{mdframed}[]\relax%
}%
{%
\end{mdframed}%
}

%%% Command to add a new unit definition %%%

\newcommand{\defunit}{
	\setkeys{unit}{%
		notinQRS=, name=, QRSname=, profile=, cost=, invocation=, costpermodel=, maxmodels=, type=, unitsize=, basesize=, commonspecialrules=, commontype=, commonspecialrulesB=, commontypeB=, specialrules=, magiclevel=, magicpaths=, alignment=, greenhiderace=, equipment=, weapons=, armour=, wizardconclave=, unitequipment=, options=, mounts=, commandgroup=, unitrules=, additional=%
	}%
	\setkeys{unit}%
}

\newcommand{\showunit}[1]{
	\defunit{#1}
	\begin{unitframe}[\unit@name]{\unit@cost}
	\mdfsetup{style=defaultoptions}
	\expandafter\ifblank\expandafter{\unit@unitsize}{}{%
	\expandafter\ifstrequal\expandafter{\unit@unitsize}{1}{% single model
		% Can you add model to this single model ?
		\expandafter\ifblank\expandafter{\unit@maxmodels}{% no		
			{\hspace*{0.25cm}\labels@Singlemodel}%
		}{% yes
			{\hspace*{0.25cm}\mincostfor{} \textbf{1} \labels@model{}. \maxunitsize{}\spacebeforecolon{}: \textbf{\unit@maxmodels} \labels@models{}.\hfill \additionalfigscost{} {\largefontsize\pts{\textbf{\unit@costpermodel{}}}\permodel}\hspace*{0.1cm}}%
		}%
	}{% not single model
		% Test if we wanna print a sentence instead of unit number
		\ifsubstring{\unit@unitsize}{SPECIAL-}{%
			\hspace*{0.25cm}\StrDel{\unit@unitsize}{SPECIAL-}%
		}{%	
			{\hspace*{0.25cm}\mincostfor{} \textbf{\unit@unitsize} \labels@models{}. \maxunitsize{}\spacebeforecolon{}: \textbf{\unit@maxmodels} \labels@models{}.\hfill \additionalfigscost{} {\largefontsize\pts{\textbf{\unit@costpermodel{}}}\permodel}\hspace*{0.1cm}}%
		}%
	}%
	}%
	\vspace*{-0.1cm}
	\noindent\begin{center}\textcolor{black!30}{\rule{\columnwidth}{1pt}}\end{center}
		\expandafter\ifblank\expandafter{\unit@invocation}{%
			\expandafter\profile\expandafter{\unit@profile}
		}{%
			\expandafter\invocprofile\expandafter{\unit@profile}
		}
	\noindent\begin{center}\textcolor{black!30}{\rule{\columnwidth}{1pt}}\end{center}
	\vspace*{-0.2cm}
	\setlength\multicolsep{0pt}
	\begin{multicols}{2}
		\raggedcolumns
		\vspace*{-0.3cm}{\setlength{\parskip}{0.3cm}
		\expandafter\ifblank\expandafter{\unit@alignment}{}{\noindent\parbox{\columnwidth}{\alignment{\unit@alignment}}}
		
		\expandafter\ifblank\expandafter{\unit@greenhiderace}{}{\noindent\parbox{\columnwidth}{\greenhideraceentry{\unit@greenhiderace}}}
		
		\expandafter\ifblank\expandafter{\unit@equipment}{}{\noindent\parbox{\columnwidth}{\equipment{\unit@equipment}}}
				
		\expandafter\ifblank\expandafter{\unit@weapons}{}{\noindent\parbox{\columnwidth}{\weapons{\unit@weapons}}}
		
		\expandafter\ifblank\expandafter{\unit@armour}{}{\noindent\parbox{\columnwidth}{\armour{\unit@armour}}}
		
		\expandafter\ifblank\expandafter{\unit@commonspecialrules}{}{\noindent\parbox{\columnwidth}{\commonspecialrules{\unit@commontype}{\unit@commonspecialrules}}}
		
		\expandafter\ifblank\expandafter{\unit@commonspecialrulesB}{}{\noindent\parbox{\columnwidth}{\commonspecialrules{\unit@commontypeB}{\unit@commonspecialrulesB}}}
		
		\expandafter\ifblank\expandafter{\unit@specialrules}{}{\noindent\parbox{\columnwidth}{\specialrules{\unit@specialrules}}}
		
		\expandafter\ifblank\expandafter{\unit@magicpaths}{}{\noindent\parbox{\columnwidth}{\magic{\unit@magiclevel}{\unit@magicpaths}}}
		
		\expandafter\ifblank\expandafter{\unit@wizardconclave}{}{\noindent\parbox{\columnwidth}{\magicwizardconclave{\unit@wizardconclave}}}
		}
		\vspace{0.1cm}
		\mounts{\unit@mounts}
		\options{\unit@options}
		\expandafter\ifblank\expandafter{\unit@commandgroup}{}{\expandafter\commandgroup\expandafter{\unit@commandgroup}}
		\unitrules{\unit@unitrules}
		\unitequipment{\unit@unitequipment}
	\end{multicols}
	\vspace*{0.1cm}\unit@additional
	\end{unitframe}
	% Database filling for auto QRS
	\expandafter\ifblank\expandafter{\unit@notinQRS}{%
	\DTLnewrow{profiles}%
	\expandafter\ifblank\expandafter{\unit@QRSname}{%
		\expandafter\profiledtbfillname\expandafter{\unit@name}%
	}{%
		\expandafter\profiledtbfillname\expandafter{\unit@QRSname}%
	}
	\expandafter\profiledtbfillcategory\expandafter{\profilecategory}%
	\expandafter\profiledtbfilltrooptype\expandafter{\unit@type}%
	\expandafter\ifblank\expandafter{\unit@invocation}{}{\expandafter\profiledtbfillinvocation\expandafter{\unit@invocation}}%
	\expandafter\profiledtbfillcarac\expandafter{\unit@profile}
	}{}%
}


%%% Changelog commands %%%

\newcommand{\newlog}[2]{%
\vspace*{0.2cm}\noindent{\antiquefont\Large\textbf{V#1}}
\parselist[,]{#2}{\locallists@changelist}%
\begin{itemize}[itemsep=0pt]%
\forlistloop{\item[-]}{\locallists@changelist}%
\end{itemize}%
}

\newcommand{\startchangelog}{\begin{multicols}{2}\vspace*{-0.2cm}}
\def\endchangelog{\end{multicols}}


\newcommand{\booktitle}{Conclaves Vampiriques}
\newcommand{\version}{0.99.2}
\newcommand{\frenchversion}{2.1}
\newcommand{\translationteam}{\item \og AEnoriel \fg \item \og Anglachel \fg \item \og Astadriel \fg \item \og Batcat \fg \item \og Eru \fg  \item \og Gandarin \fg \item \og Groumbahk \fg \item \og Iluvatar \fg \item \og Lamronchak \fg \item \og Mammstein \fg}

\newcommand{\logosize}{2.5cm}

% Army wide special rules

\newnamemacro{\masterofundeath}{Maître de la Non-Vie}
\newcommand{\master}{Maître}
\newcommand{\invocation}{Invocation}

% Army special rules

\newnamemacro{\ashestoashes}{Poussière Tu Redeviendras Poussière}
\newcommand{\wailofwoe}{Plainte Stridente}
\newcommand{\awaken}[1]{Sombre Réveil\ifblank{#1}{}{ (#1)}}
\newnamemacro{\reaper}{Moissonneur}
\newcommand{\vampiric}[1]{Vampirique\ifblank{#1}{}{ (#1+)}}
\newnamemacro{\necromanticaura}{Aura de Stabilité}

% Army common type special rules

\newcommand{\vampiriccommonrules}{Règles spéciales Vampiriques}
\newcommand{\undeadcommonrules}{Règles spéciales des Morts-Vivants}

% Vampiric Bloodlines

\newnamemacro{\bloodpower}{Pouvoir Dynastique}
\newnamemacro{\bloodpowers}{Pouvoirs Dynastiques}
\newnamemacro{\ancientbloodpower}{Pouvoir Dynastique Ancien}
\newnamemacro{\ancientbloodpowers}{Pouvoirs Dynastiques Anciens}
\newnamemacro{\bloodline}{Dynastie}
\newnamemacro{\bloodlines}{Dynasties}
\newcommand{\bloodtie}[1]{Unité Dynastique\ifblank{#1}{}{ (#1)}}
\newcommand{\bloodties}[1]{Unités Dynastiques\ifblank{#1}{}{ (#1)}}
\newnamemacro{\brotherhood}{Confrérie du Dragon}
\newnamemacro{\strigois}{Stryges}
\newnamemacro{\strigoi}{Stryge}
\newnamemacro{\vonkarnstein}{Von Karnstein}
\newnamemacro{\lamia}{Lamia}
\newnamemacro{\nosferatu}{Nosferatu}
\newcommand{\vampires}{Vampires}

% Other rules

\newnamemacro{\unlivingshield}{Bouclier Non-Vivant}
\newnamemacro{\stormofwings}{Déluge d'Ailes}
\newnamemacro{\wakethedead}{Debout les Morts}
\def\endlesshorde{Horde sans Fin} % \def necessary: a \newcommand name can't start with "end"
\newnamemacro{\bonepyre}{Bûcher Infernal}
\newnamemacro{\bringoutyourdead}{Sortez vos Macchabées}
\newnamemacro{\cart}{Charrette}
\newcommand{\bloodpool}{Vasque de Sang}
\newcommand{\ghoststeeds}{Montures Fantômes}
\newnamemacro{\undeadconstruct}{Construction Nécromantique}
\newcommand{\autonomous}{Autonome}
\newnamemacro{\darktome}{Sombre Recueil}
\newnamemacro{\auraofundeath}{Aura Nécrotique}
\newcommand{\extendedchassis}{Chassis Renforcé}
\newnamemacro{\soulsyphon}{Syphon d'Âmes}
\newcommand{\chillingshriek}{Hurlement Glaçant}
\newcommand{\greatmonstrousrevenant}{Revenant Immense}
\newnamemacro{\colossalzombiedragon}{Gigantesque Dragon Zombie}

%%% Names

% Spells

\newnamemacro{\necromancysignaturespell}{Adjuration des Morts}
\newnamemacro{\necromancyattribute}{Tromper la Faucheuse}
\newnamemacro{\necromancyspelltwo}{Sarabande Macabre}

\newnamemacro{\deathsignaturespell}{Le Baiser de la Faucheuse}
\newnamemacro{\shadowssignaturespell}{Miasmes Obscurs}

\newnamemacro{\heavensspelltwo}{Choc Foudroyant}

\newnamemacro{\lustspellfour}{Fantasmagorie}

% Characters

\newnamemacro{\vampirelord}{Comte Vampire}
\newnamemacro{\vampirelords}{Comtes Vampires}
\newnamemacro{\necromancerlord}{Maître Nécromant}
\newnamemacro{\vampirehero}{Baron Vampire}
\newnamemacro{\vampireheroes}{Barons Vampires}
\newnamemacro{\barrowking}{Roi des Tertres}
\newnamemacro{\barrowkings}{Rois des Tertres}
\newnamemacro{\necromancer}{Nécromant}
\newnamemacro{\fellwraith}{Apparition Lugubre}
\newnamemacro{\banshee}{Âme en Peine}

% Core

\newnamemacro{\zombie}{Zombie}
\newnamemacro{\zombies}{Zombies}
\newnamemacro{\skeletons}{Squelettes}
\newnamemacro{\ghouls}{Goules}
\newnamemacro{\direwolves}{Loups Sinistres}
\newnamemacro{\batswarm}{Essaim de Chauves-sou\-ris}
\newnamemacro{\batswarms}{Essaims de Chauves-sou\-ris}

% Special

\newnamemacro{\barrowguards}{Gardes des Tertres}
\newnamemacro{\barrowknights}{Chevaliers des Tertres}
\newnamemacro{\greatbats}{Grandes Chauves-Sou\-ris}
\newnamemacro{\ghasts}{Atrocités}
\newnamemacro{\cadaverwagon}{Charrette à Cadavres}
\newnamemacro{\cadaverwagons}{Charrettes à Cadavres}
\newnamemacro{\varkolak}{Varkolak}
\newnamemacro{\phantomhost}{Horde de Spectres}
\newnamemacro{\vampirespawn}{Progéniture Vampi\-ri\-que}
\newnamemacro{\courtofthedamned}{Cortège de Damnées}

% Rare

\newnamemacro{\vampireknights}{Chevaliers Vampires}
\newnamemacro{\darkcoach}{Carrosse Impie}
\newnamemacro{\wraiths}{Apparitions Vaga\-bon\-des}
\newnamemacro{\wingedreapers}{Faucheurs Ailés}
\newnamemacro{\altarofundeath}{Autel de la Non-Vie}
\newnamemacro{\shriekinghorror}{Terreur Hurlante}

% Mounts

\newnamemacro{\skeletalsteed}{Monture Squelette}
\newnamemacro{\spectralsteed}{Monture Spectrale}
\newnamemacro{\monstrousrevenant}{Revenant Monstrueux}
\newnamemacro{\zombiedragon}{Dragon Zombie}

% Short names and multiprofiles names

\newnamemacro{\steed}{Monture}
\newnamemacro{\rider}{Cavalier}
\newnamemacro{\cadavermaster}{Maître des Cadavres}
\newnamemacro{\shamblinghorde}{Horde Chancelante}
\newcommand{\floatingcourt}{Cortège Flottant}
\newnamemacro{\paramour}{Favorite}
\newnamemacro{\undeadmount}{Palefroi Mort-Vivant}
\newnamemacro{\wraithprofile}{Apparition}
\newcommand{\ghoststeed}{Monture Fantôme}
\newcommand{\altar}{Autel}
\newcommand{\coach}{Carrosse}
\newcommand{\coachman}{Cocher}
\newnamemacro{\awakenedvampire}{Vampire Réveillé}


% QRS Invocation table

\newcommand{\allchariots}{Tous les \chariots{}}


% Profile wording

\newcommand{\bloodlinechoice}{Peut choisir une \bloodline}
\newcommand{\asinglebloodpower}{Un seul \bloodpower}
\newcommand{\asingleancientbloodpower}{Un seul \ancientbloodpower}
\newcommand{\bloodpowerchoice}{Peut choisir un seul \bloodpower}
\newcommand{\bloodlinenote}{%
Ne peut être pris que si l'armée représente une Dynastie.
}
\newcommand{\strigoivanguardnote}{%
Les Vampires \strigois{} ayant rejoint cette unité gagnent la règle \vanguard{}.
}
\newcommand{\maytakebloodpool}{Peut prendre la \bloodpool}
\newcommand{\seeshriekinghorrorinraresection}{(voir la \shriekinghorror{} dans les unités rares)}
\newcommand{\seecadaverwagoninspecialsectionforupgdraderules}{Voir la \cadaverwagon{} dans les unités spéciales pour la description des règles.}

% Profile rules

\newcommand{\unlivingshieldrule}{%
Les figurines ennemies qui pourraient allouer des attaques de Corps à Corps à une figurine avec cette règle et un \necromancer{} ou un \necromancerlord{} ne peuvent pas choisir de cibler le \necromancer{} ou le \necromancerlord{}. Cette règle ne peut pas être utilisée s'il y a des figurines avec la règle \vampiric{} dans la même unité.
}

\newcommand{\stormofwingsrule}{%
Les unités ennemies au contact d'une ou plusieurs figurines d'\batswarm{} subissent -1 en Capacité de Combat (jusqu'à un minimum de 1).
}

\newcommand{\cartrule}{%
La \cadaverwagon{} perd les règles \swiftstride{} et \cannotmarch{}.
}

\newcommand{\wakethedeadrule}{%
À chaque fois qu'un sort d'Augmentation de la Discipline \necromancy{} (cela inclut \necromancyattribute{}) affecte une unité qui contient une \cadaverwagon{}, vous pouvez choisir une unité à moins de \distance{6} de cette unité. Jusqu'à la fin du tour de joueur suivant, toutes les figurines de l'unité choisie bénéficient de la règle \lightningreflexes{}.
}

\def\endlesshorderule{%
La \cadaverwagon{} gagne la règle \warplatform{}, mais ne peut rejoindre que des \zombies{}. La \cadaverwagon{} peut alors lancer et accepter des défis comme si elle était le champion de l'unité de \zombies{} rejointe. Si cette option est prise, la taille du socle de la figurine devient \unit{60x100}{\milli\meter}.
}

\newcommand{\bonepyrerule}{%
Les \wizards{} ennemis dans un rayon de \distance{24} d'une ou plusieurs \cadaverwagons{} avec cette option souffrent d'un malus de -1 à leurs tentatives pour lancer les sorts.
}

\newcommand{\bringoutyourdeadrule}{%
Les \wizards{} amis font Ressusciter deux PVs supplémentaires aux cibles de taille Petite et un aux cibles de taille Moyenne lorsqu'ils lancent le sort \necromancysignaturespell{} avec succès sur des cibles dans un rayon de \distance{6} d'une ou plusieurs \cadaverwagons{} avec cette option.
}

\newcommand{\bloodpoolrule}{%
Les figurines ordinaires d'Infanterie appartenant à des unités alliées situées dans un rayon de \distance{6} d'une ou plusieurs figurines avec une \bloodpool{} gagnent +1 en Capacité de Combat. Les unités ennemies à moins de \distance{6} d'une ou plusieurs figurines avec une \bloodpool{} subissent -2 en Initiative, jusqu'à un minimum de 1.
}

\newcommand{\vampireknightbloodtiesoption}{%
Peut gagner une \platearmour{} et \devastatingcharge{} (\rider{} seulement)%
}

\newcommand{\ghoststeedsrule}{%
La figurine monte une \ghoststeed{}. Son type d'unité change en \cavalry{}, sa taille de socle en \unit{25x50}{\milli\meter}, elle perd la règle \skirmishers{} et gagne une \mountsprotection{6}.%
}

\newcommand{\undeadconstructrule}{%
Les figurines avec cette règle subissent une blessure de moins lorsqu'elles appliquent les règles \unstable{} et \ashestoashes{}.
}

\newcommand{\autonomousrule}{%
L'unité peut effectuer une Marche Forcée, même si elle n'est pas à portée de la \inspiringpresence{} du Général.%
}

\newcommand{\chillingshriekrule}{%
L'élément de figurine avec cette règle spéciale peut effectuer une Attaque Spéciale de Tir et une Attaque Spéciale de Corps à Corps détaillées ci-dessous.
\begin{itemize}
\item Attaque Spéciale de Tir (normalement durant la Phase de Tir) : Choisissez une cible en utilisant les règles habituelles des Attaques de Tir. La Portée de l'attaque est de \distance{8}. Le tir est possible même si la figurine a effectué une Marche Forcée durant la Phase de Mouvement précédente.
\item Attaque Spéciale de Corps à Corps (normalement durant la Phase de Corps à Corps) : Si le joueur choisit de l'utiliser, cette attaque est portée à l'Initiative de l'élément de figurine et remplace les attaques normales non-Spéciales de l'élément de figurine. Sélectionnez une unique unité en contact socle à socle avec la figurine comme cible.
\end{itemize}
Que ce soit en tant qu'Attaque de Tir ou de Corps à Corps, le \chillingshriek{} inflige une touche automatique à la cible pour chaque Point de Vie que la \shriekinghorror{} possède au moment de l'attaque. Ces touches sont résolues à Force 10 et ont \magicalattacks{} et \armourpiercing{6}. Les jets pour blesser sont faits contre le Commandement de la cible au lieu de son Endurance.
}

\newcommand{\bansheerule}{%
Une unique \banshee{} avec la règle spéciale \wailofwoe{} rejoint l'équipage.%
}

\newcommand{\darktomerule}{%
Les \wizards{} amis à moins de \distance{12} d'au moins un \altarofundeath{} équipé d'un \darktome{} ajoutent +2 à leurs tentatives de lancement de Sorts de la Discipline \necromancy{}. Tout \wizard{} subissant un Fiasco à moins de \distance{12} d'un \altarofundeath{} équipé d'un \darktome{} compte comme ayant utilisé deux Dés de Pouvoir supplémentaires pour lancer le sort (maximum 5).
}

\newcommand{\auraofundeathrule}{%
Au début de chacun de vos Tours de Joueur, vous pouvez choisir l'un des effets suivants. Dans les deux cas, X est la valeur du Tour de Jeu actuel.
	\begin{itemize}
		\item Toutes les unités amies à moins de \distance{6+X} gagnent la règle spéciale \regeneration{6} jusqu'à la fin du tour de joueur suivant. Placez un marqueur à côté des unités affectées par la \regeneration{} après avoir déterminé l'aire d'effet. Une unité ayant déjà la règle spéciale \regeneration{} améliore sa sauvegarde de \regeneration{} d'un point pour obtenir 4+ au mieux.
		\item Toutes les unités ennemies à moins de \distance{12} subissent 1D6 touches de Force X.
	\end{itemize}
}

\newcommand{\extendedchassisrule}{%
La figurine gagne +1 Point de Vie et sa taille de socle change en \unit{50x150}{\milli\meter}.
}

\newcommand{\soulsyphonrule}{%
Pour connaître les effets de cette règle spéciale, le joueur doit tenir le décompte des blessures infligées par la figurine pendant la partie. À la fin de chaque Phase de Corps à Corps, observez le nombre total de blessures infligées pour déterminer le niveau de \soulsyphon{} qu'obtient la figurine. Le \darkcoach{} accumule toutes les améliorations jusqu'à son niveau actuel de \soulsyphon{}.
}

\newcommand{\soulsyphonchart}{%
\vspace*{0.1cm}
\renewcommand{\arraystretch}{1.5}
\begin{center}\begin{tabular}{M{2.5cm}m{10cm}}
\hline
\textbf{Blessures causées} & \centering\textbf{Bonus} \tabularnewline

\textbf{1 - 3}		   & \textIT{L'air est empli de pulsations meurtrières}. Le \darkcoach{} gagne \multiplewounds{2}{\infantry{}, \cavalry{}, \warbeast{}} et \lethalstrike{}. \tabularnewline

\textbf{4 - 6}		   & \textIT{La nuit s'illumine de feux impurs}. Le \darkcoach{} gagne \grindingattacks{1D3} (résolues à l'Initiative du \coachman{}) et \flamingattacks{}. \tabularnewline

\textbf{7 - 9}		   & \textIT{Un mal ancien se réveille!} Le \darkcoach{} ajoute le \awakenedvampire{} à son équipage. Il possède \vampiric{2}. \tabularnewline

\textbf{10 - 12}	   & \textIT{Un vent de terreur souffle dans la nuit et des ombres menaçantes percent dans le ciel}. Le \darkcoach{} gagne \fly{8}. \tabularnewline

\textbf{13+}		   & \textIT{Terrorisant}. Le \darkcoach{} devient \ethereal{}. \tabularnewline
\hline
\end{tabular}\end{center}
}

\newcommand{\greatmonstrousrevenantrule}{%
Le \monstrousrevenant{} gagne \thunderouscharge{} et sa taille de socle change en \unit{60x100}{\milli\meter}.
}

\newcommand{\colossalzombiedragonrule}{%
La figurine gagne +1 en Capacité de Combat, la valeur de sa \innatedefence{} passe à ($ 3+ $), et sa taille de socle change en \unit{100x150}{\milli\meter}.
}

% QRS

\newcommand{\QRSnote}{%
\noindent$^{1}$ En Monture, le Personnage remplace le \cadavermaster{}.

\noindent$^{2}$ En Monture, le Personnage remplace une \paramour{}.
}





\begin{document}

\newgeometry{margin=1in}

% Table options
\arrayrulecolor{black!30}
\setlength{\arrayrulewidth}{0.5pt}
\renewcommand{\arraystretch}{1.2}

\begin{titlepage}
\begin{center}

\ifdef{\booktitle}{}{\newcommand{\booktitle}{Missing title}}
\ifdef{\version}{}{\newcommand{\version}{Missing version}}

{\antiquefont\fontsize{40}{48}\selectfont\noindent\labels@fantasybattles

\labels@NinthAge}

\vspace*{0.5cm}
\ifdef{\booklogo}{%
\includegraphics[height=10cm]{\booklogo}%
}{%
\includegraphics[height=10cm]{../Layout/pics/logo_9th.png}%
}

\vspace*{-1cm}
{\antiquefont\fontsize{50}{60}\selectfont \booktitle
\vspace{0.4cm}

\fontsize{14}{16.8}\selectfont \labels@armyrules{}

Beta v\version{} - \today{}}

\ifdef{\frenchversion}{{\fontsize{14}{16.8}\selectfont \vspace{0.2cm}\noindent\texttt{VF \frenchversion}}}{}
\vfill

\begin{tabular}{@{}m{2cm}@{\hskip 20pt}m{13cm}@{}}
\includegraphics[width=2cm]{../Layout/pics/seal_9th.png} &
{\fontsize{10}{12}\selectfont \textcolor{black!50}{\noindent\labels@frontpagecredits}}

\ifdef{\frontpageaddstuff}{{\fontsize{10}{12}\selectfont \noindent\textcolor{black!50}{\frontpageaddstuff}}}{}

\vspace*{10pt}
\noindent{\fontsize{10}{12}\selectfont \textcolor{black!50}{\labels@license}}
\tabularnewline
\end{tabular}


\end{center}

\newpage

\thispagestyle{empty}

{\fontsize{10}{12}\selectfont

\begin{center}\noindent{\Largerfontsize\textbf{\labels@tableofcontents}}\end{center}

\vspace*{0.2cm}\begin{multicols}{2}

\tocfirstcolumn

\vspace*{\fill}\columnbreak

\tocentry{lordtitle}{\labels@lords}

\tocentry{herotitle}{\labels@heroes}

\ifdef{\tocmounts}{\tocentry{mountstitle}{\tocmounts}}{}

\tocentry{coretitle}{\labels@coreunits}

\tocentry{specialtitle}{\labels@specialunits}

\tocentry{raretitle}{\labels@rareunits}

\vspace*{\fill}\end{multicols}

\ifdef{\labels@introduction}{\vspace{0.7cm}\labels@introduction}{\vphantom{1pt}}
\vfill

\noindent\newrule{\labels@rulechanges}

\bigskip
\noindent \labels@latexcredit
}


\end{titlepage}

\restoregeometry

\startarmywiderules

\armywideruleentry{\masterofundeath}

Un Personnage doit être le \textbf{\master{}} de l'armée. Au début de la partie, le Général est toujours le \master{}.

\armywideruleentry{\invocation}

Les profils d'unité contiennent une catégorie appelée \invocation{} qui donne le nombre de Points de Vie Ressuscités grâce à l'\necromancysignaturespell{} de la Discipline \necromancy{}.

\closearmywiderules

\vspace*{1.5cm}
\startarmyspecialrules

\armyspecialruleentry{\ashestoashes}

À la fin de n'importe quelle phase durant laquelle le \master{} est retiré du jeu en tant que perte, toute unité de l'armée qui possède au moins une figurine avec la règle \ashestoashes{} doit faire un test de Commandement. Si le test échoue, l'unité subit un nombre de blessures équivalent à la différence entre le résultat obtenu et la valeur de Commandement du test, sans aucune sauvegarde autorisée. Ces blessures sont réparties comme pour la règle \unstable{}, mais ne peuvent pas être assignées à une figurine n'ayant pas la règle \ashestoashes{}. Le montant de blessures est réduit de un si l'unité est à portée de \holdyourground{}.

À la fin du Tour de Joueur suivant la mort du \master{}, un nouveau \master{} peut être choisi. Pour faire ceci, vous devez nommer un autre Personnage éligible, c'est à dire un \wizard{} utilisant la Discipline \necromancy{}. Ce Personnage est le nouveau \master{}.

Au début de chacun de vos Tours de Joueur sans nouveau \master{}, toute unité qui possède au moins une figurine avec la règle \ashestoashes{} doit faire un nouveau test de Commandement, et subir des blessures comme décrit ci-dessus.

\armyspecialruleentry{\wailofwoe}

Attaque Spéciale. Les éléments de figurine possédant cette règle peuvent effectuer une Attaque Spéciale de Tir. Elle peut être faite après une Marche Forcée et touche automatiquement avec le profil suivant : \range{8}, Force 4, \magicalattacks{}, \multipleshots{1D6+2}.


\armyspecialruleentry{\awaken{X}}

Une figurine avec cette règle spéciale peut Ressusciter des PVs au-delà de l'effectif de départ de toutes les unités mentionnées entre parenthèses, en suivant les modalités de \raisewounds{}. L'effectif de départ est le nombre de figurines choisi sur la liste d'armée. La taille des unités ne peut pas être augmentée au delà de deux fois l'effectif de départ.

\newpage
\armyspecialruleentry{\reaper}

Une unité composée uniquement de figurines possédant cette règle peut ignorer les décors et unités entre points de départ et d'arrivée durant l'étape des Autres Mouvements, mais doit respecter la règle des \distance{1} à la fin du déplacement. L'unité peut effectuer une \sweepingattack{}, à l'exception qu'elle est considérée comme une attaque spéciale de Corps à Corps plutôt qu'une attaque spéciale de Tir. L'ennemi subit une touche par figurine dans l'unité possédant la règle \reaper{}. Ces touches comptent comme si elles avaient été infligées au corps à corps, avec la Force, les règles spéciales et les bonus d'arme des figurines.

\armyspecialruleentry{\vampiric{X}}

Une unité \vampiric{} peut effectuer des Marches Forcées même lorsqu'elle se trouve hors de portée de la \inspiringpresence{} du Général. Elle doit toujours effectuer un test de Commandement si elle se trouve dans un rayon de \distance{8} d'une unité ennemie.

À la fin de chaque Phase de Corps à Corps, les unités avec cette règle spéciale peuvent faire des jets pour la règle \vampiric{}. Lancez 1D6 pour chaque Personnage \vampiric{} qui a infligé au moins une blessure non sauvegardée durant cette Phase de Corps à Corps, et 1D6 si au moins une figurine ordinaire \vampiric{} de l'unité l'a fait. Un jet \vampiric{} est réussi si le résultat du dé est de X ou plus, où X est le nombre entre parenthèses. Un résultat de \result{1} est toujours un échec, tandis qu'un \result{6} est toujours un succès. Les figurines avec la règle \largetarget{} ont un malus de -2 sur le résultat des dés de leur jet \vampiric{}, jusqu'à un minimum de 1. Un Personnage qui réussit un jet \vampiric{} Récupère un Point de Vie. Une unité qui a réussi un jet \vampiric{} provenant de ses figurines ordinaires Ressuscite un unique PV dans l'unité.

\armyspecialruleentry{\necromanticaura}

Toutes les unités amies dans un rayon de \distance{6} d'une ou plusieurs figurines avec cette règle spéciale réduisent le nombre de blessures qu'elle subissent par les règles \ashestoashes{} et \unstable{} de 1. Les figurines avec la règle \necromanticaura{} ne peuvent pas en bénéficier elles-mêmes.


\closearmyspecialrules


\startarmynewsection{Dynasties Vampiriques}

\spaceaftersection{}

Une armée des Conclaves Vampiriques peut choisir de représenter une unique Dynastie Vampirique. Dans ce cas, tous les \vampirelords{} et \vampireheroes{} de l'armée doivent prendre l'amélioration correspondant à la Dynastie.

\armynewsubsection{Pouvoir Dynastique Ancien}

Les \vampirelords{} appartenant à une \bloodline{} peuvent accéder au \ancientbloodpower{} de leur \bloodline{} plutôt que de prendre un \bloodpower{}. Chaque \ancientbloodpower{} est \oneofakind{}.

\armynewsubsection{\bloodties{X}}

Certaines troupes du Livre d'Armée sont notées comme étant des \bloodties{}, suivies entre parenthèses de la \bloodline{} à laquelle elles appartiennent. Si la \bloodline{} des Personnages \vampires{} de l'armée correspond à celle entre parenthèses, vous pouvez prendre l'option décrite à la suite.

\armynewsubsection{\brotherhood{}\dotfill\pts{30/10}}

\begin{wrapfigure}[5]{L}{0cm}
\centering
\includegraphics[width=\logosize]{logos/logo_brotherhood.png}
\end{wrapfigure}
Un Vampire de la \brotherhood{} gagne +2 en Capacité de Combat et porte une \platearmour{}. Il ne peut acheter qu'un seul Niveau de Magie supplémentaire et ne peut utiliser que la Discipline \necromancy{}. Un Vampire de la \brotherhood{} ne peut jamais refuser de Défi et doit en lancer dès qu'il le peut, à moins qu'une autre figurine ne le fasse en premier.

\vspace{0.5cm}
\bloodtie{} : \textbf{\vampireknights{}}.

\vspace{0.5cm}
\ancientbloodpower{} : \textbf{Rage Écarlate}\dotfill\pts{65}\newline%
Chaque blessure non sauvegardée infligée par des attaques normales, avant d'appliquer les \multiplewounds{}{} du Vampire autorise une nouvelle attaque à la même Initiative. Faites ces attaques avant de retirer les pertes. Ces attaques additionnelles ne génèrent pas d'autres attaques.

\armynewsubsection{\bloodline{} \vonkarnstein{}\dotfill\pts{25/10}}

\begin{wrapfigure}[5]{R}{0cm}
\centering
\includegraphics[width=\logosize]{logos/logo_vonkarnstein.png}
\end{wrapfigure}
La présence d'un ou plusieurs Vampires \vonkarnstein{} dans un corps à corps octroie un bonus de +1 au Résultat de Combat. Une unité avec la règle \undead{} rejointe par un \vonkarnstein{} peut effectuer une Marche Forcée comme si elle était \vampiric{}. La portée des éventuelles \inspiringpresence{} et \holdyourground{} du Vampire est augmentée de \distance{6} et il peut relancer ses jets ratés pour la règle \vampiric{}. 

\vspace{0.5cm}
\bloodties{} : \textbf{\darkcoach{}}.

\vspace{0.5cm}
\ancientbloodpower{} : \textbf{Appel de la Tempête}\dotfill\pts{65}\newline%
Le Vampire possède le sort \heavensspelltwo{} (Discipline \heavens{}) comme \boundspell{4}. Toutes les unités dans un rayon de \distance{12} autour du Vampire gagnent la règle spéciale \hardtarget{}. De plus, une fois par partie, le Vampire peut donner à toutes les figurines ordinaires de son unité et lui-même les règles spéciales \lightningattacks{} et \lightningreflexes{}. Cette capacité doit être activée au début d'une Phase de Corps à Corps et dure jusqu'à la fin du tour de joueur suivant.

\newpage
\armynewsubsection{\bloodline{} \lamia{}\dotfill\pts{40/25}}

\begin{wrapfigure}[7]{L}{0cm}
\centering
\includegraphics[width=\logosize]{logos/logo_lamia.png}
\end{wrapfigure}
La Vampire perd une Attaque et obtient les \lightningreflexes{}. Si elle ne porte aucune Armure, sans compter la \mountsprotection{} et la \innatedefence{}, elle obtient également la règle \distracting{}. Les Défis lancés par la Vampire doivent être acceptés si possible et les figurines en Défi contre une \lamia{} doivent réussir un test de Commandement avec un malus additionnel de -1. Dans le cas d'un échec, elles doivent relancer leurs jets pour toucher réussis pendant cette Manche de Corps à Corps. La Vampire doit choisir la Discipline \light{}, \shadows{} ou \necromancy{}.

\vspace{0.5cm}
\bloodties{} : \textbf{\courtofthedamned{}}.

\vspace{0.5cm}
\ancientbloodpower{} : \textbf{Commandement}\dotfill\pts{50}\newline%
Toute les figurines ordinaires de l'unité rejointe par la Vampire obtiennent une Capacité de Combat de 5. Si la Vampire n'est pas engagée au corps à corps, elle peut à la place, au début d'une Phase de Corps à Corps et pour la durée de cette phase, donner ce bonus à une unique unité amie située dans un rayon de \distance{6}. L'effet dure jusqu'à la fin de la Phase de Corps à Corps.

\armynewsubsection{\bloodline{} des \strigois{}\dotfill\pts{50/30}}

\begin{wrapfigure}[5]{R}{0cm}
\centering
\includegraphics[width=\logosize]{logos/logo_strigoi.png}
\end{wrapfigure}
Le Vampire et sa monture gagnent +1 Point de Vie, obtiennent une \regeneration{5} et la \hatred{}. Le Vampire ne peut pas prendre de monture à l'exception d'une \shriekinghorror{}, ne peut porter aucune Armure, sauf une \mountsprotection{}, et ne peut acheter qu'un seul Niveau de Magie supplémentaire. Il doit utiliser la Discipline \wilderness{} ou \necromancy{}.

\vspace{0.5cm}
\bloodties{} : \textbf{\ghouls{}}.

\vspace{0.5cm}
\ancientbloodpower{} : \textbf{Roi des Goules}\dotfill\pts{65}\newline%
Le Vampire et sa monture font des \poisonedattacks{} et gagnent \armourpiercing{1}. Les \ghouls{} de l'unité qu'il rejoint gagnent la \hatred{} et \armourpiercing{1}.

\armynewsubsection{\bloodline{} \nosferatu{}\dotfill\pts{120/60}}

\begin{wrapfigure}[7]{L}{0cm}
\centering
\includegraphics[width=\logosize]{logos/logo_nosferatu.png}
\end{wrapfigure}
Le Vampire perd une Attaque, subit un malus de -2 en Capacité de Combat et ne peut porter aucune Armure, à l'exception de la \mountsprotection{} et de la \innatedefence{}. Il ne peut pas porter d'Arme standard. Il devient un \magiclevel{2} si c'est un Héros ; un \magiclevel{4} si c'est un Seigneur. Il génère un sort supplémentaire et gagne \awaken{\skeletons{}, \zombies{}}. Un Vampire \nosferatu{} peut choisir ses sorts dans toutes les Disciplines qui lui sont ouvertes au lieu d'une seule. Les informations sur les Disciplines choisies et combien de sorts sont générés dans chaque Discipline doivent figurer dans la Liste d'Armée.

\vspace{0.5cm}
\bloodties{} : \textbf{\wraiths{}}.

\vspace{0.5cm}
\ancientbloodpower{} : \textbf{Occultisme du Sang}\dotfill\pts{75}\newline%
Le Vampire compte toujours comme ayant utilisé un Dé de Pouvoir de moins en cas de Fiasco. Immédiatement après avoir lancé les dés pour déterminer la force des Flux Magiques pendant votre tour, vous pouvez choisir de relancer un des dés. Dans ce cas, le Vampire subit une blessure sans aucune sauvegarde possible à la fin de la Phase de Magie.

\closearmynewsection

\newpage
\startarmynewsection{Pouvoirs Dynastiques}

\spaceaftersection{}

Les \vampirelords{} et les \vampireheroes{} peuvent  acheter une unique amélioration appelé Pouvoir Dynastique Vampirique. Dans le cas d'une armée Indépendante, c'est à dire sans Dynastie, tous les Pouvoirs Dynastiques sont de type \oneofakind{}. Si votre armée représente une Dynastie, seuls les Pouvoirs de cette Dynastie peuvent être pris par vos Vampires, mais ils peuvent être dupliqués.

\begin{multicols}{2}\raggedcolumns

\begin{center}\includegraphics[width=2cm]{logos/logo_brotherhood.png}\end{center}
\vspace*{-1.2cm}
\armynewsubsection{\begin{center}Indépendant ou \brotherhood{}\end{center}}

\startpricelist

\pricelistitem{Combattant Ultime}{35} Le Vampire obtient les règles \weaponmaster{} et \lethalstrike{}. Il est équipé d'une \pw{}, d'une \halberd{}, d'une \gw{}, d'une \lance{} et d'un \shield{}.

\pricelistitem{Fine Lame}{30} Lorsqu'il combat en défi, le Vampire peut relancer ses jets pour toucher et pour blesser ratés.

\endpricelist

\begin{center}\includegraphics[width=2cm]{logos/logo_vonkarnstein.png}\end{center}
\vspace*{-1.2cm}
\armynewsubsection{\begin{center}Indépendant ou \vonkarnstein{}\end{center}}

\startpricelist

\pricelistitem{Goût Raffiné}{25} Le Vampire obtient \vampiric{2}.

\pricelistitem{L'Heure du Loup}{20} Le Vampire gagne \awaken{\zombies, \direwolves, \batswarms, \greatbats}. Le Vampire ainsi que les figurines de l'unité qu'il rejoint gagnent la règle \swiftstride{}, à l'exception des autres Personnages ayant la règle \vampiric{}.

\endpricelist

\begin{center}\includegraphics[width=2cm]{logos/logo_lamia.png}\end{center}
\vspace*{-1.2cm}
\armynewsubsection{\begin{center}Indépendant ou \lamia{}\end{center}}

\startpricelist

\pricelistitem{Regard Hypnotique}{35} Le Vampire possède le sort \lustspellfour{} (Discipline \lust{}) comme \boundspell{4}.

\pricelistitem{Masque d'Innocence}{25} Les unités au contact d'un ou plusieurs Vampires avec ce pouvoir subissent un malus de -1 en Commandement.

\endpricelist


\begin{center}\includegraphics[width=2cm]{logos/logo_strigoi.png}\end{center}
\vspace*{-1.4cm} % Manual adjustment, change it for your own translation
\armynewsubsection{\begin{center}Indépendant ou \strigoi{}\end{center}}
\vspace*{-0.15cm} % Manual adjustment, change it for your own translation
\startpricelist

\pricelistitem{Malédiction du Sang}{70} Le Vampire gagne \regeneration{5}. S'il possédait déjà une \regeneration{}, il gagne \regeneration{4}. Toutes les \ghouls{} dans la même unité que le Vampire ainsi que son éventuelle monture gagnent \regeneration{6}. Si une figurine affectée par cette capacité possède déjà la \regeneration{}, sa sauvegarde de \regeneration{} est améliorée d'un point, pour obtenir 4+ au mieux.

\pricelistitem{Forme Bestiale}{55} Figurines à pied uniquement. Le type de troupe du Vampire devient \monstrousinfantry{} et la taille de son socle passe à \unit{40x40}{\milli\meter}. Il gagne +1 PV, +1 en Force, \regeneration{5} et son Endurance passe à 5. Il est équipé d'une \pw{} et ne peut manier aucune autre Arme, standard ou magique. Il n'a accès à aucune monture et ne peut porter aucune sorte d'Armure.

\endpricelist

\begin{center}\includegraphics[width=2cm]{logos/logo_nosferatu.png}\end{center}
\vspace*{-1.2cm}
\armynewsubsection{\begin{center}Indépendant ou \nosferatu{}\end{center}}

\startpricelist

\pricelistitem{Connaissance des Arcanes}{25} Les sorts lancés par le Vampire ont une Portée augmentée de \distance{6}. Cet effet est réduit à \distance{3} pour les sorts de type Aura. Les sorts de Vortex, les Objets de Sort et les sorts sans Portée ne sont pas affectés.

\pricelistitem{Discipline Interdite}{20} Sélectionnez une Discipline \battle{} autre que celle \nature{}. Le Vampire peut également choisir cette Discipline pour générer ses sorts en plus de celles normalement accessibles.

\endpricelist

\end{multicols}

\closearmynewsection

\startarmymagicalitems

\armymagicalweapons

\startpricelist

\pricelistitem{Lame de la Soif Rouge}{40} Vampires uniquement.\newline Type: \hw{}. Le Vampire maniant cette lame gagne \vampiric{3}. L'élément de figurine lance 1D6 pour chaque blessure qu'il inflige dans le cadre de sa règle \vampiric{}, au lieu d'un seul dé. Chaque jet \vampiric{} réussi lui rend un PV, puis tout excès de PV ainsi Ressuscité peut être utilisé pour \raisewounds{} dans l'unité que le personnage a rejointe.

\endpricelist

\armymagicalarmour

\startpricelist

\pricelistitem{Haubert Sanglant de Gilles de Raux}{40} Type : \platearmour{}. Le porteur gagne +1 Point de Vie.

\endpricelist

\armytalismans

\startpricelist

\pricelistitem{Anneau d'Éternité}{60/50} Vampires uniquement.\newline Le porteur possède une \wardsave{2} contre la première blessure subie (après les Sauvegardes d'Armure) de la partie. Il est immunisé aux effets du \lethalstrike{} et des \multiplewounds{}{}.

\pricelistitem{Linceul de Nuit}{40} Figurine à pied uniquement.\newline Les figurines au contact du porteur ainsi que toute figurine allouant des attaques de Corps à Corps au porteur ne reçoivent aucun bonus de type +X en Force lié à leurs Armes, qu'elles soient magiques ou ordinaires.

\endpricelist

\armyenchanteditems

\startpricelist

\pricelistitem{Dents de Tullius}{50} Le porteur et toutes les figurines ordinaires de son unité gagnent la règle \distracting{}.

\endpricelist

\armyarcaneitems

\startpricelist

\pricelistitem{Baguette de Gerhard Langue Noire}{50} Une armée possédant cet objet peut relancer ses jets de Canalisation ratés. De plus, quand le porteur lance le sort \necromancysignaturespell{} (Discipline \necromancy{}), le dé pour déterminer le nombre de Points de Vie Ressuscités peut être relancé pour chaque unité ciblée.

\pricelistitem{Tome Impie}{35} \boundspell{4}. Cet objet permet de lancer \necromancyspelltwo{} (Discipline \necromancy{}).

\pricelistitem{Œil de Setesh}{20} À la fin de n'importe quelle Phase de Magie, vous pouvez mettre un Dé de Magie inutilisé de côté pour l'ajouter à votre réserve du tour suivant, juste après avoir déterminé les Flux de Magie.

\endpricelist

\armymagicalbanners

\startpricelist

\pricelistitem{Étendard Noir de Zagvozd}{40} Toutes les figurines de l'unité du porteur obtiennent une \wardsave{4} contre les Attaques de Tir. 

\pricelistitem{Bannière des Rois des Tertres}{25} Les figurines de \barrowknights{} et de \barrowguards{} de l'unité obtiennent +1 pour toucher au Corps à Corps.

\endpricelist

\closearmymagicalitems



%%% START OF THE ARMYLIST - Translators shouldn't have to edit it %%%


%%% v0.99.9

\armylist

\lordstitle

\showunit{
	name={\vampirelord},
	cost={205},
	profile={ < 6 7 5 5 5 3 7 5 10},
	type=\infantry{},
	basesize=20x20,
	unitsize=1,
	commontype=\vampiriccommonrules{},
	commonspecialrules={\undead,\fear,\vampiric{6}},
	specialrules={\awaken{\zombies}},
	magiclevel=1,
	magicpaths={\necromancy, \shadows, \death},
	options={
		\magiclevelchoice{
			\magiclevel{2}=25,
			\magiclevel{3}=90,
		},
		\magicalitemsallowance =\upto{}<100,
		\bloodlinechoice{}\refsymbol{}=\unlimited{},
		\onechoiceonly{
			\asinglebloodpower{}=\unlimited{},
			\asingleancientbloodpower{}\refsymbol{}=\unlimited{},
		},
		\shield{}=5,
		\la{}=5,
		\ha{}=10,
		\weapononechoice{
			\pw{}=10,
			\halberd{}=15,
			\gw{}=20,
			\lance{}=20,
		},
	},
	mounts={
		\skeletalsteed{}=20,
		\spectralsteed{}=55,
		\monstrousrevenant{}=90,
		\courtofthedamned{}=140,
		\shriekinghorror{} \only{\strigoi}=245,
		\zombiedragon{}=270,
	},
	additional={\refsymbol{} \bloodlinenote{}},
}

\showunit{
	name={\necromancerlord},
	cost={170},
	profile={ < 4 3 3 3 4 3 3 1 8},
	type= \infantry{},
	basesize=20x20,
	unitsize=1,
	commontype=\undeadcommonrules{},
	commonspecialrules={\undead},
	specialrules={\awaken{\zombies{}, \skeletons{}}},
	magiclevel=3,
	magicpaths={\necromancy{}, \fire{}, \death{}},
	options={
		\magiclevel{4}=30,
		\magicalitemsallowance{}=\upto{}<100,
	},
	mounts={
		\skeletalsteed{}=20,
		\cadaverwagon{}=50,
		\monstrousrevenant{}=90,
	},
}





\heroestitle

\showunit{
	name={\vampirehero},
	cost=75,
	profile={ < 6 6 4 5 4 2 6 4 8},
	type=\infantry{},
	basesize=20x20,
	unitsize=1,
	commontype=\vampiriccommonrules{},
	commonspecialrules={\undead{},\fear{},\vampiric{6}},
	specialrules={\awaken{\zombies}},
	magicpaths={\necromancy{}, \shadows{}, \death{}},
	options={
		\bsb{}=25,
		\magiclevelchoice{
			\magiclevel{1}=30,
			\magiclevel{2}=55,
		},
		\magicalitemsallowance{}=\upto{}<50,
		\bloodlinechoice{}\refsymbol{}=\unlimited{},
		\asinglebloodpower{}=\unlimited{},
		\shield{}=5,
		\la{}=5,
		\ha{}=10,
		\weapononechoice{
			\pw{}=5,
			\halberd{}=10,
			\gw{}=10,
			\lance{}=15,
		},
	},
	mounts={
		\skeletalsteed{}=15,
		\spectralsteed{}=55,
		\monstrousrevenant{}=110,
		\courtofthedamned{}=140,
	},
	additional={\refsymbol{} \bloodlinenote{}},
}

\showunit{
	name={\necromancer},
	cost={65},
	profile={< 4 3 3 3 3 2 3 1 7},
	type= \infantry{},
	basesize=20x20,
	unitsize=1,
	commontype=\undeadcommonrules{},
	commonspecialrules={\undead},
	specialrules={\awaken{\skeletons{}, \zombies{}}},
	magiclevel=1,
	magicpaths={\necromancy{}, \fire{}, \death{}},
	options={
		\magiclevel{2}=25,
		\magicalitemsallowance{}=\upto{}<50,
	},
	mounts={
		\skeletalsteed{}=15,
		\cadaverwagon{}=50,
	},
}

\showunit{
	name={\barrowking},
	cost=80,
	profile={< 4 5 - 4 5 3 4 3 9},
	type=\infantry{},
	basesize=20x20,
	unitsize=1,
	commontype=\undeadcommonrules{},
	commonspecialrules={\undead{},\ashestoashes{}},
	armour={\ha{}, \shield{}},
	options={
		\bsb{}=25,
		\magicalitemsallowance{}=\upto{}<50,
		\weapononechoice{
			\pw{}=3,
			\halberd{}=4,
			\gw{}=6,
			\lance{}=6,
		},
		\unlivingshield{}=20,
	},
	mounts={\skeletalsteed{}=15},
	additional={
		\def\tempspecialrules{\multiplewounds{2}{\infantry{}, \cavalry{}, \warbeast{}}, \lethalstrike{}, \notaleader{}, \magicalattacks{}}
		\def\tempunitrules{\unitrule{\unlivingshield}{\unlivingshieldrule}}
		
		\vspace*{-0.1cm}\specialrules{\tempspecialrules}
		\vspace*{0.2cm}\unitrules{\tempunitrules}
	},
}

\showunit{
	name={\fellwraith},
	profile={\fellwraith{}< 6 4 - 3 3 2 2 3 5,
		     \banshee{}< 6 3 - 3 3 2 3 1 5
	},
	type=\infantry{},
	basesize=20x20,
	unitsize=1,
	commontype=\undeadcommonrules{},
	commonspecialrules={\undead{}, \ashestoashes{}},
	specialrules={\ethereal{}, \reaper{}, \notaleader{}, \terror{}},
	additional={%
		\begin{center}\mustbecomeoneofthefollowingNOC{}\end{center}
		\setlength{\columnseprule}{0.5pt}
		\setlength{\columnsep}{1cm}
		\renewcommand{\columnseprulecolor}{\color{black!30}}
		\begin{multicols}{2}\raggedcolumns
		\begin{center}\largerfontsize\antiquefont\fellwraith{} (\pts{55})\end{center}%
		
		\def\tempspecialrules{\armourpiercing{6}}%
		\specialrules{\tempspecialrules}

		\def\tempoptions{%
			\magicalweaponallowance{}=\upto{}<50,
			\gw{}=10,		
		}%

		\def\tempmounts{\skeletalsteed{} \wordwith{} \lighttroops{}=20}%	
		
		\vspace*{0.2cm}\mounts{\tempmounts}\options{\tempoptions}	
		\columnbreak
		\begin{center}\largerfontsize\antiquefont\banshee{} (\pts{85})\end{center}

		\def\tempspecialrules{\wailofwoe}%
		\specialrules{\tempspecialrules}%
		\end{multicols}
		\setlength{\columnseprule}{0pt}
	},
}





\newpage
\toctarget{mountstitle}{\mountstitle}

\showunit{
	name={\skeletalsteed},
	profile={< 8 2 - 3 3 1 2 1 3,},
	type=\warbeast{},
	basesize=25x50,
	commontype=\undeadcommonrules{},
	commonspecialrules={\undead{}},
	specialrules={\ethereal{}},
	armour={\mountsprotection{6}},
	options={
		\mountsprotection{5}=15,
	}
}

\showunit{
	name={\spectralsteed},
	profile={< 8 2 - 3 3 1 2 1 3,},
	type=\warbeast{},
	basesize=25x50,
	commontype=\undeadcommonrules{},
	commonspecialrules={\undead},
	specialrules={\ethereal{}, \fly{8}},
	armour={\mountsprotection{6}},
}

\showunit{
	notinQRS=yes,
	name={\shriekinghorror},
	profile={< 6 4 - 5 6 6 2 4 4,},
	type=\monster{},
	basesize=100x150,
	commontype=\undeadcommonrules{},
	commonspecialrules={\undead},
	specialrules={\chillingshriek{} \seeshriekinghorrorinraresection{}, \regeneration{6}, \fly{8}, \discordantchorus{}},
	unitrules={\unitrule{\discordantchorus}{\discordantchorusrule}},
}

\showunit{
	name={\zombiedragon},
	profile={< 6  4 - 6 6 6 2 5 4,},
	type=\monster{},
	unitsize=SPECIAL-{\textbf{(\oneofakind)}},
	basesize=50x100,
	commontype=\undeadcommonrules{},
	commonspecialrules={\undead},
	specialrules={\breathweapon{\Strength{} 2, \armourpiercing{6}}, \distracting{}, \regeneration{6}, \fly{7}},
	armour={\innatedefence{4}},
	options={
		\colossalzombiedragon{}=20,
	},
	unitrules={
		\unitrule{\colossalzombiedragon}{\colossalzombiedragonrule}
	}
}

\showunit{
	name={\monstrousrevenant},
	profile={< 6  4 - 5 5 4 2 4 4,},
	type=\monstrousbeast{},
	basesize=50x50,
	commontype=\undeadcommonrules{},
	commonspecialrules={\undead},
	specialrules={\largetarget{},\fear{}},
	unitrules={\unitrule{\greatmonstrousrevenant}{\greatmonstrousrevenantrule}},
	additional={%
		\def\tempoptions{%
			\uptotwoofthefollowingTWOCOL{
				\poisonedattacks{}=5,
				\greatmonstrousrevenant{}=10,
				\lethalstrike{}=10,
				\wailofwoe{}=30,
				\randomattacks{1D6+2}=30,
				\fly{8}=40,
			},		
		}
		\vspace*{0.1cm}\options{\tempoptions}
	}
}

\showunit{
	notinQRS=yes,
	name={\cadaverwagon},
	profile={
		\cart{}< - - - 4 4 4 - - -,
		\shamblinghorde{}< 4 1 - 3 - -  1 \starsymbol{} 2	
	},
	type=\chariot{},
	basesize=50x100,
	commontype=\undeadcommonrules{},
	commonspecialrules={\undead},
	specialrules={\randomattacks{2D6} \only{\shamblinghorde}, \wakethedead{}, \cart{}, \regeneration{4}},
	armour={\mountsprotection{5}},
	options={
		\endlesshorde{}=10,
		\onechoiceonly{
			\bonepyre{}=15,
			\bringoutyourdead{}=20,
			\necromanticaura{}=20,
		},
	},
	additional={
		\vspace*{0.22cm}\seecadaverwagoninspecialsectionforupgdraderules{}
	},
}

\showunit{
	notinQRS=yes,
	name={\courtofthedamned},
	profile={
		\floatingcourt{}	< - - - 5 5 5 - - -,
       		\paramour{} (2) 	< - 5 5 5 - - 6 2 7,
		\ghoststeeds{}	< 8 2 - 3 - - 2 \starsymbol{} 3,
	},
	type=\chariot{},
	basesize=50x100,
	commontype=\vampiriccommonrules{},
	commonspecialrules={\undead{}, \vampiric{6}, \fear{}},
	armour={\innatedefence{5}},
	options={
		\unholydominion{}=40,
		\textbf{\bloodtie{\lamia}}\spacebeforecolon{}:\newline\wardsave{4}=55,
	},
	additional={%
		\def\tempspecialrules{\randomattacks{2D6} \only{\steeds}, \ethereal{} \only{\steeds}, \largetarget}
		\def\tempunitrules{\unitrule{\unholydominion}{\unholydominionrule}}
		
		\vspace*{0.15cm}\specialrules{\tempspecialrules}
		\vspace*{0.2cm}\unitrules{\tempunitrules}
	},
}








\coreunitstitle

\showunit{
	name={\zombies},
	QRSname={\zombie},
	cost=55,
	profile={ < 4 1 - 3 3 1 1 1 2},
	invocation={2D6+3},
	type=\infantry{},
	basesize=20x20,
	unitsize=20,
	maxmodels=60,
	costpermodel=3,
	commontype=\undeadcommonrules{},
	commonspecialrules={\undead{}, \ashestoashes{}},
	commandgroup={musician=10, banner=10}
}

\showunit{
	name={\skeletons},
	QRSname={\skeleton},
	cost=80,
	profile={< 4 2 2 3 3 1 2 1 4},
	invocation={1D6+3},
	type=\infantry{},
	basesize=20x20,
	unitsize=20,
	maxmodels=60,
	costpermodel=5,
	commontype=\undeadcommonrules{},
	commonspecialrules={\undead{}, \ashestoashes},
	armour={\la{}, \shield},
	options={
		\onechoiceonly{
			\spear{}=\free{},
			\halberd{}=\permodel{}<1,
		},
	},
	commandgroup={champion=10, musician=10, banner=10, veteranstandardbearer=yessir},
}

\showunit{
	name={\ghouls},
	QRSname={\ghoul},
	cost=65,
	profile={< 4 3 - 3 4 1 4 2 6},
	invocation={1D6+3},
	type= \infantry{},
	basesize=20x20,
	unitsize=10,
	maxmodels=40,
	costpermodel=10,
	commontype=\undeadcommonrules{},
	commonspecialrules={\undead{}, \ashestoashes},
	specialrules={\poisonedattacks},
	options={
		\skirmishers{} \ifNmodelsorless{15}=25,
		\textbf{\bloodtie{\strigoi}}\spacebeforecolon{}:\newline
		\vanguard{}\refsymbol{}=\permodel{}<2,
	},
	commandgroup={champion=10, musician=10, banner=10},
	additional={%
		\vspace*{0.1cm}\refsymbol{} \strigoivanguardnote{}
	}
}

\showunit{
	name={\direwolves},
	QRSname={\direwolf},
	cost=40,
	profile={< 9 3 - 3 3 1 3 1 3},
	invocation={1D3+3},
	type=\warbeast{},
	basesize=25x50,
	unitsize=5,
	maxmodels=15,
	costpermodel=6,
	commontype=\undeadcommonrules{},
	commonspecialrules={\undead{}, \ashestoashes},
	specialrules={\vanguard{}, \thunderouscharge},
	commandgroup={champion=10}
}

\showunit{
	name={\batswarms},
	QRSname={\batswarm},
	cost=60,
	profile={< 1 3 - 2 2 4 3 4 3},
	invocation={1D6+3},
	type=\swarm{},
	basesize=40x40,
	unitsize=2,
	maxmodels=10,
	costpermodel=20,
	commontype=\undeadcommonrules{},
	commonspecialrules={\undead{}, \ashestoashes},
	specialrules={\fly{6}, \stormofwings{}, \distracting{}},
	unitrules={
		\unitrule{\stormofwings}{\stormofwingsrule}
	}
}







\specialunitstitle

\showunit{
	name={\barrowguards},
	QRSname={\barrowguard},
	cost=60,
	profile={< 4 3 - 4 4 1 3 1 7},
	invocation=1D3+3,
	type=\infantry{},
	basesize=20x20,
	unitsize=10,
	maxmodels=40,
	costpermodel=10,
	commontype=\undeadcommonrules{},
	commonspecialrules={\undead{}, \ashestoashes},
	specialrules={\magicalattacks{}, \multiplewounds{2}{\infantry{}, \cavalry{}, \warbeast}, \lethalstrike{}, \bodyguard{\general, \barrowking}},
	armour={\ha},
	options={
		\onechoiceonly{
			\shield{}=\permodel{}<1,
			\halberd{}=\permodel{}<2,
			\gw{}=\permodel{}<3,
		},
	},
	commandgroup={champion=10, musician=10, banner=10, bannerallowance=50}
}

\showunit{
	name={\barrowknights},
	QRSname={\barrowknight},
	cost=90,
	profile={\rider{}< 4 3 - 4 4 1 3 1 7,
		\skeletalsteed{}< 8 2 - 3 3 1 2 1 3},
	invocation={1D3+1},
	type=\cavalry{},
	basesize=25x50,
	unitsize=5,
	maxmodels=12,
	costpermodel=26,
	commontype=\undeadcommonrules{},
	commonspecialrules={\undead{}, \ashestoashes},
	armour={\ha{}, \shield{}, \mountsprotection{5}},
	weapons={\lance},
	commandgroup={champion=10, musician=10, banner=10, bannerallowance=50},
	additional={%
		\def\tempspecialrules{\magicalattacks{} \only{\rider}, \multiplewounds{2}{\infantry{}, \cavalry{}, \warbeast} \only{\rider}, \lethalstrike{} \only{\rider}, \ethereal{} \only{\steed}}
		\vspace*{0.22cm}\specialrules{\tempspecialrules}
	},
}

\showunit{
	name={\ghasts},
	QRSname={\ghast},
	cost=100,
	profile={< 6 3 - 4 5 3 2 3 5},
	invocation=2,
	type=\monstrousinfantry{},
	basesize=40x40,
	unitsize=3,
	maxmodels=10,
	costpermodel=47,
	commontype=\undeadcommonrules{},
	commonspecialrules={\undead{}, \ashestoashes},
	specialrules={\poisonedattacks{}, \fear{}, \regeneration{5}},
	commandgroup={champion=10}
}

\showunit{
	name={\vampirespawn},
	cost=120,
	profile={< 6 4 - 5 4 3 4 3 8},
	invocation=2,
	type=\monstrousinfantry{},
	basesize=40x40,
	unitsize=3,
	maxmodels=8,
	costpermodel=45,
	commontype=\vampiriccommonrules{},
	commonspecialrules={\undead{}, \vampiric{6}, \fear},
	specialrules={\frenzy{}, \fly{9}},
	commandgroup={champion=10},
	options={
		\skirmisher{} \ifNmodelsorless{4}=\permodel{}<3,
	}
}

\showunit{
	name={\phantomhost},
	cost=60,
	profile={< 6 3 - 3 3 4 1 4 4},
	invocation=1D3,
	type=\infantry{},
	basesize=40x40,
	unitsize=2,
	maxmodels=6,
	costpermodel=25,
	commontype=\undeadcommonrules{},
	commonspecialrules={\undead{}, \ashestoashes},
	specialrules={\ethereal{}, \fear{}, \armourpiercing{1}}
}

\showunit{
	name={\greatbats},
	QRSname={\greatbat},
	cost=40,
	profile={< 1 3 - 3 3 2 3 2 3},
	invocation=1D3+3,
	type=\warbeast{},
	basesize=40x40,
	unitsize=2,
	maxmodels=9,
	costpermodel=14,
	commontype=\undeadcommonrules{},
	commonspecialrules={\undead{}, \ashestoashes},
	specialrules={\skirmisher{}, \fly{10}},
}

\showunit{
	name={\varkolak},
	cost=175,
	profile={< 8 5 - 6 5 4 4 5 7},
	invocation=1,
	type=\monstrousbeast{},
	basesize=50x50,
	unitsize=1,
	commontype=\vampiriccommonrules{},
	commonspecialrules={\undead{}, \vampiric{5}, \fear},
	specialrules={\hatred{}, \regeneration{4}},
	options={
		\onechoiceonly{
			\vampiric{3}=5,
			\stomp{1D3+1}=10,
			\vanguard{}=20,
			\fly{8}=20,
		},
	},
}

\showunit{
	name={\cadaverwagon},
	QRSname={\cadaverwagon{}$^{1}$},
	cost=80,
	profile={
		\cart{}< - - - 4 4 4 - - -,
   		\cadavermaster{}< - 3 - 3 - -  3 1 5,
		\shamblinghorde{}< 4 1 - 3 - -  1 \starsymbol{} 2	
	},
	invocation=1,
	type=\chariot{},
	basesize=50x100,
	unitsize=1,
	commontype=\undeadcommonrules{},
	commonspecialrules={\undead{}, \ashestoashes},
	specialrules={\randomattacks{2D6} \only{\shamblinghorde}, \wakethedead{}, \cart{}, \regeneration{4}},
	armour={\mountsprotection{5}},
	options={
		\endlesshorde{}=10,
		\onechoiceonly{
			\bonepyre{}=15,
			\bringoutyourdead{}=20,
			\necromanticaura{}=20,
		},
	},
	additional={%
		\def\tempunitrules{%
			\unitrule{\cart}{\cartrule}
			\unitrule{\wakethedead}{\wakethedeadrule}
			\unitrule{\endlesshorde}{\endlesshorderule}
			\unitrule{\bonepyre}{\bonepyrerule}
			\unitrule{\bringoutyourdead}{\bringoutyourdeadrule}		
		}
		\vspace*{0.2cm}\unitrules{\tempunitrules}
	},
}

\showunit{
	name={\courtofthedamned},
	QRSname={\courtofthedamned{}$^{2}$},
	cost=140,
	profile={
		\floatingcourt{}< - - - 5 5 5 - - -,
       		\paramour{} (3)< - 5 5 5 - - 6 2 7,
		\ghoststeeds{}< 8 2 - 3 - - 2 \starsymbol{} 3
	},
	invocation=1,
	type=\chariot{},
	basesize=50x100,
	unitsize=1,
	commontype=\vampiriccommonrules{},
	commonspecialrules={\undead{}, \vampiric{6}, \fear{}},
	armour={\innatedefence{5}},
	options={
		\unholydominion{}=40,
		\textbf{\bloodtie{\lamia}}\spacebeforecolon{}:\newline
		\wardsave{4}=55,
	},
	additional={%
		\def\tempspecialrules{\randomattacks{2D6} \only{\steeds}, \ethereal{} \only{\steeds}, \largetarget}
		\def\tempunitrules{\unitrule{\unholydominion}{\unholydominionrule}}
		
		\vspace*{0.15cm}\specialrules{\tempspecialrules}
		\vspace*{0.2cm}\unitrules{\tempunitrules}
	},
}





\rareunitstitle

\showunit{
	name={\wraiths},
	QRSname={\wraith},
	cost=75,
	profile={
		\wraithprofile{}< 6 3 - 3 3 1 2 2 5,
		[\ghoststeed{}]< 8 2 - 3 3 1 2 1 3
	},
	invocation=1,
	type=\infantry{},
	basesize=20x20,
	unitsize=5,
	maxmodels=10,
	costpermodel=20,
	commontype=\undeadcommonrules{},
	commonspecialrules={\undead{}, \ashestoashes},
	specialrules={\ethereal{}, \lighttroops{}, \reaper{} \only{\wraithprofile}, \armourpiercing{6} \only{\wraithprofile}, \terror{}, \skirmisher},
	weapons={\gw},
	wizardconclave={\deathsignature{} (\Pathof{} \death), \shadowssignature{} (\Pathof{} \shadows)},
	options={
		\flamingattacks{} \only{\wraithprofile}=\permodel{}<2,
		\ghoststeeds{}=\permodel{}<15,
	},
	commandgroup={champion=60, championprerestriction=\textbf{\bloodtie{\nosferatu}}\spacebeforecolon{}:},
	unitrules={\unitrule{\ghoststeeds}{\ghoststeedsrule}},
}

\showunit{
	name={\vampireknights},
	QRSname={\vampireknight},
	cost=150,
	profile={
		\rider{}< 4 5 3 5 4 2 5 2 7,
		\undeadmount{}< 8 3 - 4 3 1 2 1 3,
	},
	invocation=1,
	type=\cavalry{},
	basesize=25x50,
	unitsize=3,
	maxmodels=6,
	costpermodel=50,
	commontype=\vampiriccommonrules{},
	commonspecialrules={\undead{},\fear{},\vampiric{6}},
	armour={\ha{}, \shield{}, \mountsprotection{6}, \barding},
	weapons={\lance},
	commandgroup={champion=10, musician=10, banner=10, bannerallowance=50, championallowance=25},
	additional={%
		\def\tempoptions{
		\textbf{\bloodtie{\brotherhood}}\spacebeforecolon{}:\newline%
		\vampireknightbloodtiesoption{}=\permodel{}<10,
		}%
		\vspace*{0.2cm}\options{\tempoptions}
	},
}

\showunit{
	name={\wingedreapers},
	QRSname={\wingedreaper},
	cost=155,
	profile={< 6 5 3 5 5 4 4 4 10,},
	invocation=2,
	type=\monstrousinfantry{},
	basesize=50x75,
	unitsize=2,
	maxmodels=5,
	costpermodel=72,
	commontype=\undeadcommonrules{},
	commonspecialrules={\undead{}, \ashestoashes},
	specialrules={\undeadconstruct{}, \lethalstrike{}, \fear{}, \fly{6}},
	armour={\innatedefence{5}},
	options={
		\la{}=\permodel{}<10,
		\weapononechoice{
			\pw{}=\permodel{}<5,
			\halberd{}=\permodel{}<12,
		},
		\onechoiceonly{
			\autonomous{}=\permodel{}<10,
			\necromanticaura{}=20,
		},
	},
	additional={%
		\def\tempunitrule{
		\unitrule{\undeadconstruct}{\undeadconstructrule}
		\unitrule{\autonomous}{\autonomousrule}
		}
		\vspace*{0.1cm}\unitrules{\tempunitrule}
	},
}

\showunit{
	name={\darkcoach},
	cost=190,
	profile={
		\coach{}< - - - 5 6 4 - - -,
        		\coachman{} (1)< - 3 - 3 - - 2 3 5,
		[\awakenedvampire{}]< - 6 - 5 - - 6 4 8,
		\undeadmount{} (2)< 8 3 - 4 - - 2 1 3,
	},
	invocation=1,
	type=\chariot{},
	basesize=50x100,
	unitsize=1,
	commontype=\vampiriccommonrules{},
	commonspecialrules={\undead{}, \vampiric{5}},
	armour={\innatedefence{4}},
	weapons={\gw{} \only{\coachman}},
	options={%
		\extendedchassis{}=10,
		\textbf{\bloodtie{\vonkarnstein}}\spacebeforecolon{}:\newline
		\stubborn{}=30,
	},
	additional={%
		\def\tempspecialrules{\soulsyphon{}, \impacthits{+1}, \wardsave{4}, \terror{}, \armourpiercing{6} \only{\coachman}}
		\def\tempunitrule{%
			\unitrule{\extendedchassis}{\extendedchassisrule}
			\unitrule{\soulsyphon}{\soulsyphonrule}
		}%
				
		\vspace*{-0.1cm}\specialrules{\tempspecialrules}
		\vspace*{0.2cm}\unitrules{\tempunitrule}
		\soulsyphonchart
	},
}

\showunit{
	name={\altarofundeath},
	cost=200,
	profile={
		\altar{}< -  - - 5 5 5 - - -,
       		\master{}< -  3 1 3 - - 3 1 5,
		[\deathlychoir{}]< -  3 - 3 - - 3 3 5,
		\ghoststeeds{}< 8 2 - 3 - - 2 \starsymbol{} 3,
	},
	invocation=1,
	type=\chariot{},
	basesize=50x100,
	unitsize=1,
	commontype=\undeadcommonrules{},
	commonspecialrules={\undead{}, \ashestoashes},
	armour={\innatedefence{5}},
	options={
		\onechoiceonly{
			\deathlychoir{}=20,
			\darktome{}=30,
		},
	},
	additional={%
		\def\tempspecialrules{\randomattacks{2D6} \only{\steeds}, \auraofundeath{}, \ethereal{} \only{\steeds}, \largetarget{}, \terror{}, \regeneration{4}}%
		\def\tempunitrule{%
			\unitrule{\deathlychoir}{\deathlychoirrule}
			\unitrule{\darktome}{\darktomerule}
			\unitrule{\auraofundeath}{\auraofundeathrule}		
		}%
		\vspace*{0.2cm}\specialrules{\tempspecialrules}\newline%
		\unitrules{\tempunitrule}
	},
}

\showunit{
	name={\shriekinghorror},
	cost=245,
	profile={< 6 4 - 5 6 6 2 4 4},
	invocation=1,
	type=\monster{},
	basesize=100x150,
	unitsize=1,
	commontype=\undeadcommonrules{},
	commonspecialrules={\undead{}, \ashestoashes},
	specialrules={\chillingshriek{}, \regeneration{6}, \fly{8}},
	additional={%
		\def\tempunitrule{\unitrule{\chillingshriek}{\chillingshriekrule}}
		\vspace*{0.2cm}\unitrules{\tempunitrule}
	},
}




%%% Quick Reference Sheet - AB_qrs.tex is automatic and shouldn't be edited %%%

\quickrefsheettitle

\input{../Layout/AB_qrs.tex}
\bigskip
\begin{center}\noindent{\antiquefont\Largefontsize\textbf{\labels@Invocation}}\end{center}

\newcommand{\QRSinvoctable}[2]{%
\rowcolors{1}{white}{black!10}
\noindent\begin{tabular}{p{4cm}>{\centering\let\newline\\\arraybackslash\hspace{0pt}}p{1cm}@{}}%
\antiquefont\Large{\textbf{#1\spacebeforecolon{}:}}&\vspace*{-0.2cm}%
\DTLforeach*[#2]{profiles}{\rowname=name, \rowtrooptype=trooptype, \rowcategory=category, \rowinvocation=invocation}{%
\tabularnewline\rowname{} & \rowinvocation{}}%
\tabularnewline%
\end{tabular}
\medskip
}

\begin{multicols}{3}\raggedcolumns
\QRSinvoctable{\infantry}{\DTLiseq{\rowtrooptype}{\infantry}\and\not\DTLiseq{\rowcategory}{\labels@heroes}\and\not\DTLiseq{\rowcategory}{\labels@lords}}

\QRSinvoctable{\swarms}{\DTLiseq{\rowtrooptype}{\swarm}}

\QRSinvoctable{\monstrousinfantry}{\DTLiseq{\rowtrooptype}{\monstrousinfantry}}

\QRSinvoctable{\cavalry}{\DTLiseq{\rowtrooptype}{\cavalry}\and\not\DTLiseq{\rowcategory}{\labels@mounts}}

\QRSinvoctable{\warbeasts}{\DTLiseq{\rowtrooptype}{\warbeast}\and\not\DTLiseq{\rowcategory}{\labels@mounts}}

\QRSinvoctable{\monstrousbeasts}{\DTLiseq{\rowtrooptype}{\monstrousbeast}\and\not\DTLiseq{\rowcategory}{\labels@mounts}}

\QRSinvoctable{\monsters}{\DTLiseq{\rowtrooptype}{\monster}\and\not\DTLiseq{\rowcategory}{\labels@mounts}}

\rowcolors{1}{white}{white}
\noindent\begin{tabular}{p{4cm}>{\centering\let\newline\\\arraybackslash\hspace{0pt}}p{1cm}@{}}%
{\antiquefont\Large\textbf{\allchariots{}\spacebeforecolon{}:}}& 1 \tabularnewline
\end{tabular}
\medskip
\end{multicols}

\restoregeometry

\changelogtitle

\startchangelog

\newlog{0.99.2}{%
Invocation clarification,
Ashes to ashes clarification,
Awaken clarification,
Reaper clarification,
Crimson Rage clarification,
Storm Caller clarification,
Lamia Bloodline clarification,
Vampiric Bloodlines clarification,
Strigoi Bloodline clarification,
Ghoul Lord clarification,
Hour of the Wolf clarification,
Black Standard of Zagvozd clarification,
Staff of Gerhard the Black clarification,
Unliving Shield clarification,
Ghouls vanguard option clarification,
Cadaver Wagon cart clarification,
Wake the Dead clarification,
Endless Horde clarification,
Bring Out Your Dead clarification,
Wraiths special rules clarification,
Soul Syphon clarification,
}

\newlog{0.99.1}{%
Green text reverted to black,
Ld on chariot mounts,
Altar of Undeath and dark coach{,} name of charioteers,
}

\newlog{0.99.0}{%
Master of Undeath,
Chilling Shriek redesign,
Wail of Woe: new scream on Banshee,
Awaken: max limit,
Reaper: clarification,
Vampire: redistribution,
Bloodlines: Layout,
Brotherhood of the Dragon: cost and clarification,
Crimson Rage: clarification,
Eternal Duelist: clarification,
Strigoi: cost and clarification,
Curse of the Blood: clarification,
Bestial Revenant: New power that replaces Bat Form,
Von Karstein: cost,
Storm Caller: redesign,
Hour of the Wolf: clarification,
Refined Taste: redistribution,
Lamia: redesign and clarification,
Mesmerizing Gaze: Redesign,
Nosferatu: cost and clarification,
Power of the Mind: new power,
Arcane Knowledge: clarification and slight redesign,
Blade of Red Thirst: redistribution,
Eternal Ring: new item replace Bow of Nepharet,
Staff of Gerhard Black Tongue: new item replaces Staff of Vengeful Death,
Banner of the Barrows: slight redesign,
Black Standard of Zagvozd: slight redesign,
Vampire Count: cost of lvl3 wizard,
Necromancer Lord: cost,
Vampire Courtier: BSB limit,
Barrow King: Weapon Skill,
Unliving Shield: redesign,
Banshee: Wail of Woe,
Zombies: cost,
Skeletons: cost,
Ghouls: Initiative and cost,
Dire Wolves: cost,
Barrow Knights: cost and Invocation,
Barrow Guard: cost,
Ghasts: cost,
Vampire Spawn: cost and size,
Phantom Host: AP(1),
Varkolak: cost and slight redesign,
Cadaver Wagon: cost and clarification,
Court of the damned: moved to Special and redesign,
Vampire Knights: cost and size,
Wraiths: merged with Mounted Wraiths and redesign,
Winged Reapers: cost and special rules,
Shrieking Horror: cost,
Dark Coach: cost{,} base size{,} upgrade and clarification,
Monstrous Revenant: new options,
}

\endchangelog

\end{document}