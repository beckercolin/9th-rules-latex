\input{../Formatage/format_LA.tex}

\begin{document}


\newcommand{\greenhideraces}{\specialrule{Race des Peaux Vertes}\xspace}
\newcommand{\unruly}{\specialrule{Indiscipliné}\xspace}
\newcommand{\Orquea}{\specialrule{Orque Commun}\xspace}
\newcommand{\Orqueb}{\specialrule{Orqu'en Fer}\xspace}
\newcommand{\Orquec}{\specialrule{Orque Primitif}\xspace}
\newcommand{\goba}{\specialrule{Gobelin Commun}\xspace}
\newcommand{\gobb}{\specialrule{Gobelin des Cavernes}\xspace}
\newcommand{\gobc}{\specialrule{Gobelin des Forêts}\xspace}
\newcommand{\borntofight}{\specialrule{Né pour la Baston}\xspace}
\newcommand{\waaargh}{\specialrule{Waaargh!}\xspace}
\newcommand{\venomousfangs}{\specialrule{Dard Venimeux}\xspace}
\newcommand{\greentide}{\specialrule{Marée Verte}\xspace}
\newcommand{\shambolic}[1]{\specialrule{Chamboul'Tout\ifblank{#1}{}{~(#1)}}\xspace}
\newcommand{\ricochet}[1]{\specialrule{Strike\ifblank{#1}{}{~(#1)}}\xspace}
\newcommand{\nets}{\specialrule{Filets}\xspace}
\newcommand{\surprise}{\specialrule{Planké}\xspace}
\newcommand{\rowsofteeth}{\specialrule{Rangées de Dents}\xspace}
\newcommand{\theyreeverywhere}{\specialrule{Partent en Sucette}\xspace}
\newcommand{\trollbelch}{\specialrule{Bile de Troll}\xspace}
\newcommand{\smasher}{\specialrule{Frakasseur}\xspace}
\newcommand{\pointedsticks}{\specialrule{Bout Pointu}\xspace}
\newcommand{\pursuitmode}{\specialrule{Rapide et Furieux}\xspace}
\newcommand{\smellslikegreenspirit}{\specialrule{Tu me Vois, Tu me Vois Plus}\xspace}
\newcommand{\lookatemgo}{\specialrule{Perd le Contrôle}\xspace}
\newcommand{\orqueoverseer}{\specialrule{Garde Chiourme}\xspace}
\newcommand{\smashemflat}{\specialrule{Cé Nous Kon les Kraz !}\xspace}
\newcommand{\wevegotthegreenlight}{\specialrule{Vert de Rage}\xspace}
\newcommand{\spidermothershrine}{\specialrule{Autel de la Mère Araignée}\xspace}
\newcommand{\bouncers}{\specialrule{Bondisseur}\xspace}
\newcommand{\sneaky}{\specialrule{Sournois}\xspace}
\newcommand{\motherskiss}{\specialrule{Baiser de la Mère Araignée}\xspace}
\newcommand{\oiitbites}{\specialrule{Fo pas s'approcher !}\xspace}


\newcommand{\booktitle}{Peaux Vertes}
\newcommand{\version}{0.99.1}
\newcommand{\frenchversion}{1.1}
\newcommand{\translationteam}{\item \og AEnoriel \fg \item \og Anglachel \fg \item \og Astadriel \fg \item \og Batcat \fg \item \og Eru \fg\item \og Gandarin \fg \item \og Groumbahk \fg \item \og Iluvatar \fg \item \og Lamronchak \fg \item \og Mammstein \fg}


\input{../Formatage/titlepage_LA.tex}

\armyspecialrules

\armyspecialruleentry{\shambolic{X, type}}

Les unités avec cette règle ont \randommovement{X}. Les unités \shambolic{} ont la règle \immunetopsychology et ne peuvent pas être rejointes par des Personnages. Si une unité obtient le même résultat sur chaque dé lors de son \randommovement{}, elle subit 1D3 blessures sans aucune sauvegarde, puis se déplace de cette distance dans une direction aléatoire.

Une unité avec cette règle doit faire un test de \emph{Terrain Dangereux (2)} si elle arrive au contact d'un élément de décor (autre qu'un Terrain Découvert ou une Colline), si elle arrive au contact d'un bord de table ou si elle s'arrête à \distance{1} d'un Terrain Infranchissable.

Le \randommovement{} d'une unité \shambolic{} a un mode optionnel :
\newline \textbf{Nawak} : Se déplace toujours dans une direction aléatoire lors de son \randommovement{}.

\armyspecialruleentry{\venomousfangs}

Désignez une des attaques de la figurine avant de lancer les dés pour toucher. Cette attaque gagne \multiplewounds{\ordnance}{}.

\armyspecialruleentry{\unruly}

Les unités suivant cette règle ont un malus de -1 en Commandement pour les tests suivants : réfréner la \frenzy , réfréner la poursuite. Si l'unité est en formation de Horde, jetez trois dés pour les tests de Panique en enlevant le plus grand.

\armyspecialruleentry{\greentide}

Une fois par partie, le Général (s'il possède cette règle) peut déclarer une \greentide au début du tour de n'importe quel joueur. Toutes les figurines de Gobelins (de n'importe quelle \greenhideraces) gagnent la règle \fightinextrarank jusqu'à la fin du tour suivant du joueur.

\armyspecialruleentry{\borntofight}

Les figurines ou éléments de figurine avec cette règle ont +1 en Force au premier round d'un corps à corps.

\armyspecialruleentry{\greenhideraces}

Différentes figurines ont des règles spécifiques en fonction de leur race :
\begin{customdescription}
	\item[\Orquea :] \unruly, \borntofight.
	\item[\Orqueb :] \immunetopsychology, \weaponmaster, \borntofight.
	\item[\Orquec :] \frenzy, \unruly, \borntofight, \wardsave{6}.
	\item[\goba :] \unruly, \insignificant.
	\item[\gobb :] \unruly, \insignificant, \hatred (Livre d'Armée : Forteresses Naines).
	\item[\gobc :] \strider{Forêts}, \unruly, \insignificant.
\end{customdescription}

\armyspecialruleentry{\ricochet{X}}

Les figurines avec cette règle ignorent les \distance{1} d'espace entre unités lors de leur \randommovement{}. Si une figurine avec cette règle touche une autre unité, amie ou ennemie, elle continue son mouvement dans la même direction jusqu'à respecter la règle des \distance{1} d'espace entre unités. Cela peut amener cette unité à traverser plusieurs unités lors du même mouvement. Si elle ne peut être placée à la fin, elle est retirée comme perte.

Toute unité traversée, et qui est dans la distance de déplacement de la figurine ayant cette règle, est touchée. Les unités touchées subissent X touches, la valeur de X étant écrite entre les parenthèses. Si une figurine avec cette règle traverse plusieurs unités déjà engagées dans un même corps à corps, elles ne comptent que pour une seule cible en ce qui concerne la détermination des touches. Le joueur possédant ces unités répartit les touches infligées entre ses unités aussi équitablement que possible, puis applique les règles normales de répartition des touches dans chaque unité.

Les unités ne peuvent pas charger une figurine avec cette règle, mais peuvent charger, fuir, poursuivre ou bouger à travers elle. Les unités faisant ainsi subissent X+1D6 touches, puis retirez la figurine traversée comme perte.

Toutes les touches infligées par les figurines avec cette règle sont faites avec leur Force non modifiée et ont \armourpiercing{1}.


\armyspecialruleentry{\waaargh}

Une fois par partie, le Général (s'il possède cette règle) peut déclarer une \waaargh au début du tour de n'importe quel joueur. Toutes les figurines, appartenant à une \greenhideraces de l'armée, gagnent +1 en Mouvement et \swiftstride jusqu'à la fin du tour du joueur.


\armyarmoury

\begin{customdescription}
	\item[Champix de Pouvoir :] Générez 1D3+1 Champix de Pouvoir à la fin du déploiement pour le porteur. Quand celui-ci lance un sort, il peut décider d'utiliser un Champix de Pouvoir pour ajouter 1 au résultat de lancement du sort (c'est une exception à la limitation des Modificateurs magiques). Un Champix de Pouvoir utilisé ne peut pas être réutilisé à nouveau. Après avoir utilisé un Champix de Pouvoir, lancez 1D6. Si le dé donne un \result{1}, le Sorcier subit une touche de Force 3 sans sauvegarde d'aucune sorte.
	\item[Perceur 'eud Gros Trucs :] Une unité qui charge avec au moins un Rang Complet gagne \impacthits{1D3} de Force 5 avec \multiplewounds{\ordnance}{\largetarget}.
\end{customdescription}


\armymagicitems


\armynewsubsection{Armes magiques}

\begin{customitemize}
	\item \optiondef{Hache-Rage de la Furie}{65 / 50}{Arme de base. Le porteur gagne +1D3 en Force et +1D3 Attaques. Ces bonus sont déterminés et prennent effet au palier d'initiative auquel le porteur attaque avec cette arme.}
	\item \optiondef{Ark Zap d'Maza}{30}{Arc. \portee{24}, Force 3, \lightningattacks, \multipleshots 3. L'unité du porteur gagne la règle spéciale \quicktofire.}
	\item \optiondef{Dague de Sournois}{15}{Arme de base. Les attaques faites avec cette arme ont \armourpiercing{1}. Si le porteur attaque une unité ennemie de flanc ou de dos, ses attaques ont +2 en Force.}	
\end{customitemize}

\armynewsubsection{Armures magiques}

\begin{customitemize}
	\item \optiondef{Couronne du Roi des Cavernes}{40}{\only{Gobelin}. Ne peut pas être pris par une \largetarget. Sauvegarde d'armure de 6+. Le porteur ne peut rejoindre qu'une unité de gobelins de la même \greenhideraces que lui.} L'unité du porteur gagne alors la règle \vanguard et peut se déplacer après un ralliement (mais elle ne peut ni tirer, ni faire de Marche Forcée dans ce cas). De plus, la portée de la \inspiringpresence ou de \holdyourground du porteur est augmentée de 6".
	\item \optiondef{Plaques Divines de Tuktek}{35}{Armure lourde. Le porteur gagne +1 en Endurance et \impacthits{1D3}.}
\end{customitemize}

\armynewsubsection{Talismans}

\begin{customitemize}
	\item \optiondef{Batée Chapardeuse de Défens'}{25}{\only{Gobelin}. Si le porteur subit une blessure, il peut utiliser la Sauvegarde d'Armure, la \wardsave{}, la \regeneration{} et la \magicresistance{} de la figurine lui ayant infligé cette blessure.}
\end{customitemize}

\armynewsubsection{Objets enchantés}

\begin{customitemize}
	\item \optiondef{Peintures Waaargh!}{30}{\only{\Orquec}. Le porteur a la règle \frenzy et ne peut jamais la perdre. De plus, chaque \Orquec de l'unité du porteur a la règle \frenzy tant que ce dernier est dans l'unité. Enfin, chaque \Orquec de l'unité gagne la règle \swiftstride pour les Mouvements de Poursuite et de Charge Irrésistible.}
	\item \optiondef{Patt'chanceuz' d'sanglier}{20}{\only{Figurine montée}. Toutes les figurines de type Cavalerie à \distance{18} du porteur peuvent relancer leurs tests de Terrain Dangereux. }
\end{customitemize}

\armynewsubsection{Bannières magiques}

\begin{customitemize}
	\item \optiondef{Totem Sacré de Mikinok}{40}{Les autres Objets magiques des figurines (amies ou ennemies) en contact avec l'unité du porteur, ainsi que celles de l'unité du porteur, n'ont plus d'effet et redeviennent des objets ordinaires. Cet effet dure tant que les unités restent en contact.}
	\item \optiondef{Icône Tanne-fer}{25}{\wardsave{5} contre les attaques de tir.}
\end{customitemize}


\armylist

\lordstitle

\showunit{
	name={Seigneur de Guerre Orque},
	cost={120},
	profile={Seigneur de Guerre Orque : 4 6 3 5 5 3 4 4 9},
	type=Infanterie,
	unitsize={1},
	basesize=25x25,
	equipment={Armure légère \only{\Orquea}, Armure lourde \only{\Orqueb}}, 
	options={
	Peut choisir des objets magiques = \upto: 100,
	\optionschoice{Doit devenir (un seul choix)}{
		\Orquea=\free,
		\Orquec= 15,
		\Orqueb (gagne +1 en Capacité de Combat)=20},
	Le Général peur avoir la règle \waaargh =20,	
	Bouclier =5,
	Armure lourde \only{\Orquea}=12,
	Armure de plates \only{\Orqueb}=20,
	Paire d'armes=5,
	Lance de cavalerie=15,
	Arme lourde=15
	},
	mounts={
		Sanglier \only{\Orquec}=10,		
		Sanglier \only{\Orquea et \Orqueb}=20,
		Char à Sangliers \only{\Orquea et \Orqueb}=30,
		Vouivre \only{\Orquec}=105,
		Vouivre \only{\Orquea et \Orqueb}=120},
}


\showunit{
	name={Grand Chamane Orque},
	cost={175},
	profile={Grand Chamane Orque : 4 3 3 4 5 3 2 1 8},
	type=Infanterie,
	unitsize={1},
	basesize=25x25,
	magiclevelmaster=3,
	magicpaths={\thebiggreengods, \wilderness},
	options={
	Peut choisir des objets magiques = \upto: 100,
	\magiclevelmaster{4}=30,
	\optionschoice{Doit devenir (un seul choix)}{
		\Orquea=\free,
		\Orquec= 5},
	},
	mounts={
		Char à Sangliers \only{\Orquea}=20,
		Sanglier=20,
		Vouivre=120}
}

\showunit{
	name={Seigneur Gobelin},
	cost={60},
	profile={Seigneur Gobelin : 4 5 4 4 4 3 4 4 8},
	type=Infanterie,
	unitsize={1},
	basesize=20x20,
	equipment={Armure légère}, 
	options={
	Peut choisir des objets magiques = \upto: 100,
	\optionschoice{Doit devenir (un seul choix)}{
		\goba=\free,
		\gobc= \free,
		\gobb (gagne +1 en Initiative)}=5,
	Le Général peur avoir la règle \greentide =10,
	Bouclier =5,
	Armure lourde =8,
	Arc court =5,
	\poisonedattacks \only{\gobc}=10,
	\optionschoice{Peut avoir (un seul choix)}{
		Paire d'armes=5,
		Lance de cavalerie=15,
		Arme lourde=15}
	},
	mounts={
		Loup \only{\goba} = 15,
		Araignée \only{\gobc}=20,
		Gigaraignée \only{\gobc}=20,
		Gros-Gniark \only{\gobb}=20,
		Char à Loups \only{\goba}=25,
		Araignée Titanesque \only{\gobc}=250,
		}
}


\showunit{
	name={Grand Chamane Gobelin},
	cost={170},
	profile={Grand Chamane Gobelin : 4 2 3 3 4 3 2 1 7},
	type=Infanterie,
	unitsize={1},
	basesize=20x20,
	magiclevelmaster=3,
	magicpaths={\thelittlegreengods, \shadows},
	options={
	Peut choisir des objets magiques = \upto: 100,
	\magiclevelmaster{4}=30,
	\optionschoice{Doit devenir (un seul choix)}{
		\goba=\free,
		\gobb (gagne +1 en Initiative et -1 en Commandement)=\free,
		\gobc= \free},
	Champix de Pouvoir \only{\gobb}=15,
	},
	mounts={
		Araignée \only{\gobc}=15,
		Loup \only{\goba} = 15,
		Char à Loups \only{\goba}=20,
		Araignée Titanesque \only{\gobc}=250,
		}
}


\heroestitle


\showunit{
	name={Boss Orque},
	cost={50},
	profile={Boss Orque : 4 5 3 4 5 2 3 3 8},
	type=Infanterie,
	unitsize={1},
	basesize=25x25,
	equipment={Armure légère \only{\Orquea}, Armure lourde \only{\Orqueb}}, 
	options={
	Peut choisir des objets magiques = \upto: 50,
	Porteur de la Grande Bannière=25,
	\optionschoice{Doit devenir (un seul choix)}{
		\Orquea=\free,
		\Orquec= 5,
		\Orqueb (gagne +1 en Capacité de Combat)=10},
	Le Général peur avoir la règle \waaargh =10,
	Bouclier =5,
	Armure lourde \only{\Orquea}=5,
	Armure de plates \only{\Orqueb}=15,
	\optionschoice{Peut avoir}{
		Paire d'arme=5,
		Lance de cavalerie=10,
		Arme lourde=10,}
	},
	mounts={
		Sanglier = 15,
		Char à Sangliers \only{\Orquea}=60,
		Vouivre=150}
}


\showunit{
	name={Chamane Orque},
	cost={65},
	profile={Chamane Orque : 4 3 3 3 4 2 2 1 7},
	type=Infanterie,
	unitsize={1},
	basesize=25x25,
	magiclevelapprentice=1,
	magicpaths={\thebiggreengods, \wilderness},
	options={
	Peut choisir des objets magiques = \upto: 50,
	\magiclevelapprentice{2}=25,
	\optionschoice{Doit devenir (un seul choix)}{
		\Orquea=\free,
		\Orquec= 5},
	},
	mounts={
		Sanglier = 15,
		Char à Sangliers \only{\Orquea}=50,}
}


\showunit{
	name={Boss Gobelin},
	cost={35},
	profile={Boss Gobelin : 4 4 4 4 4 2 3 3 7},
	type=Infanterie,
	unitsize={1},
	basesize=20x20,
	equipment={Armure légère}, 
	options={
	Peut choisir des objets magiques = \upto: 50,
	Porteur de la Grande Bannière=25,
	\optionschoice{Doit devenir (un seul choix)}{
		\goba=\free,
		\gobb (gagne +1 en Initiative et -1 en Commandement)=\free,
		\gobc= \free},
	Le Général peur avoir la règle \greentide =20,
	Bouclier =\free,
	Armure lourde \only{\goba}=5,
	Arc court =3,
	\poisonedattacks \only{\gobc}=5,
	\optionschoice{Peut avoir (un seul choix)}{
		Paire d'armes=3,
		Lance légère=3,
		Lance de cavalerie=6,
		Arme lourde=6}
	},
	mounts={
		Araignée \only{\gobc}=15,
		Loup \only{\goba} = 20,
		Gigaraignée \only{\gobc}=25,
		Gros-Gniark \only{\gobb}=35,
		Char à Loups \only{\goba}=45,
		}
}


\showunit{
	name={Chamane Gobelin},
	cost={60},
	profile={Chamane Gobelin : 4 2 3 3 3 2 2 1 6},
	type=Infanterie,
	unitsize={1},
	basesize=20x20,
	magiclevelapprentice=1,
	magicpaths={\thelittlegreengods},
	options={
	Peut choisir des objets magiques = \upto: 50,
	\magiclevelapprentice{2}=25,
	\optionschoice{Doit devenir (un seul choix)}{
		\goba=\free,
		\gobb (gagne +1 en Initiative et -1 en Commandement)=\free,
		\gobc= \free},
	Champix de Pouvoir \only{\gobb}=15,
	},
	mounts={
		Araignée \only{\gobc}=15,
		Loup \only{\goba} = 15,
		Char à Loups \only{\goba}=40,}
}



\baseunitstitle


\showunit{
	name={Orques},
	cost=90,
	profile={Orque : 4 3 3 3 4 1 2 1 7},
	type=Infanterie,
	unitsize=20,
	additionalmodels=30,
	costpermodel=6,
	basesize=25x25,
	equipment={Armure légère \only{\Orquea}},
	options={
	\optionschoice{Doit devenir (un seul choix)}{
		\Orquea=\free,
		\Orquec= \permodel:2},
	Bouclier =\permodel:1,
	Arc = \permodel:1,
	Paire d'armes =\permodel:1,
	Lance =\permodel:1,
	Peut échanger tous ses équipements contre Arbalète et Armure lourde \only{\Orquea}=\permodel:4,
	Perceur 'eud Gros Trucs \only{\Orquec}=15,
	},
	commandgroup={\commandgroup{champion=10, banner=10, standardbeareroption=\veteranstandardbearer *25, musician=10}}
}

\showunit{
	name={Orques Kraz'Krânes},
	cost=70,
	profile={Orque Kraz'Krânes : 4 4 3 4 4 1 2 1 7},
	type=Infanterie,
	unitsize=10,
	additionalmodels=30,
	costpermodel=9,
	basesize=25x25,
	specialrules={\oneofakind},
	equipment={Armure légère \only{\Orquea}},
	options={
	\optionschoice{Doit devenir (un seul choix)}{
		\Orquea=\free,
		\Orquec= \permodel:1},
	Bouclier=\permodel:1,
	Paire d'armes =\permodel:1,
	Lance=\permodel:1,
	Perceur 'eud Gros Trucs \only{\Orquec}=15,
	},
	commandgroup={\commandgroup{champion=10, banner=10, standardbeareroption=\veteranstandardbearer *25, musician=10}}
}


\showunit{
	name={Gobelins},
	cost=60,
	profile={Gobelin : 4 2 3 3 3 1 2 1 6},
	type=Infanterie,
	unitsize=20,
	additionalmodels=40,
	costpermodel=3,
	basesize=20x20,
	unitrules={\unitrule{\motherskiss}{Au début de chaque round de corps à corps, jetez 1D6 pour pour chaque unité ayant la règle \motherskiss. Sur 2+, cette unité gagne des \poisonedattacks jusqu'à la fin du round de combat. Sur un résultat de '1', déterminer aléatoirement une unité ennemie en contact avec l'unité ayant la règle \motherskiss. Cette unité ennemie gagne des \poisonedattacks jusqu'à la fin du round de combat.}
\unitrule{\nets}{Au début du round de corps à corps, choisissez une unité en contact avec l'unité avec la règle \nets. Jetez 1D6, sur 2+ la cible subit un malus de -1 en Force (minimum 1) jusqu'à la fin du tour. Sur un résultat de '1', l'unité qui a utilisé la règle \nets subit ce malus. Une unité ne peut être affectée par la règle \nets qu'une seule fois par phase de combat.}	
	},
	equipment={Armure légère \only{\goba}},
	options={
	\optionschoice{Doit devenir (un seul choix)}{
		\goba=\free,
		\gobb (gagne +1 en Initiative et -1 en Commandement)= \free,
		\gobc= \free},
	\optionschoice{Peut avoir (un seul choix)}{
		Bouclier =\permodel:1,
		Arc court =\free,
		Lance et Bouclier =\permodel:1},
	Peut échanger toutes ses armes contre Bouclier et Arc court \only{\goba}=\permodel:1,
	Armes de jet \only{\gobc} =\permodel:1,
	\motherskiss \only{\gobc}=\permodel:1,
	\nets \only{\gobb}=\permodel:1,
	\skirmishers \only{\gobc} (pour une unité de 20 figurines maximum) =\permodel:1,
	Jusqu'à 3 figurines d'Assassin Gobelin \only{\goba}=\permodel:15,
	Jusqu'à 3 figurines de Boulet-Fou \only{\gobb}=\permodel:30,
	},
	commandgroup={\commandgroup{champion=10, banner=10, standardbeareroption=\veteranstandardbearer *25, musician=10}}
}

\showunit{
	name={Assassin Gobelin},
	cost=-,
	profile={Assassin Gobelin : 4 4 3 3 3 1 3 2 6},
	type=Infanterie,
	unitsize=1,
	basesize=20x20,
	equipment={Armure légère, Paire d'armes},
	specialrules={\lethalstrike, \goba, \sneaky},
	unitrules={\unitrule{\sneaky}{Les Assassins Gobelins possèdent les mêmes règles que des champions d'unité et sont automatiquement déployés \emph{Cachés} dans l'unité qui les a pris en option. Les Assassins Gobelins sont automatiquement révélés au premier round de corps à corps de leur unité. Ils ne peuvent pas être révélés plus tôt. Lorsqu'ils sont révélés, les Assassins Gobelins reçoivent +3 en Initiative et ils gagnent la règle spéciale \lightningreflexes jusqu'à la fin du tour. Les Assassins Gobelins ne bénéficient ni de la règle \emph{Premier entre ses pairs} ni de \emph{Meneur de charge} (voir les règles des Champions).}},
}


\showunit{
	name={Boulet-Fou},
	cost=-,
	profile={Boulet-Fou : 2D6 - - 5 3 1 3 1 5},
	type=Infanterie,
	unitsize=1,
	basesize=25,
	specialrules={\hardtarget, \shambolic{2D6, Nawak}, \gobb, \surprise, \ricochet{1D6}},
	unitrules={\unitrule{\surprise}{Un Boulet-Fou n'est pas déployé, mais est caché dans l'unité qui l'a pris en option. Un Boulet-Fou ne donne pas de points de victoire en fin de partie, son coût étant compté comme une amélioration de son unité de départ. Quand un Boulet-Fou est retiré comme perte, il n'engendre pas de test de panique pour ses alliés. Les Boulets-Fous se déplacent, agissent et sont affectés par des règles spéciales indépendamment, comme des unités normales. De plus, ils ne comptent pas parmi les figurines de l'unité qui les cache.
\unitrule{Faire sortir les Boulets-Fous}
\unitrule{Première façon}{Un nombre quelconque de Boulets-Fous peut être sorti lors d'une réaction à charge "maintenir et tirer" ou "maintenir". L'option normale "maintenir et tirer" peut aussi être prise par l'unité.}
\unitrule{Seconde façon}{Tout Boulet-Fou doit être sorti de son unité au début de votre Phase de tir si cette unité n'est pas au corps à corps, si elle n'est pas en fuite et si elle se trouve à \distance{8} d'une unité ennemie.}
\unitrule{Premier déplacement}
Les figurines de Boulets-Fous sortant d'une unité se déplacent l'une après l'autre. Placez un Boulet-Fou en contact avec son unité (notez qu'il ne lui inflige aucun dégât dans ce cas-là) et choisissez une direction. Déplacez le Boulet-Fou de \distance{2D6} dans cette direction. Les Boulets-Fous suivent ensuite leurs propres règles pour les phases de mouvements suivantes.}},
}

\showunit{
	name={Chevaucheurs Gobelins},
	cost=60,
	profile={Chevaucheur Gobelin : 4 2 3 3 3 1 2 1 6,
			 Loup : 9 3 - 3 3 1 3 1 3,
			 Araignée : 7 3 - 3 3 1 4 1 2},
	type=Cavalerie,
	unitsize=5,
	additionalmodels=15,
	costpermodel=8,
	basesize=25x50,
	specialrules={\poisonedattacks \only{Araignée}, \fastcavalry, \scout \only{\gobc}, \strider{} \only{\gobc}},
	equipment={Armure légère \only{\goba}, Loup \only{\goba}, Araignée \only{\gobc}, \mountprotection{6}},
	options={
	\optionschoice{Doit devenir (un seul choix)}{
		\goba=\free,
		\gobc= \free},
	\optionschoice{Doit choisir un équipement (ou plusieurs) parmi}{
	Bouclier =\permodel:1,
	Arc court =\permodel:1,
	Armes de jet \only{\gobc} =\permodel:1,
	Lance légère =\permodel:1},
	},
	commandgroup={\commandgroup{champion=10, banner=10, musician=10}},
}

\showunit{
	name={Chevaucheurs de Sangliers},
	cost=70,
	profile={Chevaucheur Orque : 4 3 3 3 4 1 2 1 7,
			 Sanglier : 7 3 - 3 3 1 3 1 3},
	type=Cavalerie,
	unitsize=5,
	additionalmodels=10,
	costpermodel=13,
	basesize=25x50,
	specialrules={\thunderouscharge \only{Sanglier}},
	equipment={Lance légère, Armure légère \only{\Orquea}, \mountprotection{5}},
	options={
	\optionschoice{Doit devenir (un seul choix)}{
		\Orquea=\free,
		\Orquec= \permodel:1},
	Bouclier =\permodel:3,
	Paire d'armes \only{\Orquec}=\permodel:2,
	Lance de cavalerie \only{\Orquea} =\permodel:3,
	},
	commandgroup={\commandgroup{champion=10, banner=10, standardbeareroption=\veteranstandardbearer *25, musician=10}},
}

\specialunitstitle

\showunit{
	name={Full Metal'Orques},
	cost=100,
	profile={Full Metal'Orque : 4 5 3 4 4 1 2 1 8},
	type=Infanterie,
	unitsize=10,
	additionalmodels=25,
	costpermodel=13,
	basesize=25x25,
	specialrules={\bodyguard {Seigneur \Orqueb, Boss \Orqueb},\Orqueb},
	equipment={Armure lourde, Bouclier, Arme lourde, Paire d'armes},
	options={
	Armure de plates=\permodel:2,
	},
	commandgroup={\commandgroup{champion=10, banner=10, bannerallowance=50, musician=10}}
}

\showunit{
	name={Orques Kraz'Krânes sur Sangliers},
	cost=80,
	profile={Orque Kraz'Krânes : 4 4 3 4 4 1 2 1 8,
			 Sanglier : 7 3 - 3 3 1 3 1 3},
	type=Cavalerie,
	unitsize=5,
	additionalmodels=10,
	costpermodel=16,
	basesize=25x50,
	specialrules={\thunderouscharge \only{Sanglier}},
	equipment={Lance légère, Armure légère \only{\Orquea}, \mountprotection{5}},
	options={
	\optionschoice{Doit devenir (un seul choix)}{
		\Orquea=\free,
		\Orquec= \permodel:1},
	Bouclier =\permodel:3,
	Paire d'armes \only{\Orquec}=\permodel:3,
	Armure lourde \only{\Orquea}=\permodel:3,
	Lance de cavalerie \only{\Orquea} =\permodel:3,
	},
	commandgroup={\commandgroup{champion=10, banner=10, bannerallowance=50, musician=10}},
}

\showunit{
	name={Char à Sangliers},
	cost=85,
	profile={Char : - - - 5 5 4 - - -,
			 Orque Kraz'Krânes (2) : - 4 3 4 - - 2 1 7,
			 Sanglier (2): 7 3 - 3 - - 3 1 3},
	type=Char,
	unitsize=1,
	basesize=50x100,
	specialrules={\thunderouscharge \only{Sanglier}, \Orquea \only{Kraz'Krânes}, \impacthits{+1}},
	equipment={Lance de cavalerie, Armure légère, \mountprotection{5}},
	options={
		Armure Lourde=15
	}
}

\showunit{
	name={Char à Loups},
	cost=60,
	profile={Char : - - - 5 4 4 - - -,
			 Gobelin (3): - 2 3 3 - - 2 1 6,
			 Loup (2): 9 3 - 3 - - 3 1 3},
	type=Char,
	unitsize=1,
	additionalmodels=3,
	costpermodel=60,
	basesize=50x100,
	specialrules={\goba \only{Gobelin}, \insignificant, \impacthits{+1}, \lighttroops},
	equipment={Lance légère, Armure légère, Arc court, \mountprotection{6}},
}

\showunit{
	name={Barjos sur Gniark},
	cost=60,
	profile={Barjo Gobelin : - 2 3 3 3 1 3 1 5,
			 Gniark : 5 4 - 5 3 1 4 2 5},
	type=Cavalerie,
	unitsize=5,
	additionalmodels=5,
	costpermodel=10,
	basesize=20x20,
	specialrules={\oiitbites, \gobb \only{Barjo Gobelin}, \immunetopsychology, \rowsofteeth, \skirmishers, \impacthits{1} \only{Gniark}, \fly{6}},
	unitrules={\unitrule{\oiitbites}{Aucun personnage ne peut rejoindre cette unité.}
	\unitrule{\rowsofteeth}{Les Gniarks font les Attaques de Soutien à la place des Gobelins.}},
	equipment={Armure légère, , \mountprotection{6}},
}


\showunit{
	name={Meute de Gniarks},
	cost=80,
	profile={Gniark : 5 4 - 5 3 1 4 2 5},
	type=Bête de Guerre,
	unitsize=10,
	additionalmodels=30,
	costpermodel=9,
	basesize=20x20,
	specialrules={\oiitbites, \immunetopsychology, \insignificant, \theyreeverywhere},
	unitrules={\oiitbites, \gobb \only{Barjo Gobelin}, \rowsofteeth, \skirmishers, \impacthits{1} \only{Gniark}, \fly{6}},
	unitrules={\unitrule{\oiitbites}{Aucun personnage ne peut rejoindre cette unité.}
	\unitrule{\theyreeverywhere}{Lorsque cette unité fuit à la suite de sa défaite au corps à corps, elle est retirée comme perte et toutes les unités à  \distance{6} subissent une touche de Force 5 par tranche de \newrule{5} Gniarks de l'unité.}},
}

\showunit{
	name={Catapulte des Peaux-vertes},
	cost={90},
	profile={
		Catapulte des Peaux-vertes: -  - - - 7 3 - - -,
        	 Gobelin (3): 4 2 3 3 3 - 2 1 6,
			 (Garde Chiourme) : 4 3 3 3 4 +1 2 1 7},
	type=Machine de Guerre,
	unitsize=1,
	basesize=75,
	specialrules={\goba \only{Gobelin}, \insignificant},
	unitrules={\unitrule{\orqueoverseer}{Ajoute un \Orquea dans l'équipe. Ajoute un Point de Vie à la Machine, et supprime la règle \insignificant. \newline Cette machine de guerre peut choisir de perdre un Point de Vie afin de relancer un jet sur le tableau des incidents de tir.}},
	options={
	\orqueoverseer=15,
	\optionschoice{La Catapulte des Peaux-vertes doit choisir entre deux configurations}
	{\textbf \newline \textbf{Krabouilleur}=\free,
{Maximum deux exemplaires par armée standard, une seule par patrouille et quatre par grande armée.
\newline Catapulte (\distance{3}), \portee{12-60}, Force 3[9], [\multiplewounds{\ordnance}{}].},
\textbf \newline \textbf{Lance-kamikaz'}=\free,
{Maximum deux exemplaires par armée standard, un seul par patrouille et quatre par grande armée.
\newline Catapulte (\distance{1}), \portee{12-60}, Force 5, \armourpiercing{2}. Après avoir dévié le gabarit, vous pouvez lancez 1D6 puis déplacer le gabarit de cette distance en pas dans n'importe quelle direction (sauf sur des unités en combat ou des unités amies). Ceci sera la position finale du gabarit. Au lieu de toucher les figurines sous le gabarit, chacune des unités sous le gabarit subit 1D3+1 touches.
}
}
 	}
	}

\showunit{
	name={Trolls},
	cost=55,
	profile={Troll : 6 3 2 5 4 3 1 3 4},
	type=Infanterie Monstrueuse,
	unitsize=1,
	additionalmodels=9,
	costpermodel=38,
	basesize=40x40,
	specialrules={\trollbelch, \distracting \only{Troll des Rivières}, \strider{Eaux} \only{Troll des Rivières}, \fear, \regeneration{4}, \magicresistance{3} \only{Troll des Cavernes}, \stupidity},
	unitrules={\unitrule{\trollbelch}{À la place de faire ses attaques normales, la figurine peut choisir d'effectuer une seule attaque qui touche automatiquement, à Force 5 et \armourpiercing{6}.}},
	equipment={\innatedefence{4} \only{Troll de Pierre}},
	options={
	\optionschoice{Doit devenir (un seul choix)}{
		Troll Commun =\free,
		Troll des Cavernes = \permodel:8,
		Troll des Rivières = \permodel:8,},
	}
}

\showunit{
	name={Morveux},
	cost=40,
	profile={Morveux : 4 2 3 2 2 5 2 5 4},
	type=Nuée,
	unitsize=2,
	additionalmodels=4,
	costpermodel=10,
	basesize=40x40,
	specialrules={\scout, \insignificant},
	equipment={Armes de jet},
}

\showunit{
	name={Machine à Pomper},
	cost=45,
	profile={Morveux (1) : - 2 3 2 - - 2 5 4,
			 Machine à Pomper : 3D6 - - 4 4 4 - - -},
	type=Char,
	unitsize=1,
	basesize=50x100,
	specialrules={\shambolic{3D6}, \unstable, \insignificant, \impacthits{2D6}},
	unitequipment={
		\equipmentdef{\pointedsticks}{Les \impacthits{} ont \armourpiercing{2}.}
		\equipmentdef{\smellslikegreenspirit}{Gagne \hardtarget et \distracting.}
		\equipmentdef{\pursuitmode}{Durant sa phase de mouvement, lancer 1D6 additionnel pour le mouvement aléatoire de la Machine à Pomper et ignorez le résultat le plus bas.}
		\equipmentdef{\smasher}{La Force de la Machine à Pomper est de 5.}},
	equipment={Armes de jet, \mountprotection{6}},
	options={
		\pointedsticks =10,
		\smellslikegreenspirit=10,
		\pursuitmode=10,
		\smasher=15,
	}
}

\showunit{
	name={Géant},
	cost={140},
	profile={Géant : 6 3 - 6 5 6 3 * 10},
	type=Monstre,
	unitsize={1},
	basesize=50x75,
	specialrules={\specialrule{Attaques de Géant}, \immunetopsychology, \stubborn},
	options={\wardsave{6}=10},
	additional={
		\begin{description}
			\item[*Attaques de Géant :] Quand un Géant attaque au corps à corps, \newrule{au lieu d'attaquer normalement,} choisissez une unité en contact socle à socle avec lui qui va subir son attaque, lancez 1D6 et consultez ce que donne le résultat de son attaque dans l'une des deux tables ci-dessous en fonction du type de troupe de l'unité. Il est important de noter que l'attaque du Géant compte comme une attaque de corps à corps et suit ainsi normalement les règles des attaques de corps à corps. Le Géant peut également faire son \stomp{} normalement.
			\setlength\multicolsep{12.0pt plus 4.0pt minus 3pt}
			\begin{multicols}{2}
				Si l'unité est de type Infanterie, Bête de Guerre, Nuée, Machine de Guerre, Cavalerie :	
				\renewcommand{\arraystretch}{1.5}	
				\begin{center}\begin{tabular}{cl}
   					\hline
					1 & Hurle \tabularnewline
					2 & Saute \tabularnewline
					3 & Ramasse \tabularnewline
					4-6 & Frappe \tabularnewline
					\hline
				\end{tabular}\end{center}
				Si l'unité est de type Bête Monstrueuse, Infanterie Monstrueuse, Cavalerie Monstrueuse, Monstre, Monstre Monté, Char :
				\begin{center}\begin{tabular}{cl}
					\hline	
					1 & Hurle \tabularnewline
					2-3 & Tape comme un sourd \tabularnewline
					4-6 & Fracasse \tabularnewline
					\hline
				\end{tabular}\end{center}
				\renewcommand{\arraystretch}{1.2} % return to default
			\end{multicols}
			\item[Hurle :] Ni le Géant, ni l'unité sélectionnée par le Géant ne peuvent faire d'attaques au cours de cette Phase de Corps à Corps. Les attaques déjà réalisées, incluant celles réalisées simultanément avec cette attaque, ne sont pas concernées. Le camp du Géant gagne automatiquement le combat de 2. Si deux Géants opposés, ou plus, \og Hurlent \fg , le résultat du combat est un match nul.
			\item[Saute :]  L'unité sélectionnée subit 1D6 touches avec la Force du Géant, réalisées comme si elles suivaient la règle spéciale \grindingattacks{} avec la Force du Géant. Le Géant doit faire un test de Terrain Dangereux.
			\item[Ramasse :] Choisissez une figurine dans l'unité préalablement sélectionnée et en contact socle à socle avec le Géant. Cette figurine doit faire un test de Force et un test de Capacité de Combat. Pour chaque test raté, la figurine subit une touche avec la Force du Géant et suivant la règle spéciale \multiplewounds{1D3}{}.
			\item[Frappe :] Le Géant fait 2D6 attaques sur l'unité qu'il a préalablement choisie.
			\item [Tape comme un sourd :] Choisissez une figurine en contact socle à socle avec le Géant dans l'unité préalablement sélectionnée. Cette figurine doit faire un test d'Initiative. Si elle échoue, la figurine subit 2D6 blessures avec la règle spéciale \armourpiercing{6}.
			\item[Fracasse :] Choisissez une figurine dans l'unité préalablement sélectionnée et en contact socle à socle avec le Géant. Cette figurine subit une blessure avec la règle spéciale \armourpiercing{6}. Si la figurine n'a pas encore attaqué, elle ne peut pas le faire au cours de cette manche. Si la figurine a déjà réalisé ses attaques, elle ne pourra pas attaquer au cours du tour à venir de l'autre joueur.
		\end{description}
	}
}

\rareunitstitle

\showunit{
	name={Embrocheur},
	cost=45,
	profile={Embrocheur : - - - - 7 3 - - -,
			 Gobelin (3): 4 2 3 3 3 - 3 1 6},
	type=Machine de Guerre,
	unitsize=1,
	basesize=60,
	specialrules={\goba \only{Gobelin}, \insignificant},
	unitequipment={\equipmentdef{Embrocheur}{Baliste, \portee{48}, Force 6, \multiplewounds{1D3}{}, \armourpiercing{6}}},
}



\showunit{
	name={Duo de Gniarks Broyeurs},
	cost=70,
	profile={Duo de Gniarks Broyeurs : 3D6 - - 6 4 3 3 2 3},
	type=Bête Monstrueuse,
	unitsize=1,
	basesize=60,
	specialrules={\hardtarget, \shambolic{3D6}, \lookatemgo, \ricochet{2D6}},
	unitrules={\unitrule{\lookatemgo}{Après avoir touché leur première unité de la partie, cette unité remplace leur règle \shambolic{3D6} par \shambolic{3D6, Nawak}.}},
}

\showunit{
	name={Araignée Titanesque},
	cost=225,
	profile={Araignée Titanesque : 7 4 - 5 6 8 4 8 -,
			 Gobelin (8): - 2 3 3 - - 2 1 6},
	type=Monstre Monté,
	unitsize=1,
	basesize=100x150,
	specialrules={\poisonedattacks \only{Araignée Titanesque}, \venomousfangs, \gobc \only{Gobelin}, \strider{}, \immunetopsychology, \swiftstride, \stubborn},
	unitequipment={\equipmentdef{Lance-Toile}{Catapulte (\distance{3}), \portee{6-36}, Force 3. Les unités touchées subissent un malus de -1D3 en Initiative. De plus elles traitent tous les \emph{Terrain Dangereux (1)} comme des \emph{Terrain Dangereux (2)} et tous les Terrains (y compris les Terrains Découverts) comme des \emph{Terrain Dangereux (1)}. Les effets de plusieurs Lances-Toile ne se combinent pas.}},
	equipment={Lance légère \only{Gobelin}, Arc court \only{Gobelin}, \innatedefence{4}},
	options={Lance-Toile=30}
}

\showunit{
	name={Idole des Dieux Verts},
	cost=230,
	profile={Idole : 6 2 - 6 8 6 2 3 8},
	type=Monstre,
	unitsize=1,
	basesize=100x100,
	specialrules={\crushattack, \immunetopsychology, \smashemflat, \impacthits{1D3}, \wevegotthegreenlight},
	unitrules={
	\unitrule{\wevegotthegreenlight}{Toutes les unités alliées à \distance{8} peuvent relancer leur jet de distance de charge si l'Idole des Dieux Verts déclare une charge ce tour-ci.}
	\unitrule{\smashemflat}{Toutes les unités alliées à \distance{8} gagnent la \devastatingcharge si l'Idole des Dieux Verts est engagée au corps à corps ce round-ci.}},
	equipment={\innatedefence{5}},
	options={Porteur de la Grande Bannière=50}
}




\mountstitle

La section Montures est réservée aux Personnages montés. Les figurines non-Personnage montées suivent les règles données sous leur entrée respective.


\showunit{
	name={Vouivre},
	cost=-,
	profile={Vouivre : 4 5 - 6 5 4 3 3 6},
	type=Bête Monstrueuse,
	unitsize=1,
	basesize=50x50,
	specialrules={\poisonedattacks, \venomousfangs, \largetarget, \fear, \fly{8}},
}


\showunit{
	name={Sanglier},
	cost=-,
	profile={Sanglier : 7 3 - 3 3 - 3 1 3},
	type=Bête de Guerre,
	unitsize=1,
	basesize=25x50,
	specialrules={\thunderouscharge},
	equipment={\mountprotection{5}},
}


\showunit{
	name={Char à Sangliers},
	cost=-,
	profile={Char : - - - 5 5 4 - - -,
			 Orque Kraz'Krânes (1) : - 4 3 4 - - 2 1 7,
			 Sanglier (2): 7 3 - 3 - - 3 1 3},
	type=Char,
	unitsize=1,
	basesize=50x100,
	specialrules={\thunderouscharge \only{Sanglier}, \Orquea \only{Kraz'Krânes}, \impacthits{+1}},
	equipment={Lance de cavalerie, Armure légère, \mountprotection{5}},
}


\showunit{
	name={Loup},
	cost=-,
	profile={Loup : 9 3 - 3 3 1 3 1 3},
	type=Bête de Guerre,
	unitsize=1,
	basesize=25x50,
	specialrules={\fastcavalry},
	equipment={\mountprotection{6}},
}

\showunit{
	name={Char à Loups},
	cost=-,
	profile={Char : - - - 5 4 4 - - -,
			 Gobelin (2): - 2 3 3 - - 2 1 6,
			 Loup (2): 9 3 - 3 - - 3 1 3},
	type=Char,
	unitsize=1,
	basesize=50x100,
	specialrules={\goba \only{Gobelin}, \insignificant, \lighttroops, \impacthits{+1}},
	equipment={Lance légère, Armure légère, Arc court, \mountprotection{6}},
}


\showunit{
	name={Gros-Gniark},
	cost=-,
	profile={Gros-Gniark : 5 4 - 6 4 3 3 3 3},
	type=Bête Monstrueuse,
	unitsize=1,
	basesize=40x40,
	specialrules={\bouncers, \hardtarget, \oiitbites, \impacthits{1}, \fly{6}},
	equipment={\mountprotection{6}},
	unitrules={\unitrule{\bouncers}{La figurine peut uniquement rejoindre une unité de Gros-Gniarks ou de Barjos sur Gniark (ignorez les restrictions des règles spéciales \skirmishers et \oiitbites)}.
	\unitrule{\oiitbites}{Aucun personnage ne peut rejoindre cette unité.}}
}



\showunit{
	name={Araignée},
	cost=-,
	profile={Araignée : 7 3 - 3 3 1 4 1 2},
	type=Bête de Guerre,
	unitsize=1,
	basesize=25x50,
	specialrules={\poisonedattacks, \fastcavalry, \scout, \strider{}},
	equipment={\mountprotection{6}},
}


\showunit{
	name={Gigaraignée},
	cost=-,
	profile={Gigaraignée : 7 3 - 4 4 3 4 3 7},
	type=Bête Monstrueuse,
	unitsize=1,
	basesize=50x50,
	specialrules={\poisonedattacks, \strider{}},
	equipment={\mountprotection{5}},
}

\showunit{
	name={Araignée Titanesque},
	cost=-,
	profile={Araignée Titanesque : 7 4 - 5 6 8 4 8 -,
			 Gobelin (8): - 2 3 3 - - 2 1 6},
	type=Monstre Monté,
	unitsize=\oneofakind,
	basesize=100x150,
	specialrules={\poisonedattacks \only{Araignée Titanesque}, \venomousfangs, \gobc \only{Gobelin}, \strider{}, \immunetopsychology, \insignificant, \swiftstride, \stubborn},
	unitrules={\unitrule{\spidermothershrine}{Toutes les figurines alliées avec la règle \specialrule{Canalisation} à \distance{12} d'un \spidermothershrine ajoutent +2 au lieu de +1 pour le jet de canalisation. Un Sorcier monté sur un \spidermothershrine devient \pathmaster{}.}},
	equipment={Lance légère \only{Gobelin}, Arc court \only{Gobelin}, \innatedefence{4}},
	options={Si monté par un Sorcier{,} \spidermothershrine = 40}
}


\end{document}