

\documentclass[a4paper,8pt]{extarticle} % extarticle allows to use font size of 8pt.

\usepackage[a4paper, top=1.6cm, bottom=2cm, left=1.6cm, right=1.6cm]{geometry} % Marge reduction.

%% Language specific package
\usepackage[french]{babel}
\frenchbsetup{StandardLists=true} % Necessary to use enumitem with babel/french.

%% Font and typing packages
\usepackage{fontspec}
\setmainfont[
	Ligatures=TeX,
	ItalicFont={Dancing Script},
	BoldItalicFont={Dancing Script}
	]{PT Serif} % default is Latin Modern
\newfontfamily\antiquefont[Ligatures=TeX]{Caslon Antique} % fancy font
\usepackage{microtype}			% Greatly improves general appearance of the text.
\usepackage{SIunits}			% Unit appearance.
\usepackage{xspace}				% Define commands that appear not to eat spaces.
\usepackage{ulem}				% To cross words out. Use \sout{}.

%% Array utilities
\usepackage{array}				% Additionnal options for arrays.
\usepackage{colortbl}			% Additionnal options for coloring arrays.
\usepackage[table]{xcolor}		% Auto alternate grey-white rows.
\usepackage[export]{adjustbox}		% Centered pics in tables

%% List utilities
\usepackage[inline]{enumitem}   % Display inline lists.
\usepackage{etoolbox}           % General utility. Good for lists for instance.
\usepackage{xparse}             % List utilities.
\usepackage{datatool}	% Handling alphabetical order.

%% Frames
\usepackage{framed}				% Boxes.
\usepackage[framemethod=TikZ]{mdframed}% For fancy frames.
\usepackage{tikz}				% For fancy frames.
\usepackage{wrapfig}			% Fancy insertion of pics in text.

%% Page utilities
\usepackage{multicol}			% Allows to divide a part of the page in multiple columns.
	
%% Others
\usepackage{keyval}             % Used to create maps of commands/labels/objects.
	\makeatletter                  % Mandatory for the usage of keyval.
\usepackage{xstring}            % String parsing, cutting, etc.
\usepackage{hyperref} % Links in PDF.


%%% Update of the dotfill command to always get dots

\newcommand{\predotfill}{\penalty0\hbox{}\nobreak}%


%%% Command to avoid typing \xspace when creating a new name macro

\newcommand{\newnamemacro}[2]{\newcommand{#1}{#2}} % \xspace removed for compatibility with alphabetical ordering

%%% Language specific stuff


%%% Commands %%%

\newcommand{\addtosortedlist}[1]{%
	\protected@edef\textarg{#1}%
	\protected@edef\textwithoutspaces{\expandafter\removespaces\expandafter{\textarg}}%
	\substitute\textwithoutspaces{É}{e}% Most used special characters of the language, and equivalent for alphabetical ordering
	\substitute\textwithoutspaces{È}{e}%
	\substitute\textwithoutspaces{Ê}{e}%
	\substitute\textwithoutspaces{é}{e}%
	\substitute\textwithoutspaces{è}{e}%
	\substitute\textwithoutspaces{ê}{e}%
	\substitute\textwithoutspaces{À}{a}%
	\substitute\textwithoutspaces{à}{a}%
	\substitute\textwithoutspaces{ù}{u}%
	\expandafter\sortitem\expandafter[\textwithoutspaces]{#1}%
}%


%%% Labels %%%

% Profile

\newcommand{\labels@M}{M}
\newcommand{\labels@WS}{CC}
\newcommand{\labels@BS}{CT}
\newcommand{\labels@S}{F}
\newcommand{\labels@T}{E}
\newcommand{\labels@W}{PV}
\newcommand{\labels@I}{I}
\newcommand{\labels@A}{A}
\newcommand{\labels@Ld}{Cd}
\newcommand{\labels@Invocation}{Invocation} % For Vampire Covenant profiles

\newcommand{\Strength}{Force}

% Technical

\newcommand{\labels@range}{Portée}
\newcommand{\labels@point}{pt}
\newcommand{\labels@points}{pts}
\newcommand{\labels@only}{uniquement}
\newcommand{\labels@magic}{Magie}
\newcommand{\labels@pathsused}{Génère ses sorts dans la Discipline}
\newcommand{\labels@model}{figurine}
\newcommand{\labels@models}{figurines}
\newcommand{\labels@Singlemodel}{Figurine \textbf{seule}}

% Unit entry labels

\newcommand{\labels@basesize}{Socle}
\newcommand{\labels@trooptype}{Type de troupe}
\newcommand{\labels@specialrules}{Règles spéciales}
\newcommand{\labels@alignment}{Allégeance}
\newcommand{\labels@equipment}{Équipement}
\newcommand{\labels@weapons}{Armes}
\newcommand{\labels@armour}{Armure}
\newcommand{\labels@options}{Options}
\newcommand{\labels@commandgroup}{État-Major}
\newcommand{\labels@mounts}{Montures}
\newcommand{\labels@specialequipment}{Équipement spécial}

% Command groups

\newcommand{\labels@champion}{Champion}
\newcommand{\labels@standardbearer}{Porte-étendard}
\newcommand{\labels@musician}{Musicien}
\newcommand{\labels@singlebannerallowance}{Une seule unité de ce type peut prendre une Bannière magique}
\newcommand{\labels@condsinglebannerallowance}{Une seule unité de ce type peut prendre une Bannière magique si}
\newcommand{\labels@bannerallowance}{Peut prendre une Bannière Magique}
\newcommand{\labels@veteranstandardbearer}{Peut devenir Porte-étendard Vétéran}
\newcommand{\labels@championallowance}{Peut prendre une Arme Magique}

% Titles

\newcommand{\labels@lords}{Seigneurs}
\newcommand{\labels@heroes}{Héros}
\newcommand{\labels@coreunits}{Unités de base}
\newcommand{\labels@specialunits}{Unités spéciales}
\newcommand{\labels@rareunits}{Unités rares}
\newcommand{\labels@armywiderules}{Règles communes de l'armée}
\newcommand{\labels@armyspecialrules}{Règles spéciales de l'armée}
\newcommand{\labels@armoury}{Armurerie}
\newcommand{\labels@magicalitems}{Objets magiques}
\newcommand{\labels@magicalweapons}{Armes magiques}
\newcommand{\labels@magicalarmour}{Armures magiques}
\newcommand{\labels@talismans}{Talismans}
\newcommand{\labels@enchanteditems}{Objets enchantés}
\newcommand{\labels@arcaneitems}{Objets cabalistiques}
\newcommand{\labels@magicalbanners}{Bannières magiques}
\newcommand{\labels@quickrefsheet}{Fiche de référence}
\newcommand{\labels@changelog}{Change Log}

\newcommand{\labels@lordsInitial}{S}
\newcommand{\labels@heroesInitial}{H}
\newcommand{\labels@coreunitsInitial}{B}
\newcommand{\labels@specialunitsInitial}{S}
\newcommand{\labels@rareunitsInitial}{R}
\newcommand{\labels@mountsInitial}{M}


% Titlepage

\newcommand{\labels@fantasybattles}{Batailles Fantastiques}
\newcommand{\labels@NinthAge}{Le 9\ieme Âge}
\newcommand{\labels@creators}{Une collaboration des créateurs de l'ETC et du Swedish Comp System}
\newcommand{\labels@introduction}{%
\noindent {\Largerfontsize\textbf{Note des traducteurs}}
\vspace{0.5cm}

Nous souhaitons remercier chaleureusement l'équipe à l'initiative du 9\ieme Âge pour leur motivation et leur travail continu pour faire vivre notre passion. Nous espérons que ce jeu saura développer les qualités pour plaire au plus grand nombre et réunir les joueurs, amateurs comme habitués des tournois, autour de règles amusantes et équilibrées, pour finalement s'imposer comme un standard du jeu de figurines. Une grande ambition qui ne pourra s'accomplir que \textbf{grâce à vous}, la communauté, via des retours constructifs, afin de modeler le jeu selon nos désirs. N'étant \textbf{en aucun cas à but lucratif}, le 9\ieme Âge part avec un avantage considérable. Les règles des éventuelles nouvelles sorties ne seront pas dictées par le besoin de vendre ces nouveautés. Vous pouvez choisir et acheter vos figurines où bon vous semble, il n'y a pas un unique revendeur toléré. Vous n'êtes pas bloqués dans une spirale infernale où pour continuer à jouer à un jeu, dans lequel vous vous êtes tant investis, vous devez payer toujours plus cher pour entretenir votre collection. Enfin, vous pouvez être assurés que tant que 9\ieme Âge sera joué, vous disposerez d'un \textbf{support continu et régulier}, celui-ci étant offert par la communauté.

Nous attirons votre attention sur le fait que ce jeu en est encore à ses débuts et dans un \textbf{stade de développement}. Ce document correspond à une version de brouillon \textbf{\og{} beta \fg{}}, dont le but et de tester le jeu et le modifier jusqu'à atteindre une version satisfaisante. Attendez-vous donc à trouver des déséquilibres, des incohérences, et à obtenir des mises à jour régulières avec éventuellement des changements importants. N'hésitez pas à nous donner vos avis ! Ce livre d'armée n'est utilisable qu'en compagnie du livre de Règles et du livre de Magie.

Concernant la traduction en elle-même, nous avons fait de notre mieux pour vous offrir une version de qualité, dont nous espérons qu'elle surpasse celle de la version originale ! Si vous constatez des coquilles, des erreurs, merci de nous les signaler en nous contactant sur le forum du 9\ieme Âge, dans le \textbf{sous-forum français} (\url{http://www.the-ninth-age.com/index.php?board/117-french/}). Vous y trouverez aussi les dernières mises à jour. \textbf{En cas de conflit d'interprétation avec la version originale, la version originale fait référence}.

\vspace{0.5cm}
Que ce jeu vous apporte d'innombrables heures de plaisir partagé !

\vspace{0.7cm}
\noindent {\Largerfontsize\textbf{Les traducteurs}}
\vspace{0.1cm}

\ifdef{\translationteam}{
	\begin{multicols}{3}
	\begin{itemize}
		\translationteam
	\end{itemize}
	\end{multicols}
}{}
}
\newcommand{\labels@secondpageannouncement}{%
	\labels@fantasybattles{} : \labels@NinthAge{} est un jeu créé et entretenu par la communauté qui met en scène des affrontements de figurines. Toutes les règles sont disponibles gratuitement sur le site suivant. Vos retours et suggestions sont les bienvenus.
	\newline\url{http://www.the-ninth-age.com/}
}
\newcommand{\labels@rulechanges}{%
	Les changements de règles entre versions sont colorés comme ce paragraphe. Une liste en anglais de ces changements par version est ajoutée à la fin de cet ouvrage.
}
\newcommand{\labels@latexcredit}{Document réalisé à l'aide de \LaTeX .}


%%% Technical commands

\newcommand{\only}[1]{(#1 uniquement)}
\newcommand{\free}{gratuit}
\newcommand{\upto}{jusqu'à}
\newcommand{\Upto}{Jusqu'à}
\newcommand{\unlimited}{sans limite de pts}
\newcommand{\permodel}{/fig.}
\newcommand{\listlastchoice}{ ou}
\newcommand{\notif}[1]{(pas #1)}
\newcommand{\wordand}{et}
\newcommand{\wordwith}{avec}
\newcommand{\ifNmodelsorless}[1]{(#1 figurines ou moins)}
\newcommand{\unitwith}{unité avec}
\newcommand{\From}{De} % From ... to ... models
\newcommand{\wordto}{à}
\newcommand{\wordAll}{Tous}
\newcommand{\spacebeforecolon}{ } % French put a space before colons
\newcommand{\minprice}{Coût min. :}
\newcommand{\mincostfor}{Coût min. pour}
\newcommand{\maxunitsize}{Taille max.}
\newcommand{\additionalfigscost}{Les figurines additionnelles coûtent}


%%% Special rules %%%

\newcommand{\ambush}{Embuscade}
\newcommand{\armourpiercing}[1]{Perforant\ifblank{#1}{}{ (#1)}}
\newcommand{\bodyguard}[1]{Garde du Corps\ifblank{#1}{}{ (#1)}}
\newcommand{\breathweapon}[1]{Attaque de Souffle\ifblank{#1}{}{ (#1)}}
\newcommand{\channel}{Canalisation}
\newcommand{\crushattack}{Attaque Écrasante}
\newcommand{\devastatingcharge}{Charge Dévastatrice}
\newcommand{\distracting}{Distrayant}
\newcommand{\engineer}{Ingénieur}
\newcommand{\ethereal}{Éthéré}
\newcommand{\fastcavalry}{Cavalerie Légère}
\newcommand{\fear}{Peur}
\newcommand{\fightinextrarank}{Combat avec un Rang Supplémentaire}
\newcommand{\fireborn}{Né du Feu}
\newcommand{\flamingattacks}{Attaques Enflammées}
\newcommand{\flammable}{Inflammable}
\newcommand{\lighttroops}{Troupes Légères}
\newcommand{\frenzy}{Frénésie}
\newcommand{\fly}[1]{Vol\ifblank{#1}{}{ (#1)}}
\newcommand{\grindingattacks}[1]{Attaques de Broyage\ifblank{#1}{}{ (#1)}}
\newcommand{\hardtarget}{Camouflé}
\newcommand{\hatred}{Haine}
\newcommand{\hellfire}{Flammes de l'Enfer}
\newcommand{\hidden}{Caché}
\newcommand{\holyattacks}{Attaques Divines}
\newcommand{\immunetopsychology}{Immunisé à la Psychologie}
\newcommand{\impacthits}[1]{Touches d'Impact\ifblank{#1}{}{ (#1)}}
\newcommand{\insignificant}{Insignifiant}
\newcommand{\largetarget}{Grande Cible}
\newcommand{\lethalstrike}{Coup Fatal}
\newcommand{\lightningattacks}{Attaques Foudroyantes}
\newcommand{\lightningreflexes}{Réflexes Foudroyants}
\newcommand{\magicresistance}[1]{Résistance à la Magie\ifblank{#1}{}{ (#1)}}
\newcommand{\magicalattacks}{Attaques Magiques}
\newcommand{\metalshifting}{Fusion du Métal}
\newcommand{\moveorfire}{Mouvement ou Tir}
\newcommand{\multipleshots}[1]{Tirs Multiples\ifblank{#1}{}{ (#1)}}
\newcommand{\multiplewounds}[2]{Blessures Multiples\ifblank{#1}{}{ (#1\ifblank{#2}{)}{, #2)}}}
\newcommand{\notaleader}{Pas un Meneur}
\newcommand{\otherworldly}{D'Outre-Monde}
\newcommand{\pathmaster}[1]{Maître de la Discipline\ifblank{#1}{}{ (#1)}}
\newcommand{\poisonedattacks}{Attaques Empoisonnées}
\newcommand{\quicktofire}{Tir Rapide}
\newcommand{\randommovement}[1]{Mouvement Aléatoire\ifblank{#1}{}{ (#1)}}
\newcommand{\randomattacks}[1]{Attaques Aléatoires\ifblank{#1}{}{ (#1)}}
\newcommand{\regeneration}[1]{Régénération\ifblank{#1}{}{ (#1+)}}
\newcommand{\reload}{Rechargez !}
\newcommand{\requirestwohands}{Arme à deux Mains}
\newcommand{\scythes}{Faux}
\newcommand{\scout}{Éclaireur}
\newcommand{\scouts}{Éclaireurs}
\newcommand{\stomp}[1]{Piétinement\ifblank{#1}{}{ (#1)}}
\newcommand{\strider}[1]{Guide\ifblank{#1}{}{ (#1)}}
\newcommand{\stubborn}{Tenace}
\newcommand{\stupidity}{Stupidité}
\newcommand{\skirmisher}{Tirailleur}
\newcommand{\skirmishers}{Tirailleurs}
\newcommand{\sweepingattack}{Attaque au Passage}
\newcommand{\swiftstride}{Rapide}
\newcommand{\thunderouscharge}{Charge Tonitruante}
\newcommand{\terror}{Terreur}
\newcommand{\toxicattacks}{Attaques Toxiques}
\newcommand{\unbreakable}{Indémoralisable}
\newcommand{\undead}{Mort-Vivant}
\newcommand{\unstable}{Instable}
\newcommand{\unwieldy}{Encombrant}
\newcommand{\vanguard}{Avant-Garde}
\newcommand{\volleyfire}{Tir de Volée}
\newcommand{\warplatform}{Plateforme de Guerre}
\newcommand{\wardsave}[1]{Sauvegarde Invulnérable\ifblank{#1}{}{ (#1+)}}
\newcommand{\weaponmaster}{Maître d'Ar\-mes}
\newcommand{\wizardconclave}[1]{Conclave de Sorciers\ifblank{#1}{}{ (#1)}}


%%% Magic %%%

\newnamemacro{\Pathof}{Discipline}

\newcommand{\battle}{Commune}
\newcommand{\alchemy}{de l'Alchimie}
\newcommand{\death}{de la Mort}
\newcommand{\fire}{du Feu}
\newcommand{\heavens}{des Cieux}
\newcommand{\light}{de la Lumière}
\newcommand{\nature}{de la Nature}
\newcommand{\shadows}{des Ombres}
\newcommand{\wilderness}{de la Sauvagerie Bestiale}
\newcommand{\butchery}{de la Boucherie}
\newcommand{\change}{du Changement}
\newcommand{\thebiggreengods}{des Grands Dieux Verts}
\newcommand{\thelittlegreengods}{des Petits Dieux Verts}
\newcommand{\blackmagic}{de la Magie Noire}
\newcommand{\disease}{de la Maladie}
\newcommand{\lust}{de la Luxure}
\newcommand{\necromancy}{de la Nécromancie}
\newcommand{\ruin}{de la Ruine}
\newcommand{\forge}{de la Forge}
\newcommand{\sands}{des Sables}
\newcommand{\whitemagic}{de la Magie Blanche}

\newcommand{\anyofthebattlemagic}{dans n'importe laquelle des Disciplines Communes}

\newcommand{\magiclevel}[1]{\ifnumcomp{#1}{<}{3}{Sorcier Apprenti}{Maître Sorcier} Niveau #1}
\newcommand{\Level}{Niveau}

\newcommand{\wizard}{Sorcier}
\newcommand{\wizards}{Sorciers}

\newcommand{\boundspell}[1]{Objet de Sort, Puissance #1}


%%% Other rules %%%

\newcommand{\armoursave}{Sauvegarde d'Armure}
\newcommand{\firstinrank}{Au Premier Rang}
\newcommand{\hardcover}{Couvert Lourd}
\newcommand{\holdyourground}{Tenez les Rangs}
\newcommand{\inspiringpresence}{Présence Charismatique}
\newcommand{\lightcover}{Couvert Léger}
\newcommand{\monstrousrank}{Rang Monstrueux}
\newcommand{\ordnance}{Artillerie}
\newcommand{\parry}{Parade}
\newcommand{\raisewounds}{Ressusciter des Figurines}
\newcommand{\recoverwounds}{Récupérer des PVs}
\newcommand{\aideddispel}{Dissipation Assistée}
\newcommand{\rnf}{ordinaires}
\newcommand{\general}{Général}


%%% Equipment %%%

\newcommand{\innatedefence}[1]{Protection Innée\ifblank{#1}{}{~(#1+)}}
\newcommand{\mountsprotection}[1]{Protection de Monture\ifblank{#1}{}{~(#1+)}}
\newcommand{\la}{Armure Légère}
\newcommand{\ha}{Armure Lourde}
\newcommand{\platearmour}{Armure de Plates}
\newcommand{\hw}{Arme de Base}
\newcommand{\pw}{Paire d'Armes}
\newcommand{\spear}{Lance}
\newcommand{\halberd}{Hallebarde}
\newcommand{\gw}{Arme Lourde}
\newcommand{\lance}{Lance de Cavalerie}
\newcommand{\lightlance}{Lance Légère}
\newcommand{\shield}{Bouclier}
\newcommand{\barding}{Caparaçon}
\newcommand{\throwingweapons}{Armes de Jet}
\newcommand{\shortbow}{Arc Court}
\newcommand{\flail}{Fléau}

\newcommand{\cannon}{Canon}
\newcommand{\catapult}{Catapulte}
\newcommand{\volleygun}{Batterie de Tir}
\newcommand{\boltthrower}{Baliste}
\newcommand{\artilleryweapon}{Arme d'Artillerie}


%%% Troop types %%%

\newcommand{\characters}{Personnages}
\newcommand{\infantry}{Infanterie}
\newcommand{\monstrousinfantry}{Infanterie Monstrueuse}
\newcommand{\cavalry}{Cavalerie}
\newcommand{\monstrouscavalry}{Cavalerie Monstrueuse}
\newcommand{\swarm}{Nuée}
\newcommand{\swarms}{Nuées}
\newcommand{\warbeast}{Bête de Guerre}
\newcommand{\warbeasts}{Bêtes de Guerre}
\newcommand{\monster}{Monstre}
\newcommand{\monsters}{Monstres}
\newcommand{\monstrousbeast}{Bête Monstrueuse}
\newcommand{\monstrousbeasts}{Bêtes Monstrueuses}
\newcommand{\chariot}{Char}
\newcommand{\chariots}{Chars}
\newcommand{\riddenmonster}{Monstre Monté}
\newcommand{\riddenmonsters}{Monstres Montés}
\newcommand{\warmachine}{Machine de Guerre}
\newcommand{\warmachines}{Machines de Guerre}


%%% Terrain %%%

\newcommand{\water}{Eaux peu profondes}


%%% Profile wording

\newcommand{\oneofakind}{Uni\-que}
\newcommand{\onechoiceonly}{(un seul choix)}
\newcommand{\onfootonly}{(à pied seulement)}
\newcommand{\closecombatonly}{seulement au Corps à Corps}
\newcommand{\Xmodelsorless}[1]{(#1 figurines ou moins)}
\newcommand{\magicalitemsallowance}{Peut prendre des Objets Magiques}
\newcommand{\magicalweaponallowance}{Peut prendre une Arme Magique}
\newcommand{\notmagicalarmour}{(mais pas d'Armure Magique)}
\newcommand{\anyofthefollowing}{\optionschoice{Peut prendre :}}
\newcommand{\weapononechoice}{\optionschoice{Peut prendre une arme \onechoiceonly{} :}}
\newcommand{\weaponschoice}{\optionschoice{Peut prendre des armes :}}
\newcommand{\shootingweapononechoice}{\optionschoice{Peut prendre une arme de tir \onechoiceonly{} :}}
\newcommand{\combatweapononechoice}{\optionschoice{Peut prendre une arme de corps à corps \onechoiceonly{} :}}
\newcommand{\armouronechoice}{\optionschoice{Peut prendre une armure \onechoiceonly{} :}}
\newcommand{\magiclevelchoice}{\optionschoice{Peut devenir au choix :}}
\newcommand{\bsboption}{Peut devenir Porteur de la Grande Bannière}
\newcommand{\mayupgradeto}{Peut être amélioré en}
\newcommand{\mustbecomeoneofthefollowing}{\optionschoice{Doit devenir un choix parmi :}}
\newcommand{\maybecomeoneofthefollowing}{\optionschoice{Peut devenir un choix parmi :}}
\newcommand{\maytakeoneofthefollowing}{\optionschoice{Peut prendre un choix parmi :}}
\newcommand{\maytakeuptotwoofthefollowing}{\optionschoice{Peut prendre jusqu'à deux choix parmi :}}
\newcommand{\maygain}{Peut gagner la règle}
\newcommand{\maytake}{Peut prendre}
\newcommand{\maytakeashield}{Peut prendre un Bouclier}
\newcommand{\maytakela}{Peut prendre une Armure Légère}
\newcommand{\maytakeha}{Peut prendre une Armure Lourde}
\newcommand{\maytakemountsprotectionX}[1]{Peut prendre une \mountsprotection{#1}}
\newcommand{\maytakeagw}{Peut prendre une Arme Lourde}
\newcommand{\maytakeaspear}{Peut prendre une Lance}
\newcommand{\maytakepw}{Peut prendre une Paire d'Armes}
\newcommand{\maytakethrowingweapons}{Peut prendre des Armes de Jet}
\newcommand{\maytakebarding}{Peut prendre un Caparaçon}
\newcommand{\replaceshieldwithhalberd}{Remplacer le Bouclier par une Hallebarde}
\newcommand{\maybecome}{Peut devenir}

\newcommand{\maytakeonechoiceonly}{\optionschoice{\maytake{} \onechoiceonly{}\spacebeforecolon{}:}}

\newcommand{\mountssectionannouncement}{%
La section Montures concerne les montures de Personnages. Les montures pour non-Personnages suivent les règles données dans leur description d'unité.
}

%%% Commands to handle strings, better than xstring to handle commands inside the strings %%%

\newcommand{\substitute}[3]{%
  \protected@edef\sub@temp{#1}%
  \saveexpandmode
  \expandarg\StrSubstitute{\sub@temp}{#2}{#3}[#1]%
  \restoreexpandmode
}

\newcommand{\splitatstar}[3]{%
  \protected@edef\split@temp{#1}%
  \saveexpandmode
  \expandarg\StrCut{\split@temp}{*}#2#3%
  \restoreexpandmode
}

\newcommand{\splitatinf}[3]{%
  \protected@edef\split@temp{#1}%
  \saveexpandmode
  \expandarg\StrCut{\split@temp}{<}#2#3%
  \restoreexpandmode
}

\newcommand{\splitatequal}[3]{%
  \protected@edef\split@temp{#1}%
  \saveexpandmode
  \expandarg\StrCut{\split@temp}{=}#2#3%
  \restoreexpandmode
}

\newcommand{\ifsubstring}[4]{%
  \protected@edef\split@temp{#1}%
  \protected@edef\split@tempbis{#2}%
  \saveexpandmode
  \expandarg\IfSubStr{\split@temp}{\split@tempbis}{#3}{#4}%
  \restoreexpandmode
}

\def\removespaces#1{\zap@space#1 \@empty}

%%% Commands for alphabetical ordering %%%

\newcommand{\sortitem}[2][\relax]{%
	\DTLnewrow{list}% Create a new entry
	\ifx#1\relax%
		\DTLnewdbentry{list}{sortlabel}{#2}% Add entry sortlabel (no optional argument)
	\else%
		\DTLnewdbentry{list}{sortlabel}{#1}% Add entry sortlabel (optional argument)
	\fi%
		\DTLnewdbentry{list}{description}{#2}% Add entry description
}
\newenvironment{sortedlist}{%
	\DTLifdbexists{list}{\DTLcleardb{list}}{\DTLnewdb{list}}% Create new/discard old list
}{%
	\DTLsort{sortlabel}{list}% Sort list
	\begin{itemize*}[label={}, itemjoin={,}]%
		\DTLforeach*{list}{\theDesc=description}{%
		\item\theDesc}% Print each item
	\end{itemize*}%
}

\pdfstringdefDisableCommands{\def\textcolor#1{}}

% See language specific file for \addtosortedlist

%%% Database for automatic Quick Ref Sheet %%%

\DTLnewdb{profiles} % Database containing name, category, multiprofile number, profilename (if multi), caraclist, trooptype, invocation for CV.
\newcommand{\profilecategory}{\labels@lords} % Will be updated in relevant categories

\newcommand{\profiledtbfillname}[1]{\DTLnewdbentry{profiles}{name}{#1}}
\newcommand{\profiledtbfillcategory}[1]{\DTLnewdbentry{profiles}{category}{#1}}
\newcommand{\profiledtbfilltrooptype}[1]{\DTLnewdbentry{profiles}{trooptype}{#1}}
\newcommand{\profiledtbfillinvocation}[1]{\DTLnewdbentry{profiles}{invocation}{#1}}
\newcommand{\profiledtbfillprofile}[1]{\DTLnewdbentry{profiles}{profile}{#1}}
\newcommand{\profiledtbfillmultipleprofile}[1]{\DTLnewdbentry{profiles}{multipleprofile}{#1}}

\newcommand{\void}[1]{}
\newcounter{multiprofilecounter}

\newcommand{\profiledtbfillcarac}[1]{%
	\profiledtbfillprofile{#1}
	\parselist{#1}{\locallists@profileslist}% Split of the different profiles in the case of a multiprofile.
	\setcounter{multiprofilecounter}{0}%
	\forlistloop{\stepcounter{multiprofilecounter}\void}{\locallists@profileslist}%
	\expandafter\profiledtbfillmultipleprofile\expandafter{\number\value{multiprofilecounter}}
}


%%% Technical commands %%%

\newcommand{\newrule}{\textcolor{green!50!black}}
\newcommand{\removedrule}[1]{\textcolor{green!50!black}{\sout{#1}}}
\newcommand{\starsymbol}{$\star$}
\newcommand{\refsymbol}{$^\star$}

\newcommand{\inch}{\arcsecond}
\newcommand{\foot}{\arcminute}
\newcommand{\range}[1] {\labels@range~\unit{#1}{\inch}}
\newcommand{\distance}[1] {\unit{#1}{\inch}}
\newcommand{\result}[1] {\texttt{'}#1\texttt{'}}


%%% Fonts and sizes %%%

\newcommand{\bigtitle}[1]{\vspace*{-1.5cm}\section*{}\noindent\begin{center}\Hugefontsize\textbf{\antiquefont\expandafter\uppercase\expandafter{#1}}\end{center}}

\newcommand{\subtitle}[1]{\subsection*{}\noindent{\hugefontsize\antiquefont #1}}

\newcommand{\subsubtitle}[1]{\subsubsection*{}\noindent{\Largerfontsize\antiquefont #1}}

\newcommand{\verysmallfontsize}{\fontsize{4}{4.8}\selectfont}
\newcommand{\smallfontsize}{\fontsize{6}{7.2}\selectfont}
\newcommand{\normalfontsize}{\fontsize{8}{9.6}\selectfont}
\newcommand{\largefontsize}{\fontsize{10}{12}\selectfont}
\newcommand{\largerfontsize}{\fontsize{12}{14.4}\selectfont}
\newcommand{\Largefontsize}{\fontsize{14}{16.8}\selectfont}
\newcommand{\Largerfontsize}{\fontsize{15}{18}\selectfont}
\newcommand{\hugefontsize}{\fontsize{18}{21.6}\selectfont}
\newcommand{\Hugefontsize}{\fontsize{25}{30}\selectfont}

\newcommand{\unitentryformat}[1]{\textit{\largefontsize{#1}}}
\newcommand{\textIT}[1]{\textit{\largefontsize{#1}}}


%%% Titles %%%

\newcommand{\lordstitle}{\def\logolocalpath{../Layout/pics/logo_lord.png}\bigtitle{\labels@lords}}
\newcommand{\heroestitle}{%
\def\logolocalpath{../Layout/pics/logo_hero.png}%
\clearpage\bigtitle{\labels@heroes}%
\renewcommand{\profilecategory}{\labels@heroes}%
}
\newcommand{\coreunitstitle}{%
\def\logolocalpath{../Layout/pics/logo_core.png}%
\clearpage\bigtitle{\labels@coreunits}%
\renewcommand{\profilecategory}{\labels@coreunits}%
}
\newcommand{\specialunitstitle}{%
\def\logolocalpath{../Layout/pics/logo_special.png}%
\clearpage\bigtitle{\labels@specialunits}%
\renewcommand{\profilecategory}{\labels@specialunits}%
}
\newcommand{\rareunitstitle}{%
\def\logolocalpath{../Layout/pics/logo_rare.png}%
\clearpage\bigtitle{\labels@rareunits}%
\renewcommand{\profilecategory}{\labels@rareunits}%
}
\newcommand{\mountstitle}{%
\def\logolocalpath{../Layout/pics/logo_mount.png}%
\clearpage\bigtitle{\labels@charactermounts}%
\renewcommand{\profilecategory}{\labels@mounts}%
}

\newcommand{\startarmywiderules}{\newpage\bigtitle{\labels@armywiderules}\largefontsize}
\newcommand{\closearmywiderules}{\normalfontsize}
\newcommand{\armywideruleentry}[1]{\subtitle{#1}\vspace{5pt}}

\newcommand{\startarmyspecialrules}{\bigtitle{\labels@armyspecialrules}\largefontsize}
\newcommand{\closearmyspecialrules}{\normalfontsize}
\newcommand{\armyspecialruleentry}[1]{\subtitle{#1}\vspace{5pt}}

\newcommand{\startarmyarmoury}{\bigtitle{\labels@armoury}\largefontsize\subtitle{}}
\newcommand{\closearmyarmoury}{\normalfontsize}

\newcommand{\startarmymagicalitems}{\newpage\largefontsize\bigtitle{\labels@magicalitems}\begin{multicols}{2}\raggedcolumns}
\newcommand{\closearmymagicalitems}{\end{multicols}\normalfontsize}

\newcommand{\armymagicalweapons}{\subtitle{\labels@magicalweapons}}
\newcommand{\armymagicalarmour}{\subtitle{\labels@magicalarmour}}
\newcommand{\armytalismans}{\subtitle{\labels@talismans}}
\newcommand{\armyenchanteditems}{\subtitle{\labels@enchanteditems}}
\newcommand{\armyarcaneitems}{\subtitle{\labels@arcaneitems}}
\newcommand{\armymagicalbanners}{\subtitle{\labels@magicalbanners}}

\newcommand{\startarmynewsection}[1]{\newpage\bigtitle{#1}\largefontsize}
\newcommand{\startarmynewsectionSP}[1]{\vspace{1.5cm}\bigtitle{#1}\largefontsize}
\newcommand{\closearmynewsection}{\normalfontsize}

\newcommand{\armynewsubsection}[1]{\subtitle{#1}\vspace{5pt}}
\newcommand{\armynewsubsubsection}[1]{\subsubtitle{#1}\vspace{3pt}}

\newcommand{\armylist}{\clearpage}

\newcommand{\quickrefsheettitle}{\clearpage\newgeometry{top=1.6cm, bottom=2cm, left=1cm, right=1cm}\bigtitle{\labels@quickrefsheet}\vspace*{0.4cm}}
\newcommand{\changelogtitle}{\clearpage\bigtitle{\labels@changelog}\spaceaftersection{}}

\newcommand{\spaceaftersection}{\vspace{0.8cm}}

\newcommand{\separator}{\noindent\begin{center}\textcolor{black!30}{\rule{0.7\columnwidth}{2pt}}\end{center}}


%%% Custom lists and description for first sections of the army books

\newcommand{\startpricelist}{\begin{samepage}\begin{description}[leftmargin=0.3cm, labelindent=0cm, labelsep=0.1cm]}
\def\endpricelist{\end{description}\end{samepage}}
\newcommand{\pricelistitem}[2]{\item \option{\textbf{#1}}{#2}\newline}

\newcommand{\startpricelistNSP}{\begin{description}[leftmargin=0.3cm, labelindent=0cm, labelsep=0.1cm]}
\def\endpricelistNSP{\end{description}}

\newcommand{\startitemlist}{\begin{multicols}{2}\raggedcolumns\begin{description}[leftmargin=0.3cm, labelindent=0cm, labelsep=0.1cm]}
\def\enditemlist{\end{description}\end{multicols}}
\newcommand{\listitem}[1]{\item[#1\spacebeforecolon{}:]}

\newcommand{\startitemlistonecol}{\begin{description}[leftmargin=0.3cm, labelindent=0cm, labelsep=0.1cm]}
\def\enditemlistonecol{\end{description}}
\newcommand{\listitemonecol}[1]{\item \textbf{#1\spacebeforecolon{}:}\newline}

\newenvironment{customitemize}{\begin{description}[leftmargin=0.3cm, labelindent=0cm, labelsep=0cm]}{\end{description}}
\newenvironment{customsubitemize}{\begin{itemize}[label={-}, labelsep=0.1cm, topsep=0cm, parsep=0cm, itemsep=0cm, leftmargin=0.4cm, labelindent=0cm]}{\end{itemize}}

%%% Table parameters %%%

\newcolumntype{M}[1]{>{\centering\let\newline\\\arraybackslash\hspace{0pt}}m{#1}}


%%%  Lists handling %%%

\newcommand{\addlocallist}{\listadd\locallists@dummy}%
\NewDocumentCommand{\parsespacelist}{>{\SplitList{ }} m }{%
	\ProcessList{#1}{\addlocallist}%
}%
\NewDocumentCommand{\parsecommalist}{>{\SplitList{,}} m }{%
	\ProcessList{#1}{\addlocallist}%
}%
\newcommand{\parselist}[3][,]{%
	\renewcommand\addlocallist{\listadd#3}%
  	\undef#3%
  	\ifstrequal{#1}{ }{\parsespacelist{#2}}{\parsecommalist{#2}}%
}


%%% Profiles handling %%%

% Element of a table that contains the characteristics of a model (or part of a model)
\newcommand\caraclist[1]{
	\parselist[ ]{#1}{\locallists@caraclist}%
	\forlistloop{&}{\locallists@caraclist}%
}

\newcommand\caraclistbold[1]{
	\parselist[ ]{#1}{\locallists@caraclist}%
	\forlistloop{&\bfseries}{\locallists@caraclist}%
}

% Line of a profile table, including bottom line. It is meant to contain the name of the model (or part), its characteristics (preferably, the second argument should contain the \carac macro), troop type and base size.
\newcommand{\profilefirstline}[4]{#1 & #2 &   & #3 & #4 }

% Start of a profile table. Includes the table commands, and the column labels. \profilecellsize is the size of the characteristics cells in the profile.
\newcommand{\profilecellsize}{0.56cm}
\newcommand{\profilestart}{%
	\noindent %
	\begin{tabular}{@{}p{3cm}@{}M{\profilecellsize}@{}M{\profilecellsize}@{}M{\profilecellsize}@{}M{\profilecellsize}@{}M{\profilecellsize}@{}M{\profilecellsize}@{}M{\profilecellsize}@{}M{\profilecellsize}@{}M{\profilecellsize}@{}p{2.7cm}@{}p{3.3cm}@{}p{2cm}@{}}%
	 &% \textbf{\labels@profile}
	\labels@M & \labels@WS & \labels@BS & \labels@S & \labels@T & \labels@W & \labels@I & \labels@A & \labels@Ld &%
	&%
	{\unitentryformat{\labels@trooptype}} &%
	{\unitentryformat{\labels@basesize}}%
}

% End of a profile table.
\newcommand{\profileend}{\end{tabular}}

% Algorithm to automatically use and fill previous command, with coherence check.
\providebool{profilefirst}
\newcommand{\profileitem}[1]{%
	\tabularnewline%
	\splitatinf{#1}\local@unitname\local@unitprofile%
	\local@unitname \expandafter\caraclistbold\expandafter{\local@unitprofile}%
	&%
	& \ifbool{profilefirst}{\unit@type}{}%
	& \ifbool{profilefirst}{%
		\ifsubstring{\unit@basesize}{x}{% Rectangular base
			\unit{\unit@basesize}{\milli\meter}%
		}{% Circular base
			\unit{\unit@basesize}{\milli\meter} \labels@roundbase%
		}%
	}{}%
	\global\boolfalse{profilefirst}%
}
\newcommand{\profile}[1]{%
	\parselist{#1}{\locallists@profileslist}%
	\profilestart%
	\global\booltrue{profilefirst}%
	\forlistloop{\profileitem}{\locallists@profileslist}%
	\profileend%
}


%%% Profiles handling in case of invocation %%%

\newcommand{\invocprofilestart}{%
	\noindent %
	\begin{tabular}{@{}p{3cm}@{}M{\profilecellsize}@{}M{\profilecellsize}@{}M{\profilecellsize}@{}M{\profilecellsize}@{}M{\profilecellsize}@{}M{\profilecellsize}@{}M{\profilecellsize}@{}M{\profilecellsize}@{}M{\profilecellsize}@{}M{2.2cm}@{}p{0.5cm}@{}p{3.3cm}@{}p{2cm}@{}}%
	 &% \textbf{\labels@profile}
	\labels@M & \labels@WS & \labels@BS & \labels@S & \labels@T & \labels@W & \labels@I & \labels@A & \labels@Ld & \unitentryformat{\labels@Invocation} &%
	&%
	{\unitentryformat{\labels@trooptype}} &%
	{\unitentryformat{\labels@basesize}}%
}

\newcommand{\invocprofileitem}[1]{%
	\tabularnewline%
	\splitatinf{#1}\local@unitname\local@unitprofile%
	\local@unitname \expandafter\caraclistbold\expandafter{\local@unitprofile}%
	& \ifbool{profilefirst}{\unit@invocation}{} &%
	& \ifbool{profilefirst}{\unit@type}{}%
	& \ifbool{profilefirst}{\unit{\unit@basesize}{\milli\meter}}{}%
	\global\boolfalse{profilefirst}%
}

\newcommand{\invocprofile}[1]{%
	\parselist{#1}{\locallists@profileslist}%
	\invocprofilestart%
	\global\booltrue{profilefirst}%
	\forlistloop{\invocprofileitem}{\locallists@profileslist}%
	\profileend%
}


%%%%%%%%%%%%%%%%%%
%%% Unit rules %%%
%%%%%%%%%%%%%%%%%%

%%% Entry title command %%%

\newcommand{\unitentry}[2]{\ifdefempty{#1}{}{\noindent #2}}


%%% Special rules %%%

% Special rules listing for a unit, with alphabetical order.
\newcommand{\ruleslist}[1]{%
	\parselist[,]{#1}{\locallists@ruleslist}%
	\begin{sortedlist}%
		\forlistloop{\addtosortedlist}{\locallists@ruleslist}%
	\end{sortedlist}%
}

% Special rules entry.
\newcommand{\specialrules}[1]{\unitentry{#1}{\unitentryformat{\labels@specialrules\spacebeforecolon{}:}\newline\hspace*{-\fontdimen2\font}\expandafter\ruleslist\expandafter{#1}.}}
\newcommand{\commonspecialrules}[2]{\unitentry{#2}{\unitentryformat{#1\spacebeforecolon{}:}\newline\hspace*{-\fontdimen2\font}\expandafter\ruleslist\expandafter{#2}.}}


%%% Magical abilities %%%

% Paths listing for a unit.
\newcommand{\pathslist}[1]{%
	\parselist[,]{#1}{\locallists@pathslist}%
	\begin{itemize*}[label={}, itemjoin={,}, itemjoin*={\listlastchoice}]%
		\forlistloop{\item}{\locallists@pathslist}%
	\end{itemize*}%
}

% Magic entry.
\newcommand{\magic}[2]{\unitentry{#2}{\unitentryformat{\labels@magic\spacebeforecolon{}: }\newline\ifdefempty{#1}{}{\textbf{\magiclevel{#1}}. }\labels@pathsused\expandafter\pathslist\expandafter{#2}.}}

% Wizard Conclave.
\newcommand{\magicwizardconclave}[1]{\unitentry{#1}{\unitentryformat{\labels@magic\spacebeforecolon{}: }\newline\textbf{\wizardconclave{}}\spacebeforecolon{}: #1.}}


%%% Equipment %%%

% Equipment listing.
\newcommand{\equipmentlist}[1]{%
	\parselist[,]{#1}{\locallists@equipmentlist}%
	\begin{sortedlist}%
		\forlistloop{\addtosortedlist}{\locallists@equipmentlist}%
	\end{sortedlist}%
}

% Equipment entry.
\newcommand{\weapons}[1]{\unitentry{#1}{\unitentryformat{\labels@weapons\spacebeforecolon{}:}\newline\hspace*{-\fontdimen2\font}\expandafter\equipmentlist\expandafter{#1}.}}

\newcommand{\armour}[1]{\unitentry{#1}{\unitentryformat{\labels@armour\spacebeforecolon{}:}\newline\hspace*{-\fontdimen2\font}\expandafter\equipmentlist\expandafter{#1}.}}


%%% Alignment %%%

\newcommand{\alignment}[1]{\unitentry{#1}{\unitentryformat{\labels@alignment\spacebeforecolon{}:}\newline\textbf{#1}.}}

%%% Green Hide Race %%%

\newcommand{\greenhideraceentry}[1]{\unitentry{#1}{\unitentryformat{\labels@greenhiderace\spacebeforecolon{}:}\newline\textbf{#1}.}}


%%% Options %%%

% Frame commands.
\newcommand{\optionsframestart}{\begin{innerframe}[\labels@options]}
\newcommand{\optionsframeend}{\end{innerframe}}

% Options listing.
\newcommand{\optionslist}[1]{%
	\parselist[,]{#1}{\locallists@optionslist}%
	\begin{description}[leftmargin=0.3cm, labelindent=0cm, labelsep=0cm, itemsep=0cm, parsep=0cm]%
		\forlistloop{\item\setoption}{\locallists@optionslist}%
	\end{description}%
}

% Options entry.
\newcommand{\options}[1]{\ifdefempty{#1}{}{\optionsframestart\vspace*{-0.4cm}\unitentry{#1}{\expandafter\optionslist\expandafter{#1}}\optionsframeend}}

% Option specific commands.
\newcommand{\setoption}[1]{%
	\noexpandarg\StrCut{#1}{=}\optiontext\optionvalue%
	\expandafter\ifstrequal\expandafter{\optionvalue}{}{%
		\optiontext%
	}{%
	\ifsubstring{\optionvalue}{\free}{%
		\option[\free]{\optiontext}{\optionvalue}%
	}{%
	\ifsubstring{\optionvalue}{\unlimited}{%
		\option[\unlimited]{\optiontext}{\optionvalue}%
	}{%
	\ifsubstring{\optionvalue}{\upto}{%
		\splitatinf{\optionvalue}\myoption\myvalue%
		\option[\upto]{\optiontext}{\myvalue}%
	}{%
	\ifsubstring{\optionvalue}{\permodel}{%
		\splitatinf{\optionvalue}\myoption\myvalue%
		\option[\permodel]{\optiontext}{\myvalue}%
	}{%
	\ifsubstring{\optionvalue}{\pershadygit}{% For Orcs N Goblins
		\splitatinf{\optionvalue}\myoption\myvalue%
		\option[\pershadygit]{\optiontext}{\myvalue}%
	}{%
	\ifsubstring{\optionvalue}{\permadgit}{% For Orcs N Goblins
		\splitatinf{\optionvalue}\myoption\myvalue%
		\option[\permadgit]{\optiontext}{\myvalue}%
	}{%	
	\ifsubstring{\optionvalue}{\perrune}{% For Dwarven Holds
		\splitatinf{\optionvalue}\myoption\myvalue%
		\option[\perrune]{\optiontext}{\myvalue}%
	}{%	
		\option{\optiontext}{\optionvalue}%
	}}}}}}}}%
}

\newcommand{\option}[3][]{#2\predotfill\dotfill\nobreak%
	% Add \upto token if necessary.
	\ifstrequal{#1}{\upto}{\upto~}{}%
	% The option can be free, have an unlimited cost, or have a points cost.
	\ifstrequal{#1}{\free}{\free}{\ifstrequal{#1}{\unlimited}{\unlimited}{\pts{#3}}}%
	% Add \permodel if necessary.
	\ifstrequal{#1}{\permodel}{\nobreak\permodel}{}%
	% Add \persomething if necessary.
	\ifstrequal{#1}{\pershadygit}{\nobreak\pershadygit}{}% For Orcs N Goblins
	\ifstrequal{#1}{\permadgit}{\nobreak\permadgit}{}% For Orcs N Goblins
	\ifstrequal{#1}{\perrune}{\nobreak\perrune}{}% For Dwarven Holds
}

\newcommand\optionschoice[2]{%
	\parselist[,]{#2}{\locallists@optionschoice}%
	#1%
	\begin{itemize}[label={}, parsep=0cm, labelindent=0cm, labelwidth=0cm, noitemsep, topsep=0em, leftmargin=0.3cm]%
	\forlistloop{\item\setoption}{\locallists@optionschoice}%
	\end{itemize}%
}

\newcommand\optionschoiceTWOCOL[2]{%
	\parselist[,]{#2}{\locallists@optionschoice}%
	#1%
	\begin{itemize}[label={}, parsep=0cm, labelindent=0cm, labelwidth=0cm, noitemsep, topsep=0em, leftmargin=0.3cm]%
	\setlength{\columnseprule}{0.5pt}
	\renewcommand{\columnseprulecolor}{\color{black!30}}
	\vspace*{-5pt}\begin{multicols}{2}\raggedcolumns
	\forlistloop{\item\setoption}{\locallists@optionschoice}%
	\end{multicols}\setlength{\columnseprule}{0pt}
	\end{itemize}%
}

% Option description in army desc.
\newcommand{\optiondef}[3]{\option{\textbf{#1}}{#2}\ifblank{#3}{}{\\{#3}}}


%%% Mount options %%%

% Frame commands.
\newcommand{\mountsframestart}{\begin{innerframe}[\labels@mounts]}
\newcommand{\mountsframeend}{\end{innerframe}}

% Mount listing.
\newcommand{\mountslist}[1]{%
	\parselist[,]{#1}{\locallists@mountslist}%
	\begin{description}[leftmargin=0.3cm, labelindent=0cm, labelsep=0cm, itemsep=0cm, parsep=0cm]%
		\forlistloop{\item\setoption}{\locallists@mountslist}%
	\end{description}%
}

% Mount entry.
\newcommand{\mounts}[1]{\ifdefempty{#1}{}{\mountsframestart\vspace*{-0.4cm}\unitentry{#1}{\expandafter\mountslist\expandafter{#1}}\mountsframeend}}


%%% Command group %%%

% Command group specific commands.
\define@key{commandgroup}{restriction}            {\def\commandgroup@restriction{#1}}
\define@key{commandgroup}{champion}               {\def\commandgroup@champion{#1}}
\define@key{commandgroup}{championallowance}      {\def\commandgroup@championallowance{#1}}
\define@key{commandgroup}{championoption}         {\def\commandgroup@championoption{#1}}
\define@key{commandgroup}{championprerestriction} {\def\commandgroup@championprerestriction{#1}}
\define@key{commandgroup}{championrestriction}    {\def\commandgroup@championrestriction{#1}}
\define@key{commandgroup}{banner}                 {\def\commandgroup@banner{#1}}
\define@key{commandgroup}{bannerallowance}        {\def\commandgroup@bannerallowance{#1}}
\define@key{commandgroup}{veteranstandardbearer}  {\def\commandgroup@veteranstandardbearer{#1}}
\define@key{commandgroup}{singlebannerallowance}  {\def\commandgroup@singlebannerallowance{#1}}
\define@key{commandgroup}{condsinglebannerallowance}  {\def\commandgroup@condsinglebannerallowance{#1}}
\define@key{commandgroup}{banneroption}           {\def\commandgroup@banneroption{#1}}
\define@key{commandgroup}{bannerrestriction}      {\def\commandgroup@bannerrestriction{#1}}
\define@key{commandgroup}{musician}               {\def\commandgroup@musician{#1}}
\define@key{commandgroup}{musicianrestriction}    {\def\commandgroup@musicianrestriction{#1}}
\newcommand{\defcommandgroup}{%
	\setkeys{commandgroup}{restriction=,
	                       champion=, championallowance=, championoption=, championprerestriction=, 
	                       championrestriction=, banner=, bannerallowance=, veteranstandardbearer=, 
	                       singlebannerallowance=, condsinglebannerallowance=, banneroption=, 
	                       bannerrestriction=, musician=, musicianrestriction=}%
	\setkeys{commandgroup}%
}

% Frame commands.
\newcommand{\commandgroupframestart}{\begin{innerframe}[\labels@commandgroup]}
\newcommand{\commandgroupframeend}{\end{innerframe}}

% Command group entry.
\newcommand{\commandgroup}[1]{%
	\defcommandgroup{#1}%
	\ifstrempty{#1}{}{\commandgroupframestart\vspace*{-0.2cm}%
		\begin{description}[leftmargin=0.3cm, labelindent=0cm, labelsep=0cm, itemsep=0cm, parsep=0cm]%
			% Command group title, including restrictions applying to all the command group
			\item \textbf{\expandafter\ifblank\expandafter{\commandgroup@restriction}{}{ \only{\commandgroup@restriction}\spacebeforecolon{}: }} 
			% Champion handling.
			\ifdefempty{\commandgroup@champion}{}{% We have a champion!
			\ifdefempty{\commandgroup@championprerestriction}{% There is no prerestriction to have a champion
				\item \hspace*{-0.04cm}\option{\labels@champion%
					% Possible restrictions to taking a champion
				    \expandafter\ifblank\expandafter{\commandgroup@championrestriction}{}{ \only{\commandgroup@championrestriction}}%
				    % Cost of a champion
				    }{\commandgroup@champion}%
				    % Magical allowance of the champion. Should probably not be used, champion option can do it as well and is more flexible.
					\ifdefempty{\commandgroup@championallowance}{}{\par\option[\upto]{\hspace*{0.3cm}- \labels@championallowance}{\commandgroup@championallowance}}%
					% Any option available to the champion, in the form option:cost
					\ifdefempty{\commandgroup@championoption}{}{%
						\splitatinf{\commandgroup@championoption}\local@option\local@cost%
						\par\option{\hspace*{0.3cm}- \local@option}{\local@cost}}%
			}{% There is a pre-restriction to have a champion
				\item \hspace*{-0.04cm}\commandgroup@championprerestriction	\newline%
				\option{\labels@champion}{\commandgroup@champion}%
				% Magical allowance of the champion. Should probably not be used, champion option can do it as well and is more flexible.
				\ifdefempty{\commandgroup@championallowance}{}{\par\option[\upto]{\hspace*{0.3cm}- \labels@championallowance}{\commandgroup@championallowance}}%
				% Any option available to the champion, in the form option:cost
				\ifdefempty{\commandgroup@championoption}{}{%
					\splitatinf{\commandgroup@championoption}\local@option\local@cost%
					\par\option{\hspace*{0.3cm}- \local@option}{\local@cost}}%
			} %End of the prerestriction of not condition
			}% End of champion handling
			\ifdefempty{\commandgroup@musician}{}{% We have a musician!
				\item \hspace*{-0.04cm}\option{\labels@musician%
					% Possible restrictions to taking a musician
				    \expandafter\ifblank\expandafter{\commandgroup@musicianrestriction}{}{ \only{\commandgroup@musicianrestriction}}%
				    % Cost of a musician
				    }{\commandgroup@musician}%
			}%
			\ifdefempty{\commandgroup@banner}{}{% We have a banner!
				\item \hspace*{-0.04cm}\option{\labels@standardbearer%
					% Possible restrictions to taking a banner
				    \expandafter\ifblank\expandafter{\commandgroup@bannerrestriction}{}{ \only{\commandgroup@bannerrestriction}}%
				    % Cost of a banner
				    }{\commandgroup@banner}%
				    % Magical banner, if all units of this type can take one.
					\ifdefempty{\commandgroup@bannerallowance}{}{\par\option[\upto]{\hspace*{0.3cm}- \labels@bannerallowance}{\commandgroup@bannerallowance}}%
					% Magical banner, if Veteran.
					\ifdefempty{\commandgroup@veteranstandardbearer}{}{\par\hspace*{0.3cm}- \labels@veteranstandardbearer%
					\expandafter\ifstrequal\expandafter{\commandgroup@veteranstandardbearer}{*}{*}{}%
					}%
					% Magical banner, if only one unit of this type can take one.
					\ifdefempty{\commandgroup@singlebannerallowance}{}{\par\option[\upto]{\hspace*{0.3cm}- \labels@singlebannerallowance}{\commandgroup@singlebannerallowance}}%
					% Magical banner, if only one unit of this type can take one, but with condtions.
					\ifdefempty{\commandgroup@condsinglebannerallowance}{}{%
						\splitatinf{\commandgroup@condsinglebannerallowance}\local@option\local@cost%
						\par\option[\upto]{\hspace*{0.3cm}- \labels@condsinglebannerallowance \local@option}{\local@cost}}%
					% Additional option for the banner, such as Hill Goblin Lookouts for Ogres
					\ifdefempty{\commandgroup@banneroption}{}{%
						\splitatinf{\commandgroup@banneroption}{\local@option}{\local@cost}%
						\par\option{\hspace*{0.3cm}- \local@option}{\local@cost}%
					}%
			}%
		\end{description}%
	\commandgroupframeend%
	 }%
}


%%% Unit rules %%%

% Frame commands.
\newcommand{\unitrulesframestart}{\begin{innerframe}[\labels@specialrules]}
\newcommand{\unitrulesframeend}{\end{innerframe}}

% Unit rules specific commands.
\newcommand{\unitrule}[2]{\item[#1\spacebeforecolon{}:]#2}

% Unit rule entry.
\newcommand{\unitrules}[1]{\ifdefempty{#1}{}{\unitrulesframestart\vspace*{-0.05cm}\begin{description}[leftmargin=0.3cm, labelindent=0cm, labelsep=0.1cm, itemsep=0.2cm, parsep=0cm]#1\end{description}\unitrulesframeend}}


%%% Special equipment %%%

% Frame commands.
\newcommand{\unitequipmentframestart}{\begin{innerframe}[\labels@specialequipment]}
\newcommand{\unitequipmentframeend}{\end{innerframe}}

% Special equipment specific commands.
\newcommand{\equipmentdef}[2]{\item[#1\spacebeforecolon{}:]#2}

% Special equipment entry.
\newcommand{\unitequipment}[1]{\ifdefempty{#1}{}{\unitequipmentframestart\vspace*{-0.05cm}\begin{description}[leftmargin=0.3cm, labelindent=0cm, labelsep=0.1cm, itemsep=0.2cm, parsep=0cm]#1\end{description}\unitequipmentframeend}}






%%%%%%%%%%%%%%%%%%%%%%%%%%%%%%%%
%%% Profile input and layout %%%
%%%%%%%%%%%%%%%%%%%%%%%%%%%%%%%%

%%% Input parameters %%%

\define@key{unit}{notinQRS}{\def\unit@notinQRS{#1}}
\define@key{unit}{name}{\def\unit@name{#1}}
\define@key{unit}{QRSname}{\def\unit@QRSname{#1}}
\define@key{unit}{profile}{\def\unit@profile{#1}}
\define@key{unit}{cost}{\def\unit@cost{#1}}
\define@key{unit}{invocation}{\def\unit@invocation{#1}}
\define@key{unit}{costpermodel}{\def\unit@costpermodel{#1}}
\define@key{unit}{maxmodels}{\def\unit@maxmodels{#1}}
\define@key{unit}{type}{\def\unit@type{#1}}
\define@key{unit}{unitsize}{\def\unit@unitsize{#1}}
\define@key{unit}{basesize}{\def\unit@basesize{#1}}
\define@key{unit}{commonspecialrules}{\def\unit@commonspecialrules{#1}}
\define@key{unit}{commontype}{\def\unit@commontype{#1}}
\define@key{unit}{commonspecialrulesB}{\def\unit@commonspecialrulesB{#1}}
\define@key{unit}{commontypeB}{\def\unit@commontypeB{#1}}
\define@key{unit}{specialrules}{\def\unit@specialrules{#1}}
\define@key{unit}{magiclevel}{\def\unit@magiclevel{#1}}
\define@key{unit}{magicpaths}{\def\unit@magicpaths{#1}}
\define@key{unit}{equipment}{\def\unit@equipment{#1}}
\define@key{unit}{alignment}{\def\unit@alignment{#1}}
\define@key{unit}{greenhiderace}{\def\unit@greenhiderace{#1}}
\define@key{unit}{weapons}{\def\unit@weapons{#1}}
\define@key{unit}{armour}{\def\unit@armour{#1}}
\define@key{unit}{wizardconclave}{\def\unit@wizardconclave{#1}}
\define@key{unit}{unitequipment}{\def\unit@unitequipment{#1}}
\define@key{unit}{options}{\def\unit@options{#1}}
\define@key{unit}{mounts}{\def\unit@mounts{#1}}
\define@key{unit}{commandgroup}{\def\unit@commandgroup{#1}}
\define@key{unit}{unitrules}{\def\unit@unitrules{#1}}
\define@key{unit}{additional}{\def\unit@additional{#1}}


%%% Frames definition %%%

% Unit's big frame.
\tikzset{unitprice/.style={draw=white, fill=white, rectangle, rounded corners, right, minimum height=0.7cm}}
\tikzset{unittitle/.style={draw=white, fill=white, rectangle, rounded corners, right, minimum height=0.7cm, font=\bfseries}}
\tikzset{unitlogo/.style={draw=white, fill=white, rectangle, right, minimum height=0.7cm}}

\newenvironment{unitframe}[2][]{%
	\mdfsetup{%
		nobreak=true,%
		linewidth=1pt,%
		linecolor=black!30,%
		roundcorner=5pt,%
		backgroundcolor=white,%
		innertopmargin=1.2\baselineskip,
		innerbottommargin=1.2\baselineskip,
		singleextra={
			\expandafter\ifblank\expandafter{\unit@cost}{}{%
				\node[unitprice,anchor=east,xshift=-0.5cm] at (P)%
					{%
						{{\smallfontsize\minprice} \Largefontsize\pts{\textbf{\unit@cost}}}%
					};
				}%
				\node[unittitle,xshift=0.5cm] at (P-|O)%
					{\Largefontsize\antiquefont\uppercase\expandafter\expandafter\expandafter{\unit@name}};
				\node[unitlogo, xshift=8.1cm, yshift=0.1cm] at (P-|O)%
					{\includegraphics[width=1.2cm]{\logolocalpath}};
		}
	}%
	\begin{mdframed}[]\relax%
}%
{%
\end{mdframed}%
}

% Inner small frames for options, special rules definition, ...
\tikzset{innertitle/.style={fill=white, rectangle, rounded corners, right, minimum height=8pt, xshift=0.5cm}}

\newenvironment{innerframe}[1][]{%
	\mdfsetup{%
		innerleftmargin=5pt,%
		innerrightmargin=5pt,%
		linecolor=black!30,%
		linewidth=0.5pt,%
		roundcorner=5pt,%
		backgroundcolor=white,%
		innertopmargin=1.1\baselineskip,
		singleextra={
		\node[innertitle] at (P-|O)%
			{\unitentryformat{#1}};
		}
	}%
	\vspace*{-0.2cm}\begin{mdframed}[]\relax%
}%
{%
\end{mdframed}%
}

%%% Command to add a new unit definition %%%

\newcommand{\defunit}{
	\setkeys{unit}{%
		notinQRS=, name=, QRSname=, profile=, cost=, invocation=, costpermodel=, maxmodels=, type=, unitsize=, basesize=, commonspecialrules=, commontype=, commonspecialrulesB=, commontypeB=, specialrules=, magiclevel=, magicpaths=, alignment=, greenhiderace=, equipment=, weapons=, armour=, wizardconclave=, unitequipment=, options=, mounts=, commandgroup=, unitrules=, additional=%
	}%
	\setkeys{unit}%
}

\newcommand{\showunit}[1]{
	\defunit{#1}
	\begin{unitframe}[\unit@name]{\unit@cost}
	\mdfsetup{style=defaultoptions}
	\expandafter\ifblank\expandafter{\unit@unitsize}{}{%
	\expandafter\ifstrequal\expandafter{\unit@unitsize}{1}{% single model
		% Can you add model to this single model ?
		\expandafter\ifblank\expandafter{\unit@maxmodels}{% no		
			{\hspace*{0.25cm}\labels@Singlemodel}%
		}{% yes
			{\hspace*{0.25cm}\mincostfor{} \textbf{1} \labels@model{}. \maxunitsize{}\spacebeforecolon{}: \textbf{\unit@maxmodels} \labels@models{}.\hfill \additionalfigscost{} {\largefontsize\pts{\textbf{\unit@costpermodel{}}}\permodel}\hspace*{0.1cm}}%
		}%
	}{% not single model
		% Test if we wanna print a sentence instead of unit number
		\ifsubstring{\unit@unitsize}{SPECIAL-}{%
			\hspace*{0.25cm}\StrDel{\unit@unitsize}{SPECIAL-}%
		}{%	
			{\hspace*{0.25cm}\mincostfor{} \textbf{\unit@unitsize} \labels@models{}. \maxunitsize{}\spacebeforecolon{}: \textbf{\unit@maxmodels} \labels@models{}.\hfill \additionalfigscost{} {\largefontsize\pts{\textbf{\unit@costpermodel{}}}\permodel}\hspace*{0.1cm}}%
		}%
	}%
	}%
	\vspace*{-0.1cm}
	\noindent\begin{center}\textcolor{black!30}{\rule{\columnwidth}{1pt}}\end{center}
		\expandafter\ifblank\expandafter{\unit@invocation}{%
			\expandafter\profile\expandafter{\unit@profile}
		}{%
			\expandafter\invocprofile\expandafter{\unit@profile}
		}
	\noindent\begin{center}\textcolor{black!30}{\rule{\columnwidth}{1pt}}\end{center}
	\vspace*{-0.2cm}
	\setlength\multicolsep{0pt}
	\begin{multicols}{2}
		\raggedcolumns
		\vspace*{-0.3cm}{\setlength{\parskip}{0.3cm}
		\expandafter\ifblank\expandafter{\unit@alignment}{}{\noindent\parbox{\columnwidth}{\alignment{\unit@alignment}}}
		
		\expandafter\ifblank\expandafter{\unit@greenhiderace}{}{\noindent\parbox{\columnwidth}{\greenhideraceentry{\unit@greenhiderace}}}
		
		\expandafter\ifblank\expandafter{\unit@equipment}{}{\noindent\parbox{\columnwidth}{\equipment{\unit@equipment}}}
				
		\expandafter\ifblank\expandafter{\unit@weapons}{}{\noindent\parbox{\columnwidth}{\weapons{\unit@weapons}}}
		
		\expandafter\ifblank\expandafter{\unit@armour}{}{\noindent\parbox{\columnwidth}{\armour{\unit@armour}}}
		
		\expandafter\ifblank\expandafter{\unit@commonspecialrules}{}{\noindent\parbox{\columnwidth}{\commonspecialrules{\unit@commontype}{\unit@commonspecialrules}}}
		
		\expandafter\ifblank\expandafter{\unit@commonspecialrulesB}{}{\noindent\parbox{\columnwidth}{\commonspecialrules{\unit@commontypeB}{\unit@commonspecialrulesB}}}
		
		\expandafter\ifblank\expandafter{\unit@specialrules}{}{\noindent\parbox{\columnwidth}{\specialrules{\unit@specialrules}}}
		
		\expandafter\ifblank\expandafter{\unit@magicpaths}{}{\noindent\parbox{\columnwidth}{\magic{\unit@magiclevel}{\unit@magicpaths}}}
		
		\expandafter\ifblank\expandafter{\unit@wizardconclave}{}{\noindent\parbox{\columnwidth}{\magicwizardconclave{\unit@wizardconclave}}}
		}
		\vspace{0.1cm}
		\mounts{\unit@mounts}
		\options{\unit@options}
		\expandafter\ifblank\expandafter{\unit@commandgroup}{}{\expandafter\commandgroup\expandafter{\unit@commandgroup}}
		\unitrules{\unit@unitrules}
		\unitequipment{\unit@unitequipment}
	\end{multicols}
	\vspace*{0.1cm}\unit@additional
	\end{unitframe}
	% Database filling for auto QRS
	\expandafter\ifblank\expandafter{\unit@notinQRS}{%
	\DTLnewrow{profiles}%
	\expandafter\ifblank\expandafter{\unit@QRSname}{%
		\expandafter\profiledtbfillname\expandafter{\unit@name}%
	}{%
		\expandafter\profiledtbfillname\expandafter{\unit@QRSname}%
	}
	\expandafter\profiledtbfillcategory\expandafter{\profilecategory}%
	\expandafter\profiledtbfilltrooptype\expandafter{\unit@type}%
	\expandafter\ifblank\expandafter{\unit@invocation}{}{\expandafter\profiledtbfillinvocation\expandafter{\unit@invocation}}%
	\expandafter\profiledtbfillcarac\expandafter{\unit@profile}
	}{}%
}


%%% Changelog commands %%%

\newcommand{\newlog}[2]{%
\vspace*{0.2cm}\noindent{\antiquefont\Large\textbf{V#1}}
\parselist[,]{#2}{\locallists@changelist}%
\begin{itemize}[itemsep=0pt]%
\forlistloop{\item[-]}{\locallists@changelist}%
\end{itemize}%
}

\newcommand{\startchangelog}{\begin{multicols}{2}\vspace*{-0.2cm}}
\def\endchangelog{\end{multicols}}


\newcommand{\booktitle}{Peaux Vertes}
\newcommand{\version}{0.99.2}
\newcommand{\frenchversion}{2.0}
\newcommand{\translationteam}{\item \og AEnoriel \fg \item \og Anglachel \fg \item \og Astadriel \fg \item \og Batcat \fg \item \og Eru \fg  \item \og Gandarin \fg \item \og Groumbahk \fg \item \og Iluvatar \fg \item \og Lamronchak \fg \item \og Mammstein \fg}

% blabla
% Army special rules

\newcommand{\greenhiderace}{Race Peau Verte}
\newcommand{\greenhideraces}{Races de Peaux Vertes}

\newcommand{\commonorc}{Orque Commun}
\newcommand{\ironorc}{Orqu'en Fer}
\newcommand{\feralorc}{Orque Primitif}
\newcommand{\feralorcs}{Orques Primitifs}
\newcommand{\commongoblin}{Gobelin Commun}
\newcommand{\cavegoblin}{Gobelin des Cavernes}
\newcommand{\forestgoblin}{Gobelin des Forêts}

\newcommand{\unruly}{Indiscipliné}
\newcommand{\borntofight}{Né pour la Baston}
\newcommand{\waaargh}{Waaargh !}
\newcommand{\greentide}{Marée Verte}
\newcommand{\venomousfangs}{Dard Venimeux}
\newcommand{\shambolic}[1]{Déchaîné\ifblank{#1}{}{ (#1)}}
\newcommand{\runningamok}{Nawak !}
\newcommand{\ricochet}[1]{Ricochet\ifblank{#1}{}{ (#1)}}

% Armoury

\newcommand{\powershroom}{Champix de Pouvoir}
\newcommand{\powershrooms}{Champix de Pouvoir}
\newcommand{\mammothstabber}{Crève-Mammouth}

% Other Rules

\newcommand{\nets}{Filets}
\newcommand{\motherskiss}{Baiser de la Mère-Araignée}
\newcommand{\sneaky}{Sournois}
\newcommand{\surprise}{Surprise !}
\newcommand{\oiitbites}{Attention, Gniark Méchant}
\newcommand{\rowsofteeth}{Rangées de Dents}
\newcommand{\theyreeverywhere}{Partent en Sucette !}
\newcommand{\orcoverseer}{Hortatorque}
\newcommand{\splatterer}{Écrabouilleur}
\newcommand{\splatterers}{Écrabouilleurs}
\newcommand{\gitlauncher}{Lance-Kamikaze}
\newcommand{\gitlaunchers}{Lance-Kamikazes}
\newcommand{\smasher}{Fracasseur}
\newcommand{\pointedsticks}{Bouts Pointus}
\newcommand{\pursuitmode}{Fast and Furious}
\newcommand{\smellslikegreenspirit}{Smells Like Green Spirit}
\newcommand{\commontrolls}{Trolls Communs}
\newcommand{\cavetrolls}{Trolls des Cavernes}
\newcommand{\bridgetrolls}{Trolls de Rivière}
\newcommand{\trollbelch}{Bile de Troll}
\newcommand{\giantattacks}{Attaques de Géant}
\newcommand{\bellow}{Hurle}
\newcommand{\jump}{Saute}
\newcommand{\grab}{Ramasse}
\newcommand{\swing}{Frappe}
\newcommand{\thump}{Tape comme un Sourd}
\newcommand{\smashname}{Fracasse}
\newcommand{\ballista}{Embrocheur}
\newcommand{\lookatemgo}{Libéré ! Délivré !}
\newcommand{\weblauncher}{Lance-Toile}
\newcommand{\weblaunchers}{Lance-Toiles}
\newcommand{\smashemflat}{Vert de Rage}
\newcommand{\wevegotthegreenlight}{On a le Feu Vert}
\newcommand{\bouncers}{Bondisseur}
\newcommand{\spidermothershrine}{Autel de la Mère-Araignée}

% Spells



% Characters

\newcommand{\orcwarlord}{Seigneur de Guerre Orque}
\newcommand{\orcwarlordinQRS}{Seigneur Orque} % Too long for QRS in French
\newcommand{\orcbigshaman}{Grand Chamane Orque}
\newcommand{\goblinking}{Roi Gobelin}
\newcommand{\goblinbigshaman}{Grand Chamane Gobelin}
\newcommand{\orcchief}{Caïd Orque}
\newcommand{\orcshaman}{Chamane Orque}
\newcommand{\goblinchief}{Meneur Gobelin}
\newcommand{\goblinshaman}{Chamane Gobelin}

% Core

\newcommand{\orcs}{Orques}
\newcommand{\orceadbashers}{Brutes Orques}
\newcommand{\goblin}{Gobelin}
\newcommand{\goblins}{Gobelins}
\newcommand{\shadygit}{Vengeur Masqué}
\newcommand{\madgit}{Boulet-Fou}
\newcommand{\shadygits}{Vengeurs Masqués}
\newcommand{\madgits}{Boulets-Fous}
\newcommand{\goblinraiders}{Pillards Gobelins}
\newcommand{\orcboarriders}{Orques sur Sanglier}

% Special

\newcommand{\ironorcsunit}{Full Metal'Orques}
\newcommand{\mountedeadbashers}{Brutes sur Sangliers}
\newcommand{\orcboarchariot}{Char à Sangliers}
\newcommand{\goblinwolfchariot}{Char à Loups}
\newcommand{\gnasherdashers}{Barjos sur Gniark}
\newcommand{\gnasherherd}{Meute de Gniarks}
\newcommand{\greenhidecatapults}{Catapultes des Peaux Vertes}
\newcommand{\grotlings}{Morveux}
\newcommand{\scrapwagon}{Carriole de Morveux}
\newcommand{\trolls}{Trolls}
\newcommand{\giant}{Géant}

% Rare

\newcommand{\skewerer}{Embrocheur}
\newcommand{\gnasherwreckingteam}{Duo de Gniarks Broyeurs}
\newcommand{\gargantula}{Gargantula}
\newcommand{\greatgreenidol}{Idole des Dieux Verts}

% Mounts

\newcommand{\wyvern}{Vouivre}
\newcommand{\warboar}{Sanglier}
\newcommand{\wolf}{Loup}
\newcommand{\cavegnasher}{Gros Gniark}
\newcommand{\cavegnashers}{Gros Gniarks}
\newcommand{\scuttlerspider}{Araignée}
\newcommand{\huntsmenspider}{Araignée Chasseuse}


% Profile names

\newcommand{\rider}{Cavalier}
\newcommand{\eadbasherrider}{Brute Orque}
\newcommand{\gnasher}{Gniark}
\newcommand{\gnashers}{Gniarks}
\newcommand{\machine}{Machine}
\newcommand{\spider}{Araignée}
\newcommand{\wagon}{Carriole}

% Profile race forking

\newcommand{\startraceforking}[1]{%
\begin{center}Doit appartenir à l'une des Races de Peaux Vertes suivantes :\end{center}
\setlength{\columnseprule}{1pt}
\renewcommand{\columnseprulecolor}{\color{black!30}}
\begin{multicols}{#1}\raggedcolumns
}

\newcommand{\startraceforkingrider}[1]{%
\begin{center}Doit appartenir à l'une des Races de Peaux Vertes suivantes (seul le \rider{} gagne les règles de Race) :\end{center}
\setlength{\columnseprule}{1pt}
\renewcommand{\columnseprulecolor}{\color{black!30}}
\begin{multicols}{#1}\raggedcolumns
}

\newcommand{\forkraceCommonOrc}[1]{%
\begin{center}
\begin{tabular}{@{}m{0.8cm}@{\hskip 2pt}l@{}}
\includegraphics[width=0.8cm]{pics/commonorc.png} &
\largerfontsize\antiquefont\commonorc{} (#1) \tabularnewline
\end{tabular}
\vspace*{-0.2cm}
\end{center}
}

\newcommand{\forkraceIronOrc}[1]{%
\begin{center}
\begin{tabular}{@{}m{0.8cm}@{\hskip 2pt}l@{}}
\includegraphics[width=0.8cm]{pics/ironorc.png} &
\largerfontsize\antiquefont\ironorc{} (#1) \tabularnewline
\end{tabular}
\vspace*{-0.2cm}
\end{center}
}

\newcommand{\forkraceFeralOrc}[1]{%
\begin{center}
\begin{tabular}{@{}m{0.8cm}@{\hskip 2pt}l@{}}
\includegraphics[width=0.8cm]{pics/feralorc.png} &
\largerfontsize\antiquefont\feralorc{} (#1) \tabularnewline
\end{tabular}
\vspace*{-0.2cm}
\end{center}
}

\newcommand{\forkraceCommonGoblin}[1]{%
\begin{center}
\begin{tabular}{@{}m{0.8cm}@{\hskip 2pt}l@{}}
\includegraphics[width=0.8cm]{pics/commongoblin.png} &
\largerfontsize\antiquefont\commongoblin{} (#1) \tabularnewline
\end{tabular}
\vspace*{-0.2cm}
\end{center}
}

\newcommand{\forkraceCaveGoblin}[1]{%
\begin{center}
\begin{tabular}{@{}m{0.8cm}@{\hskip 2pt}l@{}}
\includegraphics[width=0.8cm]{pics/cavegoblin.png} &
\largerfontsize\antiquefont\cavegoblin{} (#1) \tabularnewline
\end{tabular}
\vspace*{-0.2cm}
\end{center}
}

\newcommand{\forkraceForestGoblin}[1]{%
\begin{center}
\begin{tabular}{@{}m{0.8cm}@{\hskip 2pt}l@{}}
\includegraphics[width=0.8cm]{pics/forestgoblin.png} &
\largerfontsize\antiquefont\forestgoblin{} (#1) \tabularnewline
\end{tabular}
\vspace*{-0.2cm}
\end{center}
}

\newcommand{\labelforinit}{Init.} % \labels@I usually

\newcommand{\profilemodification}[1]{%
\begin{center}\textbf{#1 sur le profil}\end{center}
\vspace{-0.2cm}
}

\newcommand{\closeraceforking}{%
\end{multicols}
\setlength{\columnseprule}{0pt}	
}

% Profile wording

\newcommand{\mayexchangeallequipmentforcrossbowandha}{Peut échanger tout son équipement contre une \crossbow{} et une \ha{}}
\newcommand{\maytakeamammothstabber}{L'unité peut prendre un \mammothstabber{}}
\newcommand{\takeshadygits}{Jusqu'à 3 \shadygits{}}
\renewcommand{\pershadygit}{/Vengeur}
\newcommand{\exchangeallweaponsforshieldandshortbow}{Échanger les armes contre un Arc Court et un Bouclier}
\newcommand{\takemadgits}{Jusqu'à 3 \madgits{}}
\renewcommand{\permadgit}{/Fou}
\newcommand{\takenets}{\nets}
\newcommand{\gitnote}{\textbf{Option} pour les unités de \goblins}
\newcommand{\ironorcwarlord}{Seigneur de Guerre \ironorc{}}
\newcommand{\ironorcchief}{Caïd \ironorc{}}
\newcommand{\maytakeanorcoverseer}{Peut prendre un \orcoverseer{}}
\newcommand{\maytakeweblauncher}{Peut prendre un \weblauncher{}}
\newcommand{\spidermothershrineoption}{Si le Cavalier est un Sorcier, peut prendre un \spidermothershrine}

% Profile rules

\newcommand{\netsrule}{%
Au début de chaque Manche de Corps à Corps, choisissez une unité en contact socle à socle avec l'unité possédant des \nets{}. Jetez 1D6.

Sur 2+, la cible subit un malus de -1 en Force (jusqu'à un minimum de 1) jusqu'à la fin du tour de joueur.

Sur un résultat de \result{1}, l'unité qui a utilisé les \nets{} subit le malus à sa place. Une unité ne peut être affectée par des \nets{} qu'une seule fois par Phase.
}

\newcommand{\motherskissrule}{%
Au début de chaque Manche de Corps à Corps, jetez 1D6.

Sur 2+, l'unité gagne des \poisonedattacks{} jusqu'à la fin de la Manche de Corps à Corps.

Sur un résultat de \result{1}, déterminer aléatoirement une unité ennemie en contact socle à socle avec l'unité. Elle gagne des \poisonedattacks{} contre l'unité ayant la règle \motherskiss{} jusqu'à la fin de la Manche de Corps à Corps.
}

\newcommand{\sneakyrule}{%
Les \shadygits{} comptent comme des Champions et sont obligatoirement déployés Cachés dans leur unité. Les \shadygits{} sont automatiquement révélés à la première Manche de Corps à Corps auquel leur unité prend part. Ils ne peuvent pas être révélés plus tôt. Lorsqu'ils sont révélés, les \shadygits{} gagnent +3 en Initiative et la règle \lightningreflexes{} jusqu'à la fin du tour. Les \shadygits{} ne bénéficient ni de Premier entre ses Pairs ni de Meneur de Charge (règles habituelles des Champions).
}

\newcommand{\surpriserule}{%
Les \madgits{} ne sont pas déployés normalement mais cachés dans leur unité. Ce sont des options d'unité, donc ils sont ignorés quand ils sont détruits pour le calcul des Points de Victoire, leur valeur étant déjà incluse dans l'unité de \goblins{} les cachant. Il faut donc détruire cette dernière unité pour gagner leurs Points de Victoire. Avant d'être lâchés hors de leur unité, les \madgits{} ne peuvent pas être affectés par quoi que ce soit, ni avoir d'effet sur la partie. Les \madgits{} détruits ne provoquent pas de tests de Panique. Ils ne comptent pas dans le nombre de figurines de leur unité. Une fois relâchés, ils bougent, agissent et suivent leur règles spéciales indépendamment comme toute autre unité individuelle.

Ils peuvent être relâchés de deux façons :
\begin{customsubitemize}
\item N'importe quel nombre de \madgits{} peut être relâché quand leur unité déclare comme réaction à une charge Tenir la Position ou Tenir la Position et Tirer.
\item Tous les \madgits{} doivent être relâchés si l'unité les cachant n'est ni en fuite ni engagée au Corps à Corps au début de la Phase de Tir du propriétaire et qu'elle est à moins de \distance{8} d'une unité ennemie.
\end{customsubitemize}
Occupez-vous des \madgits{} l'un après l'autre. Placez le \madgit{} en contact socle à socle avec son unité (il ne lui inflige aucune touche pour ce contact), et choisissez une direction. Déplacez-le de \distance{2D6} dans cette direction. Le \madgit{} suit ensuite ses propres règles pour son déplacement lors des tours suivants.
}

\newcommand{\oiitbitesrule}{%
Aucun Personnage ne peut rejoindre cette unité.
}

\newcommand{\rowsofteethrule}{%
Les \gnashers{} font les Attaques de Soutien à la place de leurs Cavaliers.
}

\newcommand{\theyreeverywhererule}{%
Quand une \gnasherherd{} est démoralisée au Corps à Corps, elle est immédiatement retirée du champ de bataille comme perte, et toutes les unités à moins de \distance{6} subissent une touche de Force 5 par tranche de 5 \gnashers{} qui restaient dans la \gnasherherd{}.
}

\newcommand{\orcoverseerrule}{%
La \warmachine{} gagne un servant supplémentaire, de Race \commonorc{}. Elle gagne 1 PV et perd la règle \insignificant{}.

La \warmachine{} peut choisir de perdre 1 PV pour relancer un jet sur la Table des Incidents de Tir.
}

\newcommand{\splattererrule}{%
\textbf{\artilleryweapon} de type \textbf{\catapult{} (\distance{3})}. \newline\range{12-60}, Force 3 [9], [\multiplewounds{\ordnance}{}].
}

\newcommand{\gitlauncherrule}{%
\textbf{\artilleryweapon} de type \textbf{\catapult{} (\distance{1})}. \newline\range{12-60}, Force 5, \armourpiercing{2}.\newline
Après avoir fait dévier le gabarit, vous pouvez le déplacer dans la direction de votre choix sur une distance de \distance{1D6}. Vous ne pouvez cependant pas le déplacer délibérément sur des unités au Corps à Corps ou des unités amies, si cela peut être évité. Une fois la position finale obtenue, au lieu de suivre les règles habituelles des gabarits, toute unité sous le gabarit subit 1D3+1 touches.
}

\newcommand{\greenhidecatapultsnote}{%
Une armée ne peut pas contenir plus de deux \splatterers{}, ni plus de deux \gitlaunchers{}. Ces nombres sont changés en un et quatre respectivement pour les Patrouilles et les Grandes Armées.
}

\newcommand{\pointedsticksrule}{%
Les Touches d'Impact de la \scrapwagon{} bénéficient de \armourpiercing{2}.
}

\newcommand{\pursuitmoderule}{%
Lancez 1D6 supplémentaire pour le \randommovement{} pendant la Phase de Mouvement, et ignorez le dé ayant donné le résultat le plus bas.
}

\newcommand{\smellslikegreenspiritrule}{%
La \scrapwagon{} gagne les règles \distracting{} et \hardtarget{}.
}

\newcommand{\smasherrule}{%
La \scrapwagon{} obtient Force 5.
}

\newcommand{\trollbelchrule}{%
Au lieu d'attaquer normalement au Corps à Corps, un Troll peut choisir de faire une unique Attaque Spéciale de Corps à Corps qui touche automatiquement avec Force 5 et \armourpiercing{6}.
}


\newcommand{\giantattacksrule}{%
Quand un Géant attaque au Corps à Corps, au lieu d'attaquer normalement, choisissez une unité en contact socle à socle à attaquer et lancez 1D6. Déterminez ci-dessous la table à utiliser selon le Type de Troupe de l'unité visée, et trouvez l'attaque correspondant au résultat du dé. Il est important de noter que les Attaques de Géant comptent comme des attaques de Corps à Corps et suivent ainsi normalement les règles affectant ces attaques. Le Géant peut également faire son \stomp{}.
}

\newcommand{\giantattackstable}{%
\vspace*{-0.2cm}\begin{multicols}{2}
	\noindent Si l'unité est de type \infantry{}, \warbeast{}, \cavalry{}, \swarm{} ou \warmachine{}  :	
	\renewcommand{\arraystretch}{1.5}	
	\begin{center}\begin{tabular}{cl}
   		\hline
		1 & \bellow{} \tabularnewline
		2 & \jump{} \tabularnewline
		3 & \grab{} \tabularnewline
		4-6 & \swing{} \tabularnewline
		\hline
	\end{tabular}\end{center}
	\vspace*{\fill}
	\columnbreak
	Si l'unité est de type \monstrousinfantry{}, \monstrouscavalry{}, \monstrousbeast{}, \chariot{}, \monster{} ou \riddenmonster{} :
	\begin{center}\begin{tabular}{cl}
		\hline	
		1 & \bellow{} \tabularnewline
		2-3 & \thump{} \tabularnewline
		4-6 & \smashname{} \tabularnewline
		\hline
	\end{tabular}\end{center}
	\vspace*{\fill}
	\renewcommand{\arraystretch}{1.2} % return to default
\end{multicols}
}

\newcommand{\bellowrule}{%
Ni le Géant, ni l'unité sélectionnée ne peuvent faire d'attaques de Corps à Corps pendant cette phase. Les attaques déjà réalisées, incluant celles réalisées simultanément avec cette attaque, ne sont pas concernées. Le camp du Géant gagne automatiquement le combat de 2. Si deux Géants opposés, ou plus, \og{} Hurlent \fg{}, le combat est un match nul.
}

\newcommand{\jumprule}{%
L'unité sélectionnée subit 1D6 touches de la Force du Géant. Le Géant doit faire un test de Terrain Dangereux.
}

\newcommand{\grabrule}{%
Choisissez une figurine dans l'unité sélectionnée et en contact socle à socle avec le Géant. Cette figurine doit faire un test de Force et un test de Capacité de Combat. Pour chaque test raté, la figurine subit une touche de la Force du Géant et suivant la règle spéciale \multiplewounds{1D3}{}.
}

\newcommand{\swingrule}{%
Le Géant fait 2D6 attaques sur l'unité choisie.
}

\newcommand{\thumprule}{%
Choisissez une figurine en contact socle à socle avec le Géant dans l'unité sélectionnée. Cette figurine doit faire un test d'Initiative. Si elle échoue, la figurine subit 2D6 blessures avec \armourpiercing{6}.
}

\newcommand{\smashrule}{%
Choisissez une figurine en contact socle à socle avec le Géant dans l'unité sélectionnée. Cette figurine subit une blessure avec \armourpiercing{6}. Si la figurine n'a pas encore attaqué, elle ne peut pas le faire au cours de cette manche. Si la figurine a déjà réalisé ses attaques, elle ne pourra pas attaquer au cours du tour de  joueur à venir.
}

\newcommand{\ballistarule}{%
\textbf{\artilleryweapon} de type \textbf{\boltthrower}. \range{48}, Force 6, \multiplewounds{1D3}{}, \armourpiercing{6}.
}

\newcommand{\lookatemgorule}{%
Le \gnasherwreckingteam{} gagne la règle \runningamok{} quand il touche une unité pour la première fois de la partie.
}

\newcommand{\weblauncherrule}{%
\textbf{\artilleryweapon} de type \textbf{\catapult{} (\distance{3})}. \range{6-36}, Force 3.\newline Les unités touchées souffrent d'un malus de -1D3 à l'Initiative. Les Terrains Dangereux (1) comptent pour eux comme des Terrains Dangereux (2), et tous les autres éléments de décor, Terrain Découvert compris, comptent pour eux comme des Terrains Dangereux (1). Ces effets durent jusqu'à la fin du prochain Tour de Joueur. Les effets de différents \weblaunchers{} ne s'additionnent pas.
}

\newcommand{\smashemflatrule}{%
Si l'\greatgreenidol{} est engagée au Corps à Corps, toutes les unités amies à moins de \distance{8} gagnent la règle \devastatingcharge{}.
}

\newcommand{\wevegotthegreenlightrule}{%
Si l'\greatgreenidol{} a déclaré une charge durant cette phase, toutes les unités amies à moins de \distance{8} peuvent relancer les dés pour déterminer leur distance de charge.
}

\newcommand{\bouncersrule}{%
Ne peut rejoindre que des unités de \cavegnashers{} et de \gnasherdashers{}, ignorez pour cela les restrictions des règles \skirmishers{} et \oiitbites{}.
}

\newcommand{\spidermothershrinerule}{%
Le Sorcier gagne la règle \pathmaster{}. Toutes les figurines amies avec la capacité de \channel{} à moins de \distance{12} ajoutent +2 au lieu de +1 à la tentative de \channel{}.
}

% QRS note

\newcommand{\QRSnote}{%
\noindent$^{1}$ Un \ironorc{} gagne +1 CC.

\noindent$^{2}$ Un \cavegoblin{} gagne +1 I.

\noindent$^{3}$ Un \cavegoblin{} gagne +1 I et -1 Cd.

\noindent$^{4}$ Un membre d'équipage de moins quand il sert de monture.
}




\begin{document}

\newgeometry{margin=1in}

% Table options
\arrayrulecolor{black!30}
\setlength{\arrayrulewidth}{0.5pt}
\renewcommand{\arraystretch}{1.2}

\begin{titlepage}
\begin{center}

\ifdef{\booktitle}{}{\newcommand{\booktitle}{Missing title}}
\ifdef{\version}{}{\newcommand{\version}{Missing version}}

{\antiquefont\fontsize{40}{48}\selectfont\noindent\labels@fantasybattles

\labels@NinthAge}

\vspace*{0.5cm}
\ifdef{\booklogo}{%
\includegraphics[height=10cm]{\booklogo}%
}{%
\includegraphics[height=10cm]{../Layout/pics/logo_9th.png}%
}

\vspace*{-1cm}
{\antiquefont\fontsize{50}{60}\selectfont \booktitle
\vspace{0.4cm}

\fontsize{14}{16.8}\selectfont \labels@armyrules{}

Beta v\version{} - \today{}}

\ifdef{\frenchversion}{{\fontsize{14}{16.8}\selectfont \vspace{0.2cm}\noindent\texttt{VF \frenchversion}}}{}
\vfill

\begin{tabular}{@{}m{2cm}@{\hskip 20pt}m{13cm}@{}}
\includegraphics[width=2cm]{../Layout/pics/seal_9th.png} &
{\fontsize{10}{12}\selectfont \textcolor{black!50}{\noindent\labels@frontpagecredits}}

\ifdef{\frontpageaddstuff}{{\fontsize{10}{12}\selectfont \noindent\textcolor{black!50}{\frontpageaddstuff}}}{}

\vspace*{10pt}
\noindent{\fontsize{10}{12}\selectfont \textcolor{black!50}{\labels@license}}
\tabularnewline
\end{tabular}


\end{center}

\newpage

\thispagestyle{empty}

{\fontsize{10}{12}\selectfont

\begin{center}\noindent{\Largerfontsize\textbf{\labels@tableofcontents}}\end{center}

\vspace*{0.2cm}\begin{multicols}{2}

\tocfirstcolumn

\vspace*{\fill}\columnbreak

\tocentry{lordtitle}{\labels@lords}

\tocentry{herotitle}{\labels@heroes}

\ifdef{\tocmounts}{\tocentry{mountstitle}{\tocmounts}}{}

\tocentry{coretitle}{\labels@coreunits}

\tocentry{specialtitle}{\labels@specialunits}

\tocentry{raretitle}{\labels@rareunits}

\vspace*{\fill}\end{multicols}

\ifdef{\labels@introduction}{\vspace{0.7cm}\labels@introduction}{\vphantom{1pt}}
\vfill

\noindent\newrule{\labels@rulechanges}

\bigskip
\noindent \labels@latexcredit
}


\end{titlepage}

\restoregeometry

\startarmyspecialrules

\armyspecialruleentry{\greenhideraces}

Certaines figurines de l'armée disposent d'un ensemble de règles spéciales qui correspondent à leur race.

\newcommand{\logosize}{3cm}
\begin{multicols}{3}\raggedcolumns

\begin{center}
\includegraphics[width=\logosize]{pics/commonorc.png}
\vspace*{-1cm}\subsubtitle{\commonorc}

\unruly{}, \borntofight{}
\end{center}

\columnbreak

\begin{center}
\includegraphics[width=\logosize]{pics/ironorc.png}
\vspace*{-1cm}\subsubtitle{\ironorc}

\immunetopsychology{}, \weaponmaster{}, \borntofight{}
\end{center}

\columnbreak

\begin{center}
\includegraphics[width=\logosize]{pics/feralorc.png}
\vspace*{-1cm}\subsubtitle{\feralorc}

\frenzy{}, \unruly{}, \borntofight{}, \wardsave{6}
\end{center}

\end{multicols}

\begin{multicols}{3}\raggedcolumns

\begin{center}
\includegraphics[width=\logosize]{pics/commongoblin.png}
\vspace*{-1cm}\subsubtitle{\commongoblin}

\unruly{}, \insignificant{}
\end{center}

\columnbreak

\begin{center}
\includegraphics[width=\logosize]{pics/cavegoblin.png}
\vspace*{-1cm}\subsubtitle{\cavegoblin}

\hatred{} (Livre d'Armée : Forteresses Naines), \unruly{}, \insignificant{}
\end{center}

\columnbreak

\begin{center}
\includegraphics[width=\logosize]{pics/forestgoblin.png}
\vspace*{-1cm}\subsubtitle{\forestgoblin}

\strider{\forest}, \unruly{}, \insignificant{}
\end{center}

\end{multicols}

\armyspecialruleentry{\unruly}

L'unité a un malus de -1 en Commandement pour les tests de \frenzy{} et pour se réfréner de poursuivre. Si l'unité est en formation de Horde, jetez trois dés pour les tests de Panique et ignorez le dé ayant donné le plus grand résultat.

\armyspecialruleentry{\borntofight}

L'élément de figurine gagne +1 en Force lors de la première Manche de Corps à Corps.

\begin{multicols}{2}\raggedcolumns
\armyspecialruleentry{\waaargh}

Une fois par partie, le Général peut déclarer une \waaargh{} au début du tour de n'importe quel joueur. Toutes les figurines ayant des éléments appartenant à une Race de Peaux Vertes gagnent +1 en Mouvement et la règle \swiftstride{} jusqu'à la fin du Tour de Joueur.

\columnbreak
\armyspecialruleentry{\greentide}

Une fois par partie, le Général peut déclarer une \greentide{} au début du tour de n'importe quel joueur. Tous les éléments de figurine appartenant à une Race Peau Verte de \goblins{} gagnent la règle \fightinextrarank{} jusqu'à la fin du Tour de Joueur suivant.

\end{multicols}

\armyspecialruleentry{\venomousfangs}

Désignez une des attaques de Corps à Corps de l'élément de figurine avant de lancer les dés pour toucher. Cette attaque gagne \multiplewounds{\ordnance}{}.

\armyspecialruleentry{\shambolic{X}}

L'unité a \randommovement{X}. Elle gagne la règle \immunetopsychology{} et ne peut pas être rejointe par des Personnages. Si elle obtient le même résultat sur tous les dés lors de son jet de \randommovement{}, elle subit 1D3 blessures sans sauvegarde d'aucune sorte, puis se déplace de cette distance dans une direction aléatoire.

Si elle arrive au contact d'un élément de décor autre qu'un Terrain Découvert ou une Colline, si elle arrive au contact d'un bord de table, ou si elle s'arrête à \distance{1} d'un Terrain Infranchissable, elle doit faire un test de Terrain Dangereux (2).

\armyspecialruleentry{\runningamok}

Une unité avec \shambolic{} et \runningamok{} se déplace toujours dans une direction aléatoire lors de son \randommovement{}.

\armyspecialruleentry{\ricochet{X}}

Les figurines avec \shambolic{} et \ricochet{} ignorent l'écart d'\distance{1} entre unités pendant leur \randommovement{}. Si une figurine avec cette règle touche une autre unité, amie ou ennemie, elle continue son mouvement dans la même direction jusqu'à respecter la règle des \distance{1} d'écart entre unités. Cela peut amener cette figurine à traverser plusieurs unités lors du même mouvement. Si elle ne peut être placée à la fin, elle est retirée comme perte.

Toute unité traversée durant le déplacement jusqu'à la distance obtenue sur les dés pour le \randommovement{} est touchée. Les unités touchées subissent X touches, X étant la valeur entre les parenthèses. Les unités touchées engagées dans un même Corps à Corps ne comptent que comme une seule unité pour la détermination des touches. Le joueur possédant la figurine avec \ricochet{} répartit les touches infligées entre les unités aussi équitablement que possible, puis applique les règles normales de distribution des touches dans chaque unité.

Il n'est pas possible de charger une figurine avec \ricochet{}, mais une unité peut charger, fuir, poursuivre ou se déplacer à travers elle. L'unité subit alors X+1D6 touches, puis la figurine avec \ricochet{} est retirée du champ de bataille en tant que perte.

Toutes les touches sont infligées avec la Force non modifiée de la figurine et ont \armourpiercing{1}.


\closearmyspecialrules


\vspace{1.5cm}
\startarmyarmoury

\startitemlistonecol

\listitemonecol{\powershrooms} Générez 1D3+1 \powershrooms{} à la fin du déploiement pour le porteur. Avant de jeter les dés pour lancer un sort, le porteur peut décider d'utiliser un seul \powershroom{} pour ajouter +1 au résultat de lancement du sort. C'est une exception à la limitation des Modificateurs Magiques. Un \powershroom{} est à usage unique. Après l'avoir utilisé, lancez 1D6. Si le dé donne un \result{1}, le Sorcier subit une touche de Force 3 sans sauvegarde d'aucune sorte.

\listitemonecol{\mammothstabber} Si l'unité possède au moins un Rang Complet, elle gagne 1D3 \impacthits{} de Force 5 avec \multiplewounds{\ordnance}{\largetarget}.

\enditemlistonecol

\closearmyarmoury




\startarmymagicalitems

\armymagicalweapons

\startpricelist

\pricelistitem{Hache de l'Aporcalypse}{65/50}Type : \hw{}. Le porteur gagne +1D3 en Force et +1D3 Attaques. Ces bonus sont déterminés et prennent effet au palier d'Initiative auquel le porteur attaque avec cette arme.

\pricelistitem{Arc Zap de Maza}{30}Type : Arc. \range{24}, Force 3, \lightningattacks{}, \multipleshots{3}. L'unité du porteur gagne la règle spéciale \quicktofire{}.

\pricelistitem{Dague de Sournois}{15}Type : \hw{}. Les attaques portées avec cette arme ont \armourpiercing{1}. Si le porteur attaque une unité ennemie de flanc ou de dos, ses attaques avec cette arme gagnent +2 en Force.

\endpricelist

\armymagicalarmour

\startpricelist

\pricelistitem{Couronne du Roi des Cavernes}{40}\goblin{} uniquement. Ne peut pas être pris par une \largetarget{}.\newline Type : Aucun. Sauvegarde d'Armure de 6+. Le porteur ne peut rejoindre ou être rejoint que par une unité dont toutes les figurines ont au moins un élément qui partage la même \greenhiderace{} que lui. L'unité du porteur gagne alors la règle \vanguard{} et peut se déplacer après un ralliement, mais elle ne peut alors ni tirer, ni faire de Marche Forcée. De plus, la portée de la \inspiringpresence{} ou de \holdyourground{} du porteur est augmentée de \distance{6}.

\pricelistitem{Plaques de Tuktek}{35}Type : \ha{}. Le porteur gagne +1 en Endurance et la figurine du porteur gagne 1D3 \impacthits{}.

\endpricelist

\armytalismans

\startpricelist

\pricelistitem{Chapardeur de Défense}{25}\goblin{} uniquement.\newline Si le porteur subit une blessure, il peut utiliser la Sauvegarde d'Armure, la \wardsave{}, la \regeneration{} et la \magicresistance{} de la figurine lui ayant infligé cette blessure.

\endpricelist

\armyenchanteditems

\startpricelist

\pricelistitem{Peintures Waaargh!}{30}\feralorc{} uniquement.\newline Le porteur gagne la \frenzy{} et ne peut jamais la perdre. De plus, tous les \feralorcs{} de l'unité du porteur gagnent la \frenzy{} tant que ce dernier est dans l'unité. Enfin, l'unité gagne la règle \swiftstride{} pour les mouvements de Poursuite et de Charge Irrésistible.

\pricelistitem{Patte de Sanglier}{20}Figurine montée uniquement.\newline Toutes les figurines alliées de type Cavalerie à moins de \distance{18} du porteur peuvent relancer leurs tests de Terrain Dangereux.

\endpricelist

\armymagicalbanners

\startpricelist

\pricelistitem{Totem de Mikinok}{40}Les autres Objets Magiques dans l'unité du porteur ainsi que dans les unités amies ou ennemies en contact avec l'unité du porteur n'ont plus d'effet et redeviennent des objets ordinaires. Cet effet dure tant que les unités restent en contact.

\pricelistitem{Icône de Blindage}{25}L'unité du porteur gagne une \wardsave{5} contre les Attaques de Tir.

\endpricelist

\closearmymagicalitems



%%% START OF THE ARMYLIST - Translators shouldn't have to edit it %%%

%%% v0.99.2

\armylist

\lordstitle

\showunit{
	name={\orcwarlord},
	QRSname={\orcwarlordinQRS{}$^{1}$},
	cost={120},
	profile={ < 4 6 3 5 5 3 4 4 9},
	type=\infantry{},
	basesize=25x25,
	unitsize=1,
	options={
		\magicalitemsallowance{}=\upto{}<100,
		\maygain{} \waaargh{} \only{\general}=20,
		\anyofthefollowing{
			\shield{}=5,
			\pw{}=5,
			\gw{}=15,
			\lance{}=15,
		},
	},
	additional={%
		\startraceforking{3}%
			\forkraceCommonOrc{\free}%
		
			\def\temparmour{\la{}}%
			\armour{\temparmour}

			\def\tempoptions{\ha{}=12,}%
			\vspace*{0.1cm}\options{\tempoptions}

			\def\tempmounts{\warboar{}=20,\orcboarchariot{}=30,\wyvern{}=120}%
			\mounts{\tempmounts}

		\vspace*{\fill}
		\columnbreak
			\forkraceIronOrc{\pts{20}}%

			\profilemodification{+1 \labels@WS{}}

			\def\temparmour{\ha{}}%
			\armour{\temparmour}

			\def\tempoptions{\platearmour{}=20,}%
			\vspace*{0.1cm}\options{\tempoptions}

			\def\tempmounts{\warboar{}=20,\orcboarchariot{}=30,\wyvern{}=120}%
			\mounts{\tempmounts}

		\vspace*{\fill}
		\columnbreak
			\forkraceFeralOrc{\pts{15}}%

			\def\tempmounts{\warboar{}=10,\wyvern{}=105}%
			\mounts{\tempmounts}

		\vspace*{\fill}
		\closeraceforking{}%
	},
}

\showunit{
	name={\orcbigshaman},
	cost={175},
	profile={ < 4 3 3 4 5 3 2 1 8},
	type=\infantry{},
	basesize=25x25,
	unitsize=1,
	magiclevel=3,
	magicpaths={\thebiggreengods{},\wilderness{}},
	options={
		\magicalitemsallowance{}=\upto{}<100,
		\magiclevel{4}=30,
	},
	additional={%
		\startraceforking{2}%
			\forkraceCommonOrc{\free}%

			\def\tempmounts{\warboar{}=20,\orcboarchariot{}=20,\wyvern{}=120}%
			\mounts{\tempmounts}

		\vspace*{\fill}
		\columnbreak
			\forkraceFeralOrc{\pts{5}}%

			\def\tempmounts{\warboar{}=20,\wyvern{}=120}%
			\mounts{\tempmounts}

		\vspace*{\fill}
		\closeraceforking{}%
	},
}

\showunit{
	name={\goblinking},
	QRSname={\goblinking{}$^{2}$},
	cost={60},
	profile={ < 4 5 4 4 4 3 4 4 8},
	type=\infantry{},
	basesize=20x20,
	unitsize=1,
	armour={\la},
	options={
		\magicalitemsallowance{}=\upto{}<100,
		\maygain{} \greentide{} \only{\general}=10,
		\anyofthefollowing{
			\shield{}=5,
			\ha{}=8,
		},
		\maytakeashortbow{}=5,
		\weapononechoice{
			\pw{}=5,
			\gw{}=15,
			\lance{}=15,
		},
	},
	additional={%
		\startraceforking{3}%
			\forkraceCommonGoblin{\free}%

			\def\tempmounts{\wolf{}=15,\goblinwolfchariot{}=25}%
			\mounts{\tempmounts}

		\vspace*{\fill}		
		\columnbreak
			\forkraceCaveGoblin{\pts{5}}%

			\profilemodification{+1 \labelforinit{}}

			\def\tempmounts{\cavegnasher{}=20}%
			\mounts{\tempmounts}

		\vspace*{\fill}
		\columnbreak
			\forkraceForestGoblin{\free}%

			\def\tempoptions{\poisonedattacks{}=10}
			\options{\tempoptions}
			
			\def\tempmounts{\scuttlerspider{}=20,\huntsmenspider{}=20,\gargantula{}=250}%
			\mounts{\tempmounts}

		\vspace*{\fill}
		\closeraceforking{}%
	},
}

\showunit{
	name={\goblinbigshaman},
	QRSname={\goblinbigshaman{}$^{3}$},
	cost={170},
	profile={ < 4 2 3 3 4 3 2 1 7},
	type=\infantry{},
	basesize=20x20,
	unitsize=1,
	magiclevel=3,
	magicpaths={\thelittlegreengods{},\shadows{}},
	options={
		\magicalitemsallowance{}=\upto{}<100,
		\magiclevel{4}=30,
	},
	additional={%
		\startraceforking{3}%
			\forkraceCommonGoblin{\free}%

			\def\tempmounts{\wolf{}=15,\goblinwolfchariot{}=20}%
			\mounts{\tempmounts}

		\vspace*{\fill}		
		\columnbreak
			\forkraceCaveGoblin{\free}%

			\profilemodification{+1 \labelforinit{} \wordand{} -1 \labels@Ld{}}

			\def\tempoptions{\powershrooms{}=15}
			\options{\tempoptions}

		\vspace*{\fill}
		\columnbreak
			\forkraceForestGoblin{\free}%
			
			\def\tempmounts{\scuttlerspider{}=15,\gargantula{}=250}%
			\mounts{\tempmounts}

		\vspace*{\fill}
		\closeraceforking{}%
	},
}

\heroestitle

\showunit{
	name={\orcchief},
	QRSname={\orcchief{}$^{1}$},
	cost={50},
	profile={ < 4 5 3 4 5 2 3 3 8},
	type=\infantry{},
	basesize=25x25,
	unitsize=1,
	options={
		\bsboption{}=25,
		\magicalitemsallowance{}=\upto{}<50,
		\maygain{} \waaargh{} \only{\general}=10,
		\anyofthefollowing{
			\shield{}=5,
			\pw{}=5,
			\gw{}=10,
			\lance{}=10,
		},
	},
	additional={%
		\startraceforking{3}%
			\forkraceCommonOrc{\free}%
		
			\def\temparmour{\la{}}%
			\armour{\temparmour}

			\def\tempoptions{\ha{}=5,}%
			\vspace*{0.1cm}\options{\tempoptions}

			\def\tempmounts{\warboar{}=15,\orcboarchariot{}=60,\wyvern{}=150}%
			\mounts{\tempmounts}

		\vspace*{\fill}		
		\columnbreak
			\forkraceIronOrc{\pts{10}}%

			\profilemodification{+1 \labels@WS{}}

			\def\temparmour{\ha{}}%
			\armour{\temparmour}

			\def\tempoptions{\platearmour{}=15,}%
			\vspace*{0.1cm}\options{\tempoptions}

			\def\tempmounts{\warboar{}=15,\wyvern{}=150}%
			\mounts{\tempmounts}

		\vspace*{\fill}
		\columnbreak
			\forkraceFeralOrc{\pts{5}}%

			\def\tempmounts{\warboar{}=15,\wyvern{}=150}%
			\mounts{\tempmounts}

		\vspace*{\fill}
		\closeraceforking{}%
	},
}

\showunit{
	name={\orcshaman},
	cost={65},
	profile={ < 4 3 3 3 4 2 2 1 7},
	type=\infantry{},
	basesize=25x25,
	unitsize=1,
	magiclevel=1,
	magicpaths={\thebiggreengods{},\wilderness{}},
	options={
		\magicalitemsallowance{}=\upto{}<50,
		\magiclevel{2}=25,
	},
	additional={%
		\startraceforking{2}%
			\forkraceCommonOrc{\free}%

			\def\tempmounts{\warboar{}=15,\orcboarchariot{}=50}%
			\mounts{\tempmounts}

		\vspace*{\fill}
		\columnbreak
			\forkraceFeralOrc{\pts{5}}%

			\def\tempmounts{\warboar{}=15,}%
			\mounts{\tempmounts}

		\vspace*{\fill}
		\closeraceforking{}%
	},
}

\showunit{
	name={\goblinchief},
	QRSname={\goblinchief{}$^{3}$},
	cost={35},
	profile={ < 4 4 4 4 4 2 3 3 7},
	type=\infantry{},
	basesize=20x20,
	unitsize=1,
	armour={\la},
	options={
		\bsboption{}=25,
		\magicalitemsallowance{}=\upto{}<50,
		\maygain{} \greentide{} \only{\general}=20,
		\maytakeashield{}=\free{},
		\maytakeashortbow{}=3,
		\weapononechoice{
			\pw{}=3,
			\lightlance{}=3,
			\gw{}=6,
			\lance{}=6,
		},
	},
	additional={%
		\startraceforking{3}%
			\forkraceCommonGoblin{\free}%

			\def\tempoptions{\ha{}=5,}%
			\options{\tempoptions}
			
			\def\tempmounts{\wolf{}=20,\goblinwolfchariot{}=45}%
			\mounts{\tempmounts}

		\vspace*{\fill}		
		\columnbreak
			\forkraceCaveGoblin{\free}%

			\profilemodification{+1 \labelforinit{} \wordand{} -1 \labels@Ld{}}

			\def\tempmounts{\cavegnasher{}=35}%
			\mounts{\tempmounts}

		\vspace*{\fill}
		\columnbreak
			\forkraceForestGoblin{\free}%

			\def\tempoptions{\poisonedattacks{}=5}
			\options{\tempoptions}
			
			\def\tempmounts{\scuttlerspider{}=15,\huntsmenspider{}=25}%
			\mounts{\tempmounts}

		\vspace*{\fill}
		\closeraceforking{}%
	},
}

\showunit{
	name={\goblinshaman},
	QRSname={\goblinshaman{}$^{3}$},
	cost={60},
	profile={ < 4 2 3 3 3 2 2 1 6},
	type=\infantry{},
	basesize=20x20,
	unitsize=1,
	magiclevel=1,
	magicpaths={\thelittlegreengods{}},
	options={
		\magicalitemsallowance{}=\upto{}<50,
		\magiclevel{2}=25,
	},
	additional={%
		\startraceforking{3}%
			\forkraceCommonGoblin{\free}%

			\def\tempmounts{\wolf{}=15,\goblinwolfchariot{}=40}%
			\mounts{\tempmounts}
	
		\vspace*{\fill}	
		\columnbreak
			\forkraceCaveGoblin{\free}%

			\profilemodification{+1 \labelforinit{} \wordand{} -1 \labels@Ld{}}

			\def\tempoptions{\powershrooms{}=15}
			\options{\tempoptions}

		\vspace*{\fill}
		\columnbreak
			\forkraceForestGoblin{\free}%
			
			\def\tempmounts{\scuttlerspider{}=15}%
			\mounts{\tempmounts}

		\vspace*{\fill}
		\closeraceforking{}%
	},
}

\coreunitstitle

\showunit{
	name={\orcs},
	cost={90},
	profile={< 4 3 3 3 4 1 2 1 7},
	type=\infantry{},
	basesize=25x25,
	unitsize=20,
	maxmodels=50,
	costpermodel=6,
	options={
		\anyofthefollowing{
			\shield{}=\permodel{}<1,
			\bow{}=\permodel{}<1,
			\pw{}=\permodel{}<1,
			\spear{}=\permodel{}<1,
		},
	},
	commandgroup={champion=10, musician=10, banner=10, veteranstandardbearer=yessir},
	additional={%
		\startraceforking{2}%
			\forkraceCommonOrc{\free}%

			\def\temparmour{\la}%
			\armour{\temparmour}%
			
			\def\tempoptions{\mayexchangeallequipmentforcrossbowandha{}=\permodel{}<4,}%
			\vspace*{0.1cm}\options{\tempoptions}%

		\vspace*{\fill}
		\columnbreak
			\forkraceFeralOrc{\pts{2}\permodel}%

			\def\tempoptions{\maytakeamammothstabber{}=15,}%
			\options{\tempoptions}%

		\vspace*{\fill}
		\closeraceforking{}%
	},
}

\showunit{
	name={\orceadbashers{} (\oneofakind)},
	cost={70},
	profile={< 4 4 3 4 4 1 2 1 7},
	type=\infantry{},
	basesize=25x25,
	unitsize=10,
	maxmodels=40,
	costpermodel=9,
	commandgroup={champion=10, musician=10, banner=10, veteranstandardbearer=yessir},
	options={%
		\anyofthefollowing{
			\shield{}=\permodel{}<1,
			\pw{}=\permodel{}<1,
			\spear{}=\permodel{}<1,
		},
	},
	additional={%
		\startraceforking{2}%
			\forkraceCommonOrc{\free}%

			\def\temparmour{\la}%
			\armour{\temparmour}%

		\vspace*{\fill}
		\columnbreak
			\forkraceFeralOrc{\pts{1}\permodel}%

			\def\tempoptions{%
				\maytakeamammothstabber{}=15,
			}%
			\options{\tempoptions}%

		\vspace*{\fill}
		\closeraceforking{}%
	},
}

\showunit{
	name={\goblins},
	QRSname={\goblins{}$^{3}$},
	cost={60},
	profile={ < 4 2 3 3 3 1 2 1 6},
	type=\infantry{},
	basesize=20x20,
	unitsize=20,
	maxmodels=60,
	costpermodel=3,
	options={
		\maytakeoneofthefollowing{
			\shortbow{}=\free{},
			\shield{}=\permodel{}<1,
			\spear{} \&{} \shield{}=\permodel{}<1,
		},
	},
	commandgroup={champion=10, musician=10, banner=10, veteranstandardbearer=yessir},
	additional={%
		\startraceforking{3}%
			\forkraceCommonGoblin{\free}%

			\def\temparmour{\la}
			\armour{\temparmour}
			
			\def\tempoptions{%
				\takeshadygits{}\dotfill{}=\pershadygit{}<15,
				\exchangeallweaponsforshieldandshortbow{}=\permodel{}<1,	
			}%
			\vspace*{0.1cm}\options{\tempoptions}

		\vspace*{\fill}		
		\columnbreak
			\forkraceCaveGoblin{\free}%

			\profilemodification{+1 \labelforinit{} \wordand{} -1 \labels@Ld{}}

			\def\tempoptions{%
				\takemadgits{}=\permadgit{}<30,
				\takenets{}=\permodel{}<1,	
			}%
			\options{\tempoptions}

		\vspace*{\fill}
		\columnbreak
			\forkraceForestGoblin{\free}%

			\def\tempoptions{%
				\throwingweapons{}=\permodel{}<1,
				\motherskiss{}=\permodel{}<1,
				\skirmishers{} \ifNmodelsorless{20}\dotfill{}=\permodel{}<1,
			}
			\options{\tempoptions}

		\vspace*{\fill}
		\closeraceforking{}%
		
		\def\tempunitrules{%
			\unitrule{\nets}{\netsrule}
			\unitrule{\motherskiss}{\motherskissrule}
		}
		\vspace*{0.1cm}\unitrules{\tempunitrules}
	},
}

\showunit{
	name={\shadygit},
	profile={ < 4 4 3 3 3 1 3 2 6},
	cost={15},
	type=\infantry{},
	basesize=20x20,
	unitsize=SPECIAL-\gitnote{},
	greenhiderace={\commongoblin},
	weapons={\pw},
	armour={\la},
	specialrules={\lethalstrike{},\sneaky{}},
	additional={%
		\def\tempunitrules{%
			\unitrule{\sneaky}{\sneakyrule}
		}
		\unitrules{\tempunitrules}		
	},
}

\showunit{
	name={\madgit},
	profile={ < \starsymbol{} - - 5 3 1 3 1 5},
	cost={30},
	type=\infantry{},
	basesize=25,
	unitsize=SPECIAL-\gitnote{},
	greenhiderace={\cavegoblin},
	specialrules={\shambolic{2D6},\runningamok{},\ricochet{1D6},\hardtarget{},\surprise{}},
	additional={%
		\def\tempunitrules{%
			\unitrule{\surprise}{\surpriserule}
		}
		\vspace*{0.1cm}\unitrules{\tempunitrules}		
	},
}

\showunit{
	name={\goblinraiders},
	cost={60},
	profile={
		\rider{}< 4 2 3 3 3 1 2 1 6,
		[\wolf{}]< 9 3 - 3 3 1 3 1 3,
		[\scuttlerspider{}]< 7 3 - 3 3 1 4 1 2,
	},
	type=\cavalry{},
	basesize=25x50,
	unitsize=5,
	maxmodels=20,
	costpermodel=8,
	specialrules={\fastcavalry},
	options={
		\musttakeoneormoreofthefollowing{
			\shield{}=\permodel{}<1,
			\lightlance{}=\permodel{}<1,
			\shortbow{}=\permodel{}<1,
			\throwingweapons{} \only{\forestgoblin}=\permodel{}<1,
		},
	},
	commandgroup={champion=10, musician=10, banner=10},
	additional={%
		\startraceforkingrider{2}%
			\forkraceCommonGoblin{\free}%

			\vspace*{-0.3cm}{\setlength{\parskip}{0.3cm}
			\def\temparmour{\la{},\mountsprotection{6}}
			\armour{\temparmour}
			
			\noindent\unitentryformat{\labels@mount\spacebeforecolon{}:}\newline\wolf{}.
			}

		\vspace*{\fill}
		\columnbreak
			\forkraceForestGoblin{\free}%

			\vspace*{-0.3cm}{\setlength{\parskip}{0.3cm}
			\def\temparmour{\mountsprotection{6}}
			\armour{\temparmour}

			\def\tempspecialrules{\scout{},\strider{},\poisonedattacks{} \only{\scuttlerspider}}
			\specialrules{\tempspecialrules}
			
			\noindent\unitentryformat{\labels@mount\spacebeforecolon{}:}\newline\scuttlerspider{}.
			}

		\vspace*{\fill}			
		\closeraceforking{}%
	},
}

\showunit{
	name={\orcboarriders},
	cost={70},
	profile={%
		\rider{}< 4 3 3 3 4 1 2 1 7,
		\warboar{}< 7 3 - 3 3 1 3 1 3,
	},
	type=\cavalry{},
	basesize=25x50,
	unitsize=5,
	maxmodels=15,
	costpermodel=13,
	weapons={\lightlance},
	specialrules={\thunderouscharge{} \only{\warboar}},
	options={
		\maytakeashield{}=\permodel{}<3,
	},
	commandgroup={champion=10, musician=10, banner=10, veteranstandardbearer=yessir},
	additional={%
		\startraceforkingrider{2}%
			\forkraceCommonOrc{\free}%

			\def\temparmour{\la{},\mountsprotection{5}}%
			\armour{\temparmour}%
			
			\def\tempoptions{\maytakealance{}=\permodel{}<3,}%
			\vspace*{0.1cm}\options{\tempoptions}%

		\vspace*{\fill}
		\columnbreak
			\forkraceFeralOrc{\pts{1}\permodel}%

			\def\temparmour{\mountsprotection{5}}%
			\armour{\temparmour}%
			
			\def\tempoptions{\maytakepw{}=\permodel{}<2,}%
			\vspace*{0.1cm}\options{\tempoptions}%
			
		\vspace*{\fill}
		\closeraceforking{}%
	},
}


\specialunitstitle

\showunit{
	name={\ironorcsunit},
	cost={100},
	profile={< 4 5 3 4 4 1 2 1 8},
	type=\infantry{},
	basesize=25x25,
	unitsize=10,
	maxmodels=35,
	costpermodel=13,
	greenhiderace={\ironorc},
	weapons={\gw{},\pw{}},
	armour={\ha{},\shield{}},
	specialrules={\bodyguard{\ironorcwarlord{}, \ironorcchief{}}},
	options={
		\maytakeplatearmour{}=\permodel{}<2,
	},
	commandgroup={champion=10, musician=10, banner=10, bannerallowance=50},
}

\showunit{
	name={\mountedeadbashers},
	cost={80},
	profile={%
		\rider{}< 4 4 3 4 4 1 2 1 8,
		\warboar{}< 7 3 - 3 3 1 3 1 3,
	},
	type=\cavalry{},
	basesize=25x50,
	unitsize=5,
	maxmodels=15,
	costpermodel=16,
	weapons={\lightlance},
	specialrules={\thunderouscharge{} \only{\warboar}},
	options={
		\maytakeashield{}=\permodel{}<3,
	},
	commandgroup={champion=10, musician=10, banner=10, bannerallowance=50},
	additional={%
		\startraceforkingrider{2}%
			\forkraceCommonOrc{\free}%

			\def\temparmour{\la{},\mountsprotection{5}}%
			\armour{\temparmour}%
			
			\def\tempoptions{%
				\maytakealance{}=\permodel{}<3,
				\maytakeha{}=\permodel{}<3,
			}%
			\vspace*{0.1cm}\options{\tempoptions}%

		\vspace*{\fill}
		\columnbreak
			\forkraceFeralOrc{\pts{1}\permodel}%

			\def\temparmour{\mountsprotection{5}}%
			\armour{\temparmour}%
			
			\def\tempoptions{\maytakepw{}=\permodel{}<3,}%
			\vspace*{0.1cm}\options{\tempoptions}%

		\vspace*{\fill}
		\closeraceforking{}%
	},
}

\showunit{
	name={\orcboarchariot},
	QRSname={\orcboarchariot{}$^{4}$},
	cost={85},
	profile={%
		\chariot{}< - - - 5 5 4 - - -,
		\eadbasherrider{} (2)< - 4 3 4 - - 2 1 7,
		\warboar{} (2)< 7 3 - 3 - - 3 1 3,
	},
	type=\chariot{},
	basesize=50x100,
	unitsize=1,
	greenhiderace={\commonorc{} \textnormal{\only{\eadbasherrider}}},
	weapons={\lance},
	armour={\la{},\mountsprotection{5}},
	specialrules={\thunderouscharge{} \only{\warboar},\impacthits{+1}},
	options={
		\maytakeha{}=15,
	},
}

\showunit{
	name={\goblinwolfchariot},
	QRSname={\goblinwolfchariot{}$^{4}$},
	cost={60},
	profile={%
		\chariot{}< - - - 5 4 4 - - -,
		\goblin{} (3)< - 2 3 3 - - 2 1 6,
		\wolf{} (2)< 9 3 - 3 - - 3 1 3,
	},
	type=\chariot{},
	basesize=50x100,
	unitsize=1,
	maxmodels=4,
	costpermodel=60,
	greenhiderace={\commongoblin{} \textnormal{\only{\goblin}}},
	weapons={\shortbow{},\lightlance{}},
	armour={\la{},\mountsprotection{6}},
	specialrules={\lighttroops{},\insignificant{},\impacthits{+1}},
}

\showunit{
	name={\gnasherdashers},
	cost={60},
	profile={
		\rider{}< - 2 3 3 3 1 3 1 5,
		\gnasher{}< 5 4 - 5 3 1 4 2 5,
	},
	type=\cavalry{},
	basesize=20x20,
	unitsize=5,
	maxmodels=10,
	costpermodel=10,
	greenhiderace={\cavegoblin{} \textnormal{\only{\rider}}},
	armour={\la{},\mountsprotection{6}},
	specialrules={\impacthits{1},\immunetopsychology{},\fly{6},\skirmishers{},\oiitbites{},\rowsofteeth{}},
	unitrules={%
		\unitrule{\oiitbites}{\oiitbitesrule}
		\unitrule{\rowsofteeth}{\rowsofteethrule}
	},
}

\showunit{
	name={\gnasherherd},
	cost={80},
	profile={ < 5 4 - 5 3 1 4 2 5},
	type=\warbeast{},
	basesize=20x20,
	unitsize=10,
	maxmodels=40,
	costpermodel=9,
	specialrules={\immunetopsychology{},\insignificant{},\oiitbites{},\theyreeverywhere{}},
	unitrules={%
		\unitrule{\oiitbites}{\oiitbitesrule}
		\unitrule{\theyreeverywhere}{\theyreeverywhererule}
	},
}

\showunit{
	name={\greenhidecatapults},
	QRSname={\catapults}, %Too long in french
	cost={90},
	profile={%
		\machine{}< - - - - 7 3 - - -,
		\commongoblin{} (3)< 4 2 3 3 3 - 2 1 6,
		[\commonorc{} (1)]< 4 3 3 3 4 +1 2 1 7,
	},
	type=\warmachine{},
	basesize=75,
	unitsize=1,
	specialrules={\insignificant},
	options={\maytakeanorcoverseer{}=15,},
	unitrules={%
		\unitrule{\orcoverseer}{\orcoverseerrule}
	},
	additional={%
		\begin{center}\musttakeoneofthefollowingNOC{}\end{center}
		\setlength{\columnseprule}{1pt}
		\renewcommand{\columnseprulecolor}{\color{black!30}}
		\begin{multicols}{2}\raggedcolumns
		
			\begin{center}\largerfontsize\antiquefont\splatterer{} (0-2)\refsymbol{}\end{center}
			
			\noindent\splattererrule{}
		\vspace*{\fill}
		\columnbreak
		
			\begin{center}\largerfontsize\antiquefont\gitlauncher{} (0-2)\refsymbol{}\end{center}
			
			\noindent\gitlauncherrule{}	
		\vspace*{\fill}	
		\end{multicols}
		\setlength{\columnseprule}{0pt}
		
		\vspace*{0.2cm}\noindent\refsymbol{} \greenhidecatapultsnote{}
	},
}

\showunit{
	name={\grotlings},
	cost={40},
	profile={< 4 2 3 2 2 5 2 5 4},
	type=\swarm{},
	basesize=40x40,
	unitsize=2,
	maxmodels=6,
	costpermodel=10,
	specialrules={\insignificant{},\scouts{}},
	weapons={\throwingweapons},
}

\showunit{
	name={\scrapwagon},
	cost={45},
	profile={%
		\wagon{}< \starsymbol{} - - 4 4 4 - - -,
		\grotlings{} < - 2 3 2 - - 2 5 4,
	},
	type=\chariot{},
	basesize=50x100,
	unitsize=1,
	specialrules={\shambolic{3D6},\impacthits{2D6},\insignificant{},\unstable{}},
	weapons={\throwingweapons},
	armour={\mountsprotection{6}},
	options={\anyofthefollowing{
			\pointedsticks{}=10,
			\pursuitmode{}=10,
			\smellslikegreenspirit{}=10,
			\smasher{}=15,
			},
	},
	unitrules={
		\unitrule{\pointedsticks}{\pointedsticksrule}
		\unitrule{\pursuitmode}{\pursuitmoderule}
		\unitrule{\smellslikegreenspirit}{\smellslikegreenspiritrule}
		\unitrule{\smasher}{\smasherrule}
	},
}

\showunit{
	name={\trolls},
	cost={55},
	profile={< 6 3 2 5 4 3 1 3 4},
	type=\monstrousinfantry{},
	basesize=40x40,
	unitsize=1,
	maxmodels=10,
	costpermodel=38,
	specialrules={\fear{},\stupidity{},\regeneration{4},\trollbelch{}},
	additional={%
		\begin{center}\musttakeoneofthefollowingNOC{}\end{center}
		\setlength{\columnseprule}{1pt}
		\renewcommand{\columnseprulecolor}{\color{black!30}}
		\begin{multicols}{3}\raggedcolumns
		
			\begin{center}\largerfontsize\antiquefont\commontrolls{} (\free{})\end{center}

			\begin{center}-\end{center}

		\vspace*{\fill}
		\columnbreak
			\begin{center}\largerfontsize\antiquefont\cavetrolls{} (\pts{8}\permodel)\end{center}

			\vspace{-0.3cm}{\setlength{\parskip}{0.3cm}
			\def\temparmour{\innatedefence{4}}
			\armour{\temparmour}

			\def\tempspecialrules{\magicresistance{3}}
			\specialrules{\tempspecialrules}
			}

		\vspace*{\fill}
		\columnbreak
			\begin{center}\largerfontsize\antiquefont\bridgetrolls{} (\pts{8}\permodel)\end{center}

			\def\tempspecialrules{\distracting{},\strider{\water}}
			\specialrules{\tempspecialrules}

		\vspace*{\fill}
		\end{multicols}
		\setlength{\columnseprule}{0pt}
		
		\def\tempunitrules{\unitrule{\trollbelch}{\trollbelchrule}}
		\unitrules{\tempunitrules}
	}
}

\showunit{
	name={\giant},
	cost=140,
	profile={< 6 3 - 6 5 6 3 \starsymbol{} 10,},
	type=\monster,
	basesize=50x75,
	unitsize=1,
	specialrules={\immunetopsychology{},\stubborn{},\giantattacks{}},
	options={\maygain{} \wardsave{6}=10,},
	additional={%
		\def\tempunitrules{%
			\unitrule{\giantattacks}{\giantattacksrule}
			\unitrule{\bellow}{\bellowrule}
			\unitrule{\jump}{\jumprule}
			\unitrule{\grab}{\grabrule}
			\unitrule{\swing}{\swingrule}
			\unitrule{\thump}{\thumprule}
			\unitrule{\smashname}{\smashrule}
		}
		\vspace*{0.1cm}\unitrules{\tempunitrules}
		\giantattackstable
	},
}

\rareunitstitle

\showunit{
	name={\skewerer},
	cost={45},
	profile={%
		\machine{}< - - - - 7 3 - - -,
		\commongoblin{} (3)< 4 2 3 3 3 - 2 1 6,
	},
	type=\warmachine{},
	basesize=60,
	unitsize=1,
	specialrules={\insignificant},
	weapons={\ballista},
	unitequipment={%
		\equipmentdef{\ballista}{\ballistarule}
	},
}

\showunit{
	name={\gnasherwreckingteam},
	cost={70},
	profile={< \starsymbol{} - - 6 4 3 3 2 3},
	type=\monstrousbeast{},
	basesize=60,
	unitsize=1,
	specialrules={\shambolic{3D6},\ricochet{2D6},\hardtarget{},\lookatemgo{}},
	unitrules={%
		\unitrule{\lookatemgo}{\lookatemgorule}
	},
}

\showunit{
	name={\gargantula},
	cost={225},
	profile={
		\spider{}< 7 4 - 5 6 8 4 8 -,
		\forestgoblin{} (8)< - 2 3 3 - - 2 1 6,
	},
	type=\riddenmonster{},
	basesize=100x150,
	unitsize=1,
	greenhiderace={\forestgoblin{} \textnormal{\only{\goblins}}},
	weapons={\shortbow{},\lightlance{} \only{\goblins}},
	armour={\innatedefence{4}},
	specialrules={\venomousfangs{},\immunetopsychology{},\poisonedattacks{} \only{\spider},\strider{},\stubborn{},\swiftstride{}},
	options={\maytakeweblauncher{}=30},
	unitequipment={%
		\equipmentdef{\weblauncher{} \textnormal{\only{\spider}}}{\weblauncherrule}
	},
}

\showunit{
	name={\greatgreenidol},
	cost={230},
	profile={< 6 2 - 6 8 6 2 3 8},
	type=\monster{},
	basesize=100x100,
	unitsize=1,
	armour={\innatedefence{5}},
	specialrules={\immunetopsychology{},\crushattack{},\impacthits{1D3},\smashemflat{},\wevegotthegreenlight{}},
	options={\bsboption{}=50},
	unitrules={%
		\unitrule{\wevegotthegreenlight}{\wevegotthegreenlightrule}
		\unitrule{\smashemflat}{\smashemflatrule}
	},
}

\mountstitle

\mountssectionannouncement

\showunit{
	name={\wyvern},
	profile={< 4 5 - 6 5 4 3 3 6},
	type=\monstrousbeast{},
	basesize=50x50,
	specialrules={\fear{},\fly{8},\largetarget{},\poisonedattacks{},\venomousfangs{}},
}

\showunit{
	name={\warboar},
	profile={< 7 3 - 3 3 1 3 1 3},
	type=\warbeast{},
	basesize=25x50,
	armour={\mountsprotection{5}},
	specialrules={\thunderouscharge},
}

\showunit{
	notinQRS=yes,
	name={\orcboarchariot},
	profile={%
		\chariot{}< - - - 5 5 4 - - -,
		\eadbasherrider{} (1)< - 4 3 4 - - 2 1 7,
		\warboar{} (2)< 7 3 - 3 - - 3 1 3,
	},
	type=\chariot{},
	basesize=50x100,
	greenhiderace={\commonorc{} \textnormal{\only{\eadbasherrider}}},
	weapons={\lance},
	armour={\la{},\mountsprotection{5}},
	specialrules={\thunderouscharge{} \only{\warboar},\impacthits{+1}},
}

\showunit{
	name={\wolf},
	profile={< 9 3 - 3 3 1 3 1 3},
	type=\warbeast{},
	basesize=25x50,
	armour={\mountsprotection{6}},
	specialrules={\fastcavalry},
}

\showunit{
	notinQRS=yes,
	name={\goblinwolfchariot},
	profile={%
		\chariot{}< - - - 5 4 4 - - -,
		\goblin{} (2)< - 2 3 3 - - 2 1 6,
		\wolf{} (2)< 9 3 - 3 - - 3 1 3,
	},
	type=\chariot{},
	basesize=50x100,
	greenhiderace={\commongoblin{} \textnormal{\only{\goblin}}},
	weapons={\shortbow{},\lightlance{}},
	armour={\la{},\mountsprotection{6}},
	specialrules={\lighttroops{},\insignificant{},\impacthits{+1}},
}

\showunit{
	name={\cavegnasher},
	profile={< 5 4 - 6 4 3 3 3 3},
	type=\monstrousbeast{},
	basesize=40x40,
	armour={\mountsprotection{6}},
	specialrules={\impacthits{1},\immunetopsychology{},\fly{6},\hardtarget{},\oiitbites{},\bouncers{}},
	unitrules={%
		\unitrule{\oiitbites}{\oiitbitesrule}
		\unitrule{\bouncers}{\bouncersrule}
	}
}

\showunit{
	name={\scuttlerspider},
	profile={< 7 3 - 3 3 1 4 1 2},
	type=\warbeast{},
	basesize=25x50,
	armour={\mountsprotection{6}},
	specialrules={\fastcavalry{},\poisonedattacks{},\scout{},\strider{}},
}

\showunit{
	name={\huntsmenspider},
	profile={< 7 3 - 4 4 3 4 3 7},
	type=\monstrousbeast{},
	basesize=50x50,
	armour={\mountsprotection{5}},
	specialrules={\poisonedattacks{},\strider{}},
}

\showunit{
	notinQRS=yes,
	name={\gargantula{} (\oneofakind{})},
	profile={
		\spider{}< 7 4 - 5 6 8 4 8 -,
		\forestgoblin{} (8)< - 2 3 3 - - 2 1 6,
	},
	type=\riddenmonster{},
	basesize=100x150,
	greenhiderace={\forestgoblin{} \textnormal{\only{\goblins}}},
	weapons={\shortbow{},\lightlance{} \only{\goblins}},
	armour={\innatedefence{4}},
	specialrules={\venomousfangs{},\immunetopsychology{},\poisonedattacks{} \only{\spider},\strider{},\stubborn{},\swiftstride{}},
	options={\spidermothershrineoption{}=40},
	unitrules={%
		\unitrule{\spidermothershrine}{\spidermothershrinerule}
	},
}





%%% Quick Reference Sheet - AB_qrs.tex is automatic and shouldn't be edited %%%

\quickrefsheettitle

\input{../Layout/AB_qrs.tex}
\bigskip
\begin{center}
\noindent{\antiquefont\Largefontsize\textbf{Armes de Tir des Peaux Vertes}}
\medskip

\rowcolors{1}{white}{black!10}
\noindent\begin{tabular}{lcccccc}
\textbf{Nom} & \textbf{Artillerie} & \textbf{Portée} & \textbf{\labels@S{}} & \textbf{\multipleshots{}} & \textbf{\multiplewounds{}} & \textbf{\armourpiercing{}} \tabularnewline
\skewerer{} & \boltthrower{} & \distance{48} & 6 & - & 1D3 & 6 \tabularnewline
\splatterer{} & \catapult{} (\distance{3}) & \distance{12-60} & 3[9] & - & [\ordnance{}] & - \tabularnewline
\gitlauncher{} & \catapult{} (\distance{1}) & \distance{12-60} & 5 & 1D3+1 touches & - & 2 \tabularnewline
\weblauncher{} & \catapult{} (\distance{3}) & \distance{6-36} & 3 & - & - & - \tabularnewline
\end{tabular}
\end{center}

\restoregeometry

\changelogtitle

\startchangelog

\newlog{0.99.2}{%
Waaargh clarification,
Plates of Tuktek clarification,
Crown of the Cavern King clarification,
Lucky Boar’s Leg clarification,
Mad Git clarification,
Unruly clarification,
Smash ‘Em Flat clarification,
We’ve Got The Green Light clarification,
Venomous Fangs clarification,
Git Launcher clarification,
Troll Belch clarification,
Cave Gnasher Immune to Psychology had been removed,
Web Launcher clarified which part has what weapon,
Goblin Wolf Chariot clarified which part has what weapon,
Giant attacks clarification,
}

\endchangelog

\end{document}

