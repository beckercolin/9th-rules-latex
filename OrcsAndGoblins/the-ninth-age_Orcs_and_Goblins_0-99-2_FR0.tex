

\documentclass[a4paper,8pt]{extarticle} % extarticle allows to use font size of 8pt.

\usepackage[a4paper, top=1.6cm, bottom=2cm, left=1.6cm, right=1.6cm]{geometry} % Marge reduction.

%% Language specific package
\usepackage[french]{babel}
\frenchbsetup{StandardLists=true} % Necessary to use enumitem with babel/french.

%% Font and typing packages
\usepackage{fontspec}
\setmainfont[
	Ligatures=TeX,
	ItalicFont={Dancing Script},
	BoldItalicFont={Dancing Script}
	]{PT Serif} % default is Latin Modern
\newfontfamily\antiquefont[Ligatures=TeX]{Caslon Antique} % fancy font
\usepackage{microtype}			% Greatly improves general appearance of the text.
\usepackage{SIunits}			% Unit appearance.
\usepackage{xspace}				% Define commands that appear not to eat spaces.
\usepackage{ulem}				% To cross words out. Use \sout{}.

%% Array utilities
\usepackage{array}				% Additionnal options for arrays.
\usepackage{colortbl}			% Additionnal options for coloring arrays.
\usepackage[table]{xcolor}		% Auto alternate grey-white rows.
\usepackage[export]{adjustbox}		% Centered pics in tables

%% List utilities
\usepackage[inline]{enumitem}   % Display inline lists.
\usepackage{etoolbox}           % General utility. Good for lists for instance.
\usepackage{xparse}             % List utilities.
\usepackage{datatool}	% Handling alphabetical order.

%% Frames
\usepackage{framed}				% Boxes.
\usepackage[framemethod=TikZ]{mdframed}% For fancy frames.
\usepackage{tikz}				% For fancy frames.
\usepackage{wrapfig}			% Fancy insertion of pics in text.

%% Page utilities
\usepackage{multicol}			% Allows to divide a part of the page in multiple columns.
	
%% Others
\usepackage{keyval}             % Used to create maps of commands/labels/objects.
	\makeatletter                  % Mandatory for the usage of keyval.
\usepackage{xstring}            % String parsing, cutting, etc.
\usepackage{hyperref} % Links in PDF.


%%% Update of the dotfill command to always get dots

\newcommand{\predotfill}{\penalty0\hbox{}\nobreak}%


%%% Command to avoid typing \xspace when creating a new name macro

\newcommand{\newnamemacro}[2]{\newcommand{#1}{#2}} % \xspace removed for compatibility with alphabetical ordering

%%% Language specific stuff


%%% Commands %%%

\newcommand{\addtosortedlist}[1]{%
	\protected@edef\textarg{#1}%
	\protected@edef\textwithoutspaces{\expandafter\removespaces\expandafter{\textarg}}%
	\substitute\textwithoutspaces{É}{e}% Most used special characters of the language, and equivalent for alphabetical ordering
	\substitute\textwithoutspaces{È}{e}%
	\substitute\textwithoutspaces{Ê}{e}%
	\substitute\textwithoutspaces{é}{e}%
	\substitute\textwithoutspaces{è}{e}%
	\substitute\textwithoutspaces{ê}{e}%
	\substitute\textwithoutspaces{À}{a}%
	\substitute\textwithoutspaces{à}{a}%
	\substitute\textwithoutspaces{ù}{u}%
	\expandafter\sortitem\expandafter[\textwithoutspaces]{#1}%
}%


%%% Labels %%%

% Profile

\newcommand{\labels@M}{M}
\newcommand{\labels@WS}{CC}
\newcommand{\labels@BS}{CT}
\newcommand{\labels@S}{F}
\newcommand{\labels@T}{E}
\newcommand{\labels@W}{PV}
\newcommand{\labels@I}{I}
\newcommand{\labels@A}{A}
\newcommand{\labels@Ld}{Cd}
\newcommand{\labels@Invocation}{Invocation} % For Vampire Covenant profiles

\newcommand{\Strength}{Force}

% Technical

\newcommand{\labels@range}{Portée}
\newcommand{\labels@point}{pt}
\newcommand{\labels@points}{pts}
\newcommand{\labels@only}{uniquement}
\newcommand{\labels@magic}{Magie}
\newcommand{\labels@pathsused}{Génère ses sorts dans la Discipline}
\newcommand{\labels@model}{figurine}
\newcommand{\labels@models}{figurines}
\newcommand{\labels@Singlemodel}{Figurine \textbf{seule}}

% Unit entry labels

\newcommand{\labels@basesize}{Socle}
\newcommand{\labels@trooptype}{Type de troupe}
\newcommand{\labels@specialrules}{Règles spéciales}
\newcommand{\labels@alignment}{Allégeance}
\newcommand{\labels@equipment}{Équipement}
\newcommand{\labels@weapons}{Armes}
\newcommand{\labels@armour}{Armure}
\newcommand{\labels@options}{Options}
\newcommand{\labels@commandgroup}{État-Major}
\newcommand{\labels@mounts}{Montures}
\newcommand{\labels@specialequipment}{Équipement spécial}

% Command groups

\newcommand{\labels@champion}{Champion}
\newcommand{\labels@standardbearer}{Porte-étendard}
\newcommand{\labels@musician}{Musicien}
\newcommand{\labels@singlebannerallowance}{Une seule unité de ce type peut prendre une Bannière magique}
\newcommand{\labels@condsinglebannerallowance}{Une seule unité de ce type peut prendre une Bannière magique si}
\newcommand{\labels@bannerallowance}{Peut prendre une Bannière Magique}
\newcommand{\labels@veteranstandardbearer}{Peut devenir Porte-étendard Vétéran}
\newcommand{\labels@championallowance}{Peut prendre une Arme Magique}

% Titles

\newcommand{\labels@lords}{Seigneurs}
\newcommand{\labels@heroes}{Héros}
\newcommand{\labels@coreunits}{Unités de base}
\newcommand{\labels@specialunits}{Unités spéciales}
\newcommand{\labels@rareunits}{Unités rares}
\newcommand{\labels@armywiderules}{Règles communes de l'armée}
\newcommand{\labels@armyspecialrules}{Règles spéciales de l'armée}
\newcommand{\labels@armoury}{Armurerie}
\newcommand{\labels@magicalitems}{Objets magiques}
\newcommand{\labels@magicalweapons}{Armes magiques}
\newcommand{\labels@magicalarmour}{Armures magiques}
\newcommand{\labels@talismans}{Talismans}
\newcommand{\labels@enchanteditems}{Objets enchantés}
\newcommand{\labels@arcaneitems}{Objets cabalistiques}
\newcommand{\labels@magicalbanners}{Bannières magiques}
\newcommand{\labels@quickrefsheet}{Fiche de référence}
\newcommand{\labels@changelog}{Change Log}

\newcommand{\labels@lordsInitial}{S}
\newcommand{\labels@heroesInitial}{H}
\newcommand{\labels@coreunitsInitial}{B}
\newcommand{\labels@specialunitsInitial}{S}
\newcommand{\labels@rareunitsInitial}{R}
\newcommand{\labels@mountsInitial}{M}


% Titlepage

\newcommand{\labels@fantasybattles}{Batailles Fantastiques}
\newcommand{\labels@NinthAge}{Le 9\ieme Âge}
\newcommand{\labels@creators}{Une collaboration des créateurs de l'ETC et du Swedish Comp System}
\newcommand{\labels@introduction}{%
\noindent {\Largerfontsize\textbf{Note des traducteurs}}
\vspace{0.5cm}

Nous souhaitons remercier chaleureusement l'équipe à l'initiative du 9\ieme Âge pour leur motivation et leur travail continu pour faire vivre notre passion. Nous espérons que ce jeu saura développer les qualités pour plaire au plus grand nombre et réunir les joueurs, amateurs comme habitués des tournois, autour de règles amusantes et équilibrées, pour finalement s'imposer comme un standard du jeu de figurines. Une grande ambition qui ne pourra s'accomplir que \textbf{grâce à vous}, la communauté, via des retours constructifs, afin de modeler le jeu selon nos désirs. N'étant \textbf{en aucun cas à but lucratif}, le 9\ieme Âge part avec un avantage considérable. Les règles des éventuelles nouvelles sorties ne seront pas dictées par le besoin de vendre ces nouveautés. Vous pouvez choisir et acheter vos figurines où bon vous semble, il n'y a pas un unique revendeur toléré. Vous n'êtes pas bloqués dans une spirale infernale où pour continuer à jouer à un jeu, dans lequel vous vous êtes tant investis, vous devez payer toujours plus cher pour entretenir votre collection. Enfin, vous pouvez être assurés que tant que 9\ieme Âge sera joué, vous disposerez d'un \textbf{support continu et régulier}, celui-ci étant offert par la communauté.

Nous attirons votre attention sur le fait que ce jeu en est encore à ses débuts et dans un \textbf{stade de développement}. Ce document correspond à une version de brouillon \textbf{\og{} beta \fg{}}, dont le but et de tester le jeu et le modifier jusqu'à atteindre une version satisfaisante. Attendez-vous donc à trouver des déséquilibres, des incohérences, et à obtenir des mises à jour régulières avec éventuellement des changements importants. N'hésitez pas à nous donner vos avis ! Ce livre d'armée n'est utilisable qu'en compagnie du livre de Règles et du livre de Magie.

Concernant la traduction en elle-même, nous avons fait de notre mieux pour vous offrir une version de qualité, dont nous espérons qu'elle surpasse celle de la version originale ! Si vous constatez des coquilles, des erreurs, merci de nous les signaler en nous contactant sur le forum du 9\ieme Âge, dans le \textbf{sous-forum français} (\url{http://www.the-ninth-age.com/index.php?board/117-french/}). Vous y trouverez aussi les dernières mises à jour. \textbf{En cas de conflit d'interprétation avec la version originale, la version originale fait référence}.

\vspace{0.5cm}
Que ce jeu vous apporte d'innombrables heures de plaisir partagé !

\vspace{0.7cm}
\noindent {\Largerfontsize\textbf{Les traducteurs}}
\vspace{0.1cm}

\ifdef{\translationteam}{
	\begin{multicols}{3}
	\begin{itemize}
		\translationteam
	\end{itemize}
	\end{multicols}
}{}
}
\newcommand{\labels@secondpageannouncement}{%
	\labels@fantasybattles{} : \labels@NinthAge{} est un jeu créé et entretenu par la communauté qui met en scène des affrontements de figurines. Toutes les règles sont disponibles gratuitement sur le site suivant. Vos retours et suggestions sont les bienvenus.
	\newline\url{http://www.the-ninth-age.com/}
}
\newcommand{\labels@rulechanges}{%
	Les changements de règles entre versions sont colorés comme ce paragraphe. Une liste en anglais de ces changements par version est ajoutée à la fin de cet ouvrage.
}
\newcommand{\labels@latexcredit}{Document réalisé à l'aide de \LaTeX .}


%%% Technical commands

\newcommand{\only}[1]{(#1 uniquement)}
\newcommand{\free}{gratuit}
\newcommand{\upto}{jusqu'à}
\newcommand{\Upto}{Jusqu'à}
\newcommand{\unlimited}{sans limite de pts}
\newcommand{\permodel}{/fig.}
\newcommand{\listlastchoice}{ ou}
\newcommand{\notif}[1]{(pas #1)}
\newcommand{\wordand}{et}
\newcommand{\wordwith}{avec}
\newcommand{\ifNmodelsorless}[1]{(#1 figurines ou moins)}
\newcommand{\unitwith}{unité avec}
\newcommand{\From}{De} % From ... to ... models
\newcommand{\wordto}{à}
\newcommand{\wordAll}{Tous}
\newcommand{\spacebeforecolon}{ } % French put a space before colons
\newcommand{\minprice}{Coût min. :}
\newcommand{\mincostfor}{Coût min. pour}
\newcommand{\maxunitsize}{Taille max.}
\newcommand{\additionalfigscost}{Les figurines additionnelles coûtent}


%%% Special rules %%%

\newcommand{\ambush}{Embuscade}
\newcommand{\armourpiercing}[1]{Perforant\ifblank{#1}{}{ (#1)}}
\newcommand{\bodyguard}[1]{Garde du Corps\ifblank{#1}{}{ (#1)}}
\newcommand{\breathweapon}[1]{Attaque de Souffle\ifblank{#1}{}{ (#1)}}
\newcommand{\channel}{Canalisation}
\newcommand{\crushattack}{Attaque Écrasante}
\newcommand{\devastatingcharge}{Charge Dévastatrice}
\newcommand{\distracting}{Distrayant}
\newcommand{\engineer}{Ingénieur}
\newcommand{\ethereal}{Éthéré}
\newcommand{\fastcavalry}{Cavalerie Légère}
\newcommand{\fear}{Peur}
\newcommand{\fightinextrarank}{Combat avec un Rang Supplémentaire}
\newcommand{\fireborn}{Né du Feu}
\newcommand{\flamingattacks}{Attaques Enflammées}
\newcommand{\flammable}{Inflammable}
\newcommand{\lighttroops}{Troupes Légères}
\newcommand{\frenzy}{Frénésie}
\newcommand{\fly}[1]{Vol\ifblank{#1}{}{ (#1)}}
\newcommand{\grindingattacks}[1]{Attaques de Broyage\ifblank{#1}{}{ (#1)}}
\newcommand{\hardtarget}{Camouflé}
\newcommand{\hatred}{Haine}
\newcommand{\hellfire}{Flammes de l'Enfer}
\newcommand{\hidden}{Caché}
\newcommand{\holyattacks}{Attaques Divines}
\newcommand{\immunetopsychology}{Immunisé à la Psychologie}
\newcommand{\impacthits}[1]{Touches d'Impact\ifblank{#1}{}{ (#1)}}
\newcommand{\insignificant}{Insignifiant}
\newcommand{\largetarget}{Grande Cible}
\newcommand{\lethalstrike}{Coup Fatal}
\newcommand{\lightningattacks}{Attaques Foudroyantes}
\newcommand{\lightningreflexes}{Réflexes Foudroyants}
\newcommand{\magicresistance}[1]{Résistance à la Magie\ifblank{#1}{}{ (#1)}}
\newcommand{\magicalattacks}{Attaques Magiques}
\newcommand{\metalshifting}{Fusion du Métal}
\newcommand{\moveorfire}{Mouvement ou Tir}
\newcommand{\multipleshots}[1]{Tirs Multiples\ifblank{#1}{}{ (#1)}}
\newcommand{\multiplewounds}[2]{Blessures Multiples\ifblank{#1}{}{ (#1\ifblank{#2}{)}{, #2)}}}
\newcommand{\notaleader}{Pas un Meneur}
\newcommand{\otherworldly}{D'Outre-Monde}
\newcommand{\pathmaster}[1]{Maître de la Discipline\ifblank{#1}{}{ (#1)}}
\newcommand{\poisonedattacks}{Attaques Empoisonnées}
\newcommand{\quicktofire}{Tir Rapide}
\newcommand{\randommovement}[1]{Mouvement Aléatoire\ifblank{#1}{}{ (#1)}}
\newcommand{\randomattacks}[1]{Attaques Aléatoires\ifblank{#1}{}{ (#1)}}
\newcommand{\regeneration}[1]{Régénération\ifblank{#1}{}{ (#1+)}}
\newcommand{\reload}{Rechargez !}
\newcommand{\requirestwohands}{Arme à deux Mains}
\newcommand{\scythes}{Faux}
\newcommand{\scout}{Éclaireur}
\newcommand{\scouts}{Éclaireurs}
\newcommand{\stomp}[1]{Piétinement\ifblank{#1}{}{ (#1)}}
\newcommand{\strider}[1]{Guide\ifblank{#1}{}{ (#1)}}
\newcommand{\stubborn}{Tenace}
\newcommand{\stupidity}{Stupidité}
\newcommand{\skirmisher}{Tirailleur}
\newcommand{\skirmishers}{Tirailleurs}
\newcommand{\sweepingattack}{Attaque au Passage}
\newcommand{\swiftstride}{Rapide}
\newcommand{\thunderouscharge}{Charge Tonitruante}
\newcommand{\terror}{Terreur}
\newcommand{\toxicattacks}{Attaques Toxiques}
\newcommand{\unbreakable}{Indémoralisable}
\newcommand{\undead}{Mort-Vivant}
\newcommand{\unstable}{Instable}
\newcommand{\unwieldy}{Encombrant}
\newcommand{\vanguard}{Avant-Garde}
\newcommand{\volleyfire}{Tir de Volée}
\newcommand{\warplatform}{Plateforme de Guerre}
\newcommand{\wardsave}[1]{Sauvegarde Invulnérable\ifblank{#1}{}{ (#1+)}}
\newcommand{\weaponmaster}{Maître d'Ar\-mes}
\newcommand{\wizardconclave}[1]{Conclave de Sorciers\ifblank{#1}{}{ (#1)}}


%%% Magic %%%

\newnamemacro{\Pathof}{Discipline}

\newcommand{\battle}{Commune}
\newcommand{\alchemy}{de l'Alchimie}
\newcommand{\death}{de la Mort}
\newcommand{\fire}{du Feu}
\newcommand{\heavens}{des Cieux}
\newcommand{\light}{de la Lumière}
\newcommand{\nature}{de la Nature}
\newcommand{\shadows}{des Ombres}
\newcommand{\wilderness}{de la Sauvagerie Bestiale}
\newcommand{\butchery}{de la Boucherie}
\newcommand{\change}{du Changement}
\newcommand{\thebiggreengods}{des Grands Dieux Verts}
\newcommand{\thelittlegreengods}{des Petits Dieux Verts}
\newcommand{\blackmagic}{de la Magie Noire}
\newcommand{\disease}{de la Maladie}
\newcommand{\lust}{de la Luxure}
\newcommand{\necromancy}{de la Nécromancie}
\newcommand{\ruin}{de la Ruine}
\newcommand{\forge}{de la Forge}
\newcommand{\sands}{des Sables}
\newcommand{\whitemagic}{de la Magie Blanche}

\newcommand{\anyofthebattlemagic}{dans n'importe laquelle des Disciplines Communes}

\newcommand{\magiclevel}[1]{\ifnumcomp{#1}{<}{3}{Sorcier Apprenti}{Maître Sorcier} Niveau #1}
\newcommand{\Level}{Niveau}

\newcommand{\wizard}{Sorcier}
\newcommand{\wizards}{Sorciers}

\newcommand{\boundspell}[1]{Objet de Sort, Puissance #1}


%%% Other rules %%%

\newcommand{\armoursave}{Sauvegarde d'Armure}
\newcommand{\firstinrank}{Au Premier Rang}
\newcommand{\hardcover}{Couvert Lourd}
\newcommand{\holdyourground}{Tenez les Rangs}
\newcommand{\inspiringpresence}{Présence Charismatique}
\newcommand{\lightcover}{Couvert Léger}
\newcommand{\monstrousrank}{Rang Monstrueux}
\newcommand{\ordnance}{Artillerie}
\newcommand{\parry}{Parade}
\newcommand{\raisewounds}{Ressusciter des Figurines}
\newcommand{\recoverwounds}{Récupérer des PVs}
\newcommand{\aideddispel}{Dissipation Assistée}
\newcommand{\rnf}{ordinaires}
\newcommand{\general}{Général}


%%% Equipment %%%

\newcommand{\innatedefence}[1]{Protection Innée\ifblank{#1}{}{~(#1+)}}
\newcommand{\mountsprotection}[1]{Protection de Monture\ifblank{#1}{}{~(#1+)}}
\newcommand{\la}{Armure Légère}
\newcommand{\ha}{Armure Lourde}
\newcommand{\platearmour}{Armure de Plates}
\newcommand{\hw}{Arme de Base}
\newcommand{\pw}{Paire d'Armes}
\newcommand{\spear}{Lance}
\newcommand{\halberd}{Hallebarde}
\newcommand{\gw}{Arme Lourde}
\newcommand{\lance}{Lance de Cavalerie}
\newcommand{\lightlance}{Lance Légère}
\newcommand{\shield}{Bouclier}
\newcommand{\barding}{Caparaçon}
\newcommand{\throwingweapons}{Armes de Jet}
\newcommand{\shortbow}{Arc Court}
\newcommand{\flail}{Fléau}

\newcommand{\cannon}{Canon}
\newcommand{\catapult}{Catapulte}
\newcommand{\volleygun}{Batterie de Tir}
\newcommand{\boltthrower}{Baliste}
\newcommand{\artilleryweapon}{Arme d'Artillerie}


%%% Troop types %%%

\newcommand{\characters}{Personnages}
\newcommand{\infantry}{Infanterie}
\newcommand{\monstrousinfantry}{Infanterie Monstrueuse}
\newcommand{\cavalry}{Cavalerie}
\newcommand{\monstrouscavalry}{Cavalerie Monstrueuse}
\newcommand{\swarm}{Nuée}
\newcommand{\swarms}{Nuées}
\newcommand{\warbeast}{Bête de Guerre}
\newcommand{\warbeasts}{Bêtes de Guerre}
\newcommand{\monster}{Monstre}
\newcommand{\monsters}{Monstres}
\newcommand{\monstrousbeast}{Bête Monstrueuse}
\newcommand{\monstrousbeasts}{Bêtes Monstrueuses}
\newcommand{\chariot}{Char}
\newcommand{\chariots}{Chars}
\newcommand{\riddenmonster}{Monstre Monté}
\newcommand{\riddenmonsters}{Monstres Montés}
\newcommand{\warmachine}{Machine de Guerre}
\newcommand{\warmachines}{Machines de Guerre}


%%% Terrain %%%

\newcommand{\water}{Eaux peu profondes}


%%% Profile wording

\newcommand{\oneofakind}{Uni\-que}
\newcommand{\onechoiceonly}{(un seul choix)}
\newcommand{\onfootonly}{(à pied seulement)}
\newcommand{\closecombatonly}{seulement au Corps à Corps}
\newcommand{\Xmodelsorless}[1]{(#1 figurines ou moins)}
\newcommand{\magicalitemsallowance}{Peut prendre des Objets Magiques}
\newcommand{\magicalweaponallowance}{Peut prendre une Arme Magique}
\newcommand{\notmagicalarmour}{(mais pas d'Armure Magique)}
\newcommand{\anyofthefollowing}{\optionschoice{Peut prendre :}}
\newcommand{\weapononechoice}{\optionschoice{Peut prendre une arme \onechoiceonly{} :}}
\newcommand{\weaponschoice}{\optionschoice{Peut prendre des armes :}}
\newcommand{\shootingweapononechoice}{\optionschoice{Peut prendre une arme de tir \onechoiceonly{} :}}
\newcommand{\combatweapononechoice}{\optionschoice{Peut prendre une arme de corps à corps \onechoiceonly{} :}}
\newcommand{\armouronechoice}{\optionschoice{Peut prendre une armure \onechoiceonly{} :}}
\newcommand{\magiclevelchoice}{\optionschoice{Peut devenir au choix :}}
\newcommand{\bsboption}{Peut devenir Porteur de la Grande Bannière}
\newcommand{\mayupgradeto}{Peut être amélioré en}
\newcommand{\mustbecomeoneofthefollowing}{\optionschoice{Doit devenir un choix parmi :}}
\newcommand{\maybecomeoneofthefollowing}{\optionschoice{Peut devenir un choix parmi :}}
\newcommand{\maytakeoneofthefollowing}{\optionschoice{Peut prendre un choix parmi :}}
\newcommand{\maytakeuptotwoofthefollowing}{\optionschoice{Peut prendre jusqu'à deux choix parmi :}}
\newcommand{\maygain}{Peut gagner la règle}
\newcommand{\maytake}{Peut prendre}
\newcommand{\maytakeashield}{Peut prendre un Bouclier}
\newcommand{\maytakela}{Peut prendre une Armure Légère}
\newcommand{\maytakeha}{Peut prendre une Armure Lourde}
\newcommand{\maytakemountsprotectionX}[1]{Peut prendre une \mountsprotection{#1}}
\newcommand{\maytakeagw}{Peut prendre une Arme Lourde}
\newcommand{\maytakeaspear}{Peut prendre une Lance}
\newcommand{\maytakepw}{Peut prendre une Paire d'Armes}
\newcommand{\maytakethrowingweapons}{Peut prendre des Armes de Jet}
\newcommand{\maytakebarding}{Peut prendre un Caparaçon}
\newcommand{\replaceshieldwithhalberd}{Remplacer le Bouclier par une Hallebarde}
\newcommand{\maybecome}{Peut devenir}

\newcommand{\maytakeonechoiceonly}{\optionschoice{\maytake{} \onechoiceonly{}\spacebeforecolon{}:}}

\newcommand{\mountssectionannouncement}{%
La section Montures concerne les montures de Personnages. Les montures pour non-Personnages suivent les règles données dans leur description d'unité.
}

%%% Commands to handle strings, better than xstring to handle commands inside the strings %%%

\newcommand{\substitute}[3]{%
  \protected@edef\sub@temp{#1}%
  \saveexpandmode
  \expandarg\StrSubstitute{\sub@temp}{#2}{#3}[#1]%
  \restoreexpandmode
}

\newcommand{\splitatstar}[3]{%
  \protected@edef\split@temp{#1}%
  \saveexpandmode
  \expandarg\StrCut{\split@temp}{*}#2#3%
  \restoreexpandmode
}

\newcommand{\splitatinf}[3]{%
  \protected@edef\split@temp{#1}%
  \saveexpandmode
  \expandarg\StrCut{\split@temp}{<}#2#3%
  \restoreexpandmode
}

\newcommand{\splitatequal}[3]{%
  \protected@edef\split@temp{#1}%
  \saveexpandmode
  \expandarg\StrCut{\split@temp}{=}#2#3%
  \restoreexpandmode
}

\newcommand{\ifsubstring}[4]{%
  \protected@edef\split@temp{#1}%
  \protected@edef\split@tempbis{#2}%
  \saveexpandmode
  \expandarg\IfSubStr{\split@temp}{\split@tempbis}{#3}{#4}%
  \restoreexpandmode
}

\def\removespaces#1{\zap@space#1 \@empty}

%%% Commands for alphabetical ordering %%%

\newcommand{\sortitem}[2][\relax]{%
	\DTLnewrow{list}% Create a new entry
	\ifx#1\relax%
		\DTLnewdbentry{list}{sortlabel}{#2}% Add entry sortlabel (no optional argument)
	\else%
		\DTLnewdbentry{list}{sortlabel}{#1}% Add entry sortlabel (optional argument)
	\fi%
		\DTLnewdbentry{list}{description}{#2}% Add entry description
}
\newenvironment{sortedlist}{%
	\DTLifdbexists{list}{\DTLcleardb{list}}{\DTLnewdb{list}}% Create new/discard old list
}{%
	\DTLsort{sortlabel}{list}% Sort list
	\begin{itemize*}[label={}, itemjoin={,}]%
		\DTLforeach*{list}{\theDesc=description}{%
		\item\theDesc}% Print each item
	\end{itemize*}%
}

\pdfstringdefDisableCommands{\def\textcolor#1{}}

% See language specific file for \addtosortedlist

%%% Database for automatic Quick Ref Sheet %%%

\DTLnewdb{profiles} % Database containing name, category, multiprofile number, profilename (if multi), caraclist, trooptype, invocation for CV.
\newcommand{\profilecategory}{\labels@lords} % Will be updated in relevant categories

\newcommand{\profiledtbfillname}[1]{\DTLnewdbentry{profiles}{name}{#1}}
\newcommand{\profiledtbfillcategory}[1]{\DTLnewdbentry{profiles}{category}{#1}}
\newcommand{\profiledtbfilltrooptype}[1]{\DTLnewdbentry{profiles}{trooptype}{#1}}
\newcommand{\profiledtbfillinvocation}[1]{\DTLnewdbentry{profiles}{invocation}{#1}}
\newcommand{\profiledtbfillprofile}[1]{\DTLnewdbentry{profiles}{profile}{#1}}
\newcommand{\profiledtbfillmultipleprofile}[1]{\DTLnewdbentry{profiles}{multipleprofile}{#1}}

\newcommand{\void}[1]{}
\newcounter{multiprofilecounter}

\newcommand{\profiledtbfillcarac}[1]{%
	\profiledtbfillprofile{#1}
	\parselist{#1}{\locallists@profileslist}% Split of the different profiles in the case of a multiprofile.
	\setcounter{multiprofilecounter}{0}%
	\forlistloop{\stepcounter{multiprofilecounter}\void}{\locallists@profileslist}%
	\expandafter\profiledtbfillmultipleprofile\expandafter{\number\value{multiprofilecounter}}
}


%%% Technical commands %%%

\newcommand{\newrule}{\textcolor{green!50!black}}
\newcommand{\removedrule}[1]{\textcolor{green!50!black}{\sout{#1}}}
\newcommand{\starsymbol}{$\star$}
\newcommand{\refsymbol}{$^\star$}

\newcommand{\inch}{\arcsecond}
\newcommand{\foot}{\arcminute}
\newcommand{\range}[1] {\labels@range~\unit{#1}{\inch}}
\newcommand{\distance}[1] {\unit{#1}{\inch}}
\newcommand{\result}[1] {\texttt{'}#1\texttt{'}}


%%% Fonts and sizes %%%

\newcommand{\bigtitle}[1]{\vspace*{-1.5cm}\section*{}\noindent\begin{center}\Hugefontsize\textbf{\antiquefont\expandafter\uppercase\expandafter{#1}}\end{center}}

\newcommand{\subtitle}[1]{\subsection*{}\noindent{\hugefontsize\antiquefont #1}}

\newcommand{\subsubtitle}[1]{\subsubsection*{}\noindent{\Largerfontsize\antiquefont #1}}

\newcommand{\verysmallfontsize}{\fontsize{4}{4.8}\selectfont}
\newcommand{\smallfontsize}{\fontsize{6}{7.2}\selectfont}
\newcommand{\normalfontsize}{\fontsize{8}{9.6}\selectfont}
\newcommand{\largefontsize}{\fontsize{10}{12}\selectfont}
\newcommand{\largerfontsize}{\fontsize{12}{14.4}\selectfont}
\newcommand{\Largefontsize}{\fontsize{14}{16.8}\selectfont}
\newcommand{\Largerfontsize}{\fontsize{15}{18}\selectfont}
\newcommand{\hugefontsize}{\fontsize{18}{21.6}\selectfont}
\newcommand{\Hugefontsize}{\fontsize{25}{30}\selectfont}

\newcommand{\unitentryformat}[1]{\textit{\largefontsize{#1}}}
\newcommand{\textIT}[1]{\textit{\largefontsize{#1}}}


%%% Titles %%%

\newcommand{\lordstitle}{\def\logolocalpath{../Layout/pics/logo_lord.png}\bigtitle{\labels@lords}}
\newcommand{\heroestitle}{%
\def\logolocalpath{../Layout/pics/logo_hero.png}%
\clearpage\bigtitle{\labels@heroes}%
\renewcommand{\profilecategory}{\labels@heroes}%
}
\newcommand{\coreunitstitle}{%
\def\logolocalpath{../Layout/pics/logo_core.png}%
\clearpage\bigtitle{\labels@coreunits}%
\renewcommand{\profilecategory}{\labels@coreunits}%
}
\newcommand{\specialunitstitle}{%
\def\logolocalpath{../Layout/pics/logo_special.png}%
\clearpage\bigtitle{\labels@specialunits}%
\renewcommand{\profilecategory}{\labels@specialunits}%
}
\newcommand{\rareunitstitle}{%
\def\logolocalpath{../Layout/pics/logo_rare.png}%
\clearpage\bigtitle{\labels@rareunits}%
\renewcommand{\profilecategory}{\labels@rareunits}%
}
\newcommand{\mountstitle}{%
\def\logolocalpath{../Layout/pics/logo_mount.png}%
\clearpage\bigtitle{\labels@charactermounts}%
\renewcommand{\profilecategory}{\labels@mounts}%
}

\newcommand{\startarmywiderules}{\newpage\bigtitle{\labels@armywiderules}\largefontsize}
\newcommand{\closearmywiderules}{\normalfontsize}
\newcommand{\armywideruleentry}[1]{\subtitle{#1}\vspace{5pt}}

\newcommand{\startarmyspecialrules}{\bigtitle{\labels@armyspecialrules}\largefontsize}
\newcommand{\closearmyspecialrules}{\normalfontsize}
\newcommand{\armyspecialruleentry}[1]{\subtitle{#1}\vspace{5pt}}

\newcommand{\startarmyarmoury}{\bigtitle{\labels@armoury}\largefontsize\subtitle{}}
\newcommand{\closearmyarmoury}{\normalfontsize}

\newcommand{\startarmymagicalitems}{\newpage\largefontsize\bigtitle{\labels@magicalitems}\begin{multicols}{2}\raggedcolumns}
\newcommand{\closearmymagicalitems}{\end{multicols}\normalfontsize}

\newcommand{\armymagicalweapons}{\subtitle{\labels@magicalweapons}}
\newcommand{\armymagicalarmour}{\subtitle{\labels@magicalarmour}}
\newcommand{\armytalismans}{\subtitle{\labels@talismans}}
\newcommand{\armyenchanteditems}{\subtitle{\labels@enchanteditems}}
\newcommand{\armyarcaneitems}{\subtitle{\labels@arcaneitems}}
\newcommand{\armymagicalbanners}{\subtitle{\labels@magicalbanners}}

\newcommand{\startarmynewsection}[1]{\newpage\bigtitle{#1}\largefontsize}
\newcommand{\startarmynewsectionSP}[1]{\vspace{1.5cm}\bigtitle{#1}\largefontsize}
\newcommand{\closearmynewsection}{\normalfontsize}

\newcommand{\armynewsubsection}[1]{\subtitle{#1}\vspace{5pt}}
\newcommand{\armynewsubsubsection}[1]{\subsubtitle{#1}\vspace{3pt}}

\newcommand{\armylist}{\clearpage}

\newcommand{\quickrefsheettitle}{\clearpage\newgeometry{top=1.6cm, bottom=2cm, left=1cm, right=1cm}\bigtitle{\labels@quickrefsheet}\vspace*{0.4cm}}
\newcommand{\changelogtitle}{\clearpage\bigtitle{\labels@changelog}\spaceaftersection{}}

\newcommand{\spaceaftersection}{\vspace{0.8cm}}

\newcommand{\separator}{\noindent\begin{center}\textcolor{black!30}{\rule{0.7\columnwidth}{2pt}}\end{center}}


%%% Custom lists and description for first sections of the army books

\newcommand{\startpricelist}{\begin{samepage}\begin{description}[leftmargin=0.3cm, labelindent=0cm, labelsep=0.1cm]}
\def\endpricelist{\end{description}\end{samepage}}
\newcommand{\pricelistitem}[2]{\item \option{\textbf{#1}}{#2}\newline}

\newcommand{\startpricelistNSP}{\begin{description}[leftmargin=0.3cm, labelindent=0cm, labelsep=0.1cm]}
\def\endpricelistNSP{\end{description}}

\newcommand{\startitemlist}{\begin{multicols}{2}\raggedcolumns\begin{description}[leftmargin=0.3cm, labelindent=0cm, labelsep=0.1cm]}
\def\enditemlist{\end{description}\end{multicols}}
\newcommand{\listitem}[1]{\item[#1\spacebeforecolon{}:]}

\newcommand{\startitemlistonecol}{\begin{description}[leftmargin=0.3cm, labelindent=0cm, labelsep=0.1cm]}
\def\enditemlistonecol{\end{description}}
\newcommand{\listitemonecol}[1]{\item \textbf{#1\spacebeforecolon{}:}\newline}

\newenvironment{customitemize}{\begin{description}[leftmargin=0.3cm, labelindent=0cm, labelsep=0cm]}{\end{description}}
\newenvironment{customsubitemize}{\begin{itemize}[label={-}, labelsep=0.1cm, topsep=0cm, parsep=0cm, itemsep=0cm, leftmargin=0.4cm, labelindent=0cm]}{\end{itemize}}

%%% Table parameters %%%

\newcolumntype{M}[1]{>{\centering\let\newline\\\arraybackslash\hspace{0pt}}m{#1}}


%%%  Lists handling %%%

\newcommand{\addlocallist}{\listadd\locallists@dummy}%
\NewDocumentCommand{\parsespacelist}{>{\SplitList{ }} m }{%
	\ProcessList{#1}{\addlocallist}%
}%
\NewDocumentCommand{\parsecommalist}{>{\SplitList{,}} m }{%
	\ProcessList{#1}{\addlocallist}%
}%
\newcommand{\parselist}[3][,]{%
	\renewcommand\addlocallist{\listadd#3}%
  	\undef#3%
  	\ifstrequal{#1}{ }{\parsespacelist{#2}}{\parsecommalist{#2}}%
}


%%% Profiles handling %%%

% Element of a table that contains the characteristics of a model (or part of a model)
\newcommand\caraclist[1]{
	\parselist[ ]{#1}{\locallists@caraclist}%
	\forlistloop{&}{\locallists@caraclist}%
}

\newcommand\caraclistbold[1]{
	\parselist[ ]{#1}{\locallists@caraclist}%
	\forlistloop{&\bfseries}{\locallists@caraclist}%
}

% Line of a profile table, including bottom line. It is meant to contain the name of the model (or part), its characteristics (preferably, the second argument should contain the \carac macro), troop type and base size.
\newcommand{\profilefirstline}[4]{#1 & #2 &   & #3 & #4 }

% Start of a profile table. Includes the table commands, and the column labels. \profilecellsize is the size of the characteristics cells in the profile.
\newcommand{\profilecellsize}{0.56cm}
\newcommand{\profilestart}{%
	\noindent %
	\begin{tabular}{@{}p{3cm}@{}M{\profilecellsize}@{}M{\profilecellsize}@{}M{\profilecellsize}@{}M{\profilecellsize}@{}M{\profilecellsize}@{}M{\profilecellsize}@{}M{\profilecellsize}@{}M{\profilecellsize}@{}M{\profilecellsize}@{}p{2.7cm}@{}p{3.3cm}@{}p{2cm}@{}}%
	 &% \textbf{\labels@profile}
	\labels@M & \labels@WS & \labels@BS & \labels@S & \labels@T & \labels@W & \labels@I & \labels@A & \labels@Ld &%
	&%
	{\unitentryformat{\labels@trooptype}} &%
	{\unitentryformat{\labels@basesize}}%
}

% End of a profile table.
\newcommand{\profileend}{\end{tabular}}

% Algorithm to automatically use and fill previous command, with coherence check.
\providebool{profilefirst}
\newcommand{\profileitem}[1]{%
	\tabularnewline%
	\splitatinf{#1}\local@unitname\local@unitprofile%
	\local@unitname \expandafter\caraclistbold\expandafter{\local@unitprofile}%
	&%
	& \ifbool{profilefirst}{\unit@type}{}%
	& \ifbool{profilefirst}{%
		\ifsubstring{\unit@basesize}{x}{% Rectangular base
			\unit{\unit@basesize}{\milli\meter}%
		}{% Circular base
			\unit{\unit@basesize}{\milli\meter} \labels@roundbase%
		}%
	}{}%
	\global\boolfalse{profilefirst}%
}
\newcommand{\profile}[1]{%
	\parselist{#1}{\locallists@profileslist}%
	\profilestart%
	\global\booltrue{profilefirst}%
	\forlistloop{\profileitem}{\locallists@profileslist}%
	\profileend%
}


%%% Profiles handling in case of invocation %%%

\newcommand{\invocprofilestart}{%
	\noindent %
	\begin{tabular}{@{}p{3cm}@{}M{\profilecellsize}@{}M{\profilecellsize}@{}M{\profilecellsize}@{}M{\profilecellsize}@{}M{\profilecellsize}@{}M{\profilecellsize}@{}M{\profilecellsize}@{}M{\profilecellsize}@{}M{\profilecellsize}@{}M{2.2cm}@{}p{0.5cm}@{}p{3.3cm}@{}p{2cm}@{}}%
	 &% \textbf{\labels@profile}
	\labels@M & \labels@WS & \labels@BS & \labels@S & \labels@T & \labels@W & \labels@I & \labels@A & \labels@Ld & \unitentryformat{\labels@Invocation} &%
	&%
	{\unitentryformat{\labels@trooptype}} &%
	{\unitentryformat{\labels@basesize}}%
}

\newcommand{\invocprofileitem}[1]{%
	\tabularnewline%
	\splitatinf{#1}\local@unitname\local@unitprofile%
	\local@unitname \expandafter\caraclistbold\expandafter{\local@unitprofile}%
	& \ifbool{profilefirst}{\unit@invocation}{} &%
	& \ifbool{profilefirst}{\unit@type}{}%
	& \ifbool{profilefirst}{\unit{\unit@basesize}{\milli\meter}}{}%
	\global\boolfalse{profilefirst}%
}

\newcommand{\invocprofile}[1]{%
	\parselist{#1}{\locallists@profileslist}%
	\invocprofilestart%
	\global\booltrue{profilefirst}%
	\forlistloop{\invocprofileitem}{\locallists@profileslist}%
	\profileend%
}


%%%%%%%%%%%%%%%%%%
%%% Unit rules %%%
%%%%%%%%%%%%%%%%%%

%%% Entry title command %%%

\newcommand{\unitentry}[2]{\ifdefempty{#1}{}{\noindent #2}}


%%% Special rules %%%

% Special rules listing for a unit, with alphabetical order.
\newcommand{\ruleslist}[1]{%
	\parselist[,]{#1}{\locallists@ruleslist}%
	\begin{sortedlist}%
		\forlistloop{\addtosortedlist}{\locallists@ruleslist}%
	\end{sortedlist}%
}

% Special rules entry.
\newcommand{\specialrules}[1]{\unitentry{#1}{\unitentryformat{\labels@specialrules\spacebeforecolon{}:}\newline\hspace*{-\fontdimen2\font}\expandafter\ruleslist\expandafter{#1}.}}
\newcommand{\commonspecialrules}[2]{\unitentry{#2}{\unitentryformat{#1\spacebeforecolon{}:}\newline\hspace*{-\fontdimen2\font}\expandafter\ruleslist\expandafter{#2}.}}


%%% Magical abilities %%%

% Paths listing for a unit.
\newcommand{\pathslist}[1]{%
	\parselist[,]{#1}{\locallists@pathslist}%
	\begin{itemize*}[label={}, itemjoin={,}, itemjoin*={\listlastchoice}]%
		\forlistloop{\item}{\locallists@pathslist}%
	\end{itemize*}%
}

% Magic entry.
\newcommand{\magic}[2]{\unitentry{#2}{\unitentryformat{\labels@magic\spacebeforecolon{}: }\newline\ifdefempty{#1}{}{\textbf{\magiclevel{#1}}. }\labels@pathsused\expandafter\pathslist\expandafter{#2}.}}

% Wizard Conclave.
\newcommand{\magicwizardconclave}[1]{\unitentry{#1}{\unitentryformat{\labels@magic\spacebeforecolon{}: }\newline\textbf{\wizardconclave{}}\spacebeforecolon{}: #1.}}


%%% Equipment %%%

% Equipment listing.
\newcommand{\equipmentlist}[1]{%
	\parselist[,]{#1}{\locallists@equipmentlist}%
	\begin{sortedlist}%
		\forlistloop{\addtosortedlist}{\locallists@equipmentlist}%
	\end{sortedlist}%
}

% Equipment entry.
\newcommand{\weapons}[1]{\unitentry{#1}{\unitentryformat{\labels@weapons\spacebeforecolon{}:}\newline\hspace*{-\fontdimen2\font}\expandafter\equipmentlist\expandafter{#1}.}}

\newcommand{\armour}[1]{\unitentry{#1}{\unitentryformat{\labels@armour\spacebeforecolon{}:}\newline\hspace*{-\fontdimen2\font}\expandafter\equipmentlist\expandafter{#1}.}}


%%% Alignment %%%

\newcommand{\alignment}[1]{\unitentry{#1}{\unitentryformat{\labels@alignment\spacebeforecolon{}:}\newline\textbf{#1}.}}

%%% Green Hide Race %%%

\newcommand{\greenhideraceentry}[1]{\unitentry{#1}{\unitentryformat{\labels@greenhiderace\spacebeforecolon{}:}\newline\textbf{#1}.}}


%%% Options %%%

% Frame commands.
\newcommand{\optionsframestart}{\begin{innerframe}[\labels@options]}
\newcommand{\optionsframeend}{\end{innerframe}}

% Options listing.
\newcommand{\optionslist}[1]{%
	\parselist[,]{#1}{\locallists@optionslist}%
	\begin{description}[leftmargin=0.3cm, labelindent=0cm, labelsep=0cm, itemsep=0cm, parsep=0cm]%
		\forlistloop{\item\setoption}{\locallists@optionslist}%
	\end{description}%
}

% Options entry.
\newcommand{\options}[1]{\ifdefempty{#1}{}{\optionsframestart\vspace*{-0.4cm}\unitentry{#1}{\expandafter\optionslist\expandafter{#1}}\optionsframeend}}

% Option specific commands.
\newcommand{\setoption}[1]{%
	\noexpandarg\StrCut{#1}{=}\optiontext\optionvalue%
	\expandafter\ifstrequal\expandafter{\optionvalue}{}{%
		\optiontext%
	}{%
	\ifsubstring{\optionvalue}{\free}{%
		\option[\free]{\optiontext}{\optionvalue}%
	}{%
	\ifsubstring{\optionvalue}{\unlimited}{%
		\option[\unlimited]{\optiontext}{\optionvalue}%
	}{%
	\ifsubstring{\optionvalue}{\upto}{%
		\splitatinf{\optionvalue}\myoption\myvalue%
		\option[\upto]{\optiontext}{\myvalue}%
	}{%
	\ifsubstring{\optionvalue}{\permodel}{%
		\splitatinf{\optionvalue}\myoption\myvalue%
		\option[\permodel]{\optiontext}{\myvalue}%
	}{%
	\ifsubstring{\optionvalue}{\pershadygit}{% For Orcs N Goblins
		\splitatinf{\optionvalue}\myoption\myvalue%
		\option[\pershadygit]{\optiontext}{\myvalue}%
	}{%
	\ifsubstring{\optionvalue}{\permadgit}{% For Orcs N Goblins
		\splitatinf{\optionvalue}\myoption\myvalue%
		\option[\permadgit]{\optiontext}{\myvalue}%
	}{%	
	\ifsubstring{\optionvalue}{\perrune}{% For Dwarven Holds
		\splitatinf{\optionvalue}\myoption\myvalue%
		\option[\perrune]{\optiontext}{\myvalue}%
	}{%	
		\option{\optiontext}{\optionvalue}%
	}}}}}}}}%
}

\newcommand{\option}[3][]{#2\predotfill\dotfill\nobreak%
	% Add \upto token if necessary.
	\ifstrequal{#1}{\upto}{\upto~}{}%
	% The option can be free, have an unlimited cost, or have a points cost.
	\ifstrequal{#1}{\free}{\free}{\ifstrequal{#1}{\unlimited}{\unlimited}{\pts{#3}}}%
	% Add \permodel if necessary.
	\ifstrequal{#1}{\permodel}{\nobreak\permodel}{}%
	% Add \persomething if necessary.
	\ifstrequal{#1}{\pershadygit}{\nobreak\pershadygit}{}% For Orcs N Goblins
	\ifstrequal{#1}{\permadgit}{\nobreak\permadgit}{}% For Orcs N Goblins
	\ifstrequal{#1}{\perrune}{\nobreak\perrune}{}% For Dwarven Holds
}

\newcommand\optionschoice[2]{%
	\parselist[,]{#2}{\locallists@optionschoice}%
	#1%
	\begin{itemize}[label={}, parsep=0cm, labelindent=0cm, labelwidth=0cm, noitemsep, topsep=0em, leftmargin=0.3cm]%
	\forlistloop{\item\setoption}{\locallists@optionschoice}%
	\end{itemize}%
}

\newcommand\optionschoiceTWOCOL[2]{%
	\parselist[,]{#2}{\locallists@optionschoice}%
	#1%
	\begin{itemize}[label={}, parsep=0cm, labelindent=0cm, labelwidth=0cm, noitemsep, topsep=0em, leftmargin=0.3cm]%
	\setlength{\columnseprule}{0.5pt}
	\renewcommand{\columnseprulecolor}{\color{black!30}}
	\vspace*{-5pt}\begin{multicols}{2}\raggedcolumns
	\forlistloop{\item\setoption}{\locallists@optionschoice}%
	\end{multicols}\setlength{\columnseprule}{0pt}
	\end{itemize}%
}

% Option description in army desc.
\newcommand{\optiondef}[3]{\option{\textbf{#1}}{#2}\ifblank{#3}{}{\\{#3}}}


%%% Mount options %%%

% Frame commands.
\newcommand{\mountsframestart}{\begin{innerframe}[\labels@mounts]}
\newcommand{\mountsframeend}{\end{innerframe}}

% Mount listing.
\newcommand{\mountslist}[1]{%
	\parselist[,]{#1}{\locallists@mountslist}%
	\begin{description}[leftmargin=0.3cm, labelindent=0cm, labelsep=0cm, itemsep=0cm, parsep=0cm]%
		\forlistloop{\item\setoption}{\locallists@mountslist}%
	\end{description}%
}

% Mount entry.
\newcommand{\mounts}[1]{\ifdefempty{#1}{}{\mountsframestart\vspace*{-0.4cm}\unitentry{#1}{\expandafter\mountslist\expandafter{#1}}\mountsframeend}}


%%% Command group %%%

% Command group specific commands.
\define@key{commandgroup}{restriction}            {\def\commandgroup@restriction{#1}}
\define@key{commandgroup}{champion}               {\def\commandgroup@champion{#1}}
\define@key{commandgroup}{championallowance}      {\def\commandgroup@championallowance{#1}}
\define@key{commandgroup}{championoption}         {\def\commandgroup@championoption{#1}}
\define@key{commandgroup}{championprerestriction} {\def\commandgroup@championprerestriction{#1}}
\define@key{commandgroup}{championrestriction}    {\def\commandgroup@championrestriction{#1}}
\define@key{commandgroup}{banner}                 {\def\commandgroup@banner{#1}}
\define@key{commandgroup}{bannerallowance}        {\def\commandgroup@bannerallowance{#1}}
\define@key{commandgroup}{veteranstandardbearer}  {\def\commandgroup@veteranstandardbearer{#1}}
\define@key{commandgroup}{singlebannerallowance}  {\def\commandgroup@singlebannerallowance{#1}}
\define@key{commandgroup}{condsinglebannerallowance}  {\def\commandgroup@condsinglebannerallowance{#1}}
\define@key{commandgroup}{banneroption}           {\def\commandgroup@banneroption{#1}}
\define@key{commandgroup}{bannerrestriction}      {\def\commandgroup@bannerrestriction{#1}}
\define@key{commandgroup}{musician}               {\def\commandgroup@musician{#1}}
\define@key{commandgroup}{musicianrestriction}    {\def\commandgroup@musicianrestriction{#1}}
\newcommand{\defcommandgroup}{%
	\setkeys{commandgroup}{restriction=,
	                       champion=, championallowance=, championoption=, championprerestriction=, 
	                       championrestriction=, banner=, bannerallowance=, veteranstandardbearer=, 
	                       singlebannerallowance=, condsinglebannerallowance=, banneroption=, 
	                       bannerrestriction=, musician=, musicianrestriction=}%
	\setkeys{commandgroup}%
}

% Frame commands.
\newcommand{\commandgroupframestart}{\begin{innerframe}[\labels@commandgroup]}
\newcommand{\commandgroupframeend}{\end{innerframe}}

% Command group entry.
\newcommand{\commandgroup}[1]{%
	\defcommandgroup{#1}%
	\ifstrempty{#1}{}{\commandgroupframestart\vspace*{-0.2cm}%
		\begin{description}[leftmargin=0.3cm, labelindent=0cm, labelsep=0cm, itemsep=0cm, parsep=0cm]%
			% Command group title, including restrictions applying to all the command group
			\item \textbf{\expandafter\ifblank\expandafter{\commandgroup@restriction}{}{ \only{\commandgroup@restriction}\spacebeforecolon{}: }} 
			% Champion handling.
			\ifdefempty{\commandgroup@champion}{}{% We have a champion!
			\ifdefempty{\commandgroup@championprerestriction}{% There is no prerestriction to have a champion
				\item \hspace*{-0.04cm}\option{\labels@champion%
					% Possible restrictions to taking a champion
				    \expandafter\ifblank\expandafter{\commandgroup@championrestriction}{}{ \only{\commandgroup@championrestriction}}%
				    % Cost of a champion
				    }{\commandgroup@champion}%
				    % Magical allowance of the champion. Should probably not be used, champion option can do it as well and is more flexible.
					\ifdefempty{\commandgroup@championallowance}{}{\par\option[\upto]{\hspace*{0.3cm}- \labels@championallowance}{\commandgroup@championallowance}}%
					% Any option available to the champion, in the form option:cost
					\ifdefempty{\commandgroup@championoption}{}{%
						\splitatinf{\commandgroup@championoption}\local@option\local@cost%
						\par\option{\hspace*{0.3cm}- \local@option}{\local@cost}}%
			}{% There is a pre-restriction to have a champion
				\item \hspace*{-0.04cm}\commandgroup@championprerestriction	\newline%
				\option{\labels@champion}{\commandgroup@champion}%
				% Magical allowance of the champion. Should probably not be used, champion option can do it as well and is more flexible.
				\ifdefempty{\commandgroup@championallowance}{}{\par\option[\upto]{\hspace*{0.3cm}- \labels@championallowance}{\commandgroup@championallowance}}%
				% Any option available to the champion, in the form option:cost
				\ifdefempty{\commandgroup@championoption}{}{%
					\splitatinf{\commandgroup@championoption}\local@option\local@cost%
					\par\option{\hspace*{0.3cm}- \local@option}{\local@cost}}%
			} %End of the prerestriction of not condition
			}% End of champion handling
			\ifdefempty{\commandgroup@musician}{}{% We have a musician!
				\item \hspace*{-0.04cm}\option{\labels@musician%
					% Possible restrictions to taking a musician
				    \expandafter\ifblank\expandafter{\commandgroup@musicianrestriction}{}{ \only{\commandgroup@musicianrestriction}}%
				    % Cost of a musician
				    }{\commandgroup@musician}%
			}%
			\ifdefempty{\commandgroup@banner}{}{% We have a banner!
				\item \hspace*{-0.04cm}\option{\labels@standardbearer%
					% Possible restrictions to taking a banner
				    \expandafter\ifblank\expandafter{\commandgroup@bannerrestriction}{}{ \only{\commandgroup@bannerrestriction}}%
				    % Cost of a banner
				    }{\commandgroup@banner}%
				    % Magical banner, if all units of this type can take one.
					\ifdefempty{\commandgroup@bannerallowance}{}{\par\option[\upto]{\hspace*{0.3cm}- \labels@bannerallowance}{\commandgroup@bannerallowance}}%
					% Magical banner, if Veteran.
					\ifdefempty{\commandgroup@veteranstandardbearer}{}{\par\hspace*{0.3cm}- \labels@veteranstandardbearer%
					\expandafter\ifstrequal\expandafter{\commandgroup@veteranstandardbearer}{*}{*}{}%
					}%
					% Magical banner, if only one unit of this type can take one.
					\ifdefempty{\commandgroup@singlebannerallowance}{}{\par\option[\upto]{\hspace*{0.3cm}- \labels@singlebannerallowance}{\commandgroup@singlebannerallowance}}%
					% Magical banner, if only one unit of this type can take one, but with condtions.
					\ifdefempty{\commandgroup@condsinglebannerallowance}{}{%
						\splitatinf{\commandgroup@condsinglebannerallowance}\local@option\local@cost%
						\par\option[\upto]{\hspace*{0.3cm}- \labels@condsinglebannerallowance \local@option}{\local@cost}}%
					% Additional option for the banner, such as Hill Goblin Lookouts for Ogres
					\ifdefempty{\commandgroup@banneroption}{}{%
						\splitatinf{\commandgroup@banneroption}{\local@option}{\local@cost}%
						\par\option{\hspace*{0.3cm}- \local@option}{\local@cost}%
					}%
			}%
		\end{description}%
	\commandgroupframeend%
	 }%
}


%%% Unit rules %%%

% Frame commands.
\newcommand{\unitrulesframestart}{\begin{innerframe}[\labels@specialrules]}
\newcommand{\unitrulesframeend}{\end{innerframe}}

% Unit rules specific commands.
\newcommand{\unitrule}[2]{\item[#1\spacebeforecolon{}:]#2}

% Unit rule entry.
\newcommand{\unitrules}[1]{\ifdefempty{#1}{}{\unitrulesframestart\vspace*{-0.05cm}\begin{description}[leftmargin=0.3cm, labelindent=0cm, labelsep=0.1cm, itemsep=0.2cm, parsep=0cm]#1\end{description}\unitrulesframeend}}


%%% Special equipment %%%

% Frame commands.
\newcommand{\unitequipmentframestart}{\begin{innerframe}[\labels@specialequipment]}
\newcommand{\unitequipmentframeend}{\end{innerframe}}

% Special equipment specific commands.
\newcommand{\equipmentdef}[2]{\item[#1\spacebeforecolon{}:]#2}

% Special equipment entry.
\newcommand{\unitequipment}[1]{\ifdefempty{#1}{}{\unitequipmentframestart\vspace*{-0.05cm}\begin{description}[leftmargin=0.3cm, labelindent=0cm, labelsep=0.1cm, itemsep=0.2cm, parsep=0cm]#1\end{description}\unitequipmentframeend}}






%%%%%%%%%%%%%%%%%%%%%%%%%%%%%%%%
%%% Profile input and layout %%%
%%%%%%%%%%%%%%%%%%%%%%%%%%%%%%%%

%%% Input parameters %%%

\define@key{unit}{notinQRS}{\def\unit@notinQRS{#1}}
\define@key{unit}{name}{\def\unit@name{#1}}
\define@key{unit}{QRSname}{\def\unit@QRSname{#1}}
\define@key{unit}{profile}{\def\unit@profile{#1}}
\define@key{unit}{cost}{\def\unit@cost{#1}}
\define@key{unit}{invocation}{\def\unit@invocation{#1}}
\define@key{unit}{costpermodel}{\def\unit@costpermodel{#1}}
\define@key{unit}{maxmodels}{\def\unit@maxmodels{#1}}
\define@key{unit}{type}{\def\unit@type{#1}}
\define@key{unit}{unitsize}{\def\unit@unitsize{#1}}
\define@key{unit}{basesize}{\def\unit@basesize{#1}}
\define@key{unit}{commonspecialrules}{\def\unit@commonspecialrules{#1}}
\define@key{unit}{commontype}{\def\unit@commontype{#1}}
\define@key{unit}{commonspecialrulesB}{\def\unit@commonspecialrulesB{#1}}
\define@key{unit}{commontypeB}{\def\unit@commontypeB{#1}}
\define@key{unit}{specialrules}{\def\unit@specialrules{#1}}
\define@key{unit}{magiclevel}{\def\unit@magiclevel{#1}}
\define@key{unit}{magicpaths}{\def\unit@magicpaths{#1}}
\define@key{unit}{equipment}{\def\unit@equipment{#1}}
\define@key{unit}{alignment}{\def\unit@alignment{#1}}
\define@key{unit}{greenhiderace}{\def\unit@greenhiderace{#1}}
\define@key{unit}{weapons}{\def\unit@weapons{#1}}
\define@key{unit}{armour}{\def\unit@armour{#1}}
\define@key{unit}{wizardconclave}{\def\unit@wizardconclave{#1}}
\define@key{unit}{unitequipment}{\def\unit@unitequipment{#1}}
\define@key{unit}{options}{\def\unit@options{#1}}
\define@key{unit}{mounts}{\def\unit@mounts{#1}}
\define@key{unit}{commandgroup}{\def\unit@commandgroup{#1}}
\define@key{unit}{unitrules}{\def\unit@unitrules{#1}}
\define@key{unit}{additional}{\def\unit@additional{#1}}


%%% Frames definition %%%

% Unit's big frame.
\tikzset{unitprice/.style={draw=white, fill=white, rectangle, rounded corners, right, minimum height=0.7cm}}
\tikzset{unittitle/.style={draw=white, fill=white, rectangle, rounded corners, right, minimum height=0.7cm, font=\bfseries}}
\tikzset{unitlogo/.style={draw=white, fill=white, rectangle, right, minimum height=0.7cm}}

\newenvironment{unitframe}[2][]{%
	\mdfsetup{%
		nobreak=true,%
		linewidth=1pt,%
		linecolor=black!30,%
		roundcorner=5pt,%
		backgroundcolor=white,%
		innertopmargin=1.2\baselineskip,
		innerbottommargin=1.2\baselineskip,
		singleextra={
			\expandafter\ifblank\expandafter{\unit@cost}{}{%
				\node[unitprice,anchor=east,xshift=-0.5cm] at (P)%
					{%
						{{\smallfontsize\minprice} \Largefontsize\pts{\textbf{\unit@cost}}}%
					};
				}%
				\node[unittitle,xshift=0.5cm] at (P-|O)%
					{\Largefontsize\antiquefont\uppercase\expandafter\expandafter\expandafter{\unit@name}};
				\node[unitlogo, xshift=8.1cm, yshift=0.1cm] at (P-|O)%
					{\includegraphics[width=1.2cm]{\logolocalpath}};
		}
	}%
	\begin{mdframed}[]\relax%
}%
{%
\end{mdframed}%
}

% Inner small frames for options, special rules definition, ...
\tikzset{innertitle/.style={fill=white, rectangle, rounded corners, right, minimum height=8pt, xshift=0.5cm}}

\newenvironment{innerframe}[1][]{%
	\mdfsetup{%
		innerleftmargin=5pt,%
		innerrightmargin=5pt,%
		linecolor=black!30,%
		linewidth=0.5pt,%
		roundcorner=5pt,%
		backgroundcolor=white,%
		innertopmargin=1.1\baselineskip,
		singleextra={
		\node[innertitle] at (P-|O)%
			{\unitentryformat{#1}};
		}
	}%
	\vspace*{-0.2cm}\begin{mdframed}[]\relax%
}%
{%
\end{mdframed}%
}

%%% Command to add a new unit definition %%%

\newcommand{\defunit}{
	\setkeys{unit}{%
		notinQRS=, name=, QRSname=, profile=, cost=, invocation=, costpermodel=, maxmodels=, type=, unitsize=, basesize=, commonspecialrules=, commontype=, commonspecialrulesB=, commontypeB=, specialrules=, magiclevel=, magicpaths=, alignment=, greenhiderace=, equipment=, weapons=, armour=, wizardconclave=, unitequipment=, options=, mounts=, commandgroup=, unitrules=, additional=%
	}%
	\setkeys{unit}%
}

\newcommand{\showunit}[1]{
	\defunit{#1}
	\begin{unitframe}[\unit@name]{\unit@cost}
	\mdfsetup{style=defaultoptions}
	\expandafter\ifblank\expandafter{\unit@unitsize}{}{%
	\expandafter\ifstrequal\expandafter{\unit@unitsize}{1}{% single model
		% Can you add model to this single model ?
		\expandafter\ifblank\expandafter{\unit@maxmodels}{% no		
			{\hspace*{0.25cm}\labels@Singlemodel}%
		}{% yes
			{\hspace*{0.25cm}\mincostfor{} \textbf{1} \labels@model{}. \maxunitsize{}\spacebeforecolon{}: \textbf{\unit@maxmodels} \labels@models{}.\hfill \additionalfigscost{} {\largefontsize\pts{\textbf{\unit@costpermodel{}}}\permodel}\hspace*{0.1cm}}%
		}%
	}{% not single model
		% Test if we wanna print a sentence instead of unit number
		\ifsubstring{\unit@unitsize}{SPECIAL-}{%
			\hspace*{0.25cm}\StrDel{\unit@unitsize}{SPECIAL-}%
		}{%	
			{\hspace*{0.25cm}\mincostfor{} \textbf{\unit@unitsize} \labels@models{}. \maxunitsize{}\spacebeforecolon{}: \textbf{\unit@maxmodels} \labels@models{}.\hfill \additionalfigscost{} {\largefontsize\pts{\textbf{\unit@costpermodel{}}}\permodel}\hspace*{0.1cm}}%
		}%
	}%
	}%
	\vspace*{-0.1cm}
	\noindent\begin{center}\textcolor{black!30}{\rule{\columnwidth}{1pt}}\end{center}
		\expandafter\ifblank\expandafter{\unit@invocation}{%
			\expandafter\profile\expandafter{\unit@profile}
		}{%
			\expandafter\invocprofile\expandafter{\unit@profile}
		}
	\noindent\begin{center}\textcolor{black!30}{\rule{\columnwidth}{1pt}}\end{center}
	\vspace*{-0.2cm}
	\setlength\multicolsep{0pt}
	\begin{multicols}{2}
		\raggedcolumns
		\vspace*{-0.3cm}{\setlength{\parskip}{0.3cm}
		\expandafter\ifblank\expandafter{\unit@alignment}{}{\noindent\parbox{\columnwidth}{\alignment{\unit@alignment}}}
		
		\expandafter\ifblank\expandafter{\unit@greenhiderace}{}{\noindent\parbox{\columnwidth}{\greenhideraceentry{\unit@greenhiderace}}}
		
		\expandafter\ifblank\expandafter{\unit@equipment}{}{\noindent\parbox{\columnwidth}{\equipment{\unit@equipment}}}
				
		\expandafter\ifblank\expandafter{\unit@weapons}{}{\noindent\parbox{\columnwidth}{\weapons{\unit@weapons}}}
		
		\expandafter\ifblank\expandafter{\unit@armour}{}{\noindent\parbox{\columnwidth}{\armour{\unit@armour}}}
		
		\expandafter\ifblank\expandafter{\unit@commonspecialrules}{}{\noindent\parbox{\columnwidth}{\commonspecialrules{\unit@commontype}{\unit@commonspecialrules}}}
		
		\expandafter\ifblank\expandafter{\unit@commonspecialrulesB}{}{\noindent\parbox{\columnwidth}{\commonspecialrules{\unit@commontypeB}{\unit@commonspecialrulesB}}}
		
		\expandafter\ifblank\expandafter{\unit@specialrules}{}{\noindent\parbox{\columnwidth}{\specialrules{\unit@specialrules}}}
		
		\expandafter\ifblank\expandafter{\unit@magicpaths}{}{\noindent\parbox{\columnwidth}{\magic{\unit@magiclevel}{\unit@magicpaths}}}
		
		\expandafter\ifblank\expandafter{\unit@wizardconclave}{}{\noindent\parbox{\columnwidth}{\magicwizardconclave{\unit@wizardconclave}}}
		}
		\vspace{0.1cm}
		\mounts{\unit@mounts}
		\options{\unit@options}
		\expandafter\ifblank\expandafter{\unit@commandgroup}{}{\expandafter\commandgroup\expandafter{\unit@commandgroup}}
		\unitrules{\unit@unitrules}
		\unitequipment{\unit@unitequipment}
	\end{multicols}
	\vspace*{0.1cm}\unit@additional
	\end{unitframe}
	% Database filling for auto QRS
	\expandafter\ifblank\expandafter{\unit@notinQRS}{%
	\DTLnewrow{profiles}%
	\expandafter\ifblank\expandafter{\unit@QRSname}{%
		\expandafter\profiledtbfillname\expandafter{\unit@name}%
	}{%
		\expandafter\profiledtbfillname\expandafter{\unit@QRSname}%
	}
	\expandafter\profiledtbfillcategory\expandafter{\profilecategory}%
	\expandafter\profiledtbfilltrooptype\expandafter{\unit@type}%
	\expandafter\ifblank\expandafter{\unit@invocation}{}{\expandafter\profiledtbfillinvocation\expandafter{\unit@invocation}}%
	\expandafter\profiledtbfillcarac\expandafter{\unit@profile}
	}{}%
}


%%% Changelog commands %%%

\newcommand{\newlog}[2]{%
\vspace*{0.2cm}\noindent{\antiquefont\Large\textbf{V#1}}
\parselist[,]{#2}{\locallists@changelist}%
\begin{itemize}[itemsep=0pt]%
\forlistloop{\item[-]}{\locallists@changelist}%
\end{itemize}%
}

\newcommand{\startchangelog}{\begin{multicols}{2}\vspace*{-0.2cm}}
\def\endchangelog{\end{multicols}}


\newcommand{\booktitle}{Peaux Malades}
\newcommand{\version}{0.99.1}
\newcommand{\frenchversion}{2.0}
\newcommand{\translationteam}{\item \og AEnoriel \fg \item \og Anglachel \fg \item \og Astadriel \fg \item \og Batcat \fg \item \og Eru \fg  \item \og Gandarin \fg \item \og Groumbahk \fg \item \og Iluvatar \fg \item \og Lamronchak \fg \item \og Mammstein \fg}

% Army special rules

\newcommand{\commonorc}{Orque Commun}
\newcommand{\ironorc}{Orqu'en Fer}
\newcommand{\feralorc}{Orque Primitif}
\newcommand{\commongoblin}{Gobelin Commun}
\newcommand{\cavegoblin}{Gobelin des Cavernes}
\newcommand{\forestgoblin}{Gobelin des Forêts}

\newcommand{\unruly}{Indiscipliné}
\newcommand{\borntofight}{Né pour la Baston}
\newcommand{\waaargh}{Waaargh !}
\newcommand{\greentide}{Marée Verte}
\newcommand{\venomousfangs}{Dard Venimeux}
\newcommand{\shambolic}[1]{Chamboul' Tout (#1)} %M% Bof. Erratique ?
\newcommand{\runningamok}{Nawak !}
\newcommand{\ricochet}[1]{Strike (#1)} %M% Un terme non anglophone ? En Plein Dedans ? Plein Fer ? Touché ?

% Armoury

\newcommand{\powershrooms}{Champix de Pouvoir} %M% Champis ? Pourquoi Champix ?
\newcommand{\mammothstabber}{Perceur 'eud Gros Trucs} %M% En VO ils ont essayé de pas mal s'éloigner de la trame GW de faire des fautes pour donner un style Orques et Gobs, faudrait essayer faire pareil. Défonce-Mammouth ?

% Other Rules

\newcommand{\nets}{Filets}
\newcommand{\motherskiss}{Baiser de la Mère-Araignée}
\newcommand{\sneaky}{Sournois}
\newcommand{\surprise}{Planké} %M% Même remarque. "Coucou !" ? "Tu veux voir mon boulet ?" ? (Lire "Tu veux voir mon boul', hé ?" xD)
\newcommand{\oiitbites}{Fo pas s'approcher !} %M% Pareil. "Eh mais ça mord !"
\newcommand{\rowsofteeth}{Rangées de Dents}
\newcommand{\theyreeverywhere}{Partent en Sucette !}
\newcommand{\orcoverseer}{Garde Chiourme} %M% Chiourme ?! Gardien de Gosses ? Garderie ? Pion ?
\newcommand{\splatterer}{Krabouilleur} %M% Style. Écrabouilleur, ou Appareil à Crêpes.
\newcommand{\gitlauncher}{Lance-kamikaz'} %M% Fronde à Psychopathes
\newcommand{\smasher}{Frakasseur} %M% Fracasseur
\newcommand{\pointedsticks}{Bouts Pointus}
\newcommand{\pursuitmode}{Vif et Furieux} %M% Changé rapide pour vif pour éviter la confusion avec la règle Rapide.
\newcommand{\smellslikegreenspirit}{Tu me Vois, Tu me Vois Plus} %M% Je suis vraiment tenté de le laisser en VO ... Tout le monde ou presque reconnaitra la référence, et je la trouve très bien :D A moins que quelqu'un trouve un aussi bon jeu de mot avec une chanson. "Le Vert l'Emportera" ? Et tout disparaitraaaaaaaaa aaaaaa. Eh mine de rien c'est bien trouvé aussi non ?
\newcommand{\commontrolls}{Trolls Communs}
\newcommand{\cavetrolls}{Trolls des Cavernes}
\newcommand{\bridgetrolls}{Trolls de Rivière}
\newcommand{\trollbelch}{Bile de Troll}
\newcommand{\ballista}{Embrocheur}
\newcommand{\lookatemgo}{Perd le Contrôle} %M% J'aime bien l'idee un peu moqueuse de la VO. "A Plus Tard !" ? "Adieu !" ?
\newcommand{\weblauncher}{Lance-Toiles}
\newcommand{\smashemflat}{Cé Nous Kon les Kraz !} %M% Meme si la VO a fait une petite entorse, je prefererais qu'on ait notre propre style. Faites-en de la Purée !
\newcommand{\wevegotthegreenlight}{Vert de Rage} %M% Tres Bon !!! C'etait pas facile de trouver aussi bien que le feu vert :D
\newcommand{\bouncers}{Bondisseurs}
\newcommand{\spidermothershrine}{Autel de la Mère-Araignée}

% Spells



% Characters

\newcommand{\orcwarlord}{Seigneur de Guerre Orque}
\newcommand{\orcbigshaman}{Grand Chaman Orque}
\newcommand{\goblinking}{Seigneur Gobelin} % Pourquoi pas Roi ? ça fait une petite variation ça mange pas de pain.
\newcommand{\goblinbigshaman}{Grand Chaman Gobelin}
\newcommand{\orcchief}{Boss Orque} %M$ Si on pouvait éviter les anglicismes, ce serait cool. Un Caïd ?
\newcommand{\orcshaman}{Chaman Orque}
\newcommand{\goblinchief}{Boss Gobelin} %M% Pareil. Doyen, Meneur, Régent ? Doyen j'aime bien, les plus vieux gobelins sont les plus vicieux.
\newcommand{\goblinshaman}{Chaman Gobelin}

% Core

\newcommand{\orcs}{Orques} %M% Guerriers Orques ? ça fait un peu "race dans son ensemble comme ça", avec le marmot sous le bras et la matronne qui les pourchasse avec un bras d'ahllebardier de l'Empire. Salut, on est des Orques.
\newcommand{\orceadbashers}{Orques Kraz'Krânes} %M% Trop GW style. Orques Baraqués, Trapus ? Gaillards Orques ? Gros Durs ?
\newcommand{\goblins}{Gobelins} %M% Guerriers Gobelins ? On parle de puissants guerriers !
\newcommand{\shadygit}{Assassin Gobelin} %M% Psychopathe Gobelin ? Un peu plus Peaux Vertes style
\newcommand{\madgit}{Boulet-Fou}
\newcommand{\goblinraiders}{Chevaucheurs Gobelins} %M% Pillards Gobelins. Un peu plus de couleur :D
\newcommand{\orcboarriders}{Chevaucheurs de Sangliers} %M% Orques sur Sanglier. Chevaucheur ça fait tache avec Sanglier.

% Special

\newcommand{\ironorcs}{Full Metal'Orques} %M% Y'a pas la distinction entre iron orc et iron orc en VO :) Et si on choisissait le meilleur des deux entre Orqu'en Fer et Full Metal'Orques ? Je vote Orqu'en Fer !
\newcommand{\mountedeadbashers}{Orques Kraz'Krânes sur Sangliers} %M% voir 'eadbashers
\newcommand{\orcboarchariot}{Char à Sangliers}
\newcommand{\goblinwolfchariot}{Char à Loups}
\newcommand{\gnasherdashers}{Barjos sur Gniark}
\newcommand{\gnasherherd}{Meute de Gniarks}
\newcommand{\greenhidecatapults}{Catapultes des Peaux Vertes}
\newcommand{\grotlings}{Morveux}
\newcommand{\scrapwagon}{Machine à Pomper} %M% Trop GW a mon avis.
\newcommand{\trolls}{Trolls}
\newcommand{\giant}{Géant}

% Rare

\newcommand{\skewerer}{}
\newcommand{\gnasherwreckingteam}{}
\newcommand{\gargantula}{}
\newcommand{\greatgreenidol}{}

% Mounts

\newcommand{\wyvern}{}
\newcommand{\warboar}{}
\newcommand{\wolf}{}
\newcommand{\cavegnasher}{}
\newcommand{\scuttlerspider}{}
\newcommand{\huntsmenspider}{}


% Profile names

\newcommand{\rider}{Cavalier}
\newcommand{\}{}
\newcommand{\}{}
\newcommand{\}{}
\newcommand{\}{}


% Profile wording

\newcommand{\mayjointhecultofnabh}{Peut rejoindre le \cultofnabh}
\newcommand{\mayjointhecultofyema}{Peut rejoindre le \cultofyema}
\newcommand{\oracleyemanote}{Dans le cas d'une allégeance au \cultofyema{}, seules les Disciplines \lust{}, \shadows{}, \death{} ou \blackmagic{} sont accessibles.}
\newcommand{\acultpriestmustjoinoneofthefollowingcults}{Une \cultpriest{} doit rejoindre l'un des Cultes suivants}
\newcommand{\seeexecutionerspecialunit}{(voir l'unité spéciale \executioners{})}
\newcommand{\seedancersofyemaspecialunit}{(voir l'unité spéciale \dancersofyema{})}
\newcommand{\maypurchaseanynumberofpoisonsseebelow}{Peut utiliser n'importe quel nombre de \poisons{} (voir ci-dessous)}
\newcommand{\assassinforkpaths}{Peut devenir au choix :}
\newcommand{\canonlybeusedagainstclosecombatattacks}{Ne peut être utilisé que contre des Attaques de Corps a Corps}
\newcommand{\corsairsvanguardnote}{Une unité de \corsairs{} peut prendre cette option pour chaque Personnage amélioré en \textbf{\fleetcommander}}
\newcommand{\maytakearepeatercrossbow}{Peut prendre une \repeatercrossbow}
\newcommand{\maxmodelsoneofakind}{(max. 15 figs., \oneofakind{})}
\newcommand{\huntingchariotforkchoice}{Doit prendre une des armes suivantes :}
\newcommand{\darkacolytesnote}{%
Si l'unité rejoint le \cultofyema{}, Le Conclave de Sorciers connait à la place les sorts \blackmagicspellthree{} (Discipline \blackmagic{}) et \lustsignaturespell{} (Discipline \lust{}).
}
\newcommand{\divinealtaralignment}{%
Un \divinealtar{} doit rejoindre un Culte. La figurine gagne alors les règles spéciales, l'allégeance et l'équipage correspondant.
}
\newcommand{\elvenhorsenote}{%
Seulement si le Général a rejoint le \cultofyema{} et si le \elvenhorse{} est monté par un \dreadprince{}, un \captain{} ou une \cultpriest{}.
}
\newcommand{\beastmastersmountonly}{(seulement si monté par un \beastmaster{})}
\newcommand{\seedivinealtarrareunit}{(voir l'unité rare \divinealtar{})}

% Profile rules

\newcommand{\assassinthrowingweaponsrule}{%
\range{12}, Force de l'utilisateur, \armourpiercing{1}, \multipleshots{3}, \quicktofire{}. Peuvent être enduites de \poison{}.
}

\newcommand{\masterpoisonerrule}{%
Un \assassin{} peut être équipé d'un ou plusieurs \poisons{}. Au début de chaque Tour de Joueur, déclarez un seul \poison{} que l'\assassin{} va utiliser. Les \poisons{} ne peuvent être appliqués que sur des armes standard et font effet au tir comme au corps à corps.
}

\newcommand{\nightshaderule}{%
Les attaques faites avec ce \poison{} sont résolues avec une Force égale à l'Endurance de la cible +1, jusqu'à une Force de 6 maximum.
}

\newcommand{\wolfsbanerule}{%
Les attaques faites avec ce \poison{} ont \lethalstrike{}, et vous pouvez relancer les jets pour blesser ratés.
}

\newcommand{\bloodrootrule}{%
Les attaques faites avec ce \poison{} ont un bonus de +1 pour blesser et ont \multiplewounds{2}{\characters{}, \riddenmonsters{}}.
}

\newcommand{\gladiatorweaponsrule}{%
Arme de Corps à Corps. Le porteur gagne la règle \weaponmaster{}. Cette arme peut être utilisée comme une \hw{} et un \shield{}, un \flail{}, une \pw{}, une \spear{} et un \shield{}, une \gw{} ou enfin une \halberd{}.
}

\newcommand{\executionersbladerule}{%
\gw{}. \multiplewounds{2}{\infantry{}, \cavalry{}, \monstrousbeast{}}, \lethalstrike{}.
}

\newcommand{\dreadguardiansrule}{%
La figurine gagne +1 en Capacité de Combat et la règle \fightinextrarank{}.
}

\newcommand{\giantbowrule}{%
\textbf{\artilleryweapon} de type \textbf{\boltthrower}. \range{24}, Force 5, \multiplewounds{1D3}{}, \armourpiercing{6}, \quicktofire{}.
}

\newcommand{\harpoonlauncherrule}{%
Arme de Tir. \range{24}, Force 7, \multiplewounds{1D3}{}, \reload{}, \quicktofire{}.
}

\newcommand{\elvenboltthrowerrule}{%
\textbf{\artilleryweapon} de type \textbf{\boltthrower}. \range{48}, Force 6, \multiplewounds{1D3}{}, \armourpiercing{6}.
}

\newcommand{\repeatingshotsrule}{%
La \dreadreaper{} peut aussi tirer tirer comme une \textbf{\artilleryweapon} de type \textbf{\volleygun}. \range{48}, Force 4, \armourpiercing{1}, \multipleshots{6}.
}

\newcommand{\divineblessingsrule}{%
Au début de chaque Tour de Jeu, choisissez l'une des faveurs ci-dessous. Elle s'applique jusqu'à la fin du Tour de Jeu. 
Une seule unité amie (sauf un \monster{}) à \distance{12} ou moins de l'Autel gagne cette faveur. Une unité ne peut recevoir qu'une seule faveur à la fois. De plus, seules les figurines n'appartenant à aucun Culte ou appartenant au même Culte que l'Autel peuvent être ciblées.
\begin{customsubitemize}
	\item L'unité gagne une \wardsave{5}.
	\item L'unité gagne +1 Attaque (sauf les Montures).
	\item L'unité gagne +1 en Commandement.
\end{customsubitemize}
Sinon, une unité ennemie à moins de \distance{12} peut être ciblée à la place. Dans ce cas, l'unité ciblée subit un malus de -1 en Commandement jusqu'à la fin du Tour de Jeu.
}



\begin{document}

\newgeometry{margin=1in}

% Table options
\arrayrulecolor{black!30}
\setlength{\arrayrulewidth}{0.5pt}
\renewcommand{\arraystretch}{1.2}

\begin{titlepage}
\begin{center}

\ifdef{\booktitle}{}{\newcommand{\booktitle}{Missing title}}
\ifdef{\version}{}{\newcommand{\version}{Missing version}}

{\antiquefont\fontsize{40}{48}\selectfont\noindent\labels@fantasybattles

\labels@NinthAge}

\vspace*{0.5cm}
\ifdef{\booklogo}{%
\includegraphics[height=10cm]{\booklogo}%
}{%
\includegraphics[height=10cm]{../Layout/pics/logo_9th.png}%
}

\vspace*{-1cm}
{\antiquefont\fontsize{50}{60}\selectfont \booktitle
\vspace{0.4cm}

\fontsize{14}{16.8}\selectfont \labels@armyrules{}

Beta v\version{} - \today{}}

\ifdef{\frenchversion}{{\fontsize{14}{16.8}\selectfont \vspace{0.2cm}\noindent\texttt{VF \frenchversion}}}{}
\vfill

\begin{tabular}{@{}m{2cm}@{\hskip 20pt}m{13cm}@{}}
\includegraphics[width=2cm]{../Layout/pics/seal_9th.png} &
{\fontsize{10}{12}\selectfont \textcolor{black!50}{\noindent\labels@frontpagecredits}}

\ifdef{\frontpageaddstuff}{{\fontsize{10}{12}\selectfont \noindent\textcolor{black!50}{\frontpageaddstuff}}}{}

\vspace*{10pt}
\noindent{\fontsize{10}{12}\selectfont \textcolor{black!50}{\labels@license}}
\tabularnewline
\end{tabular}


\end{center}

\newpage

\thispagestyle{empty}

{\fontsize{10}{12}\selectfont

\begin{center}\noindent{\Largerfontsize\textbf{\labels@tableofcontents}}\end{center}

\vspace*{0.2cm}\begin{multicols}{2}

\tocfirstcolumn

\vspace*{\fill}\columnbreak

\tocentry{lordtitle}{\labels@lords}

\tocentry{herotitle}{\labels@heroes}

\ifdef{\tocmounts}{\tocentry{mountstitle}{\tocmounts}}{}

\tocentry{coretitle}{\labels@coreunits}

\tocentry{specialtitle}{\labels@specialunits}

\tocentry{raretitle}{\labels@rareunits}

\vspace*{\fill}\end{multicols}

\ifdef{\labels@introduction}{\vspace{0.7cm}\labels@introduction}{\vphantom{1pt}}
\vfill

\noindent\newrule{\labels@rulechanges}

\bigskip
\noindent \labels@latexcredit
}


\end{titlepage}

\restoregeometry

\startarmyspecialrules

\armyspecialruleentry{\killerinstinct}

L'élément de figurine peut relancer ses jets pour blesser ayant donné un \result{1} naturel au Corps à Corps.

\armyspecialruleentry{\masterofthedarkarts}

Si votre armée comporte une ou plusieurs figurines bénéficiant de cette règle, ajoutez +1 à vos jets de \channel{} pour les Dés de Pouvoir (mais pas de Dissipation).

\armyspecialruleentry{\auraofdespair}

Toute unité ennemie en contact socle à socle avec au moins une figurine dotée de cette règle doit jeter 1D6 supplémentaire lors de ses tests de Commandement, à l'exception des tests de Moral, et ignorer le dé ayant donné le résultat le plus bas.

\armyspecialruleentry{\alphapredator}

Le \monster{} gagne +1 en Capacité de Combat, Initiative et Commandement.

\begin{multicols}{2}\raggedcolumns
\armyspecialruleentry{\fleetcommander}

La figurine gagne \innatedefence{5} mais ne peut pas avoir de monture. Les unités ennemies qui sont démoralisées lors d'un combat avec cette figurine doivent lancer un dé supplémentaire pour leur jet de fuite et ignorer le dé ayant le résultat le plus grand.

Pour chaque Personnage doté de cette règle, une unité de \corsairs{} peut gagner la règle \vanguard{}.

\columnbreak
\armyspecialruleentry{\beastmaster}

Les unités alliées de figurines montées, de \monsters{} et de \warbeasts{} à moins de \distance{12} de la figurine lancent un dé supplémentaire pour leurs tests de \frenzy{} et de \stupidity{} et ignorent le dé avec le résultat le plus grand.

Au début de chaque manche de Corps à Corps, une unité alliée de \cavalry{}, \monstrouscavalry{} ou un \monster{} à moins de \distance{6} de la figurine peut gagner la \hatred{} pour le reste de la manche. Seules les montures sont affectées. Remarquez que la \hatred{} n'a d'effet que lors de la première manche de Corps à Corps et qu'il n'est pas possible de cibler un \riddenmonster{}.

\end{multicols}

\closearmyspecialrules




\vspace*{1.5cm}
\startarmyarmoury

\startitemlistonecol

\listitemonecol{\repeatercrossbow} Arme de Tir. \range{24}, Force 3, \armourpiercing{1}, \multipleshots{2}.

\listitemonecol{\petrifyingstare} Arme de Tir. \range{12}, Force 4, \armourpiercing{6}, \multipleshots{2}. Les jets pour blesser sont effectués contre l'Initiative de la cible au lieu de l'Endurance.

\enditemlistonecol

\closearmyarmoury




\startarmynewsection{Cultes}

\newcommand{\logosize}{4cm}
\begin{multicols}{2}\raggedcolumns
\begin{center}
\includegraphics[width=\logosize]{pics/cultofnabh.png}
\armyspecialruleentry{\cultofnabh}

L'élément de figurine gagne la \hatred{} mais ne peut plus bénéficier de la règle \killerinstinct{}.
\end{center}

\begin{center}
\includegraphics[width=\logosize]{pics/cultofyema.png}
\armyspecialruleentry{\cultofyema}

La figurine gagne +1 en Mouvement ainsi que la règle \strider{}, mais ne peut plus bénéficier de la règle \killerinstinct{}.
\end{center}
\end{multicols}

\armyspecialruleentry{\cultrivalry}

Une figurine ne peut rejoindre qu'un seul Culte. Une unité contenant une ou plusieurs figurines appartenant à un Culte ne peut pas bénéficier des règles \holdyourground{}, \inspiringpresence{} et \divineblessings{} provenant de figurines d'un autre Culte. Un Personnage appartenant à un Culte ne peut pas rejoindre d'unité contenant une ou plusieurs figurines d'un autre Culte.

\armyspecialruleentry{\cultistgeneral}

Si le Général de l'armée appartient à un Culte, l'armée ne peut pas contenir d'unités de l'autre Culte. Touts les éléments de figurine des unités de base ayant la règle \killerinstinct{} rejoignent gratuitement le même Culte que le Général. De plus, toute unité possédant l'option de rejoindre le Culte du Général doit le faire.

\closearmynewsection

\startarmymagicalitems

\armymagicalweapons

\startpricelist

\pricelistitem{Hache du Bourreau}{60/40}\infantry{} uniquement.\newline Type : \gw{}. Les attaques portées avec cette arme ont un bonus de +3 en Force (au lieu de +2) et \multiplewounds{2}{}.

\pricelistitem{Fouet de Domination}{40}Type : \hw{}. Le porteur gagne +1 Attaque. Les attaques de Corps à Corps effectuées avec cette arme sont toujours résolues avec une Force de 5 (ignorez tout type de modificateur). Toute figurine subissant une blessure non sauvegardée par cette arme voit sa CC réduite à 1 jusqu'à la fin de la manche de Corps à Corps.

\endpricelist

\armymagicalarmour

\startpricelist

\pricelistitem{Armure Pourpre}{20}\infantry{} uniquement.\newline Type : \ha{}. Pour chaque blessure non sauvegardée que le porteur inflige en Corps à Corps, il gagne +1 en Sauvegarde d'Armure, jusqu'à obtenir 1+ au mieux, pour le reste de la partie.

\endpricelist

\armytalismans

\startpricelist

\pricelistitem{Manteau de Minuit}{50}Le porteur gagne une \wardsave{3} qu'il peut uniquement utiliser contre les Attaques à Distance. De plus, le porteur gagne \multiplewounds{1D3}{} et \lethalstrike{} lors des premières manches de Corps à Corps.

\pricelistitem{Amulette Malveillante}{35}Si un Sorcier ennemi à moins de \distance{12} lance un sort avec succès et résout ses effets, et qu'au moins deux des Dés de Pouvoirs utilisés ont eu pour résultat non modifié \result{1}, le lanceur subit un Fiasco. Un sort ne peut pas générer plus d'un Fiasco.

\endpricelist

\armyenchanteditems

\startpricelist

\pricelistitem{Anneau d'Obscurité}{35}L'unité du porteur bénéficie d'un Couvert Léger. Si elle bénéficiait déjà d'un Couvert Léger, il passe à Couvert Lourd à la place. Les attaques de Corps à Corps contre l'unité du porteur sont résolues avec un malus de -1 en Capacité de Combat.

\endpricelist

\armyarcaneitems

\startpricelist

\pricelistitem{Dague de Moraec}{35/25}Au début de la Phase de Magie, le porteur peut infliger 1D3 blessures sans aucune sauvegarde à son unité. Dans ce cas, les sorts lancés par le porteur voient leur valeur de lancement réduite d'autant que le nombre de blessures infligées de cette manière, jusqu'à la fin de cette Phase de Magie.

\pricelistitem{Familier Mystique}{25}Au début de chacune de vos Phases de Magie, placez un Familier, figurine avec un socle de \unit{20x20}{\milli\meter}, à moins de \distance{6} du porteur, et à plus de \distance{1} de toute figurine ou de tout Terrain Infranchissable. Le Familier est de taille Petite. Quand il lance un sort, à l'exception d'un sort lié à un Objet de Sort, le porteur peut choisir de déterminer les lignes de vue, les portées et l'arc frontal depuis le Familier. À la fin de la Phase, retirez le Familier du champ de bataille.

\endpricelist

\armymagicalbanners

\startpricelist

\pricelistitem{Étendard de l'Armada Cauchemardesque}{55}\fleetcommander{} uniquement.\newline Toutes les unités de \corsairs{} et de \dreadlegionnaires{} à moins de \distance{6} reçoivent un bonus de +1 pour blesser au Corps à Corps.

\pricelistitem{Bannière Ensanglantée}{35}Les éléments de figurine dotés de la règle \killerinstinct{} de l'unité peuvent relancer les résultats de \result{1} ou \result{2} de leur jets pour blesser ratés.

\endpricelist

\closearmymagicalitems



%%% START OF THE ARMYLIST - Translators shouldn't have to edit it %%%

%%% v0.99.9

\armylist

\lordstitle

\showunit{
	name={\dreadprince},
	cost={140},
	profile={ < 5 7 7 4 3 3 8 4 10},
	type=\infantry{},
	basesize=20x20,
	unitsize=1,
	commontype=\elvencommonrules{},
	commonspecialrules={\killerinstinct{},\lightningreflexes{}},
	armour={\la},
	options={
		\magicalitemsallowance{}=\upto{}<100,
		\onechoiceonly{
			\cultofnabh{}=20,
			\cultofyema{}=20,
			\beastmaster{}=40,
			\fleetcommander{}=50,
		},
		\shield{}=5,
		\ha{}=8,
		\shootingweapononechoice{
			\repeaterhandbow{}=2,		
			\repeatercrossbow{}=4,
		},
		\combatweapononechoice{
			\pw{}=5,
			\halberd{}=8,
			\gw{}=10,
			\lance{}=15,
		},
	},
	mounts={
		\elvenhorse{}=20,
		\raptor{}=35,
		\raptorchariot{}=40,
		\pegasus{}=60,
		\manticore{}=105,
		\dragon{}=250,
	},
}

\showunit{
	name={\exaltedoracle},
	cost={185},
	profile={ < 5 4 4 3 3 3 5 1 9},
	type=\infantry{},
	basesize=20x20,
	unitsize=1,
	commontype=\elvencommonrules{},
	commonspecialrules={\killerinstinct{},\lightningreflexes{}},
	specialrules={\masterofthedarkarts{}},
	magiclevel=3,
	magicpaths={\blackmagic{}, \anyofthebattlemagic{}\refsymbol{}},
	options={
		\magiclevel{4}=30,
		\cultofyema{}=30,
		\magicalitemsallowance{}=\upto{}<100,
	},
	mounts={
		\elvenhorse{}=20,
		\raptor{}=25,
		\pegasus{}=50,
		\manticore{}=100,
		\dragon{}=300,
	},
	additional={\refsymbol{} \oracleyemanote{}},
}


\heroestitle

\showunit{
	name={\captain},
	cost={75},
	profile={ < 5 6 6 4 3 2 7 3 9},
	type=\infantry{},
	basesize=20x20,
	unitsize=1,
	commontype=\elvencommonrules{},
	commonspecialrules={\killerinstinct{},\lightningreflexes{}},
	armour={\la},
	options={
		\bsb{}=25,	
		\magicalitemsallowance{}=\upto{}<50,
		\onechoiceonly{
			\cultofnabh{}=10,
			\cultofyema{}=10,
			\beastmaster{}=40,
			\fleetcommander{}=40,
		},
		\shield{}=3,
		\ha{}=5,
		\shootingweapononechoice{
			\repeaterhandbow{}=2,		
			\repeatercrossbow{}=4,
		},
		\combatweapononechoice{
			\pw{}=5,
			\gw{}=8,
			\halberd{}=8,
			\lance{}=10,
		},
	},
	mounts={
		\elvenhorse{}=15,
		\raptor{}=25,
		\pegasus{}=60,
		\raptorchariot{}=65,
		\manticore{}=135,
	},
}

\showunit{
	name={\cultpriest},
	cost=80,
	profile={< 5 6 6 4 3 2 7 3 8},
	type=\infantry{},
	basesize=20x20,
	unitsize=1,
	commontype=\elvencommonrules{},
	commonspecialrules={\lightningreflexes{}},
	weapons={\pw},
	additional={%
		\begin{center}\acultpriestmustjoinoneofthefollowingcults{}\spacebeforecolon{}:\end{center}
		\setlength{\columnseprule}{0.5pt}
		\renewcommand{\columnseprulecolor}{\color{black!30}}
		\begin{multicols}{2}\raggedcolumns
		\nahbcultlogotitle{\cultofnabh{}}%
		
		\def\tempspecialrules{\devastatingcharge{}}%
		\specialrules{\tempspecialrules}

		\def\tempoptions{%
			\bsb{}=25,			
			\magicalitemsallowance{}=\upto{}<50,
			\la{}=4,
			\executionersblade{}=15,
			\hspace*{0.3cm}\seeexecutionerspecialunit{}=,		
		}%

		\def\tempmounts{\manticore{}=110,\divinealtarofnabh{}=200}%
		
		\vspace*{0.1cm}\mounts{\tempmounts}
		\options{\tempoptions}
			
		\vspace*{\fill}\columnbreak
		\yemacultlogotitle{\cultofyema{}}

		\def\tempspecialrules{\auraofdespair{}}%
		\specialrules{\tempspecialrules}

		\def\tempoptions{%
			\bsb{}=25,			
			\magicalitemsallowance{}=\upto{}<50,
			\shield{}=3,
			\la{}=4,
			\gladiatorweapons{}=15,
			\hspace*{0.3cm}\seedancersofyemaspecialunit{}=,	
		}%

		\def\tempmounts{\elvenhorse{}=15,\raptor{}=20,\pegasus{}=55,\divinealtarofyema{}=215}%
		
		\vspace*{0.1cm}\mounts{\tempmounts}
		\options{\tempoptions}
		
		\vspace*{\fill}\end{multicols}
		\setlength{\columnseprule}{0pt}
	},
}

\showunit{
	name={\oracle},
	cost={70},
	profile={ < 5 4 4 3 3 2 5 1 8},
	type=\infantry{},
	basesize=20x20,
	unitsize=1,
	commontype=\elvencommonrules{},
	commonspecialrules={\killerinstinct{},\lightningreflexes{}},
	specialrules={\masterofthedarkarts{}},
	magiclevel=1,
	magicpaths={\blackmagic{}, \anyofthebattlemagic{}\refsymbol{}},
	options={
		\magiclevel{2}=25,
		\cultofyema{}=20,
		\magicalitemsallowance{}=\upto{}<50,
	},
	mounts={
		\elvenhorse{}=15,
		\raptor{}=20,
		\pegasus{}=35,
	},
	additional={\refsymbol{} \oracleyemanote{}},
}

\showunit{
	name={\assassin},
	cost={75},
	profile={ < 6 7 7 4 3 2 9 3 9},
	type=\infantry{},
	basesize=20x20,
	unitsize=1,
	commontype=\elvencommonrules{},
	commonspecialrules={\killerinstinct{},\lightningreflexes{}},
	unitrules={\unitrule{\professionalcourtesy}{\professionalcourtesyrule}},	
	options={
		\cultofnabh{}=15,
		\magicalitemsallowance{} \notmagicalarmour{}=\upto{}<50,
		\pw{}=6,
	},
	additional={
		\def\tempspecialrules{\scout{},\armourpiercing{1},\poisonedattacks{},\notaleader{},\hidden{},\masterpoisoner{},\professionalcourtesy{}}
		\vspace*{0.2cm}\specialrules{\tempspecialrules}
		
		\setlength{\columnseprule}{0.5pt}
		\renewcommand{\columnseprulecolor}{\color{black!30}}
		\begin{center}\assassinforkpaths{}\end{center}
		\begin{multicols}{2}\raggedcolumns
		\begin{center}\Largefontsize{\antiquefont\pathofbloodymurder{} (\free{})}\end{center}%

		\def\tempoptions{%
			\distracting{}=25,			
			\wardsave{4}\refsymbol{}=25,
		}%
		
		\options{\tempoptions}
		\noindent\refsymbol{} \canonlybeusedagainstclosecombatattacks{}.
			
		\vspace*{\fill}\columnbreak
		\begin{center}\Largefontsize{\antiquefont\pathofsilentdeath{} (\pts{20})}\end{center}

		\def\tempweapons{\assassinthrowingweapons{}}%
		\weapons{\tempweapons}

		\def\tempequipmentrules{\equipmentdef{\assassinthrowingweapons}{\assassinthrowingweaponsrule}}
		\unitequipment{\tempequipmentrules}
		
		\vspace*{\fill}\end{multicols}
		\setlength{\columnseprule}{0pt}

		\def\tempunitrules{%
			\unitrule{\professionalcourtesy}{\professionalcourtesyrule}		
			\unitrule{\masterpoisoner}{\masterpoisonerrule}
			\unitrule{\wolfsbane{} (\pts{20})}{\wolfsbanerule}
			\unitrule{\bloodroot{} (\pts{20})}{\bloodrootrule}
			\unitrule{\nightshade{} (\pts{40})}{\nightshaderule}			
		}
		\vspace*{0.2cm}\unitrules{\tempunitrules}
	},
}

\mountstitle

\showunit{
	name={\elvenhorse},
	profile={< 9 3 - 3 3 1 4 1 3},
	type=\warbeast{},
	basesize=25x50,
	armour={\mountsprotection{6}},
	options={%
		\mountsprotection{5}=10,
		\lighttroops{}\refsymbol{}=25,
	},
	additional={\refsymbol{} \elvenhorsenote{}},
}

\showunit{
	name={\raptor},
	profile={< 7 3 - 4 4 1 2 2 5},
	type=\warbeast{},
	basesize=25x50,
	armour={\mountsprotection{5}},
	specialrules={\stupidity},
}

\showunit{
	name={\pegasus},
	profile={< 7 4 - 4 4 3 4 2 6},
	type=\monstrousbeast{},
	basesize=40x40,
	armour={\mountsprotection{6}},
	options={%
		\thunderouscharge{}=10,
		\barding{}=20,
	},
	specialrules={\fly{8}},
}

\showunit{
	name={\manticore},
	profile={< 6 5 - 5 5 4 5 3 5},
	type=\monstrousbeast{},
	basesize=50x100,
	additional={%
		\def\tempspecialrules{\fly{8},\largetarget{},\fear{},\lethalstrike{},\frenzy{},\multiplewounds{1D3}{}}
		\def\tempoptions{\alphapredator{} \beastmastersmountonly{}=10,}
		\vspace*{-0.2cm}\specialrules{\tempspecialrules}
		\vspace*{0.2cm}\options{\tempoptions}
	},
}

\showunit{
	notinQRS=yes,
	name={\raptorchariot},
	profile={%
		\chariot{}<- - - 5 5 4 - - -,
		\crew{} (2)< - 5 4 4 - - 6 1 9,
		\raptor{} (2)<7 3 - 4 - - 2 2 5,
	},
	type=\chariot{},
	basesize=50x100,
	commontype=\elvencommonrules{},
	commonspecialrules={\killerinstinct{} \only{\crew},\lightningreflexes{} \only{\crew}},
	specialrules={\stupidity{},\impacthits{+1}},
	weapons={\lance{},\repeatercrossbow{} \only{\crew}},
	armour={\ha{}, \mountsprotection{5}},
}

\showunit{
	notinQRS=yes,
	name={\divinealtar},
	profile={%
		\chariot{}< 8 - - 5 5 5 - - -,
		[\discipleofnabh{} (3)]< - 5 4 3 - - 5 1 8,
		[\discipleofyema{} (2)]< - 4 4 3 - - 5 1 8,
		[\cultofyemamedusa{} (1)]< - 5 4 4 - - 5 5 8,
	},
	type=\chariot{},
	basesize=60x100,
	armour={\la{},\mountsprotection{6}},
	commontype=\elvencommonrules{},
	commonspecialrules={\lightningreflexes{} \only{\disciples}},	
	additional={%
		\def\tempspecialrules{\largetarget{},\fear{},\impacthits{+1},\wardsave{4},\divineblessings{} \seedivinealtarspecialunit{}}
		\vspace*{0.22cm}\specialrules{\tempspecialrules}
		
		\setlength{\columnseprule}{0.5pt}
		\setlength{\columnsep}{1cm}
		\renewcommand{\columnseprulecolor}{\color{black!30}}
		\begin{center}\divinealtaralignment{}\end{center}
		\begin{multicols}{2}\raggedcolumns
		\nahbcultlogotitle{\altarofnabh{}}

		\def\tempweapons{\pw{} (\disciples{})}
		\def\tempspecialrules{\magicresistance{1},\devastatingcharge{} \only{\disciples},\poisonedattacks{} \only{\disciples}}
		\def\tempalignment{\cultofnabh}
		{\setlength{\parskip}{0.3cm}
			\alignment{\tempalignment}
			
			\weapons{\tempweapons}
			
			\specialrules{\tempspecialrules}
		}

		\yemacultlogotitle{\altarofyema{}}
		
		\def\tempweapons{\lance{} \only{\disciples}, \halberd{} \only{\medusa}}
		\def\tempspecialrules{\petrifyingstare{} \only{\medusa}}
		\def\tempalignment{\cultofyema}
		{\setlength{\parskip}{0.3cm}
			\alignment{\tempalignment}
			
			\weapons{\tempweapons}
			
			\specialrules{\tempspecialrules}
		}
		\end{multicols}
		\setlength{\columnseprule}{0pt}	
	},
}

\showunit{
	name={\dragon},
	profile={< 6 5 1 6 6 6 3 5 9},
	type=\monster{},
	basesize=50x100,
	unitsize=SPECIAL-{\textbf{(\oneofakind)}},
	armour={\innatedefence{3}},
	options={%
		\alphapredator{} \beastmastersmountonly{}=25,
	},
	specialrules={\fly{7},\breathweapon{\Strength{} 4, \flamingattacks{}}},
}



\coreunitstitle

\showunit{
	name={\dreadlegionnaires},
	QRSname={\dreadlegionnaire},
	cost={90},
	profile={< 5 4 4 3 3 1 5 1 8},
	type=\infantry{},
	basesize=20x20,
	unitsize=15,
	maxmodels=50,
	costpermodel=8,
	commontype=\elvencommonrules{},
	commonspecialrules={\killerinstinct{},\lightningreflexes{}},
	armour={\la{}, \shield{}},
	options={
		\spear{}=\permodel{}<1,
		\ha{}=\permodel{}<2,
	},
	commandgroup={champion=10, musician=10, banner=10, veteranstandardbearer=yessir},
}

\showunit{
	name={\repeaterauxiliaries},
	QRSname={\repeaterauxiliary},
	cost={110},
	profile={< 5 4 4 3 3 1 5 1 8},
	type=\infantry{},
	basesize=20x20,
	unitsize=10,
	maxmodels=30,
	costpermodel=10,
	commontype=\elvencommonrules{},
	commonspecialrules={\killerinstinct{},\lightningreflexes{}},
	weapons={\repeatercrossbow},
	armour={\la},
	options={
		\shield{}=\permodel{}<1,
	},
	commandgroup={champion=10, musician=10, banner=10, veteranstandardbearer=yessir},
}

\showunit{
	name={\corsairs},
	QRSname={\corsair},
	cost={80},
	profile={< 5 4 4 3 3 1 5 1 8},
	type=\infantry{},
	basesize=20x20,
	unitsize=10,
	maxmodels=35,
	costpermodel=10,
	commontype=\elvencommonrules{},
	commonspecialrules={\killerinstinct{},\lightningreflexes{}},
	armour={\la{}, \innatedefence{5}},
	options={
		\pw{}=\permodel{}<1,
		\repeaterhandbow{}=\permodel{}<1,
		\vanguard{}\refsymbol{}=\permodel{}<1,
	},
	commandgroup={champion=10, musician=10, banner=10, veteranstandardbearer=yessir},
	additional={%
		\vspace*{0.1cm}\refsymbol{} \corsairsvanguardnote{}.
	}
}

\showunit{
	name={\darkraiders},
	QRSname={\darkraider},
	cost={85},
	profile={%
		\rider{}< 5 4 4 3 3 1 5 1 8,
		\elvenhorse{}<9 3 - 3 3 1 4 1 3,
	},
	type=\cavalry{},
	basesize=25x50,
	unitsize=5,
	maxmodels=15,
	costpermodel=15,
	commontype=\elvencommonrules{},
	commonspecialrules={\killerinstinct{} \only{\rider},\lightningreflexes{} \only{\rider}},
	specialrules={\fastcavalry},
	weapons={\lightlance},
	armour={\la{}, \mountsprotection{6}},
	options={
		\shield{}=\permodel{}<3,
		\repeatercrossbow{}=\permodel{}<3,
	},
	commandgroup={champion=10, musician=10, banner=10},
}

\showunit{
	name={\bladesofnabh},
	QRSname={\bladeofnabh},
	cost={130},
	profile={< 5 4 4 3 3 1 5 1 8},
	type=\infantry{},
	basesize=20x20,
	unitsize=10,
	maxmodels=30,
	costpermodel=12,
	alignment=\cultofnabh{},
	alliancepic=cultofnabh,
	commontype=\elvencommonrules{},
	commonspecialrules={\lightningreflexes{}},
	specialrules={\poisonedattacks{},\frenzy{}},
	weapons={\pw},
	commandgroup={champion=10, musician=10, banner=10, veteranstandardbearer=yessir},
}



\specialunitstitle

\showunit{
	name={\towerguard},
	cost={110},
	profile={< 5 5 4 3 3 1 6 2 9},
	type=\infantry{},
	basesize=20x20,
	unitsize=10,
	maxmodels=30,
	costpermodel=15,
	commontype=\elvencommonrules{},
	commonspecialrules={\killerinstinct{},\lightningreflexes{}},
	specialrules={\immunetopsychology{},\bodyguard{},\armourpiercing{1}},
	weapons={\halberd},
	armour={\ha},
	options={\dreadguardians{}=\permodel{}<3},
	commandgroup={champion=10, musician=10, banner=10, bannerallowance=50, championallowance=25},
	unitrules={\unitrule{\dreadguardians}{\dreadguardiansrule}},
}

\showunit{
	name={\dreadknights},
	QRSname={\dreadknight},
	cost={130},
	profile={%
		\rider{}< 5 5 4 4 3 1 6 1 9,
		\raptor{}<7 3 - 4 4 1 2 2 5,
	},
	type=\cavalry{},
	basesize=25x50,
	unitsize=5,
	maxmodels=12,
	costpermodel=26,
	commontype=\elvencommonrules{},
	commonspecialrules={\killerinstinct{} \only{\rider},\lightningreflexes{} \only{\rider}},
	specialrules={\stupidity},
	weapons={\lance},
	armour={\shield{}, \ha{}, \mountsprotection{5}},
	commandgroup={champion=10, musician=10, banner=10, bannerallowance=50, championallowance=25},
}

\showunit{
	name={\medusa},
	cost={70},
	profile={< 6 5 4 4 4 3 5 5 8},
	type=\monstrousinfantry{},
	basesize=40x40,
	unitsize=1,
	alignment=\cultofyema{},
	alliancepic=cultofyema,
	specialrules={\distracting{},\swiftstride{},\petrifyingstare{},\fear{}},
	options={%
		\combatweapononechoice{
			\spear{}=5,			
			\pw{}=5,
			\halberd{}=7,
		},
	},	
}

\showunit{
	name={\dancersofyema},
	QRSname={\dancerofyema},
	cost={90},
	profile={< 5 5 4 3 3 1 5 1 8},
	type=\infantry{},
	basesize=20x20,
	unitsize=10,
	maxmodels=30,
	costpermodel=14,
	alignment=\cultofyema{},
	alliancepic=cultofyema,
	commontype=\elvencommonrules{},
	commonspecialrules={\lightningreflexes{}},
	specialrules={\wardsave{4} (\canonlybeusedagainstclosecombatattacks)},
	weapons={\gladiatorweapons},
	armour={\la{}, \shield{}},
	options={%
		\skirmisher{} \maxmodelsoneofakind{}=\permodel{}<2,
	},
	commandgroup={champion=10, musician=10, banner=10, bannerallowance=50},
	unitequipment={\equipmentdef{\gladiatorweapons}{\gladiatorweaponsrule}},
}

\showunit{
	name={\executioners},
	QRSname={\executioner},
	cost={120},
	profile={< 5 5 4 4 3 1 5 1 8},
	type=\infantry{},
	basesize=20x20,
	unitsize=10,
	maxmodels=30,
	costpermodel=15,
	alignment=\cultofnabh{},
	alliancepic=cultofnabh,
	commontype=\elvencommonrules{},
	commonspecialrules={\lightningreflexes{}},
	weapons={\executionersblade},
	armour={\ha},
	commandgroup={champion=10, musician=10, banner=10, bannerallowance=50},
	unitequipment={\equipmentdef{\executionersblade}{\executionersbladerule}},
}

\showunit{
	name={\harpies},
	QRSname={\harpie},
	cost={65},
	profile={< 5 3 - 3 3 1 5 2 6},
	type=\infantry{},
	basesize=20x20,
	unitsize=5,
	maxmodels=15,
	costpermodel=9,
	specialrules={\insignificant{},\skirmisher{},\fly{10}},
}

\showunit{
	name={\raptorchariot},
	cost={100},
	profile={%
		\chariot{}<- - - 5 5 4 - - -,
		\crew{} (2)< - 5 4 4 - - 6 1 9,
		\raptor{} (2)<7 3 - 4 - - 2 2 5,
	},
	type=\chariot{},
	basesize=50x100,
	unitsize=1,
	commontype=\elvencommonrules{},
	commonspecialrules={\killerinstinct{} \only{\crew},\lightningreflexes{} \only{\crew}},
	specialrules={\stupidity{},\impacthits{+1}},
	weapons={\lance{},\repeatercrossbow{} \only{\crew}},
	armour={\ha{}, \mountsprotection{5}},
}

\showunit{
	name={\divinealtar},
	cost={200},
	profile={%
		\chariot{}< 8 - - 5 5 5 - - -,
		[\discipleofnabh{} (3)]< - 5 4 3 - - 5 1 8,
		[\discipleofyema{} (2)]< - 4 4 3 - - 5 1 8,
		[\cultofyemamedusa{} (1)]< - 5 4 4 - - 5 5 8,
	},
	type=\chariot{},
	basesize=60x100,
	unitsize=SPECIAL-{\textbf{(\oneofakind)} \labels@Singlemodel},
	armour={\la{},\mountsprotection{6}},
	commontype=\elvencommonrules{},
	commonspecialrules={\lightningreflexes{} \only{\disciples}},	
	additional={%
		\def\tempspecialrules{\largetarget{},\fear{},\impacthits{+1},\wardsave{4},\divineblessings{}}
		\vspace*{0.3cm}\specialrules{\tempspecialrules}
	
		\def\tempunitrules{\unitrule{\divineblessings}{\divineblessingsrule}}
		\vspace*{0.2cm}\unitrules{\tempunitrules}
		
		\setlength{\columnseprule}{0.5pt}
		\setlength{\columnsep}{1cm}
		\renewcommand{\columnseprulecolor}{\color{black!30}}
		\begin{center}\divinealtaralignment{}\end{center}
		\begin{multicols}{2}\raggedcolumns
		\nahbcultlogotitle{\altarofnabh{} (\free{})}%

		\def\tempweapons{\pw{} (\disciples{})}
		\def\tempspecialrules{\magicresistance{1},\devastatingcharge{} \only{\disciples},\poisonedattacks{} \only{\disciples}}
		\def\tempalignment{\cultofnabh}
		{\setlength{\parskip}{0.3cm}
			\alignment{\tempalignment}
			
			\weapons{\tempweapons}
			
			\specialrules{\tempspecialrules}
		}

		\yemacultlogotitle{\altarofyema{} (\pts{15})}
		
		\def\tempweapons{\lance{} \only{\disciples}, \halberd{} \only{\medusa}}
		\def\tempspecialrules{\petrifyingstare{} \only{\medusa}}
		\def\tempalignment{\cultofyema}
		{\setlength{\parskip}{0.3cm}
			\alignment{\tempalignment}
			
			\weapons{\tempweapons}
			
			\specialrules{\tempspecialrules}
		}
		\end{multicols}
		\setlength{\columnseprule}{0pt}	
	},
}




\rareunitstitle

\showunit{
	name={\ravencloaks},
	QRSname={\ravencloak},
	cost={80},
	profile={< 5 5 5 3 3 1 5 1 8},
	type=\infantry{},
	basesize=20x20,
	unitsize=5,
	maxmodels=10,
	costpermodel=16,
	commontype=\elvencommonrules{},
	commonspecialrules={\killerinstinct{},\lightningreflexes{}},
	specialrules={\scout{},\skirmisher},
	weapons={\repeatercrossbow},
	options={%
		\la{}=\permodel{}<1,
		\weapononechoice{
			\pw{}=\permodel{}<1,
			\gw{}=\permodel{}<2,
		},
		\poisonedattacks{} (\closecombatonly{})=\permodel{}<1,
	},
	commandgroup={champion=10},
}

\showunit{
	name={\darkacolytes},
	QRSname={\darkacolyte},
	cost={120},
	profile={%
		\rider{}< 5 4 4 4 3 1 5 2 8,
		\elvenhorse{}<9 3 - 3 3 1 4 1 3,
	},
	type=\cavalry{},
	basesize=25x50,
	unitsize=5,
	maxmodels=10,
	costpermodel=24,
	commontype=\elvencommonrules{},
	commonspecialrules={\killerinstinct{} \only{\rider},\lightningreflexes{} \only{\rider}},
	specialrules={\lighttroops{},\wardsave{4},\poisonedattacks{} \only{\rider}},
	armour={\mountsprotection{6}},
	options={
		\cultofyema{}=\permodel{}<3,
	},
	commandgroup={champion=60},
	wizardconclave={\deathspellone{} (\Pathof{} \death{}), \blackmagicspellfive{} (\Pathof{} \blackmagic{})\refsymbol{}},
	additional={%
		\vspace*{0.1cm}\noindent\refsymbol{} \darkacolytesnote{}
	},
}

\showunit{
	name={\kraken},
	cost={165},
	profile={< 6 4 1 7 5 5 3 4 6},
	type=\monster{},
	basesize=50x100,
	unitsize=1,
	armour={\innatedefence{4}},
	options={%
		\alphapredator{}=45,
	},
	additional={%
		\def\tempspecialrules{\distracting{},\poisonedattacks{},\multiplewounds{1D3}{},\strider{\water},\hardtarget{}}
		\specialrules{\tempspecialrules}
	},
}

\showunit{
	name={\hydra},
	cost={180},
	profile={< 6 4 1 5 5 5 2 7 6},
	type=\monster{},
	basesize=50x100,
	unitsize=1,
	specialrules={\regeneration{4}},
	armour={\innatedefence{4}},
	options={%
		\breathweapon{\Strength{} 4, \flamingattacks{}}=30,
		\alphapredator{}=45,		
	},
}

\showunit{
	name={\huntingchariot},
	cost={90},
	profile={%
		\chariot{}<- - - 5 4 4 - - -,
		\crew{} (2)< - 4 4 3 - - 5 1 8,
		\elvenhorse{} (2)<9 3 - 3 - - 4 1 3,
	},
	type=\chariot{},
	basesize=50x100,
	unitsize=1,
	commontype=\elvencommonrules{},
	commonspecialrules={\killerinstinct{} \only{\crew},\lightningreflexes{} \only{\crew}},
	specialrules={\impacthits{+1}},
	weapons={\lightlance{},\repeatercrossbow{} \only{\crew}},
	armour={\la{}, \mountsprotection{5}},
	additional={%
		\setlength{\columnseprule}{0.5pt}
		\setlength{\columnsep}{1cm}
		\renewcommand{\columnseprulecolor}{\color{black!30}}
		\begin{center}\huntingchariotforkchoice{}\end{center}
		\begin{multicols}{2}\raggedcolumns
		\begin{center}\Largefontsize{\antiquefont\giantbow{} (\free{})}\end{center}%

		\noindent\giantbowrule{}

		\columnbreak
		\begin{center}\Largefontsize{\antiquefont\harpoonlauncher{} (\pts{35})}\end{center}

		\noindent\harpoonlauncherrule{}
		
		\vspace*{\fill}
		\end{multicols}
		\setlength{\columnseprule}{0pt}		
	},
}

\showunit{
	name={\dreadreaper},
	cost=60,
	profile={%
		\boltthrower{}<- - - - 7 2 - - -,
		\crew{} (2)< 5 4 4 3 3 - 5 1 8,
	},
	type={\warmachine},
	unitsize=SPECIAL-{\textbf{(\zerotoXchoice{3})} \labels@Singlemodel},
	basesize=60,
	commontype=\elvencommonrules{},
	commonspecialrules={\killerinstinct{} \only{\crew},\lightningreflexes{} \only{\crew}},
	weapons={\elvenboltthrower},
	armour={\la},
	options={\repeatingshots{}=20,},
	unitequipment={\equipmentdef{\elvenboltthrower}{\elvenboltthrowerrule}},
	unitrules={\unitrule{\repeatingshots}{\repeatingshotsrule}},
}



%%% Quick Reference Sheet - AB_qrs.tex is automatic and shouldn't be edited %%%

\quickrefsheettitle

\input{../Layout/AB_qrs.tex}
\bigskip
\begin{center}
\noindent{\antiquefont\Largefontsize\textbf{Armes de Tir des Elfes des Ténèbres}}
\medskip

\rowcolors{1}{white}{black!10}
\noindent\begin{tabular}{lcccccc}
\textbf{Nom} & \textbf{Artillerie} & \textbf{Portée} & \textbf{\labels@S{}} & \textbf{\multipleshots{}} & \textbf{\multiplewounds{}} & \textbf{\armourpiercing{}} \tabularnewline
\repeatercrossbow{} & - & \distance{24} & 3 & 2 & - & 1 \tabularnewline
\petrifyingstare{} & - & \distance{12} & 4 & 2 & - & 6 \tabularnewline
\dreadreaper{} & \boltthrower{} & \distance{48} & 6 & - & 1D3 & 6 \tabularnewline
\dreadreaper{} (\repeatingshots{}) & \volleygun{} & \distance{48} & 4 & 6 & - & 1 \tabularnewline
\giantbow{} (\huntingchariot{}) & \boltthrower{} & \distance{24} & 5 & - & 1D3 & 6 \tabularnewline
\harpoonlauncher{} (\huntingchariot{}) & - & \distance{24} & 7 & - & 1D3 & 1 \tabularnewline
\end{tabular}
\end{center}

\restoregeometry

\changelogtitle

\startchangelog

\newlog{0.99.0}{%
Fleet commander,
Beastmaster’s Lash,
Midnight Cloak,
Dagger of Moraec,
Dread Prince and cults,
Captain and cults,
Blades of Nabh,
Executioners,
Raven Cloaks,
Medusas,
Kraken,
Hunting Chariot,
Dread Reaper,
raven cloaks,
Dancers of Yema,
Harpies,
Pegasus barding option,
}

\endchangelog

\end{document}

