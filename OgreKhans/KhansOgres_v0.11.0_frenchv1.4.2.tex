\input{../Formatage/format_LA.tex}

\begin{document}

\newcommand{\scraplinglookout}{\specialrule{Guetteur \newrule{Scrapulin}}}
\newcommand{\meatavalanche}{\specialrule{Avalanche de Viande}\xspace}
\newcommand{\leaderofthepack}{\specialrule{\newrule{Chef de Meute}}\xspace}
\newcommand{\backtowork}{\specialrule{Au Boulot !}\xspace}
\newcommand{\battlescarred}{\specialrule{\newrule{Balafres de Guerre}}\xspace}
\newcommand{\wildbeasts}{\specialrule{Bêtes Sauvages}\xspace}
\newcommand{\touchoffrost}{\specialrule{\newrule{Toucher de Givre}}\xspace}
\newcommand{\itsatrap}{\specialrule{\newrule{C'est un Piège !}}\xspace}
\newcommand{\rockskeleton}{\specialrule{Ossature de Granit}\xspace}
\newcommand{\frozenaura}{\specialrule{Aura Glaciale}\xspace}
\newcommand{\dowhatyouretold}{\specialrule{Fais ce qu'on te dis}\xspace}
\newcommand{\loner}{\specialrule{Solitaire}\xspace}

\newcommand{\booktitle}{Khans Ogres}
\newcommand{\version}{0.11.0}
\newcommand{\frenchversion}{1.4.2}
\newcommand{\translationteam}{\item \og Gandarin \fg \item \og Iluvatar \fg \item \og Lamronchak \fg \item \og Keinach \fg \item \og AEnoriel \fg \item \og Eru \fg \item \og Groumbahk \fg \item \og Mammstein \fg}
\newcommand{\separator}{\medskip\noindent\textcolor{black!0}{\rule{3cm}{1pt}}}

\input{../Formatage/titlepage_LA.tex}

\armyspecialrules

\armyspecialruleentry{\backtowork}

Les Chefs \newrule{Scrapulins} bénéficient de la règle \holdyourground bien qu'ils ne soient pas Porteur de la Grande Bannière. Toutefois, seuls les \newrule{Scrapulins}, les Lances-débris et les Braconniers \newrule{Scrapulins} peuvent bénéficier du \holdyourground prodigué par un Chef \newrule{Scrapulin}.



\armyspecialruleentry{\frozenaura}

Toutes les unités ennemies dans un rayon de \distance{6} de l'\newrule{Mammouth de Givre} voient leur Initiative réduite de 3 jusqu'à un minimum de 1. De plus, la figurine dispose d'un objet de sort de Puissance 3 qui permet de lancer le sort Aquilon (Discipline \heavens).

\armyspecialruleentry{\meatavalanche}

Les figurines bénéficient de la règle spéciale \impacthits{1} si ce sont des figurines ordinaires et \impacthits{1D3} si ce sont des Personnages.

\armyspecialruleentry{\battlescarred}

Chaque unité de \newrule{Balafrés} peut choisir deux règles spéciales parmi les suivantes : \poisonedattacks , \vanguard , \thunderouscharge , \lethalstrike , \bodyguard{}, \strider{}, \immunetopsychology , \swiftstride. Chaque règle spéciale ne peut être prise qu'une seule fois dans une même armée par des unités de \newrule{Balafrés}.

\armyspecialruleentry{\wildbeasts}

Une unité qui suit cette règle spéciale ne peut jamais bénéficier de la \inspiringpresence du Général et du \holdyourground du Porteur de la Grande Bannière, mais peut utiliser le Commandement d'un \newrule{Chasseur de Mastodonte} dans un rayon de \distance{12} comme si ce dernier était le Général de l'armée.

\armyspecialruleentry{\itsatrap}

Tout Champ, Forêt ou Ruine devient un Terrain Dangereux pour le reste de la partie si, au moment du déploiement ou à la fin d'une Phase de Mouvement, une unité de Braconniers \newrule{Scrapulins} est en contact avec l'élément de terrain concerné (sauf si l'unité est en fuite). Les figurines qui traitent normalement ces éléments de terrain comme des Terrains Dangereux souffrent d'un malus de -1 à leurs tests de Terrain Dangereux. Les Braconniers \newrule{Scrapulins} n'ont pas à faire de test de Terrain Dangereux lorsqu'ils entrent ou sortent d'un élément de terrain qu'ils ont piégé.

\armyspecialruleentry{\leaderofthepack}

Si un \newrule{Chasseur de Mastodonte} est déployé dans une unité de Tigres à Dents de Sabre et seulement s'il fait partie intégrante de celle-ci, l'unité perd la règle spéciale \insignificant mais gagne la règle spéciale \vanguard . De plus, le \newrule{Chasseur de Mastodonte} compte comme étant une Bête de Guerre de manière à ce que les touches normales et les touches résultant de l'application d'un gabarit soient attribuées à l'Unité Combinée.

\armyspecialruleentry{\dowhatyouretold}

Tous les Géants Asservis peuvent relancer leur premier jet d'Attaques de Géant, mais doivent accepter le second résultat.

\armyspecialruleentry{\scraplinglookout}

Les unités avec cette règle spéciale et disposant d'un Porte-étendard doivent comprendre 3 figurines ordinaires ou moins avant que les touches ne puissent être réparties sur les Personnages ayant le même type de troupe que l'unité.

\armyspecialruleentry{\rockskeleton}

Si la figurine bénéficiant de cette règle spéciale subit une blessure assujettie à la règle spéciale \multiplewounds{}{}, le nombre de Points de Vie enlevés à la figurine est divisé par deux et arrondi au supérieur.

\armyspecialruleentry{\loner}

Les \newrule{Chasseurs de Mastodonte} à pied peuvent exclusivement rejoindre des unités de Yétis ou de Tigres à Dents de Sabre. Les \newrule{Chasseurs de Mastodonte} montés ne peuvent rejoindre aucune unité.

\armyspecialruleentry{\touchoffrost}

Toutes les unités ennemies en contact avec une figurine avec cette règle spéciale ont -1 en Initiative jusqu'à un minimum de 1.

\armyarmoury

\armynewsubsection{Armes de corps à corps}

\begin{customdescription}
\item[Poing de fer :] \requirestwohands . C'est une arme de corps à corps qui ajoute +1 à la Sauvegarde d'Armure de la figurine et compte comme une Arme de base additionnelle (également applicable aux figurines montées).
\end{customdescription}

\armynewsubsection{Armes de tir}

\begin{customdescription}
	\item[Arbalète ogre :] \portee{30}, Force 5, pénètre les rangs comme une Baliste.
	\item[Canon à la hanche :]\portee{24}, Force 4, \newrule{\cumbersome}, \armourpiercing{1}, \multipleshots{1D6}. Le Canon à la hanche ne subit pas de pénalité pour toucher due à un Mouvement ou aux \multipleshots{}. Au corps à corps, le Canon à la hanche compte comme une Hallebarde avec Initiative 0.
	\item[Canon Titan :] Cette arme peut tirer de deux manières :
		\begin{customsubitemize}
			\item[-] Comme une Arme d'artillerie du type Canon (\distance{2D6}) avec le profil suivant : \portee{36}, Force 10, \multiplewounds{\ordnance}{}, \armourpiercing{2}.
			\item[-] Comme une Arme d'artillerie du type Batterie de tir avec le profil suivant : \portee{12}, Force 5, \armourpiercing{2}, \multipleshots{2D6}.
		\end{customsubitemize}
	\item[Lance-débris :] C'est une Arme d'artillerie du type Catapulte (\distance{5}) avec le profil suivant : \portee{36}, Force 3, \lethalstrike .
	\item[Lance de chasse :] \portee{12}, Force du lanceur +1, \lethalstrike, \quicktofire .	
	\item[Paire de pistolets ogres :] \portee{24}, Force 4, \armourpiercing{1}, \multipleshots{2},\quicktofire, compte comme une Arme de base additionnelle au corps à corps.
	\item[Pistolet ogre :] \portee{24}, Force 4, \armourpiercing{1}, \quicktofire,  compte comme une Arme de base additionnelle au corps à corps.
\end{customdescription}

\armynewsection{Grands noms}

Chaque Personnage Ogre peut prendre un seul Grand Nom. Deux Personnages ne peuvent pas prendre le même Grand Nom.

\begin{customitemize}
	\item \optiondef{\newrule{Cœur Sauvage}}{50}{Seuls les \newrule{Chasseurs de Mastodonte} peuvent choisir ce Grand Nom. Un \newrule{Chasseur de Mastodonte} avec ce Grand Nom peut être promu Général de l'armée. Un autre \newrule{Chasseur de Mastodonte} dans l'armée peut être promu Porteur de la Grande Bannière pour 25 pts. Une unité de Yétis et une seule unité de Tigres à Dents de Sabre peuvent être prises comme unités de base. Votre armée ne doit pas inclure de \newrule{Grand Khan}, de \newrule{Khan}, de \newrule{Chaman avec la Bénédiction du Feu}, de \newrule{Grand Chaman avec la Bénédiction Suprême du Feu}, de \newrule{Brise-Crâne}, de \newrule{Balafrés}, de Gueules d'acier ou de Crache-Tonnerre.}
	\item \optiondef{\newrule{Poing de l'Enfer}}{50}{Seul un \newrule{Grand Khan} peut prendre ce Grand Nom. Un personnage avec ce Grand Nom prend possession d'une arme magique de type Poing de fer. Toute figurine dans l'armée peut prendre une Marque des Dieux Sombres (voir le Livre d'Armée des Guerriers des Dieux Sombres). Coût des Marques :
	\begin{customsubitemize}
		\item \unit{5}{pts} par figurine d'Infanterie Monstrueuse,
		\item \unit{10}{pts} par figurine de Cavalerie Monstrueuse ou de Char,
		\item \unit{15}{pts} par figurine de Monstre ou de Personnage,
		\item Aucun coût pour la Marque du \newrule{Chaos Intégral}.
	\end{customsubitemize}
	Les \newrule{Chamans} et les \newrule{Grands Chamans} avec une Marque autre que celle du \newrule{Chaos Intégral} peuvent choisir une Discipline à laquelle les Sorciers munis de ces Marques peuvent accéder. Ils peuvent également choisir une des Disciplines auxquelles ils ont normalement accès de par leur statut de Chaman Ogre ; cela inclut les Chamans qui ont reçu la \newrule{Bénédiction du Feu} ou la \newrule{Bénédiction Suprême du Feu}. Les \newrule{Chamans} et les \newrule{Grand Chamans} ne peuvent pas choisir la Marque du Courroux. \newline
	Si votre armée porte des Marques, elle ne peut pas inclure de \newrule{Chasseur de Mastodonte}, de Yéti, de \newrule{Scrapulin}, de Braconnier, de Tigre à Dents de Sabre ou de \newrule{Lance-débris}.}	
	\item \optiondef{Mâchoire Pourrie}{30}{Le Personnage qui choisit ce Grand Nom bénéficie de la règle spéciale \poisonedattacks . De plus, au cours d'une seule phase de corps à corps, toutes les figurines de l'unité dans laquelle il se trouve bénéficient également de la règle spéciale \poisonedattacks . Déclarez l'utilisation de cette règle spéciale pour les figurines de l'unité au début de la Phase de corps à corps.}
	\item \optiondef{\newrule{Chasseur de Têtes}}{25}{Le Personnage avec ce Grand Nom se gave de la chair de ses ennemis tués au cours du combat, même au plus fort de la bataille. A la fin de toute Phase de corps à corps dans laquelle le Personnage a tué une ou plusieurs figurines et dans la mesure où il n'est pas en fuite, lancez 1D6. Sur un résultat de 4+, il regagne un Point de Vie qu'il a perdu plus tôt au cours de la bataille.}
	\item \optiondef{Mangeur de Troll}{25}{Le Personnage connu pour son régime peu savoureux gagne les règles spéciales \regeneration{5} et \stupidity . De plus, le Personnage bénéficie de la règle spéciale \multiplewounds{2}{Infanterie Monstrueuse}.}			
	\item \optiondef{\newrule{Maître des Trésors de Guerre}}{20}{Ce Grand Nom ne peut pas être pris par un Sorcier ou un \newrule{Chasseur de Mastodonte}. Le Personnage avec ce Grand Nom gagne la règle spéciale \weaponmaster et une Armure de plates, et peut aussi acquérir autant d'armes ordinaires de corps à corps et de tir auxquelles il peut accéder plutôt qu'une seule de chaque catégorie.}
	\item \optiondef{Pourfendeur de Moelle Épinière}{20}{Ce Grand Nom ne peut pas être pris par un Sorcier. Par ailleurs, seules les figurines à pied peuvent prendre ce Grand Nom. Le Personnage bénéficie des règles spéciales \devastatingcharge et \thunderouscharge (qui affectent aussi les \impacthits{} et le \stomp{}).}		
\end{customitemize}

\armymagicitems

\armynewsubsection{Armes magiques}

\begin{customitemize}
	\item \optiondef{\newrule{Masse de Khagadai}}{50}{Type : Arme lourde. \multiplewounds{1D3}{}. Le porteur peut choisir de faire une \crushattack à la place de ses attaques normales à chaque Phase de Corps à Corps.}
	\item \optiondef{\newrule{Transperce-Cœur}}{30}{Type : Poing de fer. Les attaques réalisées avec cette arme toucheront toujours sur 3+ ou mieux en situation de corps à corps, quels que soient les malus qui peuvent s'appliquer au jet pour toucher. De plus, le porteur bénéficie des règles spéciales \lethalstrike et \newrule{\armourpiercing{1}}.}
	\item \optiondef{\newrule{Gantelet Brise-Hache}}{25}{Type : Poing de fer. Si le porteur inflige une ou plusieurs blessures \newrule{avec cette arme} à une figurine ennemie possédant une arme magique, jetez 1D6. Sur un résultat de 4+, toutes les armes magiques de la figurine ennemie sont détruites.}	
\end{customitemize}

\armynewsubsection{Armures magiques}

\begin{customitemize}
	\item \optiondef{\newrule{Armure en Peau de Mammouth}}{40}{Type : Armure lourde. Cette Armure magique ne peut être prise que par des figurines à pied. Si le porteur est touché par une attaque dont la Force est de 6 ou plus, la Force de cette attaque est réduite à 5.}
	\item \optiondef{\newrule{Fourrure de Yéti}}{15}{Type : Aucun (Sauvegarde d'armure 6+). Les figurines ennemies en contact socle à socle avec le porteur frappent avec -3 en Initiative sans dépasser un minimum de 1.}
\end{customitemize}

\armynewsubsection{Talismans}

\begin{customitemize}
	\item \optiondef{\newrule{Œil Aveugle de Nyanggai}}{25}{Une seule utilisation. Ce Talisman doit être activé au début de la Phase de Magie de l'adversaire. Le porteur de ce Talisman, l'unité dans laquelle il se trouve et tout autre Personnage également dans cette unité ne peuvent être ciblés par les Sorciers ennemis qu'avec des sorts de type Aura.}
\end{customitemize}

\armynewsubsection{Objets enchantés}

\begin{customitemize}
	\item \optiondef{\newrule{Sortilège de l'Aurochs}}{15}{Le porteur de cet Objet enchanté bénéficie de la règle spéciale \rockskeleton (voir l'unité rare \newrule{Aurochs de Pierre}).}
\end{customitemize}

\armynewsubsection{Objets cabalistiques}

\begin{customitemize}
	\item \optiondef{Cœur Démoniaque}{\newrule{50}}{Une seule utilisation. Doit être activé au début d'une des Phases de Magie de l'adversaire. Au cours de cette phase, tous les Sorciers ennemis dans un rayon de \distance{24} subiront un Fiasco lorsqu'ils obtiendront n'importe quel double lors du lancement d'un sort. Seul un double \result{6} compte comme un Pouvoir Irrésistible, et un seul Fiasco peut être subi par lancement. De plus, tous les Fiascos seront considérés comme ayant été obtenus en utilisant un dé de pouvoir supplémentaire par rapport au nombre de dés réellement utilisés. Cela aura deux conséquences ; le lanceur perdra un dé supplémentaire dans sa réserve de dés de pouvoir, et les touches résultant d'un Fiasco seront d'une Force égale au nombre de dés de pouvoir utilisés +3, au lieu de +2.}
\end{customitemize}

\armynewsubsection{Bannières magiques}

\begin{customitemize}
	\item \optiondef{Étendard en Peau de Dragon}{40}{Les figurines de l'unité peuvent relancer les jets pour toucher, pour blesser et de \armoursave ayant donné \result{1} au cours de la première manche de corps à corps de la phase où ils ont chargé. De plus, le porteur de la bannière bénéficie de la règle spéciale \breathweapon{Force 3, \flamingattacks}.}
	\item \optiondef{\newrule{Crâne de Qenghet}}{20}{Les figurines de l'unité brandissant cette Bannière magique bénéficient de la règle spéciale \fear et réussissent automatiquement les tests de \terror .}	
\end{customitemize}	



\armylist

\lordstitle

\showunit{
	name={\newrule{Grand Khan}},
	cost={\newrule{180}},
	profile={Grand Khan : 6 6 4 5 5 5 4 5 9},
	type=Infanterie Monstrueuse,
	unitsize={1},
	basesize=40x40,
	specialrules={\meatavalanche, \fear},
	equipment={Armure lourde},
	options={
		Peut prendre un seul Grand Nom = \unlimited ,
		Peut prendre des Objets magiques = \upto: 100,
		\optionschoice{Peut prendre une Arme de tir}{
			Pistolet ogre = 6,
			Paire de pistolets ogres = 8,
			Arbalète ogre = 8,
		},
		\optionschoice{Peut prendre une Arme de corps à corps}{
			Poing de fer = 15,
			Arme lourde = 20,
		}
	}
}

\showunit{
	name={\newrule{Grand Chaman}},
	cost={\newrule{275}},
	profile={Grand Chaman : 6 4 3 4 5 5 3 4 8},
	type=Infanterie Monstrueuse,
	unitsize={1},
	basesize=40x40,
	specialrules={\meatavalanche},
	magiclevelmaster=3,
	magicpaths={\butchery, \heavens , \death , \wilderness}, 
	options={
		Peut prendre un seul Grand Nom = \unlimited,
		Peut prendre des Objets magiques = \upto: 100,
		\magiclevelmaster{4} = 35,
		Peut \newrule{recevoir la Bénédiction Suprême du Feu} = 35,
		\optionschoice{Peut prendre une Arme de corps à corps}{
		Arme de base additionnelle = 5,
		Arme lourde = 15
		},
	},
	unitrules={\unitrule{\specialrule{\newrule{Bénédiction Suprême du Feu}}}{La figurine bénéficie des règles spéciales \flamingattacks, \breathweapon{Force 4, \flamingattacks} et \fireborn. Un Sorcier qui reçoit la Bénédiction Suprême du Feu doit générer ses sorts dans la Discipline \alchemy ou \fire .}}
}

\heroestitle

\showunit{
	name={\newrule{Khan}},
	cost={105},
	profile={Khan : 6 5 4 5 5 4 3 4 8},
	type=Infanterie Monstrueuse,
	unitsize={1},
	basesize=40x40,
	specialrules={\meatavalanche},
	equipment={Armure lourde},
	options={
	Peut prendre un seul Grand Nom = \unlimited , 
	Peut prendre des Objets magiques = \upto: 50,
	Peut être promu Porteur de la Grande Bannière = 25,
	Peut prendre un \scraplinglookout = 5,
	\optionschoice{Peut prendre une Arme de tir}{
	Pistolet ogre = 6,
	Paire de pistolets ogres = 8,
	Arbalète ogre = 8
	},
	\optionschoice{Peut prendre une Arme de corps à corps}{  
	Poing de fer = 10,	
	Arme lourde = 15
	},
	}
}

\showunit{
	name={\newrule{Chaman}},
	cost={\newrule{105}},
	profile={Chaman : 6 3 3 4 4 4 2 3 7},
	type=Infanterie Monstrueuse,
	unitsize={1},
	basesize=40x40,
	specialrules={\meatavalanche},
	magiclevelapprentice=1,
	magicpaths={\butchery, \wilderness},
	options={
		Peut prendre un seul Grand Nom = \unlimited ,
		Peut prendre des Objets magiques = \upto: 50,
		\magiclevelapprentice{2} = 25,
		Peut \newrule{recevoir la Bénédiction du Feu} = 25,
		\optionschoice{Peut prendre une Arme de corps à corps}{ 
		Arme de base additionnelle = 3,
		Arme lourde = 6
		}, 
	},
	unitrules={\unitrule{\specialrule{\newrule{Bénédiction du Feu}}}{Une figurine qui reçoit la Bénédiction du Feu bénéficie des règles spéciales \flamingattacks, \breathweapon{Force 3, \flamingattacks} et \fireborn . Elle doit générer ses sorts à partir de la Discipline \fire .}}
}

\showunit{
	name={\newrule{Chasseur de Mastodonte}},
	cost={120},
	profile={Chasseur de Mastodonte : 7 4 5 5 5 4 4 4 9},
	type=Infanterie Monstrueuse,
	unitsize={1},
	basesize=50x50,
	specialrules={\meatavalanche , \leaderofthepack , \notaleader , \swiftstride}, 
	equipment={Armure légère, Lance de chasse},
	options={
		Peut prendre un seul Grand Nom = \unlimited, 
		Peut prendre des Objets magiques = \upto: 50,
		Peut remplacer la Lance de chasse par une Arbalète ogre = \free ,
		\optionschoice{Peut prendre une Arme de corps à corps}{ 
		Arme de base additionnelle = 5,
		Poing de fer = 10, 
		Lance de cavalerie = 15, 
		Arme lourde = 15 
		},
		\scout (à pied uniquement) = 10
	},
	mounts={
		\newrule{Razor-Buffle}=60,
		\newrule{Aurochs de Pierre} (\oneofakind)*=250,
		},
	unitrules={
		\unitrule{\leaderofthepack}{Les \newrule{Chasseurs de Mastodonte} à pied peuvent rejoindre des unités de Tigres à Dents de Sabre. Si un \newrule{Chasseur de Mastodonte} est déployé dans une telle unité et seulement s'il fait partie intégrante de celle-ci, l'unité perd la règle spéciale \insignificant mais gagne la règle spéciale \vanguard . De plus, le \newrule{Chasseur de Mastodonte} compte comme étant une Bête de Guerre de manière à ce que les touches normales et les touches résultant de l'application d'un gabarit soient attribuées à l'Unité Combinée.}
		\separator
		\unitrule{\loner}{Les \newrule{Chasseurs de Mastodonte} à pied peuvent exclusivement rejoindre des unités de Yétis ou de Tigres à Dents de Sabre. Les \newrule{Chasseurs de Mastodonte} montés ne peuvent rejoindre aucune unité.}
		\separator
		\unitrule{*}{Un \newrule{Chasseur de Mastodonte} monté sur un \newrule{Aurochs de Pierre} ne peut pas utiliser la règle spéciale \meatavalanche. De plus, il ne peut choisir qu'un seul Grand Nom dans le cas où il souhaite en avoir un : \newrule{Cœur Sauvage}.}
}
}

\baseunitstitle

\showunit{
	name={Guerriers Tribaux},
	cost={\newrule{85}},
	profile={Guerrier Tribal : 6 3 3 4 4 3 2 3 7},
	type=Infanterie Monstrueuse,
	unitsize=3,
	additionalmodels=12,
	costpermodel=28,
	basesize=40x40,
	specialrules={\meatavalanche},
	equipment={\newrule{Arme de base additionnelle}, Armure légère},
	options={
		Peuvent porter une Armure lourde = \permodel: 5,
		\newrule{Peuvent porter un Poing de fer} = \permodel: 2,
		},
	commandgroup={\commandgroup{champion=10, banner=10, banneroption=\scraplinglookout *5, standardbeareroption=\veteranstandardbearer *25, musician=10}},
}

\showunit{
	name={\newrule{Brise-Crâne}},
	cost={120},
	profile={Brise-Crâne : 6 3 3 4 4 3 2 3 8},
	type=Infanterie Monstrueuse,
	unitsize={3},
	additionalmodels=7,
	costpermodel=40,
	basesize=40x40,
	specialrules={\meatavalanche},
	equipment={Armure lourde, Arme lourde},
	commandgroup={\commandgroup{champion=10, banner=10, banneroption=\scraplinglookout *5, standardbeareroption=\veteranstandardbearer *25, musician=10}},	
	}

\showunit{
	name={\newrule{Scrapulins}},
	cost={40},
	profile={Scrapulin : 4 2 3 3 3 1 3 1 6},
	type=Infanterie,
	unitsize={10},
	additionalmodels=50,
	costpermodel=3,
	basesize=20x20,
	specialrules={\insignificant},
	equipment={Armes de jet},
	options={
		\newrule{Peuvent porter une Armure légère} = \permodel:1,
		\newrule{ 
		\optionschoice{Peuvent remplacer les Armes de jet}{
		par un Arc court = \free,
		par une Lance et un Bouclier = \permodel:1,
		}}
	},
	commandgroup={\commandgroup{champion=10, championoption=Peut être promu Chef Scrapulin:20,  banner=10, musician=10}},
}

\showunit{
	name={Chef \newrule{Scrapulin}},
	cost={-},
	profile={Chef Scrapulin : 4 3 4 3 3 2 4 3 7},
	type=Infanterie,
	basesize=20x20,
	specialrules={\backtowork , \insignificant},
	equipment={Armure légère, Armes de jet},
	options={
		\optionschoice{Peut prendre une Arme de corps à corps}{
		Hallebarde=3,
		Arme lourde=3
		}, 
	},
	unitrules={\unitrule{\backtowork}{Les Chefs \newrule{Scrapulins} possèdent la règle \holdyourground bien qu'ils ne soient pas Porteur de la Grande Bannière. Toutefois, seuls les \newrule{Scrapulins}, les \newrule{Lances-débris} et les Braconniers \newrule{Scrapulins} peuvent bénéficier du \holdyourground prodigué par un Chef \newrule{Scrapulin}.}}
}

\specialunitstitle

\showunit{
	name={Gueules d'acier},
	cost={130},
	profile={Gueule d'Acier : 6 3 3 4 4 3 2 3 7},
	type=Infanterie Monstrueuse,
	unitsize={3},
	additionalmodels=5,
	costpermodel=43,
	basesize=40x40,
	specialrules={\meatavalanche},
	equipment={Armure légère, Canon à la hanche},
	commandgroup={\commandgroup{champion=10, banner=10, bannerallowance=25, banneroption=\scraplinglookout *5, musician=10}},
	unitequipment={\equipmentdef{Canon à la hanche}{\portee{24}, Force 4, \newrule{\cumbersome}, \armourpiercing{1}, \multipleshots{1D6}. Le Canon à la hanche ne subit pas de pénalité pour toucher due à un Mouvement ou aux \multipleshots{}. Au corps à corps, le Canon à la hanche compte comme une Hallebarde avec Initiative 0.}}
}

\showunit{
	name={\newrule{Balafrés}},
	cost={150},
	profile={Balafré : 6 4 4 5 4 3 3 4 8},
	type=Infanterie Monstrueuse,
	unitsize={3},
	additionalmodels=5,
	costpermodel=50,
	basesize=40x40,
	specialrules={\meatavalanche , \battlescarred, \weaponmaster},
	equipment={Armure lourde},
	options={
		\optionschoice{Peuvent prendre}{
		Arme de base additionnelle = \permodel: 3,
		Poing de fer = \permodel: 5,
		Pistolet ogre = \permodel: 5,
		Paire de pistolets ogres = \permodel: 8,
		Arme lourde = \permodel: 8
		},
		},
	commandgroup={\commandgroup{champion=10, banner=10, bannerallowance=50, banneroption=\scraplinglookout *5, musician=10}},
	unitrules={\unitrule{\battlescarred}{Chaque unité de \newrule{Balafrés} peut choisir deux règles spéciales parmi les suivantes : \poisonedattacks , \vanguard , \thunderouscharge , \lethalstrike , \bodyguard{}, \strider{}, \immunetopsychology , \swiftstride. Chaque règle spéciale ne peut être prise qu'une seule fois dans une même armée par des unités de \newrule{Balafrés}.}}
}

\showunit{
	name={\newrule{Chevaucheurs de Razor-Buffle}},
	cost={130},
	profile={
				Chevaucheur    :  6 3 3 4 4 3 2 3 8,
				Razor-Buffle  :  8 3 - 5 5 3 2 4 5,
	},
	type=Cavalerie Monstrueuse,
	unitsize={2},
	additionalmodels=3,
	costpermodel=68,
	basesize=50x100,
	specialrules={\fear , \impacthits{1D3}},
	equipment={Armure légère, \mountprotection{5}},
	options={
		Peuvent porter une Armure lourde = \permodel: 10,
		\optionschoice{Peuvent prendre une Arme de corps à corps}{
		Arme lourde = \permodel: 10,
		Poing de fer (seulement avec une armure légère) = \permodel: 12
		},
	},
	commandgroup={\commandgroup{champion=10, banner=10, bannerallowance=50, banneroption=\scraplinglookout *5 , musician=10}},
}

\showunit{
	name={Tigre à Dents de Sabre},
	cost={35},
	profile={Tigre à Dents de Sabre : 8 4 - 4 4 2 4 3 5},
	type=Bête de Guerre,
	unitsize={1},
	additionalmodels=9,
	costpermodel=15,
	basesize=25x50,
	specialrules={\wildbeasts , \insignificant},
	unitrules={\unitrule{\wildbeasts}{Une unité qui suit cette règle spéciale ne peut jamais bénéficier de la \inspiringpresence du Général ou du \holdyourground du Porteur de la Grande Bannière, mais peut utiliser le Commandement d'un Chasseur de Mastodonte dans un rayon de \distance{12} comme si ce dernier était le Général de l'armée.}}
}

\showunit{
	name={\newrule{Dévoreur de Chair}},
	cost={90},
	profile={Dévoreur de Chair : 6 3 - 5 5 4 3 4 8},
	type=Infanterie Monstrueuse,
	unitsize={1},
	basesize=40x40,
	specialrules={\ambush , \hatred , \unbreakable , \fear , \regeneration{5}},
}

\showunit{
	name={Yétis},
	cost={120},
	profile={Yéti : 7 3 - 5 4 3 4 3 8},
	type=Infanterie Monstrueuse,
	unitsize={3},
	additionalmodels=5,
	costpermodel=40,
	basesize=40x40,
	specialrules={\magicalattacks , \strider{} , \fear , \swiftstride, \touchoffrost},
	equipment={Arme de base additionnelle},
	options={
		Toute l'unité peut bénéficier de la règle spéciale \scout et \skirmishers (Dans ce cas{,} la taille maximale de l'unité est de 4 figurines)=30 
	},
	commandgroup={\commandgroup{champion=10, musician=10}},
	unitrules={\unitrule{\touchoffrost}{Toutes les unités ennemies en contact avec une figurine avec cette règle spéciale voient leur Initiative réduite de 1 jusqu'à un minimum de 1.}}
}

\showunit{
	name={Braconniers \newrule{Scrapulins}},
	cost={60},
	profile={Braconnier Scrapulins : 5 2 3 3 3 1 3 1 6},
	type=Infanterie,
	unitsize={10},
	additionalmodels=10,
	costpermodel=4,
	basesize=20x20,
	specialrules={\itsatrap , \scout , \insignificant , \skirmishers},
	equipment={Armes de jet},
	commandgroup={\commandgroup{champion=10}},
	unitrules={\unitrule{\itsatrap}{Tout Champ, Forêt et Ruine devient un Terrain Dangereux pour le reste de la partie si, au moment du déploiement ou à la fin d'une Phase de Mouvement, une unité de Braconniers \newrule{Scrapulins} est en contact avec l'élément de terrain concerné (sauf si l'unité est en fuite). Les figurines qui traitent normalement ces éléments de terrain comme des Terrains Dangereux souffrent d'un malus de -1 à leurs tests de Terrain Dangereux. Les Braconniers \newrule{Scrapulins} n'ont pas à faire de test de Terrain Dangereux lorsqu'ils entrent ou sortent d'un élément de terrain qu'ils ont piégé.}}
}

\rareunitstitle

\showunit{
	name={Crache-Tonnerre},
	cost={\newrule{150}},
	profile={
				Char :                              - - - 5 5 5 - - -,
				Équipage Gueule d'acier (1) :       - 3 3 4 - - 2 3 7,
				Équipage Scrapulin (1) : - 2 3 3 - - 3 1 6,
				Rhinocéros laineux (1) :            6 3 - 5 - - 2 3 5
	},
	type=Char,
	unitsize={1},
	basesize=50x100,
	equipment={Canon Titan, \mountprotection{6}, \innatedefence{5}},
	unitequipment={\equipmentdef{Canon Titan}{Cette arme peut tirer de deux manières :%
	\begin{itemize}[label={-}, topsep=0cm, itemsep=0cm, parsep=0cm, leftmargin=0.3cm]%
		\item Comme une Arme d'artillerie du type Canon (\distance{2D6}) avec le profil suivant : \newrule{\portee{48}}, Force 10, \multiplewounds{\ordnance, \armourpiercing{2}}{}.
		\item Comme une Arme d'artillerie du type Batterie de tir avec le profil suivant : \portee{12}, Force 5, \armourpiercing{2}, \multipleshots{2D6}.
	\end{itemize}}}
}

\showunit{
	name={\newrule{Lance-Débris}},
	cost={\newrule{130}},
	profile={
				Char :                        - - - 5 4 4 - - -,
				Équipage de Scrapulins (7) : - 2 3 3 - - 3 1 6,
				Rhinocéros laineux (1) :      6 3 - 5 - - 2 3 5
	},
	type=Char,
	unitsize={1},
	basesize=50x100,
	equipment={Lance-débris, \mountprotection{6}, \innatedefence{5}},
	unitequipment={\equipmentdef{Lance-Débris}{C'est une Arme d'artillerie du type Catapulte (\distance{5}) avec le profil suivant : \newrule{\portee{48}}, Force 3, \lethalstrike .}}
}

\showunit{
	name={\newrule{Aurochs de Pierre}},
	cost={250},
	profile={
				Aurochs de Pierre :  7 3 - 7 6 6 2 5 5,
				Cavalier Ogre (1) : - 3 4 4 - - 3 3 8
	},
	type=Monstre monté,
	unitsize={1},
	basesize=100x150,
	specialrules={\frenzy \only{Aurochs de Pierre}, \rockskeleton , \swiftstride , \stubborn , \impacthits{3D3}}, 
	equipment={Arbalète ogre (Cavalier), Armes de jet (Cavalier), \innatedefence{4}},
	options={
	\optionschoice{Le cavalier peut remplacer l’Arbalète ogre}{
		par une Lance de chasse = \free ,
		par une Lance de cavalerie=5
		},
	},
	unitrules={\unitrule{\rockskeleton}{Si la figurine bénéficiant de cette règle spéciale subit une blessure assujettie à la règle spéciale \multiplewounds{}{}, le nombre de PVs enlevés à la figurine est divisé par deux et arrondi à l'unité supérieure.}}
}

\showunit{
	name={\newrule{Mammouth de Givre}},
	cost={200},
	profile={
				Mammouth de Givre : 6 3 - 6 6 6 2 4 5,
				Cavalier Ogre (2) : - 3 4 4 - - 3 3 8
	},
	type=Monstre monté,
	unitsize=1,
	basesize=100x150,
	specialrules={\frozenaura , \impacthits{1D3}},
	equipment={Arbalète ogre (Cavalier 1), Lance de chasse (Cavalier 2), \innatedefence{4}},
	unitrules={\unitrule{\frozenaura}{Toutes les unités ennemies dans un rayon de \distance{6} du \newrule{Mammouth de Givre} voient leur Initiative réduite de 3 jusqu'à un minimum de 1. De plus, la figurine dispose d'un objet de sort de Puissance 3 qui permet de lancer le sort Aquilon (Discipline \heavens).}}
}

\showunit{
	name={Géant asservi},
	cost={150},
	profile={Géant Asservi : 6 3 - 6 5 6 3 * 8},
	type=Monstre,
	unitsize={1},
	basesize=50x75,
	specialrules={\dowhatyouretold , \immunetopsychology , \stubborn},
	equipment={Armure lourde, Attaques de Géant},
	additional={
		\begin{description}
			\item[*Attaques de Géant :] Quand un Géant attaque au corps à corps, choisissez une unité en contact socle à socle avec lui qui va subir son attaque, lancez 1D6 et consultez ce que donne le résultat de son attaque dans l'une des deux tables ci-dessous en fonction du type de troupe de l'unité. Il est important de noter que l'attaque du Géant compte comme une attaque de corps à corps et suit ainsi normalement les règles des attaques de corps à corps. Le Géant peut également faire son \stomp{} normalement.
			\setlength\multicolsep{12.0pt plus 4.0pt minus 3pt}
			\begin{multicols}{2}
				Si l'unité est de type Infanterie, Bête de Guerre, Nuée, Machine de Guerre, Cavalerie :	
				\renewcommand{\arraystretch}{1.5}	
				\begin{center}\begin{tabular}{cl}
   					\hline
					1 & Hurle \tabularnewline
					2 & Saute \tabularnewline
					3 & Ramasse \tabularnewline
					4-6 & Frappe \tabularnewline
					\hline
				\end{tabular}\end{center}
				Si l'unité est de type Bête Monstrueuse, Infanterie Monstrueuse, Cavalerie Monstrueuse, Monstre, Monstre Monté, Char :
				\begin{center}\begin{tabular}{cl}
					\hline	
					1 & Hurle \tabularnewline
					2-3 & Tape comme un sourd \tabularnewline
					4-6 & Fracasse \tabularnewline
					\hline
				\end{tabular}\end{center}
				\renewcommand{\arraystretch}{1.2} % return to default
			\end{multicols}
			\item[Hurle :] Ni le Géant, ni l'unité sélectionnée par le Géant ne peuvent faire d'attaques au cours de cette Phase de Corps à Corps. Les attaques déjà réalisées, incluant celles réalisées simultanément avec cette attaque, ne sont pas concernées. Le camp du Géant gagne automatiquement le combat de 2. Si deux Géants opposés, ou plus, \og Hurlent \fg , le résultat du combat est un match nul.
			\item[Saute :]  L'unité sélectionnée subit 1D6 touches avec la Force du Géant, réalisées comme si elles suivaient la règle spéciale \grindingattacks{} avec la Force du Géant. Le Géant doit faire un test de Terrain Dangereux.
			\item[Ramasse :] Choisissez une figurine dans l'unité préalablement sélectionnée et en contact socle à socle avec le Géant. Cette figurine doit faire un test de Force et un test de Capacité de Combat. Pour chaque test raté, la figurine subit une touche avec la Force du Géant et suivant la règle spéciale \multiplewounds{1D3}{}.
			\item[Frappe :] Le Géant fait 2D6 attaques sur l'unité qu'il a préalablement choisie.
			\item [Tape comme un sourd :] Choisissez une figurine en contact socle à socle avec le Géant dans l'unité préalablement sélectionnée. Cette figurine doit faire un test d'Initiative. Si elle échoue, la figurine subit 2D6 blessures avec la règle spéciale \armourpiercing{6}.
			\item[Fracasse :] Choisissez une figurine dans l'unité préalablement sélectionnée et en contact socle à socle avec le Géant. Cette figurine subit une blessure avec la règle spéciale \armourpiercing{6}. Si la figurine n'a pas encore attaqué, elle ne peut pas le faire au cours de cette manche. Si la figurine a déjà réalisé ses attaques, elle ne pourra pas attaquer au cours du tour à venir de l'autre joueur.
		\end{description}
	}
}

\mountstitle

La section Montures est réservée aux Personnages montés. Les figurines non-Personnage montées suivent les règles données sous leur entrée respective.

\showunit{
	name={\newrule{Razor-Buffle}},
	cost={-},
	profile={Razor-Buffle : 8 3 - 5 5 3 2 4 5},
	type=Bête Monstrueuse,
	unitsize={1},
	basesize=50x100,
	specialrules={\fear},
	equipment={\mountprotection{5}}
}

\showunit{
	name={\newrule{Aurochs de Pierre}},
	cost={-},
	profile={Aurochs de Pierre : 7 3 - 7 6 6 2 5 5},
	type=Monstre,
	unitsize={1},
	basesize=100x150,
	specialrules={\frenzy, \swiftstride, \stubborn, \impacthits{3D3}},
	unitrules={\unitrule{\rockskeleton}{Si la figurine bénéficiant de cette règle spéciale subit une blessure assujettie à la règle spéciale \multiplewounds{}{}, le nombre de PVs enlevés à la figurine est divisé par deux et arrondi à l'unité supérieure.}},
	equipment={\innatedefence{4}}
}

\end{document}